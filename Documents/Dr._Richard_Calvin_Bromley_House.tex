\textbf{From Wikipedia, the free encyclopedia}

https://en.wikipedia.org/wiki/Dr.\_Richard\_Calvin\_Bromley\_House\\
Licensed under CC BY-SA 3.0:\\
https://en.wikipedia.org/wiki/Wikipedia:Text\_of\_Creative\_Commons\_Attribution-ShareAlike\_3.0\_Unported\_License

\section{Dr. Richard Calvin Bromley
House}\label{dr.-richard-calvin-bromley-house}

\begin{itemize}
\item
  \emph{The house was built in 1909-1911 for Dr. Richard Calvin Bromley,
  a physician.}
\item
  \emph{The house remained in the Bromley family until 1947.}
\item
  \emph{The Dr. Richard Calvin Bromley House, also known as Hotel
  Bromley, is a historic house in Flatwoods, Tennessee, U.S..}
\item
  \emph{It has been listed on the National Register of Historic Places
  since November 29, 1995.}
\end{itemize}

The Dr. Richard Calvin Bromley House, also known as Hotel Bromley, is a
historic house in Flatwoods, Tennessee, U.S..

The house was built in 1909-1911 for Dr. Richard Calvin Bromley, a
physician. Bromley was the son of a Confederate veteran and a graduate
of the University of Nashville. The house became known as Hotel Bromley
because many of Dr. Bromley's friends visited frequently to hunt and
fish. Bromley hired two African-American "servants," who lived upstairs.
The house was inherited by his daughter, Celia, a graduate of Vanderbilt
University, who lived here with her husband, William Allen Moore of
Pulaski, Tennessee. The house remained in the Bromley family until 1947.

The house was designed in the Queen Anne architectural style. It has
been listed on the National Register of Historic Places since November
29, 1995.

\section{References}\label{references}
