\textbf{From Wikipedia, the free encyclopedia}

https://en.wikipedia.org/wiki/USB\_flash\_drive\_security\\
Licensed under CC BY-SA 3.0:\\
https://en.wikipedia.org/wiki/Wikipedia:Text\_of\_Creative\_Commons\_Attribution-ShareAlike\_3.0\_Unported\_License

\section{USB flash drive security}\label{usb-flash-drive-security}

\begin{itemize}
\item
  \emph{Secure USB flash drives protect the data stored on them from
  access by unauthorized users.}
\item
  \emph{Companies in particular are at risk when sensitive data are
  stored on unsecured USB flash drives by employees who use the devices
  to transport data outside the office.}
\item
  \emph{USB flash drive products have been on the market since 2000, and
  their use is increasing exponentially.}
\end{itemize}

Secure USB flash drives protect the data stored on them from access by
unauthorized users. USB flash drive products have been on the market
since 2000, and their use is increasing exponentially. As both consumers
and businesses have increased demand for these drives, manufacturers are
producing faster devices with greater data storage capacities.

An increasing number of portable devices are used in business, such as
laptops, notebooks, personal digital assistants (PDA), smartphones, USB
flash drives and other mobile devices.

Companies in particular are at risk when sensitive data are stored on
unsecured USB flash drives by employees who use the devices to transport
data outside the office. The consequences of losing drives loaded with
such information can be significant, including the loss of customer
data, financial information, business plans and other confidential
information, with the associated risk of reputation damage.

\section{Major dangers of USB drives}\label{major-dangers-of-usb-drives}

\begin{itemize}
\item
  \emph{USB flash drives pose two major challenges to information system
  security: data leakage owing to their small size and ubiquity and
  system compromise through infections from computer viruses, malware
  and spyware.}
\end{itemize}

USB flash drives pose two major challenges to information system
security: data leakage owing to their small size and ubiquity and system
compromise through infections from computer viruses, malware and
spyware.

\section{Data leakage}\label{data-leakage}

\begin{itemize}
\item
  \emph{Examples of security breaches resulting from USB drives
  include:}
\item
  \emph{In the United States:\\
  a USB drive was stolen with names, grades, and social security numbers
  of 6,500 former students {[}4{]}\\
  USB flash drives with US Army classified military information were up
  for sale at a bazaar outside Bagram, Afghanistan.}
\item
  \emph{Usage: tracking corporate data stored on personal flash drives
  is a significant challenge; the drives are small, common and
  constantly moving.}
\end{itemize}

The large storage capacity of USB flash drives relative to their small
size and low cost means that using them for data storage without
adequate operational and logical controls may pose a serious threat to
information availability, confidentiality and integrity. The following
factors should be taken into consideration for securing important
assets:

Storage: USB flash drives are hard to track physically, being stored in
bags, backpacks, laptop cases, jackets, trouser pockets or left at
unattended workstations.

Usage: tracking corporate data stored on personal flash drives is a
significant challenge; the drives are small, common and constantly
moving. While many enterprises have strict management policies toward
USB drives and some companies ban them outright to minimize risk, others
seem unaware of the risks these devices pose to system security.

The average cost of a data breach from any source (not necessarily a
flash drive) ranges from less than \$100,000 to about \$2.5 million.

A SanDisk survey characterized the data corporate end users most
frequently copy:

Customer data (25\%)

Financial information (17\%)

Business plans (15\%)

Employee data (13\%)

Marketing plans (13\%)

Intellectual property (6\%)

Source code (6\%)

Examples of security breaches resulting from USB drives include:

In the UK:\\
HM Revenue \& Customs lost personal details of 6,500 private pension
holders

In the United States:\\
a USB drive was stolen with names, grades, and social security numbers
of 6,500 former students {[}4{]}\\
USB flash drives with US Army classified military information were up
for sale at a bazaar outside Bagram, Afghanistan.{[}5{]}

\section{Malware infections}\label{malware-infections}

\begin{itemize}
\item
  \emph{The study found that 26 percent of all malware infections of
  Windows system were due to USB flash drives exploiting the AutoRun
  feature in Microsoft Windows.}
\item
  \emph{This has made USB flash drives a leading form of information
  system infection.}
\item
  \emph{Examples of malware spread by USB flash drives include:}
\item
  \emph{When a piece of malware gets onto a USB flash drive, it may
  infect the devices into which that drive is subsequently plugged.}
\end{itemize}

In the early days of computer viruses, malware, and spyware, the primary
means of transmission and infection was the floppy disk. Today, USB
flash drives perform the same data and software storage and transfer
role as the floppy disk, often used to transfer files between computers
which may be on different networks, in different offices, or owned by
different people. This has made USB flash drives a leading form of
information system infection. When a piece of malware gets onto a USB
flash drive, it may infect the devices into which that drive is
subsequently plugged.

The prevalence of malware infection by means of USB flash drive was
documented in a 2011 Microsoft study analyzing data from more than 600
million systems worldwide in the first half of 2011. The study found
that 26 percent of all malware infections of Windows system were due to
USB flash drives exploiting the AutoRun feature in Microsoft Windows.
That finding was in line with other statistics, such as the monthly
reporting of most commonly detected malware by antivirus company ESET,
which lists abuse of autorun.inf as first among the top ten threats in
2011.

The Windows autorun.inf file contains information on programs meant to
run automatically when removable media (often USB flash drives and
similar devices) are accessed by a Windows PC user. The default Autorun
setting in Windows versions prior to Windows 7 will automatically run a
program listed in the autorun.inf file when you access many kinds of
removable media. Many types of malware copy themselves to removable
storage devices: while this is not always the program's primary
distribution mechanism, malware authors often build in additional
infection techniques.

Examples of malware spread by USB flash drives include:

The Duqu collection of computer malware.

The Flame modular computer malware.

The Stuxnet malicious computer worm.

\section{Solutions}\label{solutions}

\begin{itemize}
\item
  \emph{One common approach is to encrypt the data for storage and
  routinely scan USB flash drives for computer viruses, malware and
  spyware with an antivirus program, although other methods are
  possible.}
\item
  \emph{Since the security of the physical drive cannot be guaranteed
  without compromising the benefits of portability, security measures
  are primarily devoted to making the data on a compromised drive
  inaccessible to unauthorized users and unauthorized processes, such as
  may be executed by malware.}
\end{itemize}

Since the security of the physical drive cannot be guaranteed without
compromising the benefits of portability, security measures are
primarily devoted to making the data on a compromised drive inaccessible
to unauthorized users and unauthorized processes, such as may be
executed by malware. One common approach is to encrypt the data for
storage and routinely scan USB flash drives for computer viruses,
malware and spyware with an antivirus program, although other methods
are possible.

\section{Software encryption}\label{software-encryption}

\begin{itemize}
\item
  \emph{Installing software on company computers may help track and
  minimize risk by recording the interactions between any USB drive and
  the computer and storing them in a centralized database.}
\item
  \emph{Additional software can be installed on an external USB drive to
  prevent access to files in case the drive becomes lost or stolen.}
\item
  \emph{Also, Windows 7 Enterprise, Windows 7 Ultimate and Windows
  Server 2008 R2 provide USB drive encryption using BitLocker to Go.}
\end{itemize}

Software solutions such as BitLocker, DiskCryptor and the popular
VeraCrypt allow the contents of a USB drive to be encrypted
automatically and transparently. Also, Windows 7 Enterprise, Windows 7
Ultimate and Windows Server 2008 R2 provide USB drive encryption using
BitLocker to Go. The Apple Computer Mac OS X operating system has
provided software for disc data encryption since Mac OS X Panther was
issued in 2003 (see also: Disk Utility).{[}citation needed{]}

Additional software can be installed on an external USB drive to prevent
access to files in case the drive becomes lost or stolen. Installing
software on company computers may help track and minimize risk by
recording the interactions between any USB drive and the computer and
storing them in a centralized database.{[}citation needed{]}

\section{Hardware encryption}\label{hardware-encryption}

\begin{itemize}
\item
  \emph{This type of functionality cannot be provided by a software
  system since the encrypted data can simply be copied from the drive.}
\item
  \emph{Some USB drives utilize hardware encryption in which microchips
  within the USB drive provide automatic and transparent encryption.}
\item
  \emph{The cost of these USB drives can be significant but is starting
  to fall due to this type of USB drive gaining popularity.}
\end{itemize}

Some USB drives utilize hardware encryption in which microchips within
the USB drive provide automatic and transparent encryption. Some
manufacturers offer drives that require a pin code to be entered into a
physical keypad on the device before allowing access to the drive. The
cost of these USB drives can be significant but is starting to fall due
to this type of USB drive gaining popularity.

Hardware systems may offer additional features, such as the ability to
automatically overwrite the contents of the drive if the wrong password
is entered more than a certain number of times. This type of
functionality cannot be provided by a software system since the
encrypted data can simply be copied from the drive. However, this form
of hardware security can result in data loss if activated accidentally
by legitimate users and strong encryption algorithms essentially make
such functionality redundant.

As the encryption keys used in hardware encryption are typically never
stored in the computer's memory, technically hardware solutions are less
subject to "cold boot" attacks than software-based systems. In reality
however, "cold boot" attacks pose little (if any) threat, assuming
basic, rudimentary, security precautions are taken with software-based
systems.

\section{Compromised systems}\label{compromised-systems}

\begin{itemize}
\item
  \emph{Flash drives that have been compromised (and claimed to now be
  fixed) include:}
\item
  \emph{The security of encrypted flash drives is constantly tested by
  individual hackers as well as professional security firms.}
\item
  \emph{Kingston offered replacement drives with a different security
  architecture.}
\item
  \emph{Verbatim Corporate Secure USB Flash Drive}
\end{itemize}

The security of encrypted flash drives is constantly tested by
individual hackers as well as professional security firms. At times (as
in January 2010) flash drives that have been positioned as secure were
found to have been poorly designed such that they provide little or no
actual security, giving access to data without knowledge of the correct
password.

Flash drives that have been compromised (and claimed to now be fixed)
include:

SanDisk Cruzer Enterprise

Kingston DataTraveler BlackBox

Verbatim Corporate Secure USB Flash Drive

Trek Technology ThumbDrive CRYPTO

All of the above companies reacted immediately. Kingston offered
replacement drives with a different security architecture. SanDisk,
Verbatim, and Trek released patches.

\section{Remote management}\label{remote-management}

\begin{itemize}
\item
  \emph{In commercial environments, where most secure USB drives are
  used, a central/remote management system may provide organizations
  with an additional level of IT asset control, significantly reducing
  the risks of a harmful data breach.}
\item
  \emph{This can include initial user deployment and ongoing management,
  password recovery, data backup, remote tracking of sensitive data and
  termination of any issued secure USB drives.}
\end{itemize}

In commercial environments, where most secure USB drives are used, a
central/remote management system may provide organizations with an
additional level of IT asset control, significantly reducing the risks
of a harmful data breach. This can include initial user deployment and
ongoing management, password recovery, data backup, remote tracking of
sensitive data and termination of any issued secure USB drives. Such
management systems are available as software as a service (SaaS), where
Internet connectivity is allowed, or as behind-the-firewall solutions.

\section{See also}\label{see-also}

\begin{itemize}
\item
  \emph{Health Insurance Portability and Accountability Act (HIPAA)
  (Moving confidential data requires encryption.)}
\item
  \emph{Data remanence}
\end{itemize}

Health Insurance Portability and Accountability Act (HIPAA) (Moving
confidential data requires encryption.)

Cruzer Enterprise

Data remanence

IronKey

Kingston Technology

\section{References}\label{references}

\section{External links}\label{external-links}

\begin{itemize}
\item
  \emph{Dataquest insight: USB flash drive market trends, worldwide,
  2001--2010, Joseph Unsworth, Gartner, 20 November 2006.}
\item
  \emph{Computerworld Review: 7 Secure USB Drives, by Bill O'Brien, Rich
  Ericson and Lucas Mearian, March 2008 (archived from the original on
  17 February 2009)}
\item
  \emph{Analysis of USB flash drives in a virtual environment, by Derek
  Bem and Ewa Huebner, Small Scale Digital Device Forensics Journal,
  Vol.}
\end{itemize}

Analysis of USB flash drives in a virtual environment, by Derek Bem and
Ewa Huebner, Small Scale Digital Device Forensics Journal, Vol. 1, No 1,
June 2007 (archived from the original on 19 October 2013)

Dataquest insight: USB flash drive market trends, worldwide, 2001--2010,
Joseph Unsworth, Gartner, 20 November 2006.

Computerworld Review: 7 Secure USB Drives, by Bill O'Brien, Rich Ericson
and Lucas Mearian, March 2008 (archived from the original on 17 February
2009)

BadUSB - On Accessories that Turn Evil on YouTube, by Karsten Nohl and
Jakob Lell
