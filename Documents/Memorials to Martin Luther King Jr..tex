\textbf{From Wikipedia, the free encyclopedia}

https://en.wikipedia.org/wiki/Memorials\%20to\%20Martin\%20Luther\%20King\%20Jr.\\
Licensed under CC BY-SA 3.0:\\
https://en.wikipedia.org/wiki/Wikipedia:Text\_of\_Creative\_Commons\_Attribution-ShareAlike\_3.0\_Unported\_License

\section{Memorials to Martin Luther King
Jr.}\label{memorials-to-martin-luther-king-jr.}

\begin{itemize}
\item
  \emph{This is a list of memorials to Martin Luther King Jr.}
\end{itemize}

This is a list of memorials to Martin Luther King Jr.

\section{United States}\label{united-states}

\begin{itemize}
\item
  \emph{The Martin Luther King Jr. Memorial Bridge in Fort Wayne,
  Indiana}
\item
  \emph{In 1980, the U.S. Department of the Interior designated King's
  boyhood home in Atlanta and several nearby buildings the Martin Luther
  King Jr. National Historic Site.}
\item
  \emph{Martin Luther King Jr. Memorial on the National Mall in
  Washington, D.C. King was the first African American and the fourth
  non-president honored with his own memorial in the National Mall
  area.}
\end{itemize}

There are numerous memorials to King in the United States, including:

Martin Luther King Jr. Memorial on the National Mall in Washington, D.C.
King was the first African American and the fourth non-president honored
with his own memorial in the National Mall area. In 1996, Congress
authorized the Alpha Phi Alpha fraternity, of which King is still a
member, to establish a foundation to manage fundraising and design of a
national memorial to King. The memorial opened in August 2011 and is
administered by the National Park Service. The address of the monument,
1964 Independence Avenue, SW, commemorates the year that the Civil
Rights Act of 1964 became law.

The Landmark for Peace Memorial in Indianapolis, Indiana

King County, Washington, rededicated its name in his honor in 1986 and
changed its logo to an image of his face in 2007.

The city government center in Harrisburg, Pennsylvania, is named in
honor of King.

In 1980, the U.S. Department of the Interior designated King's boyhood
home in Atlanta and several nearby buildings the Martin Luther King Jr.
National Historic Site.

A bust of Dr. King was added to the "gallery of notables" in the United
States Capitol in 1986, portraying him in a "restful, nonspeaking pose."

The beginning words of King's "I Have a Dream" speech are etched on the
steps of the Lincoln Memorial, at the place where King stood during that
speech. These words from the speech---"five short lines of text carved
into the granite on the steps of the Lincoln Memorial"---were etched in
2003, on the 40th anniversary of the march to Washington, by stone
carver Andy Del Gallo, after a law was passed by Congress providing
authorization for the inscription.

The Homage to King sculpture in Atlanta, Georgia

The Dream sculpture in Portland, Oregon

The Martin Luther King Jr. Memorial Bridge in Fort Wayne, Indiana

The National Civil Rights Museum, at the Lorraine Motel in Memphis,
Tennessee, where King died

Brown Chapel A.M.E. Church in Selma, Alabama

On October 11, 2015, the Atlanta Journal-Constitution reported a
proposed "Freedom Bell" may be installed atop Stone Mountain honoring
King and his "I Have a Dream" speech, specifically the line "Let freedom
ring from Stone Mountain of Georgia."

A bust of Martin Luther King Jr. has been in the collection of the
collection of the Smithsonian Institution's National Portrait Gallery
since 1974, and displayed in the White House since 2000; a second cast
is in the collection of the National Museum of African American History
and Culture.

In Norfolk, Virginia stands a memorial in honor of King. The
83-foot-high granite obelisk was conceived by former Norfolk Councilman
and General District Court Judge Joseph A. Jordan Jr.

The Martin Luther King, Jr. Memorial at Yerba Buena Gardens in San
Francisco is located behind a waterfall, which is the largest fountain
on the West Coast.{[}citation needed{]} The King memorial consists of
large, etched glass excerpts of King's speeches in the languages of San
Francisco's sister cities, and also includes a large green space where
performance arts events are held throughout the year. The entire
memorial was a collaborative project between Sculptor Houston Conwill,
Poet Estella Majoza and Architect Joseph De Pace. The memorial is
located on the gardens' second block, between Howard and Folsom Streets,
which was opened in 1998, with a dedication to Martin Luther King, Jr.
by Mayor Willie Brown.

\section{Internationally}\label{internationally}

\begin{itemize}
\item
  \emph{Martin Luther King Jr. Forest in Israel's Southern Galilee
  region (along with the Coretta Scott King Forest in Biriya Forest,
  Israel)}
\item
  \emph{The Martin Luther King Jr. School in Accra, Ghana}
\item
  \emph{The Reverend Martin Luther King Jr. Church in Debrecen, Hungary}
\end{itemize}

Numerous other memorials honor him around the world, including:

The Reverend Martin Luther King Jr. Church in Debrecen, Hungary

The King-Luthuli Transformation Center in Johannesburg, South Africa

The Rev. Martin Luther King Jr. Forest in Israel's Southern Galilee
region (along with the Coretta Scott King Forest in Biriya Forest,
Israel)

The Martin Luther King Jr. School in Accra, Ghana

The Gandhi-King Plaza (garden), at the India International Center in New
Delhi, India

One of the 10 statues of 20th-century martyrs on the façade of
Westminster Abbey, London, UK

\section{See also}\label{see-also}

\begin{itemize}
\item
  \emph{List of streets named after Martin Luther King Jr.}
\end{itemize}

Civil rights movement in popular culture

List of streets named after Martin Luther King Jr.

\section{References}\label{references}
