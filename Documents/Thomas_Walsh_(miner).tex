\textbf{From Wikipedia, the free encyclopedia}

https://en.wikipedia.org/wiki/Thomas\_Walsh\_\%28miner\%29\\
Licensed under CC BY-SA 3.0:\\
https://en.wikipedia.org/wiki/Wikipedia:Text\_of\_Creative\_Commons\_Attribution-ShareAlike\_3.0\_Unported\_License

\section{Thomas Walsh (miner)}\label{thomas-walsh-miner}

\begin{itemize}
\item
  \emph{He was also famous for giving the famed Hope Diamond to his
  daughter Evalyn Walsh McLean as a wedding present.}
\item
  \emph{Thomas Francis Walsh (April 2, 1850 -- April 8, 1910) was an
  Irish-American miner who discovered one of the largest gold mines in
  America.}
\end{itemize}

Thomas Francis Walsh (April 2, 1850 -- April 8, 1910) was an
Irish-American miner who discovered one of the largest gold mines in
America. He was also famous for giving the famed Hope Diamond to his
daughter Evalyn Walsh McLean as a wedding present.

\section{Early life}\label{early-life}

\begin{itemize}
\item
  \emph{Michael Walsh, who died in 1904 in Denver, Colorado of dropsy of
  the liver}
\item
  \emph{Walsh was born April 2, 1850 to Michael Walsh, a farmer, and
  Bridget Scully.}
\item
  \emph{Walsh had the following siblings:}
\item
  \emph{Maria Walsh, who married Arthur Lafferty, a two-gun police
  sergeant in Leadville, Colorado}
\end{itemize}

Walsh was born April 2, 1850 to Michael Walsh, a farmer, and Bridget
Scully. He was most likely born on his father's farm, Baptistgrange, in
Lisronagh, Tipperary, Ireland. Walsh had the following siblings:

Maria Walsh, who married Arthur Lafferty, a two-gun police sergeant in
Leadville, Colorado

Michael Walsh, who died in 1904 in Denver, Colorado of dropsy of the
liver

According to his daughter's book, Father Struck It Rich, he became an
apprentice to a millwright at the age of twelve and grew into a fine
carpenter.

In 1869, he emigrated to the United States with his sister, Maria, after
the death of his father. For a time, he settled in Worcester,
Massachusetts, with his aunts, Catherine and Bridget Walsh Power, who
helped "shake the greenhorn off him".

\section{Career}\label{career}

\begin{itemize}
\item
  \emph{After becoming an expert in the subject in gold mining, Walsh
  was overcome by gold fever and took to the hills.}
\item
  \emph{It has been said that at first Walsh was attracted to the
  opportunities that came with the gold rush, including trading goods
  and services at inflated prices, as opposed to the gold rush itself.}
\end{itemize}

In the early 1870s, he heeded the call to "go west, young man" and found
himself in Colorado getting paid well for his carpentry skills. During
the 1870s, the Black Hills of South Dakota saw a gold rush that
attracted hordes of hopeful men afflicted with gold fever. It has been
said that at first Walsh was attracted to the opportunities that came
with the gold rush, including trading goods and services at inflated
prices, as opposed to the gold rush itself.

Gradually, he became more and more immersed in the world of gold and was
soon trading mining equipment to prospectors for mining claims as
payment. He also studied mining technology at night. In 1877, he moved
to Leadville, Colorado with a small fortune between \$75,000 (equivalent
to \$1,765,000 in 2018) and \$100,000 (equivalent to \$2,353,000 in
2018). Along with his wife, he ran the Grand Central Hotel in Leadville.

After becoming an expert in the subject in gold mining, Walsh was
overcome by gold fever and took to the hills. Unlike other prospectors
he took a far more methodical and careful approach to prospecting which
soon paid off. In 1896, he came home and uttered the words which later
became the title of his daughter's book, "Daughter, I've struck it
rich!" The Camp Bird Gold Mine near Ouray, Colorado soon turned out
\$5,000/day (equivalent to \$151,000 in 2018) in ore and produced riches
for the Walsh family "beyond the dreams of avarice". In a short period
of time, Walsh extracted a fortune totaling \$3,000,000 (equivalent to
\$90,348,000 in 2018).

\section{Washington, DC}\label{washington-dc}

\begin{itemize}
\item
  \emph{The wealth that Walsh discovered soon provided the family with a
  lavish lifestyle that included trips to Europe, fine clothes, and
  expensive motor cars.}
\item
  \emph{Around 1898, the family moved to Washington, D.C. where in 1900,
  he was appointed by President William McKinley as a commissioner to
  the Paris Exposition of 1899.}
\end{itemize}

The wealth that Walsh discovered soon provided the family with a lavish
lifestyle that included trips to Europe, fine clothes, and expensive
motor cars. Around 1898, the family moved to Washington, D.C. where in
1900, he was appointed by President William McKinley as a commissioner
to the Paris Exposition of 1899.

\includegraphics[width=3.85733in,height=5.50000in]{media/image1.jpg}\\
\emph{Carrie Reed Walsh}

\section{Personal life}\label{personal-life}

\begin{itemize}
\item
  \emph{Evalyn Walsh, August 1, 1886 -- April 24, 1947}
\item
  \emph{Thomas Francis Walsh died on April 8, 1910, at his home in
  Washington, D.C.}
\item
  \emph{On January 23, 1909, The Aero Club of Washington was founded,
  with Walsh as serving president, to promote the new technology of
  Aviation.}
\item
  \emph{Vinson Walsh, April 9, 1888 -- August 19, 1905, who died in a
  car accident}
\end{itemize}

On July 11, 1879 in Leadville, Colorado, he married Carrie Bell Reed.
The couple had two children:

Evalyn Walsh, August 1, 1886 -- April 24, 1947

Vinson Walsh, April 9, 1888 -- August 19, 1905, who died in a car
accident

In 1903 the family moved into the ornate mansion at 2020 Massachusetts
Avenue. Later, the house became the Indonesian Embassy. On January 23,
1909, The Aero Club of Washington was founded, with Walsh as serving
president, to promote the new technology of Aviation. Due to his
involvement with the Paris Exposition of 1899, Walsh became friends with
King Leopold of Belgium, whom he created a suite in his home to host.
Unfortunately, the King never made a trip to the United States. However,
when King Albert, Leopold's nephew, and Queen Elizabeth traveled to the
United States in 1919, Walsh's wife, then widowed, was decorated by the
King for her service during World War I.

In 1908, Walsh's daughter Evalyn, and only living child at the time,
married Edward Beale McLean, the son of John Roll McLean, who became the
publisher and owner of The Washington Post newspaper in 1916 until 1933.

Thomas Francis Walsh died on April 8, 1910, at his home in Washington,
D.C.

\section{Extended family}\label{extended-family}

\begin{itemize}
\item
  \emph{Thomas Walsh is a cousin twice removed to W. Arthur Garrity,
  Jr., the federal judge who issued the famous 1974 order that Boston
  schools desegregate by means of busing.}
\end{itemize}

Thomas Walsh is a cousin twice removed to W. Arthur Garrity, Jr., the
federal judge who issued the famous 1974 order that Boston schools
desegregate by means of busing.

\section{References}\label{references}

\section{Sources}\label{sources}

\begin{itemize}
\item
  \emph{Father Struck it Rich, by Evalyn Walsh McLean (1936)}
\end{itemize}

An informal family history written by Margaret Kennedy (c.1972)

Father Struck it Rich, by Evalyn Walsh McLean (1936)

Hope by Mary Ryan (c.1998)

\section{External links}\label{external-links}

\begin{itemize}
\item
  \emph{Works by or about Thomas Walsh in libraries (WorldCat catalog)}
\end{itemize}

Works by or about Thomas Walsh in libraries (WorldCat catalog)
