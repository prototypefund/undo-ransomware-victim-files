\textbf{From Wikipedia, the free encyclopedia}

https://en.wikipedia.org/wiki/Sojourner\%20\%28rover\%29\\
Licensed under CC BY-SA 3.0:\\
https://en.wikipedia.org/wiki/Wikipedia:Text\_of\_Creative\_Commons\_Attribution-ShareAlike\_3.0\_Unported\_License

\includegraphics[width=5.50000in,height=2.62625in]{media/image1.jpg}\\
\emph{Sojourner rover on Mars, as stowed on one of the station petals}

\includegraphics[width=2.32067in,height=5.50000in]{media/image2.jpg}\\
\emph{A color image taken by the Sojourner rover of its wheel leaving
tracks on Mars}

\includegraphics[width=5.50000in,height=3.46615in]{media/image3.jpg}\\
\emph{Sojourner view of Mars Pathfinder base station (Sagan Memorial
Station) after driving off the ramps onto Mars}

\section{Sojourner (rover)}\label{sojourner-rover}

\begin{itemize}
\item
  \emph{Sojourner is the Mars Pathfinder robotic Mars rover that landed
  on July 4, 1997 in the Ares Vallis region, and explored Mars for
  around three months.}
\end{itemize}

Sojourner is the Mars Pathfinder robotic Mars rover that landed on July
4, 1997 in the Ares Vallis region, and explored Mars for around three
months. It has front and rear cameras and hardware to conduct several
scientific experiments. Designed for a mission lasting 7~sols, with
possible extension to 30~sols, it was in fact active for 83~sols. The
base station had its last communication session with Earth at 3:23~a.m.
Pacific Daylight Time on September 27, 1997. The rover needed the base
station to communicate with Earth, despite still functioning at the time
communications ended.

Sojourner traveled a distance of just over 100 meters (330~ft) by the
time communication was lost. It was instructed to stay stationary until
October 5, 1997 (sol 91) and then drive around the lander.

\section{Overview}\label{overview}

\begin{itemize}
\item
  \emph{The rover was imaged on Mars by the base station's IMP camera
  system, which also helped determine where the rover should go.}
\item
  \emph{The batteries also allowed the health of the rover to be checked
  while enclosed in the cruise stage while en route to Mars.}
\item
  \emph{The rover had a mass of 11.5~kg (weighing about 25 pounds on
  Earth), which equates to a weight of 4.5 kgf (10 pounds) on Mars.}
\end{itemize}

The word Sojourner first appeared in print in the first Bible printed by
Gutenberg in 1454-1455.{[}clarification needed{]} It specifically deals
with the travels of Abraham. It is found in the first chapter of
Genesis. It means "traveler", and was selected in an essay contest won
by Valerie Ambroise, a 12-year-old from U.S. state of Connecticut. It is
named for abolitionist and women's rights activist Sojourner Truth. The
second-place prize went to Deepti Rohatgi, 18, of Rockville, who
proposed Marie Curie, a Nobel Prize-winning Polish chemist. Third place
went to Adam Sheedy, 16, of Round Rock, TX, who chose Judith Resnik, a
United States' astronaut and shuttle crew-member. The rover was also
known as Microrover Flight Experiment abbreviated MFEX.

Sojourner has solar panels and a non-rechargeable battery, which allowed
limited nocturnal operations. Once the batteries were depleted, it could
only operate during the day. The batteries are lithium-thionyl chloride
(LiSOCl2) and could provide 150 watt-hours. The batteries also allowed
the health of the rover to be checked while enclosed in the cruise stage
while en route to Mars.

0.22 square meters of solar cells could produce a maximum of about 15
watts on Mars, depending on conditions. The cells were GaAs/Ge (Gallium
Arsenide/Germanium) and capable of about 18 percent efficiency. They
could survive down to about −140° Celsius (−220~°F).

Its central processing unit (CPU) is an 80C85 with a 2~MHz clock,
addressing 64 Kbytes of memory. It has four memory stores; the
previously mentioned 64 Kbytes of RAM (made by IBM) for the main
processor, 16 Kbytes of radiation-hardened PROM (made by Harris), 176
Kbytes of non-volatile storage (made by Seeq Technology), and 512 Kbytes
of temporary data storage (made by Micron). The electronics were housed
inside the Warm Electronics Box inside the rover.

It communicated with the base station with 9,600 baud radio modems. The
practical rate was closer to 2,600 baud with a theoretical range of
about half a kilometer. The rover could travel out of range of the
lander, but its software would need to be changed to that mode. Under
normal driving, it would periodically send a "heartbeat" message to the
lander.

The UHF radio modems worked similar to walkie-talkies, but sent data,
not voice. It could send or receive, but not both at same time, which is
known as half-duplex. The data was communicated in bursts of 2
kilobytes.

The Alpha Proton X-ray Spectrometer (APXS) is nearly identical to the
one on Mars 96, and was a collaboration between the Max Planck Institute
for Solar System Research in Lindau, Germany (formally known as the Max
Planck Institute For Aeronomy) and the University of Chicago in the
United States. APXS could determine elemental composition of Mars rocks
and dust, except for hydrogen. It works by exposing a sample to alpha
particles, then measuring the energies of emitted protons, X-rays, and
backscattered alpha particles.

The rover had three cameras: two monochrome cameras in front, and a
color camera in the rear. Each front camera had an array 484 pixels high
by 768 wide. The optics consisted of a window, lens, and field
flattener. The window was made of sapphire, while the lens objective and
flattener were made of zinc selenide. The rover was imaged on Mars by
the base station's IMP camera system, which also helped determine where
the rover should go.

Sojourner operation was supported by "Rover Control Software", which ran
on a Silicon Graphics Onyx2 computer back on Earth, and allowed command
sequences to be generated using a graphical interface. The rover driver
would wear 3D goggles supplied with imagery from the base station and
move a virtual model with the spaceball controller, a specialized
joystick. The control software allowed the rover and surrounding terrain
to be viewed from any angle or position, supporting the study of terrain
features, placing waypoints, or doing virtual flyovers.

The rover had a mass of 11.5~kg (weighing about 25 pounds on Earth),
which equates to a weight of 4.5 kgf (10 pounds) on Mars.

\section{In popular culture}\label{in-popular-culture}

\begin{itemize}
\item
  \emph{In the 2005 season 4 Star Trek: Enterprise episode "Terra
  Prime", Sojourner is briefly seen on the surface of Mars as a
  monument.}
\item
  \emph{In the 2011 novel The Martian by Andy Weir, and the 2015 film
  based on it, the protagonist Mark Watney, stranded on Mars, recovers
  the Pathfinder lander, and is able to use it to contact Earth.}
\item
  \emph{In the movie, he is later seen in his Mars outpost, the Ares III
  Hab, with the Sojourner roving around.}
\end{itemize}

In the 2000 film Red Planet, the crew of the first manned mission to
Mars survives the crash-landing of their entry vehicle, but their
communications equipment is destroyed so they cannot contact their
recovery vehicle in orbit. To reestablish contact before being presumed
dead and left behind by the pilot of their recovery vehicle, the crew
goes to the site of the Pathfinder rover, from which they salvage parts
to make a basic radio.

In the 2005 season 4 Star Trek: Enterprise episode "Terra Prime",
Sojourner is briefly seen on the surface of Mars as a monument.\\
Sojourner also features in Enterprises's opening sequence.

In the 2011 novel The Martian by Andy Weir, and the 2015 film based on
it, the protagonist Mark Watney, stranded on Mars, recovers the
Pathfinder lander, and is able to use it to contact Earth. In the movie,
he is later seen in his Mars outpost, the Ares III Hab, with the
Sojourner roving around.

\includegraphics[width=5.50000in,height=2.86260in]{media/image4.jpg}\\
\emph{The Sol 2 "insurance panorama" of Sojourner, taken on 530,600, and
750 nm filters}

\section{Panorama}\label{panorama}

\section{Sojourner ' s location in
context}\label{sojourner-s-location-in-context}

\includegraphics[width=5.50000in,height=3.13500in]{media/image5.jpg}\\
\emph{A comparison of sizes for the Sojourner rover, the Mars
Exploration Rovers (Spirit and Opportunity), the Phoenix lander and the
Mars Science Laboratory (Curiosity).}

\section{Comparison to later
Mars-craft}\label{comparison-to-later-mars-craft}

\section{See also}\label{see-also}

\begin{itemize}
\item
  \emph{List of missions to Mars}
\end{itemize}

Alpha particle X-ray spectrometer

List of missions to Mars

Materials Adherence Experiment (an experiment carried on Sojourner)

List of spacecraft powered by non-rechargeable batteries

\section{References}\label{references}

\section{External links}\label{external-links}

\begin{itemize}
\item
  \emph{Directory of Pathfinder Images}
\end{itemize}

Official website

Directory of Pathfinder Images

Edible Spacecraft
