\textbf{From Wikipedia, the free encyclopedia}

https://en.wikipedia.org/wiki/School\_of\_Army\_Aviation\_\%28Germany\%29\\
Licensed under CC BY-SA 3.0:\\
https://en.wikipedia.org/wiki/Wikipedia:Text\_of\_Creative\_Commons\_Attribution-ShareAlike\_3.0\_Unported\_License

\section{School of Army Aviation
(Germany)}\label{school-of-army-aviation-germany}

\begin{itemize}
\item
  \emph{The German School of Army Aviation (Heeresfliegerwaffenschule)
  based at Bückeburg, is one of the schools of the German Army and is
  responsible for the training and development of the German Army
  Aviation Corps' personnel and equipment.}
\item
  \emph{Furthermore, the basic training of helicopter pilots of the
  other components of the German Armed Forces also takes place at the
  School of Army Aviation.}
\end{itemize}

The German School of Army Aviation (Heeresfliegerwaffenschule) based at
Bückeburg, is one of the schools of the German Army and is responsible
for the training and development of the German Army Aviation Corps'
personnel and equipment. Furthermore, the basic training of helicopter
pilots of the other components of the German Armed Forces also takes
place at the School of Army Aviation.

The School of Army Aviation was founded on 1 July 1959 and based first
at Mendig before being transferred on 12 January 1960 to its current
location at Bückeburg Air Base, utilising the facilities of RAF
Bückeburg, constructed by the Royal Air Force in 1946 and closed in the
mid-1950s.

In October 2011 the German Federal Ministry of Defence announced a
reorganisation/reduction of the German Armed Forces. Due to the
reduction of helicopter units within the German Army, some of which are
to be disbanded whereas others are to be transferred to the German Air
Force, and also in light of the already existing international character
of training helicopter pilots, the School of Army Aviation
(Heeresfliegerwaffenschule) will be renamed to International Helicopter
Training Centre. Training of helicopter pilots of the other components
of the German Armed Forces (Air Force and Navy) at Bückeburg started in
2011.

\section{Organisation}\label{organisation}

\begin{itemize}
\item
  \emph{Together with the French Army Light Aviation the Franco-German
  Training Centre TIGER at Le Luc -- Le Cannet Airport in Le
  Cannet-des-Maures was established, solely dedicated to provide flying
  instruction on the new helicopter Eurocopter Tiger.}
\item
  \emph{The head and commanding officer the German School of Army
  Aviation has the rank of Brigadier General and the position of General
  of the Army Aviation Corps (General der Heeresflieger).}
\end{itemize}

The head and commanding officer the German School of Army Aviation has
the rank of Brigadier General and the position of General of the Army
Aviation Corps (General der Heeresflieger). The School is structured in:

Staff

Support Group

Technical Maintenance Department

Development Group

Instruction Group

Staff of the German Armed Forces' Central Medical Services, amongst
which are a high-ranking flight surgeon and flight psychologist, are
also incorporated into the school.

Apart from the actual flight training, modern high-end flight simulators
are extensively used during the training of future military helicopter
pilots.

Together with the French Army Light Aviation the Franco-German Training
Centre TIGER at Le Luc -- Le Cannet Airport in Le Cannet-des-Maures was
established, solely dedicated to provide flying instruction on the new
helicopter Eurocopter Tiger.

\includegraphics[width=5.50000in,height=4.12500in]{media/image1.jpg}\\
\emph{Eurocopter Tiger of the German Army}

\section{Units}\label{units}

\begin{itemize}
\item
  \emph{* (disbanded on 31 December 2008)}
\end{itemize}

* (disbanded on 31 December 2008)

\section{See also}\label{see-also}

\begin{itemize}
\item
  \emph{Army Aviation School}
\item
  \emph{History of the German Army Aviation Corps}
\item
  \emph{German Army Aviation Corps}
\item
  \emph{German Army}
\item
  \emph{Army aviation}
\end{itemize}

Army Aviation School

German Army Aviation Corps

History of the German Army Aviation Corps

German Army

Army aviation

\section{References}\label{references}

\section{Further reading}\label{further-reading}

\begin{itemize}
\item
  \emph{Rudolph, Christin-Desirëe (2012), Soldaten unterm Rotor: die
  Huschrauberverbände der Bundeswehr, Stuttgart: Motorbuch-Verlag,
  ISBN~978-3-613-03413-6}
\item
  \emph{Schütt, Kurt W. (1985), Heeresflieger: Truppengattung der
  dritten Dimension; die Geschichte der Heeresfliegertruppe der
  Bundeswehr, Koblenz: Bernard und Graefe, ISBN~3-7637-5451-2}
\end{itemize}

Garben, Fritz (2006), Fünf Jahrzehnte Heeresflieger: Typen, Taktik und
Geschichte, Lemwerder: Stedinger-Verlag, ISBN~3-927697-45-1

Rudolph, Christin-Desirëe (2012), Soldaten unterm Rotor: die
Huschrauberverbände der Bundeswehr, Stuttgart: Motorbuch-Verlag,
ISBN~978-3-613-03413-6

Schütt, Kurt W. (1985), Heeresflieger: Truppengattung der dritten
Dimension; die Geschichte der Heeresfliegertruppe der Bundeswehr,
Koblenz: Bernard und Graefe, ISBN~3-7637-5451-2

Vetter, Bernd; Vetter, Frank (2001), Die deutschen Heeresflieger:
Geschichte, Typen und Verbände, Stuttgart: Motorbuch-Verlag,
ISBN~3-613-02146-3

\section{External links}\label{external-links}

\begin{itemize}
\item
  \emph{Official website of the German Army Aviation School (in German)}
\end{itemize}

Official website of the German Army Aviation School (in German)

Coordinates: 52°17′03″N 9°05′09″E / 52.28417°N 9.08583°E / 52.28417;
9.08583
