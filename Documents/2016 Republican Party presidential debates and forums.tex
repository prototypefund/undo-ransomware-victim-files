\textbf{From Wikipedia, the free encyclopedia}

https://en.wikipedia.org/wiki/2016\%20Republican\%20Party\%20presidential\%20debates\%20and\%20forums\\
Licensed under CC BY-SA 3.0:\\
https://en.wikipedia.org/wiki/Wikipedia:Text\_of\_Creative\_Commons\_Attribution-ShareAlike\_3.0\_Unported\_License

\section{2016 Republican Party presidential debates and
forums}\label{republican-party-presidential-debates-and-forums}

\begin{itemize}
\item
  \emph{The twelve Republican presidential debates, and the nine forums,
  were a series of political debates held between the candidates for the
  Republican Party's nomination for the 2016 United States presidential
  election.}
\end{itemize}

The twelve Republican presidential debates, and the nine forums, were a
series of political debates held between the candidates for the
Republican Party's nomination for the 2016 United States presidential
election.

\section{Presidential debates}\label{presidential-debates}

\section{Schedule}\label{schedule}

\begin{itemize}
\item
  \emph{The GOP candidates debated twice in January and three times in
  February 2016.}
\item
  \emph{The Republican National Committee announced the 2015--2016
  debate schedule on January 16, 2015.}
\item
  \emph{One debate per month followed through December 2015.}
\item
  \emph{Due to the number of candidates running for nomination, Fox News
  aired two separate debates on August 6, with the less popular
  candidates going first, followed by the candidates with more support
  in the 'prime time' debate.}
\end{itemize}

The Republican National Committee announced the 2015--2016 debate
schedule on January 16, 2015. It revealed that 12 debates would be held,
in contrast to the 20 debates that were held from 2011 to 2012. The
announcement included which news organizations would host each debate,
with Fox News and CNN having three each; and one each for ABC, CBS, NBC,
CNBC, Fox Business Network, and a conservative media outlet to be
announced. It had some changes during the primary.

The first live-broadcast debate occurred on Thursday, August 6, 2015, at
the Quicken Loans Arena in Cleveland, Ohio. It was seen on the Fox News
Channel by 24 million viewers, making the debate the most watched live
broadcast for a non-sporting event in cable television history. Due to
the number of candidates running for nomination, Fox News aired two
separate debates on August 6, with the less popular candidates going
first, followed by the candidates with more support in the 'prime time'
debate.

One debate per month followed through December 2015.\\
The GOP candidates debated twice in January and three times in February
2016. On February 20, 2016, the RNC announced a thirteenth debate, which
was to be held in Salt Lake City, Utah, on Monday, March 21. This would
have made for three debates in March, but the event was later canceled
due to all but one of the candidates opting not to attend.

The following table lists the twelve RNC debates which took place, along
with the dates, times, places, hosts, and participants.

\section{Polling effect}\label{polling-effect}

\begin{itemize}
\item
  \emph{The rhetoric about the pros and cons of the debate criteria, and
  the use of polls to winnow the field, partially displaced more
  substantive discussions of concrete policies that candidates are
  proposing.}
\item
  \emph{Candidates ranked from 8th to 12th place in the polls prior to
  the August 2015 debate, which included Chris Christie, Rick Perry, and
  John Kasich, downplayed the importance of being invited to any
  specific debate by emphasizing that delegate selection in early states
  is more important.}
\end{itemize}

The use of polling data had initially been criticized by polling firms
such as highly regarded Marist. Prior to the first debate, Marist
decided to temporarily suspend its national polling of preferences for
the Republican nominee on the basis that using non-scientific polling
data to select the bipartisan debate field puts polling firms such as
Marist under pressure to produce high-precision results that are
inherently impossible to provide at that point in time.

For instance, it would be difficult to determine the margin of error in
any statistical sampling process like a preference poll (see statistical
tie for tenth place, and more generally the independence of clones).

FiveThirtyEight made the point of the varying degrees of discretion the
television networks gave themselves with their distinct debate
invitation criteria, noting that polling data can only be seen as an
objective method for selection of the debate participants, if the full
and exact criteria are made clear in advance. The rhetoric about the
pros and cons of the debate criteria, and the use of polls to winnow the
field, partially displaced more substantive discussions of concrete
policies that candidates are proposing.

In terms of many GOP candidates, the use of polls to winnow the field
was criticized, especially by candidates with relatively low-polling
numbers in August, including Rick Santorum and Lindsey Graham, who both
said through the media that their exclusion from the main debates could
prevent them from being competitive in the primaries and caucuses.

Candidates ranked from 8th to 12th place in the polls prior to the
August 2015 debate, which included Chris Christie, Rick Perry, and John
Kasich, downplayed the importance of being invited to any specific
debate by emphasizing that delegate selection in early states is more
important. Christie and fellow one-time candidate Rand Paul both had
made the point that the early debates would give candidates a chance to
communicate policy ideas to voters, and would subsequently be helpful in
giving voters the information needed to decide which candidate to
support.

Some in the media questioned Donald Trump's seriousness as a candidate,
and pondered as to whether or not he should be included in the debates.
Trump filed FEC paperwork to make his run official; however, despite
doing well in the early polling which effectively guaranteed him an
invitation to the Fox News and CNN debates, Trump expressed ambivalence
about the value of the debates to his own campaign (saying he was not a
debater and therefore did not know how well he would perform in one),
and to the process in general (saying that politicians are always
debating with little in the way of results).

\section{Logistics}\label{logistics}

\begin{itemize}
\item
  \emph{For each of the debates held from August 2015 through January
  2016, the sponsoring television network conducted both a debate
  broadcast in prime time preceded by another debate in the afternoon or
  early evening; the candidates who ranked higher in the polls were
  invited to the prime time debate, with lower-ranking candidates
  admitted to the earlier debate only.}
\end{itemize}

With as many as 17 major candidates vying for the nomination, the
prospect of including all the candidates in a debate presented
logistical difficulties. For each of the debates held from August 2015
through January 2016, the sponsoring television network conducted both a
debate broadcast in prime time preceded by another debate in the
afternoon or early evening; the candidates who ranked higher in the
polls were invited to the prime time debate, with lower-ranking
candidates admitted to the earlier debate only. The earlier debates for
the lower-ranked candidates were nicknamed the "undercard" debates or
the "kids' table" debates.

\section{Ratings}\label{ratings}

\begin{itemize}
\item
  \emph{The following table lists the ratings (number of estimated
  viewers) of the debates to date.}
\end{itemize}

The following table lists the ratings (number of estimated viewers) of
the debates to date.

\section{August 6, 2015 -- Cleveland,
Ohio}\label{august-6-2015-cleveland-ohio}

\begin{itemize}
\item
  \emph{The lower tier debate was the first and only debate appearance
  of former Texas governor Rick Perry, who dropped out of the race less
  than a month later, after he failed to qualify for the second
  primetime debate and said that this was damaging to his fundraising
  abilities.}
\item
  \emph{The first debate was hosted by Fox News Channel, Facebook, and
  the Ohio Republican Party at the Quicken Loans Arena in Cleveland,
  Ohio -- the same location as the future 2016 Republican National
  Convention.}
\end{itemize}

The first debate was hosted by Fox News Channel, Facebook, and the Ohio
Republican Party at the Quicken Loans Arena in Cleveland, Ohio -- the
same location as the future 2016 Republican National Convention. The
two-hour debate invited the 10 highest-polling candidates, as measured
by the average of the top five national polls selected by Fox. In
addition, all other candidates who were "consistently being offered" as
choices in national polls were invited to a one-hour debate earlier that
same day. (Originally, the non-primetime debate had a minimum
requirement that invitees were averaging at least 1\% in Fox-recognized
national polls, and was to be aired at noon for a total of two hours in
duration.) The two-tiered debate hosted by Fox News on the 6th was
qualitatively different from the C-SPAN forum held on the 3rd, for at
least three reasons: it was a debate rather than a forum, where
candidates were allowed to challenge each other, not just speak one at a
time sequentially; it was divided into two tiers based on national
polling numbers, only a subset of the candidates were on-stage (during
each of the two distinct Fox News airtimes); and finally, Donald Trump
and Mike Huckabee were participants in the primetime tier, but did not
appear at the C-SPAN forum.

The candidates in the main debate were Donald Trump, Jeb Bush, Scott
Walker, Mike Huckabee, Ben Carson, Ted Cruz, Marco Rubio, Rand Paul,
Chris Christie, and John Kasich. The moderators were Bret Baier, Megyn
Kelly, and Chris Wallace. Seven candidates who did not qualify were
invited to participate in the 5 p.m. forum; these were Rick Perry, Bobby
Jindal, Rick Santorum, Lindsey Graham, Carly Fiorina, Jim Gilmore, and
George Pataki; the moderators for this debate were Bill Hemmer and
Martha McCallum. Because of a rule-change announced by FOX one week
before the debate-invitations went out, Graham, Pataki, and Gilmore were
allowed to participate at 5 p.m. despite averaging below 1\% in the five
selected polls. (Former IRS Commissioner Mark Everson was excluded from
the 5 p.m. tier, along with other relatively unknown candidates who did
not meet the updated invitation-criteria of "consistently being offered
to respondents in major national polls as recognized by Fox News.") The
five selected polls were conducted by Fox News, Bloomberg, CBS News,
Monmouth University, and Quinnipiac University.

In the main event, Trump was afforded the most time to speak at the
debate by the Fox moderators (at 10 minutes, 32 seconds) followed by
Bush (8:31), Huckabee (6:40), Cruz (6:39), Kasich (6:31), Carson (6:23),
Rubio (6:22), Christie (6:10), Walker (5:51), and Paul (5:10). The
debate itself was viewed by 24 million people at its peak, setting
records for the most-watched presidential primary debate ever and the
highest-rated non-sports telecast in cable television history.

The two different debates received rather different analyses in terms of
the performances of the candidates. In the lower tier debate with only 7
candidates, Carly Fiorina was overwhelmingly considered the best
debater, while Perry and Jindal were also praised, and Gilmore, Graham,
Pataki, and Santorum were criticized. In the primetime debate,
frontrunner Donald Trump's overall performance was criticized as rude
and erratic by many pundits, while others said his comments were popular
and his criticisms were overdue, including his criticism of Bush's
description of illegal immigration as an "act of love." Cruz, Rubio,
Christie, and Huckabee received praise. Notable conflicts between
candidates included Rand Paul vs. Christie over the NSA surveillance
program, Paul vs. Trump on the latter's possible third-party run, Paul
vs. Trump on healthcare, and Christie vs. Huckabee on the issue of
welfare reform. Trump also clashed with two of the moderators -- Kelly
and Wallace -- on the issue of sexism with Kelly, and the issue of
illegal immigration with Wallace (specifically, Trump's claims that the
Mexican government was deliberately sending criminals into America
illegally).

The lower tier debate was the first and only debate appearance of former
Texas governor Rick Perry, who dropped out of the race less than a month
later, after he failed to qualify for the second primetime debate and
said that this was damaging to his fundraising abilities.

\section{September 16, 2015 -- Simi Valley,
California}\label{september-16-2015-simi-valley-california}

\begin{itemize}
\item
  \emph{Similar to the Fox News-sponsored debate in Cleveland, but with
  slightly different ranking-criteria, the debate was split into
  primetime and pre-primetime groups based on averaged polling numbers.}
\item
  \emph{This 2015 debate was aired on CNN, and simulcast on the Salem
  Radio Network.}
\item
  \emph{The primary focus of the debate was on Carly Fiorina, the one
  and only candidate who rose from the "undercard" tier of the previous
  debate into the primetime debate this time around.}
\end{itemize}

The second debate took place at (and was co-sponsored by) the Ronald
Reagan Presidential Library, which previously hosted two of the
Republican debates in 2008 -- the first and penultimate ones. This 2015
debate was aired on CNN, and simulcast on the Salem Radio Network.
Similar to the Fox News-sponsored debate in Cleveland, but with slightly
different ranking-criteria, the debate was split into primetime and
pre-primetime groups based on averaged polling numbers. The primetime
debate was originally planned to include the candidates ranking in the
top ten, as measured by nationwide polling performed by specific firms,
averaged across polls that are released between July 16 and September
10; the rules were later changed to allow candidates placing in the top
ten in polls from August 7 through September 10 to qualify as well. This
change was made due to an unexpected scarcity of polls taken after the
August 6 debate, which would otherwise have been particularly
disadvantageous to Carly Fiorina, who had significantly increased her
support in polls taken after that debate but who would otherwise have
been kept out of the primetime debate due to her minimal support in the
large number of polls taken before the August 6 debate.

Eleven candidates participated in the prime time debate: Jeb Bush, Ben
Carson, Chris Christie, Ted Cruz, Carly Fiorina, Mike Huckabee, John
Kasich, Rand Paul, Marco Rubio, Donald Trump, and Scott Walker.\\
The candidates in the undercard debate were Bobby Jindal, Lindsey
Graham, Rick Santorum, and George Pataki. Rick Perry had been invited to
the undercard debate but suspended his campaign on September 11,
effectively ending his candidacy. Former governor Jim Gilmore did not
qualify for either debate.

The undercard broadcast took place at 3 PM PDT, while the main card
broadcast took place at 5 PM PDT. The two-tiered CNN broadcasts were
consecutive, with the primetime debate immediately following the
second-tier broadcast. The moderator was Jake Tapper of CNN, with
participation by Hugh Hewitt and Dana Bash. The primetime debate, like
the first on Fox News, was a massive ratings success with nearly 23
million viewers, roughly 1 million less than the previous debate, and
setting the record for the highest-rated broadcast in CNN's history.

The primary focus of the debate was on Carly Fiorina, the one and only
candidate who rose from the "undercard" tier of the previous debate into
the primetime debate this time around. After the debate, most analysts
believed that she successfully solidified her newfound status as a
top-tier candidate, and successfully defended herself against attacks by
Donald Trump. Marco Rubio was also largely viewed as the other strong
performer of the night, and both Fiorina's and Rubio's poll numbers
began to increase significantly in the wake of this debate. Additional
candidates who received praise included Mike Huckabee and Chris
Christie, while frontrunner Donald Trump, former Florida governor Jeb
Bush, and Ohio governor John Kasich were largely criticized. Notable
exchanges in the debate were Trump vs Fiorina on the former's comments
about her face and on each other's business records, Trump vs Paul on
the formers insults, Trump vs Bush on immigration, Trump vs Bush on
casino gambling in Florida, women's health issues, and foreign policy,
as well as Christie and Bush against Paul on marijuana legalization.
This was the second and final debate appearance by Wisconsin governor
Scott Walker, who dropped out of the race five days later, saying that
the subsequent decrease of his own poll numbers and fundraising were
largely due to his two debate performances being largely panned by
commentators.

During the less formal section of the debate the candidates were asked
what secret service nicknames they'd choose were they to become
President of the United States. The candidates answered as follows; Jeb
Bush: Eveready (a reference to Donald Trump's earlier "low-energy"
criticism), Ben Carson: One Nation, Chris Christie: True Heart, Ted
Cruz: Cohiba, Carly Fiorina: Secretariat, Mike Huckabee: Duck Hunter,
John Kasich: Unit 2 (his wife is Unit 1), Rand Paul: Justice Never
Sleeps (Tapper suggest it might be a mouthful), Marco Rubio: Gator,
Donald Trump: Humble, and Scott Walker: Harley.

\section{October 28, 2015 -- Boulder,
Colorado}\label{october-28-2015-boulder-colorado}

\begin{itemize}
\item
  \emph{Many commentators considered the winners of the primetime debate
  to be Rubio, Cruz, and Christie, primarily for their attacks on the
  debate's moderator's questions.}
\item
  \emph{On October 30, 2015, the RNC announced that NBC News would no
  longer host the February 26, 2016, debate in Houston.}
\item
  \emph{In response to the previous debate on CNN running over three
  hours in length, the top two highest-polling candidates -- Donald
  Trump and Ben Carson -- teamed up in a threatened boycott of the CNBC
  debate.}
\end{itemize}

The third debate was held on October 28 at the University of Colorado in
Boulder, which is also one of the sponsors. CNBC stated that the debate
would focus on the economy. The moderators were announced as John
Harwood, Carl Quintanilla, and Becky Quick, with additional questions to
be asked by Rick Santelli, Sharon Epperson, and Jim Cramer.

On September 30, CNBC announced that all candidates with an average of
2.5\% or better in polls conducted by NBC, Fox, CNN, ABC, CBS, and
Bloomberg released in the five weeks (from September 17 through October
21 inclusive) before the October 28 debate would be invited to
participate in the primetime debate at 8 p.m. EDT. All other candidates
receiving at least 1\% in any one of the polls conducted by the six
recognized firms would be invited to the undercard debate at 6 p.m. EDT.

In response to the previous debate on CNN running over three hours in
length, the top two highest-polling candidates -- Donald Trump and Ben
Carson -- teamed up in a threatened boycott of the CNBC debate. They
demanded that the debate be limited to no longer than two hours, and
also that opening and closing statements be included in those two hours;
otherwise, if these conditions were not met, both Trump and Carson would
withdraw from the debate. On October 16, CNBC announced that it had
accepted the demands of Trump and Carson, setting the two-hour maximum
and allowing for opening and closing statements.

On October 21, CNBC announced that 10 candidates (Bush, Carson,
Christie, Cruz, Fiorina, Huckabee, Kasich, Paul, Rubio and Trump) would
take the stage shortly after 8 p.m. EDT, with four candidates (Graham,
Jindal, Santorum, and Pataki) on stage about two hours earlier. Whether
Jindal would participate was unclear; he said on October 20 that he
might skip the debate if the criteria for the main group was not
changed. On October 27, Jindal's press secretary said he would
participate: "We just thought about it and he's always been ready to
debate at any time."

The primetime debate featured numerous clashes between the candidates
and the moderators, and the moderators were criticized -- both by the
candidates and by commentators in the aftermath -- for perceived
rudeness towards the candidates, asking questions that were perceived as
biased. Cruz, Rubio, Christie, Huckabee, Trump, and Carson all
criticized the moderators at some point or another, and often received
the loudest applause as a result. The CNBC moderators were also
criticized by some news outlets, such as The Daily Beast, with Bill
Maher, Byron York, and Stuart Rothenberg also criticizing the
moderators. Other news sources have come out in support of CNBC's
vigorous vetting of the candidates, such as The Guardian. At Salon,
Huffington Post Senior Media Editor Jack Mirkinson described CNBC's
handling of the debate as "catastrophic", while feminist blogger and
former John Edwards Campaign Blogmaster Amanda Marcotte described the
questions as "substantive".

Many commentators considered the winners of the primetime debate to be
Rubio, Cruz, and Christie, primarily for their attacks on the debate's
moderator's questions. The two front-runners in the polls at the time,
Trump and Carson, did not receive as much focus nor had as many
memorable moments, but were still viewed as doing fairly well. Bush and
Kasich were largely criticized for their performances, particularly when
the former argued with Rubio over Rubio's attendance record as a
Senator, and when the latter clashed with Trump over Kasich's record as
Governor, and his sudden shift in debate strategy from passive to
aggressive, which Trump said was simply being done in response to his
falling poll numbers.

On October 30, 2015, the RNC announced that NBC News would no longer
host the February 26, 2016, debate in Houston. RNC chairman Reince
Priebus showed concerns that an NBC-hosted debate could result in a
"repeat" of the CNBC debate, as both are divisions of NBCUniversal,
although NBC News is editorially separate from CNBC. Priebus explained
that CNBC had conducted the October 28 debate in "bad faith", arguing
that "while debates are meant to include tough questions and contrast
candidates' visions and policies for the future of America, CNBC's
moderators engaged in a series of 'gotcha' questions, petty and
mean-spirited in tone, and designed to embarrass our candidates. What
took place Wednesday night was not an attempt to give the American
people a greater understanding of our candidates' policies and ideas."

\section{November 10, 2015 -- Milwaukee,
Wisconsin}\label{november-10-2015-milwaukee-wisconsin}

\begin{itemize}
\item
  \emph{The undercard debate was the fourth and final debate appearance
  of Governor Bobby Jindal, who ended his campaign on November 17,
  stating "this is not my time."}
\item
  \emph{The fourth debate was held on November 10, 2015, at the
  Milwaukee Theatre in Milwaukee, Wisconsin, airing on the Fox Business
  Network and sponsored by The Wall Street Journal.}
\item
  \emph{The official debate lineup was unveiled on November 5.}
\end{itemize}

The fourth debate was held on November 10, 2015, at the Milwaukee
Theatre in Milwaukee, Wisconsin, airing on the Fox Business Network and
sponsored by The Wall Street Journal. This debate focused on jobs,
taxes, and the general health of the U.S. economy, as well as on
domestic and international policy issues. The moderators were Neil
Cavuto, Maria Bartiromo and Gerard Baker.

To participate in the main debate, a candidate needed to have an average
of at least 2.5\% in the four most recent recognized national polls
conducted through November 4. Candidates who failed to reach that
average but who scored at least 1\% in any of those four polls were
invited to the secondary debate, which was moderated by Sandra Smith,
Trish Regan, and the Wall Street Journal′s Washington bureau chief,
Gerald Seib.

The official debate lineup was unveiled on November 5. This lineup
noticeably differed from previous lineups in several significant ways:
For the first time in the debate season, there were fewer than ten
candidates in the primetime lineup; that consists of eight candidates:
Trump, Carson, Rubio, Cruz, Bush, Fiorina, Kasich, and Paul. Also, over
half of the original set of lower-tier candidates polling less than 1\%
were excluded from the undercard debate due to failing to reach at least
1\% in some polls, those being Graham, Pataki, and Gilmore. Thus, the
lower-tier debate lineup instead featured Christie and Huckabee -- both
removed from the main stage for the first time -- alongside previous
lower-tier candidates Santorum and Jindal. Notable exchanges included
Trump vs Kasich and Bush on immigration, Trump vs Bush on foreign
policy, Fiorina vs Paul on Russia, and Rubio vs Paul on the formers tax
plan as well as military spending. Cruz and Kasich also had an exchange
on whether they would bail out the banks.

The undercard debate was the fourth and final debate appearance of
Governor Bobby Jindal, who ended his campaign on November 17, stating
"this is not my time."

\includegraphics[width=5.50000in,height=4.12500in]{media/image1.jpg}\\
\emph{Spin room at the Las Vegas debate}

\section{December 15, 2015 -- Las Vegas,
Nevada}\label{december-15-2015-las-vegas-nevada}

\begin{itemize}
\item
  \emph{Eighteen million people watched the debate, making it the
  third-largest audience ever for a presidential primary debate.}
\item
  \emph{The debate lineup was announced on December 13 to include Trump,
  Cruz, Rubio, Carson, Bush, Fiorina, Christie, Paul, and Kasich in the
  primetime debate, and Huckabee, Santorum, Graham, and Pataki in the
  undercard debate.}
\end{itemize}

The fifth debate was held on December 15, 2015, at the Venetian Resort
in Las Vegas, Nevada. It was the second debate to air on CNN, and was
also broadcast by Salem Radio. The debate was moderated solely by Wolf
Blitzer with Dana Bash and Hugh Hewitt serving alongside as questioners.

The debate was split into primetime and pre-primetime groups based on
averaged polling numbers; in order to participate in the main debate,
candidates had to meet one of three criteria in polls conducted between
October 29 and December 13 which were recognized by CNN---either an
average of at least 3.5\% nationally, or at least 4\% in either Iowa or
New Hampshire. The secondary debate featured candidates that had reached
at least 1\% in four separate national, Iowa, or New Hampshire polls
that are recognized by CNN. Paul was included in the main debate after
not qualifying under the original rules because he received 5\% support
in Iowa in a Fox News poll.

The debate lineup was announced on December 13 to include Trump, Cruz,
Rubio, Carson, Bush, Fiorina, Christie, Paul, and Kasich in the
primetime debate, and Huckabee, Santorum, Graham, and Pataki in the
undercard debate. Commentators suggested that the key confrontation
would be between Trump and Cruz, based on their respective polling in
Iowa.

Eighteen million people watched the debate, making it the third-largest
audience ever for a presidential primary debate. During the debate, the
audible coughing was attributed to Ben Carson. His campaign admitted
that they all got sick a month prior and Carson had kept the cough for
weeks. The cough was "almost gone" and Carson was not really sick at the
time. Notable exchanges included Trump vs Bush on foreign policy as well
as the formers proposed Muslim ban, Paul vs Rubio on the NSA and
immigration, as well as Cruz vs Rubio on the NSA metadata program,
foreign policy, and immigration.

The undercard debate was the fourth and final debate appearance of
Senator Lindsey Graham and former Governor George Pataki, who suspended
their campaigns on December 21 and December 29, respectively.

\section{January 14, 2016 -- North Charleston, South
Carolina}\label{january-14-2016-north-charleston-south-carolina}

\begin{itemize}
\item
  \emph{The participants were introduced in order of their poll rankings
  at the debate.}
\item
  \emph{On December 8, 2015, it was announced that Fox Business Network
  would host an additional debate two days after the State of the Union
  address.}
\item
  \emph{On January 11, 2016, seven candidates were revealed to have been
  invited to the prime-time debate: Jeb Bush, Ben Carson, Chris
  Christie, Ted Cruz, John Kasich, Marco Rubio, and Donald Trump.}
\end{itemize}

On December 8, 2015, it was announced that Fox Business Network would
host an additional debate two days after the State of the Union address.
The debate was held in the North Charleston Coliseum in North
Charleston, South Carolina. The anchor and managing editor of Business
News, Neil Cavuto, and anchor and global markets editor, Maria
Bartiromo, reprised their roles as moderators for the prime-time debate,
which began at 9 p.m. EST. The earlier debate, which started at 6 p.m.
EST, was again moderated by anchors Trish Regan and Sandra Smith.

On December 22, 2015, Fox Business Network announced that in order to
qualify for the prime-time debate, candidates had to either: place in
the top six nationally, based on an average of the five most recent
national polls recognized by FOX News; place in the top five in Iowa,
based on an average of the five most recent Iowa state polls recognized
by FOX News; or place in the top five in New Hampshire, based on an
average of the five most recent New Hampshire state polls recognized by
FOX News. In order to qualify for the first debate, candidates must have
registered at least one percent in one of the five most recent national
polls.

On January 11, 2016, seven candidates were revealed to have been invited
to the prime-time debate: Jeb Bush, Ben Carson, Chris Christie, Ted
Cruz, John Kasich, Marco Rubio, and Donald Trump. The participants were
introduced in order of their poll rankings at the debate.

Carly Fiorina, Mike Huckabee, and Rick Santorum participated in the
undercard debate. Rand Paul was also invited to the undercard debate,
but said, "I won't participate in anything that's not first tier because
we have a first tier campaign." The candidates were introduced in order
of their poll rankings. Notable exchanges included Rubio vs Christie on
the latters record, Trump vs Cruz on Cruz's eligibility and "New York
Values", Trump vs Bush and Rubio on trade with China, as well as Cruz vs
Rubio on the formers tax plan as well as illegal immigration records.

\section{January 28, 2016 -- Des Moines,
Iowa}\label{january-28-2016-des-moines-iowa}

\begin{itemize}
\item
  \emph{The undercard debate was only the second of the 2016 cycle to
  which Gilmore was invited; he was also in the August 6, 2015,
  undercard debate, also hosted by Fox News.}
\item
  \emph{It was the last debate before actual voting began with the Iowa
  caucuses on February 1, 2016.}
\item
  \emph{The seventh debate was held in Iowa, which holds the first
  caucuses, and was the second debate to air on Fox News Channel.}
\end{itemize}

The seventh debate was held in Iowa, which holds the first caucuses, and
was the second debate to air on Fox News Channel. As in Fox's first
debate, the moderators were Bret Baier, Megyn Kelly, and Chris Wallace.
It was the last debate before actual voting began with the Iowa caucuses
on February 1, 2016.

The debate was, again, divided into undercard and primetime rounds; to
qualify for the primetime debate, candidates must have, in polls
recognized by FNC, either placed in the top six nationally based on an
average of the five most recent national polls; place in the top five in
Iowa, based on an average of the five most recent Iowa state polls, or
place in the top five in New Hampshire, based on an average of the five
most recent New Hampshire state polls. In order to qualify for the first
debate, candidates must have registered at least one percent in one of
the five most recent national polls.

On January 26, 2016, Jeb Bush, Ben Carson, Chris Christie, Ted Cruz,
John Kasich, Rand Paul, Marco Rubio, and Donald Trump were invited to
the primetime debate. Trump, however, declined to participate due to
prior confrontations with the network and moderator Megyn Kelly, and
instead hosted a town hall with charitable proceeds going to veterans
groups. Carly Fiorina, Jim Gilmore, Mike Huckabee, and Rick Santorum
were invited to the undercard debate. The undercard debate was only the
second of the 2016 cycle to which Gilmore was invited; he was also in
the August 6, 2015, undercard debate, also hosted by Fox News.

Immigration and foreign policy featured prominently in this debate, and
many candidates used the opportunity to criticize the second-place Cruz,
(Rand Paul, Marco Rubio, Chris Christie, and Jeb Bush) who had also been
subjected to attack ads in the weeks before Iowa from prominent
Republican leaders. Christie took on Cruz on the issue of the NSA
metadata program, and Bush took on both Cruz and Rubio on their senate
records and also took on Rubio on his immigration record. Rand Paul and
Ted Cruz had 2 exchanges: the 1st on the NSA and Audit the Fed, and the
2nd on immigration. In particular, Cruz and Rubio (the third-place
candidate at this point in the race) attacked each other's immigration
records.

This was the final debate appearance of Fiorina, Gilmore, Huckabee,
Paul, and Santorum. Huckabee suspended his campaign on February 1, while
Paul and Santorum ended their presidential bids on February 3. Fiorina
and Gilmore were excluded from the following debate, and suspended their
campaigns on February 10 and 12, respectively.

\section{February 6, 2016 -- Goffstown, New
Hampshire}\label{february-6-2016-goffstown-new-hampshire}

\begin{itemize}
\item
  \emph{The eighth debate was held in New Hampshire, the first state to
  hold primaries, was organized by ABC News and the Independent Journal
  Review.}
\item
  \emph{The eighth debate was the first to not feature an undercard
  event for minor candidates.}
\item
  \emph{On February 4, 2016, Jeb Bush, Ben Carson, Chris Christie, Ted
  Cruz, John Kasich, Marco Rubio, and Donald Trump were invited to the
  debate.}
\end{itemize}

The eighth debate was held in New Hampshire, the first state to hold
primaries, was organized by ABC News and the Independent Journal Review.
It was scheduled to be held in the St Anselm's College Institute of
Politics. The eighth debate was the first to not feature an undercard
event for minor candidates. David Muir and Martha Raddatz were
moderaters, along with WMUR political director Josh McElveen and Mary
Katherine Ham.

To participate in the debate, a candidate had to have either placed
among the top three candidates in the popular vote of the Iowa caucus,
or placed among the top six candidates in an average of New Hampshire or
national polls recognized by ABC News. Only polls conducted no earlier
than January 1 and released by February 4 were included in the averages.

On February 4, 2016, Jeb Bush, Ben Carson, Chris Christie, Ted Cruz,
John Kasich, Marco Rubio, and Donald Trump were invited to the debate.
Carly Fiorina and Jim Gilmore were not invited as they did not meet the
criteria.

The introduction of the candidates caused several mishaps, including
Carson missing his introduction and Kasich's introduction being skipped
by the announcers. During the debate, Rubio -- who was perceived as
gaining significant momentum after a close third-place finish in Iowa --
faced attacks from Bush and particularly Christie, who criticized Rubio
for repeating popular talking points rather than debating specifics.
Rubio's poor response to Christie's criticisms led many to consider
Rubio as the loser of the debate, with most of his post-Iowa momentum
severely blunted by the performance. Conversely, Cruz faced attacks from
Carson over allegations that Cruz's campaign, on the night of the Iowa
caucuses, was spreading rumors that Carson had already dropped out of
the race, so as to switch Carson voters to Cruz. There was also another
clash between Trump and Bush over the issue of eminent domain, to the
point where Trump was booed by the audience. Trump subsequently accused
the audience of consisting mostly of Bush's donors and supporters.

Despite Christie's perceived strong performance, this would ultimately
be his final debate appearance, as the governor suspended his campaign
four days later, after finishing 6th in New Hampshire.

\section{February 13, 2016 -- Greenville, South
Carolina}\label{february-13-2016-greenville-south-carolina}

\begin{itemize}
\item
  \emph{This was the ninth and final debate appearance of Bush, who
  suspended his campaign on February 20.}
\item
  \emph{The day before the debate, Ben Carson was invited to join the
  other participants: Jeb Bush, Ted Cruz, John Kasich, Marco Rubio, and
  Donald Trump.}
\item
  \emph{The ninth debate was held in another early primary state, South
  Carolina, and organized by CBS News.}
\end{itemize}

The ninth debate was held in another early primary state, South
Carolina, and organized by CBS News. The debate was moderated by John
Dickerson in the Peace Center in Greenville, started at 9 pm ET and went
for 90 minutes. Major Garrett of CBS and Kimberley Strassel of WSJ also
asked questions. To participate in the debate, a candidate had to have
either (1) placed among the top five candidates in the popular vote of
the New Hampshire primary, (2) placed among the top three candidates in
the popular vote of the Iowa caucuses, or (3) be among the top five
candidates in an average of national and South Carolina polls over the
four weeks beginning January 15 (that are recognized by CBS) and have
received at least 3\% in Iowa or New Hampshire or the South Carolina or
national polls. The day before the debate, Ben Carson was invited to
join the other participants: Jeb Bush, Ted Cruz, John Kasich, Marco
Rubio, and Donald Trump.

The debate was particularly combative, with Trump attacking Bush for his
stances on illegal immigration, his defense of former president George
W. Bush, and his record as governor. Bush also took on Kasich on the
subject of Medicaid. Cruz had exchanges with Trump over Planned
Parenthood and other issues. Cruz also reiterated comments made by Rubio
on Univision and when the latter claimed he could not speak Spanish,
Cruz retorted using the language and demonstrating that he was
bilingual. This was the ninth and final debate appearance of Bush, who
suspended his campaign on February 20.

\section{February 25, 2016 -- Houston,
Texas}\label{february-25-2016-houston-texas}

\begin{itemize}
\item
  \emph{By these criteria, all five remaining candidates, Carson, Cruz,
  Kasich, Rubio, and Trump, qualified for invitation to the debate.}
\item
  \emph{The debate was shifted a day earlier at the same time.}
\item
  \emph{This was the tenth and final debate appearance of Carson, who
  skipped the following debate on March 3, and dropped out of the race
  the following day.}
\end{itemize}

After the caucus in Nevada, the tenth debate was held at the University
of Houston in Houston and broadcast by CNN as its third of four debates,
in conjunction with Telemundo. The debate aired five days before 14
states voted on Super Tuesday, March 1. While the debate was to be held
in partnership with Telemundo's English-language counterpart NBC, RNC
Chairman Reince Priebus announced on October 30, 2015, that it had
suspended the partnership in response to CNBC's "bad faith" in handling
the October 28, 2015, debate. On January 18, 2016, the RNC announced
that CNN would replace NBC News as the main host of the debate, in
partnership with Telemundo and Salem Communications (CNN's conservative
media partner). The debate was shifted a day earlier at the same time.
National Review was disinvited by the Republican National Committee from
co-hosting the debate over its criticism of GOP front-runner Donald
Trump. On February 19, the criteria for invitation to the debate was
announced: in addition to having official statements of candidacy with
the Federal Election Commission and accepting the rules of the debate,
candidates must have received at least 5\% support in one of the first
four election contests held in Iowa, New Hampshire, South Carolina, and
Nevada. By these criteria, all five remaining candidates, Carson, Cruz,
Kasich, Rubio, and Trump, qualified for invitation to the debate. This
was the tenth and final debate appearance of Carson, who skipped the
following debate on March 3, and dropped out of the race the following
day.

Rubio and Cruz both were considered winners, while Trump struggled. Both
attacked Trump on illegal immigration, Israel, his business record,
religious liberty, Planned Parenthood, and healthcare. Trump, after
criticizing Rubio for repeating himself in the New Hampshire debate,
promptly repeated himself several times when talking about healthcare,
which Rubio quickly pointed out, leading to almost a minute of applause.
Cruz also attacked Trump on his record of giving to Democrats as well as
the polls, and attacked Rubio on illegal immigration and his vote to
confirm John Kerry as secretary of state.

\section{March 3, 2016 -- Detroit,
Michigan}\label{march-3-2016-detroit-michigan}

\begin{itemize}
\item
  \emph{Ben Carson said on March 2 he would not be attending the
  debate.}
\item
  \emph{The eleventh debate was held on March 3, 2016, at the Fox
  Theatre in downtown Detroit, Michigan.}
\item
  \emph{The debate drew controversy for an allusion Trump made to his
  penis in response to Rubio's comment about the size of his hands.}
\item
  \emph{It was the third debate to air on Fox News Channel.}
\end{itemize}

The eleventh debate was held on March 3, 2016, at the Fox Theatre in
downtown Detroit, Michigan. It was the third debate to air on Fox News
Channel. Special Report anchor Bret Baier, The Kelly File anchor Megyn
Kelly and Fox News Sunday host Chris Wallace served as moderators. It
led into the Maine, Kansas, Kentucky, Louisiana, Michigan, Mississippi,
Idaho, and Hawaii contests. Fox announced that in order for candidates
to qualify, they must have at least 3 percent support in the five most
recent national polls by March 1 at 5 pm. Ben Carson said on March 2 he
would not be attending the debate. The debate drew controversy for an
allusion Trump made to his penis in response to Rubio's comment about
the size of his hands.

\section{March 10, 2016 -- Coral Gables,
Florida}\label{march-10-2016-coral-gables-florida}

\begin{itemize}
\item
  \emph{The Washington Times cohosted the debate.}
\item
  \emph{This was the twelfth and final debate appearance of Rubio, who
  suspended his campaign on March 15.}
\item
  \emph{It was also the twelfth and final debate appearance of Cruz who
  suspended his campaign on May 3 and the twelfth and final debate
  appearance of Kasich who suspended his campaign on May 4.}
\end{itemize}

The twelfth debate was the fourth and final debate to air on CNN and led
into the Florida, Illinois, North Carolina, Missouri, and Ohio primaries
on March 15. The candidates debated at the University of Miami's
BankUnited Center arena, moderated by Jake Tapper and questioned by CNN
chief political correspondent Dana Bash, Salem Radio Network talk-show
host Hugh Hewitt, and Washington Times contributor Stephen Dinan. The
Washington Times cohosted the debate. The debate was originally
scheduled considering the unlikelihood that a candidate would clinch the
Republican nomination before March 15, due to the overall size of the
field. On the day of the debate, CNN summarized the immediate stakes:
"This debate comes just five days ahead of 'Super Tuesday 3', when more
than 350 delegates are decided, including winner-take-all contests in
Florida and Ohio. Both Trump and Rubio are predicting {[}a win in{]}
Florida. For Trump, a win here would fuel his growing momentum and
further grow his delegate lead; for Rubio, losing his home state could
be the death knell for his campaign." This was the twelfth and final
debate appearance of Rubio, who suspended his campaign on March 15. It
was also the twelfth and final debate appearance of Cruz who suspended
his campaign on May 3 and the twelfth and final debate appearance of
Kasich who suspended his campaign on May 4.

\section{Canceled debate}\label{canceled-debate}

\section{March 21, 2016 -- Salt Lake City,
Utah}\label{march-21-2016-salt-lake-city-utah}

\begin{itemize}
\item
  \emph{As most of the candidates were in Washington for the AIPAC
  conference, several news channels obtained interviews with the
  candidates, to air on the night of the cancelled debate instead.}
\item
  \emph{However, on March 16, 2016, Donald Trump announced that he would
  skip the debate because he was planning to give a "very major speech"
  at an AIPAC conference.}
\end{itemize}

A thirteenth debate was originally announced to take place at the Salt
Palace in Salt Lake City, Utah, airing on Fox News Channel, and
moderated by Bret Baier, Chris Wallace, and Megyn Kelly to lead into the
Arizona, Idaho, and Utah primaries. However, on March 16, 2016, Donald
Trump announced that he would skip the debate because he was planning to
give a "very major speech" at an AIPAC conference. He also said he had
debated enough, with the same questions/answers. Additionally, Kasich
stated that he would only participate in the debate if Trump were in
attendance, while Rubio suspended his campaign on March 15 following his
loss in the Florida primaries, leaving only Ted Cruz. As a result, the
RNC and Fox News Channel announced that the debate had been cancelled
due to the lack of participants, stating that "obviously, there needs to
be more than one participant".

As most of the candidates were in Washington for the AIPAC conference,
several news channels obtained interviews with the candidates, to air on
the night of the cancelled debate instead. Both CNN and Fox News Channel
aired interviews with the three remaining Republican candidates, while
CNN also aired interviews with the two remaining Democratic candidates.

\section{Forums}\label{forums}

\section{Schedule}\label{schedule-1}

\section{August 3, 2015 -- Goffstown, New
Hampshire}\label{august-3-2015-goffstown-new-hampshire}

\begin{itemize}
\item
  \emph{The 2016 Voters First Presidential Forum moderator was Jack
  Heath of WGIR radio, who asked questions of each of the participating
  candidates based on a random draw.}
\item
  \emph{Three major Republican candidates who did not participate were
  Donald Trump (who chose not to attend), Jim Gilmore (who missed the
  cutoff deadline) and Mike Huckabee (who was invited, but did not
  respond).}
\end{itemize}

The 2016 Voters First Presidential Forum moderator was Jack Heath of
WGIR radio, who asked questions of each of the participating candidates
based on a random draw. Candidates each had three opportunities to
speak: two rounds of questions, and a closing statement. Topics of
discussion during the forum were partially selected based on the results
of an online voter survey. The facilities were provided by the New
Hampshire Institute of Politics and Political Library of St. Anselm
College. The forum was organized in response to the top-ten invitation
limitations placed by Fox News and CNN on their first televised debates
(see descriptions below).

Eleven of the candidates were present in person; Senators Ted Cruz, Rand
Paul, and Marco Rubio participated in the forum via satellite to avoid
missing a vote. Three major Republican candidates who did not
participate were Donald Trump (who chose not to attend), Jim Gilmore
(who missed the cutoff deadline) and Mike Huckabee (who was invited, but
did not respond). Mark Everson did not receive an invitation, albeit
after a "serious look".

The Voters First forum was broadcast nationally by C-SPAN as the
originating source media entity, beginning at 6:30~p.m. EDT and
lasting{[}citation needed{]} from 7 to 9 p.m. The event was also
simulcast and/or co-sponsored by television stations KCRG-TV in Iowa,
New England Cable News in the northeast, WBIN-TV in New Hampshire,
WLTX-TV in South Carolina, radio stations New Hampshire Public Radio,
WGIR in New Hampshire, iHeartRadio on the internet (C-SPAN is also
offering an online version of the broadcast), and newspapers the Cedar
Rapids Gazette in Iowa, the Union Leader in New Hampshire, and the Post
and Courier in Charleston, South Carolina. There was a live audience,
with tickets to the event awarded via a lottery.

\section{November 20, 2015 -- Des Moines,
Iowa}\label{november-20-2015-des-moines-iowa}

\begin{itemize}
\item
  \emph{The Presidential Family Forum was held in the Community Choice
  Credit Union Convention Center in Des Moines, Iowa.}
\item
  \emph{Ben Carson, Ted Cruz, Carly Fiorina, Mike Huckabee, Rand Paul,
  Marco Rubio, and Rick Santorum attended the forum hosted by
  evangelical Christian advocacy group The Family Leader.}
\end{itemize}

The Presidential Family Forum was held in the Community Choice Credit
Union Convention Center in Des Moines, Iowa. Ben Carson, Ted Cruz, Carly
Fiorina, Mike Huckabee, Rand Paul, Marco Rubio, and Rick Santorum
attended the forum hosted by evangelical Christian advocacy group The
Family Leader. It was hosted by politician and political activist Bob
Vander Plaats and moderated by political consultant and pollster Frank
Luntz. Protesters interrupted the beginning of the event and were
removed by police.

\section{December 3, 2015 -- Washington,
D.C.}\label{december-3-2015-washington-d.c.}

\begin{itemize}
\item
  \emph{All candidates except Rand Paul attended the eight-hour-long
  forum.}
\item
  \emph{The Republican Jewish Coalition Presidential Candidates Forum
  was held in the Ronald Reagan Building and International Trade Center
  by the lobbyist group Republican Jewish Coalition.}
\end{itemize}

The Republican Jewish Coalition Presidential Candidates Forum was held
in the Ronald Reagan Building and International Trade Center by the
lobbyist group Republican Jewish Coalition. All candidates except Rand
Paul attended the eight-hour-long forum.

\section{January 9, 2016 -- Columbia, South
Carolina}\label{january-9-2016-columbia-south-carolina}

\begin{itemize}
\item
  \emph{Bush, Carson, Christie, Huckabee, Kasich, and Rubio attended.}
\end{itemize}

The Kemp Forum was held in the Columbia Metropolitan Convention Center
by the Jack Kemp Foundation. Bush, Carson, Christie, Huckabee, Kasich,
and Rubio attended. Fiorina was scheduled to attend the forum, but
missed her flight. The forum was moderated by Speaker of the House Paul
Ryan and Senator Tim Scott.

\section{February 17 -- 18, 2016 -- Greenville/Columbia, South
Carolina}\label{february-17-18-2016-greenvillecolumbia-south-carolina}

\begin{itemize}
\item
  \emph{The CNN Republican town halls were held in Greenville, South
  Carolina, and Columbia, South Carolina.}
\end{itemize}

The CNN Republican town halls were held in Greenville, South Carolina,
and Columbia, South Carolina.

\section{February 24, 2016 -- Houston,
Texas}\label{february-24-2016-houston-texas}

\begin{itemize}
\item
  \emph{Trump did not participate in the forum.}
\end{itemize}

Megyn Kelly hosted a two-hour town hall event on the Kelly File with
Kasich, Cruz, Rubio, and Carson in attendance. Trump did not participate
in the forum.

\section{March 29, 2016 -- Milwaukee,
Wisconsin}\label{march-29-2016-milwaukee-wisconsin}

\begin{itemize}
\item
  \emph{The three remaining candidates, Trump, Cruz, and Kasich, all
  participated in the town hall.}
\end{itemize}

CNN hosted a town hall moderated by Anderson Cooper airing from 8-11 PM
EST. The three remaining candidates, Trump, Cruz, and Kasich, all
participated in the town hall.

\section{See also}\label{see-also}

\begin{itemize}
\item
  \emph{Republican Party presidential candidates, 2016}
\item
  \emph{Republican Party presidential primaries, 2016}
\item
  \emph{Libertarian Party presidential debates and forums, 2016}
\item
  \emph{Democratic Party presidential debates and forums, 2016}
\item
  \emph{Green Party presidential debates and forums, 2016}
\end{itemize}

Republican Party presidential primaries, 2016

Republican Party presidential candidates, 2016

Democratic Party presidential debates and forums, 2016

Green Party presidential debates and forums, 2016

Libertarian Party presidential debates and forums, 2016

\section{References}\label{references}

\section{External links}\label{external-links}

\begin{itemize}
\item
  \emph{Video of February 6, 2016 debate in Goffstown, New Hampshire}
\item
  \emph{Video of February 25, 2016 debate in Houston, Texas}
\item
  \emph{Video of March 10, 2016 debate in Miami, Florida}
\item
  \emph{Video of January 28, 2016 debate in Des Moines, Iowa}
\item
  \emph{Video of February 13, 2016 debate in Greenville, South Carolina}
\item
  \emph{Video of March 3, 2016 debate in Detroit, Michigan}
\end{itemize}

Video of August 3, 2015 forum in Goffstown, New Hampshire

Video of August 6, 2015 debate in Cleveland, Ohio

Video of September 16, 2015 debate in Simi Valley, California

Video of October 28, 2015 debate in Boulder, Colorado

Video of November 10, 2015 debate in Milwaukee, Wisconsin

Video of November 20, 2015 forum in Des Moines, Iowa

Video of December 3, 2015 forum in Washington, D.C.

Video of December 15, 2015 debate in Las Vegas, Nevada

Video of January 14, 2016 debate in North Charleston, South Carolina

Video of January 28, 2016 debate in Des Moines, Iowa

Video of February 6, 2016 debate in Goffstown, New Hampshire

Video of February 13, 2016 debate in Greenville, South Carolina

Video of February 25, 2016 debate in Houston, Texas

Video of March 3, 2016 debate in Detroit, Michigan

Video of March 10, 2016 debate in Miami, Florida
