\textbf{From Wikipedia, the free encyclopedia}

https://en.wikipedia.org/wiki/Chittenden\%20County\%2C\%20Vermont\\
Licensed under CC BY-SA 3.0:\\
https://en.wikipedia.org/wiki/Wikipedia:Text\_of\_Creative\_Commons\_Attribution-ShareAlike\_3.0\_Unported\_License

\section{Chittenden County, Vermont}\label{chittenden-county-vermont}

\begin{itemize}
\item
  \emph{The county has over a quarter of Vermont's population and more
  than twice the population of Vermont's second most populous county,
  Rutland.}
\item
  \emph{Chittenden County (/ˈtʃɪtəndən/) is the most populous county in
  the U.S. state of Vermont.}
\item
  \emph{The county also has more than twice the population density of
  Vermont's second most dense county, Washington.}
\end{itemize}

Chittenden County (/ˈtʃɪtəndən/) is the most populous county in the U.S.
state of Vermont. As of the 2010 census, the population was 156,545. The
county's population estimate for 2018 was 164,572. Its shire town (seat)
is Vermont's most populous municipality, the city of Burlington. The
county has over a quarter of Vermont's population and more than twice
the population of Vermont's second most populous county, Rutland. The
county also has more than twice the population density of Vermont's
second most dense county, Washington. The county is named for Vermont's
first governor and one of the framers of its constitution as an
independent republic and later U.S. state, Thomas Chittenden.

The county has most of Vermont's fastest growing municipalities. It is
one of the three counties that comprise the Burlington metropolitan
area, along with the counties of Franklin and Grand Isle to the north
and northwest, respectively. The University of Vermont (UVM), Vermont's
largest university, is located in the county, as well as its affiliated
hospital, the UVM Medical Center (which is Vermont's largest hospital).
Vermont's largest private employer (GlobalFoundries) and largest airport
(Burlington International Airport) are in the localities of Essex
Junction and South Burlington, respectively.

The Vermont Army National Guard is based at Camp Johnson in the town of
Colchester. The Vermont Air National Guard is based at the Burlington
Air National Guard Base on the grounds of the Burlington International
Airport in South Burlington.

\section{Geography}\label{geography}

\begin{itemize}
\item
  \emph{Originally, Chittenden County contained parts of other
  counties.}
\item
  \emph{It is the third-smallest county in Vermont by area.}
\end{itemize}

According to the U.S. Census Bureau, the county has a total area of 619
square miles (1,600~km2), of which 537 square miles (1,390~km2) is land
and 83 square miles (210~km2) (13\%) is water. It is the third-smallest
county in Vermont by area.

Originally, Chittenden County contained parts of other counties. It
included all of today's Franklin, Grand Isle, and Lamoille counties, and
parts of today's Orleans, Washington, and Addison counties.

The town of Underhill in Chittenden County is home to the highest summit
within the state, Mount Mansfield, which has a peak elevation of 4,393
feet (1,339~m) above sea level.

\section{Adjacent counties}\label{adjacent-counties}

\begin{itemize}
\item
  \emph{Washington County -- southeast}
\item
  \emph{Franklin County -- north}
\item
  \emph{Addison County -- south}
\item
  \emph{Grand Isle County -- northwest}
\item
  \emph{Essex County, New York -- west}
\item
  \emph{Lamoille County -- northeast}
\item
  \emph{Clinton County, New York -- northwest}
\end{itemize}

Addison County -- south

Clinton County, New York -- northwest

Essex County, New York -- west

Franklin County -- north

Grand Isle County -- northwest

Lamoille County -- northeast

Washington County -- southeast

\section{Major highways}\label{major-highways}

\begin{itemize}
\item
  \emph{US~2}
\end{itemize}

I-89

I-189

US~2

US~7

US~7 Alt.

VT~2A

VT~F-5

VT~15

VT~17

VT~116

VT~117

VT~127

VT~128

VT~289

\section{Demographics}\label{demographics}

\begin{itemize}
\item
  \emph{2018 U.S. Census Estimates}
\item
  \emph{From 2000 to 2008, residents left Chittenden in high numbers for
  places outside Vermont.}
\item
  \emph{In 2007, census department estimates that Chittenden had the
  youngest average age in the state, 37.5.}
\item
  \emph{In the county, age distribution was as follows: 18.7\% under the
  age of 18, 15.23\% from 18 to 24, 32.05\% from 25 to 44, 20.82\% from
  45 to 64, and 13.2\% who were 65 years of age or older.}
\end{itemize}

2018 U.S. Census Estimates

In 2018, there were 164,572 people, and 67,271 households. There were
67,271 households of which 36.23\% had children under age 18 living with
them, 52.9\% were married couples living together, 7.70\% had a female
householder with no husband present, and 37.70\% were non-families.
24.31\% of all households were made up of individuals and 6.72\% had
someone living alone who was age 65 or older. Average household size was
2.67 and average family size was 3.13.

In 2014, the county was 91.7\% White, 2.4\% Black or African American,
0.3\% Native American and Alaska Native, 3.5\% Asian, 0.01\% Native
Hawaiian and Other Pacific Islander and 2.1\% Two or more races.
Hispanic or Latino of any race made up 2.2\% of the population.

In the county, age distribution was as follows: 18.7\% under the age of
18, 15.23\% from 18 to 24, 32.05\% from 25 to 44, 20.82\% from 45 to 64,
and 13.2\% who were 65 years of age or older. The median age was 34
years. For every 100 females, there were 94.06 males. For every 100
females age 18 and over, there were 92.30 males.

In 2007, census department estimates that Chittenden had the youngest
average age in the state, 37.5. This compares with the actual census in
2000 of 34.2 years.

In 2008, about 29\% of the population lives alone. 59\% of households
consist of families. 38\% of men and 35\% of women, age 15 or older,
have never married. 6\% of the population were born in a foreign
country, 8\% of residents speak a language other than English at home.

From 2000 to 2008, residents left Chittenden in high numbers for places
outside Vermont. Still, population increased slightly, in part due to
immigration from foreign countries.

\section{2010 census}\label{census}

\begin{itemize}
\item
  \emph{The per capita income for the county was \$31,095.}
\item
  \emph{The median income for a household in the county was \$59,878 and
  the median income for a family was \$78,283.}
\item
  \emph{As of the 2010 United States Census, there were 156,545 people,
  61,827 households, and 36,582 families residing in the county.}
\end{itemize}

As of the 2010 United States Census, there were 156,545 people, 61,827
households, and 36,582 families residing in the county. The population
density was 291.7 inhabitants per square mile (112.6/km2). There were
65,722 housing units at an average density of 122.5 per square mile
(47.3/km2).

Of the 61,827 households, 28.8\% had children under the age of 18 living
with them, 46.4\% were married couples living together, 9.1\% had a
female householder with no husband present, 40.8\% were non-families,
and 27.7\% of all households were made up of individuals. The average
household size was 2.37 and the average family size was 2.92. The median
age was 36.2 years.

The median income for a household in the county was \$59,878 and the
median income for a family was \$78,283. Males had a median income of
\$49,991 versus \$39,213 for females. The per capita income for the
county was \$31,095. About 6.6\% of families and 10.8\% of the
population were below the poverty line, including 11.8\% of those under
age 18 and 6.8\% of those age 65 or over.

\section{Government}\label{government}

\begin{itemize}
\item
  \emph{This was the highest in Vermont.}
\item
  \emph{There are no "county taxes."}
\item
  \emph{Remaining county government is judicial.}
\item
  \emph{There is a County Sheriff and Chittenden County Sheriff's
  Department.}
\end{itemize}

As in all Vermont counties, there is a small executive function which is
mostly consolidated at the state level. There is a County Sheriff and
Chittenden County Sheriff's Department. The elected Sheriff is Democrat
Kevin McLaughlin. Remaining county government is judicial. There are no
"county taxes."

In 2007, median property taxes in the county were \$3,809, placing it
265th out of 1,817 counties in the nation with populations over 20,000.
This was the highest in Vermont.

\section{Judicial}\label{judicial}

\begin{itemize}
\item
  \emph{The state's attorney is Sarah George.}
\end{itemize}

The state's attorney is Sarah George.

\section{Elections}\label{elections}

\begin{itemize}
\item
  \emph{In 1928, Chittenden County was won by Democratic Party Al Smith,
  making him the first Democratic candidate to carry the county.}
\item
  \emph{Since Michael Dukakis won the county in 1988, it has been won by
  Democratic candidates ever since and along with Windham County have
  been considered the bluest county in the state of Vermont.}
\end{itemize}

In 1828, Chittenden County voted for National Republican Party candidate
John Quincy Adams and in 1832 voted for Henry Clay.

From William Henry Harrison in 1836 to Winfield Scott in 1852, the
county would vote the Whig Party candidates.

From John C. Frémont in 1856 to Calvin Coolidge in 1924, the Republican
Party would have a 68 year winning streak in the county.

In 1928, Chittenden County was won by Democratic Party Al Smith, making
him the first Democratic candidate to carry the county. The county would
also vote for Franklin D. Roosevelt in all four of his presidential runs
from 1932 to 1944. During that time, Chittenden County, along with
Franklin and Grand Isle Counties would become Democratic enclaves in an
otherwise Republican-voting Vermont. The county would also be won by
Harry S. Truman in 1948.

Dwight D. Eisenhower was able to win back Chittenden County for the
Republicans during the 1952 and 1956.

The county would go to Democratic candidates John F. Kennedy in 1960,
Lyndon B. Johnson in 1964, and Hubert H. Humphrey in 1968.

Incumbent President Richard Nixon would carry the county in 1972 as
would Gerald Ford in 1976.

In 1980, Jimmy Carter would narrowly win the county.

In 1984, Ronald Reagan would become the last Republican presidential
candidate to win Chittenden County.

Since Michael Dukakis won the county in 1988, it has been won by
Democratic candidates ever since and along with Windham County have been
considered the bluest county in the state of Vermont.

\section{Economy}\label{economy}

\section{Personal income}\label{personal-income}

\begin{itemize}
\item
  \emph{The per capita income for the county was \$33,281.}
\item
  \emph{As of the 2010 U.S. Census, the median income for a household in
  the county was \$63,989, and the median income for a family was
  \$59,460.}
\end{itemize}

According to the U.S. Census, the median household income for the years
2007 and 2011 was \$62,260. The per capita income for the same period
was \$32,533.

As of the 2010 U.S. Census, the median income for a household in the
county was \$63,989, and the median income for a family was \$59,460.
Males had a median income of \$38,541 versus \$27,853 for females. The
per capita income for the county was \$33,281. About 4.90\% of families
and 8.80\% of the population were below the poverty line, including
8.00\% of those under age 18 and 8.20\% of those age 65 or over.

\section{Industry}\label{industry}

\begin{itemize}
\item
  \emph{GlobalFoundries is the largest private employer in the state of
  Vermont, with approximately 3,000 employees.}
\item
  \emph{Essex Junction is home to GlobalFoundries' Burlington Design
  Center and 200~mm wafer fabrication plant.}
\end{itemize}

Essex Junction is home to GlobalFoundries' Burlington Design Center and
200~mm wafer fabrication plant. GlobalFoundries is the largest private
employer in the state of Vermont, with approximately 3,000 employees.

Burton Snowboards employs 500 people with a payroll of \$28 million in
2008.

\section{Retailing}\label{retailing}

\begin{itemize}
\item
  \emph{In 2007, Chittenden led the state with 29\% of sales, as
  measured by sales tax reports.}
\item
  \emph{Four local cities stood among the top five areas in the state:
  1- Williston, 2-South Burlington, 4-Colchester, and 5-Burlington.}
\end{itemize}

One measure of economic activity is retail sales. In 2007, Chittenden
led the state with 29\% of sales, as measured by sales tax reports. This
amounted to US\$1.52 billion. Four local cities stood among the top five
areas in the state: 1- Williston, 2-South Burlington, 4-Colchester, and
5-Burlington.

\section{Real estate}\label{real-estate}

\begin{itemize}
\item
  \emph{In 2008, a vacancy rate for office space reached 11\%, and was
  called "historic."}
\end{itemize}

In 2008, a vacancy rate for office space reached 11\%, and was called
"historic."

\section{Education}\label{education}

\begin{itemize}
\item
  \emph{There are several school districts within the county, including
  Burlington, Winooski and Chittenden East.}
\end{itemize}

There are several school districts within the county, including
Burlington, Winooski and Chittenden East. Teachers salaries in 2007--8
varied from lows of \$33,000 to \$38,000 annually. Top salaries ranged
from \$66,000 to \$79,000. Teachers pay from 10--20\% of their health
premiums with many contracts at 12\%.

\section{Higher education}\label{higher-education}

\begin{itemize}
\item
  \emph{A branch of the Community College of Vermont is located in
  Winooski and a satellite campus of Vermont Technical College is in
  Williston.}
\item
  \emph{Chittenden County is home to the University of Vermont and
  Champlain College, which are located in the city of Burlington.}
\end{itemize}

Chittenden County is home to the University of Vermont and Champlain
College, which are located in the city of Burlington. Saint Michael's
College, the Vermont Center of Southern New Hampshire University, and a
branch campus of Albany College of Pharmacy and Health Sciences
(Vermont's first pharmacy school) are in the town of Colchester. A
branch of the Community College of Vermont is located in Winooski and a
satellite campus of Vermont Technical College is in Williston.

\section{Personal health and safety}\label{personal-health-and-safety}

\begin{itemize}
\item
  \emph{The top county in Vermont was Chittenden.}
\end{itemize}

In the first national survey by Robert Wood Johnson and the University
of Wisconsin in 2010, Vermont ranked the highest in the country for
health outcomes. The top county in Vermont was Chittenden.

\section{Infrastructure}\label{infrastructure}

\begin{itemize}
\item
  \emph{Consistent with the rest of New England and other counties in
  the state of Vermont, the county has little formal county government.}
\item
  \emph{One is the Chittenden County Solid Waste District.}
\end{itemize}

Consistent with the rest of New England and other counties in the state
of Vermont, the county has little formal county government. There are a
few agencies that serve county-wide. One is the Chittenden County Solid
Waste District.

\section{Solid waste}\label{solid-waste}

\begin{itemize}
\item
  \emph{In 2008, the Solid Waste District announced that it would charge
  trash haulers \$17/ton for recyclables.}
\end{itemize}

In 2008, the Solid Waste District announced that it would charge trash
haulers \$17/ton for recyclables. Formerly it was paying \$7/ton. The
global economy has reduced the demand for recycled materials.

\section{Roads}\label{roads}

\begin{itemize}
\item
  \emph{Four of the interchanges provide direct access to U.S. Route 2,
  which parallels the interstate throughout most of the county.}
\item
  \emph{Interstate 89 crosses Chittenden County initially from east to
  west, then makes a northward turn in South Burlington to run north
  along the Lake Champlain shoreline.}
\item
  \emph{There are seven interchanges within the county.}
\end{itemize}

Interstate 89 crosses Chittenden County initially from east to west,
then makes a northward turn in South Burlington to run north along the
Lake Champlain shoreline. The full trajectory is generally from
southeast to northwest. There are seven interchanges within the county.
Four of the interchanges provide direct access to U.S. Route 2, which
parallels the interstate throughout most of the county. U.S. Route 7,
the county's main north-south surface route, is also directly accessible
from two interchanges.

The Chittenden County Metropolitan Planning Organization measures
traffic, analyzes road conditions, and allocates federal and state funds
accordingly.

\section{Athletics}\label{athletics}

\begin{itemize}
\item
  \emph{There is a private, amateur Champlain Valley Swim League with
  nine members, mostly from Chittenden.}
\end{itemize}

There is a private, amateur Champlain Valley Swim League with nine
members, mostly from Chittenden.

\section{Communities}\label{communities}

\section{Cities}\label{cities}

\begin{itemize}
\item
  \emph{South Burlington}
\item
  \emph{Burlington (shire town / county seat)}
\end{itemize}

Burlington (shire town / county seat)

South Burlington

Winooski

\section{Towns}\label{towns}

\section{Villages}\label{villages}

\begin{itemize}
\item
  \emph{Essex Junction}
\item
  \emph{Jericho}
\end{itemize}

Essex Junction

Jericho

\section{Census-designated places}\label{census-designated-places}

\begin{itemize}
\item
  \emph{Richmond}
\item
  \emph{Hinesburg}
\item
  \emph{Shelburne}
\item
  \emph{Milton}
\end{itemize}

Hinesburg

Milton

Richmond

Shelburne

\section{Unincorporated communities}\label{unincorporated-communities}

\begin{itemize}
\item
  \emph{In Vermont, gores and grants are unincorporated portions of a
  county which are not part of any town and have limited self-government
  (if any, as many are uninhabited).}
\end{itemize}

In Vermont, gores and grants are unincorporated portions of a county
which are not part of any town and have limited self-government (if any,
as many are uninhabited).

Buels Gore

Jonesville

\section{See also}\label{see-also}

\begin{itemize}
\item
  \emph{USS Chittenden County (LST-561)}
\item
  \emph{National Register of Historic Places listings in Chittenden
  County, Vermont}
\end{itemize}

National Register of Historic Places listings in Chittenden County,
Vermont

USS Chittenden County (LST-561)

\section{References}\label{references}

\section{External links}\label{external-links}

\begin{itemize}
\item
  \emph{National Register of Historic Places listing for Chittenden Co.,
  Vermont}
\item
  \emph{Chittenden County Sheriff's Office}
\end{itemize}

Chittenden County Sheriff's Office

National Register of Historic Places listing for Chittenden Co., Vermont

Lake Champlain Regional Chamber of Commerce Business and tourism
information.

Coordinates: 44°27′N 73°05′W / 44.45°N 73.09°W / 44.45; -73.09
