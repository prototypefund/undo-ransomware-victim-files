\textbf{From Wikipedia, the free encyclopedia}

https://en.wikipedia.org/wiki/Aeronautics\%20Research\%20Mission\%20Directorate\\
Licensed under CC BY-SA 3.0:\\
https://en.wikipedia.org/wiki/Wikipedia:Text\_of\_Creative\_Commons\_Attribution-ShareAlike\_3.0\_Unported\_License

\section{Aeronautics Research Mission
Directorate}\label{aeronautics-research-mission-directorate}

\begin{itemize}
\item
  \emph{The Aeronautics Research Mission Directorate (ARMD) is one of
  four mission directorates within NASA, the other three being the Human
  Exploration and Operations Directorate, the Science Directorate, and
  the Space Technology Directorate.}
\item
  \emph{ARMD performs its aeronautics research at four NASA facilities:
  Ames Research Center and Armstrong Flight Research Center in
  California, Glenn Research Center in Ohio, and Langley Research Center
  in Virginia.}
\end{itemize}

The Aeronautics Research Mission Directorate (ARMD) is one of four
mission directorates within NASA, the other three being the Human
Exploration and Operations Directorate, the Science Directorate, and the
Space Technology Directorate. The ARMD is responsible for NASA's
aeronautical research, which benefits the commercial, military, and
general aviation sectors.

ARMD is involved in the creation of the Next Generation Air
Transportation System (NextGen).

The current NASA associate administrator heading ARMD is Jaiwon Shin. He
has held the position since 2008, after serving four years as deputy
associate administrator for the directorate.

A 2014 audit by the NASA Office of Inspector General reported that ARMD
"solicits input from industry, academia, and other Federal agencies
regarding research needs and...uses this information to develop its
research plans", and concluded that the directorate supported
"advancement of the nation's civil aeronautics research and technology
objectives consistent with the National Plan" established in 2006.

ARMD performs its aeronautics research at four NASA facilities: Ames
Research Center and Armstrong Flight Research Center in California,
Glenn Research Center in Ohio, and Langley Research Center in Virginia.

\section{Funding}\label{funding}

\begin{itemize}
\item
  \emph{\$5 million will go for hypersonics research.}
\item
  \emph{The result was the elimination of much flight research,
  hindering the advance of technologies and causing some research
  projects to collapse.}
\item
  \emph{As of 2011, 56\% of NASA's aeronautics budget went to
  fundamental research, 25\% to integrated systems research, and 14\% to
  facility maintenance.}
\end{itemize}

According to a 2012 report by the National Academies of Sciences,
Engineering, and Medicine, NASA's aeronautics budget declined from over
\$1 billion in 2000 to \$570 million in 2010, while shrinking from
approximately seven percent of NASA's total budget in 2000 to around
three percent in 2010. Its staffing decreased by approximately four
percent between 2006 and 2010. The result was the elimination of much
flight research, hindering the advance of technologies and causing some
research projects to collapse. In addition, the ambition of flight
research projects decreased with respect to technical complexity, risk,
and benefit to the nation. This decreased ambition was attributed to a
risk-averse culture within NASA's aeronautics programs, as well as to
budget reductions.

As of 2011, 56\% of NASA's aeronautics budget went to fundamental
research, 25\% to integrated systems research, and 14\% to facility
maintenance. Its budget breakdown by NASA Center was 32\% to Langley,
25\% to Glenn, 23\% to Ames, 13\% to Dryden (Armstrong), and 7\% to NASA
Headquarters. By expense category, 56\% of the budget was dedicated to
labor costs, 13\% to research announcements, and 30\% to procurement.

For fiscal 2019, its budget request for aeronautics research was cut by
3.3\% to \$634 million after four years between \$640 and \$660 million
before being cut by 2.5\% to \$609 million from fiscal 2020.\\
The supersonic demonstrator for low sonic boom will get \$88 million:
after a preliminary Lockheed Martin design was reviewed in June 2017, a
contract should be awarded in early April 2018 to design and build the
single-seat, single-engine craft before its critical design review
scheduled for fiscal 2019, and flying in January 2021.\\
\$5 million will go for hypersonics research.

\$101 million will be spent on other flight research including the X-57
Maxwell to demonstrate a three times lower energy usage with electric
aircraft in 2019.\\
The AAVP seeks \$231 million for 2019, targeting a 5--10~MW
(6,700--13,400~hp) hybrid airliner turbine-electric propulsion system
focused on superconducting motors.\\
The NEAT should test a megawatt powertrain in fiscal 2019 before the
2.6-megawatt STARC-ABL ingestion system.\\
Boeing's Mach 0.78 Truss-Braced wing concept High-speed wind-tunnel
testing is planned for fiscal 2019.\\
The Airspace Operations and Safety Program (\$91 million in 2019)
includes ATM-X to support urban air mobility in national airspace:
automated trajectory negotiation and management flights are planned for
January 2019, followed by dynamic scheduling and congestion management.

\section{Programs}\label{programs}

\begin{itemize}
\item
  \emph{The Transformative Aeronautics Concepts Program (TACP), which
  creates early-stage concepts, develops computational and experimental
  tools, and awards research grants to industry and university teams.}
\item
  \emph{The ARMD oversees four mission programs:}
\item
  \emph{AAVP projects include research into aeronautics, composite
  materials, supersonic technology, and vertical lift technology.}
\end{itemize}

The ARMD oversees four mission programs:

The Advanced Air Vehicles Program (AAVP), which develops technologies to
improve vehicle performance. AAVP projects include research into
aeronautics, composite materials, supersonic technology, and vertical
lift technology.

The Airspace Operations and Safety Program (AOSP), which works with the
FAA to develop technologies to support NextGen and improve navigation
automation and safety.

The Integrated Aviation Systems Program (IASP), which includes the
Environmentally Responsible Aviation (ERA) project and the integration
of unmanned aircraft systems into the National Airspace System, and
conducts flight test operations.

The Transformative Aeronautics Concepts Program (TACP), which creates
early-stage concepts, develops computational and experimental tools, and
awards research grants to industry and university teams.

\section{Advanced Air Transportation Technology
project}\label{advanced-air-transportation-technology-project}

\begin{itemize}
\item
  \emph{It could be coupled with active gust load alleviation from NASA
  Langley and the X-56A flexible wing for active flutter-suppression.}
\item
  \emph{The Passive Aeroelastic Tailored (PAT) wing was designed for
  more structural efficiency by a team of the ARMD, the University of
  Michigan and Boeing-owned Aurora Flight Sciences.}
\end{itemize}

The Passive Aeroelastic Tailored (PAT) wing was designed for more
structural efficiency by a team of the ARMD, the University of Michigan
and Boeing-owned Aurora Flight Sciences.\\
A 39~ft (12~m) long, 29\% scale of a Boeing 777-like wing was built by
Aurora in Columbus, Mississippi, with a conventional configuration: two
spars and 58 ribs.\\
The skin thickness varies with the load from 0.75~in (19~mm) inboard
tapering to 4~mm (0.16~in) at the tip.\\
To aligns fibers with the load, tow-steered laminates curve along the
wing span unlike current composites with 0°, ±45° and ±90° laid down and
cut plies.\\
Being more flexible but with controlled stiffness, gust loads and
flutter are passively suppressed.\\
Loads tests began in September 2018 and went up to 85\% of the design
limit in October, halted by load oscillations.\\
It could be coupled with active gust load alleviation from NASA Langley
and the X-56A flexible wing for active flutter-suppression.

\section{See also}\label{see-also}

\begin{itemize}
\item
  \emph{NASA GL-10 Greased Lightning}
\item
  \emph{NASA X-57 Maxwell}
\end{itemize}

ecoDemonstrator

NASA GL-10 Greased Lightning

NASA X-57 Maxwell

\section{References}\label{references}
