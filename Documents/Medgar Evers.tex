\textbf{From Wikipedia, the free encyclopedia}

https://en.wikipedia.org/wiki/Medgar\%20Evers\\
Licensed under CC BY-SA 3.0:\\
https://en.wikipedia.org/wiki/Wikipedia:Text\_of\_Creative\_Commons\_Attribution-ShareAlike\_3.0\_Unported\_License

\section{Medgar Evers}\label{medgar-evers}

\begin{itemize}
\item
  \emph{Evers was awarded the 1963 NAACP Spingarn Medal.}
\item
  \emph{A college graduate, Evers became active in the Civil Rights
  Movement in the 1950s.}
\item
  \emph{Medgar's widow, Myrlie Evers, became a noted activist in her own
  right, serving as national chair of the NAACP.}
\item
  \emph{Medgar Wiley Evers (July 2, 1925~-- June 12, 1963) was an
  American civil rights activist in Mississippi, the state's field
  secretary for the NAACP, and a World War II veteran who had served in
  the United States Army.}
\end{itemize}

Medgar Wiley Evers (July 2, 1925~-- June 12, 1963) was an American civil
rights activist in Mississippi, the state's field secretary for the
NAACP, and a World War II veteran who had served in the United States
Army. He worked to overturn segregation at the University of
Mississippi, end the segregation of public facilities, and expand
opportunities for African Americans, which included the enforcement of
voting rights.

A college graduate, Evers became active in the Civil Rights Movement in
the 1950s. Following the 1954 ruling of the United States Supreme Court
in Brown v. Board of Education that segregated public schools were
unconstitutional, Evers challenged the segregation of the
state-supported public University of Mississippi, applying to law school
there. He also worked for voting rights, economic opportunity, access to
public facilities, and other changes in the segregated society. Evers
was awarded the 1963 NAACP Spingarn Medal.

Evers was assassinated in 1963 by Byron De La Beckwith, a member of the
White Citizens' Council. This group was formed in 1954 in Mississippi to
resist the integration of schools and civil rights activism. As a
veteran, Evers was buried with full military honors at Arlington
National Cemetery. His murder and the resulting trials inspired civil
rights protests; his life and these events inspired numerous works of
art, music, and film. All-white juries failed to reach verdicts in the
first two trials of Beckwith in the 1960s. He was convicted in 1994 in a
new state trial based on new evidence.

Medgar's widow, Myrlie Evers, became a noted activist in her own right,
serving as national chair of the NAACP. His brother Charles Evers was
the first African American to be elected as mayor of a city in
Mississippi in the post-Reconstruction era; he won the office in 1969 in
Fayette.

\section{Early life}\label{early-life}

\begin{itemize}
\item
  \emph{Evers was born on July 2, 1925, in Decatur, Mississippi, the
  third of five children (including elder brother Charles Evers) of
  Jesse (Wright) and James Evers.}
\item
  \emph{After the end of the war, Evers was honorably discharged as a
  sergeant.}
\item
  \emph{The Evers family owned a small farm and James also worked at a
  sawmill.}
\end{itemize}

Evers was born on July 2, 1925, in Decatur, Mississippi, the third of
five children (including elder brother Charles Evers) of Jesse (Wright)
and James Evers. The family included Jesse's two children from a
previous marriage. The Evers family owned a small farm and James also
worked at a sawmill. Evers and his siblings walked 12 miles to attend
segregated schools; eventually Medgar earned his high school diploma.

Evers served in the United States Army during World War II from 1943 to
1945. He was sent to the European Theater where he fought in the Battle
of Normandy in June 1944. After the end of the war, Evers was honorably
discharged as a sergeant.

In 1948, Evers enrolled at Alcorn Agricultural and Mechanical College (a
historically black college, now Alcorn State University), majoring in
business administration. He also competed on the debate, football, and
track teams, sang in the choir, and was junior class president. He
earned his Bachelor of Arts in 1952.

On December 24, 1951, he married classmate Myrlie Beasley. Together they
had three children: Darrell Kenyatta, Reena Denise, and James Van Dyke
Evers.

\section{Activism}\label{activism}

\begin{itemize}
\item
  \emph{On June 7, 1963, Evers was nearly run down by a car after he
  came out of the NAACP office in Jackson, Mississippi.}
\item
  \emph{In the weeks before Evers was killed, he encountered new levels
  of hostility.}
\item
  \emph{Evers's civil rights leadership, along with his investigative
  work, made him a target of white supremacists.}
\item
  \emph{On November 24, 1954, Evers was named as the NAACP's first field
  secretary for Mississippi.}
\end{itemize}

The couple moved to Mound Bayou, Mississippi, a town developed by
African Americans, where Evers became a salesman for T. R. M. Howard's
Magnolia Mutual Life Insurance Company.\\
Evers was also president of the Regional Council of Negro Leadership
(RCNL), which began to organize actions for civil rights; Evers helped
organize the RCNL's boycott of gasoline stations that denied blacks the
use of the stations' restrooms.\\
Evers and his brother Charles attended the RCNL's annual conferences in
Mound Bayou between 1952 and 1954, which drew crowds of 10,000 or more.

In 1954, following the U.S. Supreme Court decision that segregated
public schools were unconstitutional, Evers applied to the
state-supported University of Mississippi Law School, but his
application was rejected because of his race. He submitted his
application as part of a test case by the NAACP.

On November 24, 1954, Evers was named as the NAACP's first field
secretary for Mississippi. In this position, he helped organize boycotts
and set up new local chapters of the NAACP. He was involved with James
Meredith's efforts to enroll in the University of Mississippi in the
early 1960s.

Evers also encouraged Dr. Gilbert Mason Sr. in his organizing of the
Biloxi wade-ins from 1959 to 1963, protests against segregation of the
city's public beaches on the Mississippi Gulf Coast. Evers conducted
actions to help integrate Jackson's privately owned buses and tried to
integrate the public parks. He led voter registration drives, and used
boycotts to integrate Leake County schools and the Mississippi State
Fair.

Evers's civil rights leadership, along with his investigative work, made
him a target of white supremacists. Following the Brown v. Board of
Education decision, local whites founded the White Citizens' Council in
Mississippi, and numerous local chapters were started, to resist the
integration of schools and facilities. In the weeks before Evers was
killed, he encountered new levels of hostility. His public
investigations into the 1955 lynching of Chicago teenager Emmett Till in
Mississippi, and his vocal support of Clyde Kennard, had made him a
prominent black leader. On May 28, 1963, a Molotov cocktail was thrown
into the carport of his home. On June 7, 1963, Evers was nearly run down
by a car after he came out of the NAACP office in Jackson, Mississippi.

\section{Assassination}\label{assassination}

\begin{itemize}
\item
  \emph{Evers's family had worried for his safety that day, and Evers
  himself had warned his wife that he felt in greater danger than
  usual.}
\item
  \emph{Medgar Evers lived with the constant threat of death.}
\item
  \emph{When he arrived home, Evers' family was waiting for him and his
  children exclaimed to his wife, Myrlie, that he had arrived.}
\end{itemize}

Medgar Evers lived with the constant threat of death. A large white
supremacist population and the Ku Klux Klan were present in Jackson and
its suburbs. The risk was so high that before his death, Evers and his
wife Myrlie had trained their children on what to do in case of a
shooting, bombing or other kind of attack on their lives. Evers, who was
regularly followed home by at least two FBI cars and one police car,
arrived at his home on the morning of his death without an escort. None
of his usual protection was present, for reasons unspecified by the FBI
or local police. There has been speculation that many members of the
police force at the time were members of the Klan.

In the early morning of June 12, 1963, just hours after President John
F. Kennedy's nationally televised Civil Rights Address, Evers pulled
into his driveway after returning from a meeting with NAACP lawyers.
Evers's family had worried for his safety that day, and Evers himself
had warned his wife that he felt in greater danger than usual. When he
arrived home, Evers' family was waiting for him and his children
exclaimed to his wife, Myrlie, that he had arrived. Emerging from his
car and carrying NAACP T-shirts that read "Jim Crow Must Go", Evers was
struck in the back with a bullet fired from an Enfield 1917 rifle; the
bullet passed through his heart. Initially thrown to the ground by the
impact of the shot, Evers rose and staggered 30 feet (10 meters) before
collapsing outside his front door. His wife Myrlie was the first to find
him. He was taken to the local hospital in Jackson, where he was
initially refused entry because of his race. His family explained who he
was and he was admitted; he died in the hospital 50 minutes
later.{[}full citation needed{]} Evers was the first African American to
be admitted to an all-white hospital in Mississippi, a questionable
achievement for the dying activist. Mourned nationally, Evers was buried
on June 19 in Arlington National Cemetery, where he received full
military honors before a crowd of more than 3,000.

After Evers was assassinated, an estimated 5,000 people marched from the
Masonic Temple on Lynch Street to the Collins Funeral Home on North
Farish Street in Jackson. Allen Johnson, Reverend Martin Luther King and
other civil rights leaders led the procession. The Mississippi police
came prepared with riot gear and rifles in case the protests turned
violent. While tensions were initially high in the stand-off between
police and marchers, both in Jackson and in many similar marches around
the state, leaders of the movement maintained nonviolence among their
followers.

\section{Trials}\label{trials}

\begin{itemize}
\item
  \emph{In 1997, De La Beckwith appealed his conviction in the Evers
  case, but the Mississippi Supreme Court upheld it.}
\item
  \emph{On June 21, 1963, Byron De La Beckwith, a fertilizer salesman
  and member of the White Citizens' Council (and later of the Ku Klux
  Klan), was arrested for Evers' murder.}
\item
  \emph{During the trial, the body of Evers was exhumed for an autopsy.}
\item
  \emph{Myrlie Evers never gave up the fight for a conviction of her
  husband's murderer.}
\end{itemize}

On June 21, 1963, Byron De La Beckwith, a fertilizer salesman and member
of the White Citizens' Council (and later of the Ku Klux Klan), was
arrested for Evers' murder. District Attorney and future governor Bill
Waller prosecuted De La Beckwith. All-white juries in February and April
1964 deadlocked on De La Beckwith's guilt and failed to reach a verdict.
At the time, most blacks were still disenfranchised by Mississippi's
constitution and voter registration practices; this meant they were also
excluded from juries, which were drawn from the pool of registered
voters.

Myrlie Evers never gave up the fight for a conviction of her husband's
murderer. She waited until a new judge had been assigned in the county
to take her case against de la Beckwith back into the courtroom. In
1994, De La Beckwith was prosecuted by the state based on new evidence.
Bobby DeLaughter was the prosecutor. During the trial, the body of Evers
was exhumed for an autopsy. De La Beckwith was convicted of murder on
February 5, 1994, after having lived as a free man for much of the three
decades following the killing. (He had been imprisoned from 1977 to 1980
on separate charges: conspiring to murder A.I. Botnick.) In 1997, De La
Beckwith appealed his conviction in the Evers case, but the Mississippi
Supreme Court upheld it. He died at age 80 in prison on January 21,
2001.

\section{Legacy}\label{legacy}

\begin{itemize}
\item
  \emph{In 2017, the Medgar and Myrlie Evers House was named as a
  National Historic Landmark.}
\item
  \emph{Evers's widow Myrlie Evers co-wrote the book For Us, the Living
  with William Peters in 1967.}
\item
  \emph{Celebrating Evers's life and career, it starred Howard Rollins
  Jr. and Irene Cara as Medgar and Myrlie Evers, airing on PBS.}
\end{itemize}

Evers was memorialized by leading Mississippi and national authors both
black and white: James Baldwin, Margaret Walker, Eudora Welty, and Anne
Moody. In 1963, Evers was posthumously awarded the Spingarn Medal by the
NAACP. In 1969, Medgar Evers College was established in Brooklyn, New
York as part of the City University of New York.

Evers's widow Myrlie Evers co-wrote the book For Us, the Living with
William Peters in 1967. In 1983, a television movie was made based on
the book. Celebrating Evers's life and career, it starred Howard Rollins
Jr. and Irene Cara as Medgar and Myrlie Evers, airing on PBS. The film
won the Writers Guild of America award for Best Adapted Drama.

In 1969, a community pool in the Central District neighborhood of
Seattle, Washington was named after Evers, honoring his life.

On June 28, 1992, the city of Jackson, Mississippi erected a statue in
honor of Evers. All of Delta Drive (part of U.S. Highway 49) in Jackson
was renamed in Evers's honor. In December 2004, the Jackson City Council
changed the name of the city's airport to "Jackson-Medgar Wiley Evers
International Airport" (Jackson-Evers International Airport) in his
honor.

His widow Myrlie Evers became a noted activist in her own right,
eventually serving as national chairperson of the NAACP. Medgar's
brother Charles Evers returned to Jackson in July 1963, and served
briefly with the NAACP in his slain brother's place. He remained
involved in Mississippi civil rights activities for many years, and in
1969, was the first African-American mayor elected in the state. He now
resides in Jackson.

On the 40th anniversary of Evers's assassination, hundreds of civil
rights veterans, government officials, and students from across the
country gathered around his grave site at Arlington National Cemetery to
celebrate his life and legacy. Barry Bradford and three
students---Sharmistha Dev, Jajah Wu, and Debra Siegel, formerly of Adlai
E. Stevenson High School in Lincolnshire, Illinois---planned and hosted
the commemoration in his honor. Evers was the subject of the students'
research project.

In October 2009, Navy Secretary Ray Mabus, a former Mississippi
governor, announced that USNS~Medgar Evers~(T-AKE-13), a Lewis and
Clark-class dry cargo ship, would be named in the activist's honor. The
ship was christened by Myrlie Evers-Williams on November 12, 2011.

In June 2013, a statue of Evers was erected at his alma mater, Alcorn
State University, to commemorate the 50th anniversary of his death.
Alumni and guests from around the world gathered to recognize his
contributions to American society.

Evers was honored in a tribute at Arlington National Cemetery on the
50th anniversary of his death. Former President Bill Clinton, Attorney
General Eric Holder, Navy Secretary Ray Mabus, Senator Roger Wicker, and
NAACP President Benjamin Jealous all spoke commemorating Evers. Evers's
widow, Myrlie Evers-Williams, spoke of his contributions to the
advancement of civil rights:

He was identified as a Freedom hero by The My Hero Project.

In 2017, the Medgar and Myrlie Evers House was named as a National
Historic Landmark. Two years later, in 2019, the site was designated a
National Monument.

\section{In popular culture}\label{in-popular-culture}

\section{Music}\label{music}

\begin{itemize}
\item
  \emph{Wadada Leo Smith's album Ten Freedom Summers contains a track
  called "Medgar Evers: A Love-Voice of a Thousand Years' Journey for
  Liberty and Justice".}
\item
  \emph{Phil Ochs referred to Evers in the song "Love Me, I'm a Liberal"
  and wrote the songs "Another Country" and "Too Many Martyrs" (also
  titled "The Ballad of Medgar Evers") in response to the killing.}
\item
  \emph{Nina Simone wrote and sang "Mississippi Goddam" about the Evers
  case.}
\end{itemize}

Musician Bob Dylan wrote his 1963 song "Only a Pawn in Their Game" about
the assassination. Nina Simone wrote and sang "Mississippi Goddam" about
the Evers case. Phil Ochs referred to Evers in the song "Love Me, I'm a
Liberal" and wrote the songs "Another Country" and "Too Many Martyrs"
(also titled "The Ballad of Medgar Evers") in response to the killing.
Matthew Jones and the Student Nonviolent Coordinating Committee Freedom
Singers recorded a version of the latter song. Wadada Leo Smith's album
Ten Freedom Summers contains a track called "Medgar Evers: A Love-Voice
of a Thousand Years' Journey for Liberty and Justice".

\section{Essays and books}\label{essays-and-books}

\begin{itemize}
\item
  \emph{", in which the speaker is the imagined assassin of Medgar
  Evers, was published in The New Yorker in July 1963.}
\item
  \emph{He added to this account in a book, Never Too Late: A
  Prosecutor's Story of Justice in the Medgar Evers Case (2001).}
\end{itemize}

Eudora Welty's short story, "Where Is the Voice Coming From?", in which
the speaker is the imagined assassin of Medgar Evers, was published in
The New Yorker in July 1963.

Attorney Robert DeLaughter wrote a first-person narrative article
entitled "Mississippi Justice" published in Reader's Digest about his
experiences as state prosecutor in the murder trial. He added to this
account in a book, Never Too Late: A Prosecutor's Story of Justice in
the Medgar Evers Case (2001).

\section{Film}\label{film}

\begin{itemize}
\item
  \emph{Evers was portrayed by James Pickens Jr.}
\item
  \emph{Baldwin recounts the circumstances of and his reaction to
  Evers's assassination.}
\item
  \emph{Beckwith and DeLaughter were played by James Woods and Alec
  Baldwin, respectively; Whoopi Goldberg played Myrlie Evers.}
\end{itemize}

The film Ghosts of Mississippi (1996), directed by Rob Reiner, explores
the 1994 trial of Beckwith in which prosecutor DeLaughter of the Hinds
County District Attorney's office secured a conviction in state court.
Beckwith and DeLaughter were played by James Woods and Alec Baldwin,
respectively; Whoopi Goldberg played Myrlie Evers. Evers was portrayed
by James Pickens Jr. The film was based on a book of the same name.

In the documentary film I Am Not Your Negro (2016), Evers is one of
three black activists (the other two are Martin Luther King Jr and
Malcolm X) who are the focus of reminiscences by author James Baldwin.
Baldwin recounts the circumstances of and his reaction to Evers's
assassination.

\section{See also}\label{see-also}

\begin{itemize}
\item
  \emph{List of civil rights leaders}
\end{itemize}

List of civil rights leaders

\section{References}\label{references}

\section{Further reading}\label{further-reading}

\begin{itemize}
\item
  \emph{Remembering Medgar Evers: Writing the Long Civil Rights
  Movement.}
\end{itemize}

Gwin, Minrose (2013). Remembering Medgar Evers: Writing the Long Civil
Rights Movement. University of Georgia Press. ISBN~9780820335636.

\section{External links}\label{external-links}

\begin{itemize}
\item
  \emph{Medgar Evers on IMDb}
\item
  \emph{Medgar Evers at Find a Grave Retrieved February 22, 2010}
\item
  \emph{Myrlie Evers (28 June 1963).}
\item
  \emph{FBI article: Civil Rights in the `60s: Justice for Medgar Evers}
\item
  \emph{Medgar Evers's FBI file hosted at the Internet Archive}
\item
  \emph{Medgar Evers in the U.S. Federal Census American Civil Rights
  Pioneers}
\item
  \emph{Medgar Evers biography at africawithin.com}
\end{itemize}

SNCC Digital Gateway: Medgar Evers, Documentary website created by the
SNCC Legacy Project and Duke University, telling the story of the
Student Nonviolent Coordinating Committee \& grassroots organizing from
the inside-out

JFK First Draft Condolence Letter to Medgar Evers' Widow, June 12, 1963
Shapell Manuscript Foundation

Audio recording of T. R. M. Howard's eulogy at the memorial service for
Medgar Evers, June 15, 1963, Jackson, Mississippi.

Myrlie Evers (28 June 1963). 'He said he wouldn't mind dying - if ... '.
LIFE. pp.~34--47.

Medgar Evers in the U.S. Federal Census American Civil Rights Pioneers

Medgar Evers biography at africawithin.com

Medgar Evers on IMDb

imbd.com

FBI article: Civil Rights in the `60s: Justice for Medgar Evers

Medgar Evers's FBI file hosted at the Internet Archive

Medgar Evers at Find a Grave Retrieved February 22, 2010
