\textbf{From Wikipedia, the free encyclopedia}

https://en.wikipedia.org/wiki/Northern\_Irish\_football\_clubs\_in\_European\_competitions\\
Licensed under CC BY-SA 3.0:\\
https://en.wikipedia.org/wiki/Wikipedia:Text\_of\_Creative\_Commons\_Attribution-ShareAlike\_3.0\_Unported\_License

\section{Northern Irish football clubs in European
competitions}\label{northern-irish-football-clubs-in-european-competitions}

\begin{itemize}
\item
  \emph{In total, 16 different clubs have represented Northern Ireland
  in European competition.}
\item
  \emph{\\[3\baselineskip]Northern Irish football clubs have
  participated in European football competitions since 1957, when in the
  1957--58 season, Glenavon took part in the European Cup -- the first
  Northern Irish club to do so.}
\end{itemize}

Northern Irish football clubs have participated in European football
competitions since 1957, when in the 1957--58 season, Glenavon took part
in the European Cup -- the first Northern Irish club to do so. In total,
16 different clubs have represented Northern Ireland in European
competition.

All statistics and records are accurate as of 19 June 2017.

\section{History}\label{history}

\begin{itemize}
\item
  \emph{This qualified them to take part in European competition for the
  first, and to date only time -- in the 1976--77 European Cup Winners'
  Cup.}
\item
  \emph{The two most notable successes in Europe are Linfield reaching
  the quarter-finals of the 1966--67 European Cup and Glentoran reaching
  the quarter-finals of the 1973--74 European Cup Winners' Cup.}
\end{itemize}

As of the 2017--18 season, the NIFL Premiership champions qualify for
the UEFA Champions League. The runners-up, the Europa League play-off
winners and the Irish Cup winners qualify for the UEFA Europa League.
If, however, the cup winners have already qualified for Europe by
finishing first or second in the league, the Europa League place goes to
the league's third-placed club.

Only the league champions have ever represented Northern Ireland in the
European Cup/Champions League. Glenavon were the first team ever to
represent Northern Ireland in any of the four competitions, when in the
first round of the 1957--58 European Cup they played out a 0--0 draw
against Danish side AGF in Aarhus, Denmark on 11 September 1957. They
played the home leg two weeks later, losing 3--0. In the 1959--60
season, Linfield became the first Northern Irish club to win a match in
the competition, in what was their first ever European Cup match. They
defeated Swedish side Göteborg 2--1 at Windsor Park on 9 September 1959,
however two weeks later they lost the away leg 6--1 which meant that
they lost the tie 7--3 on aggregate. Linfield hold the record of the
most participations in the European Cup/Champions League by any Northern
Irish club to date, having appeared in 27 different seasons of the
competition up to and including 2012--13.

Participation in the secondary competitions began with Glentoran in the
1962--63 Inter-Cities Fairs Cup against Spanish side Real Zaragoza. A
2--0 defeat in the first leg at the Oval was followed by a 6--2 defeat
away from home in the second leg. They lost the tie 8--2 on aggregate.
Glentoran hold the record of the most participations in the Inter-Cities
Fairs Cup/UEFA Cup/UEFA Europa League by any Northern Irish club to
date, having appeared in 21 different seasons of the competition. They
also hold the record of the most appearances by a Northern Irish club in
European competitions overall, having appeared in 42 different seasons
up to and including 2013--14 -- one season more than Linfield who have
made 41 appearances.

In 1960, Glenavon were drawn to face East German side Wismut in the
first round of the 1960--61 European Cup. However, they were forced to
withdraw when they were refused visas for East Germany and Wismut were
refused visas for the UK. UEFA had allowed the matches to take place in
neutral countries but that was not financially viable for Glenavon, so
they were left with no option but to withdraw. A similar issue arose the
following season when in the first round of the 1961--62 European Cup,
Linfield were drawn to face another East German team, Vorwärts. The away
leg was played, which Linfield lost 3--0. However, Vorwärts were denied
visas for the UK to play the second leg, and similarly to Glenavon the
previous season, travelling to play the game in a neutral country was
not financially viable for Linfield so they were also forced to withdraw
from the competition.

In 1965, Derry City became the first Northern Irish club to win a
two-legged European tie. In the 1965--66 European Cup, the club's last
ever appearance in European competition as a Northern Irish club, they
defeated Lyn 8--6 on aggregate in a high-scoring tie. In the second
round, they faced Anderlecht from Belgium but suffered a huge 9--0 loss
in the away leg and later withdrew from the competition before the
second leg was played, when the Irish Football Association ruled that
their home ground was not up to standard. In 1972, the club withdrew
from senior Northern Irish football and after 13 years of playing
amateur football in lower leagues, they joined the League of Ireland in
the Irish Republic in 1985.

In 1969, both Coleraine and Glentoran entered the Inter-Cities Fairs
Cup, the first time more than one Northern Irish club had ever been
entered into the same European competition. Glentoran lost in the first
round, but Coleraine impressively got through to the second round. The
two most notable successes in Europe are Linfield reaching the
quarter-finals of the 1966--67 European Cup and Glentoran reaching the
quarter-finals of the 1973--74 European Cup Winners' Cup. Since the
restructuring of the competitions based around the UEFA coefficient
system however, the league's relatively low ranking has meant that the
clubs have entered in the early qualifying rounds of either the UEFA
Champions League or the UEFA Cup/Europa League. The third qualifying
round is the furthest any club has progressed in either competition in
their current formats. This was achieved by Cliftonville in the 2010--11
UEFA Europa League.

In 1976, Irish League B Division club Carrick Rangers won the Irish Cup
by defeating strong favourites Linfield 2--1 in the final. This
qualified them to take part in European competition for the first, and
to date only time -- in the 1976--77 European Cup Winners' Cup. They
defeated Aris from Luxembourg 4--3 on aggregate in the first round,
before going out 9--3 on aggregate to English side Southampton in the
second round. To date, this is the only occasion that a club from
outside the top division of football in Northern Ireland has represented
the country in European competition.

The 1993--94 UEFA Champions League saw Linfield drawn to face Dynamo
Tbilisi of Georgia in the preliminary round. After losing 3--2 on
aggregate, they were reinstated when their opponents were expelled from
the competition for allegedly attempting to bribe match officials.
Linfield then went on to face Copenhagen in the first round proper. They
won the first leg 3--0, but lost the second leg 4--0 after extra time.
This proved costly, as victory would have meant a financially lucrative
tie against eventual champions Milan in the next round. In July 2013,
Linfield became the first Northern Irish club to win both the home leg
and the away leg of a European tie. In the 2013--14 UEFA Europa League
first qualifying round they were drawn to face ÍF Fuglafjørður from the
Faroe Islands. They won the away leg 2--0 and then won the home leg 3--0
at Windsor Park to complete a comfortable 5--0 aggregate victory. In the
second qualifying round they were drawn to face Skoda Xanthi of Greece
and won the first leg 1--0 away from home despite being massive
underdogs for the tie. This made it three consecutive victories in
Europe for the club, without conceding a goal in the process -- another
first for a Northern Irish club. However, in the second leg at home they
went down 2--1 after extra time, which eliminated them on the away goals
rule.

\section{Statistics}\label{statistics}

\begin{itemize}
\item
  \emph{Biggest winning margin (aggregate): 5 goals, joint record:\\
  Linfield 9--4 Aris (1966--67 European Cup first round)\\
  Coleraine 7--2 Sant Julià (2002 UEFA Intertoto Cup first round)\\
  Linfield 5--0 ÍF Fuglafjørður (2013--14 UEFA Europa League first
  qualifying round)}
\item
  \emph{Heaviest defeat (aggregate): 13 goals, joint record:\\
  Ards 1--14 PSV Eindhoven (1974--75 European Cup Winners' Cup first
  round)\\
  Glentoran 1--14 Ajax (1975--76 UEFA Cup first round)}
\end{itemize}

Overall

First match played: AGF 0--0 Glenavon, 11 September 1957 (1957--58
European Cup first round, first leg)

Most competitions appeared in: 45, Linfield

Most games played: 107 -- Linfield

Most wins: 18 -- Linfield

Most draws: 31 -- Linfield

Most defeats: 63 -- Glentoran

Biggest winning margin (match): 5 goals, joint record:\\
Linfield 6--1 Aris (1966--67 European Cup first round)\\
Coleraine 5--0 Sant Julià (2002 UEFA Intertoto Cup first round)

Biggest winning margin (aggregate): 5 goals, joint record:\\
Linfield 9--4 Aris (1966--67 European Cup first round)\\
Coleraine 7--2 Sant Julià (2002 UEFA Intertoto Cup first round)\\
Linfield 5--0 ÍF Fuglafjørður (2013--14 UEFA Europa League first
qualifying round)

Heaviest defeat (match): 11 goals -- Dinamo Bucharest 11--0 Crusaders
(1973--74 European Cup first round)

Heaviest defeat (aggregate): 13 goals, joint record:\\
Ards 1--14 PSV Eindhoven (1974--75 European Cup Winners' Cup first
round)\\
Glentoran 1--14 Ajax (1975--76 UEFA Cup first round)

\section{UEFA coefficient and
ranking}\label{uefa-coefficient-and-ranking}

\begin{itemize}
\item
  \emph{For the 2015--16 UEFA competitions, the associations will be
  allocated places according to their 2014 UEFA country coefficients,
  which will take into account their performance in European
  competitions from 2009--10 to 2013--14.}
\end{itemize}

For the 2015--16 UEFA competitions, the associations will be allocated
places according to their 2014 UEFA country coefficients, which will
take into account their performance in European competitions from
2009--10 to 2013--14. In the 2014 rankings used to allocate berths for
the 2015--16 European competitions, Northern Ireland's coefficient
points total will be 3.625. After earning a score of 0.875 during the
2013--14 European campaign, Northern Ireland will remain as the 47th
best association in Europe out of 54 for the second consecutive season.

45 Malta 4.833

46 Liechtenstein 4.500

47 Northern Ireland 3.625

48 Wales 3.000

49 Armenia 2.875\\
Full list

\section{Appearances in UEFA
competitions}\label{appearances-in-uefa-competitions}

\section{Active competitions}\label{active-competitions}

\begin{itemize}
\item
  \emph{PR = Preliminary round; QR = Qualifying round; 1R/2R =
  First/Second round; 1QR/2QR/3QR = First/Second/Third qualifying round}
\end{itemize}

PR = Preliminary round; QR = Qualifying round; 1R/2R = First/Second
round; 1QR/2QR/3QR = First/Second/Third qualifying round

\textbar{}\}

\section{European Cup/Champions
League}\label{european-cupchampions-league}

\begin{itemize}
\item
  \emph{PR = Preliminary round; 1R/2R = First/Second round; 1QR/2QR =
  First/Second qualifying round; QF = Quarter-finals}
\end{itemize}

PR = Preliminary round; 1R/2R = First/Second round; 1QR/2QR =
First/Second qualifying round; QF = Quarter-finals

\section{Inter-Cities Fairs Cup/UEFA Cup/Europa
League}\label{inter-cities-fairs-cupuefa-cupeuropa-league}

\section{Defunct competitions}\label{defunct-competitions}

\section{Cup Winners' Cup}\label{cup-winners-cup}

\begin{itemize}
\item
  \emph{PR = Preliminary round; QR = Qualifying round; 1R/2R =
  First/Second round; QF = Quarter-finals}
\end{itemize}

PR = Preliminary round; QR = Qualifying round; 1R/2R = First/Second
round; QF = Quarter-finals

\section{UEFA Intertoto Cup}\label{uefa-intertoto-cup}

\begin{itemize}
\item
  \emph{GS = Group stage; 1R/2R = First/Second round}
\end{itemize}

GS = Group stage; 1R/2R = First/Second round

\section{Notes}\label{notes}

\section{See also}\label{see-also}

\begin{itemize}
\item
  \emph{List of football matches between British clubs in UEFA
  competitions}
\end{itemize}

List of football matches between British clubs in UEFA competitions

List of association football competitions

\section{References}\label{references}

\section{External links}\label{external-links}

\begin{itemize}
\item
  \emph{UEFA Website}
\end{itemize}

UEFA Website

Rec.Sport.Soccer Statistics Foundation
