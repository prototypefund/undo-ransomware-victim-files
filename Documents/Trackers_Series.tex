\textbf{From Wikipedia, the free encyclopedia}

https://en.wikipedia.org/wiki/Trackers\_Series\\
Licensed under CC BY-SA 3.0:\\
https://en.wikipedia.org/wiki/Wikipedia:Text\_of\_Creative\_Commons\_Attribution-ShareAlike\_3.0\_Unported\_License

\section{Trackers Series}\label{trackers-series}

\begin{itemize}
\item
  \emph{The books appear to be an interview with Detective Ganz.}
\item
  \emph{Trackers is a series of untitled books written by Patrick
  Carman.}
\end{itemize}

Trackers is a series of untitled books written by Patrick Carman. The
books appear to be an interview with Detective Ganz. and Adam sitting in
a room on Tuesday, August 12, which is most likely in 2009.

\section{Synopsis}\label{synopsis}

\begin{itemize}
\item
  \emph{Formed to take down Shantorian, a hacker once in their ranks,
  the ISDl literally begs the Trackers to join them.}
\item
  \emph{The second book, Shantorian, was released in 2011 and is
  possibly the final book in the series.}
\item
  \emph{The first book, Trackers is about four friends with an acute
  sense for technology who form a team called the Trackers.}
\item
  \emph{At the end of the book, the Trackers are confused as to whether
  to join the ISD.}
\end{itemize}

The first book, Trackers is about four friends with an acute sense for
technology who form a team called the Trackers. They usually do field
tests with their newly high-tech technology. At the same time, the team
is infiltrated and, in a rush to try and capture the perpetrator, find a
not well-known association; the ISD, the Internet Security Directive.
Formed to take down Shantorian, a hacker once in their ranks, the ISDl
literally begs the Trackers to join them. At the end of the book, the
Trackers are confused as to whether to join the ISD.{[}citation
needed{]}

The second book, Shantorian, was released in 2011 and is possibly the
final book in the series.{[}citation needed{]}

\section{Trackers}\label{trackers}

\begin{itemize}
\item
  \emph{The story begins in medias res, with Adam Henderson being
  interrogated by Inspector H. Ganz, a member of an unidentified
  government agency.}
\item
  \emph{After confirming with Ganz that the interrogation is being
  recorded, Adam decides to give them\\
  the "whole story," which with his "videographic" memory is very
  detailed.}
\end{itemize}

The story begins in medias res, with Adam Henderson being interrogated
by Inspector H. Ganz, a member of an unidentified government agency.
After confirming with Ganz that the interrogation is being recorded,
Adam decides to give them\\
the "whole story," which with his "videographic" memory is very
detailed. This causes the inspector to believe that he might be making
it up (which is untrue and rebutted after he shows video
evidence.).{[}citation needed{]}

\section{Missions}\label{missions}

\begin{itemize}
\item
  \emph{Code ADAM - with the game Chromatix}
\item
  \emph{Code LEWIS - with the game Memory Maze}
\item
  \emph{All Trackers Missions have been released.}
\item
  \emph{Code EMILY - with the game Glyph Warp}
\item
  \emph{Code FINN - with the game Glyph Smash}
\end{itemize}

Training Mission (Code TM101)

Code ADAM - with the game Chromatix

Code LEWIS - with the game Memory Maze

Code EMILY - with the game Glyph Warp

Code FINN - with the game Glyph Smash

All Trackers Missions have been released.

\section{See also}\label{see-also}

\section{References}\label{references}
