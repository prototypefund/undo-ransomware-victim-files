\textbf{From Wikipedia, the free encyclopedia}

https://en.wikipedia.org/wiki/Police\%20Athletic\%20League\\
Licensed under CC BY-SA 3.0:\\
https://en.wikipedia.org/wiki/Wikipedia:Text\_of\_Creative\_Commons\_Attribution-ShareAlike\_3.0\_Unported\_License

\section{Police Athletic League}\label{police-athletic-league}

\begin{itemize}
\item
  \emph{The Police Athletic League (PAL; Police Activities League) is an
  organization in many American police departments in which members of
  the police force coach young people, both boys and girls, in sports,
  and help with homework and other school-related activities.}
\item
  \emph{The Police Athletic League provides participation in many
  sports.}
\end{itemize}

The Police Athletic League (PAL; Police Activities League) is an
organization in many American police departments in which members of the
police force coach young people, both boys and girls, in sports, and
help with homework and other school-related activities. The purpose is
to build character, help strengthen police-community relations, and keep
children off illegal drugs.

Most PAL programs now call themselves "Police Activities Leagues"
because many of the programs are now focused on youth enrichment,
educational and youth leadership programs and not just sports. Some
organizations are also called Sheriff Activities Leagues "SAL" because
their sponsoring agency is a Sheriff's Department.

According to local Police Athletic Leagues, the program generally
solicits funds, equipment, and volunteer help from members of the
community, so that the cost to taxpayers is small while the returns are
great. Participants in the League's activities are supposedly much less
likely to engage in crime, far more likely to praise the character of
the police force, and discourage their friends from either committing
crimes or covering up criminal activity.

Police Athletic League programs usually have a competitive component.
Although the vast majority of the League's contests are with other
youths in the same city, there are regularly scheduled national contests
between teams in different parts of the country. For many of the young
people who participate, it is their first chance to travel to where the
contests are held.

The Police Athletic League provides participation in many sports. These
include Soccer, Basketball, Football, and many other sports throughout
the U.S.

The National Association of Police Athletics/Activities Leagues, Inc. is
based in Jupiter, Florida.

\section{See also}\label{see-also}

\begin{itemize}
\item
  \emph{Police Athletic League of New York City}
\end{itemize}

Drug Abuse Resistance Education

Gang Resistance Education and Training

Police Athletic League of New York City

\section{References}\label{references}

\section{External links}\label{external-links}

\begin{itemize}
\item
  \emph{"National Association of Police Athletics/Activities Leagues,
  Inc".}
\item
  \emph{January 19, 2010.}
\item
  \emph{Retrieved 24 January 2010.}
\item
  \emph{"Mayor Makes Plea for Financial Support Of the Police Athletic
  League Campaign".}
\item
  \emph{"Police Athletic League installs officers".}
\item
  \emph{"Support for Police Athletic League Urged By Valentine as Curb
  on Delinquency".}
\end{itemize}

"National Association of Police Athletics/Activities Leagues, Inc".
Retrieved 24 January 2010.

"Support for Police Athletic League Urged By Valentine as Curb on
Delinquency". The New York Times. November 30, 1944. p.~15. Retrieved 24
January 2010.

"Mayor Makes Plea for Financial Support Of the Police Athletic League
Campaign". The New York Times. April 16, 1946. p.~39. Retrieved 24
January 2010.

"Police Athletic League installs officers". St. Petersburg Times. St.
Petersburg, Florida: Google News Archive. January 25, 1988. Retrieved 24
January 2010.

"Fort Myers PAL wins award". Cape Coral Daily Breeze. Cape Coral,
Florida. January 19, 2010. Retrieved 24 January 2010.
