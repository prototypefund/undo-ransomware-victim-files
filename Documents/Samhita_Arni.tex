\textbf{From Wikipedia, the free encyclopedia}

https://en.wikipedia.org/wiki/Samhita\_Arni\\
Licensed under CC BY-SA 3.0:\\
https://en.wikipedia.org/wiki/Wikipedia:Text\_of\_Creative\_Commons\_Attribution-ShareAlike\_3.0\_Unported\_License

\section{Samhita Arni}\label{samhita-arni}

\begin{itemize}
\item
  \emph{Samhita Arni is an Indian writer who writes in English.}
\end{itemize}

Samhita Arni is an Indian writer who writes in English. She is best
known for her adaptations of Indian epic poetry. She started writing and
illustrating her first book The Mahabharatha - A Child's View at the age
of eight. The book then went on to be translated in seven languages and
sold 50,000 copies worldwide. Her second book -- Sita's Ramayana -- was
on the New York Times Bestseller list for Graphic Novels for two weeks
in 2011.

\section{Career}\label{career}

\begin{itemize}
\item
  \emph{The first, Sita's Ramayana, is a graphic novel developed in
  collaboration with Patua artist Moyna Chitrakar.}
\item
  \emph{Arni's first book, The Mahabharata: A Child's View, was
  published in 1996, when she was 11 years old.}
\item
  \emph{Arni watched this unfold, disturbed by the tragedy of what had
  taken place and staggered by the collective fury.}
\end{itemize}

Arni's first book, The Mahabharata: A Child's View, was published in
1996, when she was 11 years old. It was translated into several European
languages, and won the Elsa Morante Literary Award from the Department
of Culture of Campania, Italy, among other accoladates.

As an adult, she has written two books based on the Ramayana. The first,
Sita's Ramayana, is a graphic novel developed in collaboration with
Patua artist Moyna Chitrakar.

Her second Ramayana adaptation is The Missing Queen, a Speculative
fiction mythological thriller. It was published by Penguin/Zubaan in
2013.

The Prince, her latest book, is the culmination of a five-year journey
spent studying Ilango Adigal's Silappatikaram.It was the year 2014 and
the Nirbhaya case had taken centre-stage, unleashing a wave of wrath
from women across the country. Arni watched this unfold, disturbed by
the tragedy of what had taken place and staggered by the collective
fury. It led to what she calls an obsession with Silappatikaram, the
Sangam period epic, a story of love, betrayal, grief and above all,
wrath. Retellings, she feels, are important for a culture to evolve.
``Each generation is different, and for a myth, story, or an epic to
resonate with that generation, it must be told in a way that relates
with that generation's experience, in order to touch them, and for that
story to remain part of our cultural psyche.

She has been a regular columnist for The Bangalore Mirror and The Hindu.

\section{Education}\label{education}

\begin{itemize}
\item
  \emph{Arni is an alumnus of the United World Colleges in Italy.}
\end{itemize}

Arni is an alumnus of the United World Colleges in Italy. She has lived
in Indonesia, Pakistan, India, Thailand, Italy and the United States.

\section{References}\label{references}

\section{External links}\label{external-links}

\begin{itemize}
\item
  \emph{Interview~-- In Conversation with Samhita Arni at GALF 2016,
  2016-02-03 (video, 09 mins)}
\item
  \emph{Conversation~-- Finding a Voice: Samhita Arni \& Abeer Hoque in
  conversation with Supriya Nair, 2017-01-24 (video, 41 mins)}
\item
  \emph{The Prince - Book launch ~-- The author, Samhita Arni, in
  conversation with Arshia Sattar, 2019-04-05 (video, 63 mins)}
\end{itemize}

Personal website Samhita's website featuring some of her works

Interviews and talks

The Prince - Book launch ~-- The author, Samhita Arni, in conversation
with Arshia Sattar, 2019-04-05 (video, 63 mins)

Interview~-- In Conversation with Samhita Arni at GALF 2016, 2016-02-03
(video, 09 mins)

Conversation~-- Finding a Voice: Samhita Arni \& Abeer Hoque in
conversation with Supriya Nair, 2017-01-24 (video, 41 mins)

Columns and articles

Gender doesn't come in the way of Nirvana Column in the national daily
The Hindu

Sita's Freedom Struggle Column in the national daily The Indian Express

Temple Run: The Sacred Structures of the Chola Dynasty in Tamil Nadu
Article in National Geographic Traveler

Why do majority of Indian writers remain obsessed with myth? Article in
the national daily The Hindu

Breaking up with the Ramayana: Why I chose texts that promote empathy
and looking beyond one's identity Article in the news and media website
Firstpost
