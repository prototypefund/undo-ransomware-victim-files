\textbf{From Wikipedia, the free encyclopedia}

https://en.wikipedia.org/wiki/Albert\%20Turner\%20\%28civil\%20rights\%20activist\%29\\
Licensed under CC BY-SA 3.0:\\
https://en.wikipedia.org/wiki/Wikipedia:Text\_of\_Creative\_Commons\_Attribution-ShareAlike\_3.0\_Unported\_License

\section{Albert Turner (activist)}\label{albert-turner-activist}

\begin{itemize}
\item
  \emph{Albert Turner (February 29, 1936 -- April 13, 2000) was an
  American civil rights activist and an advisor to Dr. Martin Luther
  King, Jr.}
\end{itemize}

Albert Turner (February 29, 1936 -- April 13, 2000) was an American
civil rights activist and an advisor to Dr. Martin Luther King, Jr. He
was Alabama field secretary for the Southern Christian Leadership
Conference and helped lead the voting rights march from Selma to
Montgomery; he was beaten on Bloody Sunday.

\section{Early life}\label{early-life}

\begin{itemize}
\item
  \emph{He graduated from Alabama A\&M.}
\item
  \emph{Turner was born outside of Marion, Alabama, the son of a Perry
  County farmer; he was the fourth of 12 children.}
\end{itemize}

Turner was born outside of Marion, Alabama, the son of a Perry County
farmer; he was the fourth of 12 children. He graduated from Alabama
A\&M.

\section{Career}\label{career}

\begin{itemize}
\item
  \emph{That incident galvanized Turner to organize local voter
  registration efforts and educate prospective voters about the voter
  registration tests.}
\item
  \emph{Albert Turner attempted to register to vote in 1962, but despite
  his college education, could not pass the literacy test given.}
\item
  \emph{After the assassination of Dr. King, Turner led the mule train
  which bore his body to its final resting place.}
\end{itemize}

Albert Turner attempted to register to vote in 1962, but despite his
college education, could not pass the literacy test given. That incident
galvanized Turner to organize local voter registration efforts and
educate prospective voters about the voter registration tests. Turner
was one of the marchers credited with leading the Selma to Montgomery
procession in March 1965 while Dr. King attended a ceremony honoring his
work in Cleveland.

He served as State Director of the SCLC from 1965 to 1972. After the
assassination of Dr. King, Turner led the mule train which bore his body
to its final resting place. Turner attracted national attention in 1978
as the manager of the Southwest Alabama Farmers Cooperative Association,
when he and an ex-moonshine distiller teamed up to cheaply produce
ethanol for gasohol from corn mash.

\section{Voting rights activism in
Alabama}\label{voting-rights-activism-in-alabama}

\begin{itemize}
\item
  \emph{Perry County and the other counties forming the Black Belt,
  initially named for its soil, are some Alabama's poorest counties, and
  residents of the Black Belt typically travel out of their home
  counties for work, making it difficult to vote at local polls, which
  were only open in Perry County for four hours in the afternoon.}
\item
  \emph{The initial conviction, though, depressed black voter
  participation in Pickens County.}
\item
  \emph{As a result, local civil rights organizations, including the
  Perry County Civic League (PCCL), began to register black voters for
  and assist them with absentee ballots.}
\end{itemize}

Traditionally, white voters in the Black Belt region of Alabama had
taken advantage of absentee ballots to retain control of elected
offices; even with the increased registration of black voters, white
absentee nonresident voters would vote in local elections at the behest
of resident relatives and friends. Wilcox County Commissioner Bobby Joe
Johnson joked, "Do you know why the roads to white folks' cemeteries are
paved in the Black Belt? It's so people won't get their feet wet if it
rains on election day." Despite complaints of fraud, election officials
failed to obtain any indictments for absentee voter fraud. When Turner
went to Washington D.C. to complain about fraud to lawyers at the
Justice Department, he was told the government "{[}couldn't{]} do
anything ... Y'all need to learn to use the absentee-ballot process
yourselves." Perry County and the other counties forming the Black Belt,
initially named for its soil, are some Alabama's poorest counties, and
residents of the Black Belt typically travel out of their home counties
for work, making it difficult to vote at local polls, which were only
open in Perry County for four hours in the afternoon. In addition, many
potential black voters in the Black Belt are elderly and/or illiterate
due to inferior education, again limiting poll access. As a result,
local civil rights organizations, including the Perry County Civic
League (PCCL), began to register black voters for and assist them with
absentee ballots.

In 1979, Maggie Bozeman and Julia Wilder were convicted of absentee
voter fraud in a 1978 election held in Pickens County. The verdict was
reached by an all-white jury, and the pair were sentenced to four
(Bozeman) and five (Wilder) years imprisonment, believed to be the
strongest sentences handed out to-date for voting fraud in Alabama. The
conviction was later upheld by the Alabama Supreme Court before being
overturned in federal court, but the pair, known as the Carrollton Two,
started to serve their sentences in January 1982 and were freed in
November 1982 after spending 11 days in prison, with the balance of time
served on work release in Tuskegee. The initial conviction, though,
depressed black voter participation in Pickens County.

\section{The Marion Three}\label{the-marion-three}

\begin{itemize}
\item
  \emph{Turner renewed his contention that the trial was instigated by
  freshman Senator Jeremiah Denton (R-AL), who, Turner said, had the
  most to gain by intimidating black voters.}
\item
  \emph{At the time, Turner led the PCCL, and a rival group complained
  to the local district attorney that Turner and other PCCL officials
  were altering votes on absentee ballots.}
\end{itemize}

State district attorney Roy Johnson convened a grand jury during the
fall and winter of 1982--83 to investigate PCCL-led voter assistance
programs and absentee ballots, but failed to obtain an indictment.
Johnson took his suspicions to Assistant Attorney General William
Bradford Reynolds, head of the Civil Rights Division of the Department
of Justice in October 1982, asking Reynolds to deploy federal marshals
to supervise elections, and Reynolds demurred, saying it was not
possible under the Voting Rights Act. In the summer of 1984, US Justice
Department officials promulgated a new policy to "investigate political
participants who seek out the elderly, socially disadvantaged, or the
illiterate, for the purpose of subjugating their electoral will,"
applying it in a selective manner against civil rights activists seeking
to register black voters in the Black Belt.

At the time, Turner led the PCCL, and a rival group complained to the
local district attorney that Turner and other PCCL officials were
altering votes on absentee ballots. In the 1984 election, according to
Congressional testimony in 1986 from Jeff Sessions at his confirmation
hearing, out of 4,000 ballots total in Perry County, 729 were absentee
ballots, which seemed like a disproportionately high number since more
absentee ballots were cast in Perry than in Jefferson County, which had
more than 45 times the number of residents of Perry County. At
Sessions's suggestion, the FBI watched the Marion post office during the
September 1984 primary election, where they saw the Turner, his wife
Evelyn and Spencer Hogue Jr. drop off absentee ballots; election
officials, under a court order, marked those envelopes and identified 75
altered ballots in the group of ballots dropped of by PCCL officials.

FBI agents took the marked ballots and confronted many of the voters
with allegations of "ballot tampering." Of the eighteen counties counted
as part of the Black Belt, the investigation concentrated on the five
counties which had posted the largest gains in black elected officials.
In a later interview, Evelyn Turner explained the alterations were
performed at the request of the voters, assistance permitted by state
law. Nevertheless the prosecution moved forward and Turner, his wife
Evelyn, and Hogue, who became known as the Marion Three, were indicted
on twenty-nine counts by a Mobile-based federal grand jury on January
25, 1985 based on allegations of ballot tampering in the September 4,
1984 primary election. The maximum punishment was 115 years in prison
and \$40,000 in fines.

Turner's brother Robert, a lawyer who was part of the defense team for
the Marion Three, alleged in 2016 that Sessions had jailed the Marion
Three to send a message: "You got a person like Albert Turner that's out
front, spearheading all these voter registration drives, why not put him
in jail and see how many people get scared?"

The Marion Three were defended by a team of lawyers experienced in civil
rights litigation, including Turner's brother Robert, J. L. Chestnut,
Lani Guinier, James Liebman, Deval Patrick, and Hank Sanders. During the
grand jury investigation in October 1984, more than two dozen subpoenaed
witnesses, many of them black senior citizens familiar with
segregation-era intimidation techniques, were bused more than 160 miles
(260~km) to Mobile to be photographed, fingerprinted, provide
handwriting samples, and testify before the federal grand jury. Grand
jury records indicate that several of the questioned witnesses were so
intimidated they would no longer vote in the future.

The case moved to trial on June 19, 1985. The federal jury deliberated
for three hours before handing down "not guilty" verdicts for all three
defendants on all charges on July 5, 1985. Turner renewed his contention
that the trial was instigated by freshman Senator Jeremiah Denton
(R-AL), who, Turner said, had the most to gain by intimidating black
voters. Sessions later claimed that his team, which consisted of two
lawyers, was understaffed and unprepared to tackle a "vigorous defense."

\section{Later life and legacy}\label{later-life-and-legacy}

\begin{itemize}
\item
  \emph{A stretch of Perry County Road 45 south from Marion is named the
  Albert Turner Sr Memorial Highway.}
\item
  \emph{Perry County has also named an elementary school for Turner.}
\item
  \emph{His youngest son, Albert Turner Jr., was appointed to fill
  Turner Sr.'s seat on the Perry County Commission after Turner Sr.'s
  death.}
\end{itemize}

Turner died on April 13, 2000, while being prepared for an operation to
stop abdominal bleeding. His youngest son, Albert Turner Jr., was
appointed to fill Turner Sr.'s seat on the Perry County Commission after
Turner Sr.'s death.

A stretch of Perry County Road 45 south from Marion is named the Albert
Turner Sr Memorial Highway. Perry County has also named an elementary
school for Turner.

\section{References}\label{references}

\section{External links}\label{external-links}

\begin{itemize}
\item
  \emph{"Albert Turner 1936-2000".}
\item
  \emph{Civil Rights Movement Veterans.}
\item
  \emph{Retrieved 16 February 2017.}
\end{itemize}

Hartford, Bruce. "Albert Turner 1936-2000". Civil Rights Movement
Veterans. Retrieved 16 February 2017. ~{[}...{]}~We had not been there
{[}in Brantley, Alabama{]} more than 20 minutes when a car skidded up to
the cabin in a cloud of red dust. A Black woman jumped out, she was the
Mayor's maid she told us. He had sent her to warn us out of town. He was
calling out the mob, and we'd better stay out if we knew what was good
for us.~′You tell the Mayor that if he wants us, he can find us here,′
Al told her softly. He went out on the porch and sat down in an old
rocking chair facing the street. Slowly he rocked back and forth, back
and forth. Cars and trucks filled with hard-eyed white men, the same
ones that had chased us the day before, drove at low speed past the
house. They stared hard at Al and the rest of us while he rocked back
and forth, back and forth. They drove by again, and again, and again,
and Al just slowly rocked back and forth in that old rocking chair.

A Message from Albert Turner, the "Root Doctor" (excerpt from Street
Heat documentary, 1989) on Vimeo

United States of America v. Albert Turner, et al., 812 F.2d 1552 (11th
Cir. 30 March 1987).
