\textbf{From Wikipedia, the free encyclopedia}

https://en.wikipedia.org/wiki/C-SPAN\\
Licensed under CC BY-SA 3.0:\\
https://en.wikipedia.org/wiki/Wikipedia:Text\_of\_Creative\_Commons\_Attribution-ShareAlike\_3.0\_Unported\_License

\section{C-SPAN}\label{c-span}

\begin{itemize}
\item
  \emph{Cable-Satellite Public Affairs Network (C-SPAN; /ˈsiːˌspæn/) is
  an American cable and satellite television network that was created in
  1979 by the cable television industry as a nonprofit public service.}
\item
  \emph{The C-SPAN network includes the television channels C-SPAN,
  C-SPAN2, and C-SPAN3, the radio station WCSP-FM, and a group of
  websites which provide streaming media and archives of C-SPAN
  programs.}
\end{itemize}

Cable-Satellite Public Affairs Network (C-SPAN; /ˈsiːˌspæn/) is an
American cable and satellite television network that was created in 1979
by the cable television industry as a nonprofit public service. It
televises many proceedings of the United States federal government, as
well as other public affairs programming. The C-SPAN network includes
the television channels C-SPAN, C-SPAN2, and C-SPAN3, the radio station
WCSP-FM, and a group of websites which provide streaming media and
archives of C-SPAN programs. C-SPAN's television channels are available
to approximately 100 million cable and satellite households within the
United States, while WCSP-FM is broadcast on FM radio in Washington,
D.C. and is available throughout the U.S. on SiriusXM via Internet
streaming, and globally through apps for iOS, BlackBerry, and Android
devices.

The network televises U.S. political events, particularly live and
"gavel-to-gavel" coverage of the U.S. Congress, as well as occasional
proceedings of the Canadian, Australian and British Parliaments
(including the weekly Prime Minister's Questions) and other major events
worldwide. Its coverage of political and policy events is unmoderated,
providing the audience with unfiltered information about politics and
government. Non-political coverage includes historical programming,
programs dedicated to non-fiction books, and interview programs with
noteworthy individuals associated with public policy. C-SPAN is a
private, non-profit organization funded by its cable and satellite
affiliates, and it does not have advertisements on any of its networks,
radio stations, or websites, nor does it solicit donations or pledges.
The network operates independently, and neither the cable industry nor
Congress has control of its programming content.

\section{History}\label{history}

\section{Development}\label{development}

\begin{itemize}
\item
  \emph{Upon its debut, only 3.5~million homes were wired for C-SPAN,
  and the network had just three employees.}
\item
  \emph{Brian Lamb, C-SPAN's chairman and former chief executive
  officer, conceived C-SPAN in 1975 while working as the Washington,
  D.C. bureau chief of the cable industry trade magazine Cablevision.}
\item
  \emph{C-SPAN began full-time operations in September 14, 1982.}
\end{itemize}

Brian Lamb, C-SPAN's chairman and former chief executive officer,
conceived C-SPAN in 1975 while working as the Washington, D.C. bureau
chief of the cable industry trade magazine Cablevision. It was a time of
rapid growth in the number of cable television channels available in the
United States, and Lamb envisioned a cable-industry financed nonprofit
network for televising sessions of the U.S. Congress and other public
affairs event and policy discussions. Lamb shared his idea with several
cable executives, who helped him launch the network. Among them were Bob
Rosencrans, who provided \$25,000 of initial funding in 1979, and John
D. Evans, who provided the wiring and access to the headend needed for
the distribution of the C-SPAN signal. According to a report from
commentator Jeff Greenfield on Nightline in 1980, C-SPAN was launched to
provide televised coverage of U.S. political events in its entirety,
thus helping viewers maintaining a thorough view of politics and
especially presidential campaigns, unlike television newscasts which
"does not really inform us about what the candidates mean to do with the
power they ask of us."

C-SPAN was launched on March 19, 1979, in time for the first televised
session made available by the House of Representatives, beginning with a
speech by then-Tennessee representative Al Gore. Upon its debut, only
3.5~million homes were wired for C-SPAN, and the network had just three
employees. C-SPAN began full-time operations in September 14, 1982. The
second C-SPAN channel, C-SPAN2, followed on June 2, 1986 when the U.S.
Senate permitted itself to be televised. It began full-time operations
in January 5, 1987. C-SPAN3, the most recent expansion channel, began
full-time operations on January 22, 2001, and shows live/taped public
policy and government-related events on weekdays, with historical
programming being shown on weeknights and weekends. It has also
sometimes served as an overflow channel for live programming conflicts
on C-SPAN and C-SPAN2. C-SPAN3 is the successor of a digital channel
called C-SPAN Extra, which was launched in the Washington D.C. area in
1997, and televised live and recorded political events from 9:00~a.m. to
6:00~p.m. Eastern Time Monday through Friday.

C-SPAN Radio began operations on October 9, 1997, covering similar
events as the television networks and often simulcasting their
programming. The station broadcasts on WCSP (90.1 FM) in Washington,
D.C., is also available on XM Satellite Radio channel 120 and is
streamed live at c-span.org. It was formerly available on Sirius
Satellite Radio from 2002 to 2006.

Lamb semi-retired in March 2012, coinciding with the channel's 33rd
anniversary, and gave executive control of the network to his two
lieutenants, Rob Kennedy and Susan Swain.

On January 12, 2017, the online feed for C-SPAN1 was interrupted and
replaced by a feed from the Russian television network RT America for
approximately 10 minutes. C-SPAN announced that they were
troubleshooting the incident and were "operating under the assumption
that it was an internal routing issue."

\section{Anniversaries}\label{anniversaries}

\begin{itemize}
\item
  \emph{On the anniversary date, C-SPAN repeated the first televised
  hour of floor debate in the House of Representatives from 1979 and,
  throughout the month, 25th anniversary features included "then and
  now" segments with journalists who had appeared on C-SPAN during its
  early years.}
\item
  \emph{C-SPAN celebrated its 10th anniversary in 1989 with a three-hour
  retrospective, featuring Lamb recalling the development of the
  network.}
\end{itemize}

C-SPAN celebrated its 10th anniversary in 1989 with a three-hour
retrospective, featuring Lamb recalling the development of the network.
The 15th anniversary was commemorated in an unconventional manner as the
network facilitated a series of re-enactments of the seven historic
Lincoln-Douglas debates of 1858, which were televised from August to
October 1994, and have been rebroadcast from time to time ever since.
Five years later, the series American Presidents: Life Portraits, which
won a Peabody Award, served as a year-long observation of C-SPAN's 20th
anniversary.

In 2004, C-SPAN celebrated its 25th anniversary, by which time the
flagship network was viewed in 86 million homes, C-SPAN2 was in
70~million homes and C-SPAN3 was in eight million homes. On the
anniversary date, C-SPAN repeated the first televised hour of floor
debate in the House of Representatives from 1979 and, throughout the
month, 25th anniversary features included "then and now" segments with
journalists who had appeared on C-SPAN during its early years. Also
included in the 25th anniversary was an essay contest for viewers to
write in about how C-SPAN has influenced their life regarding community
service. For example, one essay contest winner wrote about how C-SPAN's
non-fiction book programming serves as a resource in his charitable
mission to record non-fiction audio books for people who are blind.

To commemorate 25 years of taking viewer telephone calls, in 2005,
C-SPAN had a 25-hour "call-in marathon", from 8:00~pm. Eastern Time on
Friday, October 7, concluding at 9:00~pm. Eastern Time on Saturday,
October 8. The network also had a viewer essay contest, the winner of
which was invited to co-host an hour of the broadcast from C-SPAN's
Capitol Hill studios.

\section{Scope and limitations of
coverage}\label{scope-and-limitations-of-coverage}

\begin{itemize}
\item
  \emph{Committee meetings on health care were broadcast subsequently by
  C-SPAN and may be viewed on the C-SPAN website.}
\item
  \emph{In particular, C-SPAN asked to add some of its own robotically
  operated cameras to the existing government-controlled cameras in the
  House chamber.}
\item
  \emph{In December 2009, Lamb wrote to leaders in the House and Senate,
  requesting that negotiations for health care reform be televised by
  C-SPAN.}
\end{itemize}

C-SPAN continues to expand its coverage of government proceedings, with
a history of requests to government officials for greater access,
especially to the U.S. Supreme Court. In December 2009, Lamb wrote to
leaders in the House and Senate, requesting that negotiations for health
care reform be televised by C-SPAN. Committee meetings on health care
were broadcast subsequently by C-SPAN and may be viewed on the C-SPAN
website. In November 2010, Lamb wrote to incoming House Speaker John
Boehner requesting changes to restrictions on cameras in the House. In
particular, C-SPAN asked to add some of its own robotically operated
cameras to the existing government-controlled cameras in the House
chamber. In February 2011, Boehner denied the request. A previous
request to Speaker Designate Nancy Pelosi in 2006, to add C-SPAN's
cameras in the House chamber to record floor proceedings, was also
denied. Although C-SPAN uses the congressional chamber feed cables, the
cameras are owned and controlled by each respective body of Congress.
Requests by C-SPAN for camera access to non-government events such as
the annual dinner by the Gridiron Club have also been denied.

On June 22 and into June 23, 2016, C-SPAN took video footage of the
House floor from individual House representatives via streaming services
Periscope and Facebook Live during a sit-in by House Democrats asking
for a vote on gun control measures after the 2016 Orlando nightclub
shooting. This needed to be done because---as the sit-in was done out of
formal session and while the House was in official recess---the existing
House cameras could not be utilized for coverage of the event by rule.\\
Although the use of electronic devices to create the Periscope feeds by
House Democrats violated House rules that prohibit their use on the
floor, C-SPAN did not state why it chose to broadcast those feeds. The
network ran disclaimers on-air and on their official social media feeds
noting the restrictions.

\section{Expansion and technology}\label{expansion-and-technology}

\begin{itemize}
\item
  \emph{In January 1997, C-SPAN began real-time streaming of C-SPAN and
  C-SPAN2 on its website, the first time that Congress had been live
  streamed online.}
\item
  \emph{The network provided C-SPAN and C-SPAN2 in high definition on
  June 1, 2010, and C-SPAN3 in July 2010.}
\item
  \emph{In June 2010, C-SPAN joined with the website Foursquare to
  provide users of the application with access to geotagged C-SPAN
  content at various locations in Washington, D.C.}
\end{itemize}

Since the late 1990s, C-SPAN has significantly expanded its online
presence. In January 1997, C-SPAN began real-time streaming of C-SPAN
and C-SPAN2 on its website, the first time that Congress had been live
streamed online. To cover the Democratic and Republican conventions and
the presidential debates of 2008, C-SPAN created two standalone
websites: the Convention Hub and the Debate Hub. In addition to
real-time streams of C-SPAN's television networks online, c-span.org
features further live programming such as committee hearings and
speeches that are broadcast later in the day, after the House and Senate
have left.

C-SPAN began promoting audience interaction early in its history, by the
regular incorporation of viewer telephone calls in its programming. It
has since expanded into social media. In March 2009, viewers began
submitting questions live via Twitter to guests on C-SPAN's morning
call-in show Washington Journal. The network also has a Facebook page to
which it added occasional live streaming in January 2011. The live
stream is intended to show selected well-publicized events of Congress.
In June 2010, C-SPAN joined with the website Foursquare to provide users
of the application with access to geotagged C-SPAN content at various
locations in Washington, D.C.

In 2010, C-SPAN began a transition to high definition telecasts, planned
to take place over an 18-month period. The network provided C-SPAN and
C-SPAN2 in high definition on June 1, 2010, and C-SPAN3 in July 2010.

As part of the network's 40th anniversary, C-SPAN instituted the first
logo change in the network's history on March 18, 2019.

\section{Programming}\label{programming}

\section{Senate and House of
Representatives}\label{senate-and-house-of-representatives}

\begin{itemize}
\item
  \emph{When the House or Senate are not in session, C-SPAN channels
  broadcast other public affairs programming and recordings of previous
  events.}
\item
  \emph{The C-SPAN network's core programming is live coverage of the
  U.S. House and Senate, with the C-SPAN channel emphasizing the United
  States House of Representatives.}
\end{itemize}

The C-SPAN network's core programming is live coverage of the U.S. House
and Senate, with the C-SPAN channel emphasizing the United States House
of Representatives. Between 1979 and May 2011, the network televised
more than 24,246 hours of floor action. C-SPAN2, the first of the C-SPAN
spin-off networks, provides uninterrupted live coverage of the United
States Senate. With coverage of the House and Senate, viewers can track
legislation as it moves through both bodies of Congress. Important
debates in Congress that C-SPAN has covered live include the Persian
Gulf conflict during 1991, and the House impeachment vote and Senate
trial of President Bill Clinton in 1998 and 1999. When the House or
Senate are not in session, C-SPAN channels broadcast other public
affairs programming and recordings of previous events.

\section{Public affairs}\label{public-affairs}

\begin{itemize}
\item
  \emph{The public affairs coverage on the C-SPAN networks other than
  the House and Senate floor debates is wide-ranging.}
\item
  \emph{Due to this policy, C-SPAN hosts do not state their names on
  television.}
\item
  \emph{C-SPAN also covers midterm elections.}
\item
  \emph{When Lipstadt ended media access to her speech, C-SPAN canceled
  coverage of both.}
\end{itemize}

The public affairs coverage on the C-SPAN networks other than the House
and Senate floor debates is wide-ranging. C-SPAN is considered a useful
source of information for journalists, lobbyists, educators and
government officials as well as casual viewers interested in politics,
due to its unedited coverage of political events. C-SPAN has been
described by media observers as a "window into the world of Washington
politics" and it characterizes its own mission as being "to provide
public access to the political process". The networks cover U.S.
political campaigns, including the Republican, Democratic, and
Libertarian presidential nominating conventions in their entirety.
Coverage of presidential campaign events are provided during the
duration of the campaign, both by a weekly television program, Road to
the White House, and at its dedicated politics website. C-SPAN also
covers midterm elections.

All three channels televise events such as congressional hearings, White
House press briefings and presidential speeches, as well as other
government meetings including Federal Communications Commission hearings
and Pentagon press conferences. Other U.S. political coverage includes
State of the Union speeches, and presidential press conferences.
According to the results of a survey after the 1992 presidential
election, 85\% of C-SPAN viewers voted in that election. The results of
a similar survey in 2013 found that 89\% of C-SPAN viewers voted in the
2012 presidential election. In addition to this political coverage, the
network broadcasts press conferences and meetings of various news media
and nonprofit organizations, including those at the National Press Club,
public policy seminars and the White House Correspondents' Dinner. While
C-SPAN does not have video access to the Supreme Court, the network has
used the Court's audio recordings accompanied by still photographs of
the justices and lawyers to cover the Court in session on significant
cases, and has covered individual Supreme Court justices' speaking
engagements.

Occasionally, proceedings of the Parliament of Australia, Parliament of
Canada, Parliament of the United Kingdom (usually Prime Minister's
Questions and the State Opening of Parliament) and other governments are
shown on C-SPAN when they discuss matters of importance to viewers in
the U.S. Similarly, the networks will sometimes broadcast news reports
from around the world when major events occur -- for instance, C-SPAN
broadcast CBC Television coverage of the September 11 attacks. C-SPAN
also covers lying in state in the Capitol Rotunda and funerals of former
presidents and other notable individuals. In 2005, C-SPAN covered
Hurricane Katrina through NBC affiliate WDSU in New Orleans, as well as
coverage of Hurricane Ike via CBS affiliate KHOU in Houston. C-SPAN also
carries CBC coverage during events that affect Canadians, such as the
Canadian federal elections, the death and state funeral of Pierre
Trudeau, and the 2003 North America blackout. During early 2011, C-SPAN
carried broadcasts by Al Jazeera to cover the events in Egypt, Tunisia,
and other Arab nations. Additionally, C-SPAN simulcasts NASA Space
Shuttle mission launches and landings live, using video footage and
audio sourced from NASA TV.

With its public affairs programming, C-SPAN intends to offer different
viewpoints by allowing time for multiple opinions to be discussed on a
given topic. For example, in 2004 C-SPAN intended to televise a speech
by Holocaust historian Deborah Lipstadt adjacent to a speech by
Holocaust denier David Irving, who had unsuccessfully sued Lipstadt for
libel in the United Kingdom four years earlier; C-SPAN was criticized
for its use of the word "balance" to describe the plan to cover both
Lipstadt and Irving. When Lipstadt ended media access to her speech,
C-SPAN canceled coverage of both.

The network strives for neutrality and a lack of bias; in all
programming when on-camera hosts are present their role is simply to
facilitate and explain proceedings to the viewer. Due to this policy,
C-SPAN hosts do not state their names on television.

\section{C-SPAN and C-SPAN2 flagship
programs}\label{c-span-and-c-span2-flagship-programs}

\begin{itemize}
\item
  \emph{While many hours of programming on C-SPAN are dedicated to
  coverage of the House, the network's daily programming begins with the
  political phone-in and interview program Washington Journal from 7:00
  to 10:00~a.m. Eastern Time.}
\end{itemize}

While many hours of programming on C-SPAN are dedicated to coverage of
the House, the network's daily programming begins with the political
phone-in and interview program Washington Journal from 7:00 to
10:00~a.m. Eastern Time. Washington Journal premiered on January 4, 1995
and has been broadcast every morning since then, with guests including
elected officials, government administrators, and journalists. The
program covers current events, with guests answering questions on topics
presented by the hosts, as well as questions from members of the general
public. On weeknights C-SPAN2 dedicates its schedule to Politics and
Public Policy Today (9:00 p.m. -- midnight for the East Coast primetime,
replayed immediately for the West Coast primetime), which is a block of
recordings of the day's noteworthy events in rapid succession. On the
weekend schedule, C-SPAN's main programs are: America and the Courts,
which is shown each Saturday at 7:00~p.m. Eastern Time, Newsmakers, a
Sunday morning interview program with newsworthy guests; Q\&A, a Sunday
evening interview program hosted by Brian Lamb, with guests including
journalists, politicians, authors, and other public figures; and The
Communicators, which features interviews with journalists, government
officials, and businesspeople involved with the communications industry
and related legislation.

On weekends C-SPAN2 dedicates its schedule to Book TV, which is 48 hours
of programming about non-fiction books, book events, and authors. Book
TV was launched in September 1998. Booknotes was originally broadcast
from 1989 to 2004, as a one-hour one-on-one interview of a non-fiction
author. Repeats of the interviews remain a regular part of the Book TV
schedule with the title Encore Booknotes. Other Book TV programs feature
political and historical books and biographies of public figures. These
include In Depth, a live, monthly, three-hour interview with a single
author, and After Words, an author interview program featuring guest
hosts interviewing authors on topics with which both are familiar. After
Words was developed as a new type of author interview program after the
end of production of Booknotes. Weekend programming on Book TV also
includes coverage of book events such as panel discussions, book fairs,
book signings, readings by authors and tours of bookstores around the
U.S.

\section{C-SPAN3}\label{c-span3}

\begin{itemize}
\item
  \emph{The weekday programming on C-SPAN3 (from the morning (anywhere
  from 6:00-8:30 a.m.) to 8:00~p.m. Eastern Time) features uninterrupted
  live public affairs events, in particular political events from
  Washington, D.C. Each weekend since January 8, 2011, the network has
  broadcast 48 hours of programming dedicated to the history of the
  United States, under the umbrella title American History TV.}
\end{itemize}

The weekday programming on C-SPAN3 (from the morning (anywhere from
6:00-8:30 a.m.) to 8:00~p.m. Eastern Time) features uninterrupted live
public affairs events, in particular political events from Washington,
D.C. Each weekend since January 8, 2011, the network has broadcast 48
hours of programming dedicated to the history of the United States,
under the umbrella title American History TV. The programming covers the
history of the U.S. from the founding of the nation through the late
20th century. Programs include American Artifacts, which is dedicated to
exploring museums, archives and historical sites, and Lectures in
History, featuring major university history professors giving lectures
on U.S. history. In 2009, C-SPAN3 aired an eight-installment series of
interviews from the Robert J. Dole Institute of Politics at the
University of Kansas, which featured historian Richard Norton Smith and
Vice President Walter Mondale, among other interviewees.

\section{Special programming}\label{special-programming}

\begin{itemize}
\item
  \emph{C-SPAN has occasionally produced spinoff programs from Booknotes
  focusing on specific topics.}
\item
  \emph{In 2013, C-SPAN introduced a new program, First Ladies:
  Influence \& Image.}
\end{itemize}

C-SPAN has occasionally produced spinoff programs from Booknotes
focusing on specific topics. In 1994, Booknotes collaborated with
Lincoln scholar Harold Holzer to produce re-creations of the seven
Lincoln--Douglas debates. Several years later, a similar series retraced
the journey of Alexis de Tocqueville described in Democracy in America.
Another special series was American Writers, a 38-week tour of the U.S.
based on the works of 40 famous American writers.

During 2008 and 2009, as part of programming specially commissioned for
the 200th anniversary of the birth of Abraham Lincoln, C-SPAN produced a
series titled Lincoln 200 Years, which featured episodes on a variety of
topics relating to the life of Lincoln including his career, his homes
and his opinions of slavery.

The network has also produced special feature documentaries of American
institutions and historical landmarks, exploring their history to the
present day. These programs include: The Capitol emphasizing the
history, art, and architecture of the U.S. Capitol Building; The White
House, featuring film footage inside the White House and exploring the
history of the building and its occupants; The Supreme Court, focusing
on the history and personalities of the court; and Inside Blair House,
an examination of the president's guest house.

In 2013, C-SPAN introduced a new program, First Ladies: Influence \&
Image. 35 episodes profiling the First Ladies are planned for the
series, which was created with support from the White House Historical
Association.

\section{Radio broadcasts}\label{radio-broadcasts}

\begin{itemize}
\item
  \emph{C-SPAN Radio has a selective policy regarding its broadcast
  content, rather than duplicating the television network programming,
  although it does offer some audio simulcasts of programs such as
  Washington Journal.}
\item
  \emph{In addition to the three television networks, C-SPAN also
  broadcasts via C-SPAN Radio, which is carried on their
  owned-and-operated station WCSP (90.1 FM) in the Washington, D.C. area
  with all three cable network feeds airing via HD Radio subchannels,
  and nationwide on XM Satellite Radio.}
\end{itemize}

In addition to the three television networks, C-SPAN also broadcasts via
C-SPAN Radio, which is carried on their owned-and-operated station WCSP
(90.1 FM) in the Washington, D.C. area with all three cable network
feeds airing via HD Radio subchannels, and nationwide on XM Satellite
Radio. Its programming is also livestreamed at c-span.org and is
available via apps for iPhone, BlackBerry and Android devices. C-SPAN
Radio has a selective policy regarding its broadcast content, rather
than duplicating the television network programming, although it does
offer some audio simulcasts of programs such as Washington Journal.
Unique programming on the radio station includes oral histories, and
some committee meetings and press conferences not shown on television
due to programming commitments. The station also compiles the Sunday
morning talk shows for a same-day rebroadcast without commercials, in
rapid succession.

\section{Online availability}\label{online-availability}

\begin{itemize}
\item
  \emph{In addition to the programming available in the C-SPAN Video
  Library, all C-SPAN programming is available as a live feed streamed
  on its website in Flash Video format.}
\item
  \emph{Prior to the initiation of the C-SPAN Video Library, websites
  such as Metavid and voterwatch.org hosted House and Senate video
  records, however C-SPAN contested Metavid's usage of C-SPAN
  copyrighted footage.}
\end{itemize}

C-SPAN archival video is available through the C-SPAN Video Library,
maintained at the Purdue Research Park in West Lafayette, Indiana.
Unveiled in August 2007, the C-SPAN Video Library contains all of the
network's programming since 1987, totaling more than 160,000 hours at
its completion of digitization and public debut in March 2010. Older
C-SPAN programming continues to be added to the library, dating back to
the beginning of the network in 1979, and some limited earlier footage
from the National Archives, such as film clips of Richard Nixon's 1972
trip to China, is available as well. Most of the recordings before 1987
(when the C-SPAN Archive was established) were not saved, except for
approximately 10,000 hours of video which are slated to be made
available online. As of April~2019{[}update{]}, the C-SPAN Video Library
held over 249,000 hours of programming, and they have been viewed over
246,6 million times. Described by media commentators as a major
educational service and a valuable resource for researchers of politics
and history, the C-SPAN Video Library has also had a major role in media
and opposition research in several U.S. political campaigns. It won a
Peabody Award in 2010 "for creating an enduring archive of the history
of American policymaking, and for providing it as a free, user-friendly
public service."

Prior to the initiation of the C-SPAN Video Library, websites such as
Metavid and voterwatch.org hosted House and Senate video records,
however C-SPAN contested Metavid's usage of C-SPAN copyrighted footage.
The result was Metavid's removal of portions of the archive produced
with C-SPAN's cameras, while preserving its archive of
government-produced content. C-SPAN also engaged in actions to stop
parties from making unauthorized uses of its content online, including
its video of House and Senate proceedings. Most notably, in May 2006,
C-SPAN requested the removal of Stephen Colbert's performance at the
White House Correspondents' Association Dinner from YouTube. After
concerns by some webloggers, C-SPAN gave permission for Google Video to
host the full event. On March 7, 2007 C-SPAN liberalized its copyright
policy for current, future, and past coverage of any official events
sponsored by Congress and any federal agency and now allows for
attributed non-commercial copying, sharing, and posting of C-SPAN video
on the Internet, excluding re-syndication of live video streams. The new
policy did not affect the public's right to use the public domain video
coverage of the floor proceedings of the U.S. House and Senate.

In 2008, C-SPAN's online political coverage was expanded just prior to
the elections, with the introduction of three special pages on the
C-SPAN website: the C-SPAN Convention Hubs and C-SPAN Debate Hub, which
offered video of major events as well as discussion from weblogs and
social media about the major party conventions and candidate debates.
C-SPAN brought back the Convention Hub for the 2012 presidential
election.

In addition to the programming available in the C-SPAN Video Library,
all C-SPAN programming is available as a live feed streamed on its
website in Flash Video format.

On July 29, 2014, C-SPAN announced that it would begin restricting
access to the live feeds of the main channel, C-SPAN2 and C-SPAN3 to
subscribers of cable or satellite providers later that summer, citing
concerns with the slow shift in viewing habits from cable television to
the internet due to its reliance on carriage fees from cable and
satellite providers. However, it will continue to allow all government
meetings, hearings and conferences to be streamed live online and via
archived on the C-SPAN Video Library without requiring an authenticated
login by a provider; live audio feeds of all three channels are also
available for free through the network's mobile app. The decision drew
some criticism from public interest and government transparency
advocates, citing the fact that C-SPAN was designed as a public service.

\section{Organization and operations}\label{organization-and-operations}

\begin{itemize}
\item
  \emph{As of 2012{[}update{]}, C-SPAN received 6¢ of each subscriber's
  cable bill for an annual budget of \$60~million.}
\item
  \emph{C-SPAN is led by co-CEOs Rob Kennedy and Susan Swain.}
\item
  \emph{The majority of C-SPAN's employees are based at C-SPAN's
  headquarters located on Capitol Hill in Washington, D.C., however in
  2003 television studios were opened in New York City and Denver,
  Colorado.}
\end{itemize}

C-SPAN is operated by the National Cable Satellite Corporation, a
nonprofit organization, the board of directors of which consists
primarily of representatives of the largest cable companies. Early
chairmen of C-SPAN include Bob Rosencrans, John Saeman, Ed Allen and
Gene Schneider. C-SPAN does not sell commercials or solicit donations on
air; instead, it receives nearly all of its funding from subscriber fees
charged to cable and direct-broadcast satellite (DBS) operators. As of
2012{[}update{]}, C-SPAN received 6¢ of each subscriber's cable bill for
an annual budget of \$60~million. As the network is an independent
entity, neither the cable industry nor Congress controls the content of
its programming.

As of January~2013{[}update{]}, the network has 282 employees. C-SPAN is
led by co-CEOs Rob Kennedy and Susan Swain. Founder and former CEO Brian
Lamb serves as the executive chairman of the board of directors. The
majority of C-SPAN's employees are based at C-SPAN's headquarters
located on Capitol Hill in Washington, D.C., however in 2003 television
studios were opened in New York City and Denver, Colorado. These studios
use digital equipment that can be controlled from Washington.

C-SPAN also maintains archives in West Lafayette, Indiana at the Purdue
Research Park under the direction of Dr. Robert X. Browning.

\section{Availability}\label{availability}

\begin{itemize}
\item
  \emph{More than 28 million people said they watched C-SPAN programming
  each week.}
\item
  \emph{A 2010 poll conducted by C-SPAN and Penn Schoen Berland
  estimates that 79 million adults in the U.S. watched C-SPAN at some
  time from 2009 to 2010.}
\item
  \emph{The C-SPAN networks are available in more than 100 million
  households as of 2010{[}update{]}, not including access to the C-SPAN
  websites.}
\end{itemize}

The C-SPAN networks are available in more than 100 million households as
of 2010{[}update{]}, not including access to the C-SPAN websites. More
than 7,000 telephone callers have participated with discussion on
Washington Journal as of March~18, 2009{[}update{]}. There are no
official viewing statistics for C-SPAN because the network, which has no
commercials or underwriting advertisements, does not use the Nielsen
ratings. However, there have been a number of surveys providing
estimates:

A 1994 survey found that 8.6\% of the U.S. population regularly watched
C-SPAN.

In 2004 this figure increased to 12\% of the U.S. population, according
to a Pew Research Center survey, while 31\% of the population was
categorized as occasional viewers. More than 28 million people said they
watched C-SPAN programming each week.

A March 2009 Hart Research survey found that 20\% of homes with cable
television watch C-SPAN at least once a week, for an estimated 39
million Americans.

A 2010 poll conducted by C-SPAN and Penn Schoen Berland estimates that
79 million adults in the U.S. watched C-SPAN at some time from 2009 to
2010.

In January 2013, Hart Research conducted another survey which showed
that 47 million adults, or 24\% of adults with access to cable
television, watch C-SPAN weekly. Of the 47 million regular C-SPAN
viewers, 51\% are male and 49\% female; 26\% are liberal, 31\%
conservative, and 39\% moderate. About half are college graduates. 28\%
of 18-to-49-year-olds report watching at least once a week, as do 19\%
of 50- to 64-year-olds, and 22\% of those over age 65.

In February 2017, Ipsos Audience conducted another survey which showed
that 70 million adults, or 36\% of adults with access to cable
television, watch C-SPAN on a given six-month period. Of the 70 million
regular C-SPAN viewers, 52\% are male and 48\% female; 25\% are West
viewers, 22\% Midwest, 20\% Northeast and 33\% South. 28\% identified
themselves as liberal, 27\% conservative, and 36\% moderate. 51\% of all
viewers are 18-44 years old.

\section{Public and media opinion}\label{public-and-media-opinion}

\begin{itemize}
\item
  \emph{C-SPAN's public service nature has been praised as an enduring
  contribution to national knowledge.}
\item
  \emph{According to The New York Times, C-SPAN's mission to record
  official events in Washington, D.C. makes it "one of a kind",
  particularly in the creation of the C-SPAN Video Library, which
  received significant press coverage.}
\item
  \emph{A 2009 C-SPAN survey of viewers found that the network's
  most-valued attribute was its balanced programming.}
\end{itemize}

A 2009 C-SPAN survey of viewers found that the network's most-valued
attribute was its balanced programming. The survey's respondents were a
mixed group, with 31\% describing themselves as "liberal," while 28\%
described themselves as "conservative", and the survey found that C-SPAN
viewers are an equal mixture of men and women of all age groups.

C-SPAN's public service nature has been praised as an enduring
contribution to national knowledge. In 1987, Andrew Rosenthal wrote for
The New York Times about C-SPAN's influence in political elections,
arguing that C-SPAN's "blanket coverage" had expanded television
journalism "into areas once shielded from general view". The network has
received positive media coverage for providing public access to
proceedings such as the Goldman Sachs Senate hearings, and the U.S. 2010
Healthcare Summit, while its everyday programming has been credited with
providing the media and the general public with an intimate knowledge of
U.S. political proceedings and people. The ability of C-SPAN to provide
this service without federal funding, advertising or soliciting viewer
contributions has been remarked by local newspapers and online news
services, with the Daily Beast terming C-SPAN's \$55~million annual
budget (in 2009), "an astounding bargain." In an article on the 25th
anniversary of the network, The Washington Post noted that C-SPAN's
programming has been copied by television networks worldwide and credits
the network with providing information about foreign politics to
American viewers. According to The New York Times, C-SPAN's mission to
record official events in Washington, D.C. makes it "one of a kind",
particularly in the creation of the C-SPAN Video Library, which received
significant press coverage.

Despite its stated commitment to providing politically balanced
programming, C-SPAN and its shows such as Washington Journal, Booknotes,
Q \& A, and After Words have been accused by some liberal organizations
of having a conservative bias. In 2005, the media criticism organization
Fairness and Accuracy in Reporting (FAIR) released a study of C-SPAN's
morning telephone call-in show Washington Journal, showing that
Republicans were favored as guests over Democrats by a two-to-one margin
during a six-month period that year, and that people of color are
underrepresented. A 2007 survey released by the think tank Center for
Economic and Policy Research reported that C-SPAN covered conservative
think tanks more than left-of-center think tanks.

\section{Must-carry}\label{must-carry}

\begin{itemize}
\item
  \emph{Some communities, such as Eugene, Oregon and Alexandria,
  Virginia, were successful in restoring C-SPAN availability.}
\item
  \emph{C-SPAN availability was later restored as technological
  developments that resulted in the expansion of channel capacity on
  cable providers allowed for mandatory stations and the C-SPAN networks
  both to be broadcast.}
\end{itemize}

In 1992, Congress passed must-carry regulations, which required cable
carriers to allocate spectrum to local broadcasters. This affected the
availability of C-SPAN in some areas, in particular C-SPAN2, as some
providers chose to discontinue carriage of the channel altogether.
Between 1993 and 1994, cable systems in 95 U.S. cities dropped or
reduced broadcasts of C-SPAN and C-SPAN2, following the implementation
of the must-carry regulations. Viewers protested these decisions,
especially when the changes coincided with matters of local interest
occurring in the House or Senate. Some communities, such as Eugene,
Oregon and Alexandria, Virginia, were successful in restoring C-SPAN
availability. C-SPAN availability was later restored as technological
developments that resulted in the expansion of channel capacity on cable
providers allowed for mandatory stations and the C-SPAN networks both to
be broadcast.

\section{Other services}\label{other-services}

\begin{itemize}
\item
  \emph{The two original buses were retired in 2010, and the C-SPAN
  Digital Bus was inaugurated, introducing the public to C-SPAN's
  enhanced digital products.}
\item
  \emph{C-SPAN offers a number of public services related to the
  network's public affairs programming.}
\item
  \emph{The first C-SPAN book, C-SPAN: America's Town Hall, was
  published in 1988.}
\end{itemize}

C-SPAN offers a number of public services related to the network's
public affairs programming. C-SPAN Classroom, a free membership service
for teachers, began in July 1987 and offers help using C-SPAN resources
for classes or research. The C-SPAN School Bus, introduced in November
1993, traveled around the U.S. educating the public about government and
politics using C-SPAN resources, and served as a mobile television
studio. The bus also recorded video footage of the places that it
visited. A second bus was introduced in 1996. The two original buses
were retired in 2010, and the C-SPAN Digital Bus was inaugurated,
introducing the public to C-SPAN's enhanced digital products. C-SPAN has
also equipped six Local Content Vehicles (LCVs) to travel the country
and record unique political and historical stories, with each vehicle
containing production and web-based technologies to produce on-the-spot
content.

C-SPAN has published ten books based on its programming; these contain
original material and text taken from interview transcripts. The first
C-SPAN book, C-SPAN: America's Town Hall, was published in 1988. Other
C-SPAN books include: Gavel to Gavel: A C-SPAN Guide to Congress; Who's
Buried in Grant's Tomb?, a guide to the grave sites of U.S. presidents;
Abraham Lincoln - Great American Historians On Our Sixteenth President,
a collection of essays based on C-SPAN interviews with American
historians; and The Supreme Court, which features biographies and
interviews with past Supreme Court judges together with commentary from
legal experts. Five books have been drawn from the former Booknotes
program: Booknotes: Life Stories; Booknotes: On American Character;
Booknotes: Stories from American History; Booknotes: America's Finest
Authors on Reading, Writing and the Power of Ideas, the latter a
compilation of short monologues taken from the transcripts of Lamb's
interviews; and a companion book to the series on Tocqueville, Traveling
Tocqueville's America: A Tour Book.

\section{See also}\label{see-also}

\begin{itemize}
\item
  \emph{Public, educational, and government access}
\end{itemize}

Public, educational, and government access

Legislature broadcaster

\section{References}\label{references}

\section{External links}\label{external-links}

\begin{itemize}
\item
  \emph{Official website}
\end{itemize}

Official website
