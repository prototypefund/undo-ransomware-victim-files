\textbf{From Wikipedia, the free encyclopedia}

https://en.wikipedia.org/wiki/Tom\%20Kahn\\
Licensed under CC BY-SA 3.0:\\
https://en.wikipedia.org/wiki/Wikipedia:Text\_of\_Creative\_Commons\_Attribution-ShareAlike\_3.0\_Unported\_License

\section{Tom Kahn}\label{tom-kahn}

\begin{itemize}
\item
  \emph{Kahn died in~1992, at the age of~53.}
\item
  \emph{(This article, originally a 1964 pamphlet from the League for
  Industrial Democracy, was written by Kahn, according to Horowitz
  (2007, pp.}
\item
  \emph{Kahn's analysis of the civil rights movement influenced Bayard
  Rustin (who was the nominal author of Kahn's "From Protest to
  Politics").}
\item
  \emph{Kahn was raised in New York City.}
\end{itemize}

Tom David Kahn (September 15, 1938~-- March 27, 1992) was an American
social~democrat known for his leadership in several organizations. He
was an activist and influential strategist in the Civil Rights Movement.
He was a senior adviser and leader in the U.S. labor movement.

Kahn was raised in New York City. At Brooklyn College, he joined the
U.S. socialist movement, where he was influenced by Max Shachtman and
Michael Harrington. As an assistant to civil rights leader Bayard
Rustin, Kahn helped to organize the 1963 March on Washington, during
which Martin~Luther King delivered his "I Have a Dream" speech. Kahn's
analysis of the civil rights movement influenced Bayard Rustin (who was
the nominal author of Kahn's "From Protest to Politics"). (This article,
originally a 1964 pamphlet from the League for Industrial Democracy, was
written by Kahn, according to Horowitz (2007, pp.~223--224). It remains
widely reprinted, for example in Rustin's Down the Line of 1971 and Time
on two crosses of 2003.)

A leader in the Socialist Party of America, Kahn supported its 1972 name
change to Social Democrats,~USA (SDUSA). Like other leaders of SDUSA,
Kahn worked to support free labor-unions and democracy and to oppose
Soviet communism; he also worked to strengthen U.S. labor unions. Kahn
worked as a senior assistant to and speechwriter for Democratic Senator
Henry~"Scoop" Jackson, AFL--CIO Presidents George~Meany and
Lane~Kirkland, and other leaders of the Democratic Party, labor unions,
and civil-rights organizations.

In 1980 Lane~Kirkland appointed Kahn to organize the AFL--CIO's support
for the Polish labor-union Solidarity; this support was made despite
protests by the USSR and the Carter administration. He acted as the
Director of the AFL--CIO's Department of International Affairs in~1986
and was officially named Director in~1989. Kahn died in~1992, at the age
of~53.

\section{Biography}\label{biography}

\section{Early life}\label{early-life}

\begin{itemize}
\item
  \emph{He was adopted by Adele and David Kahn, and renamed Thomas David
  Kahn.}
\end{itemize}

Kahn was born Thomas John Marcel on September 15, 1938, and was
immediately placed for adoption at the New York Foundling Hospital. He
was adopted by Adele and David Kahn, and renamed Thomas David Kahn. His
father, a member of the Communist Party USA, became President of the
Transport Workers Local~101 of the Brooklyn Union Gas Company.

Tom Kahn was a civil libertarian who "ran for president of the Student
Organization of Erasmus Hall High School in 1955 on a platform calling
for the destruction of the student assembly, because it had no power",
an election he lost. In high school, he met Rachelle Horowitz, who would
become his lifelong friend and political ally.

\section{Democratic socialism}\label{democratic-socialism}

\begin{itemize}
\item
  \emph{As young socialists, Kahn's and Horowitz's talents were
  recognized by Michael Harrington.}
\item
  \emph{At Brooklyn College (CUNY), the undergraduate students Kahn and
  Horowitz joined the U.S. movement for democratic socialism after
  hearing Max~Shachtman denounce the 1956 Soviet invasion of Hungary:
  Shachtman described}
\item
  \emph{Kahn idolized Harrington, particularly for his erudition and
  rhetoric, both in writing and in debate.}
\end{itemize}

At Brooklyn College (CUNY), the undergraduate students Kahn and Horowitz
joined the U.S. movement for democratic socialism after hearing
Max~Shachtman denounce the 1956 Soviet invasion of Hungary: Shachtman
described

As young socialists, Kahn's and Horowitz's talents were recognized by
Michael Harrington. Harrington had joined Shachtman after working with
Dorothy Day's Catholic Worker's house of hospitality in the Bowery of
Lower Manhattan. Harrington was about to become famous in the United
States for his book on poverty in the United States, The Other America.
Kahn idolized Harrington, particularly for his erudition and rhetoric,
both in writing and in debate.

\includegraphics[width=4.67133in,height=5.50000in]{media/image1.jpg}\\
\emph{Bayard Rustin, whom Tom Kahn assisted with organizing the 1963
March on Washington}

\section{Civil rights}\label{civil-rights}

\begin{itemize}
\item
  \emph{As a leader of the American socialist movement, Michael
  Harrington sent Tom Kahn and Rachelle Horowitz to help Bayard Rustin,
  one of the leaders of the Civil Rights Movement, who became a mentor
  to Kahn.}
\item
  \emph{Kahn and Horowitz were affectionately called the "Bayard~Rustin
  Marching and~Chowder Society" by Harrington.}
\end{itemize}

As a leader of the American socialist movement, Michael Harrington sent
Tom Kahn and Rachelle Horowitz to help Bayard Rustin, one of the leaders
of the Civil Rights Movement, who became a mentor to Kahn. Kahn and
Horowitz were affectionately called the "Bayard~Rustin Marching
and~Chowder Society" by Harrington.\\
Kahn helped Rustin organize the 1957 Prayer Pilgrimage to Washington and
the 1958 and 1959 Youth March for Integrated Schools.

\section{Homosexuality and Bayard
Rustin}\label{homosexuality-and-bayard-rustin}

\begin{itemize}
\item
  \emph{As a young man, Tom Kahn "was gay but wanted to be straight~...}
\item
  \emph{Kahn accepted his homosexuality in~1956, the year that Kahn and
  Horowitz volunteered to help Bayard~Rustin with his work in the
  civil-rights movement.}
\item
  \emph{"Once he met Bayard~{[}Rustin{]}, then Kahn knew that he was gay
  and had this long-term relationship with Bayard, which went through
  many stages", according to Horowitz, who quoted Kahn's remembrance of
  Rustin:}
\end{itemize}

As a young man, Tom Kahn "was gay but wanted to be straight~... It was a
different world then", according to Rachelle~Horowitz. He had a short
relationship with a member of the Young People's Socialist League
(YPSL):

Tom Kahn was "very good~looking, a very attractive guy" according to
longtime socialist David~McReynolds, who is also an openly~gay
New~Yorker. Kahn accepted his homosexuality in~1956, the year that Kahn
and Horowitz volunteered to help Bayard~Rustin with his work in the
civil-rights movement. "Once he met Bayard~{[}Rustin{]}, then Kahn knew
that he was gay and had this long-term relationship with Bayard, which
went through many stages", according to Horowitz, who quoted Kahn's
remembrance of Rustin:

However, cohabiting in Rustin's apartment proved unsuccessful, and their
romantic relationship ended when Kahn enrolled in the historically~black
Howard~University. Kahn and Rustin remained lifelong friends and
political comrades.

\section{Howard University}\label{howard-university}

\begin{itemize}
\item
  \emph{Kahn and Carmichael worked with Howard University's chapter of
  Student Nonviolent Coordinating Committee (SNCC).}
\item
  \emph{Kahn graduated from Howard in 1961.}
\item
  \emph{Kahn introduced Carmichael and his fellow SNCC activists to
  Bayard Rustin, who became an influential adviser to SNCC.}
\item
  \emph{Kahn and Rustin's emphasis on economic inequality influenced
  Carmichael.}
\end{itemize}

Kahn, a white student, enrolled for his junior and senior years at
Howard~University, where he became a leader in student politics. Kahn
worked closely with Stokely Carmichael, who later became a national
leader of young civil-rights activists and then one of the leaders of
the Black Power movement. Kahn and Carmichael helped to fund a five-day
run of Three Penny Opera, by the Marxist playwright Berthold Brecht and
the socialist composer Kurt Weill: "Tom Kahn---very shrewdly---had
captured the position of Treasurer of the Liberal Arts Student Council
and the infinitely charismatic and popular Carmichael as floor whip was
good at lining up the votes. Before they knew what hit them the Student
Council had become a patron of the arts, having voted to buy out the
remaining performances." Kahn and Carmichael worked with Howard
University's chapter of Student Nonviolent Coordinating Committee
(SNCC). Kahn introduced Carmichael and his fellow SNCC activists to
Bayard Rustin, who became an influential adviser to SNCC. Kahn and
Rustin's emphasis on economic inequality influenced Carmichael. Kahn
graduated from Howard in 1961.

\section{Leadership}\label{leadership}

\begin{itemize}
\item
  \emph{For this march, Kahn also ghost wrote the speech of A.~Philip
  Randolph, the senior leader of the civil-rights movement and the
  African-American labor movement.}
\item
  \emph{Kahn's analysis of the civil-rights movement influenced Bayard
  Rustin (who was the nominal author of Kahn's~1964--1965 essay
  "From~protest to~politics"), Stokely Carmichael, and William~Julius
  Wilson.}
\end{itemize}

Kahn (along with Horowitz and Norman Hill) helped Rustin and A.~Philip
Randolph to plan the 1963 March on Washington, at which Martin Luther
King Jr. delivered his "I have a dream" speech. For this march, Kahn
also ghost wrote the speech of A.~Philip Randolph, the senior leader of
the civil-rights movement and the African-American labor movement.
Kahn's analysis of the civil-rights movement influenced Bayard Rustin
(who was the nominal author of Kahn's~1964--1965 essay "From~protest
to~politics"), Stokely Carmichael, and William~Julius Wilson.

\section{League for Industrial
Democracy}\label{league-for-industrial-democracy}

\begin{itemize}
\item
  \emph{Kahn's The Economics of Equality LID pamphlet gave an "incisive
  radical analysis of what it would take to end racial oppression".}
\item
  \emph{Kahn was Director of the League for Industrial Democracy after
  1964.}
\end{itemize}

Kahn was Director of the League for Industrial Democracy after 1964.
Beginning in 1960, he wrote several LID pamphlets, many of which were
published in political journals like Dissent and Commentary, and some of
which appeared in anthologies. Kahn's The Economics of Equality LID
pamphlet gave an "incisive radical analysis of what it would take to end
racial oppression".

\section{Student League for Industrial Democracy: Students for a
Democratic Society
(SDS)}\label{student-league-for-industrial-democracy-students-for-a-democratic-society-sds}

\begin{itemize}
\item
  \emph{Kahn expressed his admiration for Shachtman's intellectual
  toughness in his 1973 memorial:}
\item
  \emph{Kahn's determined style of debate emerged from the socialist
  movement led by Max Shachtman.}
\item
  \emph{In 1966, Kahn attended the Illinois Convention of SDS, where his
  forceful arguments and delivery overwhelmed and were resented by the
  other activists; Kahn was then 28~years old.}
\end{itemize}

Before Kahn became LID director in 1964, he was involved with the
Student League for Industrial Democracy, which became Students for a
Democratic Society (SDS). Along with other LID members Rachelle
Horowitz, Michael Harrington, and Don Slaiman, Kahn attended the
LID-sponsored meeting that discussed the Port Huron Statement. Kahn was
listed as a student representative from Howard University and was
elected to the National Executive Committee. The LID representatives
criticized the Port Huron Statement for promoting students as leaders of
social change, for criticizing the U.S. labor movement and its unions,
and for its criticisms of liberal and socialist opposition to Soviet
communism ("anti-communism"). Kahn believed that the SDS students were
"elitist", being overly critical of labor unions and liberals, and
attributed upper-class origins and Ivy-league schooling to them,
according to Port-Huron activist Todd Gitlin, who observes that Kahn was
the son of a "manual laborer".

LID and SDS split in 1965, when SDS voted to remove from its
constitution the "exclusion clause" that prohibited membership by
communists, against Kahn's arguments. The SDS exclusion clause had
barred "advocates of or apologists for" "totalitarianism". The clause's
removal effectively invited "disciplined cadre" to attempt to "take over
or paralyze" SDS, as had occurred to mass organizations in the thirties.
Afterward, Marxism Leninism, particularly the Progressive Labor Party,
helped to write "the death sentence" for SDS. Nonetheless Kahn continued
to argue with SDS leaders about the need for accountable leadership,
about tactics, and about strategy. In 1966, Kahn attended the Illinois
Convention of SDS, where his forceful arguments and delivery overwhelmed
and were resented by the other activists; Kahn was then 28~years old.

Kahn's determined style of debate emerged from the socialist movement
led by Max Shachtman. Kahn expressed his admiration for Shachtman's
intellectual toughness in his 1973 memorial:

"His answers, of course, could not always be correct. But they were on
target and always fundamental."

\section{Social Democrats, USA}\label{social-democrats-usa}

\begin{itemize}
\item
  \emph{Kahn worked as a senior assistant and speechwriter for Senator
  Henry~"Scoop" Jackson, AFL--CIO Presidents George Meany and Lane
  Kirkland, and other leaders of the Democratic Party, labor unions, and
  civil rights organizations.}
\item
  \emph{Kahn and Horowitz were leaders in the Socialist Party USA, and
  supported its change of name to Social Democrats,~USA (SDUSA), despite
  Harrington's opposition.}
\end{itemize}

Kahn and Horowitz were leaders in the Socialist Party USA, and supported
its change of name to Social Democrats,~USA (SDUSA), despite
Harrington's opposition.\\
Ben Wattenberg commented that SDUSA members seemed to be

Kahn worked as a senior assistant and speechwriter for Senator
Henry~"Scoop" Jackson, AFL--CIO Presidents George Meany and Lane
Kirkland, and other leaders of the Democratic Party, labor unions, and
civil rights organizations. He was an effective speechwriter because he
was able to express ideas to an American audience, according to
Wattenberg.

\section{Estrangement with
Harrington}\label{estrangement-with-harrington}

\begin{itemize}
\item
  \emph{Maurice Isserman's biography of Harrington also described this
  speech as Kahn's self~hatred, as "Kahn's resort to gay~bashing".}
\item
  \emph{This gay-baiting taunt was attributed to Kahn by Harrington, and
  repeated by Newfield in his autobiography.}
\item
  \emph{Horowitz stated, "It is in fact inconceivable that Kahn wrote
  those words."}
\end{itemize}

Another protégé of Shachtman's, Michael Harrington, called for an
immediate withdrawal of U.S. forces from Vietnam in 1972. His proposal
was rejected by the majority, who criticized the war's conduct and
called for a negotiated peace treaty, the position associated with
Shachtman and Kahn. Harrington resigned his honorary chairmanship of the
Socialist Party and organized a caucus for like-minded socialists.\\
The conflict between Kahn and Harrington became "pretty bad", according
to Irving Howe.

Harrington handed former SDS activist and New York City journalist Jack
Newfield a speech by AFL--CIO President George Meany. Addressing the
September 1972 Convention of the United Steelworkers of America, Meany
ridiculed the Democratic Party Convention, which had been held in Miami:

This gay-baiting taunt was attributed to Kahn by Harrington, and
repeated by Newfield in his autobiography. Maurice Isserman's biography
of Harrington also described this speech as Kahn's self~hatred, as
"Kahn's resort to gay~bashing".

The blaming of Kahn for Meany's speech and Isserman's scholarship have
been criticized by Rachelle Horowitz, Kahn's friend, and by Joshua
Muravchik, then an officer of the Young People's Socialist League
(1907). According to Horowitz, Meany had many speechwriters---two
specialists besides Kahn and even more writers from the AFL--CIO's
Committee on Political Education (COPE) Department. Horowitz stated, "It
is in fact inconceivable that Kahn wrote those words." She quoted a
concurring assessment from Arch Puddington: {[}Isserman{]} "assumes that
because Kahn was not publicly gay he had to be a gay basher. He never
was." According to Muravchik, "there is no reason to believe that Kahn
wrote those lines, and Isserman presents none."

Harrington failed to support an anti-discrimination (gay rights) plank
in the 1978 platform of the Democratic Party Convention, but noted his
personal support after being criticized in The~Nation. Along with others
in the AFL--CIO and SDUSA, Kahn was accused of criticizing Harrington's
application for his Democratic Socialist Organizing Committee to join
the Socialist International and to organize a 1983 conference on
European socialism; Harrington complained for six pages in his
autobiography The Long Distance Runner, and "brooded" about Kahn's
opposition, exaggerating the importance of the Socialist International
to America, according to Isserman's biography. In 1991, even after
Harrington's 1989 death, Howe warned Harrington's biographer, Maurice
Isserman, that Kahn's description of Harrington "may well be a little
nasty" and "hard line".

\section{AFL--CIO support for free
trade-unions}\label{aflcio-support-for-free-trade-unions}

\begin{itemize}
\item
  \emph{After becoming an assistant to the President of the AFL--CIO in
  1972, a position he held until 1986, Kahn developed an expertise in
  international affairs.}
\item
  \emph{In 1980 AFL--CIO officer Lane Kirkland appointed Kahn to
  organize the AFL--CIO's support for the Polish labor-union Solidarity,
  which was maintained and indeed increased even after protests by the
  USSR and Carter administration.}
\end{itemize}

After becoming an assistant to the President of the AFL--CIO in 1972, a
position he held until 1986, Kahn developed an expertise in
international affairs. In 1980 AFL--CIO officer Lane Kirkland appointed
Kahn to organize the AFL--CIO's support for the Polish labor-union
Solidarity, which was maintained and indeed increased even after
protests by the USSR and Carter administration.

\section{Support of Solidarity, the Polish
union}\label{support-of-solidarity-the-polish-union}

\begin{itemize}
\item
  \emph{The AFL--CIO's autonomous support of Solidarity was so
  successful that by 1984 both Democrats and Republicans agreed that it
  deserved public support.}
\item
  \emph{Kahn was heavily involved in supporting the Polish
  labor-movement.}
\item
  \emph{In 1980 AFL--CIO President Lane Kirkland appointed Kahn to
  organize the AFL--CIO's support of Solidarity.}
\end{itemize}

Kahn was heavily involved in supporting the Polish labor-movement. The
trade union Solidarity (Solidarność) began in 1980. The Soviet-backed
Communist regime headed by General Wojciech Jaruzelski declared martial
law in December~1981.

In 1980 AFL--CIO President Lane Kirkland appointed Kahn to organize the
AFL--CIO's support of Solidarity. The AFL--CIO sought approval in
advance from Solidarity's leadership, to avoid jeopardizing their
position with unwanted or surprising American help. Politically, the
AFL--CIO supported the twenty-one demands of the Gdansk workers, by
lobbying to stop further U.S. loans to Poland unless those demands were
met. Materially, the AFL--CIO established the Polish Workers Aid Fund.
By 1981 it had raised almost \$300,000, which was used to purchase
printing presses and office supplies. The AFL--CIO donated typewriters,
duplicating machines, a minibus, an offset press, and other supplies
requested by Solidarity.

In testimony to the Joint Congressional Commission on Security and
Cooperation in Europe, Kahn suggested policies to support the Polish
people, in particular by supporting Solidarity's demand that the
Communist regime finally establish legality, by respecting the
twenty-one rights guaranteed by the Polish constitution.

The AFL--CIO's support enraged the Communist regimes of Eastern Europe
and the Soviet Union, and worried the Carter Administration, whose
Secretary of State Edmund Muskie told Kirkland that the AFL--CIO's
continued support of Solidarity could trigger a Soviet invasion of
Poland. After Kirkland refused to withdraw support to Solidarity, Muskie
met with the USSR's Ambassador, Anatoly Dobyrnin, to clarify that the
AFL--CIO's aid did not have the support of the U.S. government. Aid to
Solidarity was also initially opposed by neo-conservatives Norman
Podhoretz and Jeane Kirkpatrick, who before 1982 argued that communism
could not be overthrown and that Solidarity was doomed.

The AFL--CIO's autonomous support of Solidarity was so successful that
by 1984 both Democrats and Republicans agreed that it deserved public
support. The AFL--CIO's example of open support was deemed to be
appropriate for a democracy, and much more suitable than the clandestine
funding through the CIA that had occurred before 1970. Both parties and
President Ronald Reagan supported a non-governmental organization,
National Endowment for Democracy (NED), through which Congress would
openly fund Solidarity through an allocation in the State Department's
budget, beginning in 1984. The NED was designed with four core
institutions, associated with the two major parties and with the AFL-CIO
and the U.S. Chamber of Commerce (representing business). The NED's
first president was Carl Gershman, a former Director of Social
Democrats, USA and former U.S. Representative to the United Nations
committee on human rights. From 1984 until 1990, the NED and the
AFL--CIO channeled equipment and support worth \$4~million to
Solidarity.

\section{Director of the AFL--CIO's Department of International
Affairs}\label{director-of-the-aflcios-department-of-international-affairs}

\begin{itemize}
\item
  \emph{In 1986 Kahn became the Director of the AFL--CIO Department of
  International Affairs, where he implemented Kirkland's program of
  having a consensus foreign policy.}
\item
  \emph{Kahn acted as Director of the AFL--CIO's Department of
  International Affairs in 1986, after Irving Brown suffered a stroke
  and resigned that same year; after Brown's death in~1989, Kahn was
  officially named the Director.}
\end{itemize}

In 1986 Kahn became the Director of the AFL--CIO Department of
International Affairs, where he implemented Kirkland's program of having
a consensus foreign policy. Working with leaders from member unions,
Kahn helped to draft resolutions that represented consensus decisions
for nearly all issues.

Kahn acted as Director of the AFL--CIO's Department of International
Affairs in 1986, after Irving Brown suffered a stroke and resigned that
same year; after Brown's death in~1989, Kahn was officially named the
Director.

\section{Living with AIDS}\label{living-with-aids}

\begin{itemize}
\item
  \emph{An upgrade of the International Department's computer systems
  was to have allowed Kahn to work from home.}
\item
  \emph{Kahn planned most of his own memorial service, which was held in
  the AFL--CIO headquarters.}
\item
  \emph{Kahn longed to spend his remaining years with his "new and most
  beloved partner", who was "the love of his life".}
\end{itemize}

Earlier in 1986, Kahn had learned that he was infected with human
immunodeficiency virus (HIV), "which was then a death sentence". Kahn
longed to spend his remaining years with his "new and most beloved
partner", who was "the love of his life". However, he accepted the
office of Director out of a feeling of duty, knowing that he was taking
"a job that would most surely work him to death". He warned his
co-workers that his terminal condition would bring intellectual
degeneration, and asked that they monitor him for signs of debilitation.
An upgrade of the International Department's computer systems was to
have allowed Kahn to work from home.

Kahn died from acquired immunodeficiency syndrome (AIDS) in Silver
Spring, Maryland on March~27,~1992, at the age of~53, after having been
cared for by his partner and supported by his friends and colleagues. He
was survived by his partner and also his sister and his niece. Kahn
planned most of his own memorial service, which was held in the AFL--CIO
headquarters.

\section{Works}\label{works}

\begin{itemize}
\item
  \emph{From Protest to Politics: The Future of the Civil Rights
  Movement.}
\item
  \emph{---Ghost written by Kahn, according to Horowitz (2007), pp.}
\item
  \emph{1 (Winter 1964), pp.}
\end{itemize}

"The Power of the March --- And After," Dissent, vol. 10, no. 4 (Autumn
1963), pp.~316--320.

"Problems of the Negro Movement," Dissent, vol. 11, no. 1 (Winter 1964),
pp.~108--138.

The Economics of Equality. Foreword by A. Philip Randolph and Michael
Harrington. New York: League for Industrial Democracy, 1964.

From Protest to Politics: The Future of the Civil Rights Movement. New
York: League for Industrial Democracy, Feb. 1965. ---Ghost written by
Kahn, according to Horowitz (2007), pp.~223--224.

"Problems of the Negro Movement," in Irving Howe (ed.), The Radical
Papers. Garden City, NY: Doubleday and Co., 1966; pp.~144--169.

"Direct Action and Democratic Values," Dissent, vol. 13, no. 1, whole
no. 50 (Jan.-Feb. 1966), pp.~22--30.

"The Riots and the Radicals," Dissent, vol. 14, no. 5, whole no. 60
(Sept.-Oct. 1967), pp.~517--526.

"The Problem of the New Left," Commentary, vol. 42 (July 1966),
pp.~30--38.

"Max Shachtman: His Ideas and His Movement," New America, Nov. 15, 1972.

"Farewell to a Decade of Illusions," New America, vol. 11 (Dec. 1980),
pp.~6--9.

"How to Support Solidarnosc: A Debate." With Norman Podhoretz;
introduction by Midge Decter; moderated by Carl Gershman. Democratiya,
vol. 13 (Summer 2008), pp.~230--261.

"Moral Duty," Transaction, vol. 19, no. 3 (March 1982), pg. 51.

"Beyond the Double Standard: A Social Democratic View of the
Authoritarianism versus Totalitarianism Debate," New America, July 1985.
---Speech of January 1985.

\section{Notes}\label{notes}

\section{References}\label{references}

\begin{itemize}
\item
  \emph{"Tom Kahn obituary".}
\item
  \emph{"A death in the office: AIDS took Tom Kahn.}
\item
  \emph{"Tom Kahn and the fight for democracy: A political portrait and
  personal recollection".}
\item
  \emph{Kahn, Tom; Podhoretz, Norman (Summer 2008).}
\item
  \emph{"Tom Kahn, leader in labor and rights movements, was 53".}
\item
  \emph{"The gallant warrior: In memoriam Tom Kahn" (PDF).}
\end{itemize}

Anderson, Jervis (1997). Bayard Rustin: Troubles I've seen. New York:
HarperCollins Publishers. ISBN~978-0-06-016702-8.

Bernstein, Carl (February 24, 1992). "The holy alliance: Ronald Reagan
and John Paul II; How Reagan and the Pope conspired to assist Poland's
Solidarity movement and hasten the demise of Communism". Time (U.S.
ed.): 28--35.

Chenoweth, Eric (Summer 1992). "The gallant warrior: In memoriam Tom
Kahn" (PDF). Uncaptive minds: A journal of information and opinion on
Eastern Europe. 1718 M Street, NW, No. 147, Washington DC 20036, USA:
Institute for Democracy in Eastern Europe (IDEE). 5 (2 (issue 20)):
5--16. ISSN~0897-9669. Archived from the original (pdf) on 2015-10-19.

D'Emilio, John (2003). Lost prophet: Bayard Rustin and the quest for
peace and justice in America. New York: The Free Press.
ISBN~0-684-82780-8. ISBN~9780684827803. Republished as Lost prophet: The
life and times of Bayard Rustin (Chicago: The University of Chicago
Press, 2004). ISBN~0-226-14269-8

Domber, Gregory F. (2008). Supporting the revolution: America,
democracy, and the end of the Cold War in Poland, 1981--1989 (Ph.D.
dissertation (12 September 2007), George Washington University).
ProQuest. pp.~1--506. ISBN~0-549-38516-9. Winner of the "2009 Betty M.
Unterberger Prize for Best Dissertation on United States Foreign Policy
from the Society for Historians of American Foreign Relations". Revised
and incorporated in Domber, Gregory F. (2014). Empowering Revolution:
America, Poland, and the End of the Cold War. The New Cold War History.
University of North Carolina Press books. ISBN~9781469618517.

Gitlin, Todd (1987). The Sixties: Years of hope, days of rage. Bantam
Books. ISBN~0-553-37212-2.

Hardesty, Rex (28 March 1992). "Tom Kahn obituary". PR Newswire. News
release from the "AFL--CIO International {[}Affairs Department{]}".
Washington.

Horowitz, Rachelle (Winter 2007) {[}2005{]}. "Tom Kahn and the fight for
democracy: A political portrait and personal recollection". Democratiya
(merged with Dissent in 2009). 11: 204--251. Archived from the original
on March 3, 2011.

Howe, Irving (2010). Rodden, John; Goffman, Ethan (eds.). Politics and
the intellectual: Conversations with Irving Howe. Shofar Supplements in
Jewish Studies. Purdue University Press. ISBN~978-1-55753-551-1.

Isserman, Maurice (1987). If I had a hammer...The death of the old left
and the birth of the new left. Basic Books. ISBN~0-465-03197-8.

Isserman, Maurice (2000). The other American: The life of Michael
Harrington. Public Affairs. ISBN~1-58648-036-7.

Kahn, Tom; Podhoretz, Norman (Summer 2008). sponsored by the Committee
for the Free World and the League for Industrial Democracy, with
introduction by Midge Decter and moderation by Carl Gershman, and held
at the Polish Institute for Arts and Sciences, New York City in March
1981. "How to support Solidarnosc: A debate" (PDF). Democratiya (merged
with Dissent in 2009). 13: 230--261. Archived from the original (PDF) on
2012-03-20.

Kastor, Elizabeth (12 August 1992). "A death in the office: AIDS took
Tom Kahn. For his co-workers, the pain goes on". The Washington Post.
Washington, D.C. p.~C1. (subscription required).

Miller, James (1987). Democracy is in the streets: From Port Huron to
the siege of Chicago (1994 ed.). Cambridge, Mass.: Harvard University
Press. ISBN~978-0-674-19725-1.

Muravchik, Joshua (January 2006). "Comrades". Commentary Magazine.
Retrieved 15 June 2007.

Phelps, Christopher (January 2007). "A neglected document on socialism
and sex (Appendix reprinting: H. L. Small, "Socialism and sex", Young
Socialist (Winter 1952) p. 21)". Journal of the History of Sexuality. 16
(1): 1--13. doi:10.1353/sex.2007.0042. JSTOR~30114199. PMID~19241620.
Reprinted in New Politics. Summer 2008 Vol:XII, 1; Issue 45.

Puddington, Arch (July 1992). "A hero of the cold war". The American
Spectator. 25 (7): 42--44. (subscription required).

Puddington, Arch (2005). "Surviving the underground: How American unions
helped Solidarity win". American Educator. American Federation of
Teachers (Summer). Retrieved 4 June 2011.

Puddington, Arch (2005K). Lane Kirkland: Champion of American labor.
Hoboken, N.J.: John Wiley and Sons. pp.~73, 134, 176, 182, 186, and 206.
ISBN~0-471-41694-0.

Saxon, Wolfgang (1 April 1992). "Tom Kahn, leader in labor and rights
movements, was 53". New York Times.

Shevis, James M. (1981). "The AFL-CIO and Poland's Solidarity". World
Affairs. World Affairs Institute. 144 (1, Summer): 31--35.
JSTOR~20671880.

Thiel, Rainer (2010). "U.S. democracy assistance in the Polish
liberalization process 1980-1989 (Chapter 6)". Nested games of external
democracy promotion: The United States and the Polish liberalization
1980--1989. VS Verlag. pp.~179--235, especially 204 and 231.
ISBN~978-3-531-17769-4.

Wattenberg, Ben (22 April 1992). "A man whose ideas helped change the
world". Baltimore Sun. Syndicated: (Thursday 23 April 1993).
"Remembering a man who mattered". The Indiana Gazette p. 2 (pdf format).
Retrieved 19 November 2011.

\section{External links}\label{external-links}

\begin{itemize}
\item
  \emph{"AFL-CIO support for Solidarity: Political, financial, moral".}
\item
  \emph{"Tom~Kahn papers: A finding aid to the collection in the Library
  of Congress" (PDF).}
\item
  \emph{"The papers of Tom~Kahn, civil~rights and labor activist, were
  given to the Library of Congress by Rachelle~Horowitz and
  Eric~Chenoweith {[}sic.}
\end{itemize}

Chenoweth, Eric (October 2010). "AFL-CIO support for Solidarity:
Political, financial, moral". 1718 M Street, NW, No. 147, Washington DC
20036, USA: Institute for Democracy in Eastern Europe (IDEE)

Cartledge, Connie~L. (2010) {[}2009{]}. "Tom~Kahn papers: A finding aid
to the collection in the Library of Congress" (PDF). "The papers of
Tom~Kahn, civil~rights and labor activist, were given to the Library of
Congress by Rachelle~Horowitz and Eric~Chenoweith {[}sic.; Chenoweth{]}
in 2006." (p.~2) ("Finding aid encoded by Library~of~Congress
Manuscript~Division, 2010" ed.). Washington,~D.C.: Manuscript Division,
Library of Congress: 1--5. MSS85310

\section{Photographs}\label{photographs}

\begin{itemize}
\item
  \emph{Tom Kahn with Donald Slaiman of Social Democrats, USA.}
\item
  \emph{Bayard Rustin and the civil~rights movement.}
\end{itemize}

Picture of Tom Kahn---with Rachelle Horowitz, James Farmer (CORE
leader), and Ernest Green---at 1964 World's Fair, protesting poverty,
before their arrest. in Levine, Daniel (2000). Bayard Rustin and the
civil~rights movement. New Jersey: Rutgers University Press. p.~352.
ISBN~0-8135-2718-X.

Tom Kahn with Donald Slaiman of Social Democrats, USA.
