\textbf{From Wikipedia, the free encyclopedia}

https://en.wikipedia.org/wiki/List\%20of\%20elected\%20socialist\%20mayors\%20in\%20the\%20United\%20States\\
Licensed under CC BY-SA 3.0:\\
https://en.wikipedia.org/wiki/Wikipedia:Text\_of\_Creative\_Commons\_Attribution-ShareAlike\_3.0\_Unported\_License

\section{List of elected socialist mayors in the United
States}\label{list-of-elected-socialist-mayors-in-the-united-states}

\begin{itemize}
\item
  \emph{The following is a list of mayors who have declared themselves
  to be socialists or have been a member of a socialist party in the
  United States.}
\end{itemize}

The following is a list of mayors who have declared themselves to be
socialists or have been a member of a socialist party in the United
States.

In 1911 it was estimated that there were twenty-eight such mayors and in
1913 thirty-four. In 1967, however, James Weinstein's table of "Cities
and Towns Electing Socialist Mayors or Other Major Municipal Officers,
1911--1920" counted 74 such municipalities in 1911 and 32 in 1913, with
smaller peaks in 1915 (22) and 1917 (18).

\section{List of mayors}\label{list-of-mayors}

\section{Notes}\label{notes}

\begin{itemize}
\item
  \emph{IV\^{} Chase and Coulter were both elected mayor for the Social
  Democratic Party, but the party later merged itself with a dissident
  faction of the Socialist Labor Party in 1901 and founded the Socialist
  Party of America.}
\item
  \emph{IX\^{} Van Lear was expelled from the Socialist Party in 1918}
\item
  \emph{VII\^{} Lumumba was self-described as a socialist.}
\end{itemize}

I\^{} Barewald resigned from the Socialist Party during the first week
of January 1921 and captured national headlines by declaring radicals
"insane" and instructing local police to greet unwanted members of the
Industrial Workers of the World with "hot lead." See: "Wants Town Rid of
IWW: Mayor Barewald Advises Use of Riot Guns," Eugene Morning Register,
Jan. 9, 1921, pg. 1.

II\^{} Ran for the Rockford Progressive Party, which was formed by
dissidents of the Rockford Labor Party in 1929.

III\^{} Clavelle became a member of the Democratic Party in 2004.

IV\^{} Chase and Coulter were both elected mayor for the Social
Democratic Party, but the party later merged itself with a dissident
faction of the Socialist Labor Party in 1901 and founded the Socialist
Party of America.

V\^{} His name is alternatively spelled Lewis J. Duncan.

VI\^{} Was running for the Rockford Labor Legion from 1921--1927, in
1929 the Labor Party refused to nominate him on the grounds that he had
moved from some of the party's principles. He ran as an independent from
1929-33.\\
The Rockford Labor Legion was a coalition of local trade unions,
socialist organizations and temperance societies.{[}10{]}

VII\^{} Lumumba was self-described as a socialist.

VIII\^{} Sanders has declared himself to be a democratic socialist.

IX\^{} Van Lear was expelled from the Socialist Party in 1918

\section{Footnotes}\label{footnotes}

\section{Bibliography}\label{bibliography}

\begin{itemize}
\item
  \emph{Jack Ross, The Socialist Party of America: A Complete History.}
\end{itemize}

Benjamin F. Arrington, Municipal History of Essex County in
Massachusetts. Chicago: Lewis Historical Publishing Company, 1922; pg.
976.

Henry F. Bedford, Socialism and the Workers in Massachusetts, 1886-1912.
Amherst, MA: University of Massachusetts Press, 1966.

Henry Bengston, On the Left in America: Memoirs of the
Scandinavian-American Labor Movement. SIU Press, 1999; pg. 237.

Hiram Taylor French, History of Idaho: A Narrative Account of Its
Historical Progress, Its People and Its Principal Interests. New York:
New York Public Library, 1914; pg. 976.

C. Hal Nelson, Sinnissippi Saga: A History of Rockford and Winnebago
County, Illinois. Winnebago County Illinois Sesquicentennial Committee,
1968; pg. 536.

Jack Ross, The Socialist Party of America: A Complete History. Lincoln,
NE: Potomac Books, 2015; pp.~609--638.

James Weinstein, The Decline of Socialism in America 1912--1925. New
York: Monthly Review Press, 1967; pp.~116--118.

\section{External links}\label{external-links}
