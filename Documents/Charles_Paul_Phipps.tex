\textbf{From Wikipedia, the free encyclopedia}

https://en.wikipedia.org/wiki/Charles\_Paul\_Phipps\\
Licensed under CC BY-SA 3.0:\\
https://en.wikipedia.org/wiki/Wikipedia:Text\_of\_Creative\_Commons\_Attribution-ShareAlike\_3.0\_Unported\_License

\section{Charles Paul Phipps}\label{charles-paul-phipps}

\begin{itemize}
\item
  \emph{\\
  Charles Paul Phipps (1815--1880), of Chalcot House, Westbury,
  Wiltshire, was an English merchant in Brazil and later Conservative MP
  for Westbury (1869--1874) and High Sheriff of Wiltshire (1875).}
\end{itemize}

Charles Paul Phipps (1815--1880), of Chalcot House, Westbury, Wiltshire,
was an English merchant in Brazil and later Conservative MP for Westbury
(1869--1874) and High Sheriff of Wiltshire (1875).

\section{Origins}\label{origins}

\begin{itemize}
\item
  \emph{Charles Paul Phipps was the eighth son of Thomas Henry Hele
  Phipps (1777--1841), of Leighton House, Westbury, Wiltshire, and Mary
  Michael Joseph Leckonby (1777--1835).}
\end{itemize}

Charles Paul Phipps was the eighth son of Thomas Henry Hele Phipps
(1777--1841), of Leighton House, Westbury, Wiltshire, and Mary Michael
Joseph Leckonby (1777--1835). The Phippses had originally emerged as
prominent Wiltshire clothiers in the 16th century. Over the next hundred
years prosperity propelled them into the ranks of the landed gentry but,
by the early 19th century, they found themselves in rather reduced
financial circumstances.

\section{Coffee merchant}\label{coffee-merchant}

\begin{itemize}
\item
  \emph{In 1830, at the age of 15, Phipps was sent to Rio de Janeiro
  with twenty pounds in his pocket to seek his fortune.}
\item
  \emph{In 1837 he went into partnership with his brother, John Lewis
  Phipps, buying out the Brazilian coffee business of Heyworth
  Brothers.}
\end{itemize}

In 1830, at the age of 15, Phipps was sent to Rio de Janeiro with twenty
pounds in his pocket to seek his fortune. In 1837 he went into
partnership with his brother, John Lewis Phipps, buying out the
Brazilian coffee business of Heyworth Brothers. Despite a number of
alarms, the business eventually flourished, becoming for a while one of
the largest coffee exporters from Brazil. Between 1850 and the
mid-1870s, the volume of coffee exported by the firm increased from
94,000 to about half a million bags per annum (valued at £2,000,000).

\section{Political career}\label{political-career}

\begin{itemize}
\item
  \emph{Phipps died on 8 June 1880, having suffered a stroke the
  previous year.}
\item
  \emph{In 1869, Phipps was elected as the Conservative Member of
  Parliament for Westbury, by 499 votes to 488 for the Liberal
  candidate, Abraham Laverton.}
\end{itemize}

In 1869, Phipps was elected as the Conservative Member of Parliament for
Westbury, by 499 votes to 488 for the Liberal candidate, Abraham
Laverton. He lost his seat to Laverton in 1874 by 22 votes.

Phipps died on 8 June 1880, having suffered a stroke the previous year.

\section{Family}\label{family}

\begin{itemize}
\item
  \emph{Their second son, William Wilton Phipps, was the grandfather of
  both Joyce Grenfell and Simon Wilton Phipps MC, the Bishop of
  Lincoln.}
\item
  \emph{Their eldest son, Charles Nicholas Paul Phipps, was also
  subsequently MP for Westbury and High Sheriff of Wiltshire.}
\item
  \emph{In 1844, Phipps married Emma Mary Benson, who came from a
  mercantile family, being the daughter of Moses Benson of Liverpool and
  granddaughter of Moses Benson (1738--1806).}
\end{itemize}

In 1844, Phipps married Emma Mary Benson, who came from a mercantile
family, being the daughter of Moses Benson of Liverpool and
granddaughter of Moses Benson (1738--1806). Their eldest son, Charles
Nicholas Paul Phipps, was also subsequently MP for Westbury and High
Sheriff of Wiltshire. Their second son, William Wilton Phipps, was the
grandfather of both Joyce Grenfell and Simon Wilton Phipps MC, the
Bishop of Lincoln.

\section{References}\label{references}

\section{Sources}\label{sources}

\begin{itemize}
\item
  \emph{Papers of the Phipps Family of Chalcot (1574--1988) (Ref.540),
  Wiltshire and Swindon Record Office}
\item
  \emph{Notes on the 'Westbury' Phipps Pedigrees by John C. Phipps
  (1983, unpublished)}
\end{itemize}

Notes on the 'Westbury' Phipps Pedigrees by John C. Phipps (1983,
unpublished)

Papers of the Phipps Family of Chalcot (1574--1988) (Ref.540), Wiltshire
and Swindon Record Office

\section{External links}\label{external-links}

\begin{itemize}
\item
  \emph{Hansard 1803--2005: contributions in Parliament by Charles Paul
  Phipps}
\end{itemize}

Hansard 1803--2005: contributions in Parliament by Charles Paul Phipps
