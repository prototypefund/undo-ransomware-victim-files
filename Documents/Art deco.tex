\textbf{From Wikipedia, the free encyclopedia}

https://en.wikipedia.org/wiki/Art\%20deco\\
Licensed under CC BY-SA 3.0:\\
https://en.wikipedia.org/wiki/Wikipedia:Text\_of\_Creative\_Commons\_Attribution-ShareAlike\_3.0\_Unported\_License

\section{Art Deco}\label{art-deco}

\begin{itemize}
\item
  \emph{Art Deco, sometimes referred to as Deco, is a style of visual
  arts, architecture and design that first appeared in France just
  before World War I.}
\item
  \emph{The Chrysler Building and other skyscrapers of New York built
  during the 1920s and 1930s are monuments of the Art Deco style.}
\item
  \emph{In the 1930s, during the Great Depression, the Art Deco style
  became more subdued.}
\end{itemize}

Art Deco, sometimes referred to as Deco, is a style of visual arts,
architecture and design that first appeared in France just before World
War I. Art Deco influenced the design of buildings, furniture, jewelry,
fashion, cars, movie theatres, trains, ocean liners, and everyday
objects such as radios and vacuum cleaners. It took its name, short for
Arts Décoratifs, from the Exposition internationale des arts décoratifs
et industriels modernes (International Exhibition of Modern Decorative
and Industrial Arts) held in Paris in 1925. It combined modern styles
with fine craftsmanship and rich materials. During its heyday, Art Deco
represented luxury, glamour, exuberance, and faith in social and
technological progress.

Art Deco was a pastiche of many different styles, sometimes
contradictory, united by a desire to be modern. From its outset, Art
Deco was influenced by the bold geometric forms of Cubism; the bright
colors of Fauvism and of the Ballets Russes; the updated craftsmanship
of the furniture of the eras of Louis Philippe I and Louis XVI; and the
exotic styles of China and Japan, India, Persia, ancient Egypt and Maya
art. It featured rare and expensive materials, such as ebony and ivory,
and exquisite craftsmanship. The Chrysler Building and other skyscrapers
of New York built during the 1920s and 1930s are monuments of the Art
Deco style.

In the 1930s, during the Great Depression, the Art Deco style became
more subdued. New materials arrived, including chrome plating, stainless
steel, and plastic. A sleeker form of the style, called Streamline
Moderne, appeared in the 1930s; it featured curving forms and smooth,
polished surfaces. Art Deco is one of the first truly international
styles, but its dominance ended with the beginning of World War II and
the rise of the strictly functional and unadorned styles of modern
architecture and the International Style of architecture that followed.

\section{Naming}\label{naming}

\begin{itemize}
\item
  \emph{Art Deco gained currency as a broadly applied stylistic label in
  1968 when historian Bevis Hillier published the first major academic
  book on the style: Art Deco of the 20s and 30s.}
\item
  \emph{Art Deco took its name, short for Arts Décoratifs, from the
  Exposition Internationale des Arts Décoratifs et Industriels Modernes
  held in Paris in 1925, though the diverse styles that characterize Art
  Deco had already appeared in Paris and Brussels before World War I.}
\end{itemize}

Art Deco took its name, short for Arts Décoratifs, from the Exposition
Internationale des Arts Décoratifs et Industriels Modernes held in Paris
in 1925, though the diverse styles that characterize Art Deco had
already appeared in Paris and Brussels before World War I.

The term arts décoratifs was first used in France in 1858; published in
the Bulletin de la Société française de photographie.

In 1868, Le Figaro newspaper used the term objets d'art décoratifs with
respect to objects for stage scenery created for the Théâtre de l'Opéra.

In 1875, furniture designers, textile, jewelry and glass designers, and
other craftsmen were officially given the status of artists by the
French government. In response to this, the École royale gratuite de
dessin (Royal Free School of Design) founded in 1766 under King Louis
XVI to train artists and artisans in crafts relating to the fine arts,
was renamed the National School of Decorative Arts (l'École nationale
des arts décoratifs). It took its present name of ENSAD (École nationale
supérieure des arts décoratifs) in 1927.

During the 1925 Exposition the architect Le Corbusier wrote a series of
articles about the exhibition for his magazine L'Esprit Nouveau under
the title, "1925 EXPO. ARTS. DÉCO." which were combined into a book,
"L'art décoratif d'aujourd'hui" (Decorative Art Today). The book was a
spirited attack on the excesses of the colorful and lavish objects at
the Exposition; and on the idea that practical objects such as furniture
should have any decoration at all; his conclusion was that "Modern
decoration has no decoration".

The actual phrase "Art déco" did not appear in print until 1966, when it
featured in the title of the first modern exhibit on the subject, called
Les Années 25~: Art déco, Bauhaus, Stijl, Esprit nouveau, which covered
the variety of major styles in the 1920s and 1930s. The term Art déco
was then used in a 1966 newspaper article by Hillary Gelson in the Times
(London, 12 November), describing the different styles at the exhibit.

Art Deco gained currency as a broadly applied stylistic label in 1968
when historian Bevis Hillier published the first major academic book on
the style: Art Deco of the 20s and 30s. Hillier noted that the term was
already being used by art dealers and cites The Times (2 November 1966)
and an essay named "Les Arts Déco" in Elle magazine (November 1967) as
examples of prior usage. In 1971, Hillier organized an exhibition at the
Minneapolis Institute of Arts, which he details in his book about it,
The World of Art Deco.

\section{Origins}\label{origins}

\section{Society of Decorative Artists
(1901--1913)}\label{society-of-decorative-artists-19011913}

\begin{itemize}
\item
  \emph{The emergence of Art Deco was closely connected with the rise in
  status of decorative artists, who until late in the 19th century had
  been considered simply as artisans.}
\item
  \emph{The early art deco style featured luxurious and exotic materials
  such as ebony, and ivory and silk, very bright colors and stylized
  motifs, particularly baskets and bouquets of flowers of all colors,
  giving a modernist look.}
\end{itemize}

The emergence of Art Deco was closely connected with the rise in status
of decorative artists, who until late in the 19th century had been
considered simply as artisans. The term "arts décoratifs" had been
invented in 1875, giving the designers of furniture, textiles, and other
decoration official status. The Société des artistes décorateurs
(Society of decorative artists), or SAD, was founded in 1901, and
decorative artists were given the same rights of authorship as painters
and sculptors. A similar movement developed in Italy. The first
international exhibition devoted entirely to the decorative arts, the
Esposizione international d'Arte decorative moderna, was held in Turin
in 1902. Several new magazines devoted to decorative arts were founded
in Paris, including Arts et décoration and L'Art décoratif moderne.
Decorative arts sections were introduced into the annual salons of the
Sociéte des artistes français, and later in the Salon d'automne. French
nationalism also played a part in the resurgence of decorative arts;
French designers felt challenged by the increasing exports of less
expensive German furnishings. In 1911, the SAD proposed the holding of a
major new international exposition of decorative arts in 1912. No copies
of old styles were to be permitted; only modern works. The exhibit was
postponed until 1914, then, because of the war, postponed until 1925,
when it gave its name to the whole family of styles known as Déco.

Parisian department stores and fashion designers also played an
important part in the rise of Art Déco. Established firms including the
luggage maker Louis Vuitton, silverware firm Christofle, glass designer
René Lalique, and the jewelers Louis Cartier and Boucheron, who all
began designing products in more modern styles. Beginning in 1900,
department stores had recruited decorative artists to work in their
design studios. The decoration of the 1912 Salon d'Automne had been
entrusted to the department store Printemps. During the same year
Printemps created its own workshop called "Primavera". By 1920 Primavera
employed more than three hundred artists. The styles ranged from the
updated versions of Louis XIV, Louis XVI and especially Louis Philippe
furniture made by Louis Süe and the Primavera workshop to more modern
forms from the workshop of the Au Louvre department store. Other
designers, including Émile-Jacques Ruhlmann and Paul Foliot refused to
use mass production, and insisted that each piece be made individually
by hand. The early art deco style featured luxurious and exotic
materials such as ebony, and ivory and silk, very bright colors and
stylized motifs, particularly baskets and bouquets of flowers of all
colors, giving a modernist look.

\section{Théâtre des Champs-Élysées
(1910--1913)}\label{thuxe9uxe2tre-des-champs-uxe9lysuxe9es-19101913}

\begin{itemize}
\item
  \emph{Perret and Sauvage became the leading Art Deco architects in
  Paris in the 1920s.}
\item
  \emph{Perret's building had clean rectangular form, geometric
  decoration and straight lines, the future trademarks of Art Deco.}
\item
  \emph{The Théâtre des Champs-Élysées (1910--1913), by Auguste Perret
  was the first landmark Art Deco building completed in Paris.}
\end{itemize}

Antoine Bourdelle, La Danse, facade of the Théâtre des Champs-Elysées,
Paris (1912)

Théâtre des Champs-Élysées, by Auguste Perret, 15 avenue Montaigne,
Paris, (1910--13). Reinforced concrete gave architects the ability to
create new forms and bigger spaces.

Interior of the Théâtre des Champs-Élysées, with Bourdelle bas-reliefs
over the stage

Dome of the Theater, with Art-Deco rose design by Maurice Denis

The Théâtre des Champs-Élysées (1910--1913), by Auguste Perret was the
first landmark Art Deco building completed in Paris. Previously
reinforced concrete had been used only for industrial and apartment
buildings, Perret had built the first modern reinforced concrete
apartment building in Paris on rue Benjamin Franklin in 1903--04. Henri
Sauvage, another important future Art Deco architect, built another in
1904 at 7 rue Trétaigne (1904). From 1908 to 1910, the 21-year old Le
Corbusier worked as a draftsman in Perret's office, learning the
techniques of concrete construction. Perret's building had clean
rectangular form, geometric decoration and straight lines, the future
trademarks of Art Deco. The decor of the theater was also revolutionary;
the facade was decorated with high reliefs by Antoine Bourdelle, a dome
by Maurice Denis, paintings by Édouard Vuillard, and an Art Deco curtain
Ker-Xavier Roussel. The theater became famous as the venue for many of
the first performances of the Ballets Russes. Perret and Sauvage became
the leading Art Deco architects in Paris in the 1920s.

\section{Salon d'Automne (1912--1913)}\label{salon-dautomne-19121913}

\begin{itemize}
\item
  \emph{Art Deco armchair made for art collector Jacques Doucet
  (1912--13)}
\item
  \emph{Bright colors were a feature of the work of fashion designer
  Paul Poiret, whose work influenced both Art Deco fashion and interior
  design.}
\item
  \emph{Display of early Art Deco furnishings by the Atelier Français at
  the 1913 Salon d'Automne from Art et décoration magazine (1914)}
\end{itemize}

Set design for Sheherazade (1910) by Leon Bakst

Table and chairs by Maurice Dufrene and carpet by Paul Follot at the
1912 Salon des artistes décorateurs

Art Deco armchair made for art collector Jacques Doucet (1912--13)

Display of early Art Deco furnishings by the Atelier Français at the
1913 Salon d'Automne from Art et décoration magazine (1914)

At its birth between 1910 and 1914, Art Deco was an explosion of colors,
featuring bright and often clashing hues, frequently in floral designs,
presented in furniture upholstery, carpets, screens, wallpaper and
fabrics. Many colorful works, including chairs and a table by Maurice
Dufrene and a bright Gobelin carpet by Paul Follot were presented at the
1912 Salon des artistes décorateurs. In 1912--1913 designer Adrien
Karbowsky made a floral chair with a parrot design for the hunting lodge
of art collector Jacques Doucet. The furniture designers Louis Süe and
André Mare made their first appearance at the 1912 exhibit, under the
name of the Atelier Française, combining colorful fabrics with exotic
and expensive materials, including ebony and ivory. After World War I
they became one of the most prominent French interior design firms,
producing the furniture for the first-class salons and cabins of the
French transatlantic ocean liners.

The vivid colors of Art Deco came from many sources, including the
exotic set designs by Leon Bakst for the Ballets Russes, which caused a
sensation in Paris just before World War I. Some of the colors were
inspired by the earlier Fauvism movement led by Henri Matisse; others by
the Orphism of painters such as Sonia Delaunay; others by the movement
known as the Nabis, and in the work of symbolist painter Odilon Redon,
who designed fireplace screens and other decorative objects. Bright
colors were a feature of the work of fashion designer Paul Poiret, whose
work influenced both Art Deco fashion and interior design.

\section{Cubist influence}\label{cubist-influence}

\begin{itemize}
\item
  \emph{Le Nouveau style, published in the journal L'Art décoratif, he
  expressed the rejection of Art Nouveau forms (asymmetric, polychrome
  and picturesque) and called for simplicité volontaire, symétrie
  manifeste, l'ordre et l'harmonie, themes that would eventually become
  common within Art Deco; though the Deco style was often extremely
  colorful and anything but simple.}
\item
  \emph{The art style known as Cubism appeared in France between 1907
  and 1912, influencing the development of Art Deco.}
\item
  \emph{The Cubist influence continued within Art Deco, even as Deco
  branched out in many other directions.}
\item
  \emph{The geometric forms of Cubism had an important influence on Art
  Deco}
\end{itemize}

Design for the facade of La Maison Cubiste (Cubist House) by Raymond
Duchamp-Villon (1912)

Raymond Duchamp-Villon, 1912, La Maison Cubiste (Cubist House) at the
Salon d'Automne, 1912, detail of the entrance

Le Salon Bourgeois, designed by André Mare inside La Maison Cubiste, in
the decorative arts section of the Salon d'Automne, 1912, Paris.
Metzinger's Femme à l'Éventail on the left wall

Stairway in the hôtel particulier of fashion designer-art collector
Jacques Doucet (1927). Design by Joseph Csaky. The geometric forms of
Cubism had an important influence on Art Deco

Jacques Doucet's hôtel particulier, 1927. Picasso's Les Demoiselles
d'Avignon can be seen hanging in the background

The art style known as Cubism appeared in France between 1907 and 1912,
influencing the development of Art Deco. The Cubists, themselves under
the influence of Paul Cézanne, were interested in the simplification of
forms to their geometric essentials: the cylinder, the sphere, the cone.

In 1912, the artists of the Section d'Or exhibited works considerably
more accessible to the general public than the analytical Cubism of
Picasso and Braque. The Cubist vocabulary was poised to attract fashion,
furniture and interior designers.

In the 1912 writings of André Vera. Le Nouveau style, published in the
journal L'Art décoratif, he expressed the rejection of Art Nouveau forms
(asymmetric, polychrome and picturesque) and called for simplicité
volontaire, symétrie manifeste, l'ordre et l'harmonie, themes that would
eventually become common within Art Deco; though the Deco style was
often extremely colorful and anything but simple.

In the Art Décoratif section of the 1912 Salon d'Automne, an
architectural installation was exhibited known as the La Maison Cubiste.
The facade was designed by Raymond Duchamp-Villon. The decor of the
house was by André Mare. La Maison Cubiste was a furnished installation
with a facade, a staircase, wrought iron banisters, a bedroom, a living
room---the Salon Bourgeois, where paintings by Albert Gleizes, Jean
Metzinger, Marie Laurencin, Marcel Duchamp, Fernand Léger and Roger de
La Fresnaye were hung. Thousands of spectators at the salon passed
through the full-scale model.

The facade of the house, designed by Duchamp-Villon, was not very
radical by modern standards; the lintels and pediments had prismatic
shapes, but otherwise the facade resembled an ordinary house of the
period. For the two rooms, Mare designed the wallpaper, which featured
stylized roses and floral patterns, along with upholstery, furniture and
carpets, all with flamboyant and colorful motifs. It was a distinct
break from traditional decor. The critic Emile Sedeyn described Mare's
work in the magazine Art et Décoration: "He does not embarrass himself
with simplicity, for he multiplies flowers wherever they can be put. The
effect he seeks is obviously one of picturesqueness and gaiety. He
achieves it." The Cubist element was provided by the paintings. The
installation was attacked by some critics as extremely radical, which
helped make for its success. This architectural installation was
subsequently exhibited at the 1913 Armory Show, New York, Chicago and
Boston. Thanks largely to the exhibition, the term "Cubist" began to be
applied to anything modern, from women's haircuts to clothing to theater
performances.

The Cubist influence continued within Art Deco, even as Deco branched
out in many other directions. In 1927, Cubists Joseph Csaky, Jacques
Lipchitz, Louis Marcoussis, Henri Laurens, the sculptor Gustave Miklos,
and others collaborated in the decoration of a Studio House, rue
Saint-James, Neuilly-sur-Seine, designed by the architect Paul Ruaud and
owned by the French fashion designer Jacques Doucet, also a collector of
Post-Impressionist art by Henri Matisse and Cubist paintings (including
Les Demoiselles d'Avignon, which he bought directly from Picasso's
studio). Laurens designed the fountain, Csaky designed Doucet's
staircase, Lipchitz made the fireplace mantel, and Marcoussis made a
Cubist rug.

Besides the Cubist artists, Doucet brought in other Deco interior
designers to help in decorating the house, including Pierre Legrain, who
was in charge of organizing the decoration, and Paul Iribe, Marcel
Coard, André Groult, Eileen Gray and Rose Adler to provide furniture.
The decor included massive pieces made of macassar ebony, inspired by
African art, and furniture covered with Morocco leather, crocodile skin
and snakeskin, and patterns taken from African designs.

\section{Influences}\label{influences}

\begin{itemize}
\item
  \emph{Art Deco was not a single style, but a collection of different
  and sometimes contradictory styles.}
\item
  \emph{Stylized floral designs and bright colors were a feature of
  early Art Deco.}
\item
  \emph{Art Deco also used the clashing colors and designs of Fauvism,
  notably in the work of Henri Matisse and André Derain, inspired the
  designs of art deco textiles, wallpaper, and painted ceramics.}
\item
  \emph{In decoration, many different styles were borrowed and used by
  Art Deco.}
\end{itemize}

The exoticism of the Ballets Russes had a strong influence on early
Deco. A drawing of the dancer Vaslav Nijinsky by Paris fashion artist
Georges Barbier (1913)

Illustration by Georges Barbier of a gown by Paquin (1914). Stylized
floral designs and bright colors were a feature of early Art Deco.

Lobby of 450 Sutter Street in San Francisco by Timothy Pflueger, (1929)
inspired by ancient Maya art

The gilded bronze Prometheus at Rockefeller Center by Paul Manship
(1934), a stylized Art Deco update of classical sculpture (1936)

A ceramic vase inspired by motifs of traditional African carved wood
sculpture, by Emile Lenoble (1937), Museum of Decorative Arts, Paris

Art Deco was not a single style, but a collection of different and
sometimes contradictory styles. In architecture, Art Deco was the
successor to and reaction against Art Nouveau, a style which flourished
in Europe between 1895 and 1900, and also gradually replaced the
Beaux-Arts and neoclassical that were predominant in European and
American architecture. In 1905 Eugène Grasset wrote and published
Méthode de Composition Ornementale, Éléments Rectilignes, in which he
systematically explored the decorative (ornamental) aspects of geometric
elements, forms, motifs and their variations, in contrast with (and as a
departure from) the undulating Art Nouveau style of Hector Guimard, so
popular in Paris a few years earlier. Grasset stressed the principle
that various simple geometric shapes like triangles and squares are the
basis of all compositional arrangements. The reinforced concrete
buildings of Auguste Perret and Henri Sauvage, and particularly the
Theatre des Champs-Elysees, offered a new form of construction and
decoration which was copied worldwide.

In decoration, many different styles were borrowed and used by Art Deco.
They included pre-modern art from around the world and observable at the
Musée du Louvre, Musée de l'Homme and the Musée national des Arts
d'Afrique et d'Océanie. There was also popular interest in archeology
due to excavations at Pompeii, Troy, and the tomb of the 18th dynasty
Pharaoh Tutankhamun. Artists and designers integrated motifs from
ancient Egypt, Mesopotamia, Greece, Rome, Asia, Mesoamerica and Oceania
with Machine Age elements.

Other styles borrowed included Russian Constructivism and Italian
Futurism, as well as Orphism, Functionalism, and Modernism in general.
Art Deco also used the clashing colors and designs of Fauvism, notably
in the work of Henri Matisse and André Derain, inspired the designs of
art deco textiles, wallpaper, and painted ceramics. It took ideas from
the high fashion vocabulary of the period, which featured geometric
designs, chevrons, zigzags, and stylized bouquets of flowers. It was
influenced by discoveries in Egyptology, and growing interest in the
Orient and in African art. From 1925 onwards, it was often inspired by a
passion for new machines, such as airships, automobiles and ocean
liners, and by 1930 this influence resulted in the style called
Streamline Moderne.

\section{Style of luxury and
modernity}\label{style-of-luxury-and-modernity}

\begin{itemize}
\item
  \emph{A good example of the luxury style of Art Deco is the boudoir of
  the fashion designer Jeanne Lanvin, designed by Armand-Albert Rateau
  (1882--1938) made between 1922--25.}
\item
  \emph{Nothing was cheap about Art Deco: pieces of furniture included
  ivory and silver inlays, and pieces of Art Deco jewelry combined
  diamonds with platinum, jade, and other precious materials.}
\item
  \emph{An Art Deco study by the Paris design firm of Alavoine, now in
  the Brooklyn Museum (1928--30)}
\end{itemize}

The boudoir of fashion designer Jeanne Lanvin (1922--25) now in the
Museum of Decorative Arts, Paris

Bath of Jeanne Lanvin, of Sienna marble, with decoration of carved
stucco and bronze (1922--25)

An Art Deco study by the Paris design firm of Alavoine, now in the
Brooklyn Museum (1928--30)

Glass Salon (Le salon de verre) designed by Paul Ruaud with furniture by
Eileen Gray, for Madame Mathieu-Levy (milliner of the boutique J.
Suzanne Talbot), 9, rue de Lota, Paris, 1922 (published in
L'Illustration, 27 May 1933)

Art Deco was associated with both luxury and modernity; it combined very
expensive materials and exquisite craftsmanship put into modernistic
forms. Nothing was cheap about Art Deco: pieces of furniture included
ivory and silver inlays, and pieces of Art Deco jewelry combined
diamonds with platinum, jade, and other precious materials. The style
was used to decorate the first-class salons of ocean liners, deluxe
trains, and skyscrapers. It was used around the world to decorate the
great movie palaces of the late 1920s and 1930s. Later, after the Great
Depression, the style changed and became more sober.

A good example of the luxury style of Art Deco is the boudoir of the
fashion designer Jeanne Lanvin, designed by Armand-Albert Rateau
(1882--1938) made between 1922--25. It was located in her house at 16
rue Barbet de Jouy, in Paris, which was demolished in 1965. The room was
reconstructed in the Museum of Decorative Arts in Paris. The walls are
covered with molded lambris below sculpted bas-reliefs in stucco. The
alcove is framed with columns of marble on with bases and a plinth of
sculpted wood. The floor is of white and black marble, and in the
cabinets decorative objects are displayed against a background of blue
silk. Her bathroom had a tub and washstand made of sienna marble, with a
wall of carved stucco and bronze fittings.

By 1928 the style had become more comfortable, with deep leather club
chairs. The study designed by the Paris firm of Alavoine for an American
businessman in 1928--30, now in the Brooklyn Museum, had a unique
American feature. Since it was constructed during Prohibition, when
serving alcohol was prohibited, it included a secret bar hidden behind
the panels.{[}not in citation given{]}

By the 1930s, the style had been somewhat simplified, but it was still
extravagant. In 1932 the decorator Paul Ruoud made the Glass Salon for
Suzanne Talbot. It featured a serpentine armchair and two tubular
armchairs by Eileen Gray, a floor of mat silvered glass slabs, a panel
of abstract patterns in silver and black lacquer, and an assortment of
animal skins.

\section{International Exhibition of Modern Decorative and Industrial
Arts
(1925)}\label{international-exhibition-of-modern-decorative-and-industrial-arts-1925}

\begin{itemize}
\item
  \emph{Postcard of the International Exhibition of Modern Decorative
  and Industrial Arts in Paris (1925)}
\item
  \emph{The Hôtel du Collectionneur was a popular attraction at the
  Exposition; it displayed the new furniture designs of Emile-Jacques
  Ruhlmann, as well as Art Deco fabrics, carpets, and a painting by Jean
  Dupas.}
\end{itemize}

Postcard of the International Exhibition of Modern Decorative and
Industrial Arts in Paris (1925)

Entrance to the 1925 Exposition from Place de la Concorde by Pierre
Patout

Polish pavilion (1925)

Pavilion of the Galeries Lafayette Department Store at the 1925
Exposition

The Hotel du Collectionneur, pavilion of the furniture manufacturer
Émile-Jacques Ruhlmann, designed by Pierre Patout.

Salon of the Hôtel du Collectionneur from the 1925 International
Exposition of Decorative Arts, furnished by Émile-Jacques Ruhlmann,
painting by Jean Dupas, design by Pierre Patout

The event that marked the zenith of the style and gave it its name was
the International Exhibition of Modern Decorative and Industrial Arts
which took place in Paris from April to October in 1925. This was
officially sponsored by the French government, and covered a site in
Paris of 55 acres, running from the Grand Palais on the right bank to
Les Invalides on the left bank, and along the banks of the Seine. The
Grand Palais, the largest hall in the city, was filled with exhibits of
decorative arts from the participating countries. There were 15,000
exhibitors from twenty different countries, including England, Italy,
Spain, Poland, Czechoslovakia, Belgium, Japan, and the new Soviet Union,
though Germany was not invited because of tensions after the war and the
United States, misunderstanding the purpose of the exhibit, declined to
participate. It was visited by sixteen million people during its
seven-month run. The rules of the exhibition required that all work be
modern; no historical styles were allowed. The main purpose of the
Exhibit was to promote the French manufacturers of luxury furniture,
porcelain, glass, metal work, textiles and other decorative products. To
further promote the products, all the major Paris department stores and
major designers had their own pavilions. The Exposition had a secondary
purpose in promoting products from French colonies in Africa and Asia,
including ivory and exotic woods.

The Hôtel du Collectionneur was a popular attraction at the Exposition;
it displayed the new furniture designs of Emile-Jacques Ruhlmann, as
well as Art Deco fabrics, carpets, and a painting by Jean Dupas. The
interior design followed the same principles of symmetry and geometric
forms which set it apart from Art Nouveau, and bright colors, fine
craftsmanship rare and expensive materials which set it apart from the
strict functionality of the Modernist style. While most of the pavilions
were lavishly decorated and filled with hand-made luxury furniture, two
pavilions, those of the Soviet Union and Pavilion du Nouveau Esprit,
built by the magazine of that name run by Le Corbusier, were built in an
austere style with plain white walls and no decoration; they were among
the earliest examples of modernist architecture.

\section{Skyscrapers}\label{skyscrapers}

\begin{itemize}
\item
  \emph{The tops of the buildings were decorated with Art Deco crowns
  and spires covered with stainless steel, and, in the case of the
  Chrysler building, with Art Deco gargoyles modeled after radiator
  ornaments, while the entrances and lobbies were lavishly decorated
  with Art Deco sculpture, ceramics, and design.}
\end{itemize}

The American Radiator Building in New York City by Raymond Hood (1924)

Chrysler Building in New York City, by William Van Alen (1928--30)

New York City skyline (1931-1933)

Crown of the General Electric Building (also known as 570 Lexington
Avenue) by Cross \& Cross (1933)

30 Rockefeller Center, now the Comcast Building, by Raymond Hood (1933)

American skyscrapers marked the summit of the Art Deco style; they
became the tallest and most recognizable modern buildings in the world.
They were designed to show the prestige of their builders through their
height, their shape, their color, and their dramatic illumination at
night. The first New York skyscraper, the Woolworth Building, in a
neoclassical style, was completed in 1913, and the American Telephone
and Telegraph Building (1924) had ionic and doric columns and a
classical Doric hypostyle with a frieze. The American Radiator Building
by Raymond Hood (1924) combined Gothic and Deco modern elements in the
design of the building. Black brick on the frontage of the building
(symbolizing coal) was selected to give an idea of solidity and to give
the building a solid mass. Other parts of the facade were covered in
gold bricks (symbolizing fire), and the entry was decorated with marble
and black mirrors. Another early Art Deco skyscraper was Detroit's
Guardian Building, which opened in 1929. Designed by modernist Wirt C.
Rowland, the building was the first to employ stainless steel as a
decorative element, and the extensive use of colored designs in place of
traditional ornaments.

The New York skyline was radically changed by the Chrysler Building in
Manhattan (completed in 1930), designed by William Van Alen. It was a
giant seventy-seven floor tall advertisement for Chrysler automobiles.
The top was crowned by a stainless steel spire, and was ornamented by
deco "gargoyles" in the form of stainless steel radiator cap
decorations. The base of the tower, thirty-three stories above the
street, was decorated with colorful art deco friezes, and the lobby was
decorated with art deco symbols and images expressing modernity.

The Chrysler Building was followed by the Empire State Building by
William F. Lamb (1931) and the RCA Building (now the Comcast Building)
in Rockefeller Center, by Raymond Hood (1933) which together completely
changed the skyline of New York. The tops of the buildings were
decorated with Art Deco crowns and spires covered with stainless steel,
and, in the case of the Chrysler building, with Art Deco gargoyles
modeled after radiator ornaments, while the entrances and lobbies were
lavishly decorated with Art Deco sculpture, ceramics, and design.
Similar buildings, though not quite as tall, soon appeared in Chicago
and other large American cities. The Chrysler Building was soon
surpassed in height by the Empire State Building, in a slightly less
lavish Deco style. Rockefeller Center added a new design element:
several tall building grouped around an open plaza, with a fountain in
the center.

\section{Late Art Deco}\label{late-art-deco}

\begin{itemize}
\item
  \emph{Le Corbusier's ideas were gradually adopted by architecture
  schools, and the aesthetics of Art Deco were abandoned.}
\item
  \emph{The Art Deco interior designer Paul Follot defended Art Deco in
  this way: "We know that man is never content with the indispensable
  and that the superfluous is always needed...If not, we would have to
  get rid of music, flowers, and perfumes..!"}
\end{itemize}

Lincoln Theater in Miami Beach, Florida by Thomas W. Lamb (1936)

The Palais de Chaillot by Louis-Hippolyte Boileau, Jacques Carlu and
Léon Azéma from the 1937 Paris International Exposition

Stairway of the Economic and Social Council in Paris, originally the
Museum of Public Works, built for the 1937 Paris International
Exposition by Auguste Perret (1937)

High School in King City, California, built by Robert Stanton for the
Works Progress Administration (1939)

In 1925 two different competing schools coexisted within Art Deco: the
traditionalists, who had founded the Society of Decorative Artists;
included the furniture designer Emile-Jacques Ruhlmann, Jean Dunard, the
sculptor Antoine Bourdelle, and designer Paul Poiret; they combined
modern forms with traditional craftsmanship and expensive materials. On
the other side were the modernists, who increasingly rejected the past
and wanted a style based upon advances in new technologies, simplicity,
a lack of decoration, inexpensive materials, and mass production. The
modernists founded their own organization, The French Union of Modern
Artists, in 1929. Its members included architects Pierre Chareau,
Francis Jourdain, Robert Mallet-Stevens, Corbusier, and, in the Soviet
Union, Konstantin Melnikov; the Irish designer Eileen Gray, and French
designer Sonia Delaunay, the jewelers Jean Fouquet and Jean Puiforcat.
They fiercely attacked the traditional art deco style, which they said
was created only for the wealthy, and insisted that well-constructed
buildings should be available to everyone, and that form should follow
function. The beauty of an object or building resided in whether it was
perfectly fit to fulfill its function. Modern industrial methods meant
that furniture and buildings could be mass-produced, not made by hand.

The Art Deco interior designer Paul Follot defended Art Deco in this
way: "We know that man is never content with the indispensable and that
the superfluous is always needed...If not, we would have to get rid of
music, flowers, and perfumes..!" However, Le Corbusier was a brilliant
publicist for modernist architecture; he stated that a house was simply
"a machine to live in", and tirelessly promoted the idea that Art Deco
was the past and modernism was the future. Le Corbusier's ideas were
gradually adopted by architecture schools, and the aesthetics of Art
Deco were abandoned. The same features that made Art Deco popular in the
beginning, its craftsmanship, rich materials and ornament, led to its
decline. The Great Depression that began in the United States in 1929,
and reached Europe shortly afterwards, greatly reduced the number of
wealthy clients who could pay for the furnishings and art objects. In
the Depression economic climate, few companies were ready to build new
skyscrapers. Even the Ruhlmann firm resorted to producing pieces of
furniture in series, rather than individual hand-made items. The last
buildings built in Paris in the new style were the Museum of Public
Works by Auguste Perret (now the French Economic, Social and
Environmental Council) and the Palais de Chaillot by Louis-Hippolyte
Boileau, Jacques Carlu and Léon Azéma, and the Palais de Tokyo of the
1937 Paris International Exposition; they looked out at the grandiose
pavilion of Nazi Germany, designed by Albert Speer, which faced the
equally grandiose socialist-realist pavilion of Stalin's Soviet Union.

After World War II the dominant architectural style became the
International Style pioneered by Le Corbusier, and Mies Van der Rohe. A
handful of Art Deco hotels were built in Miami Beach after World War II,
but elsewhere the style largely vanished, except in industrial design,
where it continued to be used in automobile styling and products such as
jukeboxes. In the 1960s, it experienced a modest academic revival,
thanks in part to the writings of architectural historians such as Bevis
Hillier. In the 1970s efforts were made in the United States and Europe
to preserve the best examples of Art Deco architecture, and many
buildings were restored and repurposed. Postmodern architecture, which
first appeared in the 1980s, like Art Deco, often includes purely
decorative features. Deco continues to inspire designers, and is often
used in contemporary fashion, jewelry, and toiletries.

\section{Painting}\label{painting}

\begin{itemize}
\item
  \emph{Art deco painting was by definition decorative, designed to
  decorate a room or work of architecture, so few painters worked
  exclusively in the style, but two painters are closely associated with
  Art Deco.}
\item
  \emph{In the 1930s a dramatic new form of Art Deco painting appeared
  in the United States.}
\item
  \emph{She painted portraits in a realistic, dynamic and colorful Art
  Deco style.}
\end{itemize}

Detail of Time, 1941, ceiling mural in lobby of Rockefeller Center by
the Spanish painter Josep Maria Sert

Reginald Marsh, 1936, Workers sorting the mail, a mural in the U.S.
Customs House in New York

Rockwell Kent, 1938, Art in the Tropics, mural in the William Jefferson
Clinton Federal Building

There was no section set aside for painting at the 1925 Exposition. Art
deco painting was by definition decorative, designed to decorate a room
or work of architecture, so few painters worked exclusively in the
style, but two painters are closely associated with Art Deco. Jean Dupas
painted Art Deco murals for the Bordeaux Pavilion at the 1925 Decorative
Arts Exposition in Paris, and also painted the picture over the
fireplace in the Maison de la Collectioneur exhibit at the 1925
Exposition, which featured furniture by Ruhlmann and other prominent Art
Deco designers. His murals were also prominent in the decor of the
French ocean liner SS Normandie. His work was purely decorative,
designed as a background or accompaniment to other elements of the
decor.

The other painter closely associated with the style is Tamara de
Lempicka. Born in Poland, she emigrated to Paris after the Russian
Revolution. She studied under Maurice Denis and André Lhote, and
borrowed many elements from their styles. She painted portraits in a
realistic, dynamic and colorful Art Deco style.

In the 1930s a dramatic new form of Art Deco painting appeared in the
United States. During the Great Depression, the Federal Art Project of
the Works Progress Administration was created to give work to unemployed
artists. Many were given the task of decorating government buildings,
hospitals and schools. There was no specific art deco style used in the
murals; artists engaged to paint murals in government buildings came
from many different schools, from American regionalism to social
realism; they included Reginald Marsh, Rockwell Kent and the Mexican
painter Diego Rivera. The murals were Art Deco because they were all
decorative and related to the activities in the building or city where
they were painted: Reginald Marsh and Rockwell Kent both decorated U.S.
postal buildings, and showed postal employees at work while Diego Rivera
depicted automobile factory workers for the Detroit Institute of Arts.
Diego Rivera's mural Man at the Crossroads (1933) for Rockefeller Center
featured an unauthorized portrait of Lenin. When Rivera refused to
remove Lenin, the painting was destroyed and a new mural was painted by
the Spanish artist Josep Maria Sert.

\section{Sculpture}\label{sculpture}

\section{Monumental and public
sculpture}\label{monumental-and-public-sculpture}

\begin{itemize}
\item
  \emph{Public art deco sculpture was almost always representational,
  usually of heroic or allegorical figures related to the purpose of the
  building or room.}
\item
  \emph{In the United States, the most prominent Art Deco sculptor for
  public art was Paul Manship, who updated classical and mythological
  subjects and themes in an Art Deco style.}
\item
  \emph{Sculpture was a very common and integral feature of Art Deco
  architecture.}
\end{itemize}

Sculpture was a very common and integral feature of Art Deco
architecture. In France, allegorical bas-reliefs representing dance and
music by Antoine Bourdelle decorated the earliest Art Deco landmark in
Paris, the Théâtre des Champs-Élysées in Paris, in 1912. The 1925 had
major sculptural works placed around the site, pavilions were decorated
with sculptural friezes, and several pavilions devoted to smaller studio
sculpture. In the 1930s, a large group of prominent sculptors made works
for the 1937 Exposition Internationale des Arts et Techniques dans la
Vie Moderne at Chaillot. Alfred Janniot made the relief sculptures on
the facade of the Palais de Tokyo. The Musée d'Art Moderne de la Ville
de Paris, and the esplanade in front of the Palais de Chaillot, facing
the Eiffel Tower, was crowded with new statuary by Charles Malfray,
Henry Arnold, and many others.

Public art deco sculpture was almost always representational, usually of
heroic or allegorical figures related to the purpose of the building or
room. The themes were usually selected by the patrons, not the artist.
Abstract sculpture for decoration was extremely rare.

In the United States, the most prominent Art Deco sculptor for public
art was Paul Manship, who updated classical and mythological subjects
and themes in an Art Deco style. His most famous work was the statue of
Prometheus at Rockefeller Center in New York, a 20th-century adaptation
of a classical subject. Other important works for Rockefeller Center
were made by Lee Lawrie, including the sculptural facade and the Atlas
statue.

During the Great Depression in the United States, many sculptors were
commissioned to make works for the decoration of federal government
buildings, with funds provided by the WPA, or Works Progress
Administration. They included sculptor Sidney Biehler Waugh, who created
stylized and idealized images of workers and their tasks for federal
government office buildings. In San Francisco, Ralph Stackpole provided
sculpture for the facade of the new San Francisco Stock Exchange
building. In Washington DC, Michael Lantz made works for the Federal
Trade Commission building.

In Britain, Deco public statuary was made by Eric Gill for the BBC
Broadcasting House, while Ronald Atkinson decorated the lobby of the
former Daily Express Building in London (1932).

One of the best known and certainly the largest public Art Deco
sculpture is the Christ the Redeemer by the French sculptor Paul
Landowski, completed between 1922 and 1931, located on a mountain top
overlooking Rio de Janeiro, Brazil.

Aluminum statue of Ceres by John Storrs atop the Chicago Board of Trade
Building (1930)

The gilded bronze Prometheus at Rockefeller Center by Paul Manship
(1934), a stylized Art Deco update of classical sculpture (1936)

Portal decoration Wisdom by Lee Lawrie, Rockefeller Center, New York
(1933)

Lee Lawrie, 1936--37, Atlas statue, in front of the Rockefeller Center
in New York (installed 1937)

Man Controlling Trade by Michael Lantz at the Federal Trade Commission
building, Washington, D.C. (1942)

Mail Delivery East, by Edmond Amateis, one of four bas-relief sculptures
on the Robert N. C. Nix, Sr., Federal Building in Philadelphia, 1937

Ralph Stackpole's sculpture group over the door of the San Francisco
Stock Exchange (1930)

'Aerial between Wisdom and Gaiety' by Eric Gill, Facade of BBC
Broadcasting House, London (1932)

Christ the Redeemer by Paul Landowski, (1931), soapstone, Corcovado
Mountain, Rio de Janeiro

\section{Studio sculpture}\label{studio-sculpture}

\begin{itemize}
\item
  \emph{Parallel with these Art Deco sculptors, more avant-garde and
  abstract modernist sculptors were at work in Paris and New York.}
\item
  \emph{Many early Art Deco sculptures were small, designed to decorate
  salons.}
\item
  \emph{One of the best-known Art Deco salon sculptors was the
  Romanian-born Demétre Chiparus, who produced colorful small sculptures
  of dancers.}
\end{itemize}

Many early Art Deco sculptures were small, designed to decorate salons.
One genre of this sculpture was called the Chryselephantine statuette,
named for a style of ancient Greek temple statues made of gold and
ivory. They were sometimes made of bronze, or sometimes with much more
lavish materials, such as ivory. onyx alabaster, and gold leaf.

One of the best-known Art Deco salon sculptors was the Romanian-born
Demétre Chiparus, who produced colorful small sculptures of dancers.
Other notable salon sculptors included Ferdinand Preiss, Josef Lorenzl,
Alexander Kelety, Dorothea Charol and Gustav Schmidtcassel. Another
important American sculptor in the studio format was Harriet Whitney
Frishmuth, who had studied with Auguste Rodin in Paris.

Pierre Le Paguays was a prominent Art Dco studio sculptor, whose work
was shown at the 1925 Exposition. he worked with bronze, marble, ivory,
onyx, gold, alabaster and other precious materials.

Joseph Csaky, Tête (front and side view), limestone, Kröller-Müller
Museum, Otterlo (c. 1920)

"The Hunter" by Pierre Le Faguays (1920s)

Bronze nude of a dancer on an onyx plinth by Josef Lorenzl, c. 1925

Speed, a design for a radiator ornament by the American sculptor Harriet
Whitney Frishmuth (1925)

The Flight of Europa by Paul Manship, bronze with gold leaf, Whitney
Museum (1925)

Demétre Chiparus, Tanara, bronze, ivory and onyx (c. 1925)

Demétre Chiparus, Dancer, bronze, ivory (c. 1925)

François Pompon was a pioneer of modern stylized animalier sculpture. He
was not fully recognized for his artistic accomplishments until the age
of 67 at the Salon d'Automne of 1922 with the work Ours blanc, also
known as The White Bear, now in the Musée d'Orsay in Paris.

Parallel with these Art Deco sculptors, more avant-garde and abstract
modernist sculptors were at work in Paris and New York. The most
prominent were Constantin Brâncuși, Joseph Csaky, Alexander Archipenko,
Henri Laurens, Jacques Lipchitz, Gustave Miklos, Jean Lambert-Rucki, Jan
et Joël Martel, Chana Orloff and Pablo Gargallo.

\section{Graphic arts}\label{graphic-arts}

\begin{itemize}
\item
  \emph{Among the best known French Art Deco poster designers was
  Cassandre, who made the celebrated poster of the ocean liner SS
  Normandie in 1935.}
\item
  \emph{In France popular Art Deco designers included, Charles Loupot
  and Paul Colin, who became famous for his posters of American singer
  and dancer Josephine Baker.}
\item
  \emph{The Art Deco style appeared early in the graphic arts, in the
  years just before World War I.}
\end{itemize}

Program for the Ballets Russes by Leon Bakst (1912)

Peter Behrens, Deutscher Werkbund exhibition poster (1914)

A Vanity Fair cover by Georges Lepape (1919)

Interpretation of Harlem Jazz I by Winold Reiss (c.1920)

Cover of Harper's Bazaar by Erté (1922)

London Underground poster by Horace Taylor (1924)

Moulin Rouge poster by Charles Gesmar (1925)

Poster for Chicago World's Fair (1933)

The Art Deco style appeared early in the graphic arts, in the years just
before World War I. It appeared in Paris in the posters and the costume
designs of Leon Bakst for the Ballets Russes, and in the catalogs of the
fashion designers Paul Poiret. The illustrations of Georges Barbier, and
Georges Lepape and the images in the fashion magazine La Gazette du bon
ton perfectly captured the elegance and sensuality of the style. In the
1920s, the look changed; the fashions stressed were more casual,
sportive and daring, with the woman models usually smoking cigarettes.
American fashion magazines such as Vogue, Vanity Fair and Harper's
Bazaar quickly picked up the new style and popularized it in the United
States. It also influenced the work of American book illustrators such
as Rockwell Kent. In Germany, the most famous poster artist of the
period was Ludwig Hohlwein, who created colorful and dramatic posters
for music festivals, beers, and, late in his career, for the Nazi Party.

During the Art Nouveau period, posters usually advertised theatrical
products or cabarets. In the 1920s, travel posters, made for steamship
lines and airlines, became extremely popular. The style changed notably
in the 1920s, to focus attention on the product being advertised. The
images became simpler, precise, more linear, more dynamic, and were
often placed against a single color background. In France popular Art
Deco designers included, Charles Loupot and Paul Colin, who became
famous for his posters of American singer and dancer Josephine Baker.
Jean Carlu designed posters for Charlie Chaplin movies, soaps, and
theaters; in the late 1930s he emigrated to the United States, where,
during the World War, he designed posters to encourage war production.
The designer Charles Gesmar became famous making posters for the singer
Mistinguett and for Air France. Among the best known French Art Deco
poster designers was Cassandre, who made the celebrated poster of the
ocean liner SS Normandie in 1935.

In the 1930s a new genre of posters appeared in the United States during
the Great Depression. The Federal Art Project hired American artists to
create posters to promote tourism and cultural events.

\section{Architecture}\label{architecture}

\begin{itemize}
\item
  \emph{The Art Deco style was not limited to buildings on land; the
  ocean liner SS Normandie, whose first voyage was in 1935, featured Art
  Deco design, including a dining room whose ceiling and decoration were
  made of glass by Lalique.}
\item
  \emph{Between 1925 and 1928 he constructed the new art deco facade of
  the La Samaritaine department store in Paris.}
\end{itemize}

La Samaritaine department store, by Henri Sauvage, Paris, (1925--28)

Los Angeles City Hall by John Parkinson, John C. Austin, and Albert C.
Martin, Sr. (1928)

Interior of the Palacio de Bellas Artes (Palace of Fine Arts) in Mexico
City (1934)

National Diet Building in Tokyo, Japan (1936)

Mayakovskaya Metro Station in Moscow (1936)

The architectural style of art deco made its debut in Paris in 1903--04,
with the construction of two apartment buildings in Paris, one by
Auguste Perret on rue Trétaigne and the other on rue Benjamin Franklin
by Henri Sauvage. The two young architects used reinforced concrete for
the first time in Paris residential buildings; the new buildings had
clean lines, rectangular forms, and no decoration on the facades; they
marked a clean break with the art nouveau style. Between 1910 and 1913,
Perret used his experience in concrete apartment buildings to construct
the Théâtre des Champs-Élysées, 15 avenue Montaigne. Between 1925 and
1928 he constructed the new art deco facade of the La Samaritaine
department store in Paris.

After the First World War, art deco buildings of steel and reinforced
concrete began to appear in large cities across Europe and the United
States. In the United States the style was most commonly used for office
buildings, government buildings, movie theaters, and railroad stations.
It sometimes was combined with other styles; Los Angeles City Hall
combined Art Deco with a roof based on the ancient Greek Mausoleum at
Halicarnassus, while the Los Angeles railroad station combined Deco with
Spanish mission architecture. Art Deco elements also appeared in
engineering projects, including the towers of the Golden Gate Bridge and
the intake towers of Hoover Dam. In the 1920s and 1930s it became a
truly international style, with examples including the Palacio de Bellas
Artes (Palace of Fine Arts) in Mexico City by Federico
Mariscal~{[}es{]}, the Mayakovskaya Metro Station in Moscow and the
National Diet Building in Tokyo by Watanabe Fukuzo.{[}citation needed{]}

The Art Deco style was not limited to buildings on land; the ocean liner
SS Normandie, whose first voyage was in 1935, featured Art Deco design,
including a dining room whose ceiling and decoration were made of glass
by Lalique.

\section{"Cathedrals of Commerce"}\label{cathedrals-of-commerce}

\begin{itemize}
\item
  \emph{The exterior facade was entirely covered with sculpture, and the
  lobby created an Art Deco harmony with a wood parquet floor in a
  geometric pattern, a mural depicting the people of French colonies;
  and a harmonious composition of vertical doors and horizontal
  balconies.}
\item
  \emph{In France, the best example of an Art Deco interior during
  period was the Palais de la Porte Dorée (1931) by Albert Laprade, Léon
  Jaussely and Léon Bazin.}
\item
  \emph{The grand showcases of Art deco interior design were the lobbies
  of government buildings, theaters, and particularly office buildings.}
\end{itemize}

The Fisher Building in Detroit by Joseph Nathaniel French (1928)

Lower lobby of the Guardian Building in Detroit by Wirt Rowland (1929)

Lobby of 450 Sutter Street in San Francisco by Timothy Pflueger (1929)

Lobby of the Chrysler Building by William Van Alen in New York City
(1930)

Elevator of the Chrysler Building (1930)

The grand showcases of Art deco interior design were the lobbies of
government buildings, theaters, and particularly office buildings.
Interiors were extremely colorful and dynamic, combining sculpture,
murals, and ornate geometric design in marble, glass, ceramics and
stainless steel. An early example was the Fisher Building in Detroit, by
Joseph Nathaniel French; the lobby was highly decorated with sculpture
and ceramics. The Guardian Building (originally the Union Trust
Building) in Detroit, by Wirt Rowland (1929), decorated with red and
black marble and brightly colored ceramics, highlighted by highly
polished steel elevator doors and counters. The sculptural decoration
installed in the walls illustrated the virtues of industry and saving;
the building was immediately termed the "Cathedral of Commerce". The
Medical and Dental Building called 450 Sutter Street in San Francisco by
Timothy Pflueger was inspired by Mayan architecture, in a highly
stylized form; it used pyramid shapes, and the interior walls were
covered highly stylized rows of hieroglyphs.

In France, the best example of an Art Deco interior during period was
the Palais de la Porte Dorée (1931) by Albert Laprade, Léon Jaussely and
Léon Bazin. The building (now the National Museum of Immigration, with
an aquarium in the basement) was built for the Paris Colonial Exposition
of 1931, to celebrate the people and products of French colonies. The
exterior facade was entirely covered with sculpture, and the lobby
created an Art Deco harmony with a wood parquet floor in a geometric
pattern, a mural depicting the people of French colonies; and a
harmonious composition of vertical doors and horizontal balconies.

\section{Movie palaces}\label{movie-palaces}

\begin{itemize}
\item
  \emph{Movie palaces in the 1920s often combined exotic themes with art
  deco style; Grauman's Egyptian Theater in Hollywood (1922) was
  inspired by ancient Egyptian tombs and pyramids, while the Fox Theater
  in Bakersfield, California attached a tower in California Mission
  style to an Art Deco hall.}
\item
  \emph{Many of the best surviving examples of Art Deco are movie
  theaters built in the 1920s and 1930s.}
\end{itemize}

Grauman's Egyptian Theater in Hollywood (1922)

Grand Rex movie theater in Paris (1932)

Four-story high grand lobby of the Paramount Theater, Oakland (1932)

Auditorium and stage of Radio City Music Hall, New York City (1932)

Gaumont State Cinema in London (1937)

The Paramount in Shanghai, China (1933)

Many of the best surviving examples of Art Deco are movie theaters built
in the 1920s and 1930s. The Art Deco period coincided with the
conversion of silent films to sound, and movie companies built enormous
theaters in major cities to capture the huge audience that came to see
movies. Movie palaces in the 1920s often combined exotic themes with art
deco style; Grauman's Egyptian Theater in Hollywood (1922) was inspired
by ancient Egyptian tombs and pyramids, while the Fox Theater in
Bakersfield, California attached a tower in California Mission style to
an Art Deco hall. The largest of all is Radio City Music Hall in New
York City, which opened in 1932. Originally designed as a stage theater,
it quickly transformed into a movie theater, which could seat 6,015
persons The interior design by Donald Deskey used glass, aluminum,
chrome, and leather to create a colorful escape from reality The
Paramount Theater in Oakland, California, by Timothy Pflueger, had a
colorful ceramic facade a lobby four stories high, and separate Art Deco
smoking rooms for gentlemen and ladies. Similar grand palaces appeared
in Europe. The Grand Rex in Paris (1932), with its imposing tower, was
the largest movie theater in Europe. The Gaumont State Cinema in London
(1937) had a tower modeled after the Empire State building, covered with
cream-colored ceramic tiles and an interior in an Art Deco-Italian
Renaissance style. The Paramount Theater in Shanghai, China (1933) was
originally built as a dance hall called The gate of 100 pleasures; it
was converted to a movie theater after the Communist Revolution in 1949,
and now is a ballroom and disco. In the 1930s Italian architects built a
small movie palace, the Cinema Impero, in Asmara in what is now Eritrea.
Today, many of the movie theaters have been subdivided into multiplexes,
but others have been restored and are used as cultural centers in their
communities.

\section{Streamline Moderne}\label{streamline-moderne}

\begin{itemize}
\item
  \emph{Paris Building in the Paquebot or ocean liner style, 3 boulevard
  Victor (1935), by Pierre Patout}
\item
  \emph{In the late 1930s, a new variety of Art Deco architecture became
  common; it was called Streamline Moderne or simply Streamline, or, in
  France, the Style Paqueboat, or Ocean Liner style.}
\end{itemize}

Paris Building in the Paquebot or ocean liner style, 3 boulevard Victor
(1935), by Pierre Patout

Pan-Pacific Auditorium in Los Angeles (1936)

The Marine Air Terminal at La Guardia Airport (1937) was the New York
terminal for the flights of Pan Am Clipper flying boats to Europe

The Hoover Building canteen in Perivale in the London suburbs, by
Wallis, Gilbert and Partners (1938)

The Ford Pavilion at the 1939 New York World's Fair

The nautical-style rounded corner of BBC Broadcasting House (1931)

Streamline Moderne church, First Church of Deliverance, Chicago, IL
(1939), by Walter T. Bailey. Towers added 1948.

In the late 1930s, a new variety of Art Deco architecture became common;
it was called Streamline Moderne or simply Streamline, or, in France,
the Style Paqueboat, or Ocean Liner style. Buildings in the style were
had rounded corners, long horizontal lines; they were built of
reinforced concrete, and were almost always white; and sometimes had
nautical features, such as railings that resembled those on a ship. The
rounded corner was not entirely new; it had appeared in Berlin in 1923
in the Mossehaus by Erich Mendelsohn, and later in the Hoover Building,
an industrial complex in the London suburb of Perivale. In the United
States, it became most closely associated with transport; Streamline
moderne was rare in office buildings, but was often used for bus
stations and airport terminals, such as terminal at La Guardia airport
in New York City that handled the first transatlantic flights, via the
PanAm clipper flying boats; and in roadside architecture, such as gas
stations and diners. In the late 1930s a series of diners, modeled after
streamlined railroad cars, were produced and installed in towns in New
England; at least two examples still remain and are now registered
historic buildings.

\section{Decoration and motifs}\label{decoration-and-motifs}

\begin{itemize}
\item
  \emph{Decoration in the Art Deco period went through several distinct
  phases.}
\item
  \emph{Throughout the Art Deco period, and particularly in the 1930s,
  the motifs of the decor expressed the function of the building.}
\item
  \emph{A second tendency of Art Deco, also from 1910 to 1920, was
  inspired by the bright colors of the artistic movement known as the
  Fauves and by the colorful costumes and sets of the Ballets Russes.}
\end{itemize}

Iron fireplace screen, Rose Iron Works, Cleveland (1930)

Elevator doors of the Chrysler Building, by William Van Alen (1927--30)

Sunrise motif from the Wisconsin Gas Building (1930)

Detail of mosaic facade of Paramount Theater (Oakland, California)
(1931)

Decoration in the Art Deco period went through several distinct phases.
Between 1910 and 1920, as Art Nouveau was exhausted, design styles saw a
return to tradition, particularly in the work of Paul Iribe. In 1912
André Vera published an essay in the magazine L'Art Décoratif calling
for a return to the craftsmanship and materials of earlier centuries,
and using a new repertoire of forms taken from nature, particularly
baskets and garlands of fruit and flowers. A second tendency of Art
Deco, also from 1910 to 1920, was inspired by the bright colors of the
artistic movement known as the Fauves and by the colorful costumes and
sets of the Ballets Russes. This style was often expressed with exotic
materials such as sharkskin, mother of pearl, ivory, tinted leather,
lacquered and painted wood, and decorative inlays on furniture that
emphasized its geometry. This period of the style reached its high point
in the 1925 Paris Exposition of Decorative Arts. In the late 1920s and
the 1930s, the decorative style changed, inspired by new materials and
technologies. It became sleeker and less ornamental. Furniture, like
architecture, began to have rounded edges and to take on a polished,
streamlined look, taken from the streamline modern style. New materials,
such as chrome-plated steel, aluminum and bakelite, an early form of
plastic, began to appear in furniture and decoration.

Throughout the Art Deco period, and particularly in the 1930s, the
motifs of the decor expressed the function of the building. Theaters
were decorated with sculpture which illustrated music, dance, and
excitement; power companies showed sunrises, the Chrysler building
showed stylized hood ornaments; The friezes of Palais de la Porte Dorée
at the 1931 Paris Colonial Exposition showed the faces of the different
nationalities of French colonies. The Streamline style made it appear
that the building itself was in motion. The WPA murals of the 1930s
featured ordinary people; factory workers, postal workers, families and
farmers, in place of classical heroes.

\section{Furniture}\label{furniture}

\begin{itemize}
\item
  \emph{He introduced the style of lacquered art deco furniture at the
  end of in the late 1920s, and in the late 1930s introduced furniture
  made of metal with panels of smoked glass.}
\item
  \emph{An Art Deco club chair (1930s)}
\item
  \emph{Late Art Deco furniture and rug by Jules Leleu (1930s)}
\item
  \emph{The Waterfall style was popular the 1930s and 1940s, the most
  prevalent Art Deco form of furniture at the time.}
\end{itemize}

Chair by Paul Follot (1912--14)

Armchair by Louis Süe (1912) and painted screen by André Mare (1920)

Dressing table and chair of marble and encrusted, lacquered, and gilded
wood by Paul Follot (1919-1920

Corner cabinet of Mahogany with rose basket design of inlaid ivory by
Émile-Jacques Ruhlmann (1923)

Cabinet by Émile-Jacques Ruhlmann (1926)

Cabinet design by Émile-Jacques Ruhlmann

Cabinet covered with shagreen or sharkskin, by André Groult (1925)

Furniture by Gio Ponti (1927)

Desk of an administrator, by Michel Roux-Spitz for the 1930 Salon of
Decorative Artists

An Art Deco club chair (1930s)

Late Art Deco furniture and rug by Jules Leleu (1930s)

A Waterfall style buffet table

French furniture from 1910 until the early 1920s was largely an updating
of French traditional furniture styles, and the art nouveau designs of
Louis Majorelle, Charles Plumet and other manufacturers. French
furniture manufacturers felt threatened by the growing popularity of
German manufacturers and styles, particularly the Biedermeier style,
which was simple and clean-lined. The French designer Frantz Jourdain,
the President of the Paris Salon d'Automne, invited designers from
Munich to participate in the 1910 Salon. French designers saw the new
German style, and decided to meet the German challenge. The French
designers decided to present new French styles in the Salon of 1912. The
rules of the Salon indicated that only modern styles would be permitted.
All of the major French furniture designers took part in Salon: Paul
Follot, Paul Iribe, Maurice Dufrene, André Groult, André Mare and Louis
Suë took part, presenting new works that updated the traditional French
styles of Louis XVI and Louis Philippe with more angular corners
inspired by Cubism and brighter colors inspired by Fauvism and the
Nabis.

The painter André Mare and furniture designer Louis Suë both
participated the 1912 Salon. After the war the two men joined together
to form their own company, formally called the Compagnie des Arts
Française, but usually known simply as Suë and Mare. Unlike the
prominent art nouveau designers like Louis Majorelle, who personally
designed every piece, they assembled a team of skilled craftsmen and
produced complete interior designs, including furniture, glassware,
carpets, ceramics, wallpaper and lighting. Their work featured bright
colors and furniture and fine woods, such ebony encrusted with mother of
pearl, abalone and silvered metal to create bouquets of flowers. They
designed everything from the interiors of ocean liners to perfume
bottles for the label of Jean Patou.The firm prospered in the early
1920s, but the two men were better craftsmen than businessmen. The firm
was sold in 1928, and both men left.

The most prominent furniture designer at the 1925 Decorative Arts
Exposition was Émile-Jacques Ruhlmann, from Alsace. He first exhibited
his works at the 1913 Autumn Salon, then had his own pavilion, the
"House of the Rich Collector", at the 1925 Exposition. He used only most
rare and expensive materials, including ebony, mahogany, rosewood, ambon
and other exotic woods, decorated with inlays of ivory, tortoise shell,
mother of pearl, Little pompoms of silk decorated the handles of drawers
of the cabinets. His furniture was based upon 18th-century models, but
simplified and reshaped. In all of his work, the interior structure of
the furniture was completely concealed. The framework usually of oak,
was completely covered with an overlay of thin strips of wood, then
covered by a second layer of strips of rare and expensive woods. This
was then covered with a veneer and polished, so that the piece looked as
if it had been cut out of a single block of wood. Contrast to the dark
wood was provided by inlays of ivory, and ivory key plates and handles.
According to Ruhlmann, armchairs had to be designed differently
according to the functions of the rooms where they appeared; living room
armchairs were designed to be welcoming, office chairs comfortable, and
salon chairs voluptuous. Only a small number of pieces of each design of
furniture was made, and the average price of one of his beds or cabinets
was greater than the price of an average house.

Jules Leleu was a traditional furniture designer who moved smoothly into
Art Deco in the 1920s; he designed the furniture for the dining room of
the Élysée Palace, and for the first-class cabins of the steamship
Normandie. his style was characterized by the use of ebony, Macassar
wood, walnut, with decoration of plaques of ivory and mother of pearl.
He introduced the style of lacquered art deco furniture at the end of in
the late 1920s, and in the late 1930s introduced furniture made of metal
with panels of smoked glass. In Italy, the designer Gio Ponti was famous
for his streamlined designs.

The costly and exotic furniture of Ruhlmann and other traditionalists
infuriated modernists, including the architect Le Corbusier, causing him
to write a famous series of articles denouncing the arts décoratif
style. He attacked furniture made only for the rich, and called upon
designers to create furniture made with inexpensive materials and modern
style, which ordinary people could afford. He designed his own chairs,
created to be inexpensive and mass-produced.

In the 1930s, furniture designs adapted to the form, with smoother
surfaces and curved forms. The masters of the late style included Donald
Deskey was one of the most influential designers; he created the
interior of the Radio City Music Hall. He used a mixture of traditional
and very modern materials, including aluminum, chrome, and bakelite, an
early form of plastic. The Waterfall style was popular the 1930s and
1940s, the most prevalent Art Deco form of furniture at the time. Pieces
were typically of plywood finished with blond veneer and with rounded
edges, resembling a waterfall.

\section{Design}\label{design}

\begin{itemize}
\item
  \emph{Philips Art Deco radio set (1931)}
\item
  \emph{Streamline was a variety of Art Deco which emerged during the
  mid-1930s.}
\item
  \emph{The cabins and salons featured the latest Art Deco furnishings
  and decoration.}
\item
  \emph{Ocean liners also adopted a style of Art Deco, known in French
  as the Style Paquebot, or "Ocean Liner Style".}
\end{itemize}

Philips Art Deco radio set (1931)

Chrysler Airflow sedan, designed by Carl Breer (1934)

Grand dining room of the ocean liner SS Normandie (1935), by Pierre
Patout; bas-reliefs by Raymond Delamarre

Bugatti Aérolithe (1936)

Philco Table Radio (c. 1937)

Electrolux Vacuum cleaner (1937)

Streamlined railroad locomotive (1939)

Streamline was a variety of Art Deco which emerged during the mid-1930s.
It was influenced by modern aerodynamic principles developed for
aviation and ballistics to reduce aerodynamic drag at high velocities.
The bullet shapes were applied by designers to cars, trains, ships, and
even objects not intended to move, such as refrigerators, gas pumps, and
buildings. One of the first production vehicles in this style was the
Chrysler Airflow of 1933. It was unsuccessful commercially, but the
beauty and functionality of its design set a precedent; meant modernity.
It continued to be used in car design well after World War II.

New industrial materials began to influence design of cars and household
objects. These included aluminum, chrome, and bakelite, an early form of
plastic. Bakelite could be easily molded into different forms, and soon
was used in telephones, radios and other appliances.

Ocean liners also adopted a style of Art Deco, known in French as the
Style Paquebot, or "Ocean Liner Style". The most famous example was the
SS Normandie, which made its first transatlantic trip in 1935. It was
designed particularly to bring wealthy Americans to Paris to shop. The
cabins and salons featured the latest Art Deco furnishings and
decoration. The Grand Salon of the ship, which was the restaurant for
first-class passengers, was bigger than the Hall of Mirrors of the
Palace of Versailles. It was illuminated by electric lights within
twelve pillars of Lalique crystal; thirty-six matching pillars lined the
walls. This was one of the earliest examples of illumination being
directly integrated into architecture. The style of ships was soon
adapted to buildings. A notable example is found on the San Francisco
waterfront, where the Maritime Museum building, built as a public bath
in 1937, resembles a ferryboat, with ship railings and rounded corners.
The Star Ferry Terminal in Hong Kong also used a variation of the style.

\section{Textiles and fashion}\label{textiles-and-fashion}

\begin{itemize}
\item
  \emph{Art Deco forms appeared in the clothing of Paul Poiret, Charles
  Worth and Jean Patou.}
\item
  \emph{Textile design Abundance by André Mare, (1911), Metropolitan
  Museum of Art}
\item
  \emph{The floral carpet was reinvented in Deco style by Paul Poiret.}
\item
  \emph{Fashion changed dramatically during the Art Deco period, thanks
  in particular to designers Paul Poiret and later Coco Chanel.}
\end{itemize}

Textile design Abundance by André Mare, (1911), Metropolitan Museum of
Art

Rose Pattern Textiles designed by André Mare (c. 1919), Metropolitan
Museum of Art

Rose Mousse pattern for upholstery, cotton and silk (1920), Metropolitan
Museum of Art

Design of birds from Les Ateliers de Martine by Paul Iribe (1918)

Textiles were an important part of the Art Deco style, in the form of
colorful wallpaper, upholstery and carpets, In the 1920s, designers were
inspired by the stage sets of the Ballets Russes, fabric designs and
costumes from Léon Bakst and creations by the Wiener Werkstätte. The
early interior designs of André Mare featured brightly colored and
highly stylized garlands of roses and flowers, which decorated the
walls, floors, and furniture. Stylized Floral motifs also dominated the
work of Raoul Dufy and Paul Poiret, and in the furniture designs of J.E.
Ruhlmann. The floral carpet was reinvented in Deco style by Paul Poiret.

The use of the style was greatly enhanced by the introduction of the
pochoir stencil-based printing system, which allowed designers to
achieve crispness of lines and very vivid colors. Art Deco forms
appeared in the clothing of Paul Poiret, Charles Worth and Jean Patou.
After World War I, exports of clothing and fabrics became one of the
most important currency earners of France.

Late Art Deco wallpaper and textiles sometimes featured stylized
industrial scenes, cityscapes, locomotives and other modern themes, as
well as stylized female figures, metallic colors and geometric designs.

Fashion changed dramatically during the Art Deco period, thanks in
particular to designers Paul Poiret and later Coco Chanel. Poiret
introduced an important innovation to fashion design, the concept of
draping, a departure from the tailoring and pattern-making of the past.
He designed clothing cut along straight lines and constructed of
rectangular motifs. His styles offered structural simplicity The
corseted look and formal styles of the previous period were abandoned,
and fashion became more practical, and streamlined. with the use of new
materials, brighter colors and printed designs. The designer Coco Chanel
continued the transition, popularizing the style of sporty, casual chic.

Evening coat by Paul Poiret, c. 1912, silk and metal, Metropolitan
Museum of Art

Diving Venus Annette Kellerman in Los Angeles, California, c. 1920

Cécile Sorel, at the Comédie-Française, 1920

Desiree Lubovska in a dress by Jean Patou, c. 1921

Natacha Rambova in a dress designed by Paul Poiret, 1926

Coco Chanel in a sailor's blouse and trousers (1928)

\section{Jewelry}\label{jewelry}

\begin{itemize}
\item
  \emph{Art Deco bracelet of gold, coral and jade (1925) (Musée des Arts
  Décoratifs, Paris)}
\item
  \emph{The short haircuts of women in the twenties called for elaborate
  deco earring designs.}
\item
  \emph{They were joined by many young new designers, each with his own
  idea of deco.}
\end{itemize}

Art Deco bracelet of gold, coral and jade (1925) (Musée des Arts
Décoratifs, Paris)

René Lalique (1925--30), molded glass pendants on silk cords

Boucheron (1925), a gold buckle set with diamonds and carved onyx, lapis
lazuli, jade, and coral

Cartier, (1930), Mackay Emerald Necklace, emerald, diamond and platinum,
Smithsonian National Museum of Natural History, USA

In the 1920s and 1930s, designers including René Lalique and Cartier
tried to reduce the traditional dominance of diamonds by introducing
more colorful gemstones, such as small emeralds, rubies and sapphires.
They also placed greater emphasis on very elaborate and elegant
settings, featuring less-expensive materials such as enamel, glass, horn
and ivory. Diamonds themselves were cut in less traditional forms; the
1925 Exposition saw a large number of diamonds cut in the form of tiny
rods or matchsticks. The settings for diamonds also changed; More and
more often jewelers used platinum instead of gold, since it was strong
and flexible, and could set clusters of stones. Jewelers also began to
use more dark materials, such as enamels and black onyx, which provided
a higher contrast with diamonds.

Jewelry became much more colorful and varied in style. Cartier and the
firm of Boucheron combined diamonds with colorful other gemstones cut
into the form of leaves, fruit or flowers, to make brooches, rings,
earrings, clips and pendants. Far Eastern themes also became popular;
plaques of jade and coral were combined with platinum and diamonds, and
vanity cases, cigarette cases and powder boxes were decorated with
Japanese and Chinese landscapes made with mother of pearl, enamel and
lacquer.

Rapidly changing fashions in clothing brought new styles of jewelry.
Sleeveless dresses of the 1920s meant that arms needed decoration, and
designers quickly created bracelets of gold, silver and platinum
encrusted with lapis-lazuli, onyx, coral, and other colorful stones;
Other bracelets were intended for the upper arms, and several bracelets
were often worn at the same time. The short haircuts of women in the
twenties called for elaborate deco earring designs. As women began to
smoke in public, designers created very ornate cigarette cases and ivory
cigarette holders. The invention of the wrist-watch before World War I
inspired jewelers to create extraordinary decorated watches, encrusted
with diamonds and plated with enamel, gold and silver. Pendant watches,
hanging from a ribbon, also became fashionable.

The established jewelry houses of Paris in the period, Cartier, Chaumet,
Georges Fouquet, Mauboussin, and Van Cleef \& Arpels all created jewelry
and objects in the new fashion. The firm of Chaumet made highly
geometric cigarette boxes, cigarette lighters, pillboxes and notebooks,
made of hard stones decorated with jade, lapis lazuli, diamonds and
sapphires. They were joined by many young new designers, each with his
own idea of deco. Raymond Templier designed pieces with highly intricate
geometric patterns, including silver earrings that looked like
skyscrapers. Gerard Sandoz was only 18 when he started to design jewelry
in 1921; he designed many celebrated pieces based on the smooth and
polished look of modern machinery. The glass designer René Lalique also
entered the field, creating pendants of fruit, flowers, frogs, fairies
or mermaids made of sculpted glass in bright colors, hanging on cords of
silk with tassels. The jeweler Paul Brandt contrasted rectangular and
triangular patterns, and embedded pearls in lines on onyx plaques. Jean
Despres made necklaces of contrasting colors by bringing together silver
and black lacquer, or gold with lapis lazuli. Many of his designs looked
like highly polished pieces of machines. Jean Dunand was also inspired
by modern machinery, combined with bright reds and blacks contrasting
with polished metal.

\section{Glass art}\label{glass-art}

\begin{itemize}
\item
  \emph{Like the Art Nouveau period before it, Art Deco was an
  exceptional period for fine glass and other decorative objects,
  designed to fit their architectural surroundings.}
\item
  \emph{Louis Majorelle, famous for his Art Nouveau furniture, designed
  a remarkable Art Deco stained glass window portraying steel workers
  for the offices of the Aciéries de Longwy, a steel mill in Longwy,
  France.}
\end{itemize}

The Firebird by René Lalique (1922)

Parrot vase by René Lalique (1922)

Hood ornament Victoire by René Lalique (1928)

A Daum vase with sculpted grapes (1925)

Window for a steel mill office by Louis Majorelle (1928)

Daum vase (1930--35)

Like the Art Nouveau period before it, Art Deco was an exceptional
period for fine glass and other decorative objects, designed to fit
their architectural surroundings. The most famous producer of glass
objects was René Lalique, whose works, from vases to hood ornaments for
automobiles, became symbols of the period. He had made ventures into
glass before World War I, designing bottles for the perfumes of François
Coty, but he did not begin serious production of art glass until after
World War I. In 1918, at the age of 58, he bought a large glass works in
Combs-la-Ville and began to manufacture both artistic and practical
glass objects. He treated glass as a form of sculpture, and created
statuettes, vases, bowls, lamps and ornaments. He used demi-crystal
rather than lead crystal, which was softer and easier to form, though
not as lustrous. He sometimes used colored glass, but more often used
opalescent glass, where part or the whole of the outer surface was
stained with a wash. Lalique provided the decorative glass panels,
lights and illuminated glass ceilings for the ocean liners SS Ile de
France in 1927 and the SS Normandie in 1935, and for some of the
first-class sleeping cars of the French railroads. At the 1925
Exposition of Decorative Arts, he had his own pavilion, designed a
dining room with a table settling and matching glass ceiling for the
Sèvres Pavilion, and designed a glass fountain for the courtyard of the
Cours des Métier, a slender glass column which spouted water from the
sides and was illuminated at night.

Other notable Art Deco glass manufacturers included Marius-Ernest
Sabino, who specialized in figurines, vases, bowls, and glass sculptures
of fish, nudes, and animals. For these he often used an opalescent glass
which could change from white to blue to amber, depending upon the
light. His vases and bowls featured molded friezes of animals, nudes or
busts of women with fruit or flowers. His work was less subtle but more
colorful than that of Lalique.

Other notable Deco glass designers included Edmond Etling, who also used
bright opalescent colors, often with geometric patterns and sculpted
nudes; Albert Simonet, and Aristide Colotte and Maurice Marinot, who was
known for his deeply etched sculptural bottles and vases. The firm of
Daum from the city of Nancy, which had been famous for its Art Nouveau
glass, produced a line of Deco vases and glass sculpture, solid,
geometric and chunky in form. More delicate multicolored works were made
by Gabriel Argy-Rousseau, who produced delicately colored vases with
sculpted butterflies and nymphs, and Francois Decorchemont, whose vases
were streaked and marbled.

The Great Depression ruined a large part of the decorative glass
industry, which depended upon wealthy clients. Some artists turned to
designing stained glass windows for churches. In 1937, the Steuben glass
company began the practice of commissioning famous artists to produce
glassware. Louis Majorelle, famous for his Art Nouveau furniture,
designed a remarkable Art Deco stained glass window portraying steel
workers for the offices of the Aciéries de Longwy, a steel mill in
Longwy, France.

\section{Metal art}\label{metal-art}

\begin{itemize}
\item
  \emph{Art Deco artists produced a wide variety of practical objects in
  the Art Deco style, made of industrial materials from traditional
  wrought iron to chrome-plated steel.}
\end{itemize}

A grill with two wings called The Pheasants, made by Paul Kiss and
displayed at the 1925 Exposition of Decorative and Industrial Arts

Iron and copper grill called Oasis by Edgar Brandt, displayed at the
1925 Paris Exposition

Cocktail set of chrome-plated steel by Norman Bel Geddes (1937)

Art Deco artists produced a wide variety of practical objects in the Art
Deco style, made of industrial materials from traditional wrought iron
to chrome-plated steel. The American artist Norman Bel Geddes designed a
cocktail set resembling a skyscraper made of chrome-plated steel.
Raymond Subes designed an elegant metal grille for the entrance of the
Palais de la Porte Dorée, the centerpiece of the 1931 Paris Colonial
Exposition. The French sculptor Jean Dunand produced magnificent doors
on the theme "The Hunt", covered with gold leaf and paint on plaster
(1935).

\section{Animation}\label{animation}

\begin{itemize}
\item
  \emph{Art Deco visuals and imagery was used in animated films
  including Batman, Night Hood, All's Fair at the Fair, Merry
  Mannequins, Page Miss Glory, Fantasia and Sleeping Beauty.}
\end{itemize}

Art Deco visuals and imagery was used in animated films including
Batman, Night Hood, All's Fair at the Fair, Merry Mannequins, Page Miss
Glory, Fantasia and Sleeping Beauty.

\section{Art Deco architecture around the
world}\label{art-deco-architecture-around-the-world}

\begin{itemize}
\item
  \emph{(For a comprehensive of existing buildings by country, see List
  of Art Deco architecture.)}
\item
  \emph{Art Deco architecture began in Europe, but by 1939 there were
  examples in large cities on every continent and in almost every
  country.}
\end{itemize}

Art Deco architecture began in Europe, but by 1939 there were examples
in large cities on every continent and in almost every country. This is
a selection of prominent buildings on each continent. (For a
comprehensive of existing buildings by country, see List of Art Deco
architecture.)

\section{Africa}\label{africa}

\begin{itemize}
\item
  \emph{Most Art Deco buildings in Africa were built during European
  colonial rule, and often designed by Italian and French architects.}
\end{itemize}

Fiat Tagliero Building in Asmara, Eritrea by Giuseppe Pettazzi
(1938){[}116{]}

St. Peter's Cathedral in Rabat, Morocco (1938)

Most Art Deco buildings in Africa were built during European colonial
rule, and often designed by Italian and French architects.

\section{Asia}\label{asia}

\begin{itemize}
\item
  \emph{A large number of the Art Deco buildings in Asia were designed
  by European architects, but in the Philippines local architect Juan
  Nakpil was preeminent.}
\item
  \emph{Many Art Deco landmarks in Asia were demolished during the great
  economic expansion of Asia the late 20th century, but some notable
  enclaves of the architecture still remain, particularly in Shanghai
  and Mumbai.}
\end{itemize}

New India Assurance Building in Mumbai, India (1936)

Broadway Mansions in Shanghai, China (1934)

National Diet Building in Tokyo, Japan (1936)

Kologdam Building in Bandung, Indonesia (1920)

Ankara railway station in Ankara, Turkey (1937)

Dare House in Chennai, India (1940)

A large number of the Art Deco buildings in Asia were designed by
European architects, but in the Philippines local architect Juan Nakpil
was preeminent. Many Art Deco landmarks in Asia were demolished during
the great economic expansion of Asia the late 20th century, but some
notable enclaves of the architecture still remain, particularly in
Shanghai and Mumbai.

\section{Australia and New Zealand}\label{australia-and-new-zealand}

\begin{itemize}
\item
  \emph{Wellington has retained a sizeable number of Art Deco
  buildings.}
\item
  \emph{Several towns in New Zealand, including Napier and Hastings were
  rebuilt in Art Deco style after the 1931 Hawke's Bay earthquake, and
  many of the buildings have been protected and restored.}
\item
  \emph{Melbourne and Sydney Australia have several notable Art Deco
  buildings, including the Manchester Unity Building and the former
  Russell Street Police Headquarters in Melbourne, the Castlemaine Art
  Museum in Castlemaine, central Victoria and the Grace Building, AWA
  Tower and ANZAC War Memorial in Sydney.}
\end{itemize}

Manchester Unity Building in Melbourne (1932)

Sound Shell (1931) in Napier, New Zealand at night

ANZAC War Memorial in Sydney (1934)

Melbourne and Sydney Australia have several notable Art Deco buildings,
including the Manchester Unity Building and the former Russell Street
Police Headquarters in Melbourne, the Castlemaine Art Museum in
Castlemaine, central Victoria and the Grace Building, AWA Tower and
ANZAC War Memorial in Sydney.

Several towns in New Zealand, including Napier and Hastings were rebuilt
in Art Deco style after the 1931 Hawke's Bay earthquake, and many of the
buildings have been protected and restored. Napier has been nominated
for UNESCO World Heritage Site status, the first cultural site in New
Zealand to be nominated. Wellington has retained a sizeable number of
Art Deco buildings.

\section{Canada, Mexico, and the United
States}\label{canada-mexico-and-the-united-states}

\begin{itemize}
\item
  \emph{In the United States, Art Deco buildings are found from coast to
  coast, in all the major cities.}
\item
  \emph{Examples of Art Deco residential architecture can be found in
  the Condesa neighborhood, many designed by Francisco J. Serrano.}
\item
  \emph{In Canada, surviving Art Deco structures are mainly in the major
  cities; Montreal, Toronto, Hamilton, Ontario, and Vancouver.}
\end{itemize}

The Price Building in Quebec City, Canada (1930)

Vancouver City Hall in Vancouver, British Columbia, Canada (1935)

La Nacional Buildings in Mexico City, México

Interior of the Palacio de Bellas Artes in Mexico City, Mexico (1934)

The Verizon Building in New York City, US (1923--27)

Buffalo City Hall in Buffalo, US (1931)

Bullocks Wilshire in Los Angeles, US (1929)

Louisiana State Capitol in Baton Rouge, US (1930--32)

Jefferson County Courthouse in Beaumont, US (1931)

In Canada, surviving Art Deco structures are mainly in the major cities;
Montreal, Toronto, Hamilton, Ontario, and Vancouver. They range from
public buildings like Vancouver City Hall to commercial buildings
(College Park) to public works (R. C. Harris Water Treatment Plant).\\
File:Edificios La Nacional I y II.JPG \textbar{} La Nacional Buildings
Mexico City, Mexico (1932)\\
In Mexico, the most imposing Art Deco example is interior of the Palacio
de Bellas Artes (Palace of Fine Arts), finished in 1934 with its
elaborate decor and murals. Examples of Art Deco residential
architecture can be found in the Condesa neighborhood, many designed by
Francisco J. Serrano.

In the United States, Art Deco buildings are found from coast to coast,
in all the major cities. It was most widely used for office buildings,
train stations, airport terminals, and movie theaters; residential
buildings are rare. In the 1930s, the more austere streamline style
became popular. Many buildings were demolished between 1945 and the late
1960s, but then efforts began to protect the best examples. The City of
Miami Beach established the Miami Beach Architectural District to
preserve the colorful collection of Art Deco buildings found there.

\section{Central America and the
Caribbean}\label{central-america-and-the-caribbean}

\begin{itemize}
\item
  \emph{Havana art deco building}
\item
  \emph{Art Deco buildings can be found throughout Central America.}
\end{itemize}

The Bacardi Building in Havana, Cuba (1930)

The Hotel Nacional de Cuba in Havana, Cuba (1930)

Havana art deco building

The Plaza del Mercado de Ponce in Ponce, Puerto Rico (1941)

Art Deco buildings can be found throughout Central America. A
particularly rich

collection is found in Cuba, built largely for the large number of
tourists who came to the island from the United States. One such
building is the López Serrano built between 1929 and 1932 in the Vedado
section of Havana.

\section{Europe}\label{europe}

\begin{itemize}
\item
  \emph{Spain and Portugal have some striking examples of Art Deco
  buildings, particularly movie theaters.}
\item
  \emph{The Mossehaus with Art Deco elements by Erich Mendelsohn in
  Berlin, Germany (c.1923)}
\item
  \emph{In Germany two variations of Art Deco flourished in the 1920s
  and 30s: The Neue Sachlichkeit style and Expressionist architecture.}
\end{itemize}

Théâtre des Champs-Élysées in Paris, France (1910--13)

The Mossehaus with Art Deco elements by Erich Mendelsohn in Berlin,
Germany (c.1923)

Basilica of the Sacred Heart in Brussels, Belgium (1925)

Éden Theater in Lisbon, Portugal (1931)

Palais de Tokyo, Musée d'Art Moderne de la Ville de Paris, France (1937)

Mayakovskaya Station in Moscow, Russia (1938)

Rivoli Theater in Porto, Portugal (1937)

Daily Express Building in Manchester, UK (1939)

The architectural style first appeared in Paris with the Théâtre des
Champs-Élysées (1910--13) by Auguste Perret but then spread rapidly
around Europe, until examples could be found in nearly every large city,
from London to Moscow. In Germany two variations of Art Deco flourished
in the 1920s and 30s: The Neue Sachlichkeit style and Expressionist
architecture. Notable examples include Erich Mendelsohn's Mossehaus and
Schaubühne theater in Berlin, Fritz Höger's Chilehaus in Hamburg and his
Kirche am Hohenzollernplatz in Berlin, the Anzeiger Tower in Hannover
and the Borsig Tower in Berlin.

One of the largest Art Deco buildings in Western Europe is the Basilica
of the Sacred Heart in Koekelberg, Brussels. In 1925, architect Albert
van Huffel won the Grand Prize for Architecture with his scale model of
the basilica at the Exposition Internationale des Arts Décoratifs et
Industriels Modernes in Paris.

Spain and Portugal have some striking examples of Art Deco buildings,
particularly movie theaters. Examples in Portugal are the Capitólio
Theater (1931) and the Éden Cine-Theater (1937) in Lisbon, the Rivoli
Theater (1937) and the Coliseu (1941) in Porto and the Rosa Damasceno
Theater (1937) in Santarém. An example in Spain is the Cine Rialto in
Valencia (1939).

During the 1930s, Art Deco had a noticeable effect on house design in
the United Kingdom, as well as the design of various public buildings.
Straight, white-rendered house frontages rising to flat roofs, sharply
geometric door surrounds and tall windows, as well as convex-curved
metal corner windows, were all characteristic of that period.

The London Underground is famous for many examples of Art Deco
architecture, and there are a number of buildings in the style situated
along the Golden Mile in Brentford. Also in West London is the Hoover
Building, which was originally built for The Hoover Company and was
converted into a superstore in the early 1990s.

\section{India}\label{india}

\begin{itemize}
\item
  \emph{Guided by their desire to emulate the west, the Indian
  architects were fascinated by the industrial modernity that Art Deco
  offered.}
\item
  \emph{Parallelly, the changing political climate in the country and
  the aspirational quality of the Art Deco aesthetics led to a
  whole-hearted acceptance of the building style in the city's
  development.}
\end{itemize}

The Indian Institute of Architects, founded in Bombay in 1929, played a
prominent role in propagating the Art Deco movement. In November 1937,
this institute organized the `Ideal Home Exhibition' held in the Town
Hall in Bombay which spanned over 12 days and attracted about one
hundred thousand visitors. As a result, it was declared a success by the
'Journal of the Indian Institute of Architects'. The exhibits displayed
the `ideal', or better described as the most `modern' arrangements for
various parts of the house, paying close detail to avoid architectural
blunders and present the most efficient and well-thought-out models. The
exhibition focused on various elements of a home ranging from furniture,
elements of interior decoration as well as radios and refrigerators
using new and scientifically relevant materials and methods.\\
Guided by their desire to emulate the west, the Indian architects were
fascinated by the industrial modernity that Art Deco offered. The
western elites were the first to experiment with the technologically
advanced facets of Art Deco, and architects began the process of
transformation by the early 1930s.

Bombay's expanding port commerce in the 1930s resulted in the growth of
educated middle class population. It also saw an increase of people
migrating to Bombay in search of job opportunities. This led to the
pressing need for new developments through Land Reclamation Schemes and
construction of new public and residential buildings. Parallelly, the
changing political climate in the country and the aspirational quality
of the Art Deco aesthetics led to a whole-hearted acceptance of the
building style in the city's development. Most of the buildings from
this period can be seen spread throughout the city neighbourhoods in
areas such as Churchgate, Colaba, Fort, Mohammed Ali Road, Cumbala Hill,
Dadar, Matunga, Bandra and Chembur.

\section{South America}\label{south-america}

\begin{itemize}
\item
  \emph{The Art Deco in South America is present especially at the
  countries that received a great wave of immigration on the first half
  of the 20th century, with notable works at their richest cities, like
  São Paulo and Rio de Janeiro in Brazil and Buenos Aires in
  Argentina.The Kavanagh building in Buenos Aires (1934), by Sánchez,
  Lagos and de la Torre, was the tallest reinforced concrete structure
  when it was completed, and a notable example of late Art Deco style.}
\end{itemize}

Lacerda Elevator in Salvador, Bahia, Brazil (1930)

Altino Arantes Building, in São Paulo, Brazil (1947)

Viaduto do Chá, São Paulo, Brazil (1938)

Pacaembu Stadium, São Paulo (1940)

Central do Brasil Station in Rio de Janeiro, Brazil (1943)

Kavanagh building in Buenos Aires, Argentina (1934)

Palacio Municipal and fountain, Laprida, Buenos Aires, Argentina

The Abasto Market in Buenos Aires, circa 1945

The Art Deco in South America is present especially at the countries
that received a great wave of immigration on the first half of the 20th
century, with notable works at their richest cities, like São Paulo and
Rio de Janeiro in Brazil and Buenos Aires in Argentina.The Kavanagh
building in Buenos Aires (1934), by Sánchez, Lagos and de la Torre, was
the tallest reinforced concrete structure when it was completed, and a
notable example of late Art Deco style.

\section{Preservation and Neo Art
Deco}\label{preservation-and-neo-art-deco}

\begin{itemize}
\item
  \emph{In many cities, efforts have been made to protect the remaining
  Art Deco buildings.}
\item
  \emph{Art deco neighborhood in Havana, Cuba}
\item
  \emph{In the 21st century, modern variants of Art Deco, called Neo Art
  Deco (or Neo-Art Deco), have appeared in some American cities,
  inspired by the classic Art Deco buildings of the 1920s and 1930s.}
\end{itemize}

The Miami Beach Architectural District protects historic Art Deco
buildings

The U-Drop Inn, a roadside gas station and diner on U.S. Highway 66 in
Shamrock, Texas (1936), now a historic monument

Art deco neighborhood in Havana, Cuba

Smith Center for the Performing Arts in Las Vegas, Nevada, a Neo-Art
Deco building (2012))

In many cities, efforts have been made to protect the remaining Art Deco
buildings. In many U.S. cities, historic art deco movie theaters have
been preserved and turned into cultural centers. Even more modest art
deco buildings have been preserved as part of America's architectural
heritage; an art deco cafe and gas station along Route 66 in Shamrock,
Texas is an historic monument. The Miami Beach Architectural District
protects several hundred old buildings, and requires that new buildings
comply with the style. In Havana, Cuba, a large number of Art Deco
buildings have badly deteriorated. Efforts are underway to bring the
buildings back to their original color and appearance.

In the 21st century, modern variants of Art Deco, called Neo Art Deco
(or Neo-Art Deco), have appeared in some American cities, inspired by
the classic Art Deco buildings of the 1920s and 1930s. Examples include
the NBC Tower in Chicago, inspired by 30 Rockefeller Plaza in New York
City; and Smith Center for the Performing Arts in Las Vegas, Nevada,
which includes art deco features from Hoover Dam, fifty miles away.

\section{Gallery}\label{gallery}

\begin{itemize}
\item
  \emph{Niagara Mohawk Building, Syracuse, New York.}
\item
  \emph{Federal Art Project poster promoting milk drinking in Cleveland,
  1940}
\item
  \emph{Lobby, Empire State Building, New York City.}
\end{itemize}

"Skyscraper Lamp" designed by Arnaldo dell'Ira, 1929

Guardians of Traffic pylon on Hope Memorial Bridge in Cleveland (1932)

Municipal Auditorium of Kansas City, Missouri: Hoit Price \& Barnes, and
Gentry, Voskamp \& Neville, 1935

U.S. Works Progress Administration poster, John Wagner, artist, ca. 1940

"Beau Brownie" camera, Walter Dorwin Teague 1930 design for Eastman
Kodak

Former Teatro Eden, now Aparthotel Vip Eden in Lisbon, Portugal:
Cassiano Branco and Carlo Florencio Dias, 1931

Parker Duofold desk set, c.1930

1937 Cord automobile model 812, designed in 1935 by Gordon M. Buehrig
and staff

Palacio de Bellas Artes, Mexico City, Federico Mariscal, completed 1934

Women's Smoking Room at the Paramount Theatre, Oakland. Timothy L.
Pflueger, architect, 1931

U.S. postage stamp commemorating the 1939 New York World's Fair, 1939

Henryk Kuna, Rytm ("Rhythm"), in Skaryszewski Park, Warsaw, Poland, 1925

Disused Snowdon Theatre, Montreal, Quebec, Canada. Opened 1937, closed
1984. Daniel J. Crighton, architect

Union Terminal in Cincinnati, Ohio; Paul Philippe Cret, Alfred T.
Fellheimer, Steward Wagner, Roland Wank, 1933

Lobby, Empire State Building, New York City. William F. Lamb, opened
1931

Federal Art Project poster promoting milk drinking in Cleveland, 1940

Interior drawing, Eaton's College Street department store, Toronto,
Ontario, Canada

Niagara Mohawk Building, Syracuse, New York. Melvin L. King and Bley \&
Lyman, architects, completed 1932

\section{See also}\label{see-also}

\section{References}\label{references}

\section{Bibliography}\label{bibliography}

\section{External links}\label{external-links}

\begin{itemize}
\item
  \emph{Art Deco Montreal}
\item
  \emph{Art Deco Society of Washington}
\item
  \emph{Art Deco Shanghai}
\item
  \emph{Art Deco Mumbai}
\item
  \emph{Art Deco Museum in Moscow}
\item
  \emph{Art Deco Society of California}
\item
  \emph{Art Deco Miami Beach}
\end{itemize}

Art Deco Miami Beach

Art Deco Mumbai

Art Deco Montreal

Art Deco Society of Washington

Art Deco Society of California

Art Deco Rio de Janeiro

Art Deco Shanghai

Art Deco Museum in Moscow

Art Deco Society New York

Art Deco Society of Los Angeles
