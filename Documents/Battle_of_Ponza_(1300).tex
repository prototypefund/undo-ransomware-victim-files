\textbf{From Wikipedia, the free encyclopedia}

https://en.wikipedia.org/wiki/Battle\_of\_Ponza\_\%281300\%29\\
Licensed under CC BY-SA 3.0:\\
https://en.wikipedia.org/wiki/Wikipedia:Text\_of\_Creative\_Commons\_Attribution-ShareAlike\_3.0\_Unported\_License

\section{Battle of Ponza (1300)}\label{battle-of-ponza-1300}

\begin{itemize}
\item
  \emph{One Sicilian galley fled after capturing one of Lauria's
  galleys, and it was followed by six more.}
\item
  \emph{Lauria's 40 Angevin galleys were at Naples when 32 Sicilian
  galleys under d'Oria arrived and challenged him to come out.}
\item
  \emph{This allowed 12 Apulian galleys to arrive from the south and
  seven Genoese galleys to arrive also, and join Lauria's fleet, making
  59 galleys.}
\end{itemize}

The naval Battle of Ponza took place on 14 June 1300 near the islands of
Ponza and Zannone, in the Gulf of Gaeta (north-west of Naples), when a
galley fleet commanded by Roger of Lauria defeated an Aragonese-Sicilian
galley fleet commanded by Conrad d'Oria.

Lauria's 40 Angevin galleys were at Naples when 32 Sicilian galleys
under d'Oria arrived and challenged him to come out. For the first time
he refused, probably because of a lack of confidence in his Angevin
crews, and d'Oria ravaged some offshore islands. This allowed 12 Apulian
galleys to arrive from the south and seven Genoese galleys to arrive
also, and join Lauria's fleet, making 59 galleys. Now Lauria emerged and
found the Sicilians near Zannone Island to the west. After d'Oria
dismissed a suggestion to retreat, he tried a quick attack on Lauria's
flag-galley and the banner-carrying galleys, his own galley running
alongside Lauria's, "head to toe", and Lauria's crew suffered even
though his galley couldn't be boarded. One Sicilian galley fled after
capturing one of Lauria's galleys, and it was followed by six more. Five
Genoese Ghibelline galleys held back, awaiting fortune, and 18-29 (two
sources give 28, but this adds up to too many) Sicilian galleys were
captured, d'Oria's being the last to surrender (when Lauria threatened
to burn it).

\section{Ships involved}\label{ships-involved}

\section{Aragonese and Angevin}\label{aragonese-and-angevin}

\begin{itemize}
\item
  \emph{12 Apulian galleys}
\item
  \emph{40 Angevin (Regicolae) galleys}
\item
  \emph{7 Genoese (Grimaldi) galleys}
\end{itemize}

40 Angevin (Regicolae) galleys

12 Apulian galleys

7 Genoese (Grimaldi) galleys

\section{Sicilian}\label{sicilian}

\begin{itemize}
\item
  \emph{32 galleys - about 26 captured}
\end{itemize}

32 galleys - about 26 captured

Coordinates: 40°54′00″N 12°58′00″E / 40.9000°N 12.9667°E / 40.9000;
12.9667
