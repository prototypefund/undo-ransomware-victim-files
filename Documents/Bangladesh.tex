\textbf{From Wikipedia, the free encyclopedia}

https://en.wikipedia.org/wiki/Bangladesh\\
Licensed under CC BY-SA 3.0:\\
https://en.wikipedia.org/wiki/Wikipedia:Text\_of\_Creative\_Commons\_Attribution-ShareAlike\_3.0\_Unported\_License

\section{Bangladesh}\label{bangladesh}

\begin{itemize}
\item
  \emph{Bangladesh (/ˌbæŋɡləˈdɛʃ, ˌbɑːŋ-/; Bengali: বাংলাদেশ Bangladesh
  {[}ˈbaŋladeʃ{]} (listen), lit.}
\item
  \emph{"The country of Bengal"), officially the People's Republic of
  Bangladesh (গণপ্রজাতন্ত্রী বাংলাদেশ Gônoprojatontri Bangladesh), is a
  country in South Asia.}
\end{itemize}

Coordinates: 23°48′N 90°18′E / 23.8°N 90.3°E / 23.8; 90.3

Bangladesh (/ˌbæŋɡləˈdɛʃ, ˌbɑːŋ-/; Bengali: বাংলাদেশ Bangladesh
{[}ˈbaŋladeʃ{]} (listen), lit. "The country of Bengal"), officially the
People's Republic of Bangladesh (গণপ্রজাতন্ত্রী বাংলাদেশ Gônoprojatontri
Bangladesh), is a country in South Asia. It shares land borders with
India and Myanmar. The country's maritime territory in the Bay of Bengal
is roughly equal to the size of its land area. Bangladesh is the
92nd-largest sovereign state in the world, with an area of 147,570
square kilometres (56,980~sq~mi). It is also the world's 8th-most
populous country, as well as one of its most densely-populated. Dhaka is
its capital and largest city, and is also the economic, political and
the cultural center of Bangladesh, followed by Chittagong, which has the
country's largest port. It forms the largest and eastern part of the
Bengal region. The country's geography is dominated by the Bengal delta
with many rivers; while hilly and mountainous areas make up the
north-east and south-east. The country also has the longest sea beach
and the largest mangrove forest in the world. The endangered Bengal
tiger is a national symbol.

In the ancient and classical period of the Indian subcontinent, the
territory of Bangladesh was home to many principalities, including
Gangaridai, Vanga, Pundra, Gauda, Samatata and Harikela. It was also a
Mauryan province. The principalities were notable for their overseas
trade, which involved contacts with the Roman world, the export of fine
muslin and silk to the Middle East, and spreading philosophy and art to
Southeast Asia. The principalities dominated the Bengal delta with
powerful navies. The Pala Empire, the Chandra dynasty and the Sena
dynasty were the last pre-Islamic Bengali middle kingdoms.

Islam became the largest religion in Bengal during the period spanning
the Delhi Sultanate and the Bengal Sultanate. During the Mughal era,
described as the "Paradise of Nations", Bengal Subah generated 12\% of
the world's GDP, larger than the entirety of western Europe. The
province of eastern Bengal alone accounted for 40\% of Dutch imports
from Asia. The region was later administered by the United Kingdom as
part of the Bengal Presidency (1757--1905; 1912--1947) and Eastern
Bengal and Assam Province (1905--1912) in British India. During British
India notable personalities of Bengal Renaissance played a pivotal role
in the anti-colonial movement. Bengal had the largest GDP in the British
Raj. In 1947, the Bengal Legislative Council and the Bengal Legislative
Assembly voted on the Partition of Bengal, while a referendum caused the
Sylhet region to join East Bengal. The area became part of the Dominion
of Pakistan and was renamed East Pakistan. Beginning with the Bengali
Language Movement in 1952, the pro-democracy movement in East Pakistan
thrived on Bengali nationalism, resulting in the Bangladesh Liberation
War in 1971.

Bangladeshis include people from a range of ethnic groups and religions.
Bengalis, who speak the official Bengali language, make up 98\% of the
population. The politically dominant Bengali Muslims make the nation the
world's third largest Muslim-majority country. While recognising Islam
as the country's established religion, the constitution grants freedom
of religion to non-Muslims. A middle power, Bangladesh is a unitary
parliamentary democracy and constitutional republic in the Westminster
tradition. The country is divided into eight administrative divisions
and sixty-four districts. It is one of the emerging and growth-leading
economies of the world, and is listed among the Next Eleven countries.
It has one of the fastest real GDP growth rates in the world. Its gross
domestic product ranks 39th largest in the world in terms of market
exchange rates and 29th in purchasing power parity. Its per capita
income ranks 143th and 136th in two measures. In the field of human
development, it made substantial progress. The country continues to face
challenging problems, including poverty, corruption, terrorism,
illiteracy, and inadequate public healthcare. Bangladesh is a member of
the UN, the WTO, the Commonwealth of Nations, the IMF, the World Bank,
the ADB, the OIC, the IDB, the SAARC, the BIMSTEC and the IMCTC.

\section{Etymology}\label{etymology}

\begin{itemize}
\item
  \emph{Hence, the name Bangladesh means "Land of Bengal" or "Country of
  Bengal".}
\item
  \emph{The etymology of Bangladesh (Country of Bengal) can be traced to
  the early 20th century, when Bengali patriotic songs, such as Namo
  Namo Namo Bangladesh Momo by Kazi Nazrul Islam and Aaji Bangladesher
  Hridoy by Rabindranath Tagore, used the term.}
\end{itemize}

The etymology of Bangladesh (Country of Bengal) can be traced to the
early 20th century, when Bengali patriotic songs, such as Namo Namo Namo
Bangladesh Momo by Kazi Nazrul Islam and Aaji Bangladesher Hridoy by
Rabindranath Tagore, used the term. The term Bangladesh was often
written as two words, Bangla Desh, in the past. Starting in the 1950s,
Bengali nationalists used the term in political rallies in East
Pakistan. The term Bangla is a major name for both the Bengal region and
the Bengali language. The earliest known usage of the term is the Nesari
plate in 805~AD. The term Vangaladesa is found in 11th-century South
Indian records.

The term gained official status during the Sultanate of Bengal in the
14th century. Shamsuddin Ilyas Shah proclaimed himself as the first
"Shah of Bangala" in 1342. The word Bangla became the most common name
for the region during the Islamic period. The Portuguese referred to the
region as Bengala in the 16th century.

The origins of the term Bangla are unclear, with theories pointing to a
Bronze Age proto-Dravidian tribe, the Austric word "Bonga" (Sun god),
and the Iron Age Vanga Kingdom. The Indo-Aryan suffix Desh is derived
from the Sanskrit word deśha, which means "land" or "country". Hence,
the name Bangladesh means "Land of Bengal" or "Country of Bengal".

\section{History}\label{history}

\section{Early and medieval periods}\label{early-and-medieval-periods}

\begin{itemize}
\item
  \emph{The Mughal Empire controlled Bengal by the 17th century.}
\item
  \emph{Bengal was a major hub for international trade.}
\item
  \emph{The oldest inscription in Bangladesh was found in Mahasthangarh
  and dates from the 3rd century BCE.}
\item
  \emph{Stone Age tools found in Bangladesh indicate human habitation
  for over 20,000 years, and remnants of Copper Age settlements date
  back 4,000 years.}
\end{itemize}

Stone Age tools found in Bangladesh indicate human habitation for over
20,000 years, and remnants of Copper Age settlements date back 4,000
years. Ancient Bengal was settled by Austroasiatics, Tibeto-Burmans,
Dravidians and Indo-Aryans in consecutive waves of migration.
Archaeological evidence confirms that by the second millennium BCE,
rice-cultivating communities inhabited the region. By the 11th century
people lived in systemically-aligned housing, buried their dead, and
manufactured copper ornaments and black and red pottery. The Ganges,
Brahmaputra and Meghna rivers were natural arteries for communication
and transportation, and estuaries on the Bay of Bengal permitted
maritime trade. The early Iron Age saw the development of metal
weaponry, coinage, agriculture and irrigation. Major urban settlements
formed during the late Iron Age, in the mid-first millennium BCE, when
the Northern Black Polished Ware culture developed. In 1879, Alexander
Cunningham identified Mahasthangarh as the capital of the Pundra Kingdom
mentioned in the Rigveda. The oldest inscription in Bangladesh was found
in Mahasthangarh and dates from the 3rd century BCE. It is written in
the Brahmi script.

Greek and Roman records of the ancient Gangaridai Kingdom, which
(according to legend) deterred the invasion of Alexander the Great, are
linked to the fort city in Wari-Bateshwar. The site is also identified
with the prosperous trading center of Souanagoura listed on Ptolemy's
world map. Roman geographers noted a large seaport in southeastern
Bengal, corresponding to the present-day Chittagong region.

Ancient Buddhist and Hindu states which ruled Bangladesh included the
Vanga, Samatata and Pundra kingdoms, the Mauryan and Gupta Empires, the
Varman dynasty, Shashanka's kingdom, the Khadga and Candra dynasties,
the Pala Empire, the Sena dynasty, the Harikela kingdom and the Deva
dynasty. These states had well-developed currencies, banking, shipping,
architecture and art, and the ancient universities of Bikrampur and
Mainamati hosted scholars and students from other parts of Asia.
Xuanzang of China was a noted scholar who resided at the Somapura
Mahavihara (the largest monastery in ancient India), and Atisa traveled
from Bengal to Tibet to preach Buddhism. The earliest form of the
Bengali language began to the emerge during the eighth century.

Early Muslim explorers and missionaries arrived in Bengal late in the
first millennium CE. The Islamic conquest of Bengal began with the 1204
invasion by Bakhtiar Khilji; after annexing Bengal to the Delhi
Sultanate, Khilji waged a military campaign in Tibet. Bengal was ruled
by the Delhi Sultanate for a century by governors from the Mamluk,
Balban and Tughluq dynasties. During the 14th century, an independent
Bengal Sultanate was established by rebel governors. The sultanate's
ruling houses included the Ilyas Shahi, Jalaluddin Muhammad Shah,
Hussain Shahi, Suri and Karrani dynasties, and the era saw the
introduction of a distinct mosque architecture and the tangka currency.
The Arakan region was brought under Bengali hegemony. The Bengal
Sultanate was visited by explorers Ibn Battuta, Admiral Zheng He and
Niccolo De Conti. During the late 16th century, the Baro-Bhuyan (a
confederation of Muslim and Hindu aristocrats) ruled eastern Bengal; its
leader was the Mansad-e-Ala, a title held by Isa Khan and his son Musa
Khan. The Khan dynasty are considered local heroes for resisting North
Indian invasions with their river navies.

The Mughal Empire controlled Bengal by the 17th century. During the
reign of Emperor Akbar, the Bengali agrarian calendar was reformed to
facilitate tax collection. The Mughals established Dhaka as a fort city
and commercial metropolis, and it was the capital of Mughal Bengal for
75 years. In 1666, the Mughals expelled the Arakanese from Chittagong.
Mughal Bengal attracted foreign traders for its muslin and silk goods,
and the Armenians were a notable merchant community. A Portuguese
settlement in Chittagong flourished in the southeast, and a Dutch
settlement in Rajshahi existed in the north.

During the 18th century, the Nawabs of Bengal became the region's de
facto rulers. The title of the ruler is popularly known as the Nawab of
Bengal, Bihar and Orissa, given that the Bengali Nawab's realm
encompassed much of the eastern subcontinent. The Nawabs forged
alliances with European colonial companies, which made the region
relatively prosperous early in the century. Bengal accounted for 50\% of
the gross domestic product of the empire. By 1700, the Mughal economy
surpassed Qing China and Europe to become the world's largest and at the
time Bengal accounted for 12\% of the world's GDP, which was larger than
the share held by western Europe. The Bengali economy relied on textile
manufacturing, shipbuilding, saltpeter production, craftsmanship and
agricultural produce. Bengal was a major hub for international trade.
For example, 80\% of Dutch silk imports from Asia were from Bengal. Silk
and cotton textiles from Bengal were worn in Europe, Japan, Indonesia
and Central Asia. Annual Bengali shipbuilding output was 223,250 tons,
compared to an output of 23,061 tons in the nineteen colonies of North
America. Bengali shipbuilding proved to be advanced than European
shipbuilding prior to the Industrial Revolution. The flush deck of
Bengali rice ships was later replicated in European shipbuilding to
replace the stepped deck design for ship hulls.

The Bengali Muslim population was a product of conversion and religious
evolution, and their pre-Islamic beliefs included elements of Buddhism
and Hinduism. The construction of mosques, Islamic academies (madrasas)
and Sufi monasteries (khanqahs) facilitated conversion, and Islamic
cosmology played a significant role in developing Bengali Muslim
society. Scholars have theorized that Bengalis were attracted to Islam
by its egalitarian social order, which contrasted with the Hindu caste
system. By the 15th century, Muslim poets were writing in the Bengali
language. Notable medieval Bengali Muslim poets included Daulat Qazi,
Abdul Hakim and Alaol. Syncretic cults, such as the Baul movement,
emerged on the fringes of Bengali Muslim society. The Persianate culture
was significant in Bengal, where cities like Sonargaon became the
easternmost centers of Persian influence.

The Mughals had aided France during the Seven Years' War. They were
defeated by Great Britain during the Battle of Plassey. Although they
had lost control of Bengal Subah, Shah Alam II was involved in the
Bengal War which ended once more in their defeat at the Battle of Buxar.

\includegraphics[width=4.35312in,height=5.50000in]{media/image1.jpg}\\
\emph{Panam Road, Narayanganj during British rule in 1875}

\section{Colonial period}\label{colonial-period}

\begin{itemize}
\item
  \emph{The British established numerous schools, colleges and a
  university in what is now Bangladesh.}
\item
  \emph{The first railway in what is now Bangladesh began operating in
  1862.}
\item
  \emph{Several towns in Bangladesh participated in the Indian Mutiny
  and pledged allegiance to the last Mughal emperor, Bahadur Shah Zafar,
  who was later exiled to neighbouring Burma.}
\end{itemize}

After the 1757 Battle of Plassey, Bengal was the first region of the
Indian subcontinent conquered by the British East India Company. The
company formed the Presidency of Fort William, which administered the
region until 1858. A notable aspect of company rule was the Permanent
Settlement, which established the feudal zamindari system. A number of
famines, including the great Bengal famine of 1770, occurred under
company rule. Several rebellions broke out during the early 19th century
(including one led by Titumir), but British rule displaced the Muslim
ruling class. A conservative Islamic cleric, Haji Shariatullah, sought
to overthrow the British by propagating Islamic revivalism. Several
towns in Bangladesh participated in the Indian Mutiny and pledged
allegiance to the last Mughal emperor, Bahadur Shah Zafar, who was later
exiled to neighbouring Burma.

The challenge posed to company rule by the failed Indian Mutiny led to
the creation of the British Indian Empire as a crown colony. The British
established numerous schools, colleges and a university in what is now
Bangladesh. Syed Ahmed Khan and Ram Mohan Roy promoted modern and
liberal education in the subcontinent, inspiring the Aligarh movement
and the Bengal Renaissance. During the late 19th century, novelists,
social reformers and feminists emerged from Muslim Bengali society.
Electricity and municipal water systems were introduced in the 1890s;
cinemas opened in many towns during the early 20th century. East
Bengal's plantation economy was important to the British Empire,
particularly its jute and tea. The British established tax-free river
ports, such as the Port of Narayanganj, and large seaports like the Port
of Chittagong.

Bengal had the highest gross domestic product in British India. Bengal
was one of the first regions in Asia to have a railway. The first
railway in what is now Bangladesh began operating in 1862. In
comparison, Japan saw its first railway in 1872. The main railway
companies in the region were the Eastern Bengal Railway and Assam Bengal
Railway. Railways competed with waterborne transport to become one of
the main mediums of transport.

Social tensions also increased under British rule, particularly between
wealthy Hindus and the Muslim-majority population. The Permanent
Settlement made millions of Muslim peasants tenants of Hindu estates,
and resentment of the Hindu landed gentry grew. Supported by the Muslim
aristocracy, the British government created the province of Eastern
Bengal and Assam in 1905; the new province received increased investment
in education, transport and industry. However, the first partition of
Bengal created an uproar in Calcutta and the Indian National Congress.
In response to growing Hindu nationalism, the All India Muslim League
was formed in Dhaka during the 1906 All India Muhammadan Educational
Conference. The British government reorganized the provinces in 1912,
reuniting East and West Bengal and making Assam a second province.

The Raj was slow to allow self-rule in the colonial subcontinent. It
established the Bengal Legislative Council in 1862, and the council's
native Bengali representation increased during the early 20th century.
The Bengal Provincial Muslim League was formed in 1913 to advocate civil
rights for Bengali Muslims within a constitutional framework. During the
1920s, the league was divided into factions supporting the Khilafat
movement and favoring cooperation with the British to achieve self-rule.
Segments of the Bengali elite supported Mustafa Kemal Ataturk's
secularist forces. In 1929, the All Bengal Tenants Association was
formed in the Bengal Legislative Council to counter the influence of the
Hindu landed gentry, and the Indian Independence and Pakistan Movements
strengthened during the early 20th century. After the Morley-Minto
Reforms and the diarchy era in the legislatures of British India, the
British government promised limited provincial autonomy in 1935. The
Bengal Legislative Assembly, British India's largest legislature, was
established in 1937.

Although it won a majority of seats in 1937, the Bengal Congress
boycotted the legislature. A. K. Fazlul Huq of the Krishak Praja Party
was elected as the first Prime Minister of Bengal. In 1940 Huq supported
the Lahore Resolution, which envisaged independent states in the
northwestern and eastern Muslim-majority regions of the subcontinent.
The first Huq ministry, a coalition with the Bengal Provincial Muslim
League, lasted until 1941; it was followed by a Huq coalition with the
Hindu Mahasabha which lasted until 1943. Huq was succeeded by Khawaja
Nazimuddin, who grappled with the effects of the Burma Campaign, the
Bengal famine of 1943 and the Quit India movement. In 1946, the Bengal
Provincial Muslim League won the provincial election, taking 113 of the
250-seat assembly (the largest Muslim League mandate in British India).
H. S. Suhrawardy, who made a final futile effort for a United Bengal in
1946, was the last premier of Bengal.

\includegraphics[width=5.50000in,height=2.25958in]{media/image2.png}\\
\emph{Prime Ministers of Bengal A. K. Fazlul Huq, Khawaja Nazimuddin and
H. S. Suhrawardy. One of them, Suhrawardy, proposed an independent
Bengal in 1947}

\section{Partition of Bengal (1947)}\label{partition-of-bengal-1947}

\begin{itemize}
\item
  \emph{Cyril Radcliffe was tasked with drawing the borders of Pakistan
  and India, and the Radcliffe Line established the borders of
  present-day Bangladesh.}
\item
  \emph{At another meeting of legislators from East Bengal, it was
  decided (106 votes to 35) that the province should not be partitioned
  and (107 votes to 34) that East Bengal should join the Constituent
  Assembly of Pakistan if Bengal was partitioned.}
\item
  \emph{On 20 June, the Bengal Legislative Assembly met to decide on the
  partition of Bengal.}
\end{itemize}

On 3 June 1947 Mountbatten Plan outlined the partition of British India.
On 20 June, the Bengal Legislative Assembly met to decide on the
partition of Bengal. At the preliminary joint meeting, it was decided
(120 votes to 90) that if the province remained united it should join
the Constituent Assembly of Pakistan. At a separate meeting of
legislators from West Bengal, it was decided (58 votes to 21) that the
province should be partitioned and West Bengal should join the
Constituent Assembly of India. At another meeting of legislators from
East Bengal, it was decided (106 votes to 35) that the province should
not be partitioned and (107 votes to 34) that East Bengal should join
the Constituent Assembly of Pakistan if Bengal was partitioned. On 6
July, the Sylhet region of Assam voted in a referendum to join East
Bengal.\\
Cyril Radcliffe was tasked with drawing the borders of Pakistan and
India, and the Radcliffe Line established the borders of present-day
Bangladesh.

\includegraphics[width=5.50000in,height=4.22643in]{media/image3.png}\\
\emph{Female students march in defiance of the Section 144 prohibition
on assembly during the Bengali Language Movement in early 1953}

\includegraphics[width=4.28016in,height=5.50000in]{media/image4.jpg}\\
\emph{U.S. Chief Justice Earl Warren meets Maulvi Tamizuddin Khan, the
plaintiff in Federation of Pakistan v. Maulvi Tamizuddin Khan}

\section{Union with Pakistan}\label{union-with-pakistan}

\begin{itemize}
\item
  \emph{The 1952 Bengali Language Movement was the first sign of
  friction between the country's geographically-separated wings.}
\item
  \emph{After the December 1970 elections, calls for the independence of
  East Bengal became louder; the Bengali-nationalist Awami League won
  167 of 169 East Pakistani seats in the National Assembly.}
\item
  \emph{In 1962 Dhaka became the seat of the National Assembly of
  Pakistan, a move seen as appeasing increased Bengali nationalism.}
\end{itemize}

The Dominion of Pakistan was created on 14 August 1947. East Bengal,
with Dhaka its capital, was the most populous province of the 1947
Pakistani federation (led by Governor General Muhammad Ali Jinnah, who
promised freedom of religion and secular democracy in the new state).
East Bengal was also Pakistan's most cosmopolitan province, home to
peoples of different faiths, cultures and ethnic groups. Partition gave
increased economic opportunity to East Bengalis, producing an urban
population during the 1950s.

Khawaja Nazimuddin was East Bengal's first chief minister with Frederick
Chalmers Bourne its governor. The All Pakistan Awami Muslim League was
formed in 1949. In 1950, the East Bengal Legislative Assembly enacted
land reform, abolishing the Permanent Settlement and the zamindari
system. The 1952 Bengali Language Movement was the first sign of
friction between the country's geographically-separated wings. The Awami
Muslim League was renamed the more-secular Awami League in 1953. The
first constituent assembly was dissolved in 1954; this was challenged by
its East Bengali speaker, Maulvi Tamizuddin Khan. The United Front
coalition swept aside the Muslim League in a landslide victory in the
1954 East Bengali legislative election. The following year, East Bengal
was renamed East Pakistan as part of the One Unit program and the
province became a vital part of the Southeast Asia Treaty Organization.

Pakistan adopted its first constitution in 1956. Three Bengalis were its
Prime Minister until 1957: Nazimuddin, Mohammad Ali of Bogra and
Suhrawardy. None of the three completed their terms, and resigned from
office. The Pakistan Army imposed military rule in 1958, and Ayub Khan
was the country's strongman for 11 years. Political repression increased
after the coup. Khan introduced a new constitution in 1962, replacing
Pakistan's parliamentary system with a presidential and gubernatorial
system (based on electoral college selection) known as Basic Democracy.
In 1962 Dhaka became the seat of the National Assembly of Pakistan, a
move seen as appeasing increased Bengali nationalism. The Pakistani
government built the controversial Kaptai Dam, displacing the Chakma
people from their indigenous homeland in the Chittagong Hill Tracts.
During the 1965 presidential election, Fatima Jinnah lost to Ayub Khan
despite support from the Combined Opposition alliance (which included
the Awami League). The Indo-Pakistani War of 1965 blocked cross-border
transport links with neighboring India in what is described as a second
partition. In 1966, Awami League leader Sheikh Mujibur Rahman announced
a six point movement for a federal parliamentary democracy.

According to senior World Bank officials, Pakistan practiced extensive
economic discrimination against East Pakistan: greater government
spending on West Pakistan, financial transfers from East to West
Pakistan, the use of East Pakistan's foreign-exchange surpluses to
finance West Pakistani imports, and refusal by the central government to
release funds allocated to East Pakistan because previous spending had
been under budget; East Pakistan generated 70 percent of Pakistan's
export revenue with its jute and tea. Sheikh Mujibur Rahman was arrested
for treason in the Agartala Conspiracy Case, and was released during the
1969 uprising in East Pakistan which resulted in Ayub Khan's
resignation. General Yahya Khan assumed power, reintroducing martial
law.

Ethnic and linguistic discrimination was common in Pakistan's civil and
military services, in which Bengalis were under-represented. Fifteen
percent of Pakistani central-government offices were occupied by East
Pakistanis, who formed 10 percent of the military. Cultural
discrimination also prevailed, making East Pakistan forge a distinct
political identity. Pakistan banned Bengali literature and music in
state media, including the works of Nobel laureate Rabindranath Tagore.
A cyclone devastated the coast of East Pakistan in 1970, killing an
estimated 500,000 people, and the central government was criticized for
its poor response. After the December 1970 elections, calls for the
independence of East Bengal became louder; the Bengali-nationalist Awami
League won 167 of 169 East Pakistani seats in the National Assembly. The
League claimed the right to form a government and develop a new
constitution, but was strongly opposed by the Pakistani military and the
Pakistan Peoples Party (led by Zulfikar Ali Bhutto).

\section{War of Independence}\label{war-of-independence}

\begin{itemize}
\item
  \emph{Pakistan recognized Bangladesh in 1974 after pressure from most
  of the Muslim countries.}
\item
  \emph{The Provisional Government of Bangladesh was established on 17
  April 1971, converting the 469 elected members of the Pakistani
  national assembly and East Pakistani provincial assembly into the
  Constituent Assembly of Bangladesh.}
\item
  \emph{Mujib, however, before his arrest proclaimed the Independence of
  Bangladesh at midnight on 26 March which led the Bangladesh Liberation
  War to break out within hours.}
\end{itemize}

The Bengali population was angered when Prime Minister-elect Sheikh
Mujibur Rahman was prevented from taking the office. Civil disobedience
erupted across East Pakistan, with calls for independence. Mujib
addressed a pro-independence rally of nearly 2 million people in Dacca
on 7 March 1971, where he said, "This time the struggle is for our
freedom. This time the struggle is for our independence." The flag of
Bangladesh was raised for the first time on 23 March, Pakistan's
Republic Day. During the night of 25 March, the Pakistani military junta
led by Yahya Khan launched Operation Searchlight (a sustained military
assault on East Pakistan). The Pakistan Army arrested Sheikh Mujibur
Rahman and flew him away to Karachi. Mujib, however, before his arrest
proclaimed the Independence of Bangladesh at midnight on 26 March which
led the Bangladesh Liberation War to break out within hours. The
Pakistan Army continued to massacre Bengali students, intellectuals,
politicians, civil servants and military defectors in the 1971
Bangladesh genocide, while the Mukti Bahini and other Bengali guerilla
forces created strong resistance throughout the country. During the war,
an estimated 300,000 to three million people were killed and several
million people took shelter in neighboring India. Global public opinion
turned against Pakistan as news of the atrocities spread; the Bangladesh
movement was supported by prominent political and cultural figures in
the West, including Ted Kennedy, George Harrison, Bob Dylan, Joan Baez,
Victoria Ocampo and André Malraux. The Concert for Bangladesh was held
at Madison Square Garden in New York City to raise funds for Bangladeshi
refugees. The first major benefit concert in history, it was organized
by Harrison and Indian Bengali sitarist Ravi Shankar.

During the Bangladesh Liberation War, Bengali nationalists declared
independence and formed the Mukti Bahini (the Bangladeshi National
Liberation Army). The Provisional Government of Bangladesh was
established on 17 April 1971, converting the 469 elected members of the
Pakistani national assembly and East Pakistani provincial assembly into
the Constituent Assembly of Bangladesh. The provisional government
issued a proclamation that became the country's interim constitution and
declared "equality, human dignity and social justice" as its fundamental
principles. Due to Mujib's detention, the acting president was Syed
Nazrul Islam, while Tajuddin Ahmad was Bangladesh's first prime
minister. The military wing of the provisional government was the
Bangladesh Forces that included Mukti Bahini and other Bengali guerilla
forces. Led by General M. A. G. Osmani and eleven sector commanders, the
forces held the countryside during the war and conducted wide-ranging
guerrilla operations against Pakistani forces. As a result, almost the
entire country except the capital Dacca was liberated by Bangladesh
Forces by late November. This led the Pakistan Army to attack
neighboring India's western front on 2 December. India retaliated in
both the western and eastern fronts. With a joint ground advance by
Bangladeshi and Indian forces, coupled with air strikes by both India
and the small Bengali air contingent, the capital Dacca was liberated
from Pakistani occupation in mid-December. During the last phase of the
war, the Soviet Union and the United States dispatched naval forces to
the Bay of Bengal in a Cold War standoff. The nine-months long war ended
with the surrender of Pakistani armed forces to the Bangladesh-India
Allied Forces on 16 December 1971. Under international pressure,
Pakistan released Rahman from imprisonment on 8 January 1972 and he was
flown by the British Royal Air Force to a million-strong homecoming in
Dacca. Remaining Indian troops were withdrawn by 12 March 1972, three
months after the war ended.

The cause of Bangladeshi self-determination was recognized around the
world. By August 1972, the new state was recognized by 86 countries.
Pakistan recognized Bangladesh in 1974 after pressure from most of the
Muslim countries.

\section{People's Republic of
Bangladesh}\label{peoples-republic-of-bangladesh}

\includegraphics[width=5.50000in,height=3.33278in]{media/image5.jpg}\\
\emph{Prime Minister Sheikh Mujibur Rahman and U.S. president Gerald
Ford in 1974}

\section{First parliamentary era}\label{first-parliamentary-era}

\begin{itemize}
\item
  \emph{The Bangladesh famine of 1974 also worsened the political
  situation.}
\item
  \emph{The constituent assembly adopted Bangladesh's constitution on 4
  November 1972, establishing a secular, multiparty parliamentary
  democracy.}
\item
  \emph{Bangladesh joined the Commonwealth of Nations, the UN, the OIC
  and the Non-Aligned Movement, and Rahman strengthened ties with
  India.}
\end{itemize}

The constituent assembly adopted Bangladesh's constitution on 4 November
1972, establishing a secular, multiparty parliamentary democracy. The
new constitution included references to socialism, and Prime Minister
Sheikh Mujibur Rahman nationalized major industries in 1972. A major
reconstruction and rehabilitation program was launched. The Awami League
won the country's first general election in 1973, securing a large
majority in the Jatiyo Sangshad. Bangladesh joined the Commonwealth of
Nations, the UN, the OIC and the Non-Aligned Movement, and Rahman
strengthened ties with India. Amid growing agitation by the opposition
National Awami Party and National Socialist Party, he became
increasingly authoritarian. Rahman amended the constitution, giving
himself more emergency powers (including the suspension of fundamental
rights). The Bangladesh famine of 1974 also worsened the political
situation.

\section{Presidential era and coups
(1975--1991)}\label{presidential-era-and-coups-19751991}

\begin{itemize}
\item
  \emph{Bangladesh was governed by a military junta led by the Chief
  Martial Law Administrator for three years.}
\item
  \emph{A semi-presidential system evolved, with the Bangladesh
  Nationalist Party (BNP) governing until 1982.}
\item
  \emph{After a year in office, Sattar was overthrown in the 1982
  Bangladesh coup d'état.}
\end{itemize}

In January 1975, Sheikh Mujibur Rahman introduced one-party socialist
rule under BAKSAL. Rahman banned all newspapers except four state-owned
publications, and amended the constitution to increase his power. He was
assassinated during a coup on 15 August 1975. Martial law was declared,
and the presidency passed to the usurper Khondaker Mostaq Ahmad for four
months. Ahmad is widely regarded as a quisling by Bangladeshis. Tajuddin
Ahmad, the nation's first prime minister, and four other independence
leaders were assassinated on 4 November 1975. Chief Justice Abu Sadat
Mohammad Sayem was installed as president by the military on 6 November
1975. Bangladesh was governed by a military junta led by the Chief
Martial Law Administrator for three years. In 1977, the army chief Ziaur
Rahman became president. Rahman reinstated multiparty politics,
privatized industries and newspapers, established BEPZA and held the
country's second general election in 1979. A semi-presidential system
evolved, with the Bangladesh Nationalist Party (BNP) governing until
1982. Rahman was assassinated in 1981, and was succeeded by Vice
President Abdus Sattar. Sattar received 65.5 percent of the vote in the
1981 presidential election.

After a year in office, Sattar was overthrown in the 1982 Bangladesh
coup d'état. Chief Justice A. F. M. Ahsanuddin Chowdhury was installed
as president, but army chief Hussain Muhammad Ershad became the
country's de facto leader and assumed the presidency in 1983. Ershad
lifted martial law in 1986. He governed with four successive prime
ministers (Ataur Rahman Khan, Mizanur Rahman Chowdhury, Moudud Ahmed and
Kazi Zafar Ahmed) and a parliament dominated by his Jatiyo Party.
General elections were held in 1986 and 1988, although the latter was
boycotted by the opposition BNP and Awami League. Ershad pursued
administrative decentralization, dividing the country into 64 districts,
and pushed Parliament to make Islam the state religion in 1988. A 1990
mass uprising forced him to resign, and Chief Justice Shahabuddin Ahmed
led the country's first caretaker government as part of the transition
to parliamentary rule.

\includegraphics[width=5.50000in,height=3.09375in]{media/image6.JPG}\\
\emph{Rohingya refugees entering Bangladesh from Myanmar}

\section{Current parliamentary era
(1991--present)}\label{current-parliamentary-era-1991present}

\begin{itemize}
\item
  \emph{Due to strong domestic demand, Bangladesh emerged as one of the
  fastest-growing economies in the world.}
\item
  \emph{The election was also criticized by the United States and the
  European Union, the largest export markets and foreign investment
  sources for Bangladesh.}
\item
  \emph{Between 2016 and 2017, an estimated 1 million Rohingya refugees
  took shelter in southeastern Bangladesh amid a military crackdown in
  neighbouring Rakhine State, Myanmar.}
\end{itemize}

After the 1991 general election, the twelfth amendment to the
constitution restored the parliamentary republic and Begum Khaleda Zia
became Bangladesh's first female prime minister. Zia, a former first
lady, led a BNP government from 1990 to 1996. In 1991 her finance
minister, Saifur Rahman, began a major program to liberalize the
Bangladeshi economy.

In February 1996, a general election was held which was boycotted by all
opposition parties giving a 300 (of 300) seat victory for BNP. This
election was deemed illegitimate, so a system of a caretaker government
was introduced to oversee the transfer of power and a new election was
held in June 1996, overseen by Justice Muhammad Habibur Rahman, the
first Chief Adviser of Bangladesh. The Awami League won the seventh
general election, marking its leader Sheikh Hasina's first term as Prime
Minister. Hasina's first term was highlighted by the Chittagong Hill
Tracts Peace Accord and a Ganges water-sharing treaty with India. The
second caretaker government, led by Chief Adviser Justice Latifur
Rahman, oversaw the 2001 Bangladeshi general election which returned
Begum Zia and the BNP to power.

The second Zia administration saw improved economic growth, but
political turmoil gripped the country between 2004 and 2006. A radical
Islamist militant group, the JMB, carried out a series of terror
attacks. The evidence of staging these attacks by these extremist groups
have been found in the investigation, and hundreds of suspected members
were detained in numerous security operations in 2006, including the two
chiefs of the JMB, Shaykh Abdur Rahman and Bangla Bhai, who were
executed with other top leaders in March 2007, bringing the militant
group to an end.

In 2006, at the end of the term of the BNP administration, there was
widespread political unrest related to the handover of power to a
caretaker government. As such, the Bangladeshi military urged President
Iajuddin Ahmed to impose a state of emergency and a caretaker
government, led by technocrat Fakhruddin Ahmed, was installed. Emergency
rule lasted for two years, during which time investigations into members
of both Awami League and BNP were conducted, including their leaders
Sheikh Hasina and Khaleda Zia. In 2008 the ninth general election saw a
return to power for Sheikh Hasina and the Awami League led Grand
Alliance in a landslide victory. In 2010, the Supreme Court ruled
martial law illegal and affirmed secular principles in the constitution.
The following year, the Awami League abolished the caretaker-government
system.

Citing the lack of caretaker government the 2014 general election was
boycotted by the BNP and other opposition parties, giving the Awami
League a decisive victory. The election was controversial with reports
of violence and an alleged crackdown on the opposition in the run-up to
the election and 153 seats (of 300) went uncontested in the election.
Despite the controversy Hasina went on to form a government which saw
her return for a third term as Prime Minister. Due to strong domestic
demand, Bangladesh emerged as one of the fastest-growing economies in
the world. However, human rights abuses increased under the Hasina
administration, particularly enforced disappearances. Between 2016 and
2017, an estimated 1 million Rohingya refugees took shelter in
southeastern Bangladesh amid a military crackdown in neighbouring
Rakhine State, Myanmar.

In 2018, the country saw major movements for government quota reforms
and road-safety. The Bangladeshi general election, 2018 was marred by
allegations of widespread vote rigging. The Awami League won 259 out of
300 seats and the main opposition alliance Jatiya Oikya Front secured
only 8 seats, with Sheikh Hasina becoming the longest serving prime
minister in Bangladeshi history. Pro-democracy leader Dr. Kamal Hossain
called for an annulment of the election result and for a new election to
be held in a free and fair manner. The election was also criticized by
the United States and the European Union, the largest export markets and
foreign investment sources for Bangladesh.

\includegraphics[width=5.50000in,height=3.59008in]{media/image7.jpg}\\
\emph{A satellite image showing the topography of Bangladesh}

\section{Geography}\label{geography}

\begin{itemize}
\item
  \emph{The geography of Bangladesh is divided between three regions.}
\item
  \emph{In southeastern Bangladesh, experiments have been done since the
  1960s to 'build with nature'.}
\item
  \emph{With an elevation of 1,064~m (3,491~ft), the highest peak of
  Bangladesh is Keokradong, near the border with Myanmar.}
\item
  \emph{Bangladesh is predominantly rich fertile flat land.}
\end{itemize}

The geography of Bangladesh is divided between three regions. Most of
the country is dominated by the fertile Ganges-Brahmaputra delta; the
northwest and central parts of the country are formed by the Madhupur
and the Barind plateaus. The northeast and southeast are home to
evergreen hill ranges. The Ganges delta is formed by the confluence of
the Ganges (local name Padma or Pôdda), Brahmaputra (Jamuna or Jomuna),
and Meghna rivers and their respective tributaries. The Ganges unites
with the Jamuna (main channel of the Brahmaputra) and later joins the
Meghna, finally flowing into the Bay of Bengal. Bangladesh has 57
trans-boundary rivers, making the resolution of water issues politically
complicated, in most cases, as the country is a lower riparian state to
India.

Bangladesh is predominantly rich fertile flat land. Most parts of it is
less than 12~m (39.4~ft) above sea level, and it is estimated that about
10\% of its land would be flooded if the sea level were to rise by 1~m
(3.28~ft). 17\% of the country is covered by forests and 12\% is covered
by hill systems. The country's haor wetlands are of significance to
global environmental science.

In southeastern Bangladesh, experiments have been done since the 1960s
to 'build with nature'. Construction of cross dams has induced a natural
accretion of silt, creating new land. With Dutch funding, the
Bangladeshi government began promoting the development of this new land
in the late 1970s. The effort has become a multi-agency endeavor,
building roads, culverts, embankments, cyclone shelters, toilets and
ponds, as well as distributing land to settlers. It was expected that by
fall 2010, the program would have allotted some 27,000 acres (10,927~ha)
to 21,000 families.\\
With an elevation of 1,064~m (3,491~ft), the highest peak of Bangladesh
is Keokradong, near the border with Myanmar.

\section{Administrative geography}\label{administrative-geography}

\begin{itemize}
\item
  \emph{Bangladesh is divided into eight administrative divisions, each
  named after their respective divisional headquarters: Barisal,
  Chittagong, Dhaka, Khulna, Mymensingh, Rajshahi, Rangpur, and Sylhet.}
\item
  \emph{There are 64 districts in Bangladesh, each further subdivided
  into upazila (subdistricts) or thana.}
\end{itemize}

Bangladesh is divided into eight administrative divisions, each named
after their respective divisional headquarters: Barisal, Chittagong,
Dhaka, Khulna, Mymensingh, Rajshahi, Rangpur, and Sylhet.

Divisions are subdivided into districts (zila). There are 64 districts
in Bangladesh, each further subdivided into upazila (subdistricts) or
thana. The area within each police station, except for those in
metropolitan areas, is divided into several unions, with each union
consisting of multiple villages. In the metropolitan areas, police
stations are divided into wards, which are further divided into
mahallas.

There are no elected officials at the divisional or district levels, and
the administration is composed only of government officials. Direct
elections are held in each union (or ward) for a chairperson and a
number of members. In 1997, a parliamentary act was passed to reserve
three seats (out of 12) in every union for female candidates.

\section{Climate}\label{climate}

\begin{itemize}
\item
  \emph{In September 1998, Bangladesh saw the most severe flooding in
  modern world history.}
\item
  \emph{Bangladesh is now widely recognised to be one of the countries
  most vulnerable to climate change.}
\item
  \emph{Over the course of a century, 508 cyclones have affected the Bay
  of Bengal region, 17 percent of which are believed to have caused
  landfall in Bangladesh.}
\end{itemize}

Straddling the Tropic of Cancer, Bangladesh's climate is tropical with a
mild winter from October to March, and a hot, humid summer from March to
June. The country has never recorded an air temperature below 0~°C
(32~°F), with a record low of 1.1~°C (34.0~°F) in the north west city of
Dinajpur on 3 February 1905. A warm and humid monsoon season lasts from
June to October and supplies most of the country's rainfall.

Natural calamities, such as floods, tropical cyclones, tornadoes, and
tidal bores occur almost every year, combined with the effects of
deforestation, soil degradation and erosion. The cyclones of 1970 and
1991 were particularly devastating, the latter killing some 140,000
people.

In September 1998, Bangladesh saw the most severe flooding in modern
world history. As the Brahmaputra, the Ganges and Meghna spilt over and
swallowed 300,000 houses, 9,700~km (6,000~mi) of road and 2,700~km
(1,700~mi) of embankment, 1,000~people were killed and 30~million more
were made homeless; 135,000 cattle were killed; 50~km2 (19~sq~mi) of
land were destroyed; and 11,000~km (6,800~mi) of roads were damaged or
destroyed. Effectively, two-thirds of the country was underwater.\\
The severity of the flooding was attributed to unusually high monsoon
rains, the shedding of equally unusually large amounts of melt water
from the Himalayas, and the widespread cutting down of trees (that would
have intercepted rain water) for firewood or animal husbandry.

Bangladesh is now widely recognised to be one of the countries most
vulnerable to climate change. Over the course of a century, 508 cyclones
have affected the Bay of Bengal region, 17 percent of which are believed
to have caused landfall in Bangladesh. Natural hazards that come from
increased rainfall, rising sea levels, and tropical cyclones are
expected to increase as the climate changes, each seriously affecting
agriculture, water and food security, human health, and shelter. It is
estimated that by 2050, a 3 feet rise in sea levels will inundate some
20 percent of the land and displace more than 30 million people.

There is evidence that earthquakes pose a threat to the country and that
plate tectonics have caused rivers to shift course suddenly and
dramatically. It has been shown that rainy-season flooding in
Bangladesh, on the world's largest river delta, can push the underlying
crust down by as much as 6~centimetres, and possibly perturb faults.

Bangladeshi water is frequently contaminated with arsenic because of the
high arsenic content of the soil---up to 77 million people are exposed
to toxic arsenic from drinking water.

\section{Biodiversity}\label{biodiversity}

\begin{itemize}
\item
  \emph{The Bangladesh Environment Conservation Act was enacted in
  1995.}
\item
  \emph{Bangladesh ratified the Rio Convention on Biological Diversity
  on 3 May 1994.}
\item
  \emph{Bangladesh has one of the largest population of Irrawaddy
  dolphins and Ganges dolphins.}
\item
  \emph{Bangladesh is located in the Indomalaya ecozone.}
\item
  \emph{Bangladesh has an abundance of wildlife in its forests, marshes,
  woodlands and hills.}
\end{itemize}

Bangladesh ratified the Rio Convention on Biological Diversity on 3 May
1994. As of 2014{[}update{]}, the country was set to revise its National
Biodiversity Strategy and Action Plan.

Bangladesh is located in the Indomalaya ecozone. Its ecology includes a
long sea coastline, numerous rivers and tributaries, lakes, wetlands,
evergreen forests, semi evergreen forests, hill forests, moist deciduous
forests, freshwater swamp forests and flat land with tall grass. The
Bangladesh Plain is famous for its fertile alluvial soil which supports
extensive cultivation. The country is dominated by lush vegetation, with
villages often buried in groves of mango, jackfruit, bamboo, betel nut,
coconut and date palm. The country has up to 6000 species of plant life,
including 5000 flowering plants. Water bodies and wetland systems
provide a habitat for many aquatic plants. Water lilies and lotuses grow
vividly during the monsoon season. The country has 50 wildlife
sanctuaries.

Bangladesh is home to much of the Sundarbans, the world's largest
mangrove forest, covering an area of 6,000~km2 in the southwest littoral
region. It is divided into three protected sanctuaries--the South, East
and West zones. The forest is a UNESCO World Heritage Site. The
northeastern Sylhet region is home to haor wetlands, which is a unique
ecosystem. It also includes tropical and subtropical coniferous forests,
a freshwater swamp forest and mixed deciduous forests. The southeastern
Chittagong region covers evergreen and semi evergreen hilly jungles.
Central Bangladesh includes the plainland Sal forest running along the
districts of Gazipur, Tangail and Mymensingh. St. Martin's Island is the
only coral reef in the country.

Bangladesh has an abundance of wildlife in its forests, marshes,
woodlands and hills. The vast majority of animals dwell within a habitat
of 150,000~km2. The Bengal tiger, clouded leopard, saltwater crocodile,
black panther and fishing cat are among the chief predators in the
Sundarbans. Northern and eastern Bangladesh is home to the Asian
elephant, hoolock gibbon, Asian black bear and oriental pied hornbill.

The Chital deer are widely seen in southwestern woodlands. Other animals
include the black giant squirrel, capped langur, Bengal fox, sambar
deer, jungle cat, king cobra, wild boar, mongooses, pangolins, pythons
and water monitors. Bangladesh has one of the largest population of
Irrawaddy dolphins and Ganges dolphins. A 2009 census found 6,000
Irrawaddy dolphins inhabiting the littoral rivers of Bangladesh. The
country has numerous species of amphibians (53), reptiles (139), marine
reptiles (19) and marine mammals (5). It also has 628 species of birds.

Several animals became extinct in Bangladesh during the last century,
including the one horned and two horned rhinoceros and common peafowl.
The human population is concentrated in urban areas, hence limiting
deforestation to a certain extent. Rapid urban growth has threatened
natural habitats. Although many areas are protected under law, a large
portion of Bangladeshi wildlife is threatened by this growth. The
Bangladesh Environment Conservation Act was enacted in 1995. The
government has designated several regions as Ecologically Critical
Areas, including wetlands, forests and rivers. The Sundarbans tiger
project and the Bangladesh Bear Project are among the key initiatives to
strengthen conservation.

\section{Politics and government}\label{politics-and-government}

\begin{itemize}
\item
  \emph{Bangladesh is increasingly classified as an autocracy due to the
  authoritarian practices of its government.}
\item
  \emph{Once the world's fifth largest democracy, Bangladesh experienced
  a two party system between 1990 and 2014, when the Awami League and
  the Bangladesh Nationalist Party (BNP) alternated in power.}
\item
  \emph{Bangladesh has a prominent civil society since the colonial
  period.}
\end{itemize}

Bangladesh is a de jure representative democracy under its constitution,
with a Westminster-style unitary parliamentary republic that has
universal suffrage. The head of government is the Prime Minister, who is
invited to form a government every five years by the President. The
President invites the leader of the largest party in parliament to
become Prime Minister. Once the world's fifth largest democracy,
Bangladesh experienced a two party system between 1990 and 2014, when
the Awami League and the Bangladesh Nationalist Party (BNP) alternated
in power. During this period, elections were managed by a neutral
caretaker government. But the caretaker government was abolished by the
Awami League government in 2011. The BNP boycotted the next election in
2014, arguing that it would not be fair without a caretaker government.
The BNP-led Jatiya Oikya Front participated in the 2018 election and
lost. The election saw many allegations of irregularities. Bangladesh is
increasingly classified as an autocracy due to the authoritarian
practices of its government. The democratic wave which ushered
parliamentary democracy in 1990 has been reversed by an illiberal
electoral autocracy which features a dominant party state led by the
Awami League. Bangladesh has a prominent civil society since the
colonial period. There are various interest groups, including
non-governmental organizations, human rights organizations, professional
associations, chambers of commerce, employers' associations and trade
unions.

One of the key aspects of Bangladeshi politics is the so-called "spirit
of the liberation war", which refers to the ideals of the liberation
movement. For example, the Proclamation of Independence enunciated the
values of "equality, human dignity and social justice". In 1972, the
constitution included a bill of rights and declared "nationalism,
socialism, democracy and secularity" as the principles of government
policy. Socialism was later de-emphasized and neglected by successive
governments; Bangladesh has a market-based economy. To many
Bangladeshis, especially in the younger generation, the spirit of the
liberation war is a vision for a society based on civil liberties, human
rights, the rule of law and good governance.

\section{Executive branch}\label{executive-branch}

\begin{itemize}
\item
  \emph{The Bangladesh Civil Service assists the cabinet in running the
  government.}
\item
  \emph{The Government of Bangladesh is overseen by a cabinet headed by
  the Prime Minister of Bangladesh.}
\item
  \emph{The President is the Supreme Commander of the Bangladesh Armed
  Forces and the chancellor of all universities.}
\end{itemize}

The Government of Bangladesh is overseen by a cabinet headed by the
Prime Minister of Bangladesh. The tenure of a parliamentary government
is five years. The Bangladesh Civil Service assists the cabinet in
running the government. Recruitment for the civil service is based on a
public examination. In theory, the civil service should be a
meritocracy. But politicization and preference for seniority have
affected the civil service's meritocracy. The President of Bangladesh is
the ceremonial head of state whose powers include signing bills passed
by parliament into law. The President is elected by the parliament and
has a five year term. Under the constitution, the president acts on the
advice of the prime minister. The President is the Supreme Commander of
the Bangladesh Armed Forces and the chancellor of all universities.

\section{Legislative branch}\label{legislative-branch}

\begin{itemize}
\item
  \emph{Article 70 of the Constitution of Bangladesh forbids MPs from
  voting against their party, thereby rendering the Jatiya Sangshad a
  largely rubber-stamp parliament.}
\item
  \emph{A bill proposing to declare Bangladesh as a nuclear weapons free
  zone remains pending.}
\end{itemize}

The Jatiya Sangshad (National Assembly) is the unicameral parliament. It
has 350 Members of Parliament (MPs), including 300 MPs elected on the
first past the post system and 50 MPs appointed to reserved seats for
women's empowerment. Article 70 of the Constitution of Bangladesh
forbids MPs from voting against their party, thereby rendering the
Jatiya Sangshad a largely rubber-stamp parliament. However, several laws
proposed independently by MPs have been transformed into legislation,
including the anti-torture law. A bill proposing to declare Bangladesh
as a nuclear weapons free zone remains pending. The parliament is
presided over by the Speaker of the Jatiya Sangsad, who is second in
line to the president as per the constitution. There is also a Deputy
Speaker. When a president is incapable of performing duties (i.e. due to
illness), the Speaker steps in as Acting President and the Deputy
Speaker becomes Acting Speaker. A recurring proposal suggests that the
Deputy Speaker should be a member of the opposition.

\section{Legal system}\label{legal-system}

\begin{itemize}
\item
  \emph{The Bangladesh Code includes a list of all laws in force in the
  country.}
\item
  \emph{The head of the judiciary is the Chief Justice of Bangladesh,
  who sits on the Supreme Court.}
\item
  \emph{Although most of Bangladesh's laws were compiled in English,
  after a 1987 government directive laws are now primarily written in
  Bengali.}
\end{itemize}

The Supreme Court of Bangladesh, including its High Court and Appellate
Divisions, is the high court of the land. The head of the judiciary is
the Chief Justice of Bangladesh, who sits on the Supreme Court. The
courts have wide latitude in judicial review, and judicial precedent is
supported by the Article 111 of the constitution. The judiciary includes
district and metropolitan courts, which are divided into civil and
criminal courts. Due to a shortage of judges, the judiciary has a large
backlog. The Bangladesh Judicial Service Commission is an independent
body responsible for judicial appointments, salaries and discipline.

Bangladesh's legal system is based on common law, and its principal
source of laws are acts of Parliament. The Bangladesh Code includes a
list of all laws in force in the country. The code begins in 1836, and
most of its listed laws were crafted under the British Raj by the Bengal
Legislative Council, the Bengal Legislative Assembly, the Eastern Bengal
and Assam Legislative Council, the Imperial Legislative Council and the
Parliament of the United Kingdom; one example is the 1860 Penal Code.
From 1947 to 1971, laws were enacted by Pakistan's national assembly and
the East Pakistani legislature. The Constituent Assembly of Bangladesh
was the country's provisional parliament until 1973, when the first
elected Jatiyo Sangshad was sworn in. Although most of Bangladesh's laws
were compiled in English, after a 1987 government directive laws are now
primarily written in Bengali. While most of Bangladeshi law is secular;
marriage, divorce and inheritance are governed by Islamic, Hindu and
Christian family law. The judiciary is often influenced by legal
developments in the Commonwealth of Nations, such as the doctrine of
legitimate expectation.

\section{Military}\label{military}

\begin{itemize}
\item
  \emph{The Bangladesh Air Force is equipped with several Russian
  multi-role fighter jets.}
\item
  \emph{The Bangladesh Navy has the third-largest fleet (after India and
  Thailand) of countries dependent on the Bay of Bengal, including
  guided-missile frigates, submarines, cutters and aircraft.}
\item
  \emph{For many years, Bangladesh has been the world's largest
  contributor to UN peacekeeping forces.}
\item
  \emph{Eighty percent of Bangladesh's military equipment comes from
  China.}
\end{itemize}

The Bangladesh Armed Forces have inherited the institutional framework
of the British military and the British Indian Army. It was formed in
1971 from the military regiments of East Pakistan. In 2012 the army
strength was around 300,000, including reservists, the Air Force
(22,000) and the Navy (24,000). In addition to traditional defence
roles, the military has supported civil authorities in disaster relief
and provided internal security during periods of political unrest. For
many years, Bangladesh has been the world's largest contributor to UN
peacekeeping forces. In February 2015, the country made major
deployments to Côte d'Ivoire, Cyprus, Darfur, the Democratic Republic of
Congo, the Golan Heights, Haiti, Lebanon, Liberia and South Sudan.

The Bangladesh Navy has the third-largest fleet (after India and
Thailand) of countries dependent on the Bay of Bengal, including
guided-missile frigates, submarines, cutters and aircraft. The
Bangladesh Air Force is equipped with several Russian multi-role fighter
jets. Bangladesh cooperates defensively with the United States Armed
Forces, participating in the Cooperation Afloat Readiness and Training
(CARAT) exercises. Ties between the Bangladeshi and the Indian military
have increased, with high-level visits by the military chiefs of both
countries. Eighty percent of Bangladesh's military equipment comes from
China.

\includegraphics[width=5.50000in,height=4.10448in]{media/image8.jpg}\\
\emph{US Secretary of State Mike Pompeo with Bangladeshi Foreign
Minister AK Abdul Momen in Washington, D.C., 2019}

\section{Foreign relations}\label{foreign-relations}

\begin{itemize}
\item
  \emph{In 2015, major Indian newspapers called Bangladesh a "trusted
  friend".}
\item
  \emph{Bangladesh and India are South Asia's largest trading partners.}
\item
  \emph{Bangladesh relies on multilateral diplomacy in the World Trade
  Organization.}
\item
  \emph{Bangladesh's most politically-important bilateral relationship
  is with neighboring India.}
\end{itemize}

The first major intergovernmental organization joined by Bangladesh was
the Commonwealth of Nations in 1972. The country joined the United
Nations in 1974, and has been elected twice to the UN Security Council.
Ambassador Humayun Rashid Choudhury was elected president of the UN
General Assembly in 1986. Bangladesh relies on multilateral diplomacy in
the World Trade Organization. It is a major contributor to UN
peacekeeping, providing 113,000 personnel to 54 UN missions in the
Middle East, the Balkans, Africa and the Caribbean in 2014.

In addition to membership in the Commonwealth and the United Nations,
Bangladesh pioneered regional cooperation in South Asia. Bangladesh is a
founding member of the South Asian Association for Regional Cooperation
(SAARC), an organization designed to strengthen relations and promote
economic and cultural growth among its members. It has hosted several
summits, and two Bangladeshi diplomats were the organization's
secretary-general.

Bangladesh joined the Organization of Islamic Cooperation (OIC) in 1973.
It has hosted the summit of OIC foreign ministers, which addresses
issues, conflicts and disputes affecting Muslim-majority countries.
Bangladesh is a founding member of the Developing 8 Countries, a bloc of
eight Muslim-majority republics.

Japan is Bangladesh's largest economic-aid provider, and the countries
have common political goals. The United Kingdom has longstanding
economic, cultural and military links with Bangladesh. The United States
is a major economic and security partner, including its largest export
market and foreign investor. Seventy-six percent of Bangladeshis viewed
the United States favorably in 2014, one of the highest ratings among
Asian countries. The United States views Bangladesh as a key partner in
the Indo-Pacific. The European Union is Bangladesh's largest regional
market, conducting public diplomacy and providing development
assistance.

Relations with other countries are generally positive. Shared democratic
values ease relations with Western countries, and similar economic
concerns forge ties to other developing countries. Despite poor working
conditions and war affecting overseas Bangladeshi workers, relations
with Middle Eastern countries are friendly and bounded by religion and
culture; more than a million Bangladeshis are employed in the region. In
2016, the king of Saudi Arabia called Bangladesh "one of the most
important Muslim countries".

Bangladesh's most politically-important bilateral relationship is with
neighboring India. In 2015, major Indian newspapers called Bangladesh a
"trusted friend". Bangladesh and India are South Asia's largest trading
partners. The countries are forging regional economic and infrastructure
projects, such as a regional motor-vehicle agreement in eastern South
Asia and a coastal shipping agreement in the Bay of Bengal.
Indo-Bangladesh relations often emphasize a shared cultural heritage,
democratic values and a history of support for Bangladeshi independence.
Despite political goodwill, border killings of Bangladeshi civilians and
the lack of a comprehensive water-sharing agreement for 54
trans-boundary rivers are major issues. In 2017, India joined Russia and
China in refusing to condemn Myanmar's atrocities against the Rohingya,
which contradicted with Bangladesh's demand for recognizing Rohingya
human rights. However, the Indian air force delivered aid shipments for
Rohingya refugees in Bangladesh. The rise of Hindu extremism and
Islamophobia in India has also affected Bangladesh. The Bangladeshi beef
and leather industries have seen increased prices due to the Indian BJP
government's Hindu nationalist campaign against the export of beef and
cattle skin.

Sino-Bangladesh relations date to the 1950s and are relatively warm,
despite the Chinese leadership siding with Pakistan during Bangladesh's
war of independence. China and Bangladesh established bilateral
relations in 1976 which have significantly strengthened, and the country
is considered a cost-effective source of arms for the Bangladeshi
military. Since the 1980s 80 percent of Bangladesh's military equipment
has been supplied by China (often with generous credit terms), and China
is Bangladesh's largest trading partner. Both countries are part of the
BCIM Forum.

The neighbouring country of Myanmar was one of first countries to
recognize Bangladesh. Despite common regional interests,
Bangladesh-Myanmar relations have been strained by the Rohingya refugee
issue and the isolationist policies of the Myanmar military. In 2012,
the countries came to terms at the International Tribunal for the Law of
the Sea over maritime disputes in the Bay of Bengal. In 2016 and 2017,
relations with Myanmar again strained as over 400,000 Rohingya refugees
entered Bangladesh after atrocities. The parliament, government and
civil society of Bangladesh have been at the forefront of international
criticism against Myanmar for military operations against the Rohingya,
which the United Nations has described as ethnic cleansing.

Pakistan and Bangladesh have a US\$550 million trade relationship,
particularly in Pakistani cotton imports for the Bangladeshi textile
industry. Although Bangladeshi and Pakistani businesses have invested in
each other, diplomatic relations are strained because of Pakistani
denial of the 1971 Bangladesh genocide.

Bangladeshi aid agencies work in many developing countries. An example
is BRAC in Afghanistan, which benefits 12 million people in that
country. Bangladesh has a record of nuclear nonproliferation as a party
to the Nuclear Nonproliferation Treaty (NPT) and the Comprehensive Test
Ban Treaty (CTBT). It is a state party to the Rome Statute of the
International Criminal Court.

Bangladeshi foreign policy is influenced by the principle of "friendship
to all and malice to none", first articulated by Bengali statesman H. S.
Suhrawardy in 1957. Suhrawardy led East and West Pakistan to join the
Southeast Asia Treaty Organization, CENTO and the Regional Cooperation
for Development.

\section{Human rights}\label{human-rights}

\begin{itemize}
\item
  \emph{Bangladesh also recognizes the third gender.}
\item
  \emph{Bangladesh was the third-most-peaceful South Asian country in
  the 2015 Global Peace Index.}
\item
  \emph{The National Human Rights Commission of Bangladesh was set up in
  2007. Notable human rights organizations and initiatives include the
  Centre for Law and Mediation, Odhikar, the Alliance for Bangladesh
  Worker Safety, the Bangladesh Environmental Lawyers Association, the
  Bangladesh Hindu Buddhist Christian Unity Council and the War Crimes
  Fact Finding Committee.}
\end{itemize}

A list of fundamental rights is enshrined in the country's constitution.
The drafter of the constitution in 1972, Dr. Kamal Hossain, was
influenced by the Universal Declaration of Human Rights. Bangladesh also
recognizes the third gender. Judicial activism has often upheld human
rights. In the 1970s, judges invalidated detentions under the Special
Powers Act, 1974 through cases such as Aruna Sen v. Government of
Bangladesh and Abdul Latif Mirza v. Government of Bangladesh. In 2008,
the Supreme Court paved the way for citizenship for the Stranded
Pakistanis, who were an estimated 300,000 stateless people. Despite
being a non-signatory of the UN Refugee Convention, Bangladesh has taken
in Rohingya refugees since 1978 and the country is now home to a million
refugees. Bangladesh is an active member of the International Labour
Organization (ILO) since 1972. It has ratified 33 ILO conventions,
including the seven fundamental ILO conventions. Bangladesh has ratified
the International Covenant on Civil and Political Rights and the
International Covenant on Economic, Social and Cultural Rights. In 2018,
Bangladesh came under heavy criticism for its repressive Digital
Security Act which threatened freedom of speech. The photojournalist
Shahidul Alam was jailed and tortured for criticizing the government.
Alam was featured in the 2018 Time Person of the Year issue.

The National Human Rights Commission of Bangladesh was set up in 2007.
Notable human rights organizations and initiatives include the Centre
for Law and Mediation, Odhikar, the Alliance for Bangladesh Worker
Safety, the Bangladesh Environmental Lawyers Association, the Bangladesh
Hindu Buddhist Christian Unity Council and the War Crimes Fact Finding
Committee.

Successive governments and their security forces have flouted
constitutional principles and have been accused of human rights abuses.
Bangladesh is ranked "partly free" in Freedom House's Freedom in the
World report, but its press is ranked "not free". According to the
British Economist Intelligence Unit, the country has a hybrid regime:
the third of four rankings in its Democracy Index. Bangladesh was the
third-most-peaceful South Asian country in the 2015 Global Peace Index.
Civil society and media in Bangladesh have been attacked by the ruling
Awami League government and Islamic extremists.

According to National Human Rights Commission, 70\% of alleged
human-rights violations are committed by law-enforcement agencies.
Targets have included Nobel Peace Prize winner Muhammad Yunus and the
Grameen Bank, secularist bloggers and independent and pro-opposition
newspapers and television networks. The United Nations is concerned
about government "measures that restrict freedom of expression and
democratic space".

Bangladeshi security forces, particularly the Rapid Action Battalion
(RAB), have received international condemnation for human-rights abuses
(including enforced disappearances, torture and extrajudicial killings).
Over 1,000 people have been said to have been victims of extrajudicial
killings by RAB since its inception under the last Bangladesh
Nationalist Party government. The RAB has been called a "death squad" by
Human Rights Watch and Amnesty International, which have called for the
force to be disbanded. The British and American governments have been
criticized for funding and engaging the force in counter-terrorism
operations.

The Bangladeshi government has not fully implemented the Chittagong Hill
Tracts Peace Accord. The Hill Tracts region remains heavily militarized,
despite a peace treaty with indigenous people forged by the United
People's Party of the Chittagong Hill Tracts.

Secularism is protected by the constitution of Bangladesh and religious
parties are barred from contesting elections; however, the government is
accused of courting religious extremist groups. Islam's ambiguous
position as the de facto state religion has been criticized by the
United Nations. Despite relative harmony, religious minorities have
faced occasional persecution. The Hindu and Buddhist communities have
experienced religious violence from Islamic groups, notably the
Bangladesh Jamaat-e-Islami and its student wing (Shibir). Islamic
far-right candidates peaked at 12 percent of the vote in 2001, falling
to four percent in 2008. Homosexuality is outlawed by section 377 of the
criminal code (a legacy of the colonial period), and is punishable by a
maximum of life imprisonment.

According to the 2016 Global Slavery Index, an estimated 1,531,300
people are enslaved in modern-day Bangladesh, or 0.95\% of the
population. A number of slaves in Bangladesh are forced to work in the
fish and shrimp industries.

\section{Corruption}\label{corruption}

\begin{itemize}
\item
  \emph{Bangladesh was 14th on Transparency International's 2014
  Corruption Perceptions Index.}
\item
  \emph{In 2015, bribes made up 3.7 percent of the national budget.}
\end{itemize}

Bangladesh was 14th on Transparency International's 2014 Corruption
Perceptions Index. In 2015, bribes made up 3.7 percent of the national
budget. The country's Anti-Corruption Commission was active during the
2006--08 Bangladeshi political crisis, indicting many leading
politicians, bureaucrats and businessmen for graft. After it assumed
power in 2009, the Awami League government reduced the commission's
independent power to investigate and prosecute. Land administration was
the sector with the most bribery in 2015, followed by education,
police\\
and water supply.

\section{Economy}\label{economy}

\begin{itemize}
\item
  \emph{Bangladesh's defense industry includes the Bangladesh Ordnance
  Factories and the Khulna Shipyard.}
\item
  \emph{Bangladesh has the second-highest foreign-exchange reserves in
  South Asia (after India).}
\item
  \emph{Microfinance was pioneered in Bangladesh by Muhammad Yunus.}
\item
  \emph{Bangladesh ranks with Pakistan as South Asia's second-largest
  banking sector.}
\end{itemize}

Bangladesh has the world's 39th largest economy in terms of market
exchange rates and 29th largest in terms of purchasing power parity,
which ranks second in South Asia after India. Bangladesh is also one of
the world's fastest-growing economies and one of the fastest growing
middle-income countries. The country has a market-based mixed economy. A
developing nation, Bangladesh is one of the Next Eleven emerging
markets. According to the IMF, its per-capita income was US\$1,888 in
2018, with a GDP of \$314 billion. Bangladesh has the second-highest
foreign-exchange reserves in South Asia (after India). The Bangladeshi
diaspora contributed \$15.31~billion in remittances in 2015.
Bangladesh's largest trading partners are the European Union, the United
States, Japan, India, Australia, China and ASEAN. Expat workers in the
Middle East and Southeast Asia send back a large chunk of remittances.
The economy is driven by strong domestic demand.

During its first five years of independence Bangladesh adopted socialist
policies. The subsequent military regime and BNP and Jatiya Party
governments restored free markets and promoted the country's private
sector. In 1991, finance minister Saifur Rahman introduced a programme
of economic liberalization. The Bangladeshi private sector has rapidly
expanded, with a number of conglomerates driving the economy. Major
industries include textiles, pharmaceuticals, shipbuilding, steel,
electronics, energy, construction materials, chemicals, ceramics, food
processing and leather goods. Export-oriented industrialization has
increased, with fiscal year 2014--15 exports increasing by 3.3\% over
the previous year to \$30 billion, although Bangladesh's trade deficit
ballooned by over 45\% in this same time period. Most export earnings
are from the garment-manufacturing industry. Bangladesh also has social
enterprises, including the Nobel Peace Prize-winning Grameen Bank and
BRAC (the world's largest non-governmental organisation).

However, an insufficient power supply is a significant obstacle to
Bangladesh's economic development. According to the World Bank, poor
governance, corruption and weak public institutions are also major
challenges. In April 2010, Standard \& Poor's gave Bangladesh a BB-
long-term credit rating, below India's but above those of Pakistan and
Sri Lanka.

The country is notable for its soil fertility land, including the Ganges
Delta, Sylhet Division and the Chittagong Hill Tracts. Agriculture is
the largest sector of the economy, making up 18.6 percent of
Bangladesh's GDP in November 2010 and employing about 45 percent of the
workforce. The agricultural sector impacts employment generation,
poverty alleviation, human resources development and food security. More
Bangladeshis earn their living from agriculture than from any other
sector. The country is among the top producers of rice (fourth),
potatoes (seventh), tropical fruits (sixth), jute (second), and farmed
fish (fifth).\\
Bangladesh is the seventh-largest natural gas producer in Asia, ahead of
neighboring Myanmar, and 56 percent of the country's electricity is
generated by natural gas. Major gas fields are located in the
northeastern (particularly Sylhet) and southern (including Barisal and
Chittagong) regions. Petrobangla is the national energy company. The
American multinational corporation Chevron produces 50 percent of
Bangladesh's natural gas. According to geologists, the Bay of Bengal
contains large, untapped gas reserves in Bangladesh's exclusive economic
zone. Bangladesh has substantial coal reserves, with several coal mines
operating in the northwest.\\
Jute exports remain significant, although the global jute trade has
shrunk considerably since its World War II peak. Bangladesh has one of
the world's oldest tea industries, and is a major exporter of fish and
seafood.

Bangladesh's textile and ready-made garment industries are the country's
largest manufacturing sector, with 2014 exports of \$25 billion.
Leather-goods manufacturing, particularly footwear, is the
second-largest export sector. The pharmaceutical industry meets 97
percent of domestic demand, and exports to many countries. Shipbuilding
has grown rapidly, with exports to Europe.

Steel is concentrated in the port city of Chittagong, and the ceramics
industry is prominent in international trade. In 2005 Bangladesh was the
world's 20th-largest cement producer, an industry dependent on limestone
imports from northeast India. Food processing is a major sector, with
local brands such as PRAN increasing their international market share.
The electronics industry is growing rapidly, particularly the Walton
Group. Bangladesh's defense industry includes the Bangladesh Ordnance
Factories and the Khulna Shipyard.

The service sector accounts for 51 percent of the country's GDP.
Bangladesh ranks with Pakistan as South Asia's second-largest banking
sector. The Dhaka and Chittagong Stock Exchanges are the country's twin
financial markets. Bangladesh's telecommunications industry is one of
the world's fastest-growing, with 114 million cellphone subscribers in
December 2013, and Grameenphone, Banglalink, Robi and BTTB are major
companies. Tourism is developing, with the beach resort of Cox's Bazar
the center of the industry. The Sylhet region, home to Bangladesh's tea
country, also hosts a large number of visitors. The country has three
UNESCO World Heritage Sites (the Mosque City, the Buddhist Vihara and
the Sundarbans) and five tentative-list sites.

Microfinance was pioneered in Bangladesh by Muhammad Yunus. In 2015, the
country had over 35~million microcredit borrowers.

\section{Transport}\label{transport}

\begin{itemize}
\item
  \emph{Bangladesh has three seaports and 22 river ports.}
\item
  \emph{Aviation has grown rapidly, and includes the flag carrier Biman
  Bangladesh Airlines and other privately owned airlines.}
\item
  \emph{Bangladesh has a 2,706-kilometre (1,681-mile) rail network
  operated by state-owned Bangladesh Railway.}
\item
  \emph{Bangladesh has a number of airports: three international and
  several domestic and STOL (short takeoff and landing) airports.}
\end{itemize}

Transport is a major sector of the economy. Aviation has grown rapidly,
and includes the flag carrier Biman Bangladesh Airlines and other
privately owned airlines. Bangladesh has a number of airports: three
international and several domestic and STOL (short takeoff and landing)
airports. The busiest, Shahjalal International Airport connects Dhaka
with major destinations.

Bangladesh has a 2,706-kilometre (1,681-mile) rail network operated by
state-owned Bangladesh Railway. The total length of the country's road
and highway network is nearly 21,000-kilometre (13,000-mile).

It has one of the largest inland waterway networks in the world, with
8,046 kilometres (5,000 miles) of navigable waters. The southeastern
port of Chittagong is its busiest seaport, handling over \$60 billion in
annual trade (more than 80 percent of the country's export-import
commerce). The second-busiest seaport is Mongla. Bangladesh has three
seaports and 22 river ports.

\section{Energy and infrastructure}\label{energy-and-infrastructure}

\begin{itemize}
\item
  \emph{Bangladesh had an installed electrical capacity of 10,289 MW in
  January 2014.}
\item
  \emph{Bangladesh has planned to import hydropower from Bhutan and
  Nepal.}
\end{itemize}

Bangladesh had an installed electrical capacity of 10,289 MW in January
2014. About 56 percent of the country's commercial energy is generated
by natural gas, followed by oil, hydropower and coal. Bangladesh has
planned to import hydropower from Bhutan and Nepal. Nuclear energy is
being developed with Russian support in the Ruppur Nuclear Power Plant
project. The country ranks fifth worldwide in the number of renewable
energy green jobs, and solar panels are increasingly used to power urban
and off-grid rural areas.

An estimated 98 percent of the country's population had access to
improved water sources in 2004 (a high percentage for a low-income
country), achieved largely through the construction of hand pumps with
support from external donors. However, in 1993 it was discovered that
much of Bangladesh's groundwater (the source of drinking water for 97
percent of the rural population and a significant share of the urban
population) is naturally contaminated with arsenic.

Another challenge is low cost recovery due to low tariffs and poor
economic efficiency, especially in urban areas (where water revenue does
not cover operating costs). An estimated 56 percent of the population
had access to adequate sanitation facilities in 2010. Community-led
total sanitation, addressing the problem of open defecation in rural
areas, is credited with improving public health since its introduction
in 2000.

\includegraphics[width=5.50000in,height=3.66667in]{media/image9.jpg}\\
\emph{In 2018, the first payload of SpaceX's Falcon 9 Block 5 rocket was
the Bangabandhu-1 satellite built by Thales Alenia Space}

\section{Science and technology}\label{science-and-technology}

\begin{itemize}
\item
  \emph{The Bangladesh Council of Scientific and Industrial Research,
  founded in 1973, traces its roots to the East Pakistan Regional
  Laboratories established in Dhaka (1955), Rajshahi (1965) and
  Chittagong (1967).}
\item
  \emph{The Bangladesh Atomic Energy Commission operates a TRIGA
  research reactor at its atomic-energy facility in Savar.}
\item
  \emph{In 2015, Bangladesh was ranked the 26th global IT outsourcing
  destination.}
\item
  \emph{Bangladesh's space agency, SPARRSO, was founded in 1983 with
  assistance from the United States.}
\end{itemize}

The Bangladesh Council of Scientific and Industrial Research, founded in
1973, traces its roots to the East Pakistan Regional Laboratories
established in Dhaka (1955), Rajshahi (1965) and Chittagong (1967).
Bangladesh's space agency, SPARRSO, was founded in 1983 with assistance
from the United States. The country's first communications satellite,
Bangabandhu-1, was launched from the United States in 2018. The
Bangladesh Atomic Energy Commission operates a TRIGA research reactor at
its atomic-energy facility in Savar. In 2015, Bangladesh was ranked the
26th global IT outsourcing destination.

\section{Tourism}\label{tourism}

\begin{itemize}
\item
  \emph{Bangladesh's world ranking in 2012 for travel and tourism's
  direct contribution to GDP, as a percentage of GDP, was 142 out of
  176.}
\item
  \emph{The World Travel and Tourism Council (WTTC) reported in 2013
  that the travel and tourism industry in Bangladesh directly generated
  1,281,500 jobs in 2012 or 1.8 percent of the country's total
  employment, which ranked Bangladesh 157 out of 178 countries
  worldwide.}
\end{itemize}

Bangladesh's tourist attractions include historical and monuments,
resorts, beaches, picnic spots, forests and tribal people, wildlife of
various species. Activities for tourists include angling, water skiing,
river cruising, hiking, rowing, yachting, and sea bathing.

The World Travel and Tourism Council (WTTC) reported in 2013 that the
travel and tourism industry in Bangladesh directly generated 1,281,500
jobs in 2012 or 1.8 percent of the country's total employment, which
ranked Bangladesh 157 out of 178 countries worldwide. Direct and
indirect employment in the industry totalled 2,714,500 jobs, or 3.7
percent of the country's total employment. The WTTC predicted that by
2023, travel and tourism will directly generate 1,785,000 jobs and
support an overall total of 3,891,000 jobs, or 4.2 percent of the
country's total employment. This would represent an annual growth rate
in direct jobs of 2.9 percent. Domestic spending generated 97.7 percent
of direct travel and tourism gross domestic product (GDP) in 2012.
Bangladesh's world ranking in 2012 for travel and tourism's direct
contribution to GDP, as a percentage of GDP, was 142 out of 176.

\section{Demographics}\label{demographics}

\begin{itemize}
\item
  \emph{Bangladesh is home to a significant Ismaili community.}
\item
  \emph{Rohingya refugees in Bangladesh number at around 1 million,
  making Bangladesh one of the countries with the largest refugee
  populations in the world.}
\item
  \emph{Bangladesh is the most densely-populated large country in the
  world, ranking 7th in population density when small countries and
  city-states are included.}
\item
  \emph{Bangladesh is the world's eighth-most-populous nation.}
\end{itemize}

Estimates of the Bangladeshi population vary, but UN data suggests
162,951,560 (162.9 million). The 2011 census estimated 142.3 million,
much less than 2007--2010 estimates of Bangladesh's population
(150--170~million). Bangladesh is the world's eighth-most-populous
nation. In 1951, its population was 44 million. Bangladesh is the most
densely-populated large country in the world, ranking 7th in population
density when small countries and city-states are included.

The country's population-growth rate was among the highest in the world
in the 1960s and 1970s, when its population grew from 65 to 110~million.
With the promotion of birth control in the 1980s, Bangladesh's growth
rate began to slow. Its total fertility rate is now 2.05, lower than
India's (2.58) and Pakistan's (3.07). The population is relatively
young, with 34 percent aged 15 or younger and five percent 65 or older.
Life expectancy at birth was estimated at 70 years in 2012. According to
the World Bank, as of 2016{[}update{]} 14.8\% of the country lives below
the international poverty line on less than \$1.90 per day.

Bengalis are 98 percent of the population. Of Bengalis, Muslims are the
majority, followed by Hindus, Christians and Buddhists.

The Adivasi population includes the Chakma, Marma, Tanchangya, Tripuri,
Kuki, Khiang, Khumi, Murang, Mru, Chak, Lushei, Bawm, Bishnupriya
Manipuri, Khasi, Jaintia, Garo, Santal, Munda and Oraon tribes. The
Chittagong Hill Tracts region experienced unrest and an insurgency from
1975 to 1997 in an autonomy movement by its indigenous people. Although
a peace accord was signed in 1997, the region remains militarized.

Bangladesh is home to a significant Ismaili community. It hosts many
Urdu-speaking immigrants, who migrated there after the partition of
India. Stranded Pakistanis were given citizenship by the Supreme Court
in 2008.

Rohingya refugees in Bangladesh number at around 1 million, making
Bangladesh one of the countries with the largest refugee populations in
the world.

\section{Urban centres}\label{urban-centres}

\begin{itemize}
\item
  \emph{Dhaka is Bangladesh's capital and largest city.}
\item
  \emph{Altogether there are 506 urban centres in Bangladesh among which
  43 cities have a population of more than 100000.}
\end{itemize}

Dhaka is Bangladesh's capital and largest city. There are 12 city
corporations which hold mayoral elections: Dhaka South, Dhaka North,
Chittagong, Comilla, Khulna, Mymensingh, Sylhet, Rajshahi, Barisal,
Rangpur, Gazipur and Narayanganj. Mayors are elected for five-year
terms. Altogether there are 506 urban centres in Bangladesh among which
43 cities have a population of more than 100000.

\section{Language}\label{language}

\begin{itemize}
\item
  \emph{Urdu has a significant heritage in Bangladesh.}
\item
  \emph{Today, the Bengali language is regulated by the Bangla Academy
  in Bangladesh.}
\item
  \emph{Urdu poets lived in many parts of Bangladesh.}
\item
  \emph{The predominant language of Bangladesh is Bengali (also known as
  Bangla).}
\item
  \emph{More than 98 percent of people in Bangladesh speak Bengali as
  their native language.}
\end{itemize}

The predominant language of Bangladesh is Bengali (also known as
Bangla). Bengali is the one of the easternmost branches of the
Indo-European language family. It is a part of the Eastern Indo-Aryan
languages in South Asia, which developed between the 10th and 13th
centuries. Bengali is written using the Bengali script. In ancient
Bengal, Sanskrit was the language of written communication, especially
by priests. During the Islamic period, Sanskrit was replaced by Bengali
as the vernacular language. The Sultans of Bengal promoted the
production of Bengali literature instead of Sanskrit. Bengali also
received Persian and Arabic loanwords during the Sultanate of Bengal.
Under British rule, Bengali was significantly modernized by Europeans.
Modern Standard Bengali emerged as the lingua franca of the region. A
heavily Sanskritized version of Bengali was employed by Hindu scholars
during the Bengali Renaissance. Muslim writers such as Kazi Nazrul Islam
gave attention to the Persian and Arabic vocabulary of the language.
Today, the Bengali language is regulated by the Bangla Academy in
Bangladesh. Bengali is a symbol of secular Bangladeshi identity. More
than 98 percent of people in Bangladesh speak Bengali as their native
language. Dialects of Bengali are spoken in some parts of the country,
which include non-standard dialects (sometimes viewed as separate
languages) such as Chatgaiya, Sylheti and Rangpuri. Bengali Language
Implementation Act, 1987 made it mandatory to use Bengali in all
government affairs in Bangladesh. Although laws were historically
written in English, they were not translated into Bengali until the
Bengali Language Implementation Act of 1987. All subsequent acts,
ordinances and laws have been promulgated in Bengali since 1987. English
is often used in the verdicts delivered by the Supreme Court of
Bangladesh, and is also used in higher education.

The Chakma language is another native Eastern Indo-Aryan language of
Bangladesh. It is written using the Chakma script. The unique aspect of
the language is that it is used by the Chakma people, who are a
population with similarities to the people of East Asia, rather than the
Indian subcontinent. The Chakma language is endangered due to its
decreasing use in schools and institutions.

Other tribal languages include Garo, Manipuri, Kokborok and Rakhine.
Among the Austroasiatic languages, the Santali language is spoken by the
Santali tribe. Many of these languages are written in the Bengali
script; while there is also some usage of the Latin script.

Urdu has a significant heritage in Bangladesh. The language was
introduced to Bengal in the 17th-century. Traders from North India often
spoke the language in Bengal, as did sections of the Bengali upper
class. Urdu poets lived in many parts of Bangladesh. The use of Urdu
became controversial during the Bengali Language Movement, when the
people of East Bengal resisted attempts to impose Urdu as the main
official language. In modern Bangladesh, the Urdu-speaking community is
today restricted to the country's Bihari community (formerly Stranded
Pakistanis); and some sections of the non-Bengali upper class. The
University of Dhaka operates a Department of Urdu.

\section{Religion}\label{religion}

\begin{itemize}
\item
  \emph{Bengali Christians are spread across the country; while there
  are many Christians among minority ethnic groups in the Chittagong
  Hill Tracts (southeastern Bangladesh) and within the Garo tribe of
  Mymensingh (north-central Bangladesh).}
\item
  \emph{Bangladesh has the fourth-largest Muslim population in the
  world, and is the third-largest Muslim-majority country (after
  Indonesia and Pakistan).}
\item
  \emph{In 1972, Bangladesh was South Asia's first
  constitutionally-secular country.}
\end{itemize}

Islam is the largest and the official state religion of Bangladesh,
followed by 90.4 percent of the population. Most Bangladeshis are
Bengali Muslims, who form the largest Muslim ethnoreligious group in
South Asia and the second largest in the world after the Arabs. There is
also a minority of non-Bengali Muslims. The vast majority of Bangladeshi
Muslims are Sunni, followed by minorities of Shia and Ahmadiya. About
four percent are non-denominational Muslims. Bangladesh has the
fourth-largest Muslim population in the world, and is the third-largest
Muslim-majority country (after Indonesia and Pakistan). Sufism has an
extensive heritage in the region. Most Bengali Muslims are influenced by
the liberal values of Sufism. Liberal Bengali Islam sometimes clashes
with orthodox movements. The largest gathering of Muslims in Bangladesh
is the apolitical Bishwa Ijtema, held annually by the orthodox Tablighi
Jamaat. The Ijtema is the second-largest Muslim congregation in the
world, after the Hajj. The Islamic Foundation is an autonomous
government agency responsible for some Muslim religious matters,
including sighting the moon in accordance with the lunar Islamic
calendar in order to set festival dates; as well as the charitable
tradition of zakat. Public holidays include the Islamic observances of
Eid-ul-Fitr, Eid-al-Adha, the Prophet's Birthday, Ashura and
Shab-e-Barat.

Hinduism is followed by 8.5 percent of the population; most are Bengali
Hindus, and some are members of ethnic minority groups. Bangladeshi
Hindus are the country's second-largest religious group and the
third-largest Hindu community in the world, after those in India and
Nepal. Hindus in Bangladesh are evenly distributed, with concentrations
in Gopalganj, Dinajpur, Sylhet, Sunamganj, Mymensingh, Khulna, Jessore,
Chittagong and parts of the Chittagong Hill Tracts. Hindus are the
second largest religious community in Bangladeshi cities. The festivals
of Durga's Return and Krishna's Birthday are public holidays.

Buddhism is the third-largest religion, at 0.6 percent. Bangladeshi
Buddhists are concentrated among ethnic groups in the Chittagong Hill
Tracts (particularly the Chakma, Marma and Tanchangya peoples), and
coastal Chittagong is home to a large number of Bengali Buddhists. While
the Mahayana school of Buddhism was historically prevalent in the
region, Bangladeshi Buddhists today adhere to the Theravada school.
Buddha's Birthday is a public holiday. The chief Buddhist priests are
based at a monastery in Chittagong.

Christianity is the fourth-largest religion, at 0.4 percent. Roman
Catholicism is the largest denomination among Bangladeshi Christians.
Bengali Christians are spread across the country; while there are many
Christians among minority ethnic groups in the Chittagong Hill Tracts
(southeastern Bangladesh) and within the Garo tribe of Mymensingh
(north-central Bangladesh). The country also has Protestant, Baptist and
Oriental Orthodox churches. Christmas is a public holiday.

The Constitution of Bangladesh declares Islam the state religion, but
bans religion-based politics. It proclaims equal recognition of Hindus,
Buddhists, Christians and people of all faiths. In 1972, Bangladesh was
South Asia's first constitutionally-secular country. Article 12 of the
constitution continues to call for secularism, the elimination of
interfaith tensions and prohibits the abuse of religion for political
purposes and any discrimination against, or persecution of, persons
practicing a particular religion. Article 41 of the constitution
subjects religious freedom to public order, law and morality; it gives
every citizen the right to profess, practice or propagate any religion;
every religious community or denomination the right to establish,
maintain and manage its religious institutions; and states that no
person attending any educational institution shall be required to
receive religious instruction, or to take part in or to attend any
religious ceremony or worship, if that instruction, ceremony or worship
relates to a religion other than his own.

\section{Education}\label{education}

\begin{itemize}
\item
  \emph{Bangladesh has 34 public, 64 private and two international
  universities; Bangladesh National University has the largest
  enrollment, and the University of Dhaka (established in 1921) is the
  oldest.University of Chittagong (established in 1966) is the largest
  University (Campus: Rural, 2,100 acres (8.5 km2)) .}
\item
  \emph{Bangladesh has a literacy rate of 72.9 percent as of 2018.}
\item
  \emph{BUET, CUET, KUET and RUET are Bangladesh's four public
  engineering universities.}
\item
  \emph{Bangladesh conforms with the Education For All (EFA) objectives,
  the Millennium Development Goals (MDG) and international
  declarations.}
\end{itemize}

Bangladesh has a literacy rate of 72.9 percent as of 2018. 75.7\%
percent for males and 70.09\% percent for females. The country's
educational system is three-tiered and heavily subsidized, with the
government operating many schools at the primary, secondary and
higher-secondary levels and subsidizing many private schools. In the
tertiary-education sector, the Bangladeshi government funds over 15
state universities through the University Grants Commission.

The education system is divided into five levels: primary (first to
fifth grade), junior secondary (sixth to eighth grade), secondary (ninth
and tenth grade), higher secondary (11th and 12th grade) and tertiary.
Five years of secondary education end with a Secondary School
Certificate (SSC) examination; since 2009, the Primary Education Closing
(PEC) examination has also been given. Students who pass the PEC
examination proceed to four years of secondary or matriculation
training, culminating in the SSC examination.

Students who pass the PEC examination proceed to three years of
junior-secondary education, culminating in the Junior School Certificate
(JSC) examination. Students who pass this examination proceed to two
years of secondary education, culminating in the SSC examination.
Students who pass this examination proceed to two years of
higher-secondary education, culminating in the Higher Secondary School
Certificate (HSC) examination.

Education is primarily in Bengali, but English is commonly taught and
used. Many Muslim families send their children to part-time courses or
full-time religious education in Bengali and Arabic in madrasas.

Bangladesh conforms with the Education For All (EFA) objectives, the
Millennium Development Goals (MDG) and international declarations.
Article 17 of the Bangladesh Constitution provides that all children
between the ages of six and ten years receive a basic education free of
charge.

Universities in Bangladesh are of three general types: public
(government-owned and -subsidized), private (privately owned
universities) and international (operated and funded by international
organizations). Bangladesh has 34 public, 64 private and two
international universities; Bangladesh National University has the
largest enrollment, and the University of Dhaka (established in 1921) is
the oldest.University of Chittagong (established in 1966) is the largest
University (Campus: Rural, 2,100 acres (8.5 km2)) . Islamic University
of Technology, commonly known as IUT, is a subsidiary of the
Organisation of the Islamic Cooperation (OIC, representing 57 countries
in Asia, Africa, Europe and South America). Asian University for Women
in Chittagong is the preeminent South Asian liberal-arts university for
women, representing 14 Asian countries; its faculty hails from notable
academic institutions in North America, Europe, Asia, Australia and the
Middle East. BUET, CUET, KUET and RUET are Bangladesh's four public
engineering universities. BUTex and DUET are two specialized engineering
universities; BUTex specializes in textile engineering, and DUET offers
higher education to diploma engineers. The NITER is a specialized
public-private partnership institute which provides higher education in
textile engineering. Science and technology universities include SUST,
PUST, JUST and NSTU. Bangladeshi universities are accredited by and
affiliated with the University Grants Commission (UGC), created by
Presidential Order 10 in 1973.

Medical education is provided by 29 government and private medical
colleges. All medical colleges are affiliated with the Ministry of
Health and Family Welfare.

Bangladesh's 2015 literacy rate rose to 71 percent due to education
modernization and improved funding, with 16,087 schools and 2,363
colleges receiving Monthly Pay Order (MPO) facilities. According to
education minister Nurul Islam Nahid, 27,558 madrasas and technical and
vocational institutions were enlisted for the facility. 6,036
educational institutions were outside MPO coverage, and the government
enlisted 1,624 private schools for MPO in 2010.

\section{Health}\label{health}

\begin{itemize}
\item
  \emph{Malnutrition has been a persistent problem in Bangladesh, with
  the World Bank ranking the country first in the number of malnourished
  children worldwide.}
\item
  \emph{A health watch, a pilot community-empowerment tool, was
  successfully developed and implemented in south-eastern Bangladesh to
  improve the uptake and monitoring of public-health services.}
\item
  \emph{Bangladesh's poor health conditions are attributed to the lack
  of healthcare provision by the government.}
\end{itemize}

Health and education levels remain relatively low, although they have
improved as poverty levels have decreased significantly. In rural areas,
village doctors with little or no formal training constitute 62 percent
of healthcare providers practising "modern medicine"; formally-trained
providers make up four percent of the total health workforce. A Future
Health Systems survey indicated significant deficiencies in the
treatment practices of village doctors, with widespread harmful and
inappropriate drug prescribing. Receiving health care from informal
providers is encouraged.

A 2007 study of 1,000 households in rural Bangladesh found that direct
payments to formal and informal healthcare providers and indirect costs
(loss of earnings because of illness) associated with illness were
deterrents to accessing healthcare from qualified providers. A community
survey of 6,183 individuals in rural Bangladesh found a gender
difference in treatment-seeking behaviour, with women less likely to
seek treatment than to men. The use of skilled birth attendant (SBA)
services, however, rose from 2005 to 2007 among women from all
socioeconomic quintiles except the highest. A health watch, a pilot
community-empowerment tool, was successfully developed and implemented
in south-eastern Bangladesh to improve the uptake and monitoring of
public-health services.

Bangladesh's poor health conditions are attributed to the lack of
healthcare provision by the government. According to a 2010 World Bank
report, 2009 healthcare spending was 3.35 percent of the country's GDP.
The number of hospital beds is 3 per 10,000 population. Government
spending on healthcare that year was 7.9 percent of the total budget;
out-of-pocket expenditures totaled 96.5 percent.

Malnutrition has been a persistent problem in Bangladesh, with the World
Bank ranking the country first in the number of malnourished children
worldwide. Twenty-six percent of the population (two-thirds of children
under the age of five) are undernourished, and 46 percent of children
are moderately or severely underweight. Forty-three to 60 percent of
children under five are smaller than normal; one in five preschool
children are vitamin-A deficient, and one in two are anemic. More than
45 percent of rural families and 76 percent of urban families were below
the acceptable caloric-intake level.

\section{Culture}\label{culture}

\section{Visual arts}\label{visual-arts}

\begin{itemize}
\item
  \emph{The recorded history of art in Bangladesh can be traced to the
  3rd century BCE, when terracotta sculptures were made in the region.}
\item
  \emph{East Bengal developed its own modernist painting and sculpture
  traditions, which were distinct from the art movements in West
  Bengal.}
\item
  \emph{The modern art movement in Bangladesh took shape during the
  1950s, particularly with the pioneering works of Zainul Abedin.}
\end{itemize}

The recorded history of art in Bangladesh can be traced to the 3rd
century BCE, when terracotta sculptures were made in the region. In
classical antiquity, a notable school of sculptural Hindu, Jain and
Buddhist art developed in the Pala Empire and the Sena dynasty. Islamic
art evolved since the 14th century. The architecture of the Bengal
Sultanate saw a distinct style of domed mosques with complex niche
pillars that had no minarets. Mughal Bengal's most celebrated artistic
tradition was the weaving of Jamdani motifs on fine muslin, which is now
classified by UNESCO as an intangible cultural heritage. Jamdani motifs
were similar to Iranian textile art (buta motifs) and Western textile
art (paisley). The Jamdani weavers in Dhaka received imperial patronage.
Ivory and brass were also widely used in Mughal art. Pottery is widely
used in Bengali culture.

The modern art movement in Bangladesh took shape during the 1950s,
particularly with the pioneering works of Zainul Abedin. East Bengal
developed its own modernist painting and sculpture traditions, which
were distinct from the art movements in West Bengal. The Art Institute
Dhaka has been an important center for visual art in the region. Its
annual Bengali New Year parade was enlisted as an intangible cultural
heritage by UNESCO in 2016.

Modern Bangladesh has produced many of South Asia's leading painters,
including SM Sultan, Mohammad Kibria, Shahabuddin Ahmed, Kanak Chanpa
Chakma, Kafil Ahmed, Saifuddin Ahmed, Qayyum Chowdhury, Rashid
Choudhury, Quamrul Hassan, Rafiqun Nabi and Syed Jahangir, among others.
Novera Ahmed and Nitun Kundu were the country's pioneers of modernist
sculpture.

The Chobi Mela is the largest photography festival in Asia.

\includegraphics[width=5.50000in,height=4.10038in]{media/image10.png}\\
\emph{Rabindranath Tagore, author of the national anthem, and Kazi
Nazrul Islam, the National Poet}

\section{Literature}\label{literature}

\begin{itemize}
\item
  \emph{During the Bengal Sultanate, medieval Bengali writers were
  influenced by Arabic and Persian works.}
\item
  \emph{The oldest evidence of writing in Bangladesh is the Mahasthan
  Brahmi Inscription, which dates back to the 3rd century BCE.}
\item
  \emph{Shamsur Rahman was the poet laureate of Bangladesh for many
  years.}
\item
  \emph{Begum Rokeya is regarded as the pioneer feminist writer of
  Bangladesh.}
\end{itemize}

The oldest evidence of writing in Bangladesh is the Mahasthan Brahmi
Inscription, which dates back to the 3rd century BCE. In the Gupta
Empire, Sanskrit literature thrived in the region. Bengali developed
from Sanskrit and Magadhi Prakrit in the from the 8th to 10th century.
Bengali literature is a millennium-old tradition; the Charyapadas are
the earliest examples of Bengali poetry. Sufi spiritualism inspired many
Bengali Muslim writers. During the Bengal Sultanate, medieval Bengali
writers were influenced by Arabic and Persian works. The Chandidas are
the notable lyric poets from the early Medieval Age. Syed Alaol was a
noted secular poet and translator from the Arakan region. The Bengal
Renaissance shaped the emergence of modern Bengali literature, including
novels, short stories and science fiction. Rabindranath Tagore was the
first non-European laureate of the Nobel Prize in Literature and is
described as the Bengali Shakespeare. Kazi Nazrul Islam was a
revolutionary poet who espoused political rebellion against colonialism
and fascism. Begum Rokeya is regarded as the pioneer feminist writer of
Bangladesh. Other renaissance icons included Michael Madhusudan Dutt and
Sarat Chandra Chattopadhyay.\\
The writer Syed Mujtaba Ali is noted for his cosmopolitan Bengali
worldview. Jasimuddin was a renowned pastoral poet. Shamsur Rahman was
the poet laureate of Bangladesh for many years. Al Mahmud is considered
one of the greatest Bengali poets to have emerged in the 20th century.
Farrukh Ahmed, Sufia Kamal, and Nirmalendu Goon are important figures of
modern Bangladeshi poetry. Ahmed Sofa is regarded as the most important
Bangladeshi intellectual in the post-independence era. Humayun Ahmed was
a popular writer of modern Bangladeshi magical realism and science
fiction. Notable writers of Bangladeshi fictions include Mir Mosharraf
Hossain, Akhteruzzaman Elias, Syed Waliullah, Shahidullah Kaiser,
Shawkat Osman, Selina Hossain, Taslima Nasreen, Haripada Datta, Razia
Khan, Anisul Hoque, and Bipradash Barua. Many Bangladeshi writers, such
as Muhammad Zafar Iqbal, and Farah Ghuznavi are acclaimed for their
short stories.

The annual Ekushey Book Fair and Dhaka Literature Festival, organized by
the Bangla Academy, are among the largest literary festivals in South
Asia.

\section{Women in Bangladesh}\label{women-in-bangladesh}

\begin{itemize}
\item
  \emph{Whereas in India and Pakistan women participate less in the
  workforce as their education increases, the reverse is the case in
  Bangladesh.}
\item
  \emph{Although, as of 2015{[}update{]}, several women occupied major
  political office in Bangladesh, its women continue to live under a
  patriarchal social regime where violence is common.}
\end{itemize}

Although, as of 2015{[}update{]}, several women occupied major political
office in Bangladesh, its women continue to live under a patriarchal
social regime where violence is common. Whereas in India and Pakistan
women participate less in the workforce as their education increases,
the reverse is the case in Bangladesh.

Bengal has a long history of feminist activism dating back to the 19th
century. Begum Rokeya and Faizunnessa Chowdhurani played an important
role in emancipating Bengali Muslim women from purdah, prior to the
country's division, as well as promoting girls' education. Several women
were elected to the Bengal Legislative Assembly in the British Raj. The
first women's magazine, Begum, was published in 1948.

In 2008, Bangladeshi female workforce participation stood at 26\%. Women
dominate blue collar jobs in the Bangladeshi garment industry.
Agriculture, social services, healthcare and education are also major
occupations for Bangladeshi women, while their employment in white
collar positions has steadily increased.

\section{Architecture}\label{architecture}

\begin{itemize}
\item
  \emph{The zamindar gentry in Bangladesh built numerous Indo-Saracenic
  palaces and country mansions, such as the Ahsan Manzil, Tajhat Palace,
  Dighapatia Palace, Puthia Rajbari and Natore Rajbari.}
\item
  \emph{The architectural traditions of Bangladesh have a 2,500-year-old
  heritage.}
\item
  \emph{The Sixty Dome Mosque was the largest medieval mosque built in
  Bangladesh, and is a fine example of Turkic-Bengali architecture.}
\end{itemize}

The architectural traditions of Bangladesh have a 2,500-year-old
heritage. Terracotta architecture is a distinct feature of Bengal.
Pre-Islamic Bengali architecture reached its pinnacle in the Pala
Empire, when the Pala School of Sculptural Art established grand
structures such as the Somapura Mahavihara. Islamic architecture began
developing under the Bengal Sultanate, when local terracotta styles
influenced medieval mosque construction. The Adina Mosque of united
Bengal was the largest mosque built on the Indian subcontinent.

The Sixty Dome Mosque was the largest medieval mosque built in
Bangladesh, and is a fine example of Turkic-Bengali architecture. The
Mughal style replaced indigenous architecture when Bengal became a
province of the Mughal Empire and influenced the development of urban
housing. The Kantajew Temple and Dhakeshwari Temple are excellent
examples of late medieval Hindu temple architecture. Indo-Saracenic
Revival architecture, based on Indo-Islamic styles, flourished during
the British period. The zamindar gentry in Bangladesh built numerous
Indo-Saracenic palaces and country mansions, such as the Ahsan Manzil,
Tajhat Palace, Dighapatia Palace, Puthia Rajbari and Natore Rajbari.

Bengali vernacular architecture is noted for pioneering the bungalow.
Bangladeshi villages consist of thatched roofed houses made of natural
materials like mud, straw, wood and bamboo. In modern times, village
bungalows are increasingly made of tin.

Muzharul Islam was the pioneer of Bangladeshi modern architecture. His
varied works set the course of modern architectural practice in the
country. Islam brought leading global architects, including Louis Kahn,
Richard Neutra, Stanley Tigerman, Paul Rudolph, Robert Boughey and
Konstantinos Doxiadis, to work in erstwhile East Pakistan. Louis Kahn
was chosen to design the National Parliament Complex in Sher-e-Bangla
Nagar. Kahn's monumental designs, combining regional red brick
aesthetics, his own concrete and marble brutalism and the use of lakes
to represent Bengali geography, are regarded as one of the masterpieces
of the 20th century. In more recent times, award-winning architects like
Rafiq Azam have set the course of contemporary architecture by adopting
influences from the works of Islam and Kahn.

\section{Performing arts}\label{performing-arts}

\begin{itemize}
\item
  \emph{The music of Bangladesh features the Baul mystical tradition,
  listed by UNESCO as a Masterpiece of Intangible Cultural Heritage.}
\item
  \emph{Theatre in Bangladesh includes various forms with a history
  dating back to the 4th century CE.}
\item
  \emph{Musician Ayub Bachchu is credited with popularizing Bengali rock
  music in Bangladesh.}
\end{itemize}

Theatre in Bangladesh includes various forms with a history dating back
to the 4th century CE. It includes narrative forms, song and dance
forms, supra-personae forms, performances with scroll paintings, puppet
theatre and processional forms. The Jatra is the most popular form of
Bengali folk theatre.\\
The dance traditions of Bangladesh include indigenous tribal and Bengali
dance forms, as well as classical Indian dances, including the Kathak,
Odissi and Manipuri dances.

The music of Bangladesh features the Baul mystical tradition, listed by
UNESCO as a Masterpiece of Intangible Cultural Heritage. Numerous
lyric-based musical traditions, varying from one region to the next,
exist, including Gombhira, Bhatiali and Bhawaiya. Folk music is
accompanied by a one-stringed instrument known as the ektara. Other
instruments include the dotara, dhol, flute, and tabla. Bengali
classical music includes Tagore songs and Nazrul geeti. Bangladesh has a
rich tradition of Indian classical music, which uses instruments like
the sitar, tabla, sarod and santoor. Musician Ayub Bachchu is credited
with popularizing Bengali rock music in Bangladesh.

\includegraphics[width=5.50000in,height=4.07786in]{media/image11.jpg}\\
\emph{A woman wearing jamdani in 1787. Bengal has manufactured textiles
for many centuries, as recorded in ancient hand-written and printed
documents}

\includegraphics[width=5.50000in,height=4.05206in]{media/image12.jpg}\\
\emph{Embroidery on Nakshi kantha (embroidered quilt), centuries-old
Bengali art tradition}

\section{Textiles}\label{textiles}

\begin{itemize}
\item
  \emph{Bangladesh also produces the Rajshahi silk.}
\item
  \emph{Bangladesh).}
\item
  \emph{Bangladesh is the world's second largest garments exporter.}
\end{itemize}

The Nakshi Kantha is a centuries-old embroidery tradition for quilts,
said to be indigenous to eastern Bengal (i.e. Bangladesh). The sari is
the national dress for Bangladeshi women. Mughal Dhaka was renowned for
producing the finest Muslin saris, including the famed Dhakai and
Jamdani, the weaving of which is listed by UNESCO as one of the
masterpieces of humanity's intangible cultural heritage. Bangladesh also
produces the Rajshahi silk. The shalwar kameez is also widely worn by
Bangladeshi women. In urban areas some women can be seen in western
clothing. The kurta and sherwani are the national dress of Bangladeshi
men; the lungi and dhoti are worn by them in informal settings. Aside
from ethnic wear, domestically tailored suits and neckties are
customarily worn by the country's men in offices, in schools and at
social events.

The handloom industry supplies 60--65\% of the country's clothing
demand. The Bengali ethnic fashion industry has flourished in the
changing environment of the fashion world. The retailer Aarong is one of
the most successful ethnic wear brands in South Asia. The development of
the Bangladesh textile industry, which supplies leading international
brands, has promoted the production and retail of modern Western attire
locally, with the country now having a number of expanding local brands
like Westecs and Yellow. Bangladesh is the world's second largest
garments exporter.

Among Bangladesh's fashion designers, Bibi Russell has received
international acclaim for her "Fashion for Development" shows.

\section{Cuisine}\label{cuisine}

\begin{itemize}
\item
  \emph{Bangladesh shares its culinary heritage with the neighboring
  Indian state of West Bengal.}
\item
  \emph{In Muslim-majority Bangladesh, meat consumption is greater;
  whereas in Hindu-majority West Bengal, vegetarianism is more
  prevalent.}
\item
  \emph{The Hilsa is the national fish and immensely popular across
  Bangladesh.}
\item
  \emph{Kebabs are widely popular across Bangladesh, particularly seekh
  kebabs, chicken tikka and shashliks.}
\end{itemize}

White rice is the staple of Bangladeshi cuisine, along with many
vegetables and lentils. Rice preparations also include Bengali biryanis,
pulaos, and khichuris. Mustard sauce, ghee, sunflower oil and fruit
chutneys are widely used in Bangladeshi cooking. Fish is the main source
of protein in Bengali cuisine. The Hilsa is the national fish and
immensely popular across Bangladesh. Other kinds of fish eaten include
rohu, butterfish, catfish, tilapia and barramundi. Fish eggs are a
gourmet delicacy. Seafood holds an important place in Bengali cuisine,
especially lobsters, shrimps and dried fish. Meat consumption includes
chicken, beef, mutton, venison, duck and squab. In Chittagong, Mezban
feasts are a popular tradition featuring the serving of hot beef curry.
In Sylhet, the shatkora lemons are used to marinate dishes. In the
tribal Hill Tracts, bamboo shoot cooking is prevalent. Bangladesh has a
vast spread of desserts, including distinctive sweets like Rôshogolla,
Rôshomalai, Chomchom, Mishti Doi and Kalojaam. Pithas are traditional
boiled desserts made with rice or fruits. Halwa is served during
religious festivities. Naan, paratha, luchi and bakarkhani are the main
local breads. Black tea is offered to guests as a gesture of welcome.
Kebabs are widely popular across Bangladesh, particularly seekh kebabs,
chicken tikka and shashliks.

Bangladesh shares its culinary heritage with the neighboring Indian
state of West Bengal. The two regions have several differences, however.
In Muslim-majority Bangladesh, meat consumption is greater; whereas in
Hindu-majority West Bengal, vegetarianism is more prevalent. The
Bangladeshi diaspora dominates the South Asian restaurant industry in
many Western countries, particularly in the United Kingdom.

\section{Festivals}\label{festivals}

\begin{itemize}
\item
  \emph{On Language Movement Day, people congregate at the Shaheed Minar
  in Dhaka to remember the national heroes of the Bengali Language
  Movement, and at the Jatiyo Smriti Soudho on Independence Day and
  Victory Day to remember the national heroes of the Bangladesh
  Liberation War.}
\item
  \emph{Of the major holidays celebrated in Bangladesh, only Pohela
  Boishakh comes without any preexisting expectations (specific
  religious identity, culture of gift-giving, etc.).}
\end{itemize}

Pohela Boishakh, the Bengali new year, is the major festival of Bengali
culture and sees widespread festivities. Of the major holidays
celebrated in Bangladesh, only Pohela Boishakh comes without any
preexisting expectations (specific religious identity, culture of
gift-giving, etc.). Unlike holidays like Eid al-Fitr, where dressing up
in lavish clothes has become a norm, or Christmas where exchanging gifts
has become an integral part of the holiday, Pohela Boishakh is really
about celebrating the simpler, rural roots of the Bengal. As a result,
more people can participate in the festivities together without the
burden of having to reveal one's class, religion, or financial capacity.
Other cultural festivals include Nabonno, and Poush Parbon both of which
are Bengali harvest festivals.

The Muslim festivals of Eid al-Fitr, Eid al-Adha, Milad un Nabi,
Muharram, Chand Raat, Shab-e-Barat; the Hindu festivals of Durga Puja,
Janmashtami and Rath Yatra; the Buddhist festival of Buddha Purnima,
which marks the birth of Gautama Buddha, and Christian festival of
Christmas are national holidays in Bangladesh and see the most
widespread celebrations in the country.

Alongside are national days like the remembrance of 21 February 1952
Language Movement Day (International Mother Language Day), Independence
Day and\\
Victory Day. On Language Movement Day, people congregate at the Shaheed
Minar in Dhaka to remember the national heroes of the Bengali Language
Movement, and at the Jatiyo Smriti Soudho on Independence Day and
Victory Day to remember the national heroes of the Bangladesh Liberation
War. These occasions are observed with public ceremonies, parades,
rallies by citizens, political speeches, fairs, concerts, and various
other public and private events, celebrating the history and traditions
of Bangladesh. TV and radio stations broadcast special programs and
patriotic songs, and many schools and colleges organise fairs,
festivals, and concerts that draw the participation of citizens from all
levels of Bangladeshi society.{[}citation needed{]}

\section{Sports}\label{sports}

\begin{itemize}
\item
  \emph{Women's sports saw tremendous progress in the 2010s decade in
  Bangladesh.}
\item
  \emph{Bangladesh hosted the Asia Cup on four occasions in 2000, 2012,
  2014, and 2016.}
\item
  \emph{Bangladesh has five grandmasters in chess.}
\item
  \emph{Six of Bangladesh's ten test match victories came in between the
  years 2014 to 2017.}
\item
  \emph{Cricket is one of the most popular sports in Bangladesh,
  followed by football.}
\end{itemize}

Cricket is one of the most popular sports in Bangladesh, followed by
football. The national cricket team participated in their first Cricket
World Cup in 1999, and the following year was granted elite Test cricket
status. They have however struggled, recording only ten test match
victories: one against Australia, one against England, one against Sri
Lanka in Sri Lanka, five against Zimbabwe (one in 2005, one in 2013 in
Zimbabwe, and three in 2014), two in a 2--0 series victory over the West
Indies in the West Indies in 2009. Six of Bangladesh's ten test match
victories came in between the years 2014 to 2017.

The team has been more successful in One Day International cricket
(ODI). They reached the quarter-final of the 2015 Cricket World Cup.
They also reached the semi-final of the 2017 ICC Champions Trophy. They
whitewashed Pakistan in a home ODI series in 2015 followed by home ODI
series wins against India and South Africa. They also won home ODI
series by 4--0 in 2010 against New Zealand and whitewashed them in the
home ODI series in 2013. In July 2010, they celebrated their first-ever
win over England in England. In late 2012, they won a five-match home
ODI series 3-2 against a full-strength West Indies National team. In
2011, Bangladesh successfully co-hosted the ICC Cricket World Cup 2011
with India and Sri Lanka. They also hosted the 2014 ICC World Twenty20
championship. Bangladesh hosted the Asia Cup on four occasions in 2000,
2012, 2014, and 2016. In 2012 Asia Cup, Bangladesh beat India and Sri
Lanka but lost the final game against Pakistan. However, it was the
first time Bangladesh had advanced to the final of any top-class
international cricket tournament. They reached the final again at the
2016 Asia Cup and 2018 Asia Cup. They participated at the 2010 Asian
Games in Guangzhou, defeating Afghanistan to claim their Gold Medal in
the first-ever cricket tournament held in the Asian Games. Bangladeshi
cricketer Sakib Al Hasan is No.1 on the ICC's all-rounder rankings in
all three formats of the cricket.

Women's sports saw tremendous progress in the 2010s decade in
Bangladesh. In 2018 the Bangladesh women's national cricket team the
2018 Women's Twenty20 Asia Cup defeating India women's national cricket
team in the final.

Kabaddi---very popular in Bangladesh---is the national game. Other
popular sports include field hockey, tennis, badminton, handball,
football, chess, shooting, angling. The National Sports Council
regulates 42 different sporting federations. On 4 November 2018,
Bangladesh national under-15 football team won the 2018 SAFF U-15
Championship, defeating Pakistan national under-15 football team in the
final. Bangladesh has five grandmasters in chess. Among them, Niaz
Murshed was the first grandmaster in South Asia. In another achievement,
Margarita Mamun, a Russian rhythmic gymnast of Bangladeshi origin, won
gold medal in 2016 Summer Olympics and became world champion in the
years 2013 and 2014.

\includegraphics[width=5.50000in,height=3.75833in]{media/image13.jpg}\\
\emph{Anwar Hossain playing Siraj-ud-Daulah, the last independent Nawab
of Bengal, in the 1967 film Nawab Sirajuddaulah}

\section{Media and cinema}\label{media-and-cinema}

\begin{itemize}
\item
  \emph{Bangladesh Television (BTV) is the state-owned television
  network.}
\item
  \emph{Bangladesh Betar is the state-run radio service.}
\item
  \emph{Bangladesh have very active film society culture.}
\item
  \emph{By the 2000s, Bangladesh produced 80--100 films a year.}
\item
  \emph{Federation of Film Societies of Bangladesh is the parent
  organization of the film society movement of Bangladesh.}
\end{itemize}

The Bangladeshi press is diverse, outspoken and privately owned. Over
200 newspapers are published in the country. Bangladesh Betar is the
state-run radio service. The British Broadcasting Corporation operates
the popular BBC Bangla news and current affairs service. Bengali
broadcasts from Voice of America are also very popular. Bangladesh
Television (BTV) is the state-owned television network. There more than
20 privately owned television networks, including several news channels.
Freedom of the media remains a major concern, due to government attempts
at censorship and the harassment of journalists.

The cinema of Bangladesh dates back to 1898, when films began screening
at the Crown Theatre in Dhaka. The first bioscope on the subcontinent
was established in Dhaka that year. The Dhaka Nawab Family patronized
the production of several silent films in the 1920s and 30s. In 1931,
the East Bengal Cinematograph Society released the first full-length
feature film in Bangladesh, titled the Last Kiss. The first feature film
in East Pakistan, Mukh O Mukhosh, was released in 1956. During the
1960s, 25--30 films were produced annually in Dhaka. By the 2000s,
Bangladesh produced 80--100 films a year. While the Bangladeshi film
industry has achieved limited commercial success, the country has
produced notable independent filmmakers. Zahir Raihan was a prominent
documentary-maker who was assassinated in 1971. The late Tareque Masud
is regarded as one of Bangladesh's outstanding directors due to his
numerous productions on historical and social issues. Masud was honored
by FIPRESCI at the 2002 Cannes Film Festival for his film The Clay Bird.
Tanvir Mokammel, Mostofa Sarwar Farooki, Humayun Ahmed, Alamgir Kabir,
and Chashi Nazrul Islam are some of the prominent directors of
Bangladeshi cinema. Bangladesh have very active film society culture.
its started in 1963 at Dhaka. Now around 40 Film Society active in all
over Bangladesh. Federation of Film Societies of Bangladesh is the
parent organization of the film society movement of Bangladesh. Active
film societies include the Rainbow Film Society, Children's Film
Society, Moviyana Film Society \& Dhaka University Film Society.

\section{Museums and libraries}\label{museums-and-libraries}

\begin{itemize}
\item
  \emph{The Ethnological Museum of Chittagong showcases the lifestyle of
  various tribes in Bangladesh.}
\item
  \emph{The Varendra Research Museum is the oldest museum in
  Bangladesh.}
\item
  \emph{The Bangladesh National Museum is located in Ramna, Dhaka and
  has a rich collection of antiquities.}
\item
  \emph{The National Library of Bangladesh was established in 1972.}
\end{itemize}

The Varendra Research Museum is the oldest museum in Bangladesh. It
houses important collections from both the pre-Islamic and Islamic
periods, including the sculptures of the Pala-Sena School of Art and the
Indus Valley Civilization; as well as Sanskrit, Arabic and Persian
manuscripts and inscriptions. The Ahsan Manzil, the former residence of
the Nawab of Dhaka, is a national museum housing collections from the
British Raj. It was the site of the founding conference of the All India
Muslim League and hosted many British Viceroys in Dhaka.

The Tajhat Palace Museum preserves artifacts of the rich cultural
heritage of North Bengal, including Hindu-Buddhist sculptures and
Islamic manuscripts. The Mymensingh Museum houses the personal antique
collections of Bengali aristocrats in central Bengal. The Ethnological
Museum of Chittagong showcases the lifestyle of various tribes in
Bangladesh. The Bangladesh National Museum is located in Ramna, Dhaka
and has a rich collection of antiquities. The Liberation War Museum
documents the Bangladeshi struggle for independence and the 1971
genocide.

In ancient times, manuscripts were written on palm leaves, tree barks,
parchment vellum and terracotta plates and preserved at monasteries
known as viharas. The Hussain Shahi dynasty established royal libraries
during the Bengal Sultanate. Libraries were established in each district
of Bengal by the zamindar gentry during the Bengal Renaissance in the
19th century. The trend of establishing libraries continued until the
beginning of World War II. In 1854, four major public libraries were
opened, including the Bogra Woodburn Library, the Rangpur Public
Library, the Jessore Institute Public Library and the Barisal Public
Library.

The Northbrook Hall Public Library was established in Dhaka in 1882 in
honour of Lord Northbrook, the Governor-General. Other libraries
established in the British period included the Victoria Public Library,
Natore (1901), the Sirajganj Public Library (1882), the Rajshahi Public
Library (1884), the Comilla Birchandra Library (1885), the Shah Makhdum
Institute Public Library, Rajshahi (1891), the Noakhali Town Hall Public
Library (1896), the Prize Memorial Library, Sylhet (1897), the
Chittagong Municipality Public Library (1904) and the Varendra Research
Library (1910). The Great Bengal Library Association was formed in 1925.
The Central Public Library of Dhaka was established in 1959. The
National Library of Bangladesh was established in 1972. The World
Literature Center, founded by Ramon Magsaysay Award winner Abdullah Abu
Sayeed, is noted for operating numerous mobile libraries across
Bangladesh and was awarded the UNESCO Jon Amos Comenius Medal.

\section{See also}\label{see-also}

\begin{itemize}
\item
  \emph{Outline of Bangladesh}
\item
  \emph{List of Bangladesh Test cricketers}
\end{itemize}

Index of Bangladesh-related articles

Outline of Bangladesh

List of Bangladeshi people

List of Bangladeshi Americans

List of Bengalis

List of British Bangladeshis

List of Bangladeshi actors

List of Bangladeshi architects

List of Bangladeshi painters

List of Bangladeshi poets

List of Bangladesh-related topics

List of Bangladesh Test cricketers

List of Bangladeshi writers

\section{References}\label{references}

\section{Cited sources}\label{cited-sources}

\begin{itemize}
\item
  \emph{Bangladesh: Past and Present.}
\item
  \emph{Bangladesh: Politics, Economy and Civil Society.}
\item
  \emph{Bangladesh, from a Nation to a State.}
\end{itemize}

Ahmed, Salahuddin (2004). Bangladesh: Past and Present. APH Publishing.
ISBN~978-81-7648-469-5.

Baxter, C (1997). Bangladesh, from a Nation to a State. Westview Press.
ISBN~978-0-8133-3632-9. OCLC~47885632.

Lewis, David (2011). Bangladesh: Politics, Economy and Civil Society.
Cambridge University Press. ISBN~978-1-139-50257-3.

\section{Further reading}\label{further-reading}

\section{External links}\label{external-links}

\begin{itemize}
\item
  \emph{Bangladesh at Curlie}
\item
  \emph{Bangladesh from the BBC News}
\item
  \emph{Bangladesh Encyclopædia Britannica entry}
\item
  \emph{"Bangladesh".}
\item
  \emph{Wikimedia Atlas of Bangladesh}
\item
  \emph{Bangladesh from UCB Libraries GovPubs}
\item
  \emph{Official Site of The Tourism Board of Bangladesh}
\end{itemize}

Government

Official website

Official Site of The Tourism Board of Bangladesh

Official Site of Bangladesh Investment Development Authority

General information

"Bangladesh". The World Factbook. Central Intelligence Agency.

Bangladesh from the BBC News

Bangladesh from UCB Libraries GovPubs

Bangladesh at Curlie

Bangladesh Encyclopædia Britannica entry

Wikimedia Atlas of Bangladesh

Geographic data related to Bangladesh at OpenStreetMap

Key Development Forecasts for Bangladesh from International Futures
