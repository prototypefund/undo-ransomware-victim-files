\textbf{From Wikipedia, the free encyclopedia}

https://en.wikipedia.org/wiki/Hugh\%20Dryden\\
Licensed under CC BY-SA 3.0:\\
https://en.wikipedia.org/wiki/Wikipedia:Text\_of\_Creative\_Commons\_Attribution-ShareAlike\_3.0\_Unported\_License

\section{Hugh Latimer Dryden}\label{hugh-latimer-dryden}

\begin{itemize}
\item
  \emph{He served as NASA Deputy Administrator from August 19, 1958
  until his death.}
\item
  \emph{Hugh Latimer Dryden (July 2, 1898 -- December 2, 1965) was an
  American aeronautical scientist and civil servant.}
\end{itemize}

Hugh Latimer Dryden (July 2, 1898 -- December 2, 1965) was an American
aeronautical scientist and civil servant. He served as NASA Deputy
Administrator from August 19, 1958 until his death.

\section{Biography}\label{biography}

\begin{itemize}
\item
  \emph{Dryden was portrayed by George Bartenieff in the 1998 TV
  miniseries From the Earth to the Moon.}
\item
  \emph{As a student, Dryden excelled in mathematics.}
\item
  \emph{Dryden is also a founding member of the National Academy of
  Engineering.}
\item
  \emph{Michael Gorn, chief historian at NASA Dryden Flight Research
  Center, described Dryden as a quiet, reserved man who was
  self-effacing and diligent.}
\end{itemize}

Dryden was born in Pocomoke City, Maryland, the son of Samuel Isaac and
Nova Hill Culver Dryden, and was named after a popular local Methodist
clergyman. During the financial panic of 1907, his father lost his job
and the family moved to Baltimore, Maryland.

As a student, Dryden excelled in mathematics. He graduated from
Baltimore City College, a high school, at the age of 14, and was the
youngest student ever to graduate from that school. He was awarded the
Peabody Prize for excellence in mathematics. With a scholarship, he was
admitted to Johns Hopkins University and graduated with honors after
only three years. He earned a M.S. in physics in 1916. His thesis was
titled, "Airplanes: An Introduction to the Physical Principles Embodied
in their Use."

In 1918, Dryden joined the National Bureau of Standards, becoming an
inspector of gauges. With the help and influence of Dr. Joseph S. Ames,
he obtained a transfer to the bureau's Wind Tunnel division, and began
taking graduate courses in fluid dynamics to complete his Ph.D. In 1919
at the age of 20, he was awarded his degree in physics and mathematics
from Johns Hopkins University, the youngest person ever to have received
a doctorate from that institution. His thesis was on the "Air Forces on
Circular Cylinders".

In 1920 Dryden was appointed the director of the Aerodynamics Division
of the National Bureau of Standards, a newly created section.
Collaborating with Dr. Lyman J. Briggs, he performed studies of airfoils
near the speed of sound. He also performed pioneering aerodynamics
research on the problems of airflow, turbulence, and especially the
boundary layer phenomenon. His work contributed to the design of the
wings for the P-51 Mustang, as well as other aircraft designed during
World War II.

By 1934, Dryden was appointed the bureau's Chief of the Mechanics and
Sound Division, and in 1939 he became a member of the National Advisory
Committee for Aeronautics (NACA).

With the start of World War II, Dryden served in an advisory capacity to
the Air Force. He led the development of the "Bat", a radar-homing
guided bomb program that was successfully employed in combat in April,
1945 to sink a Japanese destroyer.

After the war, Dryden became the Director of Aeronautical Research for
the National Advisory Committee for Aeronautics (NACA) in 1946. While at
the NACA he supervised the development of the North American X-15, a
rocket plane used for research and testing. He also established programs
for V/STOL aircraft, and studied the problem of atmospheric reentry.

He held the position of Director of NACA, NASA's predecessor, from 1947
until October 1958. In addition he served on numerous government
advisory committees, including the Scientific Advisory Committee to the
President. From 1941 until 1956 he was editor of the Journal of the
Institute of the Aeronautical Sciences. After NACA became NASA, he
became the Deputy Director of that organization, serving until his
death.

After John Glenn's orbital flight, an exchange of letters between
President John F. Kennedy and Soviet Premiere Nikita Khrushchev led to a
series of discussions led by Dryden and Soviet scientist Anatoli
Blagonravov. Their talks in 1962 led to the Dryden-Blagonravov
agreement, which was formalized in October of that year, the same time
the two countries were in the midst of the Cuban Missile Crisis. The
agreement was formally announced at the United Nations on December 5,
1962. It called for cooperation on the exchange of data from weather
satellites, a study of the Earth's magnetic field, and joint tracking of
the U.S. Echo II balloon satellite.. Unfortunately, as the competition
between the two nation's manned space programs heated up, efforts to
further cooperation at that point came to an end. They would be revived
in 1969 by NASA Administrator Thomas Paine and led to the 1975
Apollo-Soyuz Test Project.

He died from cancer on December 2, 1965.

Michael Gorn, chief historian at NASA Dryden Flight Research Center,
described Dryden as a quiet, reserved man who was self-effacing and
diligent. He was patient, a good teacher, and effective when
collaborating with others. He was also a devout Methodist, who, as a
result, had a dislike of self-promotion. He served as a lay minister for
his entire adult life. He was married to Mary Libbie Travers, and the
couple had four children.

Tom Wolfe, writing in 2009 at the 40th anniversary of the launch of
Apollo 11, credited Dryden with having been the individual who spoke up,
with President John F. Kennedy in April, 1961, and suggested that manned
flight to the moon was the way to "catch up" with the Soviets in the
space race. Wolfe describes President Kennedy as having been in "a
terrible funk" at the time of the meeting with James E. Webb, the NASA
administrator, and Dryden, his deputy, as the president wrestled with
the string of Soviet "firsts" in space flight which had started with
Sputnik 1 in 1957 and, that month in 1961, had extended to include human
earth-orbital flight. Within a month of the meeting with Webb and
Dryden, President Kennedy announced the Apollo Project-scale goal of
putting a man on the moon within 10 years, the goal that Apollo 11 was
ultimately to meet. In setting the goal, the president did not credit
Dryden's input, according to Wolfe.

Dryden is also a founding member of the National Academy of Engineering.

Dryden was portrayed by George Bartenieff in the 1998 TV miniseries From
the Earth to the Moon.

\section{Bibliography}\label{bibliography}

\begin{itemize}
\item
  \emph{Dryden, Hugh L., and Abbott, Ira H., "The design of
  low-turbulence wind tunnels", NACA, Technical Note 1755, Nov 1949.}
\item
  \emph{Dryden published over a hundred papers and articles.}
\end{itemize}

Dryden published over a hundred papers and articles.

"Turbulence and the Boundary Layer", Wright Brothers Lecture, 1938.

"The Role of Transition from Laminar to Turbulent Flow in Fluid
Mechanics", 1941, proceedings University of Pennsylvania Bicentennial
Conference on Fluid Mechanics and Statistical Methods in Engineering.

"Recent advances in the mechanics of boundary layer flow", Academic
Press Inc., New York, 1948.

Dryden, Hugh L., and Abbott, Ira H., "The design of low-turbulence wind
tunnels", NACA, Technical Note 1755, Nov 1949.

"General Survey of Experimental Aerodynamics", 1956, Dover.

"The International Geophysical Year: Man's most ambitious study of his
environment," National Geographic, February 1956, pp.~285--285.

"Footprints on the Moon", National Geographic, March 1964, pp.~356--401.

\section{Awards and honors}\label{awards-and-honors}

\begin{itemize}
\item
  \emph{The Western Aeronautical Test Range at the facility was renamed
  the NASA Hugh L. Dryden Aeronautical Test Range.}
\item
  \emph{The NASA Flight Research Center was renamed the NASA Hugh L.
  Dryden Flight Research Center on March 26, 1976.}
\item
  \emph{The crater Dryden on the Moon is named after him.}
\end{itemize}

President's Certificate of Merit.

Daniel Guggenheim Medal, 1950.

Wright brothers memorial trophy, 1956.

Baltimore City College Hall of Fame, 1958.

Career Service Award from the National Civil Service League, 1958.

Elliott Cresson Medal from The Franklin Institute, 1960.

Langley Gold Medal from the Smithsonian Institution, 1962.

National Medal of Science award in Engineering, 1965.

Inducted into the International Space Hall of Fame in 1976.

Inducted into the International Air \& Space Hall of Fame in 1973.

Sixteen honorary doctorates.

Member of the American Association for the Advancement of Science.

Member of the National Academy of Sciences.

Founding Member of the National Academy of Engineering.

The NASA Flight Research Center was renamed the NASA Hugh L. Dryden
Flight Research Center on March 26, 1976. This was rescinded on March 1,
2014 when the center was renamed the "Neil A. Armstrong Flight Research
Center."

The crater Dryden on the Moon is named after him.

The Western Aeronautical Test Range at the facility was renamed the NASA
Hugh L. Dryden Aeronautical Test Range.

\section{References}\label{references}

\begin{itemize}
\item
  \emph{Michael H. Gorn, "Hugh L. Dryden's Career in Aviation and
  Space", 1996, Washington, D.C., Monographs in Aerospace History.}
\item
  \emph{Michael Gorn, "A Powerful Friendship: Theodore von Kármán and
  Hugh L. Dryden", NASA TM-2003-212031.}
\end{itemize}

Michael Gorn, "A Powerful Friendship: Theodore von Kármán and Hugh L.
Dryden", NASA TM-2003-212031.

Michael H. Gorn, "Hugh L. Dryden's Career in Aviation and Space", 1996,
Washington, D.C., Monographs in Aerospace History.

\section{External links}\label{external-links}

\begin{itemize}
\item
  \emph{Hugh L. Dryden's Career in Aviation and Space, by Michael H.
  Gorn}
\end{itemize}

Official NASA biography

Hugh L. Dryden's Career in Aviation and Space, by Michael H. Gorn

New Mexico Museum of Space History

Biography

National Academy of Sciences Biographical Memoir
