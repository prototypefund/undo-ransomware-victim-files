\textbf{From Wikipedia, the free encyclopedia}

https://en.wikipedia.org/wiki/Fifth\%20Circuit\%20Four\\
Licensed under CC BY-SA 3.0:\\
https://en.wikipedia.org/wiki/Wikipedia:Text\_of\_Creative\_Commons\_Attribution-ShareAlike\_3.0\_Unported\_License

\section{Fifth Circuit Four}\label{fifth-circuit-four}

\begin{itemize}
\item
  \emph{The "Fifth Circuit Four" (or simply "The Four") were four judges
  of the United States Court of Appeals for the Fifth Circuit who,
  during the late 1950s, became known for a series of decisions (which
  continued into the late 1960s) crucial in advancing the civil and
  political rights of African Americans; in this they were opposed by
  fellow Fifth Circuit judge Ben Cameron, a strong advocate of states'
  rights.}
\end{itemize}

The "Fifth Circuit Four" (or simply "The Four") were four judges of the
United States Court of Appeals for the Fifth Circuit who, during the
late 1950s, became known for a series of decisions (which continued into
the late 1960s) crucial in advancing the civil and political rights of
African Americans; in this they were opposed by fellow Fifth Circuit
judge Ben Cameron, a strong advocate of states' rights. At that time,
the Fifth Circuit included not only Louisiana, Mississippi, and Texas
(the limits of its jurisdiction since October 1, 1981), but also
Alabama, Georgia, Florida, and the Panama Canal Zone.

"The Four" were Chief Judge Elbert Tuttle and his three colleagues John
Minor Wisdom, John Robert Brown, and Richard Rives. All but Rives were
liberal Republicans; Rives was a Democrat and, according to Jack Bass,
an intimate of Supreme Court justice Hugo Black.

\section{Quote}\label{quote}

\begin{itemize}
\item
  \emph{In that sense the Constitution is color blind.}
\item
  \emph{- Judge John Minor Wisdom, writing for the majority in United
  States v. Jefferson County Board of Education, 1967.}
\item
  \emph{"The Constitution is both color blind and color conscious.}
\end{itemize}

"The Constitution is both color blind and color conscious. To avoid
conflict with the equal protection clause, a classification that denies
a benefit, causes harm, or imposes a burden must not be based on race.
In that sense the Constitution is color blind. But the Constitution is
color conscious to prevent discrimination being perpetuated and to undo
the effects of past discrimination. The criterion is the relevancy of
color to a legitimate government purpose."

- Judge John Minor Wisdom, writing for the majority in United States v.
Jefferson County Board of Education, 1967.

\section{References}\label{references}

\begin{itemize}
\item
  \emph{Jack Bass, "The 'Fifth Circuit Four'", The Nation, May 3, 2004,
  p. 30-32.}
\end{itemize}

Jack Bass, "The 'Fifth Circuit Four'", The Nation, May 3, 2004, p.
30-32.

Jack Bass, Unlikely Heroes: The Dramatic Story of the Southern Judges of
the Fifth Circuit who Translated the Supreme Court's Brown Decision Into
a Revolution for Equality (New York: Simon and Schuster, 1981),
.mw-parser-output cite.citation\{font-style:inherit\}.mw-parser-output
.citation
q\{quotes:"\textbackslash{}"""\textbackslash{}"""'""'"\}.mw-parser-output
.citation .cs1-lock-free
a\{background:url("//upload.wikimedia.org/wikipedia/commons/thumb/6/65/Lock-green.svg/9px-Lock-green.svg.png")no-repeat;background-position:right
.1em center\}.mw-parser-output .citation .cs1-lock-limited
a,.mw-parser-output .citation .cs1-lock-registration
a\{background:url("//upload.wikimedia.org/wikipedia/commons/thumb/d/d6/Lock-gray-alt-2.svg/9px-Lock-gray-alt-2.svg.png")no-repeat;background-position:right
.1em center\}.mw-parser-output .citation .cs1-lock-subscription
a\{background:url("//upload.wikimedia.org/wikipedia/commons/thumb/a/aa/Lock-red-alt-2.svg/9px-Lock-red-alt-2.svg.png")no-repeat;background-position:right
.1em center\}.mw-parser-output .cs1-subscription,.mw-parser-output
.cs1-registration\{color:\#555\}.mw-parser-output .cs1-subscription
span,.mw-parser-output .cs1-registration span\{border-bottom:1px
dotted;cursor:help\}.mw-parser-output .cs1-ws-icon
a\{background:url("//upload.wikimedia.org/wikipedia/commons/thumb/4/4c/Wikisource-logo.svg/12px-Wikisource-logo.svg.png")no-repeat;background-position:right
.1em center\}.mw-parser-output
code.cs1-code\{color:inherit;background:inherit;border:inherit;padding:inherit\}.mw-parser-output
.cs1-hidden-error\{display:none;font-size:100\%\}.mw-parser-output
.cs1-visible-error\{font-size:100\%\}.mw-parser-output
.cs1-maint\{display:none;color:\#33aa33;margin-left:0.3em\}.mw-parser-output
.cs1-subscription,.mw-parser-output .cs1-registration,.mw-parser-output
.cs1-format\{font-size:95\%\}.mw-parser-output
.cs1-kern-left,.mw-parser-output
.cs1-kern-wl-left\{padding-left:0.2em\}.mw-parser-output
.cs1-kern-right,.mw-parser-output
.cs1-kern-wl-right\{padding-right:0.2em\}ISBN~0-671-25064-7,
ISBN~978-0-671-25064-5.

\section{See also}\label{see-also}

\begin{itemize}
\item
  \emph{Frank Minis Johnson -- a U.S. District Judge for the Middle
  District of Alabama who's rulings had a strong impact on civil rights
  in the American South}
\end{itemize}

Frank Minis Johnson -- a U.S. District Judge for the Middle District of
Alabama who's rulings had a strong impact on civil rights in the
American South
