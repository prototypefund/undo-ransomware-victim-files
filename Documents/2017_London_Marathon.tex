\textbf{From Wikipedia, the free encyclopedia}

https://en.wikipedia.org/wiki/2017\_London\_Marathon\\
Licensed under CC BY-SA 3.0:\\
https://en.wikipedia.org/wiki/Wikipedia:Text\_of\_Creative\_Commons\_Attribution-ShareAlike\_3.0\_Unported\_License

\section{2017 London Marathon}\label{london-marathon}

\begin{itemize}
\item
  \emph{The 2017 London Marathon was held on 23 April 2017.}
\item
  \emph{It was the 37th running of the London Marathon, an annual
  mass-participation race held in London, England.}
\end{itemize}

The 2017 London Marathon was held on 23 April 2017. It was the 37th
running of the London Marathon, an annual mass-participation race held
in London, England.

Mary Keitany won the women's race, setting a new women-only world record
with a time of 2:17:01, while Daniel Wanjiru came first in the men's
race in 2:05:48.

David Weir claimed a record breaking seventh win at the London Marathon
in the men's wheelchair event. The win broke a tie between Weir and
Tanni Gray Thompson for the most wins at the London Marathon.

\section{Course}\label{course}

\begin{itemize}
\item
  \emph{The London Marathon is run over a largely flat course around the
  River Thames, and spans 26 miles and 385 yards (42.195 kilometres).}
\item
  \emph{Heading into the final leg of the race, competitors pass The
  Tower of London on Tower Hill.}
\end{itemize}

The London Marathon is run over a largely flat course around the River
Thames, and spans 26 miles and 385 yards (42.195 kilometres). The route
has markers at one mile and five kilometre intervals.

The course begins at three separate points: the 'red start' in southern
Greenwich Park on Charlton Way, the 'green start' in St John's Park, and
the 'blue start' on Shooter's Hill Road. From these points around
Blackheath at 35~m (115~ft) above sea level, south of the River Thames,
the route heads east through Charlton. The three courses converge after
4.5~km (2.8 miles) in Woolwich, close to the Royal Artillery Barracks.

As the runners reach the 10~km mark (6.2-mile), they pass by the Old
Royal Naval College and head towards Cutty Sark drydocked in Greenwich.
Heading next into Deptford and Surrey Quays in the Docklands, and out
towards Bermondsey, competitors race along Jamaica Road before reaching
the half-way point as they cross Tower Bridge. Running east again along
The Highway through Wapping, competitors head up towards Limehouse and
into Mudchute in the Isle of Dogs via Westferry Road, before heading
into Canary Wharf.

As the route leads away from Canary Wharf into Poplar, competitors run
west down Poplar High Street back towards Limehouse and on through
Commercial Road. They then move back onto The Highway, onto Lower and
Upper Thames Streets. Heading into the final leg of the race,
competitors pass The Tower of London on Tower Hill. In the penultimate
mile along The Embankment, the London Eye comes into view, before the
athletes turn right into Birdcage Walk to complete the final 352~m (385
yards), catching the sights of Big Ben and Buckingham Palace, and
finishing in The Mall alongside St. James's Palace.

\section{Race summary}\label{race-summary}

\begin{itemize}
\item
  \emph{In the women's race, Keitany was rarely threatened.}
\item
  \emph{The men's wheelchair race saw David Weir claim a record breaking
  seventh win at the London Marathon when he out sprinted Marcel Hug and
  Rafael Botello Jimenez.}
\item
  \emph{The men's race was largely contested between Kenya's Wanjiru and
  Ethiopia's Kenenisa Bekele.}
\end{itemize}

In the women's race, Keitany was rarely threatened. She broke away from
the field after the first mile and maintained a comfortable lead until
the end of the race. Her final time was the second fastest in history,
and the fastest set without the help of male pacemakers, beating Paula
Radcliffe's record of 2:17:42 set in the 2005 race. The overall women's
record, 2:15:25, was also set by Radcliffe in the 2003 race.

The men's race was largely contested between Kenya's Wanjiru and
Ethiopia's Kenenisa Bekele. Bekele led until approximately halfway
through the race, when he dropped back sharply. Wanjiru stayed with a
lead pack of Bedan Karoki, Abel Kirui and Feyisa Lilesa until 21 miles
before making a break. However, Bekele was not finished and rapidly
accelerated through the field, closing the gap to eight seconds with
less than a mile left. Wanjiru however found the strength to hold Bekele
off, eventually winning by nine seconds.

There was also a surprise when a club runner, Josh Griffiths, who did
not start with the elite athletes, finished in 2:14:49, a time which
would have given him 13th place in the elite field. He qualified for the
World Championships with this time. Matthew Rees helped an exhausted
fellow runner, David Wyeth, across the finish line, an occurrence widely
mentioned in social and traditional media.

The men's wheelchair race saw David Weir claim a record breaking seventh
win at the London Marathon when he out sprinted Marcel Hug and Rafael
Botello Jimenez. Manuela Schar won her first title in London, finishing
almost 5 minutes ahead of her nearest rival.

\section{Results}\label{results}

\begin{itemize}
\item
  \emph{Results are listed below:}
\end{itemize}

Results are listed below:

\section{Elite races}\label{elite-races}

\section{Wheelchair races}\label{wheelchair-races}

\section{References}\label{references}

\section{External links}\label{external-links}

\begin{itemize}
\item
  \emph{Official website}
\end{itemize}

Official website
