\textbf{From Wikipedia, the free encyclopedia}

https://en.wikipedia.org/wiki/Deep\%20Space\%20Atomic\%20Clock\\
Licensed under CC BY-SA 3.0:\\
https://en.wikipedia.org/wiki/Wikipedia:Text\_of\_Creative\_Commons\_Attribution-ShareAlike\_3.0\_Unported\_License

\section{Deep Space Atomic Clock}\label{deep-space-atomic-clock}

\begin{itemize}
\item
  \emph{The Deep Space Atomic Clock (DSAC) is a miniaturized,
  ultra-precise mercury-ion atomic clock for precise radio navigation in
  deep space.}
\end{itemize}

The Deep Space Atomic Clock (DSAC) is a miniaturized, ultra-precise
mercury-ion atomic clock for precise radio navigation in deep space. It
is orders of magnitude more stable than existing navigation clocks, and
has been refined to limit drift of no more than 1 nanosecond in 10 days.
It is expected that a DSAC would incur no more than 1 microsecond of
error in 10 years of operations. It is expected to improve the precision
of deep space navigation, and enable more efficient use of tracking
networks. The project is managed by NASA's Jet Propulsion Laboratory and
it will be deployed as part of the U.S. Air Force's Space Test Program 2
(STP-2) mission aboard a SpaceX Falcon Heavy rocket in June 2019.

\section{Overview}\label{overview}

\begin{itemize}
\item
  \emph{Its applications in deep space include:}
\item
  \emph{Simultaneously track two spacecraft on a downlink with the Deep
  Space Network (DSN).}
\item
  \emph{The Deep Space Atomic Clock (DSAC) is a miniaturized and stable
  mercury ion atomic clock that is as stable as a ground clock.}
\item
  \emph{Current ground-based atomic clocks are fundamental to deep space
  navigation, however, they are too large to be flown in space.}
\end{itemize}

Current ground-based atomic clocks are fundamental to deep space
navigation, however, they are too large to be flown in space. This
results in tracking data being collected and processed here on Earth (a
two-way link) for most deep space navigation applications. The Deep
Space Atomic Clock (DSAC) is a miniaturized and stable mercury ion
atomic clock that is as stable as a ground clock. The technology could
enable autonomous radio navigation for spacecraft's time-critical events
such as orbit insertion or landing, promising new savings on mission
operations costs. It is expected to improve the precision of deep space
navigation, enable more efficient use of tracking networks, and yield a
significant reduction in ground support operations.

Its applications in deep space include:

Simultaneously track two spacecraft on a downlink with the Deep Space
Network (DSN).

Improve tracking data precision by an order of magnitude using the DSN's
Ka-band downlink tracking capability.

Mitigate Ka-band's weather sensitivity (as compared to two-way X band)
by being able to switch from a weather-impacted receiving antenna to one
in a different location with no tracking outages.

Track longer by using a ground antenna's entire spacecraft viewing
period. At Jupiter, this yields a 10--15\% increase in tracking; at
Saturn, it grows to 15--25\%, with the percentage increasing the farther
a spacecraft travels.

Make new discoveries as a Ka-band---capable radio science instrument
with a 10 times improvement in data precision for both gravity and
occultation science and deliver more data because of one-way tracking's
operational flexibility.

Explore deep space as a key element of a real-time autonomous navigation
system that tracks one-way radio signals on the uplink and, coupled with
optical navigation, provides for robust absolute and relative
navigation.

Fundamental to human explorers requiring real-time navigation data.

\section{Principle and development}\label{principle-and-development}

\begin{itemize}
\item
  \emph{Over 20 years, engineers at NASA's Jet Propulsion Laboratory
  have been steadily improving and miniaturizing the mercury-ion trap
  atomic clock.}
\end{itemize}

Over 20 years, engineers at NASA's Jet Propulsion Laboratory have been
steadily improving and miniaturizing the mercury-ion trap atomic clock.
The DSAC technology uses the property of mercury ions' hyperfine
transition frequency at 40.50 GHz to effectively "steer" the frequency
output of a quartz oscillator to a near-constant value. DSAC does this
by confining the mercury ions with electric fields in a trap and
protecting them by applying magnetic fields and shielding.

Its development will include a test flight in low-Earth orbit, while
using GPS signals to demonstrate precision orbit determination and
confirm its performance in radio navigation.

\section{Deployment}\label{deployment}

\begin{itemize}
\item
  \emph{It will be deployed as a secondary spacecraft during the U.S.
  Air Force's Space Test Program 2 (STP-2) mission aboard a SpaceX
  Falcon Heavy rocket, probably in June 2019.}
\end{itemize}

The flight unit will be hosted ---along with other four payloads--- on a
spacecraft called Orbital Test Bed (OTB) satellite, provided by General
Atomics Electromagnetic Systems, using the Swift satellite bus. It will
be deployed as a secondary spacecraft during the U.S. Air Force's Space
Test Program 2 (STP-2) mission aboard a SpaceX Falcon Heavy rocket,
probably in June 2019.

\section{References}\label{references}

\section{External links}\label{external-links}

\begin{itemize}
\item
  \emph{DSAC: Description and Nominal Mission Operations}
\end{itemize}

DSAC: Description and Nominal Mission Operations
