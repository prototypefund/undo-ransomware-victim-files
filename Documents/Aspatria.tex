\textbf{From Wikipedia, the free encyclopedia}

https://en.wikipedia.org/wiki/Aspatria\\
Licensed under CC BY-SA 3.0:\\
https://en.wikipedia.org/wiki/Wikipedia:Text\_of\_Creative\_Commons\_Attribution-ShareAlike\_3.0\_Unported\_License

\section{Aspatria}\label{aspatria}

\begin{itemize}
\item
  \emph{It is served by Aspatria railway station.}
\item
  \emph{Aspatria /əsˈpeɪtriə/ is a civil parish in the non-metropolitan
  district of Allerdale, and is currently embraced in the Parliamentary
  constituency of Workington, Cumbria, England.}
\item
  \emph{Aspatria is located on the fringe of the English Lake District.}
\end{itemize}

Aspatria /əsˈpeɪtriə/ is a civil parish in the non-metropolitan district
of Allerdale, and is currently embraced in the Parliamentary
constituency of Workington, Cumbria, England. Historically within
Cumberland the town rests on the north side of the Ellen Valley,
overlooking a panoramic view of the countryside, with Skiddaw to the
South and the Solway Firth to the North. Its developments are aligned
approximately east-west along the A596 Carlisle to Workington road and
these extend to approximately 2 miles (3~km) in length. It lies about 8
miles (12~km) northeast of Maryport, a similar distance to the Southwest
of Wigton, about 9 miles (14~km) north of Cockermouth and 5 miles (8~km)
from the coast and Allonby. It comprises the townships of Aspatria and
Brayton, Hayton and Mealo, and Oughterside and Allerby, the united area
being 8,345 acres (3,377~ha); while the township takes up an area of
1,600 acres (647~ha). In earlier days a Roman road leading from "Old
Carlisle" to Ellenborough passed through the hamlet.

The population has greatly increased since the mid nineteenth century.
In 1801, the village comprised 98 dwellings with a population of 321. By
1851, there were 236 family entities, comprising 1,123 residents; by
1871, the numbers had increased to 1,778; and twenty years later stood
at 2,714. By the start of the 20th century, the population had risen to
2,885; twenty years later it peaked at 3,521. Although the population
slumped in the 1930s to 3,189, it recovered to 3,500, in 1951; and by
1981, the population appeared stable at 2,745.\\
It is served by Aspatria railway station. Aspatria is located on the
fringe of the English Lake District.

The parish church of St Kentigern was completed in 1848. Fragments of
masonry and crosses from earlier structures on the same site are
preserved there.

\section{History}\label{history}

\section{Pre Norman}\label{pre-norman}

\begin{itemize}
\item
  \emph{In 1789, a surgeon by the name of Rigg employed a group of
  labourers to level a mound called Beacon Hill, situated close behind
  his house at Aspatria.}
\item
  \emph{Aspatria is an ancient settlement and seems to have been home to
  a group of Norsemen who fled to the area from Ireland around 900.}
\end{itemize}

Aspatria is an ancient settlement and seems to have been home to a group
of Norsemen who fled to the area from Ireland around 900. In 1789, a
surgeon by the name of Rigg employed a group of labourers to level a
mound called Beacon Hill, situated close behind his house at Aspatria.
After reaching a depth of about one metre they dug into a cavity walled
around with large stones and found the skeleton of a Viking chief almost
complete over two metres in length. At the head of the gigantic skeleton
lay a sword almost two metres in length, with a remarkably broad blade,
ornamented with a gold and silver handle. The scabbard of the sword was
made of wood, lined with cloth. The workmen also unearthed several
pieces of armour, a dirk with a silver studded handle, a golden buckled
belt, and a breast plate. The artefacts remain the property of the
British Museum. Further finds were made on the same site in 1997 when a
mobile phone mast was being constructed.

\section{The Manor}\label{the-manor}

\begin{itemize}
\item
  \emph{The manor of Aspatria is part of the ancient barony of Allerdale
  below Derwent.}
\item
  \emph{In 1870, one of England's first farmers' co-operatives, the
  Aspatria Agricultural Cooperative Society was established here with
  offices in the market square, facing the Aspatria Agricultural College
  which flourished from 1874 until 1925.}
\end{itemize}

The manor of Aspatria is part of the ancient barony of Allerdale below
Derwent. Awarded by Ranulph de Meschines, grantee of the whole of
Cumberland from William the Conqueror, to Waltheof, son of Gospatrick,
Earl of Dunbar, from whom the obsolete name of Aspatrick, may have been
derived. Upon the division of the estates of William Fitz Duncan, and
his wife Alice de Romney, among their three daughters, the manor passed
to Alice the youngest. However Alice died without issue and the estates
passed to an elder sister who had married into the Lucy family. The
latter family terminated in a female heir Maud de Lucy. She married
Henry Percy, the first Earl of Northumberland, who received the whole of
her estates. It remained in this family through eleven generations
before passing by the marriage of Lady Elizabeth, sole daughter and
heiress of Josceline Percy to Charles Seymour, sixth Earl of Somerset.
In recent times it again passed by a female heir to the Wyndham family,
from whom it has descended to Lord Leconfield and now Lord Egremont.

The village stands at the northern end of the West Cumberland Coalfield
and there have been mines in the area since the 16th century. The
opening of the Maryport and Carlisle Railway, in 1842, led to a rapid
expansion of the industry. The Brayton Domain Collieries sank five
different pits around the town at various times and there were also
mines near Mealsgate, Baggrow and Fletchertown. In 1902, a new mine was
sunk at Oughterside. The last pit in the town, Brayton Domain No.5,
closed in 1940.

In 1870, one of England's first farmers' co-operatives, the Aspatria
Agricultural Cooperative Society was established here with offices in
the market square, facing the Aspatria Agricultural College which
flourished from 1874 until 1925.

Sir Wilfrid Lawson MP (1829--1906) lived at Brayton Hall just outside
the town. He was a committed nonconformist and a leader of the
Temperance Movement. His memorial stands in the market square, topped by
a bronze effigy of St George slaying the dragon -- said to represent the
demon drink. Brayton Hall was destroyed by fire in 1918.

\section{Toponymy}\label{toponymy}

\begin{itemize}
\item
  \emph{According to one source the origins of the name of Aspatria lie
  in Old Scandinavian and Celtic.}
\item
  \emph{'Old Bill', as he was affectionately known, was born at Aspatria
  in 1817.}
\item
  \emph{The first entry in the parish register referring to the town as
  Aspatria in preference to the name Aspatrick or Aspatricke appears in
  1712.}
\end{itemize}

According to one source the origins of the name of Aspatria lie in Old
Scandinavian and Celtic. It translates as "Ash-tree of St Patrick", and
is composed of the elements askr (Old Scandinavian for "ash-tree") and
the Celtic saint's name. The order of the elements of the name, with the
ash-tree coming before the name of the saint, is particular to Celtic
place-names. The following forms of the name have been found in various
charters:- Estpatrick in 1224, Asepatrick 1230, Aspatric 1233, Askpatrik
1291, Assepatrick 1303, Aspatrick 1357, Aspatre 1491. The first entry in
the parish register referring to the town as Aspatria in preference to
the name Aspatrick or Aspatricke appears in 1712. It appears in the
handwriting of the then vicar David Bell. For the next fifty years the
spelling fluctuated until eventually Aspatria became the dominant name.
When Charles Dickens and Wilkie Collins passed through the town in 1857
they referred to the name Spatter which is not to dissimilar to
'Speatrie' the name locals prefer. This leads us on to the familiar
expression 'Speatrie Loup Oot', which had its genesis in the cry of
William Brough, a railway porter, discharging third class passengers
after their arrival at Aspatria from the Bolton Loop railway connection.
Second class passengers would detect, "Speatrie change ere for Measyat";
while first class passengers heard a polite invitation, "Aspatriah,
change heah for Mealsgate."

There is a legend that the name comes from the ash tree that grew up
when St. Patrick's staff, the Bachal Isu, took root in the ground
because it took so long for him to manage to convert the people from
this area to Christianity.

Throughout the second half of the nineteenth century Sir Wilfrid Lawson,
2nd Baronet, of Brayton made many political and non-political speeches
in the neighbourhood, and occasionally made reference to the above
phrase. At the opening of the West Cumberland Dairy, in Jan 1889, he
made the following reference.

"In olden days Aspatria used to be a railway terminus, and some of us
are old enough to remember that when the train pulled up from Maryport,
the porter used to come to the door and shout Speytrie, un' git oot."

'Old Bill', as he was affectionately known, was born at Aspatria in
1817. A former agricultural worker, he became a railway employee with
the Maryport \& Carlisle Railway Company in about 1850. His antics were
first brought to the attention of a wider public in October 1863, when
he became the subject of a comical sketch at a local concert. During the
Music Hall rendition, the Brothers Bouch presented a self-penned song
entitled, 'Bill, the Railway Porter'. A performance often repeated by
popular demand at the season of penny readings performed throughout the
1860s. Unfortunately the lyrics no longer exist, only the joke remains
as a curt reminder.\\
The Rev. William Slater Calverley, vicar of St. Kentigerns (1888--97),
referred to the phenomena in his seminal study, 'Early Sculptured,
Crosses, Shrines and Monuments', with the following paragraph;

"Aspatrick" which is another name for "St Patrick," and which people
pronounce "S'Patrick"; as witness what many folks remember, when the
Maryport \& Carlisle Railway terminated at Aspatria, and the porter now
deceased used quite politely to show out the first-class passengers;
but, coming to the third class carriages, threw open the doors and
shouted "S'Patrick, get out!"

Recent research at the University of Warwick by Professor William Bailey
suggests an alternative Roman {[}Latin{]} origin for the name Aspatria,
being "place where the asparagus grows" arising from his research into
the supply of foodstuffs to the garrisons along Hadrians Wall.

\section{Governance}\label{governance}

\begin{itemize}
\item
  \emph{An electoral ward exists with the same name.}
\item
  \emph{This ward stretches east to Allhallows with a total population
  taken at the 2011 Census of 3,380.}
\end{itemize}

An electoral ward exists with the same name. This ward stretches east to
Allhallows with a total population taken at the 2011 Census of 3,380.

\section{Religious worship}\label{religious-worship}

\begin{itemize}
\item
  \emph{The Wesleyan Methodists built their first chapel on the corner
  of North Road and Queen Street in 1898.}
\item
  \emph{Prior to the opening of the Brayton Domain Collieries the people
  of Aspatria had two places of worship, the long established Anglican
  parish church of St. Kentigern's and a non-conformist chapel of the
  Congregationalist persuasion, built by Sir Wilfrid Lawson, in 1826.}
\end{itemize}

Prior to the opening of the Brayton Domain Collieries the people of
Aspatria had two places of worship, the long established Anglican parish
church of St. Kentigern's and a non-conformist chapel of the
Congregationalist persuasion, built by Sir Wilfrid Lawson, in 1826. The
latter is now a café with dwelling behind. However, with the influx of
new workers came a demand for new institutions. In 1864, the Primitive
Methodists built a chapel in the lower end of Lawson Street. Twenty
years later, to cater for their expanding congregation they built a new
chapel, with adjoining manse for the minister, at the junction of Queen
Street and Brayton Road, while retaining the original building for use
as a Sunday school. In the 1980s they sold the property, which the new
owner demolished and replaced with a private house. In 1874, a group of
Bible Christians, originally from Cornwall built a chapel at the bottom
of Richmond Hill. This is also now the site of a private house. The
Wesleyan Methodists built their first chapel on the corner of North Road
and Queen Street in 1898. This proved too small and was replaced by the
existing building in 1921. Although the small numbers of Roman Catholics
have had a variety of meeting places over the years, they have never
built a church.

\section{Neighbouring parishes}\label{neighbouring-parishes}

\begin{itemize}
\item
  \emph{The parish is bounded on the North by the parishes of Bromfield
  and Westnewton; on the West by Gilcrux and Crosscanonby; on the South
  by Plumbland and Torpenhow; and on the East by Bromfield and
  Allhallows.}
\end{itemize}

The parish is bounded on the North by the parishes of Bromfield and
Westnewton; on the West by Gilcrux and Crosscanonby; on the South by
Plumbland and Torpenhow; and on the East by Bromfield and Allhallows.

\section{Industry}\label{industry}

\begin{itemize}
\item
  \emph{Aspatria Farmers Limited, (formerly the Aspatria Agricultural
  Cooperative Society) is based.}
\end{itemize}

There is a small industrial area next to the railway station where:-

Mattress manufacturer Sealy have maintained their British head office
since 1974.

First Milk creamery (formerly owned by the Milk Marketing Board), a
farmers' co-operative which produces Lake District Cheese, now the third
best-selling Cheddar Brand in the UK. 60 tonnes of cheese are produced
daily, using 800,000 litres of milk.

Aspatria Farmers Limited, (formerly the Aspatria Agricultural
Cooperative Society) is based.

\section{Sport}\label{sport}

\begin{itemize}
\item
  \emph{Aspatria FC are the town's football club who compete in the
  Tesco Cumberland County Premier League.}
\item
  \emph{Aspatria Hornets are the local rugby league team.}
\item
  \emph{Aspatria is also home to rugby union club Aspatria RUFC,
  currently playing in the RFU's North Lancashire/Cumbria Division.}
\end{itemize}

Aspatria Hornets are the local rugby league team. Aspatria is also home
to rugby union club Aspatria RUFC, currently playing in the RFU's North
Lancashire/Cumbria Division. The 'Aspatria Eagles' are the club's second
team, and the 'Aspatria Sinners' are the women's team. Aspatria FC are
the town's football club who compete in the Tesco Cumberland County
Premier League.

\section{Notable people}\label{notable-people}

\begin{itemize}
\item
  \emph{Dr William Perry Briggs, Medical Officer of Health to Aspatria
  Urban District Council (1892--1928)}
\item
  \emph{Jenny Cowern, artist, lived at Langrigg, Aspatria}
\item
  \emph{Sheila Fell, artist, born in Aspatria}
\item
  \emph{Henry J. Webb, principal of Aspatria Agricultural College}
\end{itemize}

Sheila Fell, artist, born in Aspatria

Jenny Cowern, artist, lived at Langrigg, Aspatria

Thomas Holliday, rugby international, had a drapery and ironmonger's
business in Queen Street

Sir Wilfrid Lawson, 2nd Baronet of Brayton, temperance campaigner and
Liberal Party politician

Henry Thompson MRCVS, veterinary surgeon, pioneer agriculturalist and
author

Greg Ridley, Rock musician

William Thompson Casson, coach designer and manufacturer

Rev. William Slater Calverley, antiquarian

Thomas Farrall, author, teacher and agriculturalist

Henry J. Webb, principal of Aspatria Agricultural College

Roland Stobbart, Speedway rider

Maurice Stobbart, Speedway rider

Dr William Perry Briggs, Medical Officer of Health to Aspatria Urban
District Council (1892--1928)

\section{See also}\label{see-also}

\begin{itemize}
\item
  \emph{Listed buildings in Aspatria}
\end{itemize}

Listed buildings in Aspatria

\section{References}\label{references}

\section{Bibliography}\label{bibliography}

\section{External links}\label{external-links}

\begin{itemize}
\item
  \emph{Brayton Domain -- pictures of Aspatria mines}
\end{itemize}

Durham Mining Museum Index of Mines

Brayton Domain -- pictures of Aspatria mines
