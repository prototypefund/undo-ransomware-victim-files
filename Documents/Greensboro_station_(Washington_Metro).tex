\textbf{From Wikipedia, the free encyclopedia}

https://en.wikipedia.org/wiki/Greensboro\_station\_\%28Washington\_Metro\%29\\
Licensed under CC BY-SA 3.0:\\
https://en.wikipedia.org/wiki/Wikipedia:Text\_of\_Creative\_Commons\_Attribution-ShareAlike\_3.0\_Unported\_License

\section{Greensboro station (Washington
Metro)}\label{greensboro-station-washington-metro}

\begin{itemize}
\item
  \emph{Greensboro (preliminary names Tysons Central 7, Tysons Central)
  is a Washington Metro station in Tysons, in Fairfax County, Virginia,
  on the Silver Line.}
\end{itemize}

Greensboro (preliminary names Tysons Central 7, Tysons Central) is a
Washington Metro station in Tysons, in Fairfax County, Virginia, on the
Silver Line. It opened on July 26, 2014 as part of phase 1 of the Silver
Line. Greensboro is one of four Metro stations in the Tysons area and is
to be part of the massive regeneration of the district.

\section{Station layout}\label{station-layout}

\begin{itemize}
\item
  \emph{The construction and overall design of the station have been
  likened to that of Southern Avenue on the Green Line because of its
  depressed but open-air layout.}
\item
  \emph{Like Spring Hill station, Greensboro was built in the median of
  SR 7 with a single island platform serving two tracks.}
\item
  \emph{However, unique amongst all Silver Line stations in Tysons, it
  was built partially at ground level and sub-surface.}
\end{itemize}

Like Spring Hill station, Greensboro was built in the median of SR 7
with a single island platform serving two tracks. However, unique
amongst all Silver Line stations in Tysons, it was built partially at
ground level and sub-surface. The construction and overall design of the
station have been likened to that of Southern Avenue on the Green Line
because of its depressed but open-air layout. This is the result of the
south end of the station acting as the western portal for the connecting
tunnel leading to SR 123 while SR 7 slopes upwards towards the east. A
mezzanine covering the central half of the platform will contain ticket
machines and faregates; two aerial walkway exits will cross either side
of Route 7 and meet at the mezzanine. The main platform has a height of
−10~ft (−3.0~m) at its east end and 8~ft (2.4~m) at its west end.

\section{History}\label{history}

\begin{itemize}
\item
  \emph{It was eventually decided that the majority of the line would be
  built above ground, but the station will be built partially below
  ground in order to send trains through a short tunnel connecting the
  line's Route 7 and Route 123-paralleling sections.}
\item
  \emph{Greensboro station opened as part of the first phase of the
  Silver Line to Wiehle -- Reston East in 2014.}
\end{itemize}

Greensboro station opened as part of the first phase of the Silver Line
to Wiehle -- Reston East in 2014. In the planning stages, controversy
ensued over whether to build the Metro in a tunnel or on an elevated
viaduct through Tysons. It was eventually decided that the majority of
the line would be built above ground, but the station will be built
partially below ground in order to send trains through a short tunnel
connecting the line's Route 7 and Route 123-paralleling sections.

\section{Location}\label{location}

\begin{itemize}
\item
  \emph{Much of the surrounding area is commercial in nature, with the
  Pike Seven Plaza Shopping Center to the west and Tysons Galleria to
  the east with little in the way of residential development.}
\item
  \emph{Greensboro station is located within west-central Tysons,
  specifically in the median of Route 7 (Leesburg Pike).}
\end{itemize}

Greensboro station is located within west-central Tysons, specifically
in the median of Route 7 (Leesburg Pike). Much of the surrounding area
is commercial in nature, with the Pike Seven Plaza Shopping Center to
the west and Tysons Galleria to the east with little in the way of
residential development. Traffic counts by the Virginia Department of
Transportation (VDOT) show that the section of Leesburg Pike on which
the station sits is the most heavily used in Fairfax County, with 61,000
vehicles per day using the stretch of road between Route 123 and the
Dulles Toll Road.

\section{Transit-oriented
development}\label{transit-oriented-development}

\begin{itemize}
\item
  \emph{The Tysons Central 7 District is divided into two sub-districts,
  North and South, separated by Route 7.}
\item
  \emph{In order to reduce congestion and improve walkability and
  connectivity in the area, the Fairfax County Planning Commission
  created the "Tysons Corner Urban Center Comprehensive Plan", an
  outline for the urbanization of Tysons in conjunction with the opening
  of the Silver Line.}
\end{itemize}

In order to reduce congestion and improve walkability and connectivity
in the area, the Fairfax County Planning Commission created the "Tysons
Corner Urban Center Comprehensive Plan", an outline for the urbanization
of Tysons in conjunction with the opening of the Silver Line. As one of
four Metro stations within the identified locale, Greensboro is the
focal point of one of the transit-oriented development schemes in the
plan. According to the Commission's outline, the area bounded by Route
123, Gosnell Drive, Westpark Drive, and International Drive will be
designated as the Tysons Central 7 District and contain high-density
residential and commercial mixed-use development.

The Tysons Central 7 District is divided into two sub-districts, North
and South, separated by Route 7. The south sub-district is approximately
76 acres (31~ha) large and will contain mixed-use development, with
offices predominating near the station and residential buildings in the
outer transition zone. The plan calls for a "civic commons" to be the
central open space in the sub-district with government and civi-related
buildings surrounding it. The north sub-district is similar in nature,
but is 102 acres (41~ha) in area. In contrast, the north sub-district is
planned to be more vibrant and 24-hour than the south, with a minimum
building height of 175 feet (53~m), although both sectors have a maximum
allowance of 400 feet (122~m). To connect these districts, it is
envisioned that Leesburg Pike will be reconfigured, along with Chain
Bridge Road, to a "boulevard" design, with a median separating four
lanes of traffic each way, as well as landscaping the sidewalks to
improve walkability. Radiating out from Route 7 will be a series of
avenues and collector streets, each with different regulations to create
a hierarchical street grid.

\section{Station facilities}\label{station-facilities}

\begin{itemize}
\item
  \emph{Pedestrian bridge crossing Route 7}
\item
  \emph{2 station entrances (each side of Route 7)}
\end{itemize}

2 station entrances (each side of Route 7)

Pedestrian bridge crossing Route 7

Bus dropoff/pickup

\section{References}\label{references}

\section{External links}\label{external-links}

\begin{itemize}
\item
  \emph{WMATA: Greensboro Station}
\end{itemize}

WMATA: Greensboro Station

Dulles Corridor Metrorail Project
