\textbf{From Wikipedia, the free encyclopedia}

https://en.wikipedia.org/wiki/Edward\%20Wong\\
Licensed under CC BY-SA 3.0:\\
https://en.wikipedia.org/wiki/Wikipedia:Text\_of\_Creative\_Commons\_Attribution-ShareAlike\_3.0\_Unported\_License

\section{Edward Wong}\label{edward-wong}

\begin{itemize}
\item
  \emph{Wong is the main writer on the Times' Culture \& Control series,
  about the struggle among citizens and the state to shape the culture
  of China.}
\item
  \emph{Wong graduated from the University of Virginia in 1994 with a
  B.A.}
\item
  \emph{Edward Wong (born in Washington, D.C.) is an American journalist
  and a foreign correspondent for The New York Times.}
\end{itemize}

Edward Wong (born in Washington, D.C.) is an American journalist and a
foreign correspondent for The New York Times. Wong served as one of the
Times' primary correspondents in Baghdad, covering the Iraq War from
November 2003 through June 2007. He then moved to the paper's Beijing
bureau in April 2008, following a sabbatical at Middlebury College and
the International Chinese Language Program (ICLP) in Taiwan improving
his Mandarin. He eventually became the Beijing bureau chief for The New
York Times, before leaving in 2017 to take up a Ferris Professorship of
Journalism at Princeton University. He is currently a Nieman Fellow at
Harvard.

Wong reports on China's politics, economy, environment, military,
foreign policy and culture. He has covered recent signature events in
China, including the Sichuan earthquake, the Beijing Olympics and unrest
in Tibet and Xinjiang. Wong is the main writer on the Times' Culture \&
Control series, about the struggle among citizens and the state to shape
the culture of China. He has also reported from Afghanistan, Tajikistan,
North Korea, Myanmar, Mongolia, India, Indonesia, Vietnam and Taiwan. He
has written travel stories about trekking in the mountains of Asia and
South America.

Wong graduated from the University of Virginia in 1994 with a B.A. in
English. In 1999, he earned dual master's degrees in journalism and
international studies at the University of California, Berkeley.

Wong's first newspaper job was at The Potomac Gazette in Potomac, MD.
While attending graduate school at Berkeley, he wrote freelance stories
for The Los Angeles Times, The San Francisco Chronicle, The San Jose
Mercury News, Wired magazine and The Far Eastern Economic Review. Wong
worked as an intern at The Associated Press in 1997. He started out at
The New York Times as an intern in 1998 and went on to report for the
metro, sports, business and foreign desks.

Wong received the 2005 Livingston Award for International Reporting for
his Iraq coverage. He was among a group of reporters from the Times'
Baghdad bureau named as finalists for the 2008 Pulitzer Prize in
International Reporting. He shared a 2010 Feature Writing prize from the
Society of Publishers in Asia for the Times' 10-part Uneasy Engagement
series, about China's growing influence in the world.

An essay by Wong was published in Travelers' Tales: Tibet, an anthology
of travel writing on Tibet. Wong appears in Laura Poitras' 2006
documentary about the Iraq War, My Country, My Country, and in Dexter
Filkins' book, The Forever War. He has appeared on The NewsHour with Jim
Lehrer and The Charlie Rose Show, and speaks regularly on NPR, BBC and
CBC.

\section{Notes}\label{notes}
