\textbf{From Wikipedia, the free encyclopedia}

https://en.wikipedia.org/wiki/SpaceX\\
Licensed under CC BY-SA 3.0:\\
https://en.wikipedia.org/wiki/Wikipedia:Text\_of\_Creative\_Commons\_Attribution-ShareAlike\_3.0\_Unported\_License

\section{SpaceX}\label{spacex}

\begin{itemize}
\item
  \emph{In March 2017, SpaceX became the first to successfully re-launch
  and land the first stage of an orbital rocket.}
\item
  \emph{SpaceX conducted the maiden launch of its Crew Dragon spacecraft
  on a NASA-required demonstration flight on March 2, 2019 and is set to
  launch its first crewed Crew Dragon later in 2019.}
\item
  \emph{SpaceX's achievements include the first privately funded
  liquid-propellant rocket to reach orbit (Falcon 1 in 2008), the first
  private company to successfully launch, orbit, and recover a
  spacecraft (Dragon in 2010), the first private company to send a
  spacecraft to the International Space Station (Dragon in 2012), the
  first propulsive landing for an orbital rocket (Falcon 9 in 2015), the
  first reuse of an orbital rocket (Falcon 9 in 2017), and the first
  private company to launch an object into orbit around the sun (Falcon
  Heavy's payload of a Tesla Roadster in 2018).}
\end{itemize}

Space Exploration Technologies Corp., doing business as SpaceX, is a
private American aerospace manufacturer and space transportation
services company headquartered in Hawthorne, California. It was founded
in 2002 by entrepreneur Elon Musk with the goal of reducing space
transportation costs and enabling the colonization of Mars. SpaceX has
since developed the Falcon launch vehicle family and the Dragon
spacecraft family, which both currently deliver payloads into Earth
orbit.

SpaceX's achievements include the first privately funded
liquid-propellant rocket to reach orbit (Falcon 1 in 2008), the first
private company to successfully launch, orbit, and recover a spacecraft
(Dragon in 2010), the first private company to send a spacecraft to the
International Space Station (Dragon in 2012), the first propulsive
landing for an orbital rocket (Falcon 9 in 2015), the first reuse of an
orbital rocket (Falcon 9 in 2017), and the first private company to
launch an object into orbit around the sun (Falcon Heavy's payload of a
Tesla Roadster in 2018). SpaceX has flown 16 resupply missions to the
International Space Station (ISS) under a partnership with NASA. NASA
also awarded SpaceX a further development contract in 2011 to develop
and demonstrate a human-rated Dragon, which would be used to transport
astronauts to the ISS and return them safely to Earth. SpaceX conducted
the maiden launch of its Crew Dragon spacecraft on a NASA-required
demonstration flight on March 2, 2019 and is set to launch its first
crewed Crew Dragon later in 2019. On 11 March at 8:45 a.m. EST, the
SpaceX Crew Dragon completed its first uncrewed flight that
splash-landed in the Atlantic. The flight named Crew Dragon Demo-1 has
demonstrated the Crew Dragon's ability to safely transport crew to ISS
and back..

SpaceX announced in 2011 that it was beginning a reusable launch system
technology development program. In December 2015, the first Falcon 9 was
flown back to a landing pad near the launch site, where it successfully
accomplished a propulsive vertical landing. This was the first such
achievement by a rocket for orbital spaceflight. In April 2016, with the
launch of CRS-8, SpaceX successfully vertically landed the first stage
on an ocean drone ship landing platform. In May 2016, in another first,
SpaceX again landed the first stage, but during a significantly more
energetic geostationary transfer orbit mission. In March 2017, SpaceX
became the first to successfully re-launch and land the first stage of
an orbital rocket.

In September 2016, CEO Elon Musk unveiled the mission architecture of
the Interplanetary Transport System program, an ambitious privately
funded initiative to develop spaceflight technology for use in crewed
interplanetary spaceflight. In 2017, Musk unveiled an updated
configuration of the system, now named Starship and Super Heavy, which
is planned to be fully reusable and will be the largest rocket ever on
its debut, currently scheduled for the early 2020s.

\includegraphics[width=5.50000in,height=3.66667in]{media/image1.jpg}\\
\emph{SpaceX employees with the Dragon capsule at SpaceX HQ in
Hawthorne, California, February 2015}

\includegraphics[width=3.66667in,height=5.50000in]{media/image2.jpg}\\
\emph{Launch of Falcon 9 carrying ORBCOMM OG2-M1}

\includegraphics[width=4.12133in,height=5.50000in]{media/image3.jpg}\\
\emph{Falcon 9 rocket's first stage on the landing pad after the second
successful vertical landing of an orbital rocket stage, OG2 Mission.}

\includegraphics[width=5.50000in,height=3.66667in]{media/image4.jpg}\\
\emph{Falcon 9 first stage on an ASDS barge after the first successful
landing at sea, CRS-8 Mission.}

\section{History}\label{history}

\begin{itemize}
\item
  \emph{As of March 2018, SpaceX had over 100 launches on its manifest
  representing about \$12 billion in contract revenue.}
\item
  \emph{Musk approached rocket engineer Tom Mueller (later SpaceX's CTO
  of Propulsion) and Mueller agreed to work for Musk, and thus SpaceX
  was born.}
\item
  \emph{SpaceX was first headquartered in a warehouse in El Segundo,
  California.}
\end{itemize}

In 2001, Elon Musk conceptualized Mars Oasis, a project to land a
miniature experimental greenhouse and grow plants on Mars. "This would
be the furthest that life's ever traveled" in an attempt to regain
public interest in space exploration and increase the budget of NASA.
Musk tried to buy cheap rockets from Russia but returned empty-handed
after failing to find rockets for an affordable price.\\
On the flight home, Musk realized that he could start a company that
could build the affordable rockets he needed. According to early Tesla
and SpaceX investor Steve Jurvetson, Musk calculated that the raw
materials for building a rocket actually were only three percent of the
sales price of a rocket at the time. By applying vertical integration,
producing around 85\% of launch hardware in-house, and the modular
approach from software engineering, SpaceX could cut launch price by a
factor of ten and still enjoy a 70\% gross margin.

In early 2002, Musk was seeking staff for his new space company, soon to
be named SpaceX. Musk approached rocket engineer Tom Mueller (later
SpaceX's CTO of Propulsion) and Mueller agreed to work for Musk, and
thus SpaceX was born. SpaceX was first headquartered in a warehouse in
El Segundo, California. The company has grown rapidly since it was
founded in 2002, growing from 160 employees in November 2005 to 1,100 in
2010, 3,800 employees and contractors by October 2013, nearly 5,000 by
late 2015, and about 6,000 in April 2017.\\
As of November~2017{[}update{]}, the company had grown to nearly
7,000.\\
In 2016, Musk gave a speech at the International Astronautical Congress,
where he explained that the US government regulates rocket technology as
an "advanced weapon technology", making it difficult to hire
non-Americans.

As of March 2018, SpaceX had over 100 launches on its manifest
representing about \$12 billion in contract revenue. The contracts
included both commercial and government (NASA/DOD) customers. In late
2013, space industry media quoted Musk's comments on SpaceX
"forcing\ldots{}increased competitiveness in the launch industry," its
major competitors in the commercial comsat launch market being
Arianespace, United Launch Alliance, and International Launch Services.
At the same time, Musk also said that the increased competition would
"be a good thing for the future of space." Currently, SpaceX is the
leading global commercial launch provider measured by manifested
launches.

\includegraphics[width=3.44667in,height=5.50000in]{media/image5.jpg}\\
\emph{Falcon Heavy Rocket on Launch Pad 39-A in Cape Canaveral, FL}

\section{Goals}\label{goals}

\begin{itemize}
\item
  \emph{In 2017, SpaceX formed a subsidiary, The Boring Company,\\
  and began work to construct a short underground test tunnel on and
  adjacent to the SpaceX headquarters and manufacturing facility,
  utilizing a small number of SpaceX employees, which was completed in
  May 2018, and opened to the public in December 2018.}
\item
  \emph{In 2015, SpaceX successfully landed the first orbital rocket
  stage on December 21.}
\item
  \emph{To date, SpaceX has successfully landed 25 boosters: 23 Falcon 9
  and 5 Falcon Heavy.}
\item
  \emph{A major goal of SpaceX has been to develop a rapidly reusable
  launch system.}
\end{itemize}

Musk has stated that one of his goals is to decrease the cost and
improve the reliability of access to space, ultimately by a factor of
ten. CEO Elon Musk said: "I believe \$500 per pound (\$1,100/kg) or less
is very achievable."

A major goal of SpaceX has been to develop a rapidly reusable launch
system. As of March~2013{[}update{]}, the publicly announced aspects of
this technology development effort include an active test campaign of
the low-altitude, low-speed Grasshopper vertical takeoff, vertical
landing (VTVL) technology demonstrator rocket, and a high-altitude,
high-speed Falcon 9 post-mission booster return test campaign. In 2015,
SpaceX successfully landed the first orbital rocket stage on December
21. To date, SpaceX has successfully landed 25 boosters: 23 Falcon 9 and
5 Falcon Heavy.{[}citation needed{]}

In 2017, SpaceX formed a subsidiary, The Boring Company,\\
and began work to construct a short underground test tunnel on and
adjacent to the SpaceX headquarters and manufacturing facility,
utilizing a small number of SpaceX employees, which was completed in May
2018, and opened to the public in December 2018.\\
During 2018, The Boring Company was spun out into a separate corporate
entity with 6\% of the equity going to SpaceX, less than 10\% to early
employees, and the remainder of the equity to Elon Musk.

At the 2017 International Astronautical Congress in Adelaide, Australia,
Musk announced his plans to build large spaceships to reach Mars. Using
the BFR, Musk plans to land at least two uncrewed cargo ships to Mars in
2022. The first missions will be used to seek out sources of water and
build a propellant plant. In 2024, Musk plans to fly four additional
ships to Mars including the first people. From there, additional
missions would work to establish a Mars colony. Musk's advocacy for the
long-term settlement of Mars, goes far beyond what SpaceX projects to
build; a successful colonization would ultimately involve many more
economic actors---whether individuals, companies, or governments---to
facilitate the growth of the human presence on Mars over many decades.

\section{Achievements}\label{achievements}

\begin{itemize}
\item
  \emph{The first relaunch and landing of a used orbital rocket stage
  (B1021 on Falcon 9 flight 32 on March 30, 2017)}
\item
  \emph{Landmark achievements of SpaceX include:}
\item
  \emph{The first landing of an orbital rocket's first stage on land
  (Falcon 9 flight 20 on December 22, 2015)}
\item
  \emph{The first controlled flyback and recovery of a payload fairing
  (Falcon 9 flight 32 on March 30, 2017)}
\end{itemize}

Landmark achievements of SpaceX include:

The first privately funded liquid-fueled rocket to reach orbit (Falcon 1
flight 4 on September 28, 2008)

The first privately developed liquid-fueled rocket to put a commercial
satellite in orbit (RazakSAT on Falcon 1 flight 5 on July 14, 2009)

The first private company to successfully launch, orbit, and recover a
spacecraft (Dragon capsule on COTS demo flight 1 on December 9, 2010)

The first private company to send a spacecraft to the International
Space Station (Dragon C2+ on May 25, 2012)

The first private company to send a satellite into geosynchronous orbit
(SES-8 on Falcon 9 flight 7 on December 3, 2013)

The first landing of an orbital rocket's first stage on land (Falcon 9
flight 20 on December 22, 2015)

The first landing of an orbital rocket's first stage on an ocean
platform (Falcon 9 flight 23 on April 8, 2016)

The first relaunch and landing of a used orbital rocket stage (B1021 on
Falcon 9 flight 32 on March 30, 2017)

The first controlled flyback and recovery of a payload fairing (Falcon 9
flight 32 on March 30, 2017)

The first reflight of a commercial cargo spacecraft. (Dragon C106 on
CRS-11 mission on June 3, 2017)

The first private company to send a human-rated spacecraft to space
(Crew Dragon Demo-1 Mission, SpX Flight 72 on Falcon 9 flight 69 on
March 2, 2019) and the first private company to autonomously dock a
spacecraft to the International Space Station (same flight on March 3,
2019)

\section{Setbacks}\label{setbacks}

\begin{itemize}
\item
  \emph{Though not considered an unsuccessful flight, the rocket
  explosion sent the company into a four-month launch hiatus while it
  worked out what went wrong, and SpaceX returned to flight in January
  2017.}
\item
  \emph{Musk described the event as the "most difficult and complex
  failure" ever in SpaceX's history; SpaceX reviewed nearly 3,000
  channels of telemetry and video data covering a period of 35--55
  milliseconds for the postmortem.}
\end{itemize}

In March 2013, a Dragon spacecraft in orbit developed issues with its
thrusters that limited its control capabilities. SpaceX engineers were
able to remotely clear the blockages within a short period, and the
spacecraft was able to successfully complete its mission to and from the
International Space Station.

In June 2015, CRS-7 launched a Dragon capsule atop a Falcon 9 to
resupply the International Space Station. All telemetry readings were
nominal until 2 minutes and 19 seconds into the flight, when a loss of
helium pressure was detected and a cloud of vapor appeared outside the
second stage. A few seconds after this, the second stage exploded. The
first stage continued to fly for a few seconds before disintegrating due
to aerodynamic forces. The capsule was thrown off and survived the
explosion, transmitting data until it was destroyed on impact. Later it
was revealed that the capsule could have landed intact if it had
software to deploy its parachutes in case of a launch mishap. The
problem was discovered to be a failed 2-foot-long steel strut purchased
from a supplier to hold a helium pressure vessel that broke free due to
the force of acceleration. This caused a breach and allowed
high-pressure helium to escape into the low-pressure propellant tank,
causing the failure. The Dragon software issue was also fixed in
addition to an analysis of the entire program in order to ensure proper
abort mechanisms are in place for future rockets and their payload.

In September 2016, a Falcon 9 exploded during a propellant fill
operation for a standard pre-launch static fire test. The payload, the
Spacecom Amos-6 communications satellite valued at \$200 million, was
destroyed. Musk described the event as the "most difficult and complex
failure" ever in SpaceX's history; SpaceX reviewed nearly 3,000 channels
of telemetry and video data covering a period of 35--55 milliseconds for
the postmortem. Musk reported the explosion was caused by the liquid
oxygen that is used as propellant turning so cold that it solidified and
it ignited with carbon composite helium vessels. Though not considered
an unsuccessful flight, the rocket explosion sent the company into a
four-month launch hiatus while it worked out what went wrong, and SpaceX
returned to flight in January 2017.

\section{Ownership, funding and
valuation}\label{ownership-funding-and-valuation}

\begin{itemize}
\item
  \emph{Congressional testimony by SpaceX in 2017 suggested that the
  NASA Space Act Agreement process of "setting only a high-level
  requirement for cargo transport to the space station {[}while{]}
  leaving the details to industry" had allowed SpaceX to design and
  develop the Falcon 9 rocket on its own at substantially lower cost.}
\end{itemize}

In August 2008, SpaceX accepted a \$20 million investment from Founders
Fund. In early 2012, approximately two-thirds of the company were owned
by its founder and his 70 million shares were then estimated to be worth
\$875 million on private markets, which roughly valued SpaceX at \$1.3
billion as of February 2012. After the COTS 2+ flight in May 2012, the
company private equity valuation nearly doubled to \$2.4 billion. In
January 2015, SpaceX raised \$1 billion in funding from Google and
Fidelity, in exchange for 8.333\% of the company, establishing the
company valuation at approximately \$12 billion. Google and Fidelity
joined prior investors Draper Fisher Jurvetson, Founders Fund, Valor
Equity Partners and Capricorn. In July 2017, the Company raised US\$350m
at a valuation of US\$21 billion.

As of May~2012{[}update{]}, SpaceX had operated on total funding of
approximately \$1 billion in its first ten years of operation. Of this,
private equity provided about \$200M, with Musk investing approximately
\$100M and other investors having put in about \$100M (Founders Fund,
Draper Fisher Jurvetson, ...). The remainder has come from progress
payments on long-term launch contracts and development contracts. By
March 2018, SpaceX had contracts for 100 launch missions, and each of
those contracts provide down payments at contract signing, plus many are
paying progress payments as launch vehicle components are built in
advance of mission launch, driven in part by US accounting rules for
recognizing long-term revenue.

Congressional testimony by SpaceX in 2017 suggested that the NASA Space
Act Agreement process of "setting only a high-level requirement for
cargo transport to the space station {[}while{]} leaving the details to
industry" had allowed SpaceX to design and develop the Falcon 9 rocket
on its own at substantially lower cost. "According to NASA's own
independently verified numbers, SpaceX's development costs of both the
Falcon 1 and Falcon 9 rockets were estimated at approximately US\$390
million in total. "In 2011, NASA estimated that it would have cost the
agency about US\$4 billion to develop a rocket like the Falcon 9 booster
based upon NASA's traditional contracting processes". The Falcon 9
launch system, with an estimated improvement at least four to ten times
over traditional cost-plus contracting estimates, about \$400 million
vs. \$4 billion in savings through the usage of Space Act Agreements.

In April 2019, the Wall Street Journal reported that SpaceX was raising
another \$500 million in funding. In May Space News reported SpaceX
"raised \$1.022 billion" the day after SpaceX launched 60 Satellites
towards their 12,000 satellite plan named Starlink broadband
constellation.

\section{Spacecraft and flight
hardware}\label{spacecraft-and-flight-hardware}

\begin{itemize}
\item
  \emph{SpaceX also manufactures the Dragon, a pressurized orbital
  spacecraft that is launched on top of a Falcon 9 booster to carry
  cargo to low Earth orbit, and the follow-on Dragon 2 spacecraft, or
  Crew Dragon, currently in the process of being human-rated through a
  variety of design reviews and flight tests that began in 2014.}
\item
  \emph{The Merlin powers their two main space launch vehicles: the
  Falcon 9, which flew successfully into orbit on its maiden launch in
  June 2010 and the super-heavy class Falcon Heavy, which was launched
  for the first time on February 6, 2018.}
\end{itemize}

SpaceX currently manufactures three broad classes of rocket engine
in-house: the kerosene fueled Merlin engines, the methane fueled Raptor
engines, and the hypergolic fueled Draco/SuperDraco vernier thrusters.
The Merlin powers their two main space launch vehicles: the Falcon 9,
which flew successfully into orbit on its maiden launch in June 2010 and
the super-heavy class Falcon Heavy, which was launched for the first
time on February 6, 2018. SpaceX also manufactures the Dragon, a
pressurized orbital spacecraft that is launched on top of a Falcon 9
booster to carry cargo to low Earth orbit, and the follow-on Dragon 2
spacecraft, or Crew Dragon, currently in the process of being
human-rated through a variety of design reviews and flight tests that
began in 2014.

\section{Rocket engines}\label{rocket-engines}

\begin{itemize}
\item
  \emph{Merlin is a family of rocket engines developed by SpaceX for use
  on its Falcon rocket family.}
\item
  \emph{SpaceX is currently developing two further rocket engines:
  SuperDraco and Raptor.}
\item
  \emph{SpaceX is currently the world's most prolific producer of liquid
  fuel rocket engines.}
\item
  \emph{Kestrel is a LOX/RP-1 pressure-fed rocket engine, and was used
  as the Falcon 1 rocket's second stage main engine.}
\end{itemize}

Since the founding of SpaceX in 2002, the company has developed three
families of rocket engines~--- Merlin and the retired Kestrel for launch
vehicle propulsion, and the Draco control thrusters. SpaceX is currently
developing two further rocket engines: SuperDraco and Raptor. SpaceX is
currently the world's most prolific producer of liquid fuel rocket
engines.

Merlin is a family of rocket engines developed by SpaceX for use on its
Falcon rocket family. Merlin engines use LOX and RP-1 as propellants in
a gas-generator power cycle. The Merlin engine was originally designed
for sea recovery and reuse. The injector at the heart of Merlin is of
the pintle type that was first used in the Apollo Program for the lunar
module landing engine. Propellants are fed via a single shaft, dual
impeller turbo-pump.

Kestrel is a LOX/RP-1 pressure-fed rocket engine, and was used as the
Falcon 1 rocket's second stage main engine. It is built around the same
pintle architecture as SpaceX's Merlin engine but does not have a
turbo-pump, and is fed only by tank pressure. Its nozzle is ablatively
cooled in the chamber and throat, is also radiatively cooled, and is
fabricated from a high strength niobium alloy.

Both names for the Merlin and Kestrel engines are derived from species
of North American falcons: the kestrel and the merlin.

Draco are hypergolic liquid-propellant rocket engines that utilize
monomethyl hydrazine fuel and nitrogen tetroxide oxidizer. Each Draco
thruster generates 400 newtons (90~lbf) of thrust. They are used as
reaction control system (RCS) thrusters on the Dragon spacecraft.
SuperDraco engines are a much more powerful version of the Draco
thrusters, which were initially meant to be used as landing and launch
escape system engines on the version 2 Dragon spacecraft, Dragon 2. The
concept of using retro-rockets for landing was scrapped in 2017 when it
was decided to perform a traditional parachute descent and splashdown at
sea.

Raptor is a new family of methane-fueled full flow staged combustion
cycle engines to be used in its future Interplanetary Transport System.
Development versions were test fired in late 2016. On April 3, 2019,
SpaceX conducted a successful static fire test in Texas on its
Starhopper vehicle, which ignited the engine while the vehicle remained
tethered to the ground.

\section{Falcon launch vehicles}\label{falcon-launch-vehicles}

\begin{itemize}
\item
  \emph{At the time of its first launch, SpaceX described their Falcon
  Heavy as "the world's most powerful rocket in operation".}
\item
  \emph{Since 2010, SpaceX has flown all its missions on the Falcon 9,
  except one test flight and one operational flight of Falcon Heavy.}
\item
  \emph{In 2011, SpaceX began development of the Falcon Heavy, a
  heavy-lift rocket configured using a cluster of three Falcon 9 first
  stage cores with a total 27 Merlin 1D engines and propellant
  crossfeed.}
\end{itemize}

Since 2010, SpaceX has flown all its missions on the Falcon 9, except
one test flight and one operational flight of Falcon Heavy. They
previously developed and flew the Falcon 1 pathfinder vehicle.

Falcon 1 was a small rocket capable of placing several hundred kilograms
into low earth orbit. It functioned as an early test-bed for developing
concepts and components for the larger Falcon 9. Falcon 1 attempted five
flights between 2006 and 2009. With Falcon I, when Musk announced his
plans for it before a subcommittee in the Senate in 2004, he discussed
that Falcon I would be the 'worlds only semi-reusable orbital rocket'
apart from the space shuttle. On September 28, 2008, on its fourth
attempt, the Falcon 1 successfully reached orbit, becoming the first
privately funded, liquid-fueled rocket to do so.

Falcon 9 is an EELV-class medium-lift vehicle capable of delivering up
to 22,800 kilograms (50,265~lb) to orbit, and is intended to compete
with the Delta IV and the Atlas V rockets, as well as other launch
providers around the world. It has nine Merlin engines in its first
stage. The Falcon 9 v1.0 rocket successfully reached orbit on its first
attempt on June 4, 2010. Its third flight, COTS Demo Flight 2, launched
on May 22, 2012, and was the first commercial spacecraft to reach and
dock with the International Space Station. The vehicle was upgraded to
Falcon 9 v1.1 in 2013 and again in 2015 to the current Falcon 9 Full
Thrust version. As of February~2018{[}update{]}, Falcon 9 vehicles have
flown 49 successful missions with one failure, the CRS-7 mission. An
additional vehicle was destroyed during a routine test several days
prior to a scheduled launch in 2016.

In 2011, SpaceX began development of the Falcon Heavy, a heavy-lift
rocket configured using a cluster of three Falcon 9 first stage cores
with a total 27 Merlin 1D engines and propellant crossfeed. The Falcon
Heavy successfully flew on its inaugural mission on February 6, 2018
with a payload consisting of Musk's personal Tesla Roadster into
heliocentric orbit The first stage would be capable of lifting 63,800
kilograms (140,660~lb) to LEO with the 27 Merlin 1D engines producing
22,819 kN of thrust at sea level, and 24,681~kN in space. At the time of
its first launch, SpaceX described their Falcon Heavy as "the world's
most powerful rocket in operation".

\includegraphics[width=5.50000in,height=4.37434in]{media/image6.jpg}\\
\emph{The Dragon spacecraft approaching the ISS}

\includegraphics[width=5.50000in,height=3.65771in]{media/image7.jpg}\\
\emph{The interior of the COTS 2 Dragon}

\section{Dragon capsules}\label{dragon-capsules}

\begin{itemize}
\item
  \emph{In September 2017, Elon Musk released first prototype images of
  their space suits to be used in future missions.}
\item
  \emph{SpaceX demonstrated cargo resupply and eventually crew
  transportation services using the Dragon.}
\item
  \emph{SpaceX conducted a test of an empty Crew Dragon to ISS in early
  2019, and later in the year they plan to launch a crewed Dragon which
  will send US astronauts to the ISS for the first time since the
  retirement of the Space Shuttle.}
\item
  \emph{The Crew Dragon spacecraft was first sent to space on March 2,
  2019.}
\end{itemize}

In 2005, SpaceX announced plans to pursue a human-rated commercial space
program through the end of the decade. The Dragon is a conventional
blunt-cone ballistic capsule which is capable of carrying cargo or up to
seven astronauts into orbit and beyond.

In 2006, NASA announced that the company was one of two selected to
provide crew and cargo resupply demonstration contracts to the ISS under
the COTS program. SpaceX demonstrated cargo resupply and eventually crew
transportation services using the Dragon. The first flight of a Dragon
structural test article took place in June 2010, from Launch Complex 40
at Cape Canaveral Air Force Station during the maiden flight of the
Falcon 9 launch vehicle; the mock-up Dragon lacked avionics, heat
shield, and other key elements normally required of a fully operational
spacecraft but contained all the necessary characteristics to validate
the flight performance of the launch vehicle. An operational Dragon
spacecraft was launched in December 2010 aboard COTS Demo Flight 1, the
Falcon 9's second flight, and safely returned to Earth after two orbits,
completing all its mission objectives. In 2012, Dragon became the first
commercial spacecraft to deliver cargo to the International Space
Station, and has since been conducting regular resupply services to the
ISS.

In April 2011, NASA issued a \$75 million contract, as part of its
second-round commercial crew development (CCDev) program, for SpaceX to
develop an integrated launch escape system for Dragon in preparation for
human-rating it as a crew transport vehicle to the ISS. In August 2012,
NASA awarded SpaceX a firm, fixed-price SAA with the objective of
producing a detailed design of the entire crew transportation system.
This contract includes numerous key technical and certification
milestones, an uncrewed flight test, a crewed flight test, and six
operational missions following system certification. The fully
autonomous Crew Dragon spacecraft is expected to be one of the safest
crewed spacecraft systems. Reusable in nature, the Crew Dragon will
offer savings to NASA.

SpaceX conducted a test of an empty Crew Dragon to ISS in early 2019,
and later in the year they plan to launch a crewed Dragon which will
send US astronauts to the ISS for the first time since the retirement of
the Space Shuttle. In February 2017 SpaceX announced that two would-be
space tourists had put down "significant deposits" for a mission which
would see the two tourists fly on board a Dragon capsule around the Moon
and back again.

In addition to SpaceX's privately funded plans for an eventual Mars
mission, NASA Ames Research Center had developed a concept called Red
Dragon: a low-cost Mars mission that would use Falcon Heavy as the
launch vehicle and trans-Martian injection vehicle, and the Dragon
capsule to enter the Martian atmosphere. The concept was originally
envisioned for launch in 2018 as a NASA Discovery mission, then
alternatively for 2022 The objectives of the mission would be return the
samples from Mars to Earth at a fraction of the cost of the NASA own
return-sample mission now projected at 6 billion dollars.

In September 2017, Elon Musk released first prototype images of their
space suits to be used in future missions. The suit is in testing phase
and it is designed to cope with 2 ATM pressure in vacuum.

The Crew Dragon spacecraft was first sent to space on March 2, 2019.

\includegraphics[width=5.50000in,height=3.66667in]{media/image8.jpg}\\
\emph{First test firing of a scale Raptor development engine in
September 2016 in McGregor, Texas.}

\section{Research and development}\label{research-and-development}

\begin{itemize}
\item
  \emph{For example, at the 2015 GPU Technology Conference, SpaceX
  revealed their own computational fluid dynamics (CFD) software to
  improve the simulation capability of evaluating rocket engine
  combustion design.}
\item
  \emph{SpaceX has on occasion developed new engineering development
  technologies to enable it to pursue its various goals.}
\item
  \emph{SpaceX is actively pursuing several different research and
  development programs.}
\end{itemize}

SpaceX is actively pursuing several different research and development
programs. Most notable are those intended to develop reusable launch
vehicles, an interplanetary transport system and a global
telecommunications network.

SpaceX has on occasion developed new engineering development
technologies to enable it to pursue its various goals. For example, at
the 2015 GPU Technology Conference, SpaceX revealed their own
computational fluid dynamics (CFD) software to improve the simulation
capability of evaluating rocket engine combustion design.

\includegraphics[width=5.50000in,height=3.92857in]{media/image9.jpg}\\
\emph{Autonomous spaceport drone ship in position prior to Falcon 9
Flight 17 carrying CRS-6.}

\section{Reusable launch system}\label{reusable-launch-system}

\begin{itemize}
\item
  \emph{As a result of Elon Musk's goal of crafting more cost-effective
  launch vehicles, SpaceX conceived a method to reuse the first stage of
  their primary rocket, the Falcon 9, by attempting propulsive vertical
  landings on solid surfaces.}
\item
  \emph{On March 30, 2017, SpaceX launched a "flight-proven" Falcon 9
  for the SES-10 mission.}
\item
  \emph{SpaceX continues to carry out first stage landings on every
  orbital launch that fuel margins allow.}
\end{itemize}

SpaceX's reusable launcher program was publicly announced in 2011 and
the design phase was completed in February 2012. The system returns the
first stage of a Falcon 9 rocket to a predetermined landing site using
only its own propulsion systems.

SpaceX's active test program began in late 2012 with testing
low-altitude, low-speed aspects of the landing technology. Grasshopper
and the Falcon 9 Reusable Development Vehicles (F9R Dev) were
experimental technology-demonstrator reusable rockets that performed
vertical takeoffs and landings.

High-velocity, high-altitude aspects of the booster atmospheric return
technology began testing in late 2013 and have continued through 2018,
with a 98\% success rate to date. As a result of Elon Musk's goal of
crafting more cost-effective launch vehicles, SpaceX conceived a method
to reuse the first stage of their primary rocket, the Falcon 9, by
attempting propulsive vertical landings on solid surfaces. Once the
company determined that soft landings were feasible by touching down
over the Atlantic and Pacific Ocean, they began landing attempts on a
solid platform. SpaceX leased and modified several barges to sit out at
sea as a target for the returning first stage, converting them to
autonomous spaceport drone ships (ASDS). SpaceX first achieved a
successful landing and recovery of a first stage in December 2015, and
in April 2016, the first stage booster first successfully landed on the
ASDS Of Course I Still Love You.

SpaceX continues to carry out first stage landings on every orbital
launch that fuel margins allow. By October 2016, following the
successful landings, SpaceX indicated they were offering their customers
a ten percent price discount if they choose to fly their payload on a
reused Falcon 9 first stage. On March 30, 2017, SpaceX launched a
"flight-proven" Falcon 9 for the SES-10 mission. This was the first time
a re-launch of a payload-carrying orbital rocket went back to space. The
first stage was recovered and landed on the ASDS Of Course I Still Love
You in the Atlantic Ocean, also making it the first landing of a reused
orbital class rocket. Elon Musk called the achievement an "incredible
milestone in the history of space."

\includegraphics[width=5.50000in,height=2.42000in]{media/image10.jpg}\\
\emph{Artist's impression of the Interplanetary Starship on the Jovian
moon Europa.}

\section{Interplanetary Transport System /
BFR}\label{interplanetary-transport-system-bfr}

\begin{itemize}
\item
  \emph{SpaceX is developing a super-heavy lift launch system, the BFR.}
\item
  \emph{SpaceX initially envisioned the ITS vehicle design which was
  solely aimed at Mars transit and other interplanetary uses, SpaceX in
  2017 began to focus on a vehicle support all SpaceX launch service
  provider capabilities: Earth-orbit, lunar-orbit, interplanetary
  missions, and even intercontinental passenger transport on Earth.}
\end{itemize}

SpaceX is developing a super-heavy lift launch system, the BFR. The BFR
is a fully reusable first stage launch vehicle and spacecraft intended
to replace all of the company's existing hardware by the early 2020s,
ground infrastructure for rapid launch and relaunch, and zero-gravity
propellant transfer technology in low Earth orbit (LEO).

SpaceX initially envisioned the ITS vehicle design which was solely
aimed at Mars transit and other interplanetary uses, SpaceX in 2017
began to focus on a vehicle support all SpaceX launch service provider
capabilities: Earth-orbit, lunar-orbit, interplanetary missions, and
even intercontinental passenger transport on Earth. Private passenger
Yusaku Maezawa has been signed to fly around the Moon in the BFR rocket.

Musk's long term vision for the company is the development of technology
and resources suitable for human colonization on Mars. He has expressed
his interest in someday traveling to the planet, stating "I'd like to
die on Mars, just not on impact." A rocket every two years or so could
provide a base for the people arriving in 2025 after a launch in 2024.
According to Steve Jurvetson, Musk believes that by 2035 at the latest,
there will be thousands of rockets flying a million people to Mars, in
order to enable a self-sustaining human colony.

\section{Other projects}\label{other-projects}

\begin{itemize}
\item
  \emph{\textless{}ref\textgreater{} Chris Gebhardt,
  {[}https://www.nasaspaceflight.com/2019/05/first-starlink-mission-heaviest-payload-launch-spacex/
  "Falcon 9 launches first Starlink mission -- heaviest payload launch
  by SpaceX to date"{]}, ''NASASpaceflight.com'', May 23,
  2019\textless{}/ref\textgreater{}}
\item
  \emph{Owned and operated by SpaceX, the goal of the business is to
  increase profitability and cashflow, to allow SpaceX to build its Mars
  colony.}
\item
  \emph{In May 2019, SpaceX launched the first batch of 60 satellites
  aboard a Falcon 9 from Cape Canaveral, FL.}
\end{itemize}

In January 2015, SpaceX CEO Elon Musk announced the development of a new
satellite constellation to provide global broadband internet service. In
June 2015 the company asked the federal government for permission to
begin testing for a project that aims to build a constellation of 4,425
satellites capable of beaming the Internet to the entire globe,
including remote regions which currently do not have Internet access.
The Internet service would use a constellation of 4,425 cross-linked
communications satellites in 1,100~km orbits. Owned and operated by
SpaceX, the goal of the business is to increase profitability and
cashflow, to allow SpaceX to build its Mars colony. Development began in
2015, initial prototype test-flight satellites were launched on the
SpaceX PAZ mission in 2017. Initial operation of the constellation could
begin as early as 2020. As of March~2017{[}update{]}, SpaceX filed with
the US regulatory authorities plans to field a constellation of an
additional 7,518 "V-band satellites in non-geosynchronous orbits to
provide communications services" in an electromagnetic spectrum that had
not previously been "heavily employed for commercial communications
services". Called the "V-band low-Earth-orbit (VLEO) constellation", it
would consist of "7,518 satellites to follow the {[}earlier{]} proposed
4,425 satellites that would function in Ka- and Ku-band".\\
In February 2019, SpaceX formed a sibling company, SpaceX Services,
Inc., to license the manufacture and deployment of up to 1,000,000 fixed
satellite earth stations that will communicate with its Starlink system.
In May 2019, SpaceX launched the first batch of 60 satellites aboard a
Falcon 9 from Cape Canaveral, FL. \textless{}ref\textgreater{} Chris
Gebhardt,
{[}https://www.nasaspaceflight.com/2019/05/first-starlink-mission-heaviest-payload-launch-spacex/
"Falcon 9 launches first Starlink mission -- heaviest payload launch by
SpaceX to date"{]}, ''NASASpaceflight.com'', May 23,
2019\textless{}/ref\textgreater{}

In June 2015, SpaceX announced that they would sponsor a Hyperloop
competition, and would build a 1-mile-long (1.6~km) subscale test track
near SpaceX's headquarters for the competitive events. The first
competitive event was held at the track in January 2017 and the second
in August 2017. And the third in December 2018.

\section{Infrastructure}\label{infrastructure}

\begin{itemize}
\item
  \emph{SpaceX is headquartered in Hawthorne, California, which also
  serves as its primary manufacturing plant.}
\item
  \emph{SpaceX also operates regional offices in Redmond, Texas,
  Virginia, and Washington, D.C.}
\end{itemize}

SpaceX is headquartered in Hawthorne, California, which also serves as
its primary manufacturing plant. The company owns a test site in Texas
and operates three launch sites, with another under development. SpaceX
also operates regional offices in Redmond, Texas, Virginia, and
Washington, D.C.

\includegraphics[width=5.50000in,height=3.73641in]{media/image11.jpg}\\
\emph{Falcon 9 v1.1 rocket cores under construction at the SpaceX
Hawthorne facility, November 2014.}

\section{Headquarters, manufacturing and refurbishment
facilities}\label{headquarters-manufacturing-and-refurbishment-facilities}

\begin{itemize}
\item
  \emph{In June 2017, SpaceX announced they would construct a facility
  on 0.88 hectares (2.17 acres) in Port Canaveral Florida for
  refurbishment and storage of previously-flown Falcon 9 and Falcon
  Heavy booster cores.}
\item
  \emph{Nevertheless, SpaceX still has over 3,000 suppliers with some
  1,100 of those delivering to SpaceX nearly weekly.}
\end{itemize}

SpaceX Headquarters is located in the Los Angeles suburb of Hawthorne,
California. The large three-story facility, originally built by Northrop
Corporation to build Boeing 747 fuselages, houses SpaceX's office space,
mission control, and, as of 2018, all vehicle manufacturing. In March
2018, SpaceX indicated that it would manufacture its next-generation,
9~m (30~ft)-diameter launch vehicle, the BFR at a new facility it is
building on the Los Angeles waterfront in the San Pedro area. The
company has leased an 18-acre site near Berth 240 in the Los Angeles
port for 10 years, with multiple renewals possible, and will use the
site for manufacturing, recovery from shipborne landings, and
refurbishment of both the BFR booster and the BFR spaceship.

The area has one of the largest concentrations of aerospace
headquarters, facilities, and/or subsidiaries in the U.S., including
Boeing/McDonnell Douglas main satellite building campuses, Aerospace
Corp., Raytheon, NASA's Jet Propulsion Laboratory, Air Force Space
Command's Space and Missile Systems Center at Los Angeles Air Force
Base, Lockheed Martin, BAE Systems, Northrop Grumman, and AECOM, etc.,
with a large pool of aerospace engineers and recent college engineering
graduates.

SpaceX utilizes a high degree of vertical integration in the production
of its rockets and rocket engines. SpaceX builds its rocket engines,
rocket stages, spacecraft, principal avionics and all software in-house
in their Hawthorne facility, which is unusual for the aerospace
industry. Nevertheless, SpaceX still has over 3,000 suppliers with some
1,100 of those delivering to SpaceX nearly weekly.

In June 2017, SpaceX announced they would construct a facility on 0.88
hectares (2.17 acres) in Port Canaveral Florida for refurbishment and
storage of previously-flown Falcon 9 and Falcon Heavy booster
cores.{[}needs update{]}

\section{Development and test
facilities}\label{development-and-test-facilities}

\begin{itemize}
\item
  \emph{SpaceX operates their first Rocket Development and Test Facility
  in McGregor, Texas.}
\item
  \emph{All SpaceX rocket engines are tested on rocket test stands, and
  low-altitude VTVL flight testing of the Falcon 9 Grasshopper v1.0 and
  F9R Dev1 test vehicles in 2013--2014 were carried out at McGregor.}
\item
  \emph{In the event, SpaceX decided to forego building another nose
  cone for the first test article, because at the low velocities planned
  for that rocket, it was unnecessary.}
\end{itemize}

SpaceX operates their first Rocket Development and Test Facility in
McGregor, Texas. All SpaceX rocket engines are tested on rocket test
stands, and low-altitude VTVL flight testing of the Falcon 9 Grasshopper
v1.0 and F9R Dev1 test vehicles in 2013--2014 were carried out at
McGregor. 2019 low-altitude VTVL testing of the much larger 9-meter
(30~ft)-diameter "Starhopper" is planned to occur at the SpaceX South
Texas Launch Site near Brownsville, Texas, which is currently under
construction. On January 23, 2019, strong winds at the Texas test launch
site blew over the nose cone over the first test article rocket, causing
delays that will take weeks to repair according to SpaceX
representatives. In the event, SpaceX decided to forego building another
nose cone for the first test article, because at the low velocities
planned for that rocket, it was unnecessary.

The company purchased the McGregor facilities from Beal Aerospace, where
it refitted the largest test stand for Falcon 9 engine testing. SpaceX
has made a number of improvements to the facility since purchase, and
has also extended the acreage by purchasing several pieces of adjacent
farmland. In 2011, the company announced plans to upgrade the facility
for launch testing a VTVL rocket, and then constructed a half-acre
concrete launch facility in 2012 to support the Grasshopper test flight
program. As of October~2012{[}update{]}, the McGregor facility had seven
test stands that are operated "18 hours a day, six days a week" and is
building more test stands because production is ramping up and the
company has a large manifest in the next several years. {[}needs
update{]}

In addition to routine testing, Dragon capsules (following recovery
after an orbital mission), are shipped to McGregor for de-fueling,
cleanup, and refurbishment for reuse in future missions.

\includegraphics[width=5.50000in,height=3.66667in]{media/image12.jpg}\\
\emph{SpaceX west coast launch facility at Vandenberg Air Force Base,
during the launch of CASSIOPE, September 2013.}

\section{Launch facilities}\label{launch-facilities}

\begin{itemize}
\item
  \emph{SpaceX has indicated that they see a niche for each of the four
  orbital facilities and that they have sufficient launch business to
  fill each pad.}
\item
  \emph{SpaceX currently operates three orbital launch sites, at Cape
  Canaveral, Vandenberg Air Force Base, and Kennedy Space Center, and is
  under construction on a fourth in Brownsville, Texas.}
\end{itemize}

SpaceX currently operates three orbital launch sites, at Cape Canaveral,
Vandenberg Air Force Base, and Kennedy Space Center, and is under
construction on a fourth in Brownsville, Texas. SpaceX has indicated
that they see a niche for each of the four orbital facilities and that
they have sufficient launch business to fill each pad. The Vandenberg
launch site enables highly inclined orbits (66--145°), while Cape
Canaveral enables orbits of medium inclination, up to 51.6°. Before it
was retired, all Falcon 1 launches took place at the Ronald Reagan
Ballistic Missile Defense Test Site on Omelek Island.

\includegraphics[width=5.50000in,height=3.66667in]{media/image13.jpg}\\
\emph{Falcon 9 Flight 20 landing on Landing Zone 1 in December 2015}

\section{Cape Canaveral}\label{cape-canaveral}

\begin{itemize}
\item
  \emph{Cape Canaveral Air Force Station Space Launch Complex 40
  (SLC-40) is used for Falcon 9 launches to low Earth and geostationary
  orbits.}
\item
  \emph{As part of SpaceX's booster reusability program, the former
  Launch Complex 13 at Cape Canaveral, now renamed Landing Zone 1, has
  been designated for use for Falcon 9 first-stage booster landings.}
\end{itemize}

Cape Canaveral Air Force Station Space Launch Complex 40 (SLC-40) is
used for Falcon 9 launches to low Earth and geostationary orbits. SLC-40
is not capable of supporting Falcon Heavy launches. As part of SpaceX's
booster reusability program, the former Launch Complex 13 at Cape
Canaveral, now renamed Landing Zone 1, has been designated for use for
Falcon 9 first-stage booster landings.

\section{Vandenberg}\label{vandenberg}

\begin{itemize}
\item
  \emph{The neighboring SLC-4W has been converted to Landing Zone 4,
  where SpaceX successfully landed one Falcon 9 first-stage booster, in
  October 2018.}
\item
  \emph{The Vandenberg site can launch both Falcon 9 and Falcon Heavy,
  but cannot launch to low inclination orbits.}
\end{itemize}

Vandenberg Air Force Base Space Launch Complex 4 East (SLC-4E) is used
for payloads to polar orbits. The Vandenberg site can launch both Falcon
9 and Falcon Heavy, but cannot launch to low inclination orbits. The
neighboring SLC-4W has been converted to Landing Zone 4, where SpaceX
successfully landed one Falcon 9 first-stage booster, in October 2018.

\section{Kennedy Space Center}\label{kennedy-space-center}

\begin{itemize}
\item
  \emph{The pad was subsequently modified to support Falcon 9 and Falcon
  Heavy launches.}
\item
  \emph{SpaceX has launched 13 Falcon 9 missions from Launch Pad 39A and
  more recently the Falcon Heavy Rocket, on April 11, 2019.}
\item
  \emph{SpaceX intends to launch the first crewed missions to the ISS
  from Launch Pad 39A in 2019.}
\item
  \emph{On April 14, 2014, SpaceX signed a 20-year lease for Launch Pad
  39A.}
\end{itemize}

On April 14, 2014, SpaceX signed a 20-year lease for Launch Pad 39A. The
pad was subsequently modified to support Falcon 9 and Falcon Heavy
launches. SpaceX has launched 13 Falcon 9 missions from Launch Pad 39A
and more recently the Falcon Heavy Rocket, on April 11, 2019. SpaceX
intends to launch the first crewed missions to the ISS from Launch Pad
39A in 2019.

\section{Brownsville}\label{brownsville}

\begin{itemize}
\item
  \emph{Real estate packages at the location have been named by SpaceX
  with names based on the theme "Mars Crossing".}
\item
  \emph{In August 2014, SpaceX announced they would be building a
  commercial-only launch facility at Brownsville, Texas.}
\item
  \emph{SpaceX started construction on the new launch facility in 2014
  with production ramping up in the latter half of 2015, with the first
  suborbital launches from the facility in 2019.}
\end{itemize}

In August 2014, SpaceX announced they would be building a
commercial-only launch facility at Brownsville, Texas. The Federal
Aviation Administration released a draft Environmental Impact Statement
for the proposed Texas facility in April 2013, and "found that 'no
impacts would occur' that would force the Federal Aviation
Administration to deny SpaceX a permit for rocket operations," and
issued the permit in July 2014. SpaceX started construction on the new
launch facility in 2014 with production ramping up in the latter half of
2015, with the first suborbital launches from the facility in 2019. Real
estate packages at the location have been named by SpaceX with names
based on the theme "Mars Crossing".

\section{Satellite prototyping
facility}\label{satellite-prototyping-facility}

\begin{itemize}
\item
  \emph{In July 2016, SpaceX acquired an additional 740 square meters
  (8,000~sq~ft) creative space in Irvine, California (Orange County) to
  focus on satellite communications.}
\item
  \emph{In January 2015, SpaceX announced it would be entering the
  satellite production business and global satellite internet business.}
\end{itemize}

In January 2015, SpaceX announced it would be entering the satellite
production business and global satellite internet business. The first
satellite facility is a 30,000 square foot (2800m2) office building
located in Redmond, Washington. As of January 2017, a second facility in
Redmond was acquired with 40,625 square feet (3800m2) and has become a
research and development lab for the satellites. In July 2016, SpaceX
acquired an additional 740 square meters (8,000~sq~ft) creative space in
Irvine, California (Orange County) to focus on satellite communications.

\section{Launch contracts}\label{launch-contracts}

\begin{itemize}
\item
  \emph{SpaceX is also certified for US military launches of Evolved
  Expendable Launch Vehicle-class (EELV) payloads.}
\item
  \emph{With approximately 30 missions on manifest for 2018 alone,
  SpaceX represents over \$12B under contract.}
\item
  \emph{SpaceX won demonstration and actual supply contracts from NASA
  for the International Space Station (ISS) with technology the company
  developed.}
\end{itemize}

SpaceX won demonstration and actual supply contracts from NASA for the
International Space Station (ISS) with technology the company developed.
SpaceX is also certified for US military launches of Evolved Expendable
Launch Vehicle-class (EELV) payloads. With approximately 30 missions on
manifest for 2018 alone, SpaceX represents over \$12B under contract.

\section{NASA contracts}\label{nasa-contracts}

\includegraphics[width=5.50000in,height=4.57825in]{media/image14.jpg}\\
\emph{The COTS 2 Dragon is berthed to the ISS by Canadarm2.}

\section{COTS}\label{cots}

\begin{itemize}
\item
  \emph{This contract, designed by NASA to provide "seed money" through
  Space Act Agreements for developing new capabilities, NASA paid SpaceX
  \$396 million to develop the cargo configuration of the Dragon
  spacecraft, while SpaceX self-invested more than \$500 million to
  develop the Falcon 9 launch vehicle.}
\item
  \emph{In December 2010, the launch of the COTS Demo Flight 1 mission,
  SpaceX became the first private company to successfully launch, orbit
  and recover a spacecraft.}
\end{itemize}

In 2006, NASA announced that SpaceX had won a NASA Commercial Orbital
Transportation Services (COTS) Phase 1 contract to demonstrate cargo
delivery to the ISS, with a possible contract option for crew transport.
This contract, designed by NASA to provide "seed money" through Space
Act Agreements for developing new capabilities, NASA paid SpaceX \$396
million to develop the cargo configuration of the Dragon spacecraft,
while SpaceX self-invested more than \$500 million to develop the Falcon
9 launch vehicle. These Space Act Agreements have been shown to have
saved NASA millions of dollars in development costs, making rocket
development \textasciitilde{}4-10 times cheaper than if produced by NASA
alone.

In December 2010, the launch of the COTS Demo Flight 1 mission, SpaceX
became the first private company to successfully launch, orbit and
recover a spacecraft. Dragon was successfully deployed into orbit,
circled the Earth twice, and then made a controlled re-entry burn for a
splashdown in the Pacific Ocean. With Dragon's safe recovery, SpaceX
became the first private company to launch, orbit, and recover a
spacecraft; prior to this mission, only government agencies had been
able to recover orbital spacecraft.

COTS Demo Flight 2 launched in May 2012, in which Dragon successfully
berthed with the ISS, marking the first time that a private spacecraft
had accomplished this feat.

\section{Commercial cargo}\label{commercial-cargo}

\begin{itemize}
\item
  \emph{The first CRS contracts were signed in 2008 and awarded \$1.6
  billion to SpaceX for 12 cargo transport missions, covering deliveries
  to 2016.}
\item
  \emph{After further extensions late in 2015, SpaceX is currently
  scheduled to fly a total of 20 missions.}
\item
  \emph{In 2015, NASA extended the Phase 1 contracts by ordering an
  additional three resupply flights from SpaceX.}
\end{itemize}

Commercial Resupply Services (CRS) are a series of contracts awarded by
NASA from 2008--2016 for delivery of cargo and supplies to the ISS on
commercially operated spacecraft. The first CRS contracts were signed in
2008 and awarded \$1.6 billion to SpaceX for 12 cargo transport
missions, covering deliveries to 2016. SpaceX CRS-1, the first of the 12
planned resupply missions, launched in October 2012, achieved orbit,
berthed and remained on station for 20 days, before re-entering the
atmosphere and splashing down in the Pacific Ocean. CRS missions have
flown approximately twice a year to the ISS since then. In 2015, NASA
extended the Phase 1 contracts by ordering an additional three resupply
flights from SpaceX. After further extensions late in 2015, SpaceX is
currently scheduled to fly a total of 20 missions. A second phase of
contracts (known as CRS2) were solicited and proposed in 2014. They were
awarded in January 2016, for cargo transport flights beginning in 2019
and expected to last through 2024.

\includegraphics[width=5.50000in,height=3.66667in]{media/image15.jpg}\\
\emph{Crew Dragon undergoing testing prior to flight}

\section{Commercial crew}\label{commercial-crew}

\begin{itemize}
\item
  \emph{SpaceX did not win a Space Act Agreement in the first round
  (CCDev 1), but during the second round (CCDev 2), NASA awarded SpaceX
  with a contract worth \$75 million to further develop their launch
  escape system, test a crew accommodations mock-up, and to further
  progress their Falcon/Dragon crew transportation design.}
\item
  \emph{SpaceX won \$2.6 billion to complete and certify Dragon 2 by
  2017.}
\end{itemize}

The Commercial Crew Development (CCDev) program intends to develop
commercially operated spacecraft that are capable of delivering
astronauts to the ISS. SpaceX did not win a Space Act Agreement in the
first round (CCDev 1), but during the second round (CCDev 2), NASA
awarded SpaceX with a contract worth \$75 million to further develop
their launch escape system, test a crew accommodations mock-up, and to
further progress their Falcon/Dragon crew transportation design. The
CCDev program later became Commercial Crew Integrated Capability
(CCiCap), and in August 2012, NASA announced that SpaceX had been
awarded \$440 million to continue development and testing of its Dragon
2 spacecraft.

In September 2014, NASA chose SpaceX and Boeing as the two companies
that will be funded to develop systems to transport U.S. crews to and
from the ISS. SpaceX won \$2.6 billion to complete and certify Dragon 2
by 2017. The contracts include at least one crewed flight test with at
least one NASA astronaut aboard. Once Crew Dragon achieves NASA
certification, the contract requires SpaceX to conduct at least two, and
as many as six, crewed missions to the space station. In early 2017,
SpaceX was awarded four additional crewed missions to the ISS from NASA
to shuttle astronauts back and forth. In early 2019, SpaceX successfully
conducted a test flight of Crew Dragon, which it docked (instead of
Dragon 1's method of berthing using Canadarm 2) and then splashed down
in the Atlantic ocean.

\section{US Defense contracts}\label{us-defense-contracts}

\begin{itemize}
\item
  \emph{DSCOVR was launched on a Falcon 9 launch vehicle in 2015, while
  STP-2 will be launched on a Falcon Heavy in 2019.}
\item
  \emph{In 2016 the US National Reconnaissance Office said it had
  purchased launches from SpaceX - the first (for NROL-76) took place on
  1 May 2017.}
\item
  \emph{In March 2017 SpaceX won (vs ULA) with a bid of \$96.5 million
  for the 3rd GPS III launch (due Feb 2019).}
\end{itemize}

In 2005, SpaceX announced that it had been awarded an Indefinite
Delivery/Indefinite Quantity (IDIQ) contract for Responsive Small
Spacelift (RSS) launch services by the United States Air Force, which
could allow the Air Force to purchase up to \$100 million worth of
launches from the company. In April 2008, NASA announced that it had
awarded an IDIQ Launch Services contract to SpaceX for up to \$1
billion, depending on the number of missions awarded. The contract
covers launch services ordered by June 2010, for launches through
December 2012. Musk stated in the same 2008 announcement that SpaceX has
sold 14 contracts for flights on the various Falcon vehicles. In
December 2012, SpaceX announced its first two launch contracts with the
United States Department of Defense. The United States Air Force Space
and Missile Systems Center awarded SpaceX two EELV-class missions: Deep
Space Climate Observatory (DSCOVR) and Space Test Program 2 (STP-2).
DSCOVR was launched on a Falcon 9 launch vehicle in 2015, while STP-2
will be launched on a Falcon Heavy in 2019.

In May 2015, the United States Air Force announced that the Falcon 9
v1.1 was certified for launching "national security space missions,"
which allows SpaceX to contract launch services to the Air Force for any
payloads classified under national security. This broke the monopoly
held since 2006 by ULA over the US Air Force launches of classified
payloads.

In April 2016, the U.S. Air Force awarded the first such national
security launch, an \$82.7 million contract to SpaceX to launch the 2nd
GPS III satellite in May 2018; this estimated cost was approximately
40\% less than the estimated cost for similar previous missions. Prior
to this, United Launch Alliance was the only provider certified to
launch national security payloads. ULA did not submit a bid for the May
2018 launch.

In 2016 the US National Reconnaissance Office said it had purchased
launches from SpaceX - the first (for NROL-76) took place on 1 May 2017.

In March 2017 SpaceX won (vs ULA) with a bid of \$96.5 million for the
3rd GPS III launch (due Feb 2019).

In March 2018, SpaceX secured an additional \$290 million contract from
the U.S. Air Force to launch three next-generation GPS satellites, known
as GPS III. The first of these launches is expected to take place in
March 2020.

In February 2019, SpaceX secured a \$297 million contract from the US.
Air Force to launch three national security missions, including
AFSPC-44, NROL-87, and NROL-85, all slated to launch NET FY 2021.

\section{International contracts}\label{international-contracts}

\section{Kazakhstan}\label{kazakhstan}

\begin{itemize}
\item
  \emph{According to the Kazakh Defence and Aerospace Ministry, the
  launch from SpaceX would cost the country \$1.3 million.}
\item
  \emph{SpaceX won a contract to launch two Kazakhstan's satellites
  aboard the Falcon 9 launch rocket on a rideshare with other
  satellites.}
\end{itemize}

SpaceX won a contract to launch two Kazakhstan's satellites aboard the
Falcon 9 launch rocket on a rideshare with other satellites. The takeoff
was scheduled for November 19, 2018. According to the Kazakh Defence and
Aerospace Ministry, the launch from SpaceX would cost the country \$1.3
million. The two small satellites are named KazSaySat and KazistiSat.

\section{Launch market competition and pricing
pressure}\label{launch-market-competition-and-pricing-pressure}

\begin{itemize}
\item
  \emph{SpaceX has publicly indicated that if they are successful with
  developing the reusable technology, launch prices in the US\$5 to 7
  million range for the reusable Falcon 9 are possible.}
\item
  \emph{At a side-by-side comparison, SpaceX's launch costs for
  commercial missions are considerably lower at \$62~million.}
\item
  \emph{In 2017, SpaceX had 45\% global market share for awarded
  commercial launch contracts, the estimate for 2018 is about 65\% as of
  July 2018.}
\end{itemize}

SpaceX's low launch prices, especially for communication satellites
flying to geostationary (GTO) orbit, have resulted in market pressure on
its competitors to lower their own prices. Prior to 2013, the openly
competed comsat launch market had been dominated by Arianespace (flying
Ariane 5) and International Launch Services (flying Proton). With a
published price of US\$56.5 million per launch to low Earth orbit,
"Falcon 9 rockets {[}were{]} already the cheapest in the industry.
Reusable Falcon 9s could drop the price by an order of magnitude,
sparking more space-based enterprise, which in turn would drop the cost
of access to space still further through economies of scale." SpaceX has
publicly indicated that if they are successful with developing the
reusable technology, launch prices in the US\$5 to 7 million range for
the reusable Falcon 9 are possible.

In 2014, SpaceX had won nine contracts out of 20 that were openly
competed worldwide in 2014 at commercial launch service providers. Space
media reported that SpaceX had "already begun to take market share" from
Arianespace. Arianespace has requested that European governments provide
additional subsidies to face the competition from SpaceX. European
satellite operators are pushing the ESA to reduce Ariane 5 and the
future Ariane 6 rocket launch prices as a result of competition from
SpaceX. According to one Arianespace managing director in 2015, it was
clear that "a very significant challenge {[}was{]} coming from SpaceX
... Therefore things have to change ... and the whole European industry
is being restructured, consolidated, rationalised and streamlined." Jean
Botti, Director of innovation for Airbus (which makes the Ariane 5)
warned that "those who don't take Elon Musk seriously will have a lot to
worry about." In 2014, no commercial launches were booked to fly on the
Russian Proton rocket.

Also in 2014, SpaceX capabilities and pricing began to affect the market
for launch of US military payloads. For nearly a decade the large US
launch provider United Launch Alliance (ULA) had faced no competition
for military launches. Without this competition, launch costs by the
U.S. provider rose to over \$400 million. The ULA monopoly ended when
SpaceX began to compete for national security launches. At a
side-by-side comparison, SpaceX's launch costs for commercial missions
are considerably lower at \$62~million.

In 2015, anticipating a slump in domestic military and spy launches, ULA
stated that it would go out of business unless it won commercial
satellite launch orders. To that end, ULA announced a major
restructuring of processes and workforce in order to decrease launch
costs by half.

In 2017, SpaceX had 45\% global market share for awarded commercial
launch contracts, the estimate for 2018 is about 65\% as of July 2018.

On January 11, 2019, SpaceX issued a statement announcing it will lay
off 10\% of its workforce, in order to help finance the Starship and
Starlink projects.

\section{See also}\label{see-also}

\section{References}\label{references}

\section{External links}\label{external-links}

\begin{itemize}
\item
  \emph{SpaceX on Twitter \& Elon Musk on Twitter}
\item
  \emph{SpaceX Instagram \& Musk's Instagram}
\item
  \emph{Media related to SpaceX at Wikimedia Commons}
\item
  \emph{SpaceX YouTube}
\end{itemize}

Media related to SpaceX at Wikimedia Commons

Official website

SpaceX on Twitter \& Elon Musk on Twitter

SpaceX Instagram \& Musk's Instagram

SpaceX YouTube
