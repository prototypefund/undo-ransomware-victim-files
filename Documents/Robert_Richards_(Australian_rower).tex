\textbf{From Wikipedia, the free encyclopedia}

https://en.wikipedia.org/wiki/Robert\_Richards\_\%28Australian\_rower\%29\\
Licensed under CC BY-SA 3.0:\\
https://en.wikipedia.org/wiki/Wikipedia:Text\_of\_Creative\_Commons\_Attribution-ShareAlike\_3.0\_Unported\_License

\section{Robert Richards (Australian
rower)}\label{robert-richards-australian-rower}

\begin{itemize}
\item
  \emph{Robert Richards (born 22 September 1971 in Ballarat, Victoria)
  is an Australian former lightweight rower.}
\item
  \emph{In the four years he rowed for Australia at the premier world
  regatta he won a medal each time.}
\end{itemize}

Robert Richards (born 22 September 1971 in Ballarat, Victoria) is an
Australian former lightweight rower. He is a former world champion, an
Olympic silver medallist and a national champion. In the four years he
rowed for Australia at the premier world regatta he won a medal each
time.

\section{Club and state rowing}\label{club-and-state-rowing}

\begin{itemize}
\item
  \emph{Richards' senior rowing was from the Wendouree-Ballarat Rowing
  Club.}
\item
  \emph{Richards rowed in Victorian representative men's lightweight
  fours contesting the Penrith Cup at the Interstate Regatta within the
  Australian Rowing Championships from 1997 to 2000.}
\end{itemize}

Richards' senior rowing was from the Wendouree-Ballarat Rowing Club.

Richards rowed in Victorian representative men's lightweight fours
contesting the Penrith Cup at the Interstate Regatta within the
Australian Rowing Championships from 1997 to 2000.\\
He stroked those Victorian crews in all four of those years.

\section{International representative
rowing}\label{international-representative-rowing}

\begin{itemize}
\item
  \emph{Richards first represented Australia in the lightweight eight at
  the Aiguebelette 1997 where the Australians won a thrilling final by
  0.03 seconds with only 1.5 lengths separating the field.There Richards
  won his first and only World Championship title.}
\item
  \emph{For Cologne 1998 and then at St Catharine's 1999 Richards
  stroked the Australian lightweight coxless four with his Ballarat
  team-mate Anthony Edwards, Darren Balmforth and the Tasmanian champion
  Simon Burgess.}
\end{itemize}

Richards first represented Australia in the lightweight eight at the
Aiguebelette 1997 where the Australians won a thrilling final by 0.03
seconds with only 1.5 lengths separating the field.There Richards won
his first and only World Championship title. For Cologne 1998 and then
at St Catharine's 1999 Richards stroked the Australian lightweight
coxless four with his Ballarat team-mate Anthony Edwards, Darren
Balmforth and the Tasmanian champion Simon Burgess. That four took
bronze in 1998 and silver in 1999.

That same lightweight coxless four stayed together for the Sydney 2000
Olympics. The event showcased two match races between the Australians
and the French crew. They met in a semi-final where the Australians,
with Richards setting the pace, pipped the French by 3/100ths of a
second. In the final the Australians led for much of the race. The
French tried once to break through and failed, then a second time and
failed and finally with a matter of metres to go broke through to win by
less than half a second. Both races were a superb highlight of the
regatta and won Richards Olympic silver in his last Australian
representative appearance.

\section{References}\label{references}

\section{External links}\label{external-links}

\begin{itemize}
\item
  \emph{Robert Richards at FISA WorldRowing.com}
\item
  \emph{"Robert Richards".}
\end{itemize}

Robert Richards at FISA WorldRowing.com

Evans, Hilary; Gjerde, Arild; Heijmans, Jeroen; Mallon, Bill. "Robert
Richards". Olympics at Sports-Reference.com. Sports Reference
LLC..mw-parser-output
cite.citation\{font-style:inherit\}.mw-parser-output .citation
q\{quotes:"\textbackslash{}"""\textbackslash{}"""'""'"\}.mw-parser-output
.citation .cs1-lock-free
a\{background:url("//upload.wikimedia.org/wikipedia/commons/thumb/6/65/Lock-green.svg/9px-Lock-green.svg.png")no-repeat;background-position:right
.1em center\}.mw-parser-output .citation .cs1-lock-limited
a,.mw-parser-output .citation .cs1-lock-registration
a\{background:url("//upload.wikimedia.org/wikipedia/commons/thumb/d/d6/Lock-gray-alt-2.svg/9px-Lock-gray-alt-2.svg.png")no-repeat;background-position:right
.1em center\}.mw-parser-output .citation .cs1-lock-subscription
a\{background:url("//upload.wikimedia.org/wikipedia/commons/thumb/a/aa/Lock-red-alt-2.svg/9px-Lock-red-alt-2.svg.png")no-repeat;background-position:right
.1em center\}.mw-parser-output .cs1-subscription,.mw-parser-output
.cs1-registration\{color:\#555\}.mw-parser-output .cs1-subscription
span,.mw-parser-output .cs1-registration span\{border-bottom:1px
dotted;cursor:help\}.mw-parser-output .cs1-ws-icon
a\{background:url("//upload.wikimedia.org/wikipedia/commons/thumb/4/4c/Wikisource-logo.svg/12px-Wikisource-logo.svg.png")no-repeat;background-position:right
.1em center\}.mw-parser-output
code.cs1-code\{color:inherit;background:inherit;border:inherit;padding:inherit\}.mw-parser-output
.cs1-hidden-error\{display:none;font-size:100\%\}.mw-parser-output
.cs1-visible-error\{font-size:100\%\}.mw-parser-output
.cs1-maint\{display:none;color:\#33aa33;margin-left:0.3em\}.mw-parser-output
.cs1-subscription,.mw-parser-output .cs1-registration,.mw-parser-output
.cs1-format\{font-size:95\%\}.mw-parser-output
.cs1-kern-left,.mw-parser-output
.cs1-kern-wl-left\{padding-left:0.2em\}.mw-parser-output
.cs1-kern-right,.mw-parser-output
.cs1-kern-wl-right\{padding-right:0.2em\}
