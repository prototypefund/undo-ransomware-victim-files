\textbf{From Wikipedia, the free encyclopedia}

https://en.wikipedia.org/wiki/Adam\%20Liptak\\
Licensed under CC BY-SA 3.0:\\
https://en.wikipedia.org/wiki/Wikipedia:Text\_of\_Creative\_Commons\_Attribution-ShareAlike\_3.0\_Unported\_License

\section{Adam Liptak}\label{adam-liptak}

\begin{itemize}
\item
  \emph{Adam Liptak (born September 2, 1960) is an American journalist,
  lawyer and instructor in law and journalism.}
\item
  \emph{He is the Supreme Court correspondent for The New York Times.}
\item
  \emph{Liptak has written for The New Yorker, Vanity Fair, Rolling
  Stone, the New York Observer, Business Week and other publications.}
\end{itemize}

Adam Liptak (born September 2, 1960) is an American journalist, lawyer
and instructor in law and journalism. He is the Supreme Court
correspondent for The New York Times.

Liptak has written for The New Yorker, Vanity Fair, Rolling Stone, the
New York Observer, Business Week and other publications. He was a
finalist for the Pulitzer Prize for Explanatory Journalism in 2009 for a
series of articles that examined ways in which the American legal system
differs from those of other developed nations.

\section{Early life and education}\label{early-life-and-education}

\begin{itemize}
\item
  \emph{During law school, Liptak worked as a summer clerk in The New
  York Times Company's legal department.}
\item
  \emph{Liptak spent a decade advising The New York Times and the
  company's other newspapers, television stations and new media
  properties on defamation, privacy, news gathering and related issues
  and frequently litigated media and commercial cases.}
\item
  \emph{In 1992, he returned to The New York Times Company's legal
  department.}
\end{itemize}

Liptak was born in Stamford, Connecticut. He first joined The New York
Times as a copyboy in 1984, after graduating cum laude from Yale
University, where he was an editor of the Yale Daily News, with a degree
in English. In addition to clerical work and fetching coffee, he
assisted the reporter M. A. Farber in covering the trial of a libel suit
brought by General William Westmoreland against CBS.

He returned to Yale for a J.D. degree, graduating from Yale Law School
in 1988. During law school, Liptak worked as a summer clerk in The New
York Times Company's legal department. After graduating, he spent four
years at Cahill Gordon \& Reindel, a New York City law firm, as a
litigation associate specializing in First Amendment matters.

In 1992, he returned to The New York Times Company's legal department.
Liptak spent a decade advising The New York Times and the company's
other newspapers, television stations and new media properties on
defamation, privacy, news gathering and related issues and frequently
litigated media and commercial cases.

\section{Career}\label{career}

\begin{itemize}
\item
  \emph{He has taught courses on media law and the Supreme Court at
  Columbia University Graduate School of Journalism, UCLA School of Law,
  University of Chicago Law School, University of Southern California's
  Gould School of Law, and Yale Law School.}
\item
  \emph{Liptak joined The New York Times news staff in 2002 as its
  national legal correspondent.}
\end{itemize}

Liptak joined The New York Times news staff in 2002 as its national
legal correspondent. He covered the Supreme Court nominations of John
Roberts and Samuel Alito; the investigation into the disclosure of the
identity of Valerie Plame, an undercover Central Intelligence Agency
operative; the trial of John Lee Malvo, one of the Washington-area
snipers; judicial ethics; and various aspects of the criminal justice
system, including capital punishment. He inaugurated the Sidebar column
in January 2007. The column covers and considers developments in the
law.

In 2005, he examined the rise in life sentences in the U.S. in a
three-part series. The next year, Liptak and two colleagues studied
connections between contributions to the campaigns of justices on the
Ohio Supreme Court and those justices' voting records. He was a member
of the teams that examined the reporting of Jayson Blair and Judith
Miller at The New York Times, in 2003 and 2005, respectively.

He began covering the Supreme Court in 2008. He followed Linda
Greenhouse, who had covered the Supreme Court for nearly 30 years.

Liptak has served as the chairman of the New York City Bar Association's
communications and media law committee and was a member of the board of
the Media Law Resource Center.

He has taught courses on media law and the Supreme Court at Columbia
University Graduate School of Journalism, UCLA School of Law, University
of Chicago Law School, University of Southern California's Gould School
of Law, and Yale Law School.

Liptak's work has appeared in The New Yorker, Vanity Fair, Rolling
Stone, the New York Observer, Business Week, and The American Lawyer. He
has written several law review articles on First Amendment topics.
Liptak was also featured in The Harvard Crimson's 2014 commencement
issue with his column entitled "Please Calculate Badly." In 2013, he
published an e-book, To Have and Uphold: The Supreme Court and the
Struggle for Same-Sex Marriage.

\section{Awards}\label{awards}

\begin{itemize}
\item
  \emph{Stetson University awarded Liptak an honorary doctor of laws
  degree in 2014, and Hofstra University presented him with its
  Presidential Medal in 2008.}
\item
  \emph{In 1999, he received the New York Press Club's John Peter Zenger
  award for "defending and advancing the cause of a free press".}
\end{itemize}

In 1995, Presstime magazine named him one of 20 leading newspaper
professionals under the age of 40. In 1999, he received the New York
Press Club's John Peter Zenger award for "defending and advancing the
cause of a free press". In 2006, the same group awarded him its Crystal
Gavel award for his journalistic work.

He was a finalist for the Pulitzer Prize for Explanatory Reporting in
2009, and he won the 2010 Scripps Howard Award for Washington Reporting
for a five-part series on the Roberts Court.

Stetson University awarded Liptak an honorary doctor of laws degree in
2014, and Hofstra University presented him with its Presidential Medal
in 2008.

\section{Personal life}\label{personal-life}

\begin{itemize}
\item
  \emph{Liptak lives in Washington, D.C. with his wife, Jennifer Bitman,
  a veterinarian, and their children, Katie and Ivan.}
\end{itemize}

Liptak lives in Washington, D.C. with his wife, Jennifer Bitman, a
veterinarian, and their children, Katie and Ivan.

\section{References}\label{references}

\section{External links}\label{external-links}

\begin{itemize}
\item
  \emph{New York Times biography - Adam Liptak}
\item
  \emph{Adam Liptak's New York Observer review archive}
\end{itemize}

Columbia University Graduate School of Journalism - faculty biographies

New York Times biography - Adam Liptak

Adam Liptak's New York Observer review archive

Appearances on C-SPAN
