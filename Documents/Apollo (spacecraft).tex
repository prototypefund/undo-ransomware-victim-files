\textbf{From Wikipedia, the free encyclopedia}

https://en.wikipedia.org/wiki/Apollo\%20\%28spacecraft\%29\\
Licensed under CC BY-SA 3.0:\\
https://en.wikipedia.org/wiki/Wikipedia:Text\_of\_Creative\_Commons\_Attribution-ShareAlike\_3.0\_Unported\_License

\includegraphics[width=5.50000in,height=3.50625in]{media/image1.jpg}\\
\emph{Complete Apollo spacecraft stack: launch escape system, command
module, service module, Lunar Module, and spacecraft--LM adapter}

\includegraphics[width=5.44133in,height=5.50000in]{media/image2.jpg}\\
\emph{The Apollo 17 CSM seen in lunar orbit from the ascent stage of the
lunar module}

\section{Apollo (spacecraft)}\label{apollo-spacecraft}

\begin{itemize}
\item
  \emph{The expendable (single-use) spacecraft consisted of a combined
  command and service module (CSM) and an Apollo Lunar Module (LM).}
\item
  \emph{Two additional components complemented the spacecraft stack for
  space vehicle assembly: a spacecraft--LM adapter (SLA) designed to
  shield the LM from the aerodynamic stress of launch and to connect the
  CSM to the Saturn launch vehicle; and a launch escape system (LES) to
  carry the crew in the command module safely away from the launch
  vehicle in the event of a launch emergency.}
\end{itemize}

The Apollo spacecraft was composed of three parts designed to accomplish
the American Apollo program's goal of landing astronauts on the Moon by
the end of the 1960s and returning them safely to Earth. The expendable
(single-use) spacecraft consisted of a combined command and service
module (CSM) and an Apollo Lunar Module (LM). Two additional components
complemented the spacecraft stack for space vehicle assembly: a
spacecraft--LM adapter (SLA) designed to shield the LM from the
aerodynamic stress of launch and to connect the CSM to the Saturn launch
vehicle; and a launch escape system (LES) to carry the crew in the
command module safely away from the launch vehicle in the event of a
launch emergency.

The design was based on the lunar orbit rendezvous approach: two docked
spacecraft were sent to the Moon and went into lunar orbit. While the LM
separated and landed, the CSM remained in orbit. After the lunar
excursion, the two craft rendezvoused and docked in lunar orbit, and the
CSM returned the crew to Earth. The command module was the only part of
the space vehicle that returned with the crew to the Earth's surface.

The LES was jettisoned during launch upon reaching the point where it
was no longer needed, and the SLA remained attached to the launch
vehicle's upper stage. Two unmanned CSM's, one unmanned LM and one
manned CSM were carried into space by Saturn IB launch vehicles for low
Earth orbit Apollo missions. Larger Saturn Vs launched two unmanned
CSM's on high Earth orbit test flights, the CSM on one manned lunar
mission, the complete spacecraft on one manned low Earth orbit mission
and eight manned lunar missions. After conclusion of the Apollo program,
four CSM's were launched on Saturn IBs for three Skylab Earth orbital
missions and the Apollo-Soyuz Test Project.

\section{Command and service module}\label{command-and-service-module}

\begin{itemize}
\item
  \emph{This consisted of a command module supported by a service
  module, built by North American Aviation (later North American
  Rockwell).}
\item
  \emph{The major part of the Apollo spacecraft was a three-man vehicle
  designed for Earth orbital, translunar, and lunar orbital flight, and
  return to Earth.}
\end{itemize}

The major part of the Apollo spacecraft was a three-man vehicle designed
for Earth orbital, translunar, and lunar orbital flight, and return to
Earth. This consisted of a command module supported by a service module,
built by North American Aviation (later North American Rockwell).

\includegraphics[width=5.50000in,height=4.12500in]{media/image3.jpg}\\
\emph{Apollo Command Module and its position on top of Saturn V}

\section{Command module (CM)}\label{command-module-cm}

\begin{itemize}
\item
  \emph{The command module was the control center for the Apollo
  spacecraft and living quarters for the three crewmen.}
\item
  \emph{It was the only part of the Apollo/Saturn space vehicle that
  returned to Earth intact.}
\end{itemize}

The command module was the control center for the Apollo spacecraft and
living quarters for the three crewmen. It contained the pressurized main
crew cabin, crew couches, control and instrument panel, Primary
Guidance, Navigation and Control System, communications systems,
environmental control system, batteries, heat shield, reaction control
system to provide attitude control, forward docking hatch, side hatch,
five windows, and a parachute recovery system. It was the only part of
the Apollo/Saturn space vehicle that returned to Earth intact.

\includegraphics[width=5.50000in,height=3.88418in]{media/image4.jpg}\\
\emph{Apollo Service Module}

\section{Service module (SM)}\label{service-module-sm}

\begin{itemize}
\item
  \emph{The service module remained attached to the command module
  throughout the mission.}
\item
  \emph{The service module was unpressurized and contained a main
  service propulsion engine and hypergolic propellant to enter and leave
  lunar orbit, a reaction control system to provide attitude control and
  translational capability, fuel cells with hydrogen and oxygen
  reactants, radiators to dump waste heat into space, and a high gain
  antenna.}
\item
  \emph{Capable of multiple restarts, this engine placed the Apollo
  spacecraft into and out of lunar orbit, and was used for mid-course
  corrections between the Earth and the Moon.}
\end{itemize}

The service module was unpressurized and contained a main service
propulsion engine and hypergolic propellant to enter and leave lunar
orbit, a reaction control system to provide attitude control and
translational capability, fuel cells with hydrogen and oxygen reactants,
radiators to dump waste heat into space, and a high gain antenna. The
oxygen was also used for breathing, and the fuel cells produced water
for drinking and environmental control. On Apollo 15, 16 and 17 it also
carried a scientific instrument package, with a mapping camera and a
small sub-satellite to study the Moon.

A major portion of the service module was taken up by propellant and the
main rocket engine. Capable of multiple restarts, this engine placed the
Apollo spacecraft into and out of lunar orbit, and was used for
mid-course corrections between the Earth and the Moon.

The service module remained attached to the command module throughout
the mission. It was jettisoned just prior to reentry into the Earth's
atmosphere.

\includegraphics[width=5.50000in,height=3.50625in]{media/image5.jpg}\\
\emph{Apollo Lunar Module}

\section{Lunar Module (LM)}\label{lunar-module-lm}

\begin{itemize}
\item
  \emph{The Apollo Lunar Module was a separate vehicle designed to land
  on the Moon and return to lunar orbit, and was the first true
  "spaceship" since it flew solely in the vacuum of space.}
\item
  \emph{It also had several cargo compartments used to carry, among
  other things: the Apollo Lunar Surface Experiment Packages ALSEP, the
  modularized equipment transporter (MET) (a hand-pulled equipment cart
  used on Apollo 14), the Lunar Rover (Apollo 15, 16 and 17), a surface
  television camera, surface tools, and lunar sample collection boxes.}
\end{itemize}

The Apollo Lunar Module was a separate vehicle designed to land on the
Moon and return to lunar orbit, and was the first true "spaceship" since
it flew solely in the vacuum of space. It consisted of a descent stage
and an ascent stage. It supplied life support systems for two astronauts
for up to four to five days on the Apollo 15, 16 and 17 missions. The
spacecraft was designed and manufactured by the Grumman Aircraft
Company.

The descent stage contained the landing gear, landing radar antenna,
descent propulsion system, and fuel to land on the Moon. It also had
several cargo compartments used to carry, among other things: the Apollo
Lunar Surface Experiment Packages ALSEP, the modularized equipment
transporter (MET) (a hand-pulled equipment cart used on Apollo 14), the
Lunar Rover (Apollo 15, 16 and 17), a surface television camera, surface
tools, and lunar sample collection boxes.

The ascent stage contained the crew cabin, instrument panels, overhead
hatch/docking port, forward hatch, optical and electronic guidance
systems, reaction control system, radar and communications antennas,
ascent rocket engine and propellant to return to lunar orbit and
rendezvous with the Apollo Command and Service Modules.

\includegraphics[width=5.50000in,height=4.58333in]{media/image6.png}\\
\emph{Apollo spacecraft-to-LM adapter}

\includegraphics[width=5.50000in,height=5.50000in]{media/image7.jpg}\\
\emph{One of the SLA panels on Apollo 7 did not fully open to the
designed 45 degrees.}

\section{Spacecraft--lunar module adapter
(SLA)}\label{spacecraftlunar-module-adapter-sla}

\begin{itemize}
\item
  \emph{On all flights through Apollo 7, the SLA panels remained hinged
  to the S-IVB and opened to a 45-degree angle, as originally designed.}
\item
  \emph{It protected the LM, the service propulsion system engine
  nozzle, and the launch vehicle to service module umbilical during
  launch and ascent through the atmosphere.}
\item
  \emph{Once in space, the astronauts pressed the 'CSM/LV Sep' button on
  the control panel to separate the CSM from the launch vehicle.}
\end{itemize}

The spacecraft--LM adapter (SLA), built by North American Aviation
(Rockwell), was a conical aluminum structure which supported the service
module above the Saturn S-IVB rocket stage. It protected the LM, the
service propulsion system engine nozzle, and the launch vehicle to
service module umbilical during launch and ascent through the
atmosphere.

The SLA was composed of four fixed 7-foot (2.1~m) tall panels bolted to
the Instrument Unit on top of the S-IVB stage, which were connected via
hinges to four 21-foot-tall (6.4~m) panels which would open from the top
similar to flower petals.

The SLA was made from 1.7-inch (43~mm) thick aluminum honeycomb
material. The exterior of the SLA was covered by a thin (0.03--0.2
inches or 0.76--5.08 millimetres) layer of cork and painted white to
minimize thermal stresses during launch and ascent.

The service module was bolted to a flange at the top of the longer
panels, and power to the SLA multiply-redundant pyrotechnics was
provided by an umbilical. Because a failure to separate from the S-IVB
stage could leave the crew stranded in orbit, the separation system used
multiple signal paths, multiple detonators and multiple explosive
charges where the detonation of one charge would set off another even if
the detonator on that charge failed to function.

Once in space, the astronauts pressed the 'CSM/LV Sep' button on the
control panel to separate the CSM from the launch vehicle. Detonating
cord was ignited around the flange between the SM and SLA, and along the
joints between the four SLA panels, releasing the SM and blowing apart
the connections between the panels. Dual-redundant pyrotechnic thrusters
at the lower end of the SLA panels then fired to rotate them around the
hinges at 30-60 degrees per second.

On all flights through Apollo 7, the SLA panels remained hinged to the
S-IVB and opened to a 45-degree angle, as originally designed. But as
the Apollo 7 crew practiced rendezvous with the S-IVB/SLA containing a
dummy docking target, one panel did not open to the full 45 degrees,
raising concern about the possibility of collision between the
spacecraft and the SLA panels during docking and extraction of the LM in
a lunar mission. This led to a redesign using a spring-loaded hinge
release system which released the panels at the 45 degree angle and
pushed them away from the S-IVB at a velocity of about five miles per
hour, putting them a safe distance away by the time the astronauts
pulled the CSM away, rotated it through 180 degrees, and came back for
docking.

The LM was connected to the SLA at four points around the lower panels.
After the astronauts docked the CSM to the LM, they blew charges to
separate those connections and a guillotine severed the
LM-to-instrument-unit umbilical. After the charges fired, springs pushed
the LM away from the S-IVB and the astronauts were free to continue
their trip to the Moon.

\section{Specifications}\label{specifications}

\begin{itemize}
\item
  \emph{Apex diameter: 12~ft 10 in (3.9 m) Service module end}
\end{itemize}

Height: 28~ft (8.5~m)

Apex diameter: 12~ft 10 in (3.9 m) Service module end

Base diameter: 21~ft 8 in (6.6 m) S-IVB end

Weight: 4,050~lb (1,840~kg)

Volume: 6,700~cu~ft (190~m3), 4,900~cu~ft (140~m3) usable

\includegraphics[width=4.64933in,height=5.50000in]{media/image8.jpg}\\
\emph{Pad abort test (2), showing pitch motor and launch escape motor in
operation}

\section{Launch escape system (LES)}\label{launch-escape-system-les}

\begin{itemize}
\item
  \emph{The LES was carried but never used on four unmanned Apollo
  flights, and fifteen manned Apollo, Skylab, and Apollo-Soyuz Test
  Project flights.}
\item
  \emph{The Apollo launch escape system (LES) was built by the Lockheed
  Propulsion Company.}
\item
  \emph{The LES included three wires that ran down the exterior of the
  launch vehicle.}
\end{itemize}

The Apollo launch escape system (LES) was built by the Lockheed
Propulsion Company. Its purpose was to abort the mission by pulling the
CM (the crew cabin) away from the launch vehicle in an emergency, such
as a pad fire before launch, guidance failure, or launch vehicle failure
likely to lead to an imminent explosion.

The LES included three wires that ran down the exterior of the launch
vehicle. If the signals from any two of the wires were lost, the LES
would activate automatically. Alternatively, the Commander could
activate the system manually using one of two translation controller
handles, which were switched to a special abort mode for launch. When
activated, the LES would fire a solid fuel escape rocket, and open a
canard system to direct the CM away from, and off the path of, a launch
vehicle in trouble. The LES would then jettison and the CM would land
with its parachute recovery system.

If the emergency happened on the launch pad, the LES would lift the CM
to a sufficient height to allow the recovery parachutes to deploy safely
before coming in contact with the ground.

In the absence of an emergency, the LES was routinely jettisoned about
20 or 30 seconds after the launch vehicle's second-stage ignition, using
a separate solid fuel rocket motor manufactured by the Thiokol Chemical
Company. Abort modes after this point would be accomplished without the
LES. The LES was carried but never used on four unmanned Apollo flights,
and fifteen manned Apollo, Skylab, and Apollo-Soyuz Test Project
flights.

\includegraphics[width=5.50000in,height=4.19714in]{media/image9.jpg}\\
\emph{Apollo launch escape system components}

\section{Major components}\label{major-components}

\begin{itemize}
\item
  \emph{This jettisoned the entire Launch Escape System after it was no
  longer needed, sometime after second stage ignition.}
\item
  \emph{Launch escape tower}
\item
  \emph{The apparent overengineering of this safety system was due to
  the fact that the launch escape system, which depended on the Q-ball
  data, was armed 5 minutes before launch, so retraction of the Q-ball
  cover was a life-critical part of a possible pad abort.}
\item
  \emph{Launch escape motor}
\end{itemize}

Nose cone and Q-ball

The nose cone of the LES contained an array of 8 pressure-measuring
pitot tubes in a structure known as the "Q-ball". These sensors were
connected to the CM and Saturn launch vehicle guidance computers,
allowing calculation of dynamic pressure (q) during atmospheric flight,
and also the angle of attack in the event of an abort.

Q-ball cover

A styrofoam cover, removed a few seconds before launch, protected the
pitot tubes from being clogged by debris. The cover was split in half
vertically and held together by a 2-inch (51~mm) rubber band. A razor
blade was positioned behind the rubber band, pinched between the halves
of the cover. A wire cable was connected to the top and bottom of the
razor blade and to both halves of the cover. The cable was routed
through a pulley on the hammerhead crane at the top of the launch
umbilical tower (LUT) down to a tube on the right side of the 360-foot
(110~m) level of the LUT. The cable was connected to a cylindrical
weight inside a tube. The weight rested on a lever controlled by a
pneumatic solenoid valve. When the valve was actuated from the Launch
Control Center (LCC), the pneumatic pressure of 600 PSI GN2 (nitrogen
gas) rotated the lever down allowing the weight to drop down the tube.
The dropping weight pulled the cable, which pulled the blade cutting the
rubber band, and the cable pulled the halves of the cover away from the
launch vehicle. The apparent overengineering of this safety system was
due to the fact that the launch escape system, which depended on the
Q-ball data, was armed 5 minutes before launch, so retraction of the
Q-ball cover was a life-critical part of a possible pad abort.

Canard assembly and pitch motor

These worked in combination to direct the CM off a straight path and to
the side during an emergency. This would direct the CM off the flight
path of an exploding launch vehicle. It would also direct the CM to land
off to the side of any launch pad fire and not in the middle of it.

Launch escape motor

The main solid fuel rocket motor inside a long tube, with four exhaust
nozzles mounted under a conical fairing. This would pull the CM rapidly
away from a launch emergency.

Tower jettison motor

A smaller solid fuel motor with two exhaust nozzles, mounted in the
tube, above the escape motor. This jettisoned the entire Launch Escape
System after it was no longer needed, sometime after second stage
ignition.

Launch escape tower

A truss framework of tubes that attached the escape motor fairing to the
CM.

Boost protective cover

A hollow conical fiberglass structure protecting the CM's parachute
compartment and providing a smooth aerodynamic cover over the docking
tunnel and probe. After erosion of the pilot's windows from the escape
motor exhaust was discovered during early LES flight testing, an aft
protective cover surrounding the CM's entire upper surface was added.

\section{Specifications}\label{specifications-1}

\begin{itemize}
\item
  \emph{Diameter: 2~ft 2 in (0.66 m)}
\item
  \emph{Thrust, maximum: 200,000 pounds-force (890~kN)}
\item
  \emph{Length with BPC: 39~ft 5 in (12.02 m)}
\item
  \emph{Total mass: 9,200 pounds (4,200~kg)}
\item
  \emph{Thrust, 36,000~ft: 147,000 pounds-force (650~kN)}
\item
  \emph{Length minus BPC: 32~ft 6 in (9.92 m)}
\item
  \emph{Burn time: 4.0 seconds}
\end{itemize}

Length minus BPC: 32~ft 6 in (9.92 m)

Length with BPC: 39~ft 5 in (12.02 m)

Diameter: 2~ft 2 in (0.66 m)

Total mass: 9,200 pounds (4,200~kg)

Thrust, 36,000~ft: 147,000 pounds-force (650~kN)

Thrust, maximum: 200,000 pounds-force (890~kN)

Burn time: 4.0 seconds

\section{Abort tests}\label{abort-tests}

\begin{itemize}
\item
  \emph{Pad Abort Test 1 -- LES abort test from launch pad with Apollo
  boilerplate BP-6}
\item
  \emph{Pad Abort Test 2 -- LES pad abort test of near Block-I CM with
  Apollo boilerplate B-23A}
\end{itemize}

Pad Abort Test 1 -- LES abort test from launch pad with Apollo
boilerplate BP-6

Pad Abort Test 2 -- LES pad abort test of near Block-I CM with Apollo
boilerplate B-23A

Little Joe II -- Four in-flight LES abort tests.

\section{Current locations of
spacecraft}\label{current-locations-of-spacecraft}

\begin{itemize}
\item
  \emph{The disposition of all command modules, and all unflown service
  modules is listed at Apollo command and service module\#CSMs
  produced.}
\item
  \emph{The disposition of all lunar modules is listed at Apollo Lunar
  Module\#Lunar modules produced.}
\end{itemize}

The disposition of all command modules, and all unflown service modules
is listed at Apollo command and service module\#CSMs produced. (All
flown service modules burned up in the Earth's atmosphere at termination
of the missions.)

The disposition of all lunar modules is listed at Apollo Lunar
Module\#Lunar modules produced.

\section{References}\label{references}

\begin{itemize}
\item
  \emph{Apollo Operations Handbook Lunar Module Subsystems Data}
\end{itemize}

North American Rockwell, 'Apollo Command Module News Reference', 1968.

NASA TN D-7083: Launch Escape Propulsion Subsystem

Apollo Operations Handbook Lunar Module Subsystems Data

\section{External links}\label{external-links}

\begin{itemize}
\item
  \emph{Apollo D-2 Proposal by General Electric, Encyclopedia
  Astronautica}
\item
  \emph{NASA report JSC-03600 Apollo/Skylab ASTP and Shuttle Orbiter
  Major End Items, Final Report, March 1978; NASA report listing
  dispositions of all rockets and spacecraft used in the Apollo, Skylab,
  Apollo-Soyez Test Project and early shuttle missions, as of 1978.}
\end{itemize}

NASA report JSC-03600 Apollo/Skylab ASTP and Shuttle Orbiter Major End
Items, Final Report, March 1978; NASA report listing dispositions of all
rockets and spacecraft used in the Apollo, Skylab, Apollo-Soyez Test
Project and early shuttle missions, as of 1978.

Apollo D-2 Proposal by General Electric, Encyclopedia Astronautica
