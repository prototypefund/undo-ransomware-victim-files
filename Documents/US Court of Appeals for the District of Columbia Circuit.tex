\textbf{From Wikipedia, the free encyclopedia}

https://en.wikipedia.org/wiki/United\%20States\%20Court\%20of\%20Appeals\%20for\%20the\%20District\%20of\%20Columbia\%20Circuit\\
Licensed under CC BY-SA 3.0:\\
https://en.wikipedia.org/wiki/Wikipedia:Text\_of\_Creative\_Commons\_Attribution-ShareAlike\_3.0\_Unported\_License

\section{United States Court of Appeals for the District of Columbia
Circuit}\label{united-states-court-of-appeals-for-the-district-of-columbia-circuit}

\begin{itemize}
\item
  \emph{Circuit from having to take appeals from the local D.C. trial
  court.}
\item
  \emph{The United States Court of Appeals for the District of Columbia
  Circuit (in case citations, D.C.}
\item
  \emph{Circuit, is the federal appellate court for the U.S. District
  Court for the District of Columbia.}
\item
  \emph{Circuit before their elevations to the Supreme Court.}
\item
  \emph{Circuit.}
\end{itemize}

The United States Court of Appeals for the District of Columbia Circuit
(in case citations, D.C. Cir.) known informally as the D.C. Circuit, is
the federal appellate court for the U.S. District Court for the District
of Columbia. Appeals from the D.C. Circuit, as with all U.S. Courts of
Appeals, are heard on a discretionary basis by the Supreme Court. It
should not be confused with the United States Court of Appeals for the
Federal Circuit, which is limited in jurisdiction by subject matter
rather than geography, or with the District of Columbia Court of
Appeals, which is roughly equivalent to a state supreme court in the
District of Columbia, and was established in 1970 to relieve the D.C.
Circuit from having to take appeals from the local D.C. trial court.

While it has the smallest geographic jurisdiction of any of the United
States courts of appeals, the D.C. Circuit, with eleven active
judgeships, is arguably the most important inferior appellate court. The
court is given the responsibility of directly reviewing the decisions
and rulemaking of many federal independent agencies of the United States
government based in the national capital, often without prior hearing by
a district court. Aside from the agencies whose statutes explicitly
direct review by the D.C. Circuit, the court typically hears cases from
other agencies under the more general jurisdiction granted to the Courts
of Appeals under the Administrative Procedure Act. Given the broad areas
over which federal agencies have power, this often gives the judges of
the D.C. Circuit a central role in affecting national U.S. policy and
law. Because of this, the D.C. Circuit is often referred to as the
second-most powerful court in the United States, second only to the
Supreme Court.

A judgeship on the D.C. Circuit is often thought of as a stepping-stone
for appointment to the Supreme Court. As of October~2018{[}update{]},
four of the nine justices on the Supreme Court are alumni of the D.C.
Circuit: Chief Justice John Roberts and Associate Justices Clarence
Thomas, Ruth Bader Ginsburg, and Brett Kavanaugh. Associate Justice
Elena Kagan was nominated by President Bill Clinton to the same seat
that Roberts would later fill, but was never given a vote in the Senate.
In addition, Chief Justices Fred M. Vinson and Warren Burger, as well as
Associate Justices Wiley Blount Rutledge and Antonin Scalia, served on
the D.C. Circuit before their elevations to the Supreme Court. In 1987,
President Ronald Reagan put forth two failed nominees from the D.C.
Circuit: former Judge Robert Bork, who was rejected by the Senate, and
former (2001--2008) Chief Judge Douglas H. Ginsburg (no relation to Ruth
Bader Ginsburg), who withdrew his nomination after it became known that
he had used marijuana as a college student and professor in the 1960s
and 1970s. Likewise, in 2016 President Barack Obama nominated Merrick
Garland from the D.C. Circuit to replace the late Scalia, but the Senate
controversially did not give Garland a full vote.

Because the D.C. Circuit does not represent any state, confirmation of
nominees can be procedurally and practically easier than for nominees to
the Courts of Appeals for the other geographical districts, as
home-state senators have historically been able to hold up confirmation
through the "blue slip" process. However, in recent years, several
nominees to the D.C. Circuit were stalled and some were ultimately not
confirmed because senators claimed that the court had become larger than
necessary to handle its caseload. The court has a history of reversing
the Federal Communications Commission's major policy actions.

The United States Court of Appeals for the District of Columbia Circuit
meets at the E. Barrett Prettyman United States Courthouse, near
Judiciary Square in downtown Washington, D.C.

From 1984 to 2009, there were twelve seats on the D.C. Circuit. One of
those seats was eliminated by the Court Security Improvement Act of 2007
on January 7, 2008, with immediate effect, leaving the number of
authorized judgeships at eleven. (The eliminated judgeship was assigned
to the Ninth Circuit effective January 21, 2009).

Decisions of the U.S. Courts of Appeals are published in the Federal
Reporter, an unofficial reporter from Thomson Reuters.

\section{Current composition of the
court}\label{current-composition-of-the-court}

\begin{itemize}
\item
  \emph{As of March~18, 2019{[}update{]}:}
\end{itemize}

As of March~18, 2019{[}update{]}:

\section{List of former judges}\label{list-of-former-judges}

\section{Chiefs}\label{chiefs}

\begin{itemize}
\item
  \emph{These acts made the Chief Justice a Chief Judge.}
\item
  \emph{Chief judges have administrative responsibilities with respect
  to their circuits, and preside over any panel on which they serve
  unless the circuit justice (i.e., the Supreme Court justice
  responsible for the circuit) is also on the panel.}
\item
  \emph{Unlike the Supreme Court, where one justice is specifically
  nominated to be chief, the office of chief judge rotates among the
  circuit judges.}
\end{itemize}

When Congress established this court in 1893 as the Court of Appeals of
the District of Columbia, it had a Chief Justice, and the other judges
were called Associate Justices, which was similar to the structure of
the Supreme Court. The Chief Justiceship was a separate seat: the
President would appoint the Chief Justice, and that person would stay
Chief Justice until he left the court.

On June 25, 1948, 62 Stat. 869 and 62 Stat. 985 became law. These acts
made the Chief Justice a Chief Judge. In 1954, another law, 68 Stat.
1245, clarified what was implicit in those laws: that the Chief
Judgeship was not a mere renaming of the position but a change in its
status that made it the same as the Chief Judge of other inferior
courts.

Chief judges have administrative responsibilities with respect to their
circuits, and preside over any panel on which they serve unless the
circuit justice (i.e., the Supreme Court justice responsible for the
circuit) is also on the panel. Unlike the Supreme Court, where one
justice is specifically nominated to be chief, the office of chief judge
rotates among the circuit judges. To be chief, a judge must have been in
active service on the court for at least one year, be under the age of
65, and have not previously served as chief judge. A vacancy is filled
by the judge highest in seniority among the group of qualified judges.
The chief judge serves for a term of seven years or until age 70,
whichever occurs first. The age restrictions are waived if no members of
the court would otherwise be qualified for the position.

When the office was created in 1948, the chief judge was the
longest-serving judge who had not elected to retire on what has since
1958 been known as senior status or declined to serve as chief judge.
After August 6, 1959, judges could not become or remain chief after
turning 70 years old. The current rules have been in operation since
October 1, 1982.

\section{Succession of seats}\label{succession-of-seats}

\begin{itemize}
\item
  \emph{The seat that was originally the Chief Justiceship is numbered
  as Seat 1; the other seats are numbered in order of their creation.}
\item
  \emph{That seat is filled by the next circuit judge appointed by the
  President.}
\item
  \emph{The court has eleven seats for active judges after the
  elimination of seat seven under the Court Security Improvement Act of
  2007.}
\end{itemize}

The court has eleven seats for active judges after the elimination of
seat seven under the Court Security Improvement Act of 2007. The seat
that was originally the Chief Justiceship is numbered as Seat 1; the
other seats are numbered in order of their creation. If seats were
established simultaneously, they are numbered in the order in which they
were filled. Judges who retire into senior status remain on the bench
but leave their seat vacant. That seat is filled by the next circuit
judge appointed by the President.

\section{See also}\label{see-also}

\begin{itemize}
\item
  \emph{List of current United States Circuit Judges}
\item
  \emph{Federal judicial appointment history\#DC Circuit}
\end{itemize}

Federal judicial appointment history\#DC Circuit

List of current United States Circuit Judges

\section{Notes}\label{notes}

\section{References}\label{references}

\begin{itemize}
\item
  \emph{"U. S. Court of Appeals for the District of Columbia Circuit".}
\end{itemize}

"Standard Search". Federal Law Clerk Information System. Archived from
the original on 2005-10-21. Retrieved 2005-06-02.\\
Source for the duty station for Judge Williams

"Instructions for Judicial Directory". Website of the University of
Texas Law School. Archived from the original on 2005-11-11. Retrieved
2005-07-04.\\
Source for the duty station for Judges Silberman and Buckley\\
Data is current to 2002

"U. S. Court of Appeals for the District of Columbia Circuit". Official
website of the Federal Judicial Center. Archived from the original on
2005-04-04. Retrieved 2005-05-26.\\
Source for the state, lifetime, term of active judgeship, term of chief
judgeship, term of senior judgeship, appointer, termination reason, and
seat information

\section{External links}\label{external-links}

\begin{itemize}
\item
  \emph{United States Court of Appeals for the District of Columbia
  Circuit}
\item
  \emph{What Makes the DC Circuit so Different?}
\end{itemize}

United States Court of Appeals for the District of Columbia Circuit

Recent opinions from FindLaw

What Makes the DC Circuit so Different? A Historical View - Article by
Chief Justice John G. Roberts, Jr.
