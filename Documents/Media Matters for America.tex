\textbf{From Wikipedia, the free encyclopedia}

https://en.wikipedia.org/wiki/Media\%20Matters\%20for\%20America\\
Licensed under CC BY-SA 3.0:\\
https://en.wikipedia.org/wiki/Wikipedia:Text\_of\_Creative\_Commons\_Attribution-ShareAlike\_3.0\_Unported\_License

\section{Media Matters for America}\label{media-matters-for-america}

\begin{itemize}
\item
  \emph{Media Matters for America (MMfA) is a progressive 501(c)(3),
  nonprofit organization, with the mission of "comprehensively
  monitoring, analyzing, and correcting conservative misinformation in
  the U.S.}
\item
  \emph{media."}
\item
  \emph{It is known for its aggressive criticism of conservative
  journalists and media outlets, including its "War on Fox News."}
\end{itemize}

Media Matters for America (MMfA) is a progressive 501(c)(3), nonprofit
organization, with the mission of "comprehensively monitoring,
analyzing, and correcting conservative misinformation in the U.S.
media." MMfA was founded in 2004 by journalist and political activist
David Brock as a counterweight to the conservative Media Research
Center. It is known for its aggressive criticism of conservative
journalists and media outlets, including its "War on Fox News."

\section{Founding}\label{founding}

\begin{itemize}
\item
  \emph{Media Matters for America was founded in May 2004 by David
  Brock, a former conservative journalist who became a progressive.}
\end{itemize}

Media Matters for America was founded in May 2004 by David Brock, a
former conservative journalist who became a progressive. Brock said that
he founded the organization to combat the conservative journalism sector
that he had once been a part of. He founded the group with help from the
Center for American Progress. Initial donors included Leo Hindery, Susie
Tompkins Buell, and James Hormel.

\section{Organization overview}\label{organization-overview}

\section{Research}\label{research}

\begin{itemize}
\item
  \emph{An annual feature on the Media Matters website is the title of
  "Misinformer of the Year", which is given to the journalist,
  commentator, or network that Media Matters contends was responsible
  for the most factual errors or claims.}
\item
  \emph{Diaz said Media Matters had obscured the nuanced ideological
  positions of some columnists classified in the study as conservative.}
\end{itemize}

Media Matters analyzes American news sources including NBC, ABC, CBS,
PBS, CNN, MSNBC, CNBC, OAN, Breitbart, and the Fox News channel. Its
techniques include content analysis, fact checking, monitoring, and
comparison of quotes or presentations from media figures to primary
documents such as Pentagon or Government Accountability Office reports.

Beginning in 2006, Media Matters for America has released a number of
studies which contend that Democrats and progressives are outnumbered by
Republicans and conservatives in terms of guest appearances on
television news programs.

On September 12, 2007, Media Matters released its study of 1,377 U.S.
newspapers and the 201 syndicated political columnists the papers carry
on a regular basis. Media Matters said "in paper after paper, state
after state, and region after region, conservative syndicated columnists
get more space than their progressive counterparts."

John Diaz, editorial page editor of the San Francisco Chronicle, said by
over-factoring conservative columns in smaller newspapers, Media
Matters' study had overestimated how many conservative columns appeared
in daily newspapers. Diaz said Media Matters had obscured the nuanced
ideological positions of some columnists classified in the study as
conservative.

An annual feature on the Media Matters website is the title of
"Misinformer of the Year", which is given to the journalist,
commentator, or network that Media Matters contends was responsible for
the most factual errors or claims. Recipients of this award have
included: Bill O'Reilly, Chris Matthews, ABC, Sean Hannity, Glenn Beck,
Sarah Palin, Rupert Murdoch, Rush Limbaugh, CBS News, George Will, The
Center for Medical Progress, the fake news ecosystem, and the alt-right.

\section{Funding}\label{funding}

\begin{itemize}
\item
  \emph{During a 2014 CNN interview, David Brock stated that Soros'
  contributions were "less than 10 percent" of Media Matters' budget.}
\item
  \emph{Media Matters has a policy of not comprehensively listing
  donors.}
\item
  \emph{According to Politico: "Media Matters, however has received
  funding from or formed partnerships with several groups that Soros
  funds or has funded.}
\end{itemize}

MMfA began with the help of \$2 million in donations. According to Byron
York, additional funding came from MoveOn.org and the New Democrat
Network.

In 2004, MMfA received the endorsement of the Democracy Alliance, a
partnership of wealthy and politically active progressive donors. The
Alliance itself does not fund endorsees, but many wealthy Alliance
members acted on the endorsement and donated directly to MMfA.

Media Matters has a policy of not comprehensively listing donors. In
2010, six years after the Democracy Alliance initially endorsed MMfA,
financier George Soros --- a founding and continuing member of the
Alliance --- announced that he was donating \$1 million to MMfA. Soros
said his concern over "recent evidence suggesting that the incendiary
rhetoric of Fox News hosts may incite violence" had moved him to donate
to MMfA. During a 2014 CNN interview, David Brock stated that Soros'
contributions were "less than 10 percent" of Media Matters' budget.
According to Politico: "Media Matters, however has received funding from
or formed partnerships with several groups that Soros funds or has
funded. These include the Tides Foundation, Democracy Alliance,
Moveon.org and the Center for American Progress.''

\section{Personnel}\label{personnel}

\begin{itemize}
\item
  \emph{Media Matters has hired numerous political professionals who had
  worked for Democratic politicians and for other progressive groups.}
\item
  \emph{In 2014, the staff of Media Matters voted to join the Service
  Employees International Union (SEIU).}
\item
  \emph{According to The New York Times, Media Matters "helped lay the
  groundwork" for Hillary Clinton's 2016 presidential campaign.}
\end{itemize}

John Podesta, former chief of staff to President Bill Clinton, provided
office space for Media Matters early in its formation at the Center for
American Progress, a Democratic think tank which Podesta established in
2002. Hillary Clinton advised Media Matters in its early stages out of a
belief that progressives should follow conservatives in forming think
tanks and advocacy groups to support their political goals. According to
The New York Times, Media Matters "helped lay the groundwork" for
Hillary Clinton's 2016 presidential campaign.

Media Matters has hired numerous political professionals who had worked
for Democratic politicians and for other progressive groups. In 2004,
National Review referred to MMfA staffers who had recently worked on the
presidential campaigns of John Edwards and Wesley Clark, for Congressman
Barney Frank, and for the Democratic Congressional Campaign Committee.

Eric E. Burns served as MMFA's president until 2011. Burns was succeeded
by Matt Butler, and then, in 2013, by Bradley Beychok. In late 2016,
Angelo Carusone replaced Bradley Beychok as MMFA's president. Under
Carusone, the organization's focus has shifted toward focusing on the
alt-right, conspiracy theories, and fake news.

In 2014, the staff of Media Matters voted to join the Service Employees
International Union (SEIU). Media Matters management had declined to
recognize the union through a card check process, instead exercising its
right to force a union election.

\section{Initiatives}\label{initiatives}

\begin{itemize}
\item
  \emph{At launch the site fully incorporated Media Matters's content on
  LGBT issues.}
\item
  \emph{In 2015, the formal Equality Matters program was deactivated and
  merged with the LGBT Program within Media Matters.}
\item
  \emph{David Brock established Media Matters Action Network to track
  conservative politicians and organizations.}
\item
  \emph{Media Matters runs the website DropFox.com and works to get
  advertisers to boycott Fox News.}
\end{itemize}

Established to "incubate a new generation of liberal pundits", the
Progressive Talent Initiative trains potential pundits in media skills.

David Brock established Media Matters Action Network to track
conservative politicians and organizations. Organized as a 501(c)(4)
nonprofit group, the Media Matters Action Network can lobby and engage
in political campaign work. The New York Times reported that it was "set
to take on an expanded role in the 2012 elections, including potentially
running television ads".

Brock established American Bridge 21st Century as a super PAC associated
with Media Matters for America. Kathleen Kennedy Townsend, daughter of
Robert F. Kennedy, and a former lieutenant governor of Maryland, chairs
the board of directors of American Bridge. American Bridge is focused on
opposition research.

In 2009, Media Matters Action Network launched the Conservative
Transparency website, aimed at tracking the funding of conservative
activist organizations. Media Matters Action Network established the
Political Correction project with the goal of holding conservative
politicians and advocacy groups accountable.

In December 2010, Media Matters Action Network started
EqualityMatters.org, a site "in support of gay equality". At launch the
site fully incorporated Media Matters's content on LGBT issues. Designed
to provide talking points for liberal activists and politicians, Brock
set up the Message Matters project. Media Matters runs the website
DropFox.com and works to get advertisers to boycott Fox News. One
target, Orbitz, initially referred to Media Matters' efforts as a "smear
campaign", but agreed, on June 9, 2011, following a three-week campaign
by prominent LGBT organizations, to "review the policies and process
used to evaluate where advertising is placed". In 2015, the formal
Equality Matters program was deactivated and merged with the LGBT
Program within Media Matters.

\section{Reception}\label{reception}

\begin{itemize}
\item
  \emph{Some news organizations have cited Media Matters reports and
  credited it for bringing attention to issues including the story of
  James Guckert, formerly a reporter for the web-based Talon News.}
\item
  \emph{Leftist columnists and writers such as Paul Krugman and the late
  Molly Ivins cited Media Matters or identified it as a helpful source.}
\end{itemize}

In 2008, Jacques Steinberg of The New York Times reported that David
Folkenflik of National Public Radio had told him that although Media
Matters looked "at every dangling participle, every dependent clause,
every semicolon, every quotation" for the benefit of "a cause, a party,
a candidate, that they may have some feelings for," they were still a
useful source for leads, partly due to the "breadth of their research."
Conversely, political analyst and columnist Stuart Rothenberg told
Steinberg that he did not pay attention to them, as he had no confidence
in "ideological stuff." In Steinberg's view, Media Matters is a new
weapon for the Democratic Party employing "rapid-fire, technologically
sophisticated means to call out what it considers 'conservative
misinformation' on air or in print, then feed it to a Rolodex of
reporters, cable channels and bloggers hungry for grist."

According to a 2010 opinion piece by "M. S." on the blog of The
Economist magazine, "because it is dedicated to critiquing distortions
by conservatives, its critiques carry no weight with conservatives."

Some news organizations have cited Media Matters reports and credited it
for bringing attention to issues including the story of James Guckert,
formerly a reporter for the web-based Talon News. During George W.
Bush's administration, Guckert gained White House press access using the
pseudonym Jeff Gannon and attended 155 White House press briefings. It
was revealed that he had also worked as a prostitute soliciting male
clientele on the internet with photos of himself fully naked. Leftist
columnists and writers such as Paul Krugman and the late Molly Ivins
cited Media Matters or identified it as a helpful source.

In 2016, MMfA was accused by The New Republic of lowering its research
standards in defense of Hillary Clinton.

\section{Controversies}\label{controversies}

\section{Don Imus}\label{don-imus}

\begin{itemize}
\item
  \emph{The Wall Street Journal said Imus's apology "seemed to make
  matters worse, with critics latching on to Mr. Imus's use of the
  phrase 'you people.'"}
\item
  \emph{On April 4, 2007, Media Matters posted a video clip of Don Imus
  calling the Rutgers University women's basketball team members
  "nappy-headed hoes" and made their discovery known in Media Matters'
  daily e-mailing to hundreds of journalists.}
\end{itemize}

On April 4, 2007, Media Matters posted a video clip of Don Imus calling
the Rutgers University women's basketball team members "nappy-headed
hoes" and made their discovery known in Media Matters' daily e-mailing
to hundreds of journalists. The next day, according to The Wall Street
Journal, "top news outlets didn't mention the incident." It was
objections made to CBS Radio by the National Association of Black
Journalists that led to an on-the-air apology from Imus. MSNBC, calling
Imus's comments "racist" and "abhorrent," suspended Imus' show, and
within minutes, CBS suspended Imus's radio show. The Wall Street Journal
said Imus's apology "seemed to make matters worse, with critics latching
on to Mr. Imus's use of the phrase 'you people.'" Included among those
dissatisfied with Imus's apology and suspension were the coach of the
Rutgers team and a group of MSNBC African-American employees. After
Procter \& Gamble pulled advertising from all of MSNBC's daytime
schedule, and other advertisers, including General Motors and American
Express requested CBS to cancel any upcoming advertising they had bought
for "Imus in the Morning", MSNBC and CBS dropped Imus's show.

\section{Rush Limbaugh "phony
soldiers"}\label{rush-limbaugh-phony-soldiers}

\begin{itemize}
\item
  \emph{In 2007, Media Matters' reported radio talk show host Rush
  Limbaugh saying Iraq War veterans opposed to the war as "the phony
  soldiers."}
\item
  \emph{Limbaugh said he was the victim of a "smear" by Media Matters,
  which had taken out of context and selectively edited his comments.}
\item
  \emph{After Limbaugh published what he said was the entire transcript
  of phony soldiers discussion, Media Matters reported that over a
  minute and 30 seconds was omitted without "notation or ellipsis to
  indicate that there is, in fact, a break in the transcript."}
\end{itemize}

In 2007, Media Matters' reported radio talk show host Rush Limbaugh
saying Iraq War veterans opposed to the war as "the phony soldiers."
Limbaugh later said he was speaking of only one soldier, Jesse MacBeth,
who had falsely claimed to have been decorated for valor but had never
seen combat. Limbaugh said he was the victim of a "smear" by Media
Matters, which had taken out of context and selectively edited his
comments. After Limbaugh published what he said was the entire
transcript of phony soldiers discussion, Media Matters reported that
over a minute and 30 seconds was omitted without "notation or ellipsis
to indicate that there is, in fact, a break in the transcript." Limbaugh
told National Review that the gap between referring to "phony soldiers"
and MacBeth was a delay because his staff printed out an ABC news story
that reported on what it called "phony soldiers" and that his transcript
and audio edits were "for space and relevance reasons, not to hide
anything."

The Associated Press, CNN, and ABC reported on the controversy, as
political satirist and fictional pundit Stephen Colbert lampooned
Limbaugh and his defenders saying: "Hey, Media Matters, you want to end
offensive speech? Then stop recording it for people who would be
offended."

\section{Bill O'Reilly Harlem
restaurant}\label{bill-oreilly-harlem-restaurant}

\begin{itemize}
\item
  \emph{O'Reilly said Media Matters misleadingly took comments spoken
  five minutes apart and presented them as one.}
\item
  \emph{In an appearance on NBC's Today with Matt Lauer, Media Matters
  senior fellow Paul Waldman said Media Matters had included "the full
  audio, the full transcript, nothing was taken out of context".}
\end{itemize}

In October 2007 television and radio host and commentator Bill O'Reilly
said a Media Matters' headline declaring "O'Reilly surprised 'there was
no difference' between Harlem restaurant and other New York restaurants"
took out of context comments he made regarding a pleasant dinner he
shared with Al Sharpton at a Harlem restaurant. O'Reilly said Media
Matters misleadingly took comments spoken five minutes apart and
presented them as one. In an appearance on NBC's Today with Matt Lauer,
Media Matters senior fellow Paul Waldman said Media Matters had included
"the full audio, the full transcript, nothing was taken out of context".

\section{Laura Schlessinger racial
slur}\label{laura-schlessinger-racial-slur}

\begin{itemize}
\item
  \emph{On August 12, 2010, Media Matters reported that radio host Laura
  Schlessinger said "nigger" eleven times during a discussion with an
  African-American woman, continuing to say it after the caller took
  offense at the word.}
\item
  \emph{Media Matters said that, as the boycott was not
  "government-sanctioned censorship", her First Amendment rights had not
  been violated.}
\item
  \emph{Schlessinger held Media Matters responsible for the boycott,
  which she called a typical tactic of the group to fulfill its "sole
  purpose of silencing people."}
\end{itemize}

On August 12, 2010, Media Matters reported that radio host Laura
Schlessinger said "nigger" eleven times during a discussion with an
African-American woman, continuing to say it after the caller took
offense at the word. Schlessinger told the woman she was too sensitive
and that a double standard determined who could say the word.
Schlessinger also said that those "hypersensitive" about color should
not "marry outside of their race." The caller had earlier in the
discussion said her husband was white. Schlessinger apologized for the
epithet the day after the broadcast. A joint statement of Media Matters
and other organizations noted that although Schlessinger "attempted to
apologize for using the epithet, the racist diatribe on Tuesday's show
extends far beyond the use of a single word" and urged advertisers to
boycott her show. After General Motors, OnStar, and Motel 6 pulled their
advertising, Schlessinger said she would not renew her syndication
contract set to expire December 2010. In January 2011 her show resumed
on satellite radio.

Schlessinger held Media Matters responsible for the boycott, which she
called a typical tactic of the group to fulfill its "sole purpose of
silencing people." She said the boycotts' "threat of attack on my
advertisers and stations" had violated her First Amendment free speech
rights. Media Matters said that, as the boycott was not
"government-sanctioned censorship", her First Amendment rights had not
been violated.

\section{War on Fox}\label{war-on-fox}

\begin{itemize}
\item
  \emph{MMfA said the greater attention given to Fox was part of a
  campaign to educate the public about what it regarded as the
  distortions of conservative media, and the greater attention given to
  Fox was in line with its prominence.}
\item
  \emph{MMfA said that changing Fox, not shutting it down, was its
  goal.}
\item
  \emph{In 2010, MMfA declared a "War on Fox."}
\end{itemize}

In 2010, MMfA declared a "War on Fox." David Brock said MMfA would focus
its efforts on Fox and select conservative websites in what Brock called
an "all-out campaign of 'guerrilla warfare and sabotage.'" MMfA said the
greater attention given to Fox was part of a campaign to educate the
public about what it regarded as the distortions of conservative media,
and the greater attention given to Fox was in line with its prominence.
MMfA said its Drop Fox campaign for advertisers to boycott Fox was also
part of the organisation's educational mission. MMfA said that changing
Fox, not shutting it down, was its goal.

In December 2013, the War on Fox was officially concluded, with MMfA
Executive VP Angelo Carusone claiming the "War on Fox is over. And it's
not just that it's over, but it was very successful. To a large extent,
we won," claiming to have "effectively discredited the network's desire
to be seen as 'fair and balanced.'" Around that time, Glenn Beck had
departed the network and Sean Hannity's time slot was moved from 9 p.m.
to 10 p.m.

\section{Tax-exempt status challenge}\label{tax-exempt-status-challenge}

\begin{itemize}
\item
  \emph{In another report, Politico said Fox News and Fox Business
  campaigns held, "The non-profit status as an educator is violated by
  partisan attacks.}
\item
  \emph{In an interview with Fox News, Gray said "It's not unlawful.}
\item
  \emph{Marcus Owens, former director of the IRS's Exempt Organizations
  Division, told Politico in 2011 that he believed the law was on Media
  Matters's side.}
\end{itemize}

C. Boyden Gray, former White House counsel for George H. W. Bush and Fox
consultant, sent a letter to the IRS in 2011 alleging that MMfA's
activities including the War on Fox were not primarily educational, but
instead "unlawful conduct" and asking for that MMfA's tax-exempt status
to be revoked. Prior to Gray's IRS petition, Politico reported that Fox
News had run "more than 30 segments calling for the nonprofit group to
be stripped of its tax-exempt status." In another report, Politico said
Fox News and Fox Business campaigns held, "The non-profit status as an
educator is violated by partisan attacks. That sentiment was first laid
out by a piece written by Gray for The Washington Times in June." In an
interview with Fox News, Gray said "It's not unlawful. It's just not
charitable."

MMfA vice-president Ari Rabin-Havt said "C. Boyden Gray is {[}a{]}
Koch-affiliated, former Fox News contributor whose flights of fancy have
already been discredited by actual experts in tax law." Gray denied
having been on Fox's payroll while he was a Fox consultant in 2005, but
at that time, Fox had said Gray was a contributor, adding: "We pay
contributors for strong opinions."

Marcus Owens, former director of the IRS's Exempt Organizations
Division, told Politico in 2011 that he believed the law was on Media
Matters's side. Owens told Fox Business that only an IRS probe could
reveal if partisan activity takes up a substantial enough part of MMfA's
operations to disallow its tax-free status; the IRS allows limited
political activity at nonprofits if it does not take up a substantial
amount of their operations.

\section{2016 presidential election}\label{presidential-election}

\begin{itemize}
\item
  \emph{Roller wrote that the "ferocity with which Media Matters has
  defended Clinton can verge on the absurd."}
\item
  \emph{After Clinton's loss in the 2016 presidential election, Clio
  Chang and Alex Shephard wrote in The New Republic that "in our
  numerous conversations with past Media Matters staff, there was a
  consensus that in the lead-up to Clinton's announcement of her
  candidacy in 2015, the organization's priority shifted away from
  {[}its stated mission{]} towards running defense for Clinton" which
  "damaged Media Matters's credibility and hurt the work it did in other
  areas."}
\end{itemize}

After his political conversion, Brock became a strong supporter of Bill
and Hillary Clinton. In 2015, Emma Roller in The Atlantic described MMFA
as part of Brock's "three-pronged empire", along with the super PACs
American Bridge 21st Century and Correct the Record. Roller wrote that
the "ferocity with which Media Matters has defended Clinton can verge on
the absurd." After Clinton's loss in the 2016 presidential election,
Clio Chang and Alex Shephard wrote in The New Republic that "in our
numerous conversations with past Media Matters staff, there was a
consensus that in the lead-up to Clinton's announcement of her candidacy
in 2015, the organization's priority shifted away from {[}its stated
mission{]} towards running defense for Clinton" which "damaged Media
Matters's credibility and hurt the work it did in other areas."

\section{Tucker Carlson blog posts}\label{tucker-carlson-blog-posts}

\begin{itemize}
\item
  \emph{In March 2019, MMfA released audio recordings of Fox News host
  Tucker Carlson, in which he made remarks demeaning to women between
  2006 and 2011 on the call-in show hosted by shock jock Bubba The Love
  Sponge.}
\item
  \emph{After Carlson's remarks had been widely reported, Carlson
  tweeted: "Media Matters caught me saying something naughty on a radio
  show more than a decade ago" and declined to apologize.}
\end{itemize}

In March 2019, MMfA released audio recordings of Fox News host Tucker
Carlson, in which he made remarks demeaning to women between 2006 and
2011 on the call-in show hosted by shock jock Bubba The Love Sponge.
Among other comments, Carlson called rape shield laws "unfair", defended
fundamentalist Mormon church leader Warren Jeffs, who had been charged
of child sexual assault, and called women "extremely primitive". After
Carlson's remarks had been widely reported, Carlson tweeted: "Media
Matters caught me saying something naughty on a radio show more than a
decade ago" and declined to apologize. The following day, MMfA released
a second set of audio recordings in which Carlson referred to Iraqis as
``semiliterate primitive monkeys'' and said they ``don't use toilet
paper or forks.'' Carlson also suggested that immigrants to the U.S.
should be ``hot'' or ``really smart'' and that white men ``created
civilization.''

The Daily Caller, which Carlson co-founded, responded by resurfacing
blog posts made by MMfA's president Angelo Carusone. These blog posts
included derogatory comments about transvestites, Jews, and people from
Japan and Bangladesh. Carusone responded by saying that posts were
supposed to be a "caricature of what a right wing blowhard would sound
like if he was living my life" and apologized for the "gross" remarks.
Erik Wemple, writing in The Washington Post, expressed scepticism at
Carusone's parody explanation.

\section{See also}\label{see-also}

\begin{itemize}
\item
  \emph{Accuracy in Media}
\item
  \emph{Media bias in the United States}
\item
  \emph{Media monitoring service}
\item
  \emph{Media bias}
\end{itemize}

Accuracy in Media

Fairness and Accuracy in Reporting

Journalism ethics and standards

Media bias

Media bias in the United States

People for the American Way

Media monitoring service

\section{References}\label{references}

\section{External links}\label{external-links}

\begin{itemize}
\item
  \emph{Media Matters Action Network website}
\item
  \emph{"Media Matters for America Internal Revenue Service filings".}
\end{itemize}

Official website

Media Matters Action Network website

American Bridge website

"Media Matters for America Internal Revenue Service filings". ProPublica
Nonprofit Explorer.
