\textbf{From Wikipedia, the free encyclopedia}

https://en.wikipedia.org/wiki/Golan\%20Heights\%20Law\\
Licensed under CC BY-SA 3.0:\\
https://en.wikipedia.org/wiki/Wikipedia:Text\_of\_Creative\_Commons\_Attribution-ShareAlike\_3.0\_Unported\_License

\section{Golan Heights Law}\label{golan-heights-law}

\begin{itemize}
\item
  \emph{Although the law did not use the term, it was considered by the
  international community and some members of the Israeli opposition as
  Annexation of the Golan Heights.}
\item
  \emph{The Golan Heights Law is the Israeli law which applies Israel's
  government and laws to the Golan Heights.}
\end{itemize}

The Golan Heights Law is the Israeli law which applies Israel's
government and laws to the Golan Heights. It was ratified by the Knesset
by a vote of 63―21, on December 14, 1981. Although the law did not use
the term, it was considered by the international community and some
members of the Israeli opposition as Annexation of the Golan Heights.

The law was passed half a year after the peace treaty with Egypt which
included Israeli withdrawal from the Sinai Peninsula.

In February 2018, the Prime Minister of Israel Benjamin Netanyahu stated
that "the Golan Heights will remain Israel's forever", after his
political rival Yair Lapid called on the international community to
recognize Israeli sovereignty over the Golan Heights two months earlier.

\section{The law}\label{the-law}

\begin{itemize}
\item
  \emph{The three broad provisions in the Golan Heights Law are the
  following:}
\item
  \emph{"The Law, jurisdiction and administration of the State will take
  effect in the Golan Heights, as described in the Schedule."}
\item
  \emph{"The Minister of the Interior is placed in-charge of the
  implementation of this Law, and is entitled, in consultation with the
  Minister of Justice, to enact regulations for its implementation and
  to formulate regulations on interim provisions regarding the continued
  application of regulations, directives, administrative directives, and
  rights and duties that were in effect in the Golan Heights prior to
  the acceptance of this Law."}
\end{itemize}

The three broad provisions in the Golan Heights Law are the following:

1. "The Law, jurisdiction and administration of the State will take
effect in the Golan Heights, as described in the Schedule."

2. "This Law will begin taking effect on the day of its acceptance in
the Knesset."

3. "The Minister of the Interior is placed in-charge of the
implementation of this Law, and is entitled, in consultation with the
Minister of Justice, to enact regulations for its implementation and to
formulate regulations on interim provisions regarding the continued
application of regulations, directives, administrative directives, and
rights and duties that were in effect in the Golan Heights prior to the
acceptance of this Law."

Signed:

Yitzhak Navon (President)

Menachem Begin (Prime Minister)

Yosef Burg (Interior Minister)

Passed in the Knesset with a majority of 63 in favour, 21 against.

\section{Controversies}\label{controversies}

\begin{itemize}
\item
  \emph{While the Israeli public at large, and especially the law's
  critics, viewed it as an annexation, the law avoids the use of the
  word.}
\item
  \emph{Substantially, the law has mainly been criticized for
  potentially hindering future negotiations with Syria.}
\item
  \emph{I do not use it," and noting that similar wording was used in a
  1967 law authorizing the government to apply Israeli law to any part
  of the Land of Israel.}
\end{itemize}

The law was not recognised internationally and determined null and void
by United Nations Security Council Resolution 497.

Unusually, all three readings took place on the same day. This procedure
was heavily criticized by the centre-left opposition. Substantially, the
law has mainly been criticized for potentially hindering future
negotiations with Syria.{[}citation needed{]}

While the Israeli public at large, and especially the law's critics,
viewed it as an annexation, the law avoids the use of the word. Prime
Minister Menachem Begin responded to Amnon Rubinstein's criticism by
saying, "You use the word 'annexation.' I do not use it," and noting
that similar wording was used in a 1967 law authorizing the government
to apply Israeli law to any part of the Land of Israel.{[}citation
needed{]}

\section{See also}\label{see-also}

\begin{itemize}
\item
  \emph{Jerusalem Law}
\item
  \emph{International law and the Arab-Israeli conflict}
\end{itemize}

International law and the Arab-Israeli conflict

Jerusalem Law

\section{References}\label{references}
