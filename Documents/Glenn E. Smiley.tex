\textbf{From Wikipedia, the free encyclopedia}

https://en.wikipedia.org/wiki/Glenn\%20E.\%20Smiley\\
Licensed under CC BY-SA 3.0:\\
https://en.wikipedia.org/wiki/Wikipedia:Text\_of\_Creative\_Commons\_Attribution-ShareAlike\_3.0\_Unported\_License

\section{Glenn E. Smiley}\label{glenn-e.-smiley}

\begin{itemize}
\item
  \emph{Glenn Smiley (April 19, 1910 -- September 14, 1993) was a white
  civil rights consultant and leader.}
\item
  \emph{Just three years before his 1993 death, Smiley opened the King
  Center in Los Angeles.}
\item
  \emph{After the Civil Rights Movement, Smiley continued to employ
  nonviolence and worked for several organizations promoting peace in
  South American countries.}
\end{itemize}

Glenn Smiley (April 19, 1910 -- September 14, 1993) was a white civil
rights consultant and leader. He closely studied the doctrine of Mahatma
Gandhi and became convinced that racism and segregation were most likely
to be overcome without the use of violence, and began studying and
teaching peaceful tactics. As an employee of the Fellowship of
Reconciliation (FOR), he visited Martin Luther King, Jr. in Montgomery,
Alabama in 1956 during the Montgomery bus boycott where Smiley advised
King and his associates on nonviolent tactics, and was able to convince
King that nonviolence was a feasible solution to racial tension. Smiley,
together with Bayard Rustin and others, helped convince King and his
associates that complete nonviolence and nonviolent direct action were
the most effective methods and tools to use during protest. After the
Civil Rights Movement, Smiley continued to employ nonviolence and worked
for several organizations promoting peace in South American countries.
Just three years before his 1993 death, Smiley opened the King Center in
Los Angeles.

\section{Early life life}\label{early-life-life}

\begin{itemize}
\item
  \emph{Glenn Smiley was born in Loraine, Texas on April 19, 1910.}
\end{itemize}

Glenn Smiley was born in Loraine, Texas on April 19, 1910. He attended
several universities, including McMurry College, Southwestern
University, and the University of Arizona before graduating from
University of Redlands.

\section{Career}\label{career}

\begin{itemize}
\item
  \emph{Smiley went on to have a professional relationship with Martin
  Luther King, in which he advised King on nonviolence tactics and
  emphasized the importance of nonviolence in the success of the Civil
  Rights Movement.}
\item
  \emph{Smiley was already impress with Dr. King's leadership, but was
  critical of King for having a bodyguard.}
\end{itemize}

Smiley worked as the preacher to a Methodist congregation in Arizona,
and later California for 14 years. After his work in ministry, Smiley
went on to work for several organizations, including the Congress of
Racial Equality (CORE) in 1942, and later served as the national field
secretary of the Fellowship of Reconciliation (FOR). When World War II
broke out and the time came for Smiley to enlist in armed services, he
refused to participate. He also opted not to take the clergy exception,
which was available to him as a minister. These actions classified him
as a conscientious objector and he went on to serve time in prison in
1945 for his lack of compliance. Smiley believed that prison is only
secondary to war in dehumanization and violence. In Smiley's sixties, he
suffered from 44 small strokes. These strokes greatly affected his
memory and speech for 15 years, until one day he woke up and seemed to
be completely back to his normal self and even went on to give 103 major
lectures.

During his work in ministry in the 1940s, Smiley developed an interest
for the methods of Mahatma Gandhi and his methods of self-discipline and
nonviolence. From these studies, he developed his theory that
nonviolence was the most effective way to combat discrimination. Smiley
first used his theory of nonviolence in the late 1940s when he attempted
to spur integration of tearooms of department stores in the Los Angeles
area. Smiley went on to have a professional relationship with Martin
Luther King, in which he advised King on nonviolence tactics and
emphasized the importance of nonviolence in the success of the Civil
Rights Movement. Smiley was already impress with Dr. King's leadership,
but was critical of King for having a bodyguard. In a letter that Smiley
had written to some of his friends, he was quoted writing, "If King can
really be won to a faith of nonviolence there is no end to what he can
do. Soon he will be able to direct movement by sheer force of being the
symbol of resistance. Smiley also persuaded King that there needs to be
an active dialogue between the white and black ministers in the South.
King sent Smiley around the South preaching the doctrine to church
congregations and civil-rights groups, and nonviolence quickly became a
binding premise of King's Southern Christian Leadership Conference.

During the Montgomery Bus Boycott, Smiley participated by spreading news
of the boycott to his congregation. Smiley was also charged with
appealing to Southern white people, and accessed group meetings of
organizations such as the Ku Klux Klan and the WCC. He is quoted saying,
"my assignment was to make every contact possible in the white
community." After the resolution of Browder v. Gayle on December 17,
1956, it was ruled by the Supreme Court that segregation on city busses
is unconstitutional; the MIA developed a set of guidelines to help black
residents successfully ride on the newly integrated busses. Smiley,
along with Martin Luther King and other MIA leaders, was an integral
author of these new guidelines.

After the Supreme Court's ruling in Browder v. Gayle Smiley rode with
Martin Luther King and Reverend Ralph D. Abernathy on the first day that
bus segregation ended in Montgomery. Smiley later said that he took the
bus ride to get a reaction, as his organizational work had been urging
nonviolence.\\
Later during the student sit-in movement during the 1960s, Smiley was a
strong supporter and urged the students to attend a conference at Shaw
University that would go on to be the birthplace of the Student
Nonviolent Coordinating Committee (SNCC).

In the 1960s, Smiley founded the Methodist-inspired organization called
Justice-Action-Peace Latin America, which was responsible for organizing
seminars on nonviolence in Latin American countries between the years of
1967 and the early 1970s. Smiley traveled to South American countries,
where he taught nonviolence during the time he worked under the National
Council of Churches and the National Council of Catholic Bishops.
Shortly before his death, Smiley founded the Martin Luther King Center
for Nonviolence in Los Angeles in 1990 to further his lifelong
philosophy of nonviolence. Speaking about the King Center, Smiley
stressed "nonviolence is the most effective way of achieving change
because in the process it does not rip countries apart; it builds, it
does not destroy."

\section{Death}\label{death}

\begin{itemize}
\item
  \emph{Smiley died on September 14, 1993, in Glendale, California, at
  the age of 83.}
\item
  \emph{In a statement issued by Dean Hunsell, a board member of Los
  Angeles chapter of the Martin Luther King Center for Nonviolence, it
  was announced that Smiley died of natural causes likely connected to
  complications from a previous stroke.}
\end{itemize}

Smiley died on September 14, 1993, in Glendale, California, at the age
of 83. In a statement issued by Dean Hunsell, a board member of Los
Angeles chapter of the Martin Luther King Center for Nonviolence, it was
announced that Smiley died of natural causes likely connected to
complications from a previous stroke. Smiley left behind his wife,
Helen, as well as three children, eight grandchildren, and 22
great-grandchildren.

\section{References}\label{references}
