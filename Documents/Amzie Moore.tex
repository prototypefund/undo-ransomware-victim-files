\textbf{From Wikipedia, the free encyclopedia}

https://en.wikipedia.org/wiki/Amzie\%20Moore\\
Licensed under CC BY-SA 3.0:\\
https://en.wikipedia.org/wiki/Wikipedia:Text\_of\_Creative\_Commons\_Attribution-ShareAlike\_3.0\_Unported\_License

\section{Amzie Moore}\label{amzie-moore}

\begin{itemize}
\item
  \emph{Amzie Moore (September 23, 1911 -- February 1, 1982) was an
  African-American civil rights leader and entrepreneur in the
  Mississippi Delta.}
\end{itemize}

Amzie Moore (September 23, 1911 -- February 1, 1982) was an
African-American civil rights leader and entrepreneur in the Mississippi
Delta.

\section{Early life}\label{early-life}

\begin{itemize}
\item
  \emph{After the war, Moore opened a gas station, beauty shop, and
  grocery store on Highway 61 in Cleveland, Mississippi.}
\item
  \emph{Moore was born in 1911 on the Wilkin plantation near the Grenada
  and Carroll county lines.}
\item
  \emph{Even before leaving Mississippi to fight in World War II, Moore
  was involved in race relations.}
\item
  \emph{His business also served as headquarters for the area's civil
  rights efforts.}
\end{itemize}

Moore was born in 1911 on the Wilkin plantation near the Grenada and
Carroll county lines.

Left on his own at fourteen after his mother died in 1925, Moore
completed high school but could not realize his dream of a college
education. Through the rest of his life, however, he worked hard to
educate himself. By 1935 he had gained a federal job in the U. S. Post
Office, considered a lucky opportunity during the Great Depression.

Even before leaving Mississippi to fight in World War II, Moore was
involved in race relations. He served over three and a half years in the
United States Army, including time overseas {[}citation needed{]},
before returning to his job at the post office.

After the war, Moore opened a gas station, beauty shop, and grocery
store on Highway 61 in Cleveland, Mississippi. His business also served
as headquarters for the area's civil rights efforts. At his gas station,
which was one of the few owned by an African American, he refused to
have separate white and black bathrooms.

\section{Regional Council of Negro
Leadership}\label{regional-council-of-negro-leadership}

\begin{itemize}
\item
  \emph{The RCNL sought to encourage entrepreneurship, self-help, and
  civil rights in the Delta.}
\item
  \emph{Beginning in 1951, Moore, Aaron Henry and Medgar Evers worked
  with Dr. T.R.M.}
\end{itemize}

Beginning in 1951, Moore, Aaron Henry and Medgar Evers worked with Dr.
T.R.M. Howard, a self-made entrepreneur, fraternal organization leader,
and surgeon, to build the Regional Council of Negro Leadership (RCNL).
The RCNL sought to encourage entrepreneurship, self-help, and civil
rights in the Delta. He participated in the RCNL's campaign to boycott
gas stations that failed to provide restrooms for blacks. His gas
station was one of the few that allowed blacks to use restrooms between
Memphis and Vicksburg. During this period, Moore also belonged to the
United Order of Friendship, a fraternal society headed by Howard to
provide low-cost medical care to blacks.

In August 1955, as word first got out that Emmett Till was missing,
Evers and Moore quickly became involved, disguising themselves as cotton
pickers and going into the cotton fields searching for anything that
would help find the young Delta visitor. Moore asserted, after
collecting stories first hand from the field laborers, that whites had
murdered thousands of blacks over the years and thrown their bodies into
the region's swamps, rivers, and bayous.

\section{Other Civil Rights Movement
activism}\label{other-civil-rights-movement-activism}

\begin{itemize}
\item
  \emph{Moore conceived of the voter registration campaign that was
  later the centerpiece of Freedom Summer in 1964.}
\item
  \emph{Moses later said that Moore was a guiding force from the start.}
\end{itemize}

Moore conceived of the voter registration campaign that was later the
centerpiece of Freedom Summer in 1964. The local leader welcomed outside
help including SNCC organizer Bob Moses, coming into the Delta from New
York City to build the Student Nonviolent Coordinating Committee or
SNCC. Moses later said that Moore was a guiding force from the start.

\section{Notes}\label{notes}

\section{References}\label{references}

\begin{itemize}
\item
  \emph{Howard's Fight for Civil Rights and Economic Power.}
\item
  \emph{John Dittmer, Local People: the Struggle for Civil Rights in
  Mississippi (1994 book).}
\item
  \emph{Charles M. Payne, I've Got the Light of Freedom: The Organizing
  Tradition and the Mississippi Freedom Struggle or the MFS (1995
  book).}
\end{itemize}

John Dittmer, Local People: the Struggle for Civil Rights in Mississippi
(1994 book).

Charles M. Payne, I've Got the Light of Freedom: The Organizing
Tradition and the Mississippi Freedom Struggle or the MFS (1995 book).

Beito, David and Linda (2009). Black Maverick: T.R.M. Howard's Fight for
Civil Rights and Economic Power. Urbana: University of Illinois Press.
ISBN~978-0-252-03420-6..mw-parser-output
cite.citation\{font-style:inherit\}.mw-parser-output .citation
q\{quotes:"\textbackslash{}"""\textbackslash{}"""'""'"\}.mw-parser-output
.citation .cs1-lock-free
a\{background:url("//upload.wikimedia.org/wikipedia/commons/thumb/6/65/Lock-green.svg/9px-Lock-green.svg.png")no-repeat;background-position:right
.1em center\}.mw-parser-output .citation .cs1-lock-limited
a,.mw-parser-output .citation .cs1-lock-registration
a\{background:url("//upload.wikimedia.org/wikipedia/commons/thumb/d/d6/Lock-gray-alt-2.svg/9px-Lock-gray-alt-2.svg.png")no-repeat;background-position:right
.1em center\}.mw-parser-output .citation .cs1-lock-subscription
a\{background:url("//upload.wikimedia.org/wikipedia/commons/thumb/a/aa/Lock-red-alt-2.svg/9px-Lock-red-alt-2.svg.png")no-repeat;background-position:right
.1em center\}.mw-parser-output .cs1-subscription,.mw-parser-output
.cs1-registration\{color:\#555\}.mw-parser-output .cs1-subscription
span,.mw-parser-output .cs1-registration span\{border-bottom:1px
dotted;cursor:help\}.mw-parser-output .cs1-ws-icon
a\{background:url("//upload.wikimedia.org/wikipedia/commons/thumb/4/4c/Wikisource-logo.svg/12px-Wikisource-logo.svg.png")no-repeat;background-position:right
.1em center\}.mw-parser-output
code.cs1-code\{color:inherit;background:inherit;border:inherit;padding:inherit\}.mw-parser-output
.cs1-hidden-error\{display:none;font-size:100\%\}.mw-parser-output
.cs1-visible-error\{font-size:100\%\}.mw-parser-output
.cs1-maint\{display:none;color:\#33aa33;margin-left:0.3em\}.mw-parser-output
.cs1-subscription,.mw-parser-output .cs1-registration,.mw-parser-output
.cs1-format\{font-size:95\%\}.mw-parser-output
.cs1-kern-left,.mw-parser-output
.cs1-kern-wl-left\{padding-left:0.2em\}.mw-parser-output
.cs1-kern-right,.mw-parser-output
.cs1-kern-wl-right\{padding-right:0.2em\}

\section{External links}\label{external-links}

\begin{itemize}
\item
  \emph{SNCC Digital Gateway: Amzie Moore, Documentary website created
  by the SNCC Legacy Project and Duke University, telling the story of
  the Student Nonviolent Coordinating Committee \& grassroots organizing
  from the inside-out}
\end{itemize}

SNCC Digital Gateway: Amzie Moore, Documentary website created by the
SNCC Legacy Project and Duke University, telling the story of the
Student Nonviolent Coordinating Committee \& grassroots organizing from
the inside-out
