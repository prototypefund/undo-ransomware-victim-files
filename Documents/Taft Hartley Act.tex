\textbf{From Wikipedia, the free encyclopedia}

https://en.wikipedia.org/wiki/Taft\%20Hartley\%20Act\\
Licensed under CC BY-SA 3.0:\\
https://en.wikipedia.org/wiki/Wikipedia:Text\_of\_Creative\_Commons\_Attribution-ShareAlike\_3.0\_Unported\_License

\section{Taft--Hartley Act}\label{tafthartley-act}

\begin{itemize}
\item
  \emph{The Labor Management Relations Act of 1947, better known as the
  Taft--Hartley Act, is a United States federal law that restricts the
  activities and power of labor unions.}
\item
  \emph{The Taft--Hartley Act amended the 1935 National Labor Relations
  Act (NLRA), prohibiting unions from engaging in several "unfair labor
  practices."}
\end{itemize}

The Labor Management Relations Act of 1947, better known as the
Taft--Hartley Act, is a United States federal law that restricts the
activities and power of labor unions. It was enacted by the 80th United
States Congress over the veto of President Harry S. Truman, becoming law
on June 23, 1947.

Taft-Hartley was introduced in the aftermath of a major strike wave in
1945 and 1946. Though it was enacted by the Republican-controlled 80th
Congress, the law received significant support from congressional
Democrats, many of whom joined with their Republican colleagues in
voting to override Truman's veto. The act continued to generate
opposition after Truman left office, but it remains in effect.

The Taft--Hartley Act amended the 1935 National Labor Relations Act
(NLRA), prohibiting unions from engaging in several "unfair labor
practices." Among the practices prohibited by the act are jurisdictional
strikes, wildcat strikes, solidarity or political strikes, secondary
boycotts, secondary and mass picketing, closed shops, and monetary
donations by unions to federal political campaigns. The NLRA also
allowed states to pass right-to-work laws banning union shops. Enacted
during the early stages of the Cold War, the law required union officers
to sign non-communist affidavits with the government.

\section{Background}\label{background}

\begin{itemize}
\item
  \emph{In 1945 and 1946, an an unprecedented wave of major strikes
  affected the United States; by February 1946 nearly 2 million workers
  were engaged in strikes or other labor disputes.}
\item
  \emph{Labor leaders, meanwhile, derided the act as a "slave-labor
  bill."}
\item
  \emph{Many of the newly elected congressmen were strongly conservative
  and sought to overturn or roll back New Deal legislation such as the
  National Labor Relations Act of 1935, which had established the right
  of workers to join unions, bargain collectively, and engage in
  strikes.}
\end{itemize}

In 1945 and 1946, an an unprecedented wave of major strikes affected the
United States; by February 1946 nearly 2 million workers were engaged in
strikes or other labor disputes. Organized labor had largely refrained
from striking during World War II, but with the end of the war, labor
leaders were eager to share in the gains from a postwar economic
resurgence. The strikes damaged the political standing of unions, and
the real wages of blue-collar workers fell by over 12 percent in the
year after the surrender of Japan.

The 1946 mid-term elections left Republicans in control of Congress for
the first time since the early 1930s. Many of the newly elected
congressmen were strongly conservative and sought to overturn or roll
back New Deal legislation such as the National Labor Relations Act of
1935, which had established the right of workers to join unions, bargain
collectively, and engage in strikes. Republican Senator Robert A. Taft
and Republican Congressman Fred A. Hartley Jr. each introduced measures
to curtail the power of unions and prevent strikes. Taft's bill passed
the Senate by a 68-to-24 majority, but some of its original provisions
were removed by moderates like Republican Senator Wayne Morse.
Meanwhile, the stronger Hartley bill garnered a 308-to-107 majority in
the House of Representatives. The Taft-Hartley bill that emerged from a
conference committee incorporated aspects from both the House and Senate
bills. The bill was promoted by large business lobbies including the
National Association of Manufacturers.

After spending several days considering how to respond to the bill,
President Truman vetoed Taft--Hartley with a strong message to Congress,
calling the act a "dangerous intrusion on free speech." Labor leaders,
meanwhile, derided the act as a "slave-labor bill." Despite Truman's
all-out effort to prevent a veto override, Congress overrode his veto
with considerable Democratic support, including 106 out of 177 Democrats
in the House, and 20 out of 42 Democrats in the Senate.

\section{Effects of the act}\label{effects-of-the-act}

\begin{itemize}
\item
  \emph{The amendments enacted in Taft--Hartley added a list of
  prohibited actions, or unfair labor practices, on the part of unions
  to the NLRA, which had previously only prohibited unfair labor
  practices committed by employers.}
\item
  \emph{The Taft--Hartley Act prohibited jurisdictional strikes, wildcat
  strikes, solidarity or political strikes, secondary boycotts,
  secondary and mass picketing, closed shops, and monetary donations by
  unions to federal political campaigns.}
\end{itemize}

As stated in Section 1 (29 U.S.C.~§~141), the purpose of the NLRA is:

The amendments enacted in Taft--Hartley added a list of prohibited
actions, or unfair labor practices, on the part of unions to the NLRA,
which had previously only prohibited unfair labor practices committed by
employers. The Taft--Hartley Act prohibited jurisdictional strikes,
wildcat strikes, solidarity or political strikes, secondary boycotts,
secondary and mass picketing, closed shops, and monetary donations by
unions to federal political campaigns. It also required union officers
to sign non-communist affidavits with the government. Union shops were
heavily restricted, and states were allowed to pass right-to-work laws
that ban agency fees. Furthermore, the executive branch of the federal
government could obtain legal strikebreaking injunctions if an impending
or current strike imperiled the national health or safety.

\section{Jurisdictional strikes}\label{jurisdictional-strikes}

\begin{itemize}
\item
  \emph{In jurisdictional strikes, outlawed by Taft--Hartley, a union
  strikes in order to assign particular work to the employees it
  represents.}
\item
  \emph{{[}citation needed{]} A later statute, the Labor Management
  Reporting and Disclosure Act, passed in 1959, tightened these
  restrictions on secondary boycotts still further.}
\end{itemize}

In jurisdictional strikes, outlawed by Taft--Hartley, a union strikes in
order to assign particular work to the employees it represents.
Secondary boycotts and common situs picketing, also outlawed by the act,
are actions in which unions picket, strike, or refuse to handle the
goods of a business with which they have no primary dispute but which is
associated with a targeted business.{[}citation needed{]} A later
statute, the Labor Management Reporting and Disclosure Act, passed in
1959, tightened these restrictions on secondary boycotts still further.

\section{Campaign expenditures}\label{campaign-expenditures}

\begin{itemize}
\item
  \emph{According to First Amendment scholar Floyd Abrams, the Act "was
  the first law barring unions and corporations from making independent
  expenditures in support of or {[}in{]} opposition to federal
  candidates".}
\end{itemize}

According to First Amendment scholar Floyd Abrams, the Act "was the
first law barring unions and corporations from making independent
expenditures in support of or {[}in{]} opposition to federal
candidates".

\section{Closed shops}\label{closed-shops}

\begin{itemize}
\item
  \emph{The law outlawed closed shops which were contractual agreements
  that required an employer to hire only labor union members.}
\item
  \emph{The National Labor Relations Board and the courts have added
  other restrictions on the power of unions to enforce union security
  clauses and have required them to make extensive financial disclosures
  to all members as part of their duty of fair representation.}
\end{itemize}

The law outlawed closed shops which were contractual agreements that
required an employer to hire only labor union members. Union shops,
still permitted, require new recruits to join the union within a certain
amount of time. The National Labor Relations Board and the courts have
added other restrictions on the power of unions to enforce union
security clauses and have required them to make extensive financial
disclosures to all members as part of their duty of fair
representation.{[}citation needed{]} On the other hand, Congress
repealed the provisions requiring a vote by workers to authorize a union
shop a few years after the passage of the Act when it became apparent
that workers were approving them in virtually every case.{[}citation
needed{]}

\section{Union security clauses}\label{union-security-clauses}

\begin{itemize}
\item
  \emph{The amendments also authorized individual states to outlaw union
  security clauses (such as the union shop) entirely in their
  jurisdictions by passing right-to-work laws.}
\item
  \emph{A right-to-work law, under Section 14B of Taft--Hartley,
  prevents unions from negotiating contracts or legally binding
  documents requiring companies to fire workers who refuse to join the
  union.}
\end{itemize}

The amendments also authorized individual states to outlaw union
security clauses (such as the union shop) entirely in their
jurisdictions by passing right-to-work laws. A right-to-work law, under
Section 14B of Taft--Hartley, prevents unions from negotiating contracts
or legally binding documents requiring companies to fire workers who
refuse to join the union.{[}citation needed{]} Currently all of the
states in the Deep South and a number of states in the Midwest, Great
Plains, and Rocky Mountains regions have right-to-work laws (with six
states---Alabama, Arizona, Arkansas, Florida, Mississippi, and
Oklahoma---going one step further and enshrining right-to-work laws in
their states' constitutions).{[}citation needed{]}

\section{Strikes}\label{strikes}

\begin{itemize}
\item
  \emph{The Act also authorized the President to intervene in strikes or
  potential strikes that create a national emergency, a reaction to the
  national coal miners' strikes called by the United Mine Workers of
  America in the 1940s.}
\item
  \emph{The Act also prohibited federal employees from striking.}
\end{itemize}

The amendments required unions and employers to give 80 days' notice to
each other and to certain state and federal mediation bodies before they
may undertake strikes or other forms of economic action in pursuit of a
new collective bargaining agreement; it did not, on the other hand,
impose any "cooling-off period" after a contract expired. The Act also
authorized the President to intervene in strikes or potential strikes
that create a national emergency, a reaction to the national coal
miners' strikes called by the United Mine Workers of America in the
1940s. Presidents have used that power less and less frequently in each
succeeding decade. President George W. Bush invoked the law in
connection with the employer lockout of the International Longshore and
Warehouse Union during negotiations with West Coast shipping and
stevedoring companies in 2002.

The Act also prohibited federal employees from striking.

\section{Anti-communism}\label{anti-communism}

\begin{itemize}
\item
  \emph{The amendments required union leaders to file affidavits with
  the United States Department of Labor declaring that they were not
  supporters of the Communist Party and had no relationship with any
  organization seeking the "overthrow of the United States government by
  force or by any illegal or unconstitutional means" as a condition to
  participating in NLRB proceedings.}
\item
  \emph{Just over a year after Taft--Hartley passed, 81,000 union
  officers from nearly 120 unions had filed the required affidavits.}
\end{itemize}

The amendments required union leaders to file affidavits with the United
States Department of Labor declaring that they were not supporters of
the Communist Party and had no relationship with any organization
seeking the "overthrow of the United States government by force or by
any illegal or unconstitutional means" as a condition to participating
in NLRB proceedings. Just over a year after Taft--Hartley passed, 81,000
union officers from nearly 120 unions had filed the required affidavits.
In 1965, the Supreme Court held that this provision was an
unconstitutional bill of attainder.

\section{Treatment of supervisors}\label{treatment-of-supervisors}

\begin{itemize}
\item
  \emph{The amendments maintained coverage under the act for
  professional employees, but provided for special procedures before
  they may be included in the same bargaining unit as non-professional
  employees.}
\item
  \emph{The amendments expressly excluded supervisors from coverage
  under the act, and allowed employers to terminate supervisors engaging
  in union activities or those not supporting the employer's stance.}
\end{itemize}

The amendments expressly excluded supervisors from coverage under the
act, and allowed employers to terminate supervisors engaging in union
activities or those not supporting the employer's stance. The amendments
maintained coverage under the act for professional employees, but
provided for special procedures before they may be included in the same
bargaining unit as non-professional employees.

\section{Right of employer to oppose
unions}\label{right-of-employer-to-oppose-unions}

\begin{itemize}
\item
  \emph{The Act revised the Wagner Act's requirement of employer
  neutrality, to allow employers to deliver anti-union messages in the
  workplace.}
\item
  \emph{The amendments also gave employers the right to file a petition
  asking the Board to determine if a union represents a majority of its
  employees, and allow employees to petition either to decertify their
  union, or to invalidate the union security provisions of any existing
  collective bargaining agreement.}
\end{itemize}

The Act revised the Wagner Act's requirement of employer neutrality, to
allow employers to deliver anti-union messages in the workplace. These
changes confirmed an earlier Supreme Court ruling that employers have a
constitutional right to express their opposition to unions, so long as
they did not threaten employees with reprisals for their union
activities nor offer any incentives to employees as an alternative to
unionizing. The amendments also gave employers the right to file a
petition asking the Board to determine if a union represents a majority
of its employees, and allow employees to petition either to decertify
their union, or to invalidate the union security provisions of any
existing collective bargaining agreement.

\section{National Labor Relations
Board}\label{national-labor-relations-board}

\begin{itemize}
\item
  \emph{The amendments gave the General Counsel of the National Labor
  Relations Board discretionary power to seek injunctions against either
  employers or unions that violated the Act.}
\end{itemize}

The amendments gave the General Counsel of the National Labor Relations
Board discretionary power to seek injunctions against either employers
or unions that violated the Act.{[}citation needed{]} The law made
pursuit of such injunctions mandatory, rather than discretionary, in the
case of secondary boycotts by unions.{[}citation needed{]} The
amendments also established the General Counsel's autonomy within the
administrative framework of the NLRB. Congress also gave employers the
right to sue unions for damages caused by a secondary boycott, but gave
the General Counsel exclusive power to seek injunctive relief against
such activities.{[}citation needed{]}

\section{Federal jurisdiction}\label{federal-jurisdiction}

\begin{itemize}
\item
  \emph{Although Congress passed this section to empower federal courts
  to hold unions liable in damages for strikes violating a no-strike
  clause, this part of the act has instead served as the springboard for
  creation of a "federal common law" of collective bargaining
  agreements, which favored arbitration over litigation or strikes as
  the preferred means of resolving labor disputes.}
\end{itemize}

The act provided for federal court jurisdiction to enforce collective
bargaining agreements. Although Congress passed this section to empower
federal courts to hold unions liable in damages for strikes violating a
no-strike clause, this part of the act has instead served as the
springboard for creation of a "federal common law" of collective
bargaining agreements, which favored arbitration over litigation or
strikes as the preferred means of resolving labor disputes.{[}citation
needed{]}

\section{Other}\label{other}

\begin{itemize}
\item
  \emph{The Congress that passed the Taft--Hartley Amendments considered
  repealing the Norris--La Guardia Act to the extent necessary to permit
  courts to issue injunctions against strikes violating a no-strike
  clause, but chose not to do so.}
\item
  \emph{Finally, the act imposed a number of procedural and substantive
  standards that unions and employers must meet before they may use
  employer funds to provide pensions and other employee benefit to
  unionized employees.}
\end{itemize}

The Congress that passed the Taft--Hartley Amendments considered
repealing the Norris--La Guardia Act to the extent necessary to permit
courts to issue injunctions against strikes violating a no-strike
clause, but chose not to do so. The Supreme Court nonetheless held
several decades later that the act implicitly gave the courts the power
to enjoin such strikes over subjects that would be subject to final and
binding arbitration under a collective bargaining agreement.

Finally, the act imposed a number of procedural and substantive
standards that unions and employers must meet before they may use
employer funds to provide pensions and other employee benefit to
unionized employees. Congress has since passed more extensive
protections for workers and employee benefit plans as part of the
Employee Retirement Income Security Act ("ERISA").

\section{Aftermath}\label{aftermath}

\begin{itemize}
\item
  \emph{Truman won, but a union-backed effort in Ohio to defeat Taft in
  1950 failed in what one author described as "a shattering
  demonstration of labor's political weaknesses".}
\item
  \emph{Union leaders in the Congress of Industrial Organizations (CIO)
  vigorously campaigned for Truman in the 1948 election based upon a
  (never fulfilled) promise to repeal Taft--Hartley.}
\end{itemize}

Union leaders in the Congress of Industrial Organizations (CIO)
vigorously campaigned for Truman in the 1948 election based upon a
(never fulfilled) promise to repeal Taft--Hartley. Truman won, but a
union-backed effort in Ohio to defeat Taft in 1950 failed in what one
author described as "a shattering demonstration of labor's political
weaknesses". Despite his opposition to the law, Truman relied upon it in
twelve instances during his presidency.

Organized labor nearly succeeded in pushing Congress to amend the law to
increase the protections for strikers and targets of employer
retaliation during the Carter and Clinton administrations, but failed on
both occasions because of Republican opposition and lukewarm support for
these changes from the Democratic President in office at the time.

\section{See also}\label{see-also}

\begin{itemize}
\item
  \emph{Wagner Act}
\item
  \emph{Norris--La Guardia Act}
\item
  \emph{Labor unions in the United States}
\end{itemize}

Labor unions in the United States

Norris--La Guardia Act

Wagner Act

Jurisdictional strike

Secondary boycott

Chauffeurs, Teamsters, and Helpers Local No. 391 v. Terry, 494 U.S. 558
(1990) 5 to 2 on §185 of LMRA 1947, holding that a plaintiff is entitled
to trial by jury if the trade union denies representation

\section{Notes}\label{notes}

\section{Works cited}\label{works-cited}

\begin{itemize}
\item
  \emph{The Presidency of Harry S. Truman.}
\end{itemize}

Bowen, Michael (2011). The Roots of Modern Conservatism: Dewey, Taft,
and the Battle for the Soul of the Republican Party. UNC Press Books.
ISBN~9780807869192.

McCoy, Donald R. (1984). The Presidency of Harry S. Truman. University
Press of Kansas. ISBN~978-0-7006-0252-0.

\section{References}\label{references}

\begin{itemize}
\item
  \emph{From the Wagner Act to Taft-Hartley: A Study of National Labor
  Policy and Labor Relations.}
\end{itemize}

Cockburn, Alexander. "How Many Democrats Voted for Taft-Hartley?"
Counterpunch. September 6, 2004.

Faragher, J.M.; Buhle, M.J.; Czitrom, D.; and Armitage, S.H. Out of
Many: A History of the American People. Upper Saddle River, N.J.:
Pearson Prentice Hall, 2006.

McCann, Irving G. Why the Taft-Hartley Law? New York: Committee for
Constitutional Government, 1950.

Millis, Harry A. and Brown, Emily Clark. From the Wagner Act to
Taft-Hartley: A Study of National Labor Policy and Labor Relations.
Chicago: University of Chicago Press, 1950.

\section{Further reading}\label{further-reading}

\begin{itemize}
\item
  \emph{Norman: University of Oklahoma Press, 2019.}
\item
  \emph{Caballero, Raymond.}
\item
  \emph{McCarthyism vs. Clinton Jencks.}
\end{itemize}

Caballero, Raymond. McCarthyism vs. Clinton Jencks. Norman: University
of Oklahoma Press, 2019.

\section{External links}\label{external-links}

\begin{itemize}
\item
  \emph{Text of the Taft-Hartley Act}
\end{itemize}

Text of the Taft-Hartley Act

A film clip "Longines Chronoscope with Fred A Hartley" is available at
the Internet Archive
