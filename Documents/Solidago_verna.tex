\textbf{From Wikipedia, the free encyclopedia}

https://en.wikipedia.org/wiki/Solidago\_verna\\
Licensed under CC BY-SA 3.0:\\
https://en.wikipedia.org/wiki/Wikipedia:Text\_of\_Creative\_Commons\_Attribution-ShareAlike\_3.0\_Unported\_License

\section{Solidago verna}\label{solidago-verna}

\begin{itemize}
\item
  \emph{Solidago verna is a species of flowering plant in the aster
  family known by the common names springflowering goldenrod and spring
  goldenrod.}
\item
  \emph{Solidago verna occurs in several types of habitat, including
  sandhills, pine barrens, and pocosins.}
\item
  \emph{Solidago verna is a perennial herb growing up to about 1.2
  meters (4 feet) in height.}
\end{itemize}

Solidago verna is a species of flowering plant in the aster family known
by the common names springflowering goldenrod and spring goldenrod. It
is native to North Carolina and South Carolina in the United States.

Solidago verna is a perennial herb growing up to about 1.2 meters (4
feet) in height. It produces a single hairy, erect stem from a woody,
branching caudex. The serrated leaves are up to 16 centimeters (6.4
inches) long and are borne on winged petioles. The inflorescence
contains many bell-shaped flower heads. Each flower head contains 7-12
yellow ray florets surrounding 14-27 yellow disc florets. This species
is the only goldenrod in the region that blooms in spring.

Solidago verna occurs in several types of habitat, including sandhills,
pine barrens, and pocosins. The three main habitat types are pocosin
ecotones, the river terraces along the Little River, and wet pine
flatwoods.

Threats to the species include the loss of habitat to development and
agriculture, including silviculture. Fire suppression may degrade the
habitat as well.

\section{References}\label{references}

\section{External links}\label{external-links}

\begin{itemize}
\item
  \emph{United States Department of Agriculture Plants Profile}
\end{itemize}

United States Department of Agriculture Plants Profile
