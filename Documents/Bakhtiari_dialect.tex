\textbf{From Wikipedia, the free encyclopedia}

https://en.wikipedia.org/wiki/Bakhtiari\_dialect\\
Licensed under CC BY-SA 3.0:\\
https://en.wikipedia.org/wiki/Wikipedia:Text\_of\_Creative\_Commons\_Attribution-ShareAlike\_3.0\_Unported\_License

\section{Bakhtiari dialect}\label{bakhtiari-dialect}

\begin{itemize}
\item
  \emph{The Bakhtiari dialect is considered a middle Persian dialect
  which could survive through history.}
\item
  \emph{Bakhtiari dialect is a dialect of Southern Luri spoken by
  Bakhtiari people in Chaharmahal-o-Bakhtiari, Bushehr, eastern
  Khuzestan and parts of Isfahan and Lorestan provinces.}
\item
  \emph{"Luri and Bakhtiari are much more closely related to Persian,
  than Kurdish."}
\end{itemize}

Bakhtiari dialect is a dialect of Southern Luri spoken by Bakhtiari
people in Chaharmahal-o-Bakhtiari, Bushehr, eastern Khuzestan and parts
of Isfahan and Lorestan provinces. It is closely related to the
Boir-Aḥmadī, Kohgīlūya, and Mamasanī dialects in northwestern Fars.
These dialects, together with the Lori dialects of Lorestan (e.g.
Khorramabadi dialect), are referred to as the ``Perside'' southern
Zagros group, or Lori dialects.\\[2\baselineskip]"Luri and Bakhtiari are
much more closely related to Persian, than Kurdish." The Bakhtiari
dialect is considered a middle Persian dialect which could survive
through history. There do exist transitional dialects between Southern
Kurdish and Lori-Bakhtiāri', and Lori-Bakhtiāri itself may be called a
transitional idiom between Kurdish and Persian, with most of the
language originating from Persian.

\section{References}\label{references}

\begin{itemize}
\item
  \emph{Bakhtiari dialect, Encyclopædia Iranica}
\end{itemize}

Sources

Bakhtiari dialect, Encyclopædia Iranica

\section{External links}\label{external-links}

\begin{itemize}
\item
  \emph{Bakhtiari Wikipedia Beta}
\end{itemize}

Bakhtiari Wikipedia Beta
