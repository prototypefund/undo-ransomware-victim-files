\textbf{From Wikipedia, the free encyclopedia}

https://en.wikipedia.org/wiki/C.\%20T.\%20Vivian\\
Licensed under CC BY-SA 3.0:\\
https://en.wikipedia.org/wiki/Wikipedia:Text\_of\_Creative\_Commons\_Attribution-ShareAlike\_3.0\_Unported\_License

\section{C. T. Vivian}\label{c.-t.-vivian}

\begin{itemize}
\item
  \emph{Cordy Tindell Vivian, usually known as C. T. Vivian (born July
  28, 1924), is a minister, author, and was a close friend and
  lieutenant of Martin Luther King Jr. during the Civil Rights
  Movement.}
\item
  \emph{Vivian continues to reside in Atlanta, Georgia and most recently
  founded the C. T. Vivian Leadership Institute, Inc.}
\end{itemize}

Cordy Tindell Vivian, usually known as C. T. Vivian (born July 28,
1924), is a minister, author, and was a close friend and lieutenant of
Martin Luther King Jr. during the Civil Rights Movement. Vivian
continues to reside in Atlanta, Georgia and most recently founded the C.
T. Vivian Leadership Institute, Inc. He is a member of the Alpha Phi
Alpha fraternity.

President Barack Obama, speaking at the occasion of the anniversary of
Selma to Montgomery marches in March 2007 at Selma's Brown Chapel
A.M.E., recognized Vivian in his opening remarks in the words of Martin
L. King Jr. as "the greatest preacher to ever live.''

On August 8, 2013, President Barack Obama named Vivian as a recipient of
the Presidential Medal of Freedom. The citation in the press release
reads as follows:

\section{Background}\label{background}

\begin{itemize}
\item
  \emph{Studying for the ministry at American Baptist College in
  Nashville, Tennessee in 1959, Vivian met James Lawson, who was
  teaching Mohandas Gandhi's nonviolent direct action strategy to the
  Nashville Student Movement.}
\item
  \emph{Vivian was born in Boonville, Missouri.}
\item
  \emph{There, Vivian participated in his first sit-in demonstrations,
  which successfully integrated Barton's Cafeteria in 1947.}
\end{itemize}

Vivian was born in Boonville, Missouri. As a small boy he migrated with
his mother to Macomb, Illinois, where he attended Lincoln Grade School
and Edison Junior High School. Vivian graduated from Macomb High School
in 1942 and went on to attend Western Illinois University in Macomb,
where he worked as the sports editor for the school newspaper. His first
professional job was recreation director for the Carver Community Center
in Peoria, Illinois. There, Vivian participated in his first sit-in
demonstrations, which successfully integrated Barton's Cafeteria in
1947.

Studying for the ministry at American Baptist College in Nashville,
Tennessee in 1959, Vivian met James Lawson, who was teaching Mohandas
Gandhi's nonviolent direct action strategy to the Nashville Student
Movement. Soon Lawson's students, including Diane Nash, Bernard
Lafayette, James Bevel, John Lewis and others from American Baptist,
Fisk University and Tennessee State University, organized a systematic
nonviolent sit-in campaign. On April 19, 1960, 4,000 demonstrators
marched on City Hall, where Vivian and Diane Nash challenged Nashville
Mayor Ben West. As a result, Mayor West publicly agreed that racial
discrimination was morally wrong. Many of the students who participated
in the Nashville Student Movement soon took on major leadership roles in
both the Student Nonviolent Coordinating Committee (SNCC) and the
Southern Christian Leadership Conference (SCLC).

\section{Author}\label{author}

\begin{itemize}
\item
  \emph{Vivian wrote Black Power and the American Myth in 1970, a book
  about the failings of the Civil Rights Movement.}
\end{itemize}

Vivian wrote Black Power and the American Myth in 1970, a book about the
failings of the Civil Rights Movement. The book was published by
Fortress Press of Philadelphia, Pennsylvania.

The telling prophetic citing of an ongoing challenge between Christians
and Muslims judges the previous generation's negligence on working
toward peace. "Christians and Muslims can find common ground in the
necessity to create new alternatives. Anyone who starts to struggle at
any place can go all the way to achieve the changes all desire."
(p.~125.)

\section{Work with Martin Luther King and
SCLC}\label{work-with-martin-luther-king-and-sclc}

\begin{itemize}
\item
  \emph{Some claim that the St. Augustine campaign helped lead to the
  passage of the landmark Civil Rights Act of 1964, and Vivian's role in
  it was honored when he returned to the city in 2008 to dedicate a
  Freedom Trail of historic sites of the Civil Rights Movement.}
\item
  \emph{His 1970 Black Power and the American Myth was the first book on
  the Civil Rights Movement by a member of Martin Luther King's staff.}
\end{itemize}

In 1961, Vivian, now a member of the Southern Christian Leadership
Conference (SCLC) participated in Freedom Rides replacing injured
members of the Congress of Racial Equality (CORE).

He helped found the Nashville Christian Leadership Conference, and
helped organize the first sit-ins in Nashville in 1960 and the first
civil rights march in 1961. Vivian rode the first "Freedom Bus" into
Jackson, Mississippi, and went on to work alongside Martin Luther King
Jr., James Bevel, Diane Nash, and others on SCLC's Executive Staff in
Birmingham, Selma, Chicago, Nashville, the March on Washington;
Danville, Virginia, and St. Augustine, Florida. Some claim that the St.
Augustine campaign helped lead to the passage of the landmark Civil
Rights Act of 1964, and Vivian's role in it was honored when he returned
to the city in 2008 to dedicate a Freedom Trail of historic sites of the
Civil Rights Movement.

During the summer following the Selma Voting Rights Movement, Vivian
conceived and directed an educational program, Vision, and put 702
Alabama students in college with scholarships (this program later became
Upward Bound). His 1970 Black Power and the American Myth was the first
book on the Civil Rights Movement by a member of Martin Luther King's
staff.

\section{Later life}\label{later-life}

\begin{itemize}
\item
  \emph{In 2008, Vivian founded and incorporated the C. T. Vivian
  Leadership Institute, Inc. (CTVLI) to "Create a Model Leadership
  Culture in Atlanta" Georgia.}
\item
  \emph{He was featured as an activist and an analyst in the civil
  rights documentary Eyes on the Prize, and has been featured in a PBS
  special, The Healing Ministry of Dr. C. T. Vivian.}
\end{itemize}

In the 1970s Vivian moved to Atlanta, and in 1977 founded the Black
Action Strategies and Information Center (BASICS), a consultancy on
multiculturalism and race relations in the workplace and other contexts.
In 1979 he co-founded, with Anne Braden, the Center for Democratic
Renewal (initially as the National Anti-Klan Network), an organization
where blacks and whites worked together in response to white supremacist
activity. In 1984 he served in Jesse Jackson's presidential campaign, as
the national deputy director for clergy. In 1994 he helped to establish,
and served on the board of Capitol City Bank and Trust Co., a
black-owned Atlanta bank. He serves currently on the board of Every
Church a Peace Church.

Vivian continues to speak publicly and offer workshops, and has done so
at many conferences around the country and the world, including with the
United Nations. He was featured as an activist and an analyst in the
civil rights documentary Eyes on the Prize, and has been featured in a
PBS special, The Healing Ministry of Dr. C. T. Vivian. He has made
numerous appearances on Oprah as well as the Montel Williams Show and
Donahue. He is the focus of the biography Challenge and Change by Lydia
Walker.

In 2008, Vivian founded and incorporated the C. T. Vivian Leadership
Institute, Inc. (CTVLI) to "Create a Model Leadership Culture in
Atlanta" Georgia. The C. T. Vivian Leadership Institute conceived,
developed and implemented the "Yes, We Care" campaign on December 18,
2008 (four days after the City of Atlanta turned the water off at Morris
Brown College {[}MBC{]}) and, over a period of two and a half months,
mobilized the Atlanta community to donate in excess of \$500,000
directly to Morris Brown as "bridge funding." This effort literally
saved this Historically Black College or University (HBCU) which was
founded in 1881 and allowed the college to negotiate with the City which
ultimately restored the water services to the college. Additionally,
this strategic campaign gave impetus to MBC to expand and renew its
donor base.

Subsequent to the Morris Brown campaign, Vivian began discussions with
Mosaica Educational Systems which ultimately lead to a partnership with
the Atlanta Preparatory Academy (APA) an innovative charter school based
in Atlanta at the historic Jordan Hall facility.

\section{See also}\label{see-also}

\begin{itemize}
\item
  \emph{List of civil rights leaders}
\end{itemize}

List of civil rights leaders

\section{References}\label{references}

\section{Further reading}\label{further-reading}

\begin{itemize}
\item
  \emph{Includes five-minute video interview with Vivian.}
\item
  \emph{Vivian, C. T.(1924--) - Minister, civil rights activist, The
  Online Encyclopedia.}
\item
  \emph{C. T. Vivian, The Transformation of America Project.}
\end{itemize}

Pam Adams,""Changing the Nation"". Archived from the original on October
14, 2007. Retrieved September 8, 2008.CS1 maint: BOT: original-url
status unknown (link) , The Legacy Project, Peoria Journal Star, October
24, 1999 - an interview, two articles, and a timeline of his life.

C. T. Vivian, The Transformation of America Project. Includes
five-minute video interview with Vivian.

Vivian, C. T.(1924--) - Minister, civil rights activist, The Online
Encyclopedia.

\section{External links}\label{external-links}

\begin{itemize}
\item
  \emph{Vivian's oral history video excerpts at The National Visionary
  Leadership Project}
\end{itemize}

C.T. Vivian's oral history video excerpts at The National Visionary
Leadership Project

Appearances on C-SPAN
