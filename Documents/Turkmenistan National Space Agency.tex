\textbf{From Wikipedia, the free encyclopedia}

https://en.wikipedia.org/wiki/Turkmenistan\%20National\%20Space\%20Agency\\
Licensed under CC BY-SA 3.0:\\
https://en.wikipedia.org/wiki/Wikipedia:Text\_of\_Creative\_Commons\_Attribution-ShareAlike\_3.0\_Unported\_License

\section{Turkmenistan National Space
Agency}\label{turkmenistan-national-space-agency}

\begin{itemize}
\item
  \emph{Turkmenistan National Space Agency (TNSA; Turkmen:
  Türkmenistanyň prezidentiň ýanynda Milli kosmos agentligi), is a
  governmental body that coordinates all Turkmenistan space research
  programs with scientific and commercial goals.}
\end{itemize}

Turkmenistan National Space Agency (TNSA; Turkmen: Türkmenistanyň
prezidentiň ýanynda Milli kosmos agentligi), is a governmental body that
coordinates all Turkmenistan space research programs with scientific and
commercial goals. It was established in 2011.

\section{Space programme}\label{space-programme}

\begin{itemize}
\item
  \emph{In 2011, Turkmenistan space industry boosted as new agency set
  up under the state program for development of space industry after
  President Gurbanguly Berdimuhamedow's approval.}
\end{itemize}

In 2011, Turkmenistan space industry boosted as new agency set up under
the state program for development of space industry after President
Gurbanguly Berdimuhamedow's approval.

\section{Satellites}\label{satellites}

\begin{itemize}
\item
  \emph{Turkmenistan contracted the Thales Alenia Space group to
  manufacture its first satellite launched into geosynchronous orbit in
  March 2015.}
\end{itemize}

Turkmenistan contracted the Thales Alenia Space group to manufacture its
first satellite launched into geosynchronous orbit in March 2015.

\section{TurkmenAlem52E/MonacoSAT}\label{turkmenalem52emonacosat}

\begin{itemize}
\item
  \emph{TurkmenAlem52E/MonacoSAT is a geosynchronous orbit (GSO)
  communication satellite initially intended to be Turkmenistan's first
  satellite.}
\item
  \emph{The satellite's operations will be controlled by a state-run
  Turkmenistan National Space Agency.}
\end{itemize}

TurkmenAlem52E/MonacoSAT is a geosynchronous orbit (GSO) communication
satellite initially intended to be Turkmenistan's first satellite.\\
Built by Thales Alenia Space, and contracted to launch aboard a SpaceX
Falcon 9 launch vehicle in March 2015, the satellite has an anticipated
service life of 15 years.\\
The satellite will cover Europe and significant part of Asian countries
and Africa and will have transmission for TV, radio broadcasting and the
internet.{[}needs update{]}

The satellite's operations will be controlled by a state-run
Turkmenistan National Space Agency.

\section{See also}\label{see-also}

\begin{itemize}
\item
  \emph{List of government space agencies}
\end{itemize}

List of government space agencies

\section{References}\label{references}
