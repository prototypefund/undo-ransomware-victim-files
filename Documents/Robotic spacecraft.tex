\textbf{From Wikipedia, the free encyclopedia}

https://en.wikipedia.org/wiki/Robotic\%20spacecraft\\
Licensed under CC BY-SA 3.0:\\
https://en.wikipedia.org/wiki/Wikipedia:Text\_of\_Creative\_Commons\_Attribution-ShareAlike\_3.0\_Unported\_License

\includegraphics[width=5.50000in,height=4.33754in]{media/image1.jpg}\\
\emph{An artist's interpretation of the MESSENGER spacecraft at Mercury}

\section{Robotic spacecraft}\label{robotic-spacecraft}

\begin{itemize}
\item
  \emph{A robotic spacecraft designed to make scientific research
  measurements is often called a space probe.}
\item
  \emph{Many artificial satellites are robotic spacecraft, as are many
  landers and rovers.}
\item
  \emph{A robotic spacecraft is an uncrewed spacecraft, usually under
  telerobotic control.}
\end{itemize}

A robotic spacecraft is an uncrewed spacecraft, usually under
telerobotic control. A robotic spacecraft designed to make scientific
research measurements is often called a space probe. Many space missions
are more suited to telerobotic rather than crewed operation, due to
lower cost and lower risk factors. In addition, some planetary
destinations such as Venus or the vicinity of Jupiter are too hostile
for human survival, given current technology. Outer planets such as
Saturn, Uranus, and Neptune are too distant to reach with current crewed
spacecraft technology, so telerobotic probes are the only way to explore
them.

Many artificial satellites are robotic spacecraft, as are many landers
and rovers.

\includegraphics[width=5.50000in,height=4.50820in]{media/image2.jpg}\\
\emph{A replica of Sputnik 1 at the U.S. National Air and Space Museum}

\includegraphics[width=5.50000in,height=3.52344in]{media/image3.jpg}\\
\emph{A replica of Explorer 1}

\section{History}\label{history}

\begin{itemize}
\item
  \emph{The first robotic spacecraft was launched by the Soviet Union
  (USSR) on 22 July 1951, a suborbital flight carrying two dogs Dezik
  and Tsygan.}
\item
  \emph{The first artificial satellite, Sputnik 1, was put into a
  215-by-939-kilometer (116 by 507~nmi) Earth orbit by the USSR) on 4
  October 1957.}
\item
  \emph{Since the satellite was not designed to detach from its launch
  vehicle's upper stage, the total mass in orbit was 508.3 kilograms
  (1,121~lb).}
\end{itemize}

The first robotic spacecraft was launched by the Soviet Union (USSR) on
22 July 1951, a suborbital flight carrying two dogs Dezik and Tsygan.
Four other such flights were made through the fall of 1951.

The first artificial satellite, Sputnik 1, was put into a
215-by-939-kilometer (116 by 507~nmi) Earth orbit by the USSR) on 4
October 1957. On 3 November 1957, the USSR orbited Sputnik 2. Weighing
113 kilograms (249~lb), Sputnik 2 carried the first living animal into
orbit, the dog Laika. Since the satellite was not designed to detach
from its launch vehicle's upper stage, the total mass in orbit was 508.3
kilograms (1,121~lb).

In a close race with the Soviets, the United States launched its first
artificial satellite, Explorer 1, into a 193-by-1,373-nautical-mile (357
by 2,543~km) orbit on 31 January 1958. Explorer I was a 80.75-inch
(205.1~cm) long by 6.00-inch (15.2~cm) diameter cylinder weighing 30.8
pounds (14.0~kg), compared to Sputnik 1, a 58-centimeter (23~in) sphere
which weighed 83.6 kilograms (184~lb). Explorer 1 carried sensors which
confirmed the existence of the Van Allen belts, a major scientific
discovery at the time, while Sputnik 1 carried no scientific sensors. On
17 March 1958, the US orbited its second satellite, Vanguard 1, which
was about the size of a grapefruit, and remains in a
360-by-2,080-nautical-mile (670 by 3,850~km) orbit as of
2016{[}update{]}.

Nine other countries have successfully launched satellites using their
own launch vehicles: France (1965), Japan and China (1970), the United
Kingdom (1971), India (1980), Israel (1988), Iran (2009), North Korea
(2012), and New Zealand (2018).

\section{Design}\label{design}

\begin{itemize}
\item
  \emph{In spacecraft design, the United States Air Force considers a
  vehicle to consist of the mission payload and the bus (or platform).}
\item
  \emph{JPL divides the "flight system" of a spacecraft into
  subsystems.}
\end{itemize}

In spacecraft design, the United States Air Force considers a vehicle to
consist of the mission payload and the bus (or platform). The bus
provides physical structure, thermal control, electrical power, attitude
control and telemetry, tracking and commanding.

JPL divides the "flight system" of a spacecraft into subsystems. These
include:

\includegraphics[width=5.50000in,height=3.09375in]{media/image4.jpg}\\
\emph{An illustration of NASA's planned Orion spacecraft approaching a
robotic asteroid capture vehicle}

\section{Structure}\label{structure}

\begin{itemize}
\item
  \emph{ensures spacecraft components are supported and can withstand
  launch loads}
\item
  \emph{provides overall mechanical integrity of the spacecraft}
\end{itemize}

This is the physical backbone structure. It:

provides overall mechanical integrity of the spacecraft

ensures spacecraft components are supported and can withstand launch
loads

\section{Data handling}\label{data-handling}

\begin{itemize}
\item
  \emph{collecting and reporting spacecraft telemetry data (e.g.}
\item
  \emph{maintaining the spacecraft clock}
\item
  \emph{spacecraft health)}
\end{itemize}

This is sometimes referred to as the command and data subsystem. It is
often responsible for:

command sequence storage

maintaining the spacecraft clock

collecting and reporting spacecraft telemetry data (e.g. spacecraft
health)

collecting and reporting mission data (e.g. photographic images)

\section{Attitude determination and
control}\label{attitude-determination-and-control}

\begin{itemize}
\item
  \emph{This system is mainly responsible for the correct spacecraft's
  orientation in space (attitude) despite external disturbance-gravity
  gradient effects, magnetic-field torques, solar radiation and
  aerodynamic drag; in addition it may be required to reposition movable
  parts, such as antennas and solar arrays.}
\end{itemize}

This system is mainly responsible for the correct spacecraft's
orientation in space (attitude) despite external disturbance-gravity
gradient effects, magnetic-field torques, solar radiation and
aerodynamic drag; in addition it may be required to reposition movable
parts, such as antennas and solar arrays.

\section{Landing on hazardous
terrain}\label{landing-on-hazardous-terrain}

\begin{itemize}
\item
  \emph{The robotic spacecraft must also efficiently perform hazard
  assessment and trajectory adjustments in real time to avoid hazards.}
\item
  \emph{To achieve this, the robotic spacecraft requires accurate
  knowledge of where the spacecraft is located relative to the surface
  (localization), what may pose as hazards from the terrain (hazard
  assessment), and where the spacecraft should presently be headed
  (hazard avoidance).}
\end{itemize}

In planetary exploration missions involving robotic spacecraft, there
are three key parts in the processes of landing on the surface of the
planet to ensure a safe and successful landing. This process includes a
entry into the planetary gravity field and atmosphere, a descent through
that atmosphere towards a intended/targeted region of scientific value,
and a safe landing that guarantees the integrity of the instrumentation
on the craft is preserved. While the robotic spacecraft is going through
those parts, it must also be capable of estimating its position compared
to the surface in order to ensure reliable control of itself and its
ability to maneuver well. The robotic spacecraft must also efficiently
perform hazard assessment and trajectory adjustments in real time to
avoid hazards. To achieve this, the robotic spacecraft requires accurate
knowledge of where the spacecraft is located relative to the surface
(localization), what may pose as hazards from the terrain (hazard
assessment), and where the spacecraft should presently be headed (hazard
avoidance). Without the capability for operations for localization,
hazard assessment, and avoidance, the robotic spacecraft becomes unsafe
and can easily enter dangerous situations such as surface collisions,
undesirable fuel consumption levels, and/or unsafe maneuvers.

\section{Entry, descent, and landing}\label{entry-descent-and-landing}

\begin{itemize}
\item
  \emph{Integrated sensing incorporates an image transformation
  algorithm to interpret the immediate imagery land data, perform a
  real-time detection and avoidance of terrain hazards that may impede
  safe landing, and increase the accuracy of landing at a desired site
  of interest using landmark localization techniques.}
\item
  \emph{The cameras are also used to detect any possible hazards whether
  it is increased fuel consumption or it is a physical hazard such as a
  poor landing spot in a crater or cliff side that would make landing
  very not ideal (hazard assessment).}
\end{itemize}

Integrated sensing incorporates an image transformation algorithm to
interpret the immediate imagery land data, perform a real-time detection
and avoidance of terrain hazards that may impede safe landing, and
increase the accuracy of landing at a desired site of interest using
landmark localization techniques. Integrated sensing completes these
tasks by relying on pre-recorded information and cameras to understand
its location and determine its position and whether it is correct or
needs to make any corrections (localization). The cameras are also used
to detect any possible hazards whether it is increased fuel consumption
or it is a physical hazard such as a poor landing spot in a crater or
cliff side that would make landing very not ideal (hazard assessment).

\section{Telecommunications}\label{telecommunications}

\begin{itemize}
\item
  \emph{These may be used to communicate with ground stations on Earth,
  or with other spacecraft.}
\end{itemize}

Components in the telecommunications subsystem include radio antennas,
transmitters and receivers. These may be used to communicate with ground
stations on Earth, or with other spacecraft.

\section{Electrical power}\label{electrical-power}

\begin{itemize}
\item
  \emph{The supply of electric power on spacecraft generally come from
  photovoltaic (solar) cells or from a radioisotope thermoelectric
  generator.}
\end{itemize}

The supply of electric power on spacecraft generally come from
photovoltaic (solar) cells or from a radioisotope thermoelectric
generator. Other components of the subsystem include batteries for
storing power and distribution circuitry that connects components to the
power sources.

\section{Temperature control and protection from the
environment}\label{temperature-control-and-protection-from-the-environment}

\begin{itemize}
\item
  \emph{Spacecraft are often protected from temperature fluctuations
  with insulation.}
\item
  \emph{Some spacecraft use mirrors and sunshades for additional
  protection from solar heating.}
\end{itemize}

Spacecraft are often protected from temperature fluctuations with
insulation. Some spacecraft use mirrors and sunshades for additional
protection from solar heating. They also often need shielding from
micrometeoroids and orbital debris.

\section{Propulsion}\label{propulsion}

\begin{itemize}
\item
  \emph{But, most spacecraft propulsion today is based on rocket
  engines.}
\item
  \emph{Spacecraft propulsion is a method that allows a spacecraft to
  travel through space by generating thrust to push it forward.}
\end{itemize}

Spacecraft propulsion is a method that allows a spacecraft to travel
through space by generating thrust to push it forward. However, there
isn't one universally used propulsion system: monopropellant,
bipropellant, ion propulsion, and etc. Each propulsion system generates
thrust in slightly different ways with each system having its own
advantages and disadvantages. But, most spacecraft propulsion today is
based on rocket engines. The general idea behind rocket engines is that
when an oxidizer meets the fuel source, there is explosive release of
energy and heat at high speeds, which propels the spacecraft forward.
This happens due to one basic principle known as Newton's Third Law.
According to Newton, ``to every action there is an equal and opposite
reaction.'' As the energy and heat is being released from the back of
the spacecraft, gas particles are being pushed around to allow the
spacecraft to propel forward. The main reason behind the usage of rocket
engine today is because rockets are the most powerful form of propulsion
there is.

\section{Monopropellant}\label{monopropellant}

\begin{itemize}
\item
  \emph{This way, the spacecraft propulsion is controlled.}
\end{itemize}

For a propulsion system to work, there is usually always an oxidizer
line and a fuel line. This way, the spacecraft propulsion is controlled.
But in a monopropellant propulsion, there is no need for an oxidizer
line and only requires the fuel line. This works due to the oxidizer
being chemically bonded into the fuel molecule itself. But for the
propulsion system to be controlled, the combustion of the fuel can only
occur due to a presence of a catalyst. This is quite advantageous due to
making the rocket engine lighter and cheaper, easy to control, and more
reliable. But, the downfall is that the chemical is very dangerous to
manufacture, store, and transport.

\section{Bipropellant}\label{bipropellant}

\begin{itemize}
\item
  \emph{The downside is the same as that of monopropellant propulsion
  system: very dangerous to manufacture, store, and transport.}
\item
  \emph{A bipropellant propulsion system is a rocket engine that uses a
  liquid propellent.}
\item
  \emph{The main benefit for having this technology is because that
  these kinds of liquids have relatively high density, which allows the
  volume of the propellent tank to be small, therefore increasing space
  efficacy.}
\end{itemize}

A bipropellant propulsion system is a rocket engine that uses a liquid
propellent. This means both the oxidizer and fuel line are in liquid
states. This system is unique because it requires no ignition system,
the two liquids would spontaneously combust as soon as they come into
contact with each other and produces the propulsion to push the ship
forward. The main benefit for having this technology is because that
these kinds of liquids have relatively high density, which allows the
volume of the propellent tank to be small, therefore increasing space
efficacy. The downside is the same as that of monopropellant propulsion
system: very dangerous to manufacture, store, and transport.

\section{Ion}\label{ion}

\begin{itemize}
\item
  \emph{An ion propulsion system is a type of engine that generates
  thrust by the means of electron bombardment or the acceleration of
  ions.}
\item
  \emph{The momentum of these positively charged ions provides the
  thrust to propel the spacecraft forward.}
\end{itemize}

An ion propulsion system is a type of engine that generates thrust by
the means of electron bombardment or the acceleration of ions. By
shooting high-energy electrons to a propellant atom (neutrally charge),
it removes electrons from the propellant atom and this results the
propellant atom becoming a positively charged atom. The positively
charged ions are guided to pass through positively charged grids that
contains thousands of precise aligned holes are running at high
voltages. Then, the aligned positively charged ions accelerates through
a negative charged accelerator grid that further increases the speed of
the ions up to 90,000~mph. The momentum of these positively charged ions
provides the thrust to propel the spacecraft forward. The advantage of
having this kind of propulsion is that it is incredibly efficient in
maintaining constant velocity, which is needed for deep-space travel.
However, the amount of thrust produced is extremely low and that it
needs a lot of electrical power to operate.

\section{Mechanical devices}\label{mechanical-devices}

\begin{itemize}
\item
  \emph{Mechanical components often need to be moved for deployment
  after launch or prior to landing.}
\end{itemize}

Mechanical components often need to be moved for deployment after launch
or prior to landing. In addition to the use of motors, many one-time
movements are controlled by pyrotechnic devices.

\section{Robotic vs. uncrewed
spacecraft}\label{robotic-vs.-uncrewed-spacecraft}

\begin{itemize}
\item
  \emph{Robotic spacecraft are specifically designed system for a
  specific hostile environment.}
\item
  \emph{The term 'uncrewed spacecraft' does not imply that the
  spacecraft is robotic.}
\end{itemize}

Robotic spacecraft are specifically designed system for a specific
hostile environment. Due to their specification for a particular
environment, it varies greatly in complexity and capabilities. While an
uncrewed spacecraft is a spacecraft without personnel or crew and is
operated by automatic (proceeds with an action without human
intervention) or remote control (with human intervention). The term
'uncrewed spacecraft' does not imply that the spacecraft is robotic.

\section{Control}\label{control}

\begin{itemize}
\item
  \emph{Soon after these first spacecraft, command systems were
  developed to allow remote control from the ground.}
\item
  \emph{Although generally referred to as "remotely controlled" or
  "telerobotic", the earliest orbital spacecraft -- such as Sputnik 1
  and Explorer 1 -- did not receive control signals from Earth.}
\item
  \emph{Robotic spacecraft use telemetry to radio back to Earth acquired
  data and vehicle status information.}
\end{itemize}

Robotic spacecraft use telemetry to radio back to Earth acquired data
and vehicle status information. Although generally referred to as
"remotely controlled" or "telerobotic", the earliest orbital spacecraft
-- such as Sputnik 1 and Explorer 1 -- did not receive control signals
from Earth. Soon after these first spacecraft, command systems were
developed to allow remote control from the ground. Increased autonomy is
important for distant probes where the light travel time prevents rapid
decision and control from Earth. Newer probes such as Cassini--Huygens
and the Mars Exploration Rovers are highly autonomous and use on-board
computers to operate independently for extended periods of time.

\section{Space probes}\label{space-probes}

\begin{itemize}
\item
  \emph{A space probe is a robotic spacecraft that does not orbit Earth,
  but instead, explores further into outer space.}
\item
  \emph{{[}1{]} A space probe may approach the Moon; travel through
  interplanetary space; flyby, orbit, or land on other planetary bodies;
  or enter interstellar space.}
\end{itemize}

A space probe is a robotic spacecraft that does not orbit Earth, but
instead, explores further into outer space.{[}1{]} A space probe may
approach the Moon; travel through interplanetary space; flyby, orbit, or
land on other planetary bodies; or enter interstellar space.

\includegraphics[width=5.50000in,height=5.50000in]{media/image5.jpg}\\
\emph{Departure shot of Pluto by New Horizons, showing Pluto's
atmosphere backlit by the Sun.}

\section{SpaceX's Dragon}\label{spacexs-dragon}

\begin{itemize}
\item
  \emph{A space probe is a scientific space exploration mission in which
  a spacecraft leaves Earth and explores space.}
\end{itemize}

An example of a fully robotic spacecraft in the modern world would be
SpaceX's Dragon. The SpaceX Dragon is a robotic spacecraft designed to
send not only cargo to Earth's orbit, but also humans as well. The
SpaceX Dragon's total height is 7.2~m (23.6~ft) with a diameter of 3.7~m
(12~ft). The total launch payload mass is 6,000~kg (13,228~lbs) and a
total return mass of 3,000~kg (6,614~lbs), along with a total launch
payload volume of 25m\^{}3 (883~ft\^{}3) and a total return payload
volume of 11m\^{}3 (388~ft\^{}3). The total duration of the Dragon in
Earth's orbit is two years.

In 2012 the SpaceX Dragon made history by becoming the first commercial
robotic spacecraft to deliver cargo to the International Space Station
and to safely return cargo to Earth in the same trip. This feat that the
Dragon made was only achieved previously by governments. Currently the
Dragon is meant to transfer cargo because of its capability of returning
significant amounts of cargo to Earth despite it originally being
designed to carry humans.

A space probe is a scientific space exploration mission in which a
spacecraft leaves Earth and explores space. It may approach the Moon,
enter interplanetary, flyby or orbit other bodies, or approach
interstellar space.

\includegraphics[width=5.50000in,height=5.41339in]{media/image6.jpg}\\
\emph{AERCam Sprint released from the Space Shuttle Columbia payload
bay}

\section{Robotic spacecraft service
vehicles}\label{robotic-spacecraft-service-vehicles}

\begin{itemize}
\item
  \emph{MDA Space Infrastructure Servicing vehicle --- an in-space
  refueling depot and service spacecraft for communication satellites in
  geosynchronous orbit.}
\end{itemize}

MDA Space Infrastructure Servicing vehicle --- an in-space refueling
depot and service spacecraft for communication satellites in
geosynchronous orbit. Launch planned for 2015.{[}needs update{]}

Mission Extension Vehicle is an alternative approach that does not
utilize in-space RCS fuel transfer. Rather, it would connect to the
target satellite in the same way as MDA SIS, and then use "its own
thrusters to supply attitude control for the target."

\section{See also}\label{see-also}

\begin{itemize}
\item
  \emph{Automated cargo spacecraft}
\item
  \emph{Timeline of Solar System exploration}
\item
  \emph{Space observatory}
\end{itemize}

Astrobotic Technology

Geosynchronous satellite

Human spaceflight

Space observatory

Timeline of Solar System exploration

Automated cargo spacecraft

\section{References}\label{references}

\section{External links}\label{external-links}

\begin{itemize}
\item
  \emph{Russia's unmanned Moon missions}
\item
  \emph{NASA Jet Propulsion Laboratory}
\end{itemize}

NASA Jet Propulsion Laboratory

Russia's unmanned Moon missions

NASA Home Page
