\textbf{From Wikipedia, the free encyclopedia}

https://en.wikipedia.org/wiki/Student\%20Nonviolent\%20Coordinating\%20Committee\\
Licensed under CC BY-SA 3.0:\\
https://en.wikipedia.org/wiki/Wikipedia:Text\_of\_Creative\_Commons\_Attribution-ShareAlike\_3.0\_Unported\_License

\section{Student Nonviolent Coordinating
Committee}\label{student-nonviolent-coordinating-committee}

\begin{itemize}
\item
  \emph{The Student Nonviolent Coordinating Committee (SNCC, often
  pronounced /snɪk/ SNIK) was one of the major American Civil Rights
  Movement organizations of the 1960s.}
\item
  \emph{In the later 1960s, inspired by fiery leaders such as Stokely
  Carmichael, SNCC focused on black power, and draft resistance to the
  Vietnam War.}
\end{itemize}

The Student Nonviolent Coordinating Committee (SNCC, often pronounced
/snɪk/ SNIK) was one of the major American Civil Rights Movement
organizations of the 1960s. It emerged from the first wave of student
sit-ins and formed at a May 1960 meeting organized by Ella Baker at Shaw
University. After its involvement in the Voter Education Project, SNCC
grew into a large organization with many supporters in the North who
helped raise funds to support its work in the South, allowing full-time
organizers to have a small salary. Many unpaid grassroots organizers and
activists also worked with SNCC on projects in the Deep South, often
becoming targets of racial violence and police brutality. SNCC played a
seminal role in the freedom rides, the 1963 March on Washington,
Mississippi Freedom Summer, the Selma campaigns, the March Against Fear
and other historic events. SNCC may be best known for its community
organizing, including voter registration, freedom schools, and localized
direct action all over the country, but especially in Georgia, Alabama,
and Mississippi.

In the later 1960s, inspired by fiery leaders such as Stokely
Carmichael, SNCC focused on black power, and draft resistance to the
Vietnam War. As early as 1965, executive secretary James Forman said he
"did not know how much longer we can stay nonviolent" and in 1969, SNCC
officially changed its name to the Student National Coordinating
Committee to reflect the broadening of its strategy. It passed out of
existence in the 1970s following heavy infiltration and suppression by
the Federal Bureau of Investigation (FBI), spearheaded as part of
COINTELPRO operations during the 1960s and 70s led by J. Edgar Hoover.

\section{Founding and early years}\label{founding-and-early-years}

\begin{itemize}
\item
  \emph{Out of this conference the SNCC was formed.}
\item
  \emph{Instead of being closely tied to SCLC or the NAACP as a "youth
  division", SNCC sought to stand on its own.}
\item
  \emph{But SNCC was not a branch of SCLC.}
\item
  \emph{SNCC's first chairman was Marion Barry, who later became the
  mayor of Washington DC.}
\item
  \emph{SNCC's executive secretary, James Forman, played a major role in
  running the organization.}
\end{itemize}

Founded in 1960 and inspired by the Greensboro sit-ins and Nashville
sit-ins, independent student-led groups began direct-action protests
against segregation in dozens of southern communities. SNCC focused on
mobilizing local communities,\\
a policy in which African American communities would push for change,\\
impelling the federal government to act once the injustice had become\\
apparent. The most common action of these groups was organizing sit-ins
at racially segregated lunch counters to protest the pervasiveness of
Jim Crow and other forms of racism. While in the Civil Rights Cases (109
U.S. 3 {[}1883{]}), the Court ruled that the equal protection clause
"did not cover private individuals, organizations, or establishments,"
the trials of arrested sit-in protesters created an opening for the
Court to reevaluate its earlier ruling and expand the clause to cover
acts of private discrimination. The sit-in movement was a turning point
in using the courts and jail to exert moral and economic pressure on
southern communities. In addition to sitting in at lunch counters, the
groups also organized and carried out protests at segregated White
public libraries, public parks, public swimming pools, and movie
theaters. At that time, all those facilities financed by taxes were
closed to blacks. The white response was often to close the facility,
rather than integrate it.

The Student Nonviolent Coordinating Committee (SNCC), as an
organization, began with an \$800.00 grant from the Southern Christian
Leadership Conference (SCLC) for a conference attended by 126 student
delegates from 58 sit-in centers in 12 states, along with delegates from
19 northern colleges, the SCLC, Congress of Racial Equality (CORE),
Fellowship of Reconciliation (FOR), National Student Association (NSA),
and Students for a Democratic Society (SDS). Out of this conference the
SNCC was formed.

Ella Baker, who organized the Shaw conference, was the SCLC director at
the time she helped form SNCC. But SNCC was not a branch of SCLC.
Instead of being closely tied to SCLC or the NAACP as a "youth
division", SNCC sought to stand on its own. Ms. Baker later lost her job
with SCLC, which she had helped found.

Among important SNCC leaders attending the conference were Stokely
Carmichael from Howard University; Charles F. McDew, who led student
protests at South Carolina State University; J. Charles Jones, who
organized 200 students to participate in sit-ins at department stores
throughout Charlotte, North Carolina; Julian Bond from Atlanta, Diane
Nash from Fisk University; James Lawson; and John Lewis, Bernard
Lafayette, James Bevel, and Marion Barry from the Nashville Student
Movement.

SNCC's first chairman was Marion Barry, who later became the mayor of
Washington DC. Barry served as chairman for one year. The second
chairman was Charles F. McDew, who served as the chairman from 1961 to
1963, when he was succeeded by John Lewis. Stokely Carmichael and H.
"Rap" Brown were chairmen in the late 1960s. SNCC's executive secretary,
James Forman, played a major role in running the organization.

\section{Freedom Riders}\label{freedom-riders}

\begin{itemize}
\item
  \emph{SNCC took on greater risks in 1961, after a mob of Ku Klux Klan
  members and other whites attacked integrated groups of bus passengers
  who defied local segregation laws as part of the Freedom Rides
  organized by the Congress of Racial Equality (CORE).}
\item
  \emph{In the years that followed, SNCC members were referred to as
  "shock troops of the revolution."}
\end{itemize}

In the years that followed, SNCC members were referred to as "shock
troops of the revolution." SNCC took on greater risks in 1961, after a
mob of Ku Klux Klan members and other whites attacked integrated groups
of bus passengers who defied local segregation laws as part of the
Freedom Rides organized by the Congress of Racial Equality (CORE).
Rather than allowing mob violence to stop them, New Orleans CORE and
Nashville SNCC Freedom Riders, including Dave Dennis, Oretha Castle
Haley, Jean C. Thompson, Rudy Lombard, Diane Nash, James Bevel, Marion
Barry, Angeline Butler, and John Lewis, put themselves at great personal
risk by traveling in racially integrated groups into Mississippi as they
continued the Ride. Other bus riders followed, traveling through the
deep South to test Southern compliance with Federal Law. At least 436
people took part in these Freedom Rides during the spring and summer of
1961.

\section{Voter registration}\label{voter-registration}

\begin{itemize}
\item
  \emph{The voter registration project he initiated in McComb,
  Mississippi in 1961 became the seed for much of SNCC's activities from
  1962 to 1966.}
\item
  \emph{SNCC workers lived with local families: often the homes
  providing such hospitality were firebombed.}
\item
  \emph{After the Freedom Rides, SNCC worked primarily on voter
  registration, and with local protests over segregated public
  facilities.}
\end{itemize}

Bob Moses played a central role in transforming SNCC from a coordinating
committee of student protest groups to an organization of activists
dedicated to building community-based political organizations of the
rural poor. The voter registration project he initiated in McComb,
Mississippi in 1961 became the seed for much of SNCC's activities from
1962 to 1966.

After the Freedom Rides, SNCC worked primarily on voter registration,
and with local protests over segregated public facilities. Registering
Black voters was extremely difficult and dangerous. People of Color who
attempted to register often lost their jobs and their homes, and
sometimes their lives. SNCC workers lived with local families: often the
homes providing such hospitality were firebombed.

The actions of SNCC, CORE, and SCLC forced the Kennedy Administration to
briefly provide federal protection to temporarily abate mob violence.
Local FBI offices were usually staffed by Southern whites (there were no
Black FBI agents at that time) who refused to intervene to protect civil
rights workers or local Blacks who were attempting to register to vote.

\section{Participatory democracy (group centered
leadership)}\label{participatory-democracy-group-centered-leadership}

\begin{itemize}
\item
  \emph{SNCC was unusual among civil rights groups in the way in which
  decisions were made.}
\item
  \emph{Because activities were often very dangerous and could lead to
  prison or death, SNCC wanted all participants to support each
  activity.}
\item
  \emph{Instead of "top down" control, as was the case with most
  organizations at that time, decisions in SNCC were made by consensus,
  called participatory democracy.}
\end{itemize}

SNCC was unusual among civil rights groups in the way in which decisions
were made. Instead of "top down" control, as was the case with most
organizations at that time, decisions in SNCC were made by consensus,
called participatory democracy. Ms. Ella Baker was extremely influential
in establishing that model, as was Rev. James Lawson. Group meetings
would be convened in which every participant could speak for as long as
they wanted and the meeting would continue until everyone who was left
was in agreement with the decision. Because activities were often very
dangerous and could lead to prison or death, SNCC wanted all
participants to support each activity.

\section{March on Washington}\label{march-on-washington}

\begin{itemize}
\item
  \emph{Forman's and SNCC's anger came in part from the failure of the
  federal government, FBI, and Justice Department to protect SNCC civil
  rights workers in the South at this time.}
\item
  \emph{SNCC played a significant role in the 1963 March on Washington
  for Jobs and Freedom.}
\item
  \emph{Indeed, the federal government at that time was instrumental in
  indicting SNCC workers and other civil rights activists.}
\end{itemize}

SNCC played a significant role in the 1963 March on Washington for Jobs
and Freedom. While many speakers applauded the Kennedy Administration
for the efforts it had made toward obtaining new, more effective civil
rights legislation protecting the right to vote and outlawing
segregation, John Lewis took the administration to task for how little
it had done to protect Southern blacks and civil rights workers under
attack in the Deep South. Although he was forced to tone down his speech
under pressure from the representatives of other civil rights
organizations on the march organization committee, his words still
stung. The version of the speech leaked to the press went as follows:

However, under pressure from the representatives of other groups many
changes were made to the speech as it was delivered that day. According
to James Forman, the most important of these was the change of "we
cannot support" the Kennedy Civil Rights Bill to "we support with
reservations". Forman wrote of the following explanation of this:

Forman's and SNCC's anger came in part from the failure of the federal
government, FBI, and Justice Department to protect SNCC civil rights
workers in the South at this time. Indeed, the federal government at
that time was instrumental in indicting SNCC workers and other civil
rights activists.

\section{Voting rights}\label{voting-rights}

\begin{itemize}
\item
  \emph{SNCC followed up on the Freedom Ballot with the Mississippi
  Summer Project, also known as Freedom Summer, which focused on voter
  registration and Freedom Schools.}
\item
  \emph{SNCC also established Freedom Schools to teach children to read
  and to educate them to stand up for their rights.}
\item
  \emph{It also estranged SNCC leaders from many of the mainstream
  leaders of the civil rights movement.}
\end{itemize}

In 1961 SNCC began expanding its activities from direct-action protests
against segregation into other forms of organizing, most notably voter
registration. Under the leadership of Bob Moses, SNCC's first
voter-registration project was in McComb, Mississippi, an effort
suppressed with arrests and savage white violence, resulting in the
murder of local activist Herbert Lee.

With funding from the Voter Education Project, SNCC expanded its voter
registration efforts into the Mississippi Delta around Greenwood,
Southwest Georgia around Albany, and the Alabama Black Belt around
Selma. All of these projects endured police harassment and arrests; KKK
violence including shootings, bombings, and assassinations; and economic
terrorism against those blacks who dared to try to register.

In 1962 Bob Moses worked to forge a coalition of national and regional
organizations, including the NAACP and the National Council of Churches,
that would fund and promote SNCC's voter registration work in
Mississippi. This coalition was known as the Council of Federated
Organizations. In the fall of 1963, SNCC conducted the Freedom Ballot, a
parallel election in which black Mississippians came out to show their
willingness to vote --- a right they had been denied for decades,
despite the provisions of the Fifteenth Amendment to the United States
Constitution, due to a combination of state laws and constitutional
provisions, economic reprisals and violence by white authorities and
private citizens.

SNCC followed up on the Freedom Ballot with the Mississippi Summer
Project, also known as Freedom Summer, which focused on voter
registration and Freedom Schools. The Summer Project brought hundreds of
white Northern students to the South, where they volunteered as teachers
and organizers. Their presence brought national press attention to
SNCC's work in the south. SNCC organized black Mississippians to
register to vote, almost always without success. White authorities
either rejected their applications on any pretexts available or, failing
that, simply refused to accept their applications.Tensions grew
gradually and SNCC refused to recruit white people because they thought
that they brought attention of the media only on white people.

Mississippi Summer received national attention when three civil rights
workers involved in the project - James Chaney, Andrew Goodman and
Michael Schwerner - were murdered after having been released from police
custody. Their bodies were eventually found after a reluctant J. Edgar
Hoover directed the FBI to search for them. Johnson only sent the FBI
after series of pressure and demonstrations. He favored at first to be
the leader outside of his country with the Vietnam War whereas there
were many conflicts inside of it. In the process, the FBI also found
corpses of several other missing black Mississippians, whose
disappearances had not attracted public attention outside the Delta.

SNCC also established Freedom Schools to teach children to read and to
educate them to stand up for their rights. As in the struggle to
desegregate public accommodations led by Martin Luther King Jr. and
James Bevel in Birmingham, Alabama the year before, the bolder attitudes
of the children helped shake their parents out of the fear that had
paralyzed many of them.

The goal of the Mississippi Summer Project was to organize the
Mississippi Freedom Democratic Party (MFDP), an integrated party, to win
seats at the 1964 Democratic National Convention for a slate of
delegates elected by disfranchised black Mississippians and white
sympathizers. The MFDP was, however, tremendously inconvenient for the
Johnson Administration. It had wanted to minimize the inroads that Barry
Goldwater's campaign was making into what had previously been the
Democratic stronghold of the "Solid South" and the support that George
Wallace received during the Democratic primaries in the North.

When the MFDP started to organize a fight over credentials, Johnson
originally would not budge. When Fannie Lou Hamer, the leader of the
MFDP, was in the midst of testifying about the police beatings of her
and others for attempting to exercise their right to vote, Johnson
preempted television coverage of the credentials fight. Even so, her
testimony created enough uproar that Johnson offered the MFDP a
"compromise": they would receive two non-voting seats, while the
delegation sent by the official Democratic Party would take its seats.

Johnson used all of his resources, mobilizing Walter Reuther, one of his
key supporters within the liberal wing of the Democratic Party, and his
Vice-Presidential nominee Hubert Humphrey, to pressure King and other
mainstream civil rights leaders to bring the MFDP around, while
directing Hoover to put the delegation under surveillance. The MFDP
rejected both the compromise and the pressure to accept it, and walked
out.

That experience destroyed what little faith SNCC activists had in the
federal government, even though Johnson had obtained a broad Civil
Rights Act barring discrimination in public accommodations, employment
and private education in 1964 and would go on to obtain an equally broad
Voting Rights Act in 1965. It also estranged SNCC leaders from many of
the mainstream leaders of the civil rights movement.

Those differences carried over into the voting rights struggle that
centered on Selma, Alabama in 1965. SNCC had begun organizing black
citizens to register to vote in Selma in 1963, but made little headway
against the adamant resistance of Sheriff Jim Clark and the White
Citizens' Council. In early 1965, local Selma activists asked the
Southern Christian Leadership Conference for help, and the two
organizations formed an uneasy alliance. They disagreed over tactical
and strategic issues, including the SCLC's decision not to attempt to
cross the Edmund Pettus Bridge a second time after county sheriffs and
state troopers attacked them on "Bloody Sunday" on March 7, 1965.

The civil rights activists crossed the bridge on the third attempt, with
the aid of a federal court order barring authorities from interfering
with the march. It was part of a five-day march to Montgomery, Alabama,
that helped dramatize the need for a Voting Rights Act. During this
period, SNCC activists became more and more disenchanted with
nonviolence, integration as a strategic goal, and cooperation with white
liberals or the Federal government.

\section{Change in strategy and
dissolution}\label{change-in-strategy-and-dissolution}

\begin{itemize}
\item
  \emph{SNCC was also deeply affected by the killing of Sammy Younge Jr.
  the first black college student to be killed as a result of his
  involvement in the civil rights movement.}
\item
  \emph{SNCC's experience with the COFO and Mississippi Freedom Summer
  solidified their estrangement from white liberals.}
\item
  \emph{SNCC took Younge's death as the occasion to denounce the war in
  Vietnam, the first statement of its kind by a major civil rights
  organization.}
\end{itemize}

SNCC's experience with the COFO and Mississippi Freedom Summer
solidified their estrangement from white liberals. During several points
in the Mississippi project, a team of Democratic Party operatives led by
Allard Lowenstein and Barney Frank tried to take over its management.
They sought to move decision-making power away from grassroots activists
in the South, and purge Communist-linked organizations (such as the
National Lawyers Guild) from SNCC's network, in spite of those
organizations having made crucial contributions to the movement. Dorothy
Zellner (a white radical SNCC staffer) remarked that, "What they
{[}Lowenstein and Frank{]} want is to let the Negro into the existing
society, not to change it."

SNCC was also deeply affected by the killing of Sammy Younge Jr. the
first black college student to be killed as a result of his involvement
in the civil rights movement. Younge was a Navy veteran who later
enrolled in the Tuskegee Institute and participated in the
Selma-to-Montgomery campaign, as well as other SNCC projects. His murder
by a white supremacist in January 1966, and subsequent acquittal of the
killer, furthered the group's disillusionment that the federal
government would protect them. SNCC took Younge's death as the occasion
to denounce the war in Vietnam, the first statement of its kind by a
major civil rights organization. SNCC highlighted Younge's death as an
example of the hypocrisy of fighting for freedom abroad while rights
were denied in the US and was used as a call for people to refuse the
draft and work for freedom at home instead.

Many within SNCC had grown skeptical about the tactics of nonviolence
and integration. After the Democratic convention of 1964, the group
began to split into two factions -- one favoring a continuation of
nonviolent, integration-oriented redress of grievances within the
existing political system, and the other moving towards Black Power and
Marxism.

\section{Lowndes County Freedom
Organization}\label{lowndes-county-freedom-organization}

\begin{itemize}
\item
  \emph{White supremacists regularly killed blacks, and sometimes their
  allies like white SNCC volunteer Jonathan Daniels, with impunity.}
\item
  \emph{The first SNCC project to promote the slogan "black power" was
  the Lowndes County Freedom Organization (LCFO) an African-American
  electoral organization which registered over 2,500 black voters
  between 1965 and 1969.}
\end{itemize}

The first SNCC project to promote the slogan "black power" was the
Lowndes County Freedom Organization (LCFO) an African-American electoral
organization which registered over 2,500 black voters between 1965 and
1969. This was a historic achievement given that Lowndes was the most
Klan-dominated area in Alabama and that, as a result, Lowndes had zero
registered black voters. Although the Voting Rights Act had been passed,
federal monitoring was sporadic and federal protection of black voters
inconsistent. White supremacists regularly killed blacks, and sometimes
their allies like white SNCC volunteer Jonathan Daniels, with impunity.
As such, most LCFO members did their organizing openly armed. They had
no confidence in appealing to the support of middle-class liberals (even
Martin Luther King and SCLC distanced themselves from the group) or the
national Democratic Party. LCFO co-founder John Hulett (later elected
Sheriff of Lowndes County) warned that this was the state of Alabama's
last chance to peacefully grant African Americans their rights: "We're
out to take power legally, but if we're stopped by the government from
doing it legally, we're going to take it the way everyone else took it,
including the way the Americans took it in the American Revolution."
Certain the federal government was not going to protect him and his
fellow party members, Hulett told a federal registrar, "if one of our
candidates gets touched, we're going to take care of the murderer
ourselves." Choosing a black panther as their mascot, LCFO was the first
of numerous local organizations to be known as "the black panther
party". (LCFO had no direct relationship with the later Black Panther
Party for Self-Defense founded by Huey Newton, however.)

While the LCFO candidates did not win their early campaigns, most
historians and activists regard the group's mere survival under such
hostile conditions to be a victory. In 1970 LCFO reconciled with the
local Democratic Party, and various candidates, including John Hulett,
went on to be Lowndes County officials.

\section{Stokely Carmichael as chair}\label{stokely-carmichael-as-chair}

\begin{itemize}
\item
  \emph{After a contentious debate over the meaning of "Black Power",
  issues of black nationalism and black separatism, and the
  organization's strategic direction, white SNCC members were asked to
  leave the organization in December 1966.}
\item
  \emph{The U.S. Department of Defense stated in 1967: "SNCC can no
  longer be considered a civil rights group.}
\end{itemize}

After the Watts riots in Los Angeles in 1965, more of SNCC's members
sought to break their ties with the mainstream civil rights movement and
the liberal organizations that supported it. They argued instead that
blacks needed to build power of their own, rather than seeking
accommodations from the power structure in place. SNCC migrated from a
philosophy of nonviolence to one of greater militancy after the
mid-1960s, as an advocate of the burgeoning Black Power movement, a
facet of late 20th-century black nationalism. The shift was personified
by Stokely Carmichael, who replaced John Lewis as SNCC chairman in
1966--67.

Carmichael raised the banner of Black Power nationally in a speech in
Greenwood, Mississippi in June 1966, as part of SNCC's response to the
attempted assassination of James Meredith. After a contentious debate
over the meaning of "Black Power", issues of black nationalism and black
separatism, and the organization's strategic direction, white SNCC
members were asked to leave the organization in December 1966. The vote,
characterized by some as "expelling" whites and by others as "asking
whites to work against racism in white communities," was extremely
close; 19 Aye, 18 Nay, and 24 abstentions.

SNCC continued to maintain coalition with several white radical
organizations, most notably Students for a Democratic Society (SDS), and
inspired them to focus on militant anti-draft resistance. At an
SDS-organized conference at UC Berkeley in October 1966, Carmichael
challenged the white left to escalate their resistance to the military
draft in a manner similar to the black movement. Some participants in
ghetto rebellions of the era had already associated their actions with
opposition to the Vietnam War, and SNCC had first disrupted an Atlanta
draft board in August 1966. According to historians Joshua Bloom and
Waldo Martin, SDS's first Stop the Draft Week of October 1967 was
"inspired by Black Power {[}and{]} emboldened by the ghetto rebellions."
SNCC appear to have originated the popular anti-draft slogan: "Hell no!
We won't go!" For a time in 1967, SNCC seriously considered an alliance
with Saul Alinsky's Industrial Areas Foundation, and generally supported
IAF's work in Rochester and Buffalo's black communities.

Expressing SNCC's evolving policy on nonviolence/violence, Carmichael
first argued that blacks should be free to use violence in self-defense;
later he advocated revolutionary violence to overthrow oppression.
Carmichael rejected the civil-rights legislation as mere palliatives.
The U.S. Department of Defense stated in 1967: "SNCC can no longer be
considered a civil rights group. It has become a racist organization
with black supremacy ideals and an expressed hatred for whites." (Martin
Luther King's Southern Christian Leadership Conference was classified as
a "hate-type" group by the federal government during the same period).

SNCC became a target of the Counterintelligence Program (COINTELPRO) of
the Federal Bureau of Investigation (FBI) in a concerted effort at all
levels of government to crush black radicalism -- both violent and
nonviolent -- through both overt and covert means ranging from
propaganda to assassination.

Charles E. Cobb, formerly SNCC field secretary in Mississippi, has said
that SNCC's grassroots and autonomous community work was undercut and
co-opted by Lyndon Johnson's War on Poverty: "After we got the Civil
Rights Act in 1964 and the Voting Rights Act in 1965, a lot of groups
that we had cultivated were absorbed into the Democratic Party...a lot
more money came into the states we were working in. A lot of the people
we were working with became a part of Head Start and various kinds of
poverty programs. We were too young to really know how to respond
effectively. How could we tell poor sharecroppers or maids making a few
dollars a day to walk away from poverty program salaries or stipends?"

\section{Post-1967}\label{post-1967}

\begin{itemize}
\item
  \emph{In 1968, SNCC lost numerous organizers, such as Kathleen Neal,
  Bob Brown, and Bobby Rush, to the Black Panther Party.}
\item
  \emph{The San Antonio SNCC chapter was part Black Panther Party and
  part SNCC.}
\item
  \emph{By then, SNCC was no longer an effective organization.}
\end{itemize}

By early 1967, SNCC was approaching bankruptcy as liberal funders
refused to support its overt militancy. Carmichael voluntarily stepped
down as chair in May 1967. H. Rap Brown, later known as Jamil Abdullah
Al-Amin, replaced him as the head of SNCC. Brown renamed the group the
Student National Coordinating Committee and supported violence, which he
described as "as American as cherry pie". He resigned as chair of SNCC
in 1968, after being indicted for inciting to riot in Cambridge,
Maryland, in 1967. In 1968, Carmichael was expelled from the group
completely by the new program secretary, Phil Hutchings, when Carmichael
refused to resign from the Black Panther Party. Carmichael, along with
Rap Brown and James Forman, had tried to foster an alliance between SNCC
and the Panthers, but it proved to be a failure.

By then, SNCC was no longer an effective organization. Much analysis at
the time blamed Carmichael's departure from the group for the decline,
though others would dispute this. In 1968, SNCC lost numerous
organizers, such as Kathleen Neal, Bob Brown, and Bobby Rush, to the
Black Panther Party. Ella Baker said that "SNCC came North at a time
when the North was in a ferment that led to various interpretations on
what was needed to be done. With its own frustrations, it could not take
the pace-setter role it took in the South..."

The organization largely disappeared in the early 1970s, although
chapters in some communities, such as San Antonio, Texas, continued for
several more years. Mario Marcel Salas, field secretary of the SNCC
chapter in San Antonio, operated until 1976. The San Antonio SNCC
chapter was part Black Panther Party and part SNCC. Dr. Charles Jones of
Albany State University termed it a "hybrid organization" because it had
Panther-style survival programs. Salas also worked closely with La Raza
Unida Party, running for political office and organizing demonstrations
to expose discrimination against Blacks and Latinos. Salas later helped
the New Jewel Movement in the otherthrow of Eric Gairy in 1979, the
leader of the island of Grenada. He also became the chairman of the Free
Nelson Mandela Movement in San Antonio, Texas.

Charles McDew, SNCC's second chairman, said that the organization was
not designed to last beyond its mission of winning civil rights for
blacks, and that at the founding meetings most participants expected it
to last no more than five years:

First, we felt if we go more than five years without the understanding
that the organization would be disbanded, we run the risk of becoming
institutionalized or being more concerned with trying to perpetuate the
organization and in doing so, giving up the freedom to act and to
do...The other thing is that by the end of that time you'd either be
dead or crazy\ldots{}"{[}54{]}

By the time of its conclusion, many of the controversial ideas that once
had defined SNCC's radicalism had become widely accepted among African
Americans.

\section{Geography}\label{geography}

\begin{itemize}
\item
  \emph{\\[3\baselineskip]Most of SNCC's early activity took place in
  Georgia and Mississippi.}
\item
  \emph{The 1964 voter registration project called "Freedom Summer"
  focused on Mississippi and marked the beginning of SNCC's rejection of
  Northern white volunteers.}
\item
  \emph{SNCC was eventually overshadowed by the Black Panther Party, who
  had a broader national reach.}
\end{itemize}

Most of SNCC's early activity took place in Georgia and Mississippi. In
the early 1960s, they mainly focused on voter registration projects in
the South, and multiple chapters were established throughout the region.
The Freedom Rides, in which interracial groups rode buses together and
challenged segregated seating arrangements, brought them media attention
and helped raise awareness in the North. In 1963, they played an
important role in the March on Washington.

The 1964 voter registration project called "Freedom Summer" focused on
Mississippi and marked the beginning of SNCC's rejection of Northern
white volunteers. Under Stokely Carmichael's leadership, SNCC shifted
its focus to the North, where it focused on alleviating poverty in
Northern urban areas. As SNCC became more radical in 1966 and 1967,
Carmichael established ties with several foreign governments. SNCC was
eventually overshadowed by the Black Panther Party, who had a broader
national reach.

\section{Feminism}\label{feminism}

\begin{itemize}
\item
  \emph{When Stokely Carmichael was elected Chair of SNCC, he reoriented
  the path of the organization towards Black Power.}
\item
  \emph{The degree and significance of male-domination and women's
  subordination was hotly debated within SNCC; many of SNCC's black
  women disputed the premise that women were denied leadership roles.}
\item
  \emph{Many black women held prominent positions in the movement as a
  result of their participation in SNCC.}
\end{itemize}

SNCC activist Bernice Johnson Reagon described the Civil Rights Movement
as the "'borning struggle' of the decade, in that it stimulated and
informed those that followed it," including the modern feminist
movement. The influence of the Civil Rights Movement inspired mass
protests and awareness campaigns as the main methods to obtain sexual
equality.

Many black women held prominent positions in the movement as a result of
their participation in SNCC. Some of these women include Ruby Doris
Smith Robinson, Donna Richards, Fay Bellamy, Gwen Patton, Cynthia
Washington, Jean Wiley, Muriel Tillinghast, Fannie Lou Hamer, Annie
Pearl Avery, Diane Nash, Ella Baker, Victoria Gray, Unita Blackwell,
Bettie Mae Fikes, Joyce Ladner, Dorie Ladner, Gloria Richardson, Bernice
Reagon, Prathia Hall, Gwendolyn Delores Robinson/Zoharah Simmons, Judy
Richardson, Martha Prescod Norman Noonan, Ruby Sales, Endesha Ida Mae
Holland, Eleanor Holmes Norton and Anne Moody.

"Women who were active in the lunch counter sit-in movement of 1960 led
the transformation of SNCC from a coordinating office into a cadre of
militant activists dedicated to expanding the civil rights movement
throughout the South. In February 1961, Diane Nash and Ruby Doris Smith
were among four SNCC members who joined the Rock Hill, South Carolina,
desegregation protests, which featured the jail-no-bail
tactic-demonstrators serving their jail sentences rather than accepting
bail." "In May 1961, Nash led a group of student activists to Alabama in
order to sustain the Freedom Rides after the initial group of protesters
organized by the Congress of Racial Equality (CORE) encountered mob
violence in Birmingham. During May and June, Nash, Smith, and other
student freedom riders traveled on buses from Montgomery to Jackson,
Mississippi, where they were swiftly arrested and imprisoned. In August,
when veterans of the sit-ins and the Freedom Rides met to discuss SNCC's
future, Baker helped to avoid a damaging split by suggesting separate
direct-action and voter-registration wings. Nash became the leader of
the direct-action wing of SNCC."

Young black girls also played a significant part in the SNCC
demonstrations. In July 1963, dozens of young black girls participated
in a SNCC protest of a segregated movie theater in Americus Georgia.
Over 30 of them were arrested and eventually held against their will in
the Leesburg Stockade. They were freed over a month later due to the
help of a SNCC volunteer who photographed the girls and published the
pictures in a SNCC newsletter. However, despite the multiple humans
rights violations that they enduring in the stockade, many girls
continued to fight for civil rights after they were freed.

Anne Moody published her autobiography, Coming of Age in Mississippi, in
1970, detailing her decision to participate in SNCC and later CORE, and
her experience as a woman in the movement. She described the widespread
trend of black women to become involved with SNCC at their educational
institutions. As young college students or teachers, these black women
were often heavily involved in grassroots campaign by teaching Freedom
Schools and promoting voter registration.

Young white women also became very involved with SNCC, particularly
after the Freedom Summer of 1964. Many northern white women were
inspired by the ideology of racial equality. The book Deep in Our Hearts
details the experiences of nine white women in SNCC. Some white women,
such as Mary King, Constance W. Curry, and Casey Hayden, and Latino
women such as Mary Varela and Elizabeth Sutherland Martinez, were able
to obtain status and leadership within SNCC.

Through organizations like SNCC, women of both races were becoming more
politically active than at any time in American history since the
Women's suffrage movement. A group of women in SNCC, later identified as
Mary King and Casey Hayden, openly challenged the way women were treated
when they issued the "SNCC Position Paper (Women in the Movement), or
adapted an earlier paper submitted at Waveland Meeting by Elaine DeLott
Baker." The paper was published anonymously, helping King and Hayden to
avoid unwanted attention. The paper listed 11 events in which women were
treated as subordinate to men. According to the paper, women in SNCC did
not have a chance to become the face of the organization, the top
leaders, because they were assigned to clerical and housekeeping duties,
whereas men were involved in decision-making. The degree and
significance of male-domination and women's subordination was hotly
debated within SNCC; many of SNCC's black women disputed the premise
that women were denied leadership roles. Ruby Doris Smith was often
falsely attributed as author of paper, yet Smith was looking towards
black nationalism at this time rather than interracial feminism. The
following year, King and Hayden produced another document entitled "Sex
and Caste: A Kind of Memo". The document was published in 1966 by
Liberation, the magazine of the War Resisters League. "Sex and Caste"
has since been credited as one of the generative documents that launched
second-wave feminism.

When Stokely Carmichael was elected Chair of SNCC, he reoriented the
path of the organization towards Black Power. He famously said in a
speech, "it is a call for black people to define their own goals, to
lead their own organizations." Thus, white women lost their influence
and power in SNCC; Mary King and Casey Hayden left to become active in
pursuing equality for women.

While it is often argued that the Black Power period led to a
downgrading of women generally in the organization, historian Barbara
Ransby notes that there is no real evidence of this. Carmichael
appointed several women to posts as project directors during his tenure
as chairman; by the latter half of the 1960s, more women were in charge
of SNCC projects than during the first half. Former SNCC member Kathleen
Cleaver played a key role in the central committee of the Black Panther
Party as communications secretary (1968). Her position in this "male
dominated" leadership was both effective and influential to Brown, Red
and Yellow Power groups of the late 1960s and early 1970s.

In 1968, the Third World Women's Alliance (TWWA) was originated in New
York by Frances M. Beal as a caucus of SNCC, addressing the issue of
sexism within the movement. By 1970 it had become independent from SNCC,
but maintained close ties with it. TWWA came to focus less on
specifically black power and more on Puerto Rican and Cuban liberation.
It continued to operate until 1978, with chapters in several major
cities.

\section{See also}\label{see-also}

\begin{itemize}
\item
  \emph{Mississippi Freedom Project - Samuel Proctor Oral History
  Program at the University of Florida}
\item
  \emph{Nonviolent action}
\end{itemize}

Amzie Moore

Faith Holsaert

Mississippi Freedom Project - Samuel Proctor Oral History Program at the
University of Florida

Nonviolent action

\section{References}\label{references}

\section{Further reading}\label{further-reading}

\section{Archives}\label{archives}

\begin{itemize}
\item
  \emph{SNCC History and Geography from the Mapping American Social
  Movements Project at the University of Washington.}
\item
  \emph{FBI COINTELPRO Black Extremist Records, a series of archival
  documents from the FBI that explicitly target SNCC and Stokely
  Carmichael for suppression.}
\item
  \emph{The University of Southern Mississippi Libraries Special
  Collections.}
\end{itemize}

Ellin (Joseph and Nancy) Freedom Summer Collection. Collection Number:
M323. Dates: 1963 - 1988. Volume: 1.7~ft³ (48 L)

The University of Southern Mississippi Libraries Special Collections.
Retrieved May 2, 2005.

SNCC History and Geography from the Mapping American Social Movements
Project at the University of Washington.

FBI COINTELPRO Black Extremist Records, a series of archival documents
from the FBI that explicitly target SNCC and Stokely Carmichael for
suppression.

\section{Books}\label{books}

\begin{itemize}
\item
  \emph{In Struggle, SNCC and the Black Awakening of the 1960s.}
\item
  \emph{The River of No Return: The Autobiography of a Black Militant
  and the Life and Death of SNCC.}
\item
  \emph{Ella Baker and the Black Freedom Movement: A Radical Democratic
  Vision.}
\item
  \emph{"Freedom Song: A Personal Story of the 1960s Civil Rights
  Movement".}
\item
  \emph{A Circle of Trust: Remembering SNCC.}
\item
  \emph{SNCC: The New Abolitionists.}
\end{itemize}

Carmichael, Stokely, and Michael Thelwell. Ready for Revolution: The
Life and Struggles of Stokely Carmichael (Kwame Ture). Scribner, 2005.
848 pages. ISBN~0-684-85004-4

Carson, Claybourne. In Struggle, SNCC and the Black Awakening of the
1960s. Cambridge Massachusetts: Harvard University Press, 1981.
ISBN~0-674-44727-1

Forman, James. The Making of Black Revolutionaries, 1985 and 1997, Open
Hand Publishing, Washington D.C. ISBN~0-295-97659-4 and
ISBN~0-940880-10-5

Greenberg, Cheryl Lynn, ed. A Circle of Trust: Remembering SNCC. Rutgers
University Press, 1998. 274 pages. ISBN~0-8135-2477-6

Halberstam, David. The Children, Ballantine Books, 1999.
ISBN~0-449-00439-2

Hamer, Fannie Lou, The Speeches of Fannie Lou Hamer: To Tell it Like it
is, University Press of Mississippi, 2011. ISBN~9781604738230.

Deep in Our Hearts: Nine White Women in the Freedom Movement, University
of Georgia Press, 2002. ISBN~0-8203-2419-1

Holsaert, Faith; Martha Prescod Norman Noonan, Judy Richardson, Betty
Garman Robinson, Jean Smith Young, and Dorothy M. Zellner, Hands on the
Freedom Plow: Personal Accounts by Women in SNCC. University of Illinois
Press, 2010. ISBN~978-0-252-03557-9.

Hogan, Wesley C. How Democracy travels: SNCC, Swarthmore students, and
the growth of the student movement in the North, 1961-1964.

Hogan, Wesley C. Many Minds, One Heart: SNCC's Dream for a New America,
University of North Carolina Press. 2007.

King, Mary. "Freedom Song: A Personal Story of the 1960s Civil Rights
Movement". 1987.

Lewis, John. Walking With the Wind: A Memoir of the Movement. New York:
Simon \& Schuster. 1998.

Pardun, Robert. Prairie Radical: A Journey Through the Sixties.
California: Shire Press. 2001. 376 pages. ISBN~0-918828-20-1

Ransby, Barbara. Ella Baker and the Black Freedom Movement: A Radical
Democratic Vision. University of North Carolina Press. 2003.

Salas, Mario Marcel. Masters Thesis: "Patterns of Persistence: Paternal
Colonialist Structures and the Radical Opposition in the African
American Community in San Antonio, Texas, 1937--2001", University of
Texas at San Antonio, John Peace Library 6900 Loop 1604, San Antonio,
Texas, 2002. Other SNCC material located in historical records at the
Institute of Texan Cultures, University of Texas at San Antonio as part
of the Mario Marcel Salas historical record.

Sellers, Cleveland, and Robert Terrell. The River of No Return: The
Autobiography of a Black Militant and the Life and Death of SNCC.
University Press of Mississippi; 1990 reprint. 289 pages.
ISBN~0-87805-474-X

Zinn, Howard. SNCC: The New Abolitionists. Boston: Beacon Press, 1964.
ISBN~0-89608-679-8

Payne, Charles M. I've Got the Light of Freedom: The Organizing
Tradition and the Mississippi Freedom Struggle, 2nd edition.
ISBN~0-52025-176-8

\section{Video}\label{video}

\begin{itemize}
\item
  \emph{Eighth Annual Forum on Women in Leadership Then and Now: Women
  in the Civil Rights Leadership, Joyce Ladner is one of the panelists
  and shares many stories about SNCC}
\end{itemize}

SNCC 50th Anniversary Conference 38 DVD collection documenting the
formal addresses, panel discussions and programs that took place at the
50th anniversary conference at Shaw University in Raleigh, North
Carolina.

Eighth Annual Forum on Women in Leadership Then and Now: Women in the
Civil Rights Leadership, Joyce Ladner is one of the panelists and shares
many stories about SNCC

\section{Interviews}\label{interviews}

\begin{itemize}
\item
  \emph{Interviews with civil rights workers from the Student Nonviolent
  Coordinating Committee (SNCC).}
\item
  \emph{SNCC member and Freedom Summer participant.}
\end{itemize}

Transcript: An Oral History with Terri Shaw. SNCC member and Freedom
Summer participant. The University of Southern Mississippi Libraries
Special Collections. Retrieved May 2, 2005.

Interviews with civil rights workers from the Student Nonviolent
Coordinating Committee (SNCC). Stanford University Project South oral
history collection. Microfilming Corp. of America. 1975.
ISBN~0-88455-990-4.

\section{Publications and documents}\label{publications-and-documents}

\begin{itemize}
\item
  \emph{Student Nonviolent Coordinating Committee Founding Statement.}
\item
  \emph{Memorandum: on the SNCC Mississippi Summer Project Transcript.}
\end{itemize}

Student Nonviolent Coordinating Committee Founding Statement.

Memorandum: on the SNCC Mississippi Summer Project Transcript. Oxford,
Ohio: General Materials (c. June 1964). Retrieved May 2, 2005.

\section{External links}\label{external-links}

\begin{itemize}
\item
  \emph{Civil Rights Movement Veterans}
\item
  \emph{SNCC 1960 - 1966: Six years of the Student Nonviolent
  Coordinating Committee.}
\item
  \emph{The SNCC Digital Gateway}
\item
  \emph{Americus Movement, Civil Rights Digital Library.}
\item
  \emph{SNCC Documents Online collection of original SNCC documents
  \textasciitilde{} Civil Rights Movement Veterans.}
\item
  \emph{SNCC Actions 1960-1970 (map)}
\end{itemize}

The SNCC Digital Gateway

The SNCC Project: A Year by Year History 1960-1970

SNCC Actions 1960-1970 (map)

SNCC 1960 - 1966: Six years of the Student Nonviolent Coordinating
Committee. Retrieved May 2, 2005.

Stokely Carmichael - Leader of SNCC's militant branch

Civil Rights Movement Veterans

SNCC Documents Online collection of original SNCC documents
\textasciitilde{} Civil Rights Movement Veterans.

Americus Movement, Civil Rights Digital Library.

The Story of SNCC, One Person, One Vote Project
