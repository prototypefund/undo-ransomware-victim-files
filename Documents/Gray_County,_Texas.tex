\textbf{From Wikipedia, the free encyclopedia}

https://en.wikipedia.org/wiki/Gray\_County\%2C\_Texas\\
Licensed under CC BY-SA 3.0:\\
https://en.wikipedia.org/wiki/Wikipedia:Text\_of\_Creative\_Commons\_Attribution-ShareAlike\_3.0\_Unported\_License

\section{Gray County, Texas}\label{gray-county-texas}

\begin{itemize}
\item
  \emph{The Clinton-Oklahoma-Western Railroad Company of Texas served
  Gray County with service to Hemphill County at the Oklahoma border.}
\item
  \emph{The county seat is Pampa.}
\item
  \emph{Gray County is a county located in the U.S. state of Texas.}
\item
  \emph{Gray County comprises the Pampa, TX Micropolitan Statistical
  Area.}
\end{itemize}

Gray County is a county located in the U.S. state of Texas. As of the
2010 census, its population was 22,535. The county seat is Pampa. The
county was created in 1876 and later organized in 1902. is named for
Peter W. Gray, a Confederate lawyer and soldier in the American Civil
War.

Gray County comprises the Pampa, TX Micropolitan Statistical Area.

Gray County was the center of the White Deer Lands Management Company,
which ceased operations in 1957. The history of the company is the theme
of the White Deer Land Museum in Pampa, but company archives are at the
Panhandle-Plains Historical Museum in Canyon. Timothy Dwight Hobart, the
White Deer land agent from 1903 to 1924, was elected mayor of Pampa in
1927.

The Clinton-Oklahoma-Western Railroad Company of Texas served Gray
County with service to Hemphill County at the Oklahoma border. Another
line then connected eastward to Clinton, Oklahoma. There was an
eleven-mile extension of the COW-T from rural nHeaton to the former oil
camp of Coltexo in Gray County. Originally a Frank Kell property, the
COW-T was acquired in 1928 by the Atchison, Topeka and Santa Fe Railway,
which then leased it in 1931 to the former Panhandle and Santa Fe
Railway.

\section{Geography}\label{geography}

\begin{itemize}
\item
  \emph{According to the U.S. Census Bureau, the county has a total area
  of 929 square miles (2,410~km2), of which 926 square miles (2,400~km2)
  is land and 3.4 square miles (8.8~km2) (0.4\%) is water.}
\end{itemize}

According to the U.S. Census Bureau, the county has a total area of 929
square miles (2,410~km2), of which 926 square miles (2,400~km2) is land
and 3.4 square miles (8.8~km2) (0.4\%) is water.

\section{Major highways}\label{major-highways}

\begin{itemize}
\item
  \emph{State Highway 152}
\item
  \emph{State Highway 273}
\item
  \emph{State Highway 70}
\item
  \emph{U.S. Highway 60}
\end{itemize}

Interstate 40

U.S. Highway 60

State Highway 70

State Highway 152

State Highway 273

\section{Adjacent counties}\label{adjacent-counties}

\begin{itemize}
\item
  \emph{Donley County (south)}
\item
  \emph{Hemphill County (northeast)}
\item
  \emph{Collingsworth County (southeast)}
\item
  \emph{Wheeler County (east)}
\item
  \emph{Hutchinson County (northwest)}
\item
  \emph{Carson County (west)}
\item
  \emph{Roberts County (north)}
\end{itemize}

Roberts County (north)

Wheeler County (east)

Donley County (south)

Carson County (west)

Hemphill County (northeast)

Hutchinson County (northwest)

Collingsworth County (southeast)

\section{National protected area}\label{national-protected-area}

\begin{itemize}
\item
  \emph{McClellan Creek National Grassland}
\end{itemize}

McClellan Creek National Grassland

\section{Demographics}\label{demographics}

\begin{itemize}
\item
  \emph{The per capita income for the county was \$16,702.}
\item
  \emph{In the county, the population was spread out with 24.00\% under
  the age of 18, 8.40\% from 18 to 24, 27.20\% from 25 to 44, 22.30\%
  from 45 to 64, and 18.10\% who were 65 years of age or older.}
\item
  \emph{The median income for a household in the county was \$31,368,
  and the median income for a family was \$40,019.}
\item
  \emph{As of the census of 2000, there were 22,744 people, 8,793
  households, and 6,049 families residing in the county.}
\end{itemize}

As of the census of 2000, there were 22,744 people, 8,793 households,
and 6,049 families residing in the county. The population density was 24
people per square mile (9/km²). There were 10,567 housing units at an
average density of 11 per square~mile (4/km²). The racial makeup of the
county was 82.15\% White, 5.85\% Black or African American, 0.94\%
Native American, 0.39\% Asian, 0.02\% Pacific Islander, 8.23\% from
other races, and 2.42\% from two or more races. 13.01\% of the
population were Hispanic or Latino of any race.

There were 8,793 households out of which 30.00\% had children under the
age of 18 living with them, 57.00\% were married couples living
together, 9.00\% had a female householder with no husband present, and
31.20\% were non-families. 28.70\% of all households were made up of
individuals and 15.30\% had someone living alone who was 65 years of age
or older. The average household size was 2.39 and the average family
size was 2.93.

In the county, the population was spread out with 24.00\% under the age
of 18, 8.40\% from 18 to 24, 27.20\% from 25 to 44, 22.30\% from 45 to
64, and 18.10\% who were 65 years of age or older. The median age was 39
years. For every 100 females, there were 104.00 males. For every 100
females age 18 and over, there were 103.70 males.

The median income for a household in the county was \$31,368, and the
median income for a family was \$40,019. Males had a median income of
\$32,401 versus \$20,158 for females. The per capita income for the
county was \$16,702. About 11.20\% of families and 13.80\% of the
population were below the poverty line, including 17.60\% of those under
age 18 and 9.60\% of those age 65 or over.

\section{Communities}\label{communities}

\section{City}\label{city}

\begin{itemize}
\item
  \emph{Pampa (county seat)}
\end{itemize}

Pampa (county seat)

\section{Town}\label{town}

\begin{itemize}
\item
  \emph{McLean}
\item
  \emph{Lefors}
\end{itemize}

Lefors

McLean

\section{Unincorporated community}\label{unincorporated-community}

\begin{itemize}
\item
  \emph{Alanreed}
\end{itemize}

Alanreed

\section{Politics}\label{politics}

\begin{itemize}
\item
  \emph{Starting with the 1952 election, the county has become a
  Republican stronghold along with the rest of the Texas Panhandle.}
\item
  \emph{Prior to 1952, Gray County was primarily Democratic similar to
  most of Texas \& the Solid South.}
\item
  \emph{The county only gave a Republican presidential candidate a
  majority before 1952 in 1928 when Herbert Hoover won the county thanks
  to anti-Catholic sentiment towards Al Smith.}
\end{itemize}

Prior to 1952, Gray County was primarily Democratic similar to most of
Texas \& the Solid South. The county only gave a Republican presidential
candidate a majority before 1952 in 1928 when Herbert Hoover won the
county thanks to anti-Catholic sentiment towards Al Smith. Starting with
the 1952 election, the county has become a Republican stronghold along
with the rest of the Texas Panhandle. This level of Republican dominance
has increased in recent years, as every Republican presidential
candidate in the second millennium has racked up 80 percent of the
county's vote. Additionally, in the two most recent presidential
elections, Democrats Barack Obama \& Hillary Clinton have failed to win
even 1,000 votes total in the county.

\section{See also}\label{see-also}

\begin{itemize}
\item
  \emph{Kae T. Patrick, native of Gray County who served in the Texas
  House of Representatives from San Antonio from 1981 to 1988}
\item
  \emph{Recorded Texas Historic Landmarks in Gray County}
\item
  \emph{National Register of Historic Places listings in Gray County,
  Texas}
\item
  \emph{Tom Mechler, state Republican Party chairman since 2015; former
  Gray County Republican chairman}
\end{itemize}

List of museums in the Texas Panhandle

National Register of Historic Places listings in Gray County, Texas

Recorded Texas Historic Landmarks in Gray County

Phil Cates, state representative from 1971 to 1979, born in Pampa in
1947

Tom Mechler, state Republican Party chairman since 2015; former Gray
County Republican chairman

Kae T. Patrick, native of Gray County who served in the Texas House of
Representatives from San Antonio from 1981 to 1988

\section{References}\label{references}

\section{External links}\label{external-links}

\begin{itemize}
\item
  \emph{Gray County Profile from the Texas Association of Counties}
\item
  \emph{Gray County from the Handbook of Texas Online}
\item
  \emph{Gray County government's website}
\end{itemize}

Gray County government's website

Gray County from the Handbook of Texas Online

Gray County Profile from the Texas Association of Counties

Coordinates: 35°25′N 100°49′W / 35.41°N 100.81°W / 35.41; -100.81
