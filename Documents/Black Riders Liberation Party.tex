\textbf{From Wikipedia, the free encyclopedia}

https://en.wikipedia.org/wiki/Black\%20Riders\%20Liberation\%20Party\\
Licensed under CC BY-SA 3.0:\\
https://en.wikipedia.org/wiki/Wikipedia:Text\_of\_Creative\_Commons\_Attribution-ShareAlike\_3.0\_Unported\_License

\section{Black Riders Liberation
Party}\label{black-riders-liberation-party}

\begin{itemize}
\item
  \emph{The Black Riders Liberation Party (BRLP) is a revolutionary
  black power organization, based in the United States.}
\end{itemize}

The Black Riders Liberation Party (BRLP) is a revolutionary black power
organization, based in the United States. The group claims ideological
continuity with the original Black Panther Party for Self-Defense and,
according to its official website, organizes gang members to "stop
commiting genocide against each other and to stand up against white
supremacy and capitalist oppression."

\section{History}\label{history}

\section{Establishment}\label{establishment}

\begin{itemize}
\item
  \emph{The Black Riders Liberation Party traces its origins back to a
  class conducted at the Youth Training School in Chino, California,
  conducted by the California Youth Authority for prisoners in the
  California state penal system.}
\item
  \emph{Inspired by the historic example of the Black Panther Party for
  Self-Defense, upon his release from prison in 1996 Culton sought to
  build a new political organization by gathering others from the
  predominately African-American ghettos of South Central Los Angeles
  and Watts.}
\end{itemize}

The Black Riders Liberation Party traces its origins back to a class
conducted at the Youth Training School in Chino, California, conducted
by the California Youth Authority for prisoners in the California state
penal system. Among these was Mischa Culton, an individual also using
the noms de guerre "General T.A.C.O.," an acronym for Taking All
Capitalists Out, and "Wolverine Shakur." Inspired by the historic
example of the Black Panther Party for Self-Defense, upon his release
from prison in 1996 Culton sought to build a new political organization
by gathering others from the predominately African-American ghettos of
South Central Los Angeles and Watts.

The fledgling organization started by Culton was energized by a November
17, 1997 police shooting of a mentally troubled black man in the Jordan
Downs housing complex in Watts, a suicidal individual who had lunged at
officers with a butter knife. The result was a vigilance program given
the provocative moniker "Watch a Pig," which encouraged citizens
"standing a legal distance from the pigs and making sure they don't
brutalize the people," in the words of the group's "Minister of Public
Relations."

\section{Development}\label{development}

\begin{itemize}
\item
  \emph{The new front group was inspired by the National Committee to
  Combat Fascism (NCCF) of the Black Panther Party, according to a
  representative of the organization.}
\end{itemize}

Originally limited to Southern California, in 2010 a section of the
organization based in Oakland, California was initiated.

In November 2012 the BRLP launched a mass organization called the
Inter-Communal Solidarity Committee in Los Angeles, attempting to build
broader support for a common program. The new front group was inspired
by the National Committee to Combat Fascism (NCCF) of the Black Panther
Party, according to a representative of the organization.

In March 2015 the BRLP decided to take advantage of the open gun
carrying law in Texas, traveling to Austin to conduct an armed march to
the Texas state capitol together with the Huey P. Newton Gun Club. Held
in conjunction with the heavily attended South by Southwest conference,
the joint march was conducted in an effort to "raise the cry for armed
self-defense" by the black community, according to the marchers.

\section{Ideology}\label{ideology}

\begin{itemize}
\item
  \emph{The Black Riders have adopted a party manifesto known as the
  Black Commune program which put forward many of the demands of the
  original 1966 Ten-Point Program, with the addition of new demands (for
  the proper medical care of AIDS sufferers and an end to the trade in
  crack cocaine in the black community).}
\end{itemize}

The group styles itself as an organization of "black revolutionaries"
engaged in a "people's war" against a white-dominated "oppressive
capitalistic system." The group advocates on behalf of civil rights and
social justice and actively seeks to end gang violence so as to "change
gang mentality into revolutionary mentality."

The BRLP professes a belief in the ideas of revolutionary socialism, and
on May Day 2012 were part of a small and ineffectual "General Strike"
effort in Los Angeles. The group claimed that their May 1 participation
was met with retaliation by government authorities, who are said to have
burst into the home of party leader Mischa Culton two days later with
automatic rifles during what was later explained as a routine
"compliance check" by the California Department of Corrections and
Rehabilitation.

The Black Riders have adopted a party manifesto known as the Black
Commune program which put forward many of the demands of the original
1966 Ten-Point Program, with the addition of new demands (for the proper
medical care of AIDS sufferers and an end to the trade in crack cocaine
in the black community).

The group's founder and chief theoretician, Mischa Culton, has called
Barack Obama, the first African-American President of the United States,
the "ultimate neocolonial puppet" and "the grand house negro" and
declared that the American government and political system was "designed
to enslave, massacre, and genocide our people out of this country."

Culton advocates for an autonomous, black-directed movement, encouraging
sympathetic whites to fight against police abuse and the ideology of
white supremacy. In a 2015 interview with Vice magazine, he declared:

\section{Programs and publications}\label{programs-and-publications}

\begin{itemize}
\item
  \emph{The BRLP launched its own newspaper in 2012, the eponymous Black
  Riders Liberation Party.}
\end{itemize}

In addition to its "Watch a Pig" police monitoring campaign, the BRLP
conducts ideological training under the slogan "Educate 2 Liberate" and
maintains what it calls the "Break the Lock Prisoner Support" program.

The BRLP launched its own newspaper in 2012, the eponymous Black Riders
Liberation Party.

\section{In popular culture}\label{in-popular-culture}

\begin{itemize}
\item
  \emph{The group was the subject of a 2013 documentary film, Let Um
  Hear Ya Coming.}
\end{itemize}

The group was the subject of a 2013 documentary film, Let Um Hear Ya
Coming.

\section{Criticism}\label{criticism}

\begin{itemize}
\item
  \emph{The Southern Poverty Law Center classifies the Black Riders as a
  black separatist hate group and notes that its members hold black
  supremacist views.}
\end{itemize}

The Southern Poverty Law Center classifies the Black Riders as a black
separatist hate group and notes that its members hold black supremacist
views.

\section{See also}\label{see-also}

\begin{itemize}
\item
  \emph{Black Power movement}
\item
  \emph{Black Panther Party}
\item
  \emph{New Black Panther Party}
\end{itemize}

Black Power movement

Civil Rights

Black Panther Party

Huey P. Newton Gun Club

List of organizations designated by the Southern Poverty Law Center as
hate groups

New Black Panther Party

\section{Footnotes}\label{footnotes}

\section{Further reading}\label{further-reading}

\begin{itemize}
\item
  \emph{Dennis Romero, "Black Riders Liberation Party's General TACO
  Visited by Authorities After May Day Rally," LA Weekly, May 3, 2012.}
\end{itemize}

Dennis Romero, "Black Riders Liberation Party's General TACO Visited by
Authorities After May Day Rally," LA Weekly, May 3, 2012.

\section{External links}\label{external-links}

\begin{itemize}
\item
  \emph{Soynoise, "Black Riders Liberation Party," Oct. 23, 2011.}
\item
  \emph{Jesse Roots, "Black Riders Honored by the Black Panther Alumni
  Association," April 23, 2013.}
\item
  \emph{Black Riders Liberation Party official web site}
\item
  \emph{"Black Riders Liberation Party Black Commune Program," February
  2013.}
\end{itemize}

Black Riders Liberation Party official web site

"Black Riders Liberation Party Black Commune Program," February 2013.

Jesse Roots, "Black Riders Honored by the Black Panther Alumni
Association," April 23, 2013. ---Video.

Soynoise, "Black Riders Liberation Party," Oct. 23, 2011. ---Video.
