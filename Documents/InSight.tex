\textbf{From Wikipedia, the free encyclopedia}

https://en.wikipedia.org/wiki/InSight\\
Licensed under CC BY-SA 3.0:\\
https://en.wikipedia.org/wiki/Wikipedia:Text\_of\_Creative\_Commons\_Attribution-ShareAlike\_3.0\_Unported\_License

\section{InSight}\label{insight}

\begin{itemize}
\item
  \emph{NASA officials decided in March 2016 to delay launching InSight
  to May 2018.}
\item
  \emph{By reusing technology from the Mars Phoenix lander, which
  successfully landed on Mars in 2008, mission costs and risks were
  reduced.}
\item
  \emph{The Interior Exploration using Seismic Investigations, Geodesy
  and Heat Transport (InSight) mission is a robotic lander designed to
  study the deep interior of the planet Mars.}
\end{itemize}

The Interior Exploration using Seismic Investigations, Geodesy and Heat
Transport (InSight) mission is a robotic lander designed to study the
deep interior of the planet Mars. It was manufactured by Lockheed
Martin, is managed by NASA's Jet Propulsion Laboratory, and most of its
scientific instruments were built by European agencies. The mission
launched on 5 May 2018 at 11:05~UTC aboard an Atlas V-401 rocket and
successfully landed at Elysium Planitia on Mars on 26 November 2018 at
19:52:59 UTC. InSight traveled 483~million~km (300~million~mi) during
its journey.

InSight's objectives are to place a seismometer, called SEIS, on the
surface of Mars to measure seismic activity and provide accurate 3D
models of the planet's interior; and measure internal heat flow using a
heat probe called HP3 to study Mars' early geological evolution. This
could bring a new understanding of how the Solar System's terrestrial
planets~-- Mercury, Venus, Earth, Mars~-- and Earth's Moon form and
evolve.

The lander was originally planned for launch in March 2016. Following a
persistent vacuum failure in the SEIS instrument prior to launch, with
the 2016 launch window missed, InSight was returned to Lockheed Martin's
facility in Denver, Colorado, for storage. NASA officials decided in
March 2016 to delay launching InSight to May 2018. This allowed time for
the seismometer to be fixed, although it increased the total cost from
US\$675 million to US\$830 million. By reusing technology from the Mars
Phoenix lander, which successfully landed on Mars in 2008, mission costs
and risks were reduced.

\section{History}\label{history}

\includegraphics[width=5.50000in,height=4.01265in]{media/image1.jpg}\\
\emph{InSight comes together with the backshell and surface lander being
joined, 2015.}

\section{Discovery Program selection}\label{discovery-program-selection}

\begin{itemize}
\item
  \emph{In August 2012, InSight was selected for development and
  launch.}
\item
  \emph{InSight was initially known as GEMS (Geophysical Monitoring
  Station), but its name was changed in early 2012 following a request
  by NASA.}
\item
  \emph{Managed by NASA's Jet Propulsion Laboratory (JPL) with
  participation from scientists from several countries, the mission was
  cost-capped at US\$425 million, not including launch vehicle funding.}
\end{itemize}

InSight was initially known as GEMS (Geophysical Monitoring Station),
but its name was changed in early 2012 following a request by NASA. Out
of 28 proposals from 2010, it was one of the three Discovery Program
finalists receiving US\$3 million in May 2011 to develop a detailed
concept study. In August 2012, InSight was selected for development and
launch. Managed by NASA's Jet Propulsion Laboratory (JPL) with
participation from scientists from several countries, the mission was
cost-capped at US\$425 million, not including launch vehicle funding.

\section{Schedule issues}\label{schedule-issues}

\begin{itemize}
\item
  \emph{The spacecraft was rescheduled to launch on 5 May 2018 for a
  Mars landing on 26 November at 3 p.m.}
\item
  \emph{On 9 March 2016, NASA officials announced that InSight would be
  delayed until the 2018 launch window at an estimated cost of US\$150
  million.}
\end{itemize}

Lockheed Martin began construction of the lander on 19 May 2014, with
general testing starting in 27 May 2015.

A persistent vacuum leak in the CNES-supplied seismometer known as the
Seismic Experiment for Interior Structure (SEIS) led NASA to postpone
the planned launch in March 2016 to May 2018. When InSight was delayed,
the rest of the spacecraft was returned to Lockheed Martin's factory in
Colorado for storage, and the Atlas V rocket intended to launch the
spacecraft was reassigned to the WorldView-4 mission.

On 9 March 2016, NASA officials announced that InSight would be delayed
until the 2018 launch window at an estimated cost of US\$150 million.
The spacecraft was rescheduled to launch on 5 May 2018 for a Mars
landing on 26 November at 3 p.m. The flight plan remained unchanged with
launch using an Atlas V rocket from Vandenberg Air Force Base in
California. NASA's Jet Propulsion Laboratory was tasked with redesigning
and building a new vacuum enclosure for the SEIS instrument, while CNES
conducted instrument integration and testing.

On 22 November 2017 InSight completed testing in a thermal vacuum, also
known as TVAC testing, where the spacecraft is put in simulated space
conditions with reduced pressure and various thermal loads. On 23
January 2018, after a long storage, its solar panels were once again
deployed and tested, and a second silicon chip containing 1.6~million
names from the public was added to the lander.

\includegraphics[width=5.44867in,height=5.50000in]{media/image2.jpg}\\
\emph{The Apollo 11 seismometer, 1969}

\section{Science background}\label{science-background}

\section{Vibrations}\label{vibrations}

\begin{itemize}
\item
  \emph{One of the aspects of the InSight mission is to compare the
  Earth, Moon, and Mars seismic data to learn more.}
\item
  \emph{Two other problems were the location of the lander and that a
  certain level of wind on Mars caused a loss of sensitivity for the
  Viking 2 seismometer.}
\item
  \emph{The Viking 2 seismometer did detect vibrations from Mars winds
  complementing the meteorology results.}
\end{itemize}

Seismometers on both Viking spacecraft were mounted on the lander, and
picked up vibrations from various operations of the lander and from the
wind. However, the Viking 1 lander's seismometer did not deploy properly
in 1976 after it landed; the seismometer remained locked and did not
unlock. The Viking 2 seismometer did unlock, and was able to vibrate and
return data to Earth. One problem was accounting for other data, as this
was the issue with an event detected on Sol 80 by the Viking 2
seismometer. When this event was recorded, no wind data were recorded at
the same time, so it was not possible to determine if the data indicated
a seismic event or wind gust. However, for the Sol 80 event the main
problem was not wind noise per se, but rather a lack of other data to
rule out other sources of vibrations. Two other problems were the
location of the lander and that a certain level of wind on Mars caused a
loss of sensitivity for the Viking 2 seismometer. InSight has many other
sensors, is placed directly on the surface, and also has a wind shield.

Despite the difficulties, the Viking 2 seismometer readings were used to
estimate a Martian geological crust thickness between 14 and 18~km (8.7
and 11.2~mi) at the Viking 2 lander site. The Viking 2 seismometer did
detect vibrations from Mars winds complementing the meteorology results.
There was the aforementioned candidate for a possible marsquake, but is
not particularly definitive. The wind data did prove useful in its own
right, and despite the limitations of the data, widespread and large
marsquakes were not detected.

Seismometers were also left on the Moon, starting with Apollo 11 in
1969, and also by Apollo 12, 14, 15 and 16 missions and provided many
insights into lunar seismology, including the discovery of moonquakes.
The Apollo seismic network, which was operated until 1977, detected at
least 28 moonquakes up to 5.5 on the Richter scale.

One of the aspects of the InSight mission is to compare the Earth, Moon,
and Mars seismic data to learn more.

\section{Nutation}\label{nutation}

\begin{itemize}
\item
  \emph{Radio Doppler measurements were taken with Viking and twenty
  years later with Mars Pathfinder, and in each case the axis of
  rotation of Mars was estimated.}
\end{itemize}

Radio Doppler measurements were taken with Viking and twenty years later
with Mars Pathfinder, and in each case the axis of rotation of Mars was
estimated. By combining this data the core size was constrained, because
the change in axis of rotation over 20 years allowed a precession rate
and from that the planet's moment of inertia to be estimated. InSight's
measurements of crust thickness, mantle viscosity, core radius and
density, and seismic activity should result in a three- to tenfold
increase in accuracy compared to current data.

\includegraphics[width=5.50000in,height=1.99833in]{media/image3.jpg}\\
\emph{Comparison of the interiors of Earth, Mars and the Moon (artist
concept)}

\includegraphics[width=5.50000in,height=3.09375in]{media/image4.jpg}\\
\emph{InSight lander on Mars (artist concept)}

\section{Objectives}\label{objectives}

\begin{itemize}
\item
  \emph{This data would be the first of its kind for Mars.}
\item
  \emph{InSight's primary objective is to study the earliest
  evolutionary history of the processes that shaped Mars.}
\item
  \emph{The InSight mission placed a single stationary lander on Mars to
  study its deep interior and address a fundamental issue of planetary
  and Solar System science: understanding the processes that shaped the
  rocky planets of the inner Solar System (including Earth) more than
  four~billion years ago.}
\end{itemize}

The InSight mission placed a single stationary lander on Mars to study
its deep interior and address a fundamental issue of planetary and Solar
System science: understanding the processes that shaped the rocky
planets of the inner Solar System (including Earth) more than
four~billion years ago.

InSight's primary objective is to study the earliest evolutionary
history of the processes that shaped Mars. By studying the size,
thickness, density and overall structure of Mars' core, mantle and
crust, as well as the rate at which heat escapes from the planet's
interior, InSight will provide a glimpse into the evolutionary processes
of all of the rocky planets in the inner Solar System. The rocky inner
planets share a common ancestry that begins with a process called
accretion. As the body increases in size, its interior heats up and
evolves to become a terrestrial planet, containing a core, mantle and
crust. Despite this common ancestry, each of the terrestrial planets is
later shaped and molded through a poorly understood process called
differentiation. InSight mission's goal is to improve the understanding
of this process and, by extension, terrestrial evolution, by measuring
the planetary building blocks shaped by this differentiation: a
terrestrial planet's core, mantle and crust.

The mission will determine if there is any seismic activity, measure the
rate of heat flow from the interior, estimate the size of Mars' core and
whether the core is liquid or solid. This data would be the first of its
kind for Mars. It is also expected that frequent meteor airbursts
(10--200 detectable events per year for InSight) will provide additional
seismo-acoustic signals to probe the interior of Mars. The mission's
secondary objective is to conduct an in-depth study of geophysics,
tectonic activity and the effect of meteorite impacts on Mars, which
could provide knowledge about such processes on Earth. Measurements of
crust thickness, mantle viscosity, core radius and density, and seismic
activity should result in a three- to tenfold increase in accuracy
compared to current data. This is the first time a robotic lander dug
this deep into the martian crust.

In terms of fundamental processes shaping planetary formation, it is
thought that Mars contains the most in-depth and accurate historical
record, because it is big enough to have undergone the earliest
accretion and internal heating processes that shaped the terrestrial
planets, but is small enough to have retained signs of those processes.
The science phase is expected to last for two years.

\includegraphics[width=5.50000in,height=3.66667in]{media/image5.jpg}\\
\emph{The InSight lander with solar panels deployed in a cleanroom}

\section{Design}\label{design}

\begin{itemize}
\item
  \emph{The mission includes two relay microsatellites called Mars Cube
  One (MarCO) that launched with InSight but were flying in formation
  with InSight to Mars.}
\item
  \emph{The mission further develops a design based on the 2008 Phoenix
  Mars lander.}
\item
  \emph{Three major aspects to the InSight spacecraft are the cruise
  stage, the entry, descent, and landing system, and the lander.}
\end{itemize}

The mission further develops a design based on the 2008 Phoenix Mars
lander. Because InSight is powered by solar panels, it landed near the
equator to enable maximum power for a projected lifetime of two years (1
Martian year). The mission includes two relay microsatellites called
Mars Cube One (MarCO) that launched with InSight but were flying in
formation with InSight to Mars.

Three major aspects to the InSight spacecraft are the cruise stage, the
entry, descent, and landing system, and the lander.

\section{Overall specifications}\label{overall-specifications}

\begin{itemize}
\item
  \emph{Total mass during cruise: 694~kg (1,530~lb)\\
  Lander: 358~kg (789~lb){[}3{]}\\
  Aeroshell: 189~kg (417~lb){[}3{]}\\
  Cruise stage: 79~kg (174~lb){[}3{]}\\
  Propellant and pressurant: 67~kg (148~lb){[}3{]}}
\item
  \emph{Relay probes flew separately but they weighed 13.5~kg (30~lb)
  each (there were 2)}
\end{itemize}

Mass

Total mass during cruise: 694~kg (1,530~lb)\\
Lander: 358~kg (789~lb){[}3{]}\\
Aeroshell: 189~kg (417~lb){[}3{]}\\
Cruise stage: 79~kg (174~lb){[}3{]}\\
Propellant and pressurant: 67~kg (148~lb){[}3{]}

Relay probes flew separately but they weighed 13.5~kg (30~lb) each
(there were 2)

\includegraphics[width=5.39733in,height=5.50000in]{media/image6.png}\\
\emph{Comparison of single-sol energy generated by various probes on
Mars. (30 November 2018)}

\section{Lander specifications}\label{lander-specifications}

\begin{itemize}
\item
  \emph{Tilt of lander at landing on Mars: 4 degrees}
\end{itemize}

Lander mass (Earth weight): 358~kg (789~lb)\\
Mars weight (0.376 of Earth's){[}59{]}~: 134.608~kg (296.76~lb)

About 6.0~m (19.7~ft) wide with solar panels deployed.

The science deck is about 1.56~m (5.1~ft) wide and between 0.83 and
1.08~m (2.7 and 3.5~ft) high (depending on leg compression after
landing).

The length of the robotic arm is 2.4~m (7.9~ft)

Tilt of lander at landing on Mars: 4 degrees

Power

Power is generated by two round solar panels, each 2.15~m (7.1~ft) in
diameter and consisting of SolAero ZTJ triple-junction solar cells made
of InGaP/InGaAs/Ge arranged on Orbital ATK UltraFlex arrays. After
touchdown on the Martian surface, the arrays are deployed by opening
like a folding fan.

Rechargeable batteries

Solar panels yielded 4.6 kilowatt-hours on Sol 1

Each panel is 7 feet in diameter (2.2 meters) unfurled

\includegraphics[width=5.09667in,height=5.50000in]{media/image7.jpg}\\
\emph{InSight lander with labeled instruments}

\includegraphics[width=5.50000in,height=2.77750in]{media/image8.jpg}\\
\emph{InSight collecting weather data (artist concept)}

\section{Payload}\label{payload}

\begin{itemize}
\item
  \emph{They were ejected from the stage after launch and coasted to
  Mars independent of the main InSight cruise stage with the lander.}
\item
  \emph{Laser RetroReflector for InSight (LaRRI) is a corner cube
  retroreflector provided by the Italian Space Agency and mounted on
  InSight's top deck.}
\item
  \emph{The Instrument Deployment Camera (IDC) is a color camera based
  on the Mars Exploration Rover and Mars Science Laboratory navcam
  design.}
\end{itemize}

InSight's lander payload has a total mass of 50~kg, including science
instruments and support systems such as the Auxiliary Payload Sensor
Suite, cameras, the instrument deployment system, and a laser
retroreflector.

InSight performs three major experiments using SEIS, HP3 and RISE. SEIS
is a very sensitive seismometer, measuring vibrations; HP3 involves a
burrowing probe to measure the thermal properties of the subsurface.
RISE uses the radio communication equipment on the lander and on Earth
to measure the overall movement of planet Mars that could reveal the
size and density of its core.

The Seismic Experiment for Interior Structure (SEIS) is measuring
marsquakes and other internal activity on Mars, and the response to
meteorite impacts, to better understand the planet's history and
structure. SEIS was provided by the French Space Agency (CNES), with the
participation of the Institut de Physique du Globe de Paris (IPGP), the
Swiss Federal Institute of Technology (ETH), the Max Planck Institute
for Solar System Research (MPS), Imperial College, Institut supérieur de
l'aéronautique et de l'espace (ISAE) and JPL. The seismometer will also
detect sources including atmospheric waves and tidal forces from Mars'
moon Phobos. A leak in SEIS in 2016 had forced a two-year mission
postponement. The SEIS instrument is supported by meteorological tools
including a vector magnetometer provided by UCLA that will measure
magnetic disturbances, air temperature, wind speed and wind direction
sensors based on the Spanish/Finnish Rover Environmental Monitoring
Station; and a barometer from JPL.

The Heat Flow and Physical Properties Package (HP3), provided by the
German Aerospace Center (DLR), is a self-penetrating heat flow probe.
Referred to as a "self-hammering nail" and nicknamed "the mole", it was
designed to burrow as deep as 5~m (16~ft) below the Martian surface
while trailing a tether with embedded heat sensors to measure how
efficiently heat flows through Mars' core, and thus reveal unique
information about the planet's interior and how it has evolved over
time. It trails a tether containing precise temperature sensors every
10~cm (3.9~in) to measure the temperature profile of the subsurface. The
tractor mole of the instrument was provided by the Polish company
Astronika.

The Rotation and Interior Structure Experiment (RISE) led by the Jet
Propulsion Laboratory (JPL), is a radio science experiment that will use
the lander's X band radio to provide precise measurements of planetary
rotation to better understand the interior of Mars. X band radio
tracking, capable of an accuracy under 2~cm, will build on previous
Viking program and Mars Pathfinder data. The previous data allowed the
core size to be estimated, but with more data from InSight, the nutation
amplitude can be determined. Once spin axis direction, precession, and
nutation amplitudes are better understood, it should be possible to
calculate the size and density of the Martian core and mantle. This
should increase the understanding of the formation of terrestrial
planets (e.g. Earth) and rocky exoplanets.

Temperature and Winds for InSight (TWINS), fabricated by the Spanish
Astrobiology Center, will monitor weather at the landing site.

Laser RetroReflector for InSight (LaRRI) is a corner cube retroreflector
provided by the Italian Space Agency and mounted on InSight's top deck.
It will enable passive laser range-finding by orbiters even after the
lander is retired, and would function as a node in a proposed Mars
geophysical network. This device previously flew on the Schiaparelli
lander as the Instrument for Landing-Roving Laser Retroreflector
Investigations (INRRI), and is an aluminum dome 54~mm (2.1~in) in
diameter and 25~g (0.9~oz) in mass featuring eight fused silica
reflectors.

Instrument Deployment Arm (IDA) is a 2.4~m robotic arm that will be used
to deploy the SEIS and HP3 instruments to Mars' surface. It also
features the IDC camera.

The Instrument Deployment Camera (IDC) is a color camera based on the
Mars Exploration Rover and Mars Science Laboratory navcam design. It is
mounted on the Instrument Deployment Arm and will image the instruments
on the lander's deck and provide stereoscopic views of the terrain
surrounding the landing site. It features a 45-degree field of view and
uses a 1024~×~1024 pixel CCD detector. The IDC sensor was originally
black and white for best resolution; a program was enacted that tested
with a standard hazcam and, since development deadlines and budgets were
met, it was replaced with a color sensor.

The Instrument Context Camera (ICC) is a color camera based on the
MER/MSL hazcam design. It is mounted below the lander's deck, and with
its wide-angle 120-degree panoramic field of view will provide a
complementary view of the instrument deployment area. Like the IDC, it
uses a 1024~×~1024 pixel CCD detector.

The two relay 6U cubesats were part of the overall InSight program, and
were launched at the same time as the lander but they were attached to
the centaur upper stage (InSight's second stage in the launch). They
were ejected from the stage after launch and coasted to Mars independent
of the main InSight cruise stage with the lander.

\section{Journey to Mars}\label{journey-to-mars}

\includegraphics[width=5.50000in,height=3.68304in]{media/image9.jpg}\\
\emph{Launch of the Atlas V rocket carrying InSight and MarCO from
Vandenberg Space Launch Complex 3-E.}

\section{Launch}\label{launch}

\begin{itemize}
\item
  \emph{The launch was managed by NASA's Launch Services Program.}
\item
  \emph{The lander was launched on 5 May 2018 and arrived on Mars at
  approximately 19:54 UTC on 26 November 2018.}
\item
  \emph{InSight in a Payload fairing}
\item
  \emph{The rescheduled launch window ran from 5 May to 8 June 2018.}
\end{itemize}

On 28 February 2018, InSight was shipped via C-17 cargo aircraft from
the Lockheed Martin Space Systems building in Denver to the Vandenberg
Air Force Base in California in order to be integrated to the launch
vehicle. The lander was launched on 5 May 2018 and arrived on Mars at
approximately 19:54 UTC on 26 November 2018.

The spacecraft was launched on 5 May 2018 at 11:05~UTC on an Atlas V 401
launch vehicle (AV-078) from Vandenberg Air Force Base Space Launch
Complex 3-East. This was the first American interplanetary mission to
launch from California.

The launch was managed by NASA's Launch Services Program. InSight was
originally scheduled for launch on 4 March 2016 on an Atlas V 401 (4
meter fairing/zero (0) solid rocket boosters/single (1) engine Centaur)
from Vandenberg Air Force Base in California, U.S., but was called off
in December 2015 due to a vacuum leak on the SEIS instrument. The
rescheduled launch window ran from 5 May to 8 June 2018.

Major components of the launch vehicle include:

Common Core Booster

This launch did not use additional solid rocket boosters

Centaur with Relay cubsats

InSight in a Payload fairing

The journey to Mars took 6.5 months across 484~million~km
(301~million~mi) for a touchdown on 26 November. After a successful
landing, a three-month-long deployment phase commenced as part of its
two-year (a little more than one Martian year) prime mission.

\section{Cruise}\label{cruise}

\begin{itemize}
\item
  \emph{The MarCo probes were ejected from the 2nd stage Centaur booster
  and traveled to Mars independent of the InSight cruise stage, but they
  were all launched together}
\item
  \emph{During the cruise to Mars, the InSight cruise stage made several
  course adjustments, and the first of these (TCM-1) took place on May
  22, 2018.}
\item
  \emph{The thrusters are actually on the InSight lander itself, but
  there are cutouts in the shell so the relevant rockets can vent into
  space.}
\end{itemize}

After its launch from Earth on the 5th of May in 2018, it coasted
through interplanetary space for 6.5 months traveling across
484~million~km (301~million~mi) for a touchdown on the 26th November in
that year.

InSight cruise stage departed Earth at a speed of 6,200 miles per hour
(10,000 kilometers per hour). The MarCo probes were ejected from the 2nd
stage Centaur booster and traveled to Mars independent of the InSight
cruise stage, but they were all launched together

During the cruise to Mars, the InSight cruise stage made several course
adjustments, and the first of these (TCM-1) took place on May 22, 2018.
The cruise stage that carries the lander includes solar panels, antenna,
star trackers, sun sensor, inertial measurement unit among its
technologies. The thrusters are actually on the InSight lander itself,
but there are cutouts in the shell so the relevant rockets can vent into
space.

The final course correction was November 25, 2018, the day before its
touch down. A few hours before making contact with the Martian
atmosphere, the cruise stage was jettisoned, on 26 November 2018.

\section{Entry, Descent, and Landing}\label{entry-descent-and-landing}

\begin{itemize}
\item
  \emph{There are three major stages to InSight's landing:}
\item
  \emph{On 26 November 2018 InSight successfully touched down in Elysium
  Plantia.}
\item
  \emph{A few hours after landing, NASA's 2001 Mars Odyssey orbiter
  relayed signals indicating that InSight's solar panels had
  successfully unfurled and are generating enough electrical power to
  recharge its batteries daily.}
\item
  \emph{It will then begin its mission of observing Mars, which is
  planned to last for two years.}
\end{itemize}

On 26 November 2018, at approximately 19:53 UTC, mission controllers
received a signal via the Mars Cube One (MarCO) satellites that the
spacecraft had successfully touched down at Elysium Planitia. After
landing, the mission will take three months to deploy and commission the
geophysical science instruments. It will then begin its mission of
observing Mars, which is planned to last for two years.

The mass that entered the atmosphere of Mars was 1,340 pounds (608
kilograms) .

There are three major stages to InSight's landing:

Entry: after separating from the cruise stage the aeroshell enters the
atmosphere and is subject to air and dust in the Martian atmosphere

Parachute descent: a certain speed and altitude a parachute is deployed
to slow the lander further

Rocket descent: closer to the ground the parachute is ejected and the
lander uses rocket engines to slow the lander before touchdown

Landing sequence:

25 November 2018 final course correction before EDL

26 November 2018 Cruise stage jettisoned before entering atmosphere

Several minutes later aeroshell containing lander makes contact with
upper Martian atmosphere at 12,300 mph (5.5 kilometers per second).\\
At this point it is 80 miles (about 128 kilometers) above Mars and in
the next few minutes it lands, but undergoes many stages.{[}101{]}

Aeroshell is heated to 2,700 degrees Fahrenheit (1,500 degrees Celsius)
during descent, Martian atmosphere is used to slow down.

At 861 mph (385 meters per second) and \textasciitilde{}36,400 feet
(11,100 meters) above the surface, the parachute is deployed

Several seconds later the heat shield is jettisoned from the lander

The landing legs extended

Landing radar activated

Backshell jettisoned at speed of about 134 mph (60 meters per second)
and 3,600 feet (1,100 meters) altitude

Landing rockets turned on

Roughly 164 feet (50 meters) from the ground constant velocity mode is
entered

Approaches ground at about 5 mph (2 meters per second)

Touchdown---each of the three lander legs have a sensor to detect ground
contact

Descent rockets are turned off at touchdown

Begin surface operations

The lander's mass is about 358~kg (789~lb) but on Mars, which has 0.376
of Earth's gravity, it will only weigh the equivalent of a 135~kg
(298~lb) object on Earth.

On 26 November 2018 InSight successfully touched down in Elysium
Plantia.

A few hours after landing, NASA's 2001 Mars Odyssey orbiter relayed
signals indicating that InSight's solar panels had successfully unfurled
and are generating enough electrical power to recharge its batteries
daily. Odyssey also relayed a pair of images showing InSight's landing
site. More images would be taken in stereo pairs to create 3D images,
allowing InSight to find the best locations on the surface to place the
heat probe and seismometer. Over the next few weeks, InSight would check
health indicators and monitor both weather and temperature conditions at
the landing site.

\includegraphics[width=5.50000in,height=0.51333in]{media/image10.jpg}\\
\emph{InSight Lander - panorama (9 December 2018)}

\section{Landing site}\label{landing-site}

\begin{itemize}
\item
  \emph{On 26 November 2018 the spacecraft successfully touched down at
  its landing site, and in early December 2018 InSight lander and EDL
  components were imaged from space on the surface of Mars.}
\item
  \emph{As InSight's science goals are not related to any particular
  surface feature of Mars, potential landing sites were chosen on the
  basis of practicality.}
\end{itemize}

As InSight's science goals are not related to any particular surface
feature of Mars, potential landing sites were chosen on the basis of
practicality. Candidate sites needed to be near the equator of Mars to
provide sufficient sunlight for the solar panels year round, have a low
elevation to allow for sufficient atmospheric braking during EDL, flat,
relatively rock-free to reduce the probability of complications during
landing, and soft enough terrain to allow the heat flow probe to
penetrate well into the ground.

An optimal area that meets all these requirements is Elysium Planitia,
so all 22 initial potential landing sites were located in this area. The
only two other areas on the equator and at low elevation, Isidis
Planitia and Valles Marineris, are too rocky. In addition, Valles
Marineris has too steep a gradient to allow safe landing.\\
In September 2013, the initial 22 potential landing sites were narrowed
down to four, and the Mars Reconnaissance Orbiter was then used to gain
more information on each of the four potential sites before a final
decision was made. Each site consists of a landing ellipse that measures
about 130 by 27~km (81 by 17~mi).

In March 2017, scientists from the Jet Propulsion Laboratory announced
that the landing site had been selected. It is located in western
Elysium Planitia at 4°30′N 135°54′E / 4.5°N 135.9°E / 4.5; 135.9
(InSight landing site). The landing site is about 600~km (370~mi) north
from where the Curiosity rover is operating in Gale Crater.

On 26 November 2018 the spacecraft successfully touched down at its
landing site, and in early December 2018 InSight lander and EDL
components were imaged from space on the surface of Mars. The images
provided precise position of the lander: 4°30′09″N 135°37′24″E /
4.5024°N 135.6234°E / 4.5024; 135.6234.

\section{Surface operations}\label{surface-operations}

\begin{itemize}
\item
  \emph{In April 2019, NASA reported that the Mars InSight lander
  detected its first Marsquake.}
\item
  \emph{On 26 November 2018, NASA reported that the InSight lander had
  landed successfully on Mars.}
\item
  \emph{On 19 December 2018, the SEIS instrument was deployed onto the
  surface of Mars next to the lander by its robotic arm, and it was
  commissioned on 4 February 2019.}
\end{itemize}

On 26 November 2018, NASA reported that the InSight lander had landed
successfully on Mars. The meteorological suite (TWINS) and magnetometer
are operational, and the mission will take up to three months to deploy
and commission the geophysical science instruments. One of the first
critical tasks was to unfurl the solar panels for the batteries to be
recharged. After landing, the dust was allowed to settle for a few
hours, time during which the solar array motors were warmed up and then
the solar panels were unfurled. The lander then reported its systems'
status, acquired some images, and it powered down to sleep mode for its
first night on Mars. On its first sol on Mars it set a new solar power
record of 4.6 kilowatt-hours generated for a single Martian day (known
as a "sol"). This amount is enough to support operations and deploy the
sensors.

On December 7, 2018 InSight recorded the sounds of Martian winds with
SEIS, which is able to record vibrations within human hearing range,
although rather low (aka subwoofer-type sounds), and these were sent
back to Earth. This was the first time the sound of Mars wind was heard
after two previous attempts.

On 19 December 2018, the SEIS instrument was deployed onto the surface
of Mars next to the lander by its robotic arm, and it was commissioned
on 4 February 2019. After the seismometer became fully operational, the
heat probe instrument was deployed on 12 February 2019.

On 28 February 2019, the Heat and Physical Properties Package probe
(HP³) started its drilling into the surface of Mars. The probe and its
digging mole were intended to reach a maximum depth of 5~m (16~ft) about
two months after.

On 7 March 2019, the HP³ instrument's mole paused its digging. The mole
had made it about 30~cm (12~in) or three quarters of the way out of its
housing structure and into the ground. Engineers think the mole hit a
rock which caused it to make little progress since 2 March 2019. Both
NASA and JPL continue to look into the cause of the under-performing
tool and for potential solutions. Scientifically useful measurements are
possible at a depth of 3~m (9.8~ft).

In April 2019, NASA reported that the Mars InSight lander detected its
first Marsquake.

\section{MarCO spacecraft}\label{marco-spacecraft}

\begin{itemize}
\item
  \emph{They did not enter orbit, but flew past Mars during the EDL
  phase of the mission and relayed InSight's telemetry in real time.}
\item
  \emph{They were launched along with InSight, but separated soon after
  reaching space, and they flew as a pair for redundancy while flanking
  the lander.}
\item
  \emph{The Mars Cube One (MarCO) spacecraft are a pair of 6U CubeSats
  that piggybacked with the InSight mission to test CubeSat navigation
  and endurance in deep space, and to help relay real-time
  communications (with an eight minute lightspeed delay) during the
  probe's entry, descent and landing (EDL) phase.}
\end{itemize}

The Mars Cube One (MarCO) spacecraft are a pair of 6U CubeSats that
piggybacked with the InSight mission to test CubeSat navigation and
endurance in deep space, and to help relay real-time communications
(with an eight minute lightspeed delay) during the probe's entry,
descent and landing (EDL) phase. The two 6U CubeSats, named MarCO A and
B, are identical. They were launched along with InSight, but separated
soon after reaching space, and they flew as a pair for redundancy while
flanking the lander. They did not enter orbit, but flew past Mars during
the EDL phase of the mission and relayed InSight's telemetry in real
time. On 5 February 2019, NASA reported that the CubeSats went silent,
and are unlikely to be heard from again.

Mass: 13.5~kg (30~lb) each.

Dimensions: 30~cm ×~20~cm ×~10~cm (11.8~in ×~7.9~in ×~3.9~in) each

Each has a reflectarray high gain antenna

Miniaturized radio operating in UHF (receive only) and X-band (receive
and transmit).

They carry a miniature wide-angle camera.

Cold gas thrusters for attitude adjustments.

Star tracker for navigation.

\includegraphics[width=5.50000in,height=3.55566in]{media/image11.jpg}\\
\emph{Insight team at JPL}

\includegraphics[width=5.50000in,height=4.13741in]{media/image12.jpg}\\
\emph{NASA team cheers as the InSight Lander touches down on Mars. (26
November 2018)}

\section{Team and participation}\label{team-and-participation}

\begin{itemize}
\item
  \emph{Mars Exploration Rover project scientist W. Bruce Banerdt is the
  principal investigator for the InSight mission and the lead scientist
  for the SEIS instrument.}
\item
  \emph{The InSight science and engineering team includes scientists and
  engineers from many disciplines, countries and organizations.}
\item
  \emph{National Aeronautics and Space Administration (NASA)}
\end{itemize}

The InSight science and engineering team includes scientists and
engineers from many disciplines, countries and organizations. The
science team assigned to InSight includes scientists from institutions
in the U.S., France, Germany, Austria, Belgium, Canada, Japan,
Switzerland, Spain, Poland and the United Kingdom.

Mars Exploration Rover project scientist W. Bruce Banerdt is the
principal investigator for the InSight mission and the lead scientist
for the SEIS instrument. Suzanne Smrekar, whose research focuses on the
thermal evolution of planets and who has done extensive testing and
development on instruments designed to measure the thermal properties
and heat flow on other planets, is the lead for InSight's HP3
instrument. The Principal Investigator for RISE is William Folkner at
JPL. The InSight mission team also includes project manager Tom Hoffman
and deputy project manager Henry Stone. Major contributing agencies and
institutions are:

National Aeronautics and Space Administration (NASA)

Centre National d'Études Spatiales (CNES)

German Aerospace Center (DLR)

Italian Space Agency (ASI)

Jet Propulsion Laboratory (NASA/JPL)

Lockheed Martin

Paris Institute of Earth Physics (IPGP)

Swiss Federal Institute of Technology in Zurich (ETHZ)

Max Planck Institute for Solar System Research (MPS)

Imperial College London

Institut supérieur de l'aéronautique et de l'espace (ISAE-SUPAERO)

University of Oxford

Spanish Astrobiology Center (CAB)

Space Research Centre of Polish Academy of Sciences (CBK)

\section{Name chips}\label{name-chips}

\begin{itemize}
\item
  \emph{The first chip was installed on the lander in November 2015 and
  the second on 23 January 2018.}
\item
  \emph{As part of its public outreach, NASA organized a program where
  members of the public were able to have their names sent to Mars
  aboard InSight.}
\end{itemize}

As part of its public outreach, NASA organized a program where members
of the public were able to have their names sent to Mars aboard InSight.
Due to its launch delay, two rounds of sign-ups were conducted totaling
2.4~million names: 826,923 names were registered in 2015 and a further
1.6~million names were added in 2017. An electron beam was used to etch
letters only ​1⁄1000 the width of a human hair onto 8~mm (0.3~in)
silicon wafers. The first chip was installed on the lander in November
2015 and the second on 23 January 2018.

\section{Gallery}\label{gallery}

\begin{itemize}
\item
  \emph{Entry, Descent, and Landing sequence for InSight}
\item
  \emph{InSight landing zone target with other NASA landing zones}
\item
  \emph{Global view of Mars, InSight landed in Elysium Plantia,
  Curiosity rover is in Gale crater}
\item
  \emph{InSight lander loaded on a Boeing C-17 Globemaster III (December
  2015)}
\end{itemize}

InSight lander loaded on a Boeing C-17 Globemaster III (December 2015)

InSight landing zone target with other NASA landing zones

Global view of Mars, InSight landed in Elysium Plantia, Curiosity rover
is in Gale crater

Entry, Descent, and Landing sequence for InSight

\section{Context map}\label{context-map}

\section{See also}\label{see-also}

\begin{itemize}
\item
  \emph{List of missions to Mars}
\item
  \emph{Exploration of Mars}
\end{itemize}

Exploration of Mars

List of missions to Mars

\section{References}\label{references}

\section{External links}\label{external-links}

\begin{itemize}
\item
  \emph{InSight NASA~-- InSight Raw Images}
\item
  \emph{InSight NASA (video/01:38; 26 November 2018; Landing)}
\item
  \emph{InSight NASA (video/01:39; 01 December 2018; Wind Sounds)}
\item
  \emph{InSight NASA (video/03:31; 18 November 2018; Details)}
\item
  \emph{InSight NASA}
\item
  \emph{InSight NASA -- Mars Exploration Program}
\end{itemize}

InSight NASA

InSight NASA -- Mars Exploration Program

InSight NASA~-- InSight Raw Images

InSight NASA (video/03:31; 18 November 2018; Details)

InSight NASA (video/01:38; 26 November 2018; Landing)

InSight NASA (video/01:39; 01 December 2018; Wind Sounds)
