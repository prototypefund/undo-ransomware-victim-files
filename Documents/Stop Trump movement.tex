\textbf{From Wikipedia, the free encyclopedia}

https://en.wikipedia.org/wiki/Stop\%20Trump\%20movement\\
Licensed under CC BY-SA 3.0:\\
https://en.wikipedia.org/wiki/Wikipedia:Text\_of\_Creative\_Commons\_Attribution-ShareAlike\_3.0\_Unported\_License

\section{Stop Trump movement}\label{stop-trump-movement}

\begin{itemize}
\item
  \emph{Following Trump's election in November 2016, some in the
  movement refocused their efforts on defeating Trump in 2020.}
\item
  \emph{Following unsuccessful attempts by some delegates at the
  Republican National Convention to block his nomination, Trump became
  the Republican Party's 2016 nominee for President on July 18, 2016.}
\end{itemize}

The Stop Trump movement, also called the anti-Trump, Dump Trump, or
Never Trump movement, was an effort on the part of some Republicans and
other prominent conservatives to prevent front-runner and now President
of the United States Donald Trump from obtaining the Republican Party
presidential nomination and following his nomination the presidency for
the 2016 United States presidential election. Although Trump's campaign
drew a substantial amount of criticism, Trump received 90\% of the
Republican vote, while Clinton won 89\% of Democratic voters. Moreover,
Trump was supported by 80\% of Republican members of Congress in the
general election. Following Trump's election in November 2016, some in
the movement refocused their efforts on defeating Trump in 2020.

Trump entered the Republican primaries on June 16, 2015, at a time when
Governors Jeb Bush and Scott Walker and Senator Marco Rubio were viewed
as the early frontrunners. Trump was considered a longshot to win the
nomination, but his large media profile gave him a chance to spread his
message and appear in the Republican debates. By the end of 2015, Trump
was leading the Republican field in national polls. At this point, some
Republicans, such as former Mitt Romney adviser Alex Castellanos, called
for a "negative ad blitz" against Trump and another former Romney aide
founded Our Principles PAC to attack Trump. After Trump won the New
Hampshire and South Carolina primaries, many Republican leaders called
for the party to unite around a single leader to stop Trump's
nomination. The Never Trump movement gained momentum following Trump's
wins in the March 15, 2016, Super Tuesday primaries, including his
victory over Senator Marco Rubio in Florida. After Senator Ted Cruz
dropped out of the race following Trump's primary victory in Indiana on
May 3, 2016, Trump became the presumptive nominee while internal
opposition to Trump remained as the process pivoted towards a general
election. Following unsuccessful attempts by some delegates at the
Republican National Convention to block his nomination, Trump became the
Republican Party's 2016 nominee for President on July 18, 2016. Some
members of the Never Trump movement endorsed alternative candidates in
the general election, such as Democratic nominee Hillary Clinton,
Libertarian nominee Gary Johnson, independent conservative Evan McMullin
and American Solidarity Party nominee Mike Maturen.

Research on the Never Trump movement shows that Mormon and female
Republicans were the most likely groups to oppose Trump's candidacy
while non-Mormon and male Republicans were the most supportive.

\section{Erickson meeting}\label{erickson-meeting}

\begin{itemize}
\item
  \emph{On March 17, 2016, notable conservatives under the leadership of
  Erick Erickson met at the Army and Navy Club in Washington D.C. to
  discuss strategies for preventing Trump from securing the presidential
  nomination at the Republican National Convention in July.}
\item
  \emph{Among the strategies discussed were a "unity ticket", a possible
  third-party candidate and a contested convention, especially if Trump
  did not gain the 1,237 delegates necessary to secure the nomination.}
\end{itemize}

On March 17, 2016, notable conservatives under the leadership of Erick
Erickson met at the Army and Navy Club in Washington D.C. to discuss
strategies for preventing Trump from securing the presidential
nomination at the Republican National Convention in July. Among the
strategies discussed were a "unity ticket", a possible third-party
candidate and a contested convention, especially if Trump did not gain
the 1,237 delegates necessary to secure the nomination.

The meeting was organized by Erick Erickson, Bill Wichterman and Bob
Fischer. Around two dozen people attended. Consensus was reached that
Trump's nomination could be prevented and that efforts would be made to
seek a unity ticket, possibly comprising Senator Ted Cruz and Ohio
Governor John Kasich.

\section{Efforts}\label{efforts}

\section{By political organizations}\label{by-political-organizations}

\begin{itemize}
\item
  \emph{Our Principles PAC and Club for Growth were involved in trying
  to prevent Trump's nomination.}
\item
  \emph{The Club for Growth spent \$11 million in an effort to prevent
  Trump from becoming the Republican Party's nominee.}
\item
  \emph{Our Principles PAC spent more than \$13 million on advertising
  attacking Trump.}
\end{itemize}

Our Principles PAC and Club for Growth were involved in trying to
prevent Trump's nomination. Our Principles PAC spent more than \$13
million on advertising attacking Trump. The Club for Growth spent \$11
million in an effort to prevent Trump from becoming the Republican
Party's nominee.

\section{By Republican delegates}\label{by-republican-delegates}

\begin{itemize}
\item
  \emph{Republican delegates Kendal Unruh and Steve Lonegan led an
  effort among fellow Republican delegates to change the convention
  rules "to include a 'conscience clause' that would allow delegates
  bound to Trump to vote against him, even on the first ballot at the
  July convention".}
\item
  \emph{However, the Rules Committee voted down by a vote of 84--21, a
  move to send a "minority report" to the floor allowing the unbinding
  of delegates, thereby defeating the Stop Trump activists and
  guaranteeing Trump's nomination.}
\item
  \emph{Unruh described the effort as "an 'Anybody but Trump'
  movement".}
\end{itemize}

In June 2016, activists Eric O'Keefe and Dane Waters formed a group
called Delegates Unbound, which CNN described as "an effort to convince
delegates that they have the authority and the ability to vote for
whomever they want". The effort involved the publication of a book
titled Unbound: The Conscience of a Republican Delegate by Republican
delegates Curly Haugland and Sean Parnell. The book argues that
"delegates are not bound to vote for any particular candidate based on
primary and caucus results, state party rules, or even state law".

Republican delegates Kendal Unruh and Steve Lonegan led an effort among
fellow Republican delegates to change the convention rules "to include a
'conscience clause' that would allow delegates bound to Trump to vote
against him, even on the first ballot at the July convention". Unruh
described the effort as "an 'Anybody but Trump' movement". According to
The Washington Post, Unruh's efforts started with a conference call on
June 16 "with at least 30 delegates from 15 states". Regional
coordinators for the effort were recruited in Arizona, Iowa, Louisiana,
Washington and other states. By June 19, hundreds of delegates to the
Republican National Convention calling themselves Free the Delegates,
had begun raising funds and recruiting members in support of an effort
to change party convention rules to free delegates to vote however they
want, instead of according to the results of state caucuses and
primaries. Unruh, a member of the convention's Rules Committee and one
of the group's founders, planned to propose adding the "conscience
clause" to the convention's rules effectively unhinging pledged
delegates. She needed 56 other supporters from the 112-member panel,
which determines precisely how Republicans select their nominee in
Cleveland. However, the Rules Committee voted down by a vote of 84--21,
a move to send a "minority report" to the floor allowing the unbinding
of delegates, thereby defeating the Stop Trump activists and
guaranteeing Trump's nomination. The committee then endorsed the
opposite option, voting 87--12 to include rules language specifically
stating that delegates were required to vote based on their states'
primary and caucus results.

\section{By individuals}\label{by-individuals}

\begin{itemize}
\item
  \emph{In October 2016, some individuals made third-party vote trading
  mobile applications and websites to help stop Trump.}
\item
  \emph{After Trump became the presumptive nominee in May, Graham
  announced he would not be supporting Trump in the general election,
  stating: "{[}I{]} cannot, in good conscience, support Donald Trump
  because I do not believe he is a reliable Republican conservative nor
  has he displayed the judgment and temperament to serve as Commander in
  Chief".}
\end{itemize}

At a luncheon in February 2016 attended by Republican governors and
donors, Karl Rove discussed the danger of Trump securing the Republican
nomination in July and that it may be possible to stop him, but that
there was not much time left.

Early in March 2016, Mitt Romney, the 2012 Republican presidential
nominee, directed some of his advisors to look at ways to stop Trump
from obtaining the nomination at the Republican National Convention
(RNC). Romney also gave a major speech urging voters to vote for the
Republican candidate most likely to prevent Trump from acquiring
delegates in state primaries. A few weeks later, Romney announced that
he would vote for Ted Cruz in the Utah GOP caucuses. On his Facebook
page, he posted: "Today, there is a contest between Trumpism and
Republicanism. Through the calculated statements of its leader, Trumpism
has become associated with racism, misogyny, bigotry, xenophobia,
vulgarity and, most recently, threats and violence. I am repulsed by
each and every one of these". Nevertheless, Romney said early on he
would "support the Republican nominee", though he did not "think that's
going to be Donald Trump".

Senator Lindsey Graham shifted from opposing both Ted Cruz and Donald
Trump to eventually supporting Cruz as a better alternative to Trump.
Commenting about Trump, Graham said: "I don't think he's a Republican, I
don't think he's a conservative, I think his campaign's built on
xenophobia, race-baiting and religious bigotry. I think he'd be a
disaster for our party and as Senator Cruz would not be my first choice,
I think he is a Republican conservative who I could support". After
Trump became the presumptive nominee in May, Graham announced he would
not be supporting Trump in the general election, stating: "{[}I{]}
cannot, in good conscience, support Donald Trump because I do not
believe he is a reliable Republican conservative nor has he displayed
the judgment and temperament to serve as Commander in Chief".

In October 2016, some individuals made third-party vote trading mobile
applications and websites to help stop Trump. For example, a Californian
that wants to vote for Clinton will instead vote for Jill Stein and in
exchange a Stein supporter in a swing state will vote for Clinton. The
Ninth Circuit Court of Appeals in the 2007 case Porter v. Bowen
established vote trading as a First Amendment right.

Former Republican Presidents George H.W. Bush and George W. Bush both
refused to support Trump in the general election.

\section{Republicans who left the party in opposition to the Trump
administration}\label{republicans-who-left-the-party-in-opposition-to-the-trump-administration}

\begin{itemize}
\item
  \emph{Several prominent Republicans have left the party in opposition
  to actions taken by the Trump administration.}
\end{itemize}

Several prominent Republicans have left the party in opposition to
actions taken by the Trump administration.

Joe Scarborough (host of Morning Joe)

George Will (conservative columnist)

Max Boot (conservative columnist)

Richard Painter (Bush ethics lawyer)

Steve Schmidt (Republican Party strategist and top George W. Bush aide)

Jennifer Rubin (author of the Right Turn blog for The Washington Post)

\section{General election opposition}\label{general-election-opposition}

\begin{itemize}
\item
  \emph{On May 3, 2016, one of the biggest anti-Trump groups, the Never
  Trump PAC, circulated a petition to collect the signatures of
  conservatives opposed to voting for Donald Trump in the 2016
  presidential election.}
\item
  \emph{The grassroots effort, called Republicans for Clinton in 2016,
  or R4C16, also joined the effort in defeating Trump.}
\end{itemize}

Trump was widely described as the presumptive Republican nominee after
the May 3 Indiana primary, notwithstanding the continued opposition of
groups such as Our Principles PAC. Many Republican leaders endorsed
Trump after he became the presumptive nominee, but other Republicans
looked for ways to defeat him in the general election. Stop Trump
members such as Mitt Romney, Eric Erickson, William Kristol, Mike
Murphy, Stuart Stevens and Rick Wilson pursued the possibility of an
independent candidacy by a non-Trump Republican. Potential candidates
included Senator Ben Sasse, Governor John Kasich, Senator Tom Coburn,
Congressman Justin Amash, Senator Rand Paul, retired Marine Corps
General James Mattis, retired Army General Stanley McChrystal, former
Secretary of State Condoleezza Rice, businessman Mark Cuban and 2012
Republican nominee Mitt Romney. However, many of these candidates
rejected the possibility of an independent run, pointing to difficulties
such as ballot access and the potential to help the Democratic candidate
win the presidency. One potential strategy would involve an independent
candidate gaining enough electoral votes to deny a majority to either of
the major party candidates, sending the three presidential candidates
with the most electoral votes to the House of Representatives under
procedures established by the Twelfth Amendment. Some anti-Trump
Republicans stated that they would vote for Hillary Clinton in the
general election.

On May 3, 2016, one of the biggest anti-Trump groups, the Never Trump
PAC, circulated a petition to collect the signatures of conservatives
opposed to voting for Donald Trump in the 2016 presidential election. As
of August 19, 2016, over 54,000 people had signed the petition. Gary
Johnson's campaign in the Libertarian Party attracted attention as a
possible vehicle for the Stop Trump movement's votes in the general
election after Trump became the Republican Party's presumptive nominee.
In late May, Craig Snyder, a former Republican staffer, launched the
Republicans for Hillary PAC, "aimed at convincing Republicans to choose
Hillary Clinton over {[}...{]} Donald Trump in November". The grassroots
effort, called Republicans for Clinton in 2016, or R4C16, also joined
the effort in defeating Trump.

William Kristol, editor of The Weekly Standard, promoted National Review
staff writer David A. French of Tennessee as a prospective candidate.
However, French opted not to run. On August 8, Evan McMullin, a
conservative Republican, announced that he would mount an independent
bid for President with support of the Never Trump movement. McMullin was
backed by Better for America (a Never Trump group) and supported by
former Americans Elect CEO Kahlil Byrd and Republican campaign-finance
lawyer Chris Ashby.

\section{Reactions}\label{reactions}

\begin{itemize}
\item
  \emph{Roger Stone, a political consultant who served as an advisor for
  Donald Trump's 2016 presidential campaign and who remains a
  "confidant" to Trump, put together a group called Stop the Steal and
  threatened "days of rage" if Republican Party leaders tried to deny
  the nomination to Trump at the Republican National Convention in
  Cleveland.}
\item
  \emph{Reactions to the Stop Trump movement were mixed, with other
  prominent Republicans making statements in support of preventing Trump
  from receiving the Republican nomination.}
\end{itemize}

Reactions to the Stop Trump movement were mixed, with other prominent
Republicans making statements in support of preventing Trump from
receiving the Republican nomination. Following his withdrawal as a
candidate for President, Senator Marco Rubio expressed hope that Trump's
nomination could be stopped, adding that his nomination "would fracture
the party and be damaging to the conservative movement".

Republican National Committee chairman Reince Priebus dismissed the
potential impact of Mitt Romney's efforts to block Trump at the
convention. Sam Clovis, a national co-chairman for Trump's campaign,
said that he would leave the Republican Party if it "comes into that
convention and jimmies with the rules and takes away the will of the
people". Ned Ryun, founder of conservative group American Majority,
expressed concern about a contested convention, should Trump have the
most delegates, but fail to reach the 1,237 necessary to be assured the
nomination. Ryun speculated that a contested convention would result in
Trump running as a third-party candidate, making it unlikely that
Republicans would win the presidency in the November general election,
adding that it would "blow up the party, at least in the short term".

New Jersey Governor Chris Christie expressed his opinion that the
efforts to stop Trump would ultimately fail. Relatively shortly after
his endorsement of Trump, he criticized the people who condemned his
endorsement, including the Stop Trump movement, stating that his critics
had yet to support any of the remaining Republican candidates, saying:
"I think if you're a public figure, you have the obligation to speak
out, and be 'for' something, not just 'against' something. {[}...{]}
When those folks in the 'Stop Trump' movement actually decide to be for
something, then people can make an evaluation {[}...{]} if they want to
be for one of the remaining candidates, do what I did: Be for one of the
remaining candidates".

Trump said that if he were deprived of the nomination because of falling
just short of the 1,237 delegates required, there could be "problems
like you've never seen before. I think bad things would happen" and "I
think you'd have riots". Trump made prior comments suggesting that he
might run as an independent candidate if he were not to get the
Republican nomination.

Roger Stone, a political consultant who served as an advisor for Donald
Trump's 2016 presidential campaign and who remains a "confidant" to
Trump, put together a group called Stop the Steal and threatened "days
of rage" if Republican Party leaders tried to deny the nomination to
Trump at the Republican National Convention in Cleveland. Stone also
threatened to disclose to the public the hotel room numbers of delegates
who opposed Trump.

\section{Developments following the
election}\label{developments-following-the-election}

\begin{itemize}
\item
  \emph{After Trump won the election, two Electoral College electors
  launched an effort to convince fellow electors who are allocated to
  Trump not to vote for him.}
\item
  \emph{The Congressman did admit that Trump won "fair and square", but
  he said that Trump proved himself unfit for public office.}
\item
  \emph{In the end, efforts to persuade more electors to vote against
  Trump ultimately failed and Trump won 304 electors on December 19.}
\end{itemize}

After Trump won the election, two Electoral College electors launched an
effort to convince fellow electors who are allocated to Trump not to
vote for him.

On December 11, Jim Himes, a member of the House of Representatives,
wrote on Twitter that the Electoral College should not elect Trump:
"We're 5 wks from Inauguration \& the President Elect is completely
unhinged. The Electoral College must do what it was designed for". In a
December 12 interview on CNN's New Day, Himes said that he was troubled
by several actions by the President-elect. The issue that "pushed me
over the edge" was Trump's criticism of the CIA and the intelligence
community. The Congressman did admit that Trump won "fair and square",
but he said that Trump proved himself unfit for public office. He cited
the intentions behind the creation of the Electoral College and argued
that it was created for an instance such as the election of Trump.

In the end, efforts to persuade more electors to vote against Trump
ultimately failed and Trump won 304 electors on December 19. Trump's
electoral lead over Clinton even grew because a larger number of
electors defected from her: Trump received 304 of his 306 pledged
electors, Clinton 227 of her 232.

In a National Review article titled "Never Trump Nevermore", Jonah
Goldberg stated:

Since the election, other Republicans who had resisted Trump's
candidacy, such as South Carolina Senator Lindsey Graham have since
declared their support for his presidency. Since taking office, Trump's
job approval among self-described Republicans has been consistently at
or near 90 percent.

\section{See also}\label{see-also}

\begin{itemize}
\item
  \emph{The Case for Impeachment (a book by Allan Lichtman arguing for
  the impeachment of Donald Trump)}
\item
  \emph{Fire and Fury (a book by Michael Wolff which details the first
  year of the Trump presidency)}
\end{itemize}

The Case for Impeachment (a book by Allan Lichtman arguing for the
impeachment of Donald Trump)

Fire and Fury (a book by Michael Wolff which details the first year of
the Trump presidency)

\section{References}\label{references}

\section{External links}\label{external-links}

\begin{itemize}
\item
  \emph{Neocons Paved the Way for Trump.}
\item
  \emph{How Neocons Helped Create Trump by Jim Lobe}
\end{itemize}

The Neocons Are Responsible for Trumpism by Michael Lind

The Iraq reckoning still to come by Andrew Bacevich

How Neocons Helped Create Trump by Jim Lobe

Neocons Paved the Way for Trump. Finally, One Admits It by Jacob
Heilbrunn
