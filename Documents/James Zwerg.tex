\textbf{From Wikipedia, the free encyclopedia}

https://en.wikipedia.org/wiki/James\%20Zwerg\\
Licensed under CC BY-SA 3.0:\\
https://en.wikipedia.org/wiki/Wikipedia:Text\_of\_Creative\_Commons\_Attribution-ShareAlike\_3.0\_Unported\_License

\section{James Zwerg}\label{james-zwerg}

\begin{itemize}
\item
  \emph{James Zwerg (born November 28, 1939) is an American former
  minister who was involved with the Freedom Riders in the early 1960s.}
\end{itemize}

James Zwerg (born November 28, 1939) is an American former minister who
was involved with the Freedom Riders in the early 1960s.

\section{Early life}\label{early-life}

\begin{itemize}
\item
  \emph{Zwerg was very involved in school and took part in the student
  protests in high school.}
\item
  \emph{Zwerg was born in Appleton, Wisconsin where he lived with his
  parents and older brother, Charles.}
\item
  \emph{Zwerg was also very active in the Christian church, where he
  attended services regularly.}
\item
  \emph{Jim was one of only a handful of white Christians to join the
  nonviolent movement.}
\end{itemize}

Zwerg was born in Appleton, Wisconsin where he lived with his parents
and older brother, Charles. His father was a dentist who provided free
dental care to the poor on one day per month. Zwerg was very involved in
school and took part in the student protests in high school.

Zwerg was also very active in the Christian church, where he attended
services regularly. Through the church, he became exposed to the belief
in civil equality. He was taught that all people are created equal, no
matter what race or religion they are. Jim was one of only a handful of
white Christians to join the nonviolent movement.

\section{College and SNCC}\label{college-and-sncc}

\begin{itemize}
\item
  \emph{Zwerg joined SNCC and suggested that the group attend a movie.}
\item
  \emph{SNCC members explained to Zwerg that Nashville theaters were
  segregated.}
\item
  \emph{At Fisk, Zwerg met John Lewis, who was active in the Civil
  Rights Movement, and was immediately impressed with the way Lewis
  handled himself and his commitment to the movement.}
\end{itemize}

Zwerg attended Beloit College, where he studied sociology. He developed
an interest in civil rights from his interactions with his roommate,
Robert Carter, an African-American from Alabama.\\
Zwerg recalls: " I witnessed prejudice against him\ldots{} we would go
to a lunch counter or cafeteria and people would get up and leave the
table. I had pledged a particular fraternity and then found out that he
was not allowed in the fraternity house. I decided that his friendship
was more important than that particular fraternity, so I depledged." He
participated in a one-semester student exchange program in January 1961
at Nashville's Fisk University, a predominantly black school. At Fisk,
Zwerg met John Lewis, who was active in the Civil Rights Movement, and
was immediately impressed with the way Lewis handled himself and his
commitment to the movement. Lewis was a member of the Student Nonviolent
Coordinating Committee (SNCC), a student organized Civil Rights activist
group focused on nonviolent direct action. Zwerg joined SNCC and
suggested that the group attend a movie. SNCC members explained to Zwerg
that Nashville theaters were segregated. Zwerg began attending SNCC
nonviolence workshops, often playing the angry bigot in role-play. His
first test was to buy two movie tickets and try to walk in with a black
man. When trying to enter the theater on February 21, 1961, Zwerg was
hit with a monkey wrench and knocked unconscious.

\section{Freedom Rides}\label{freedom-rides}

\begin{itemize}
\item
  \emph{Zwerg was denied prompt medical attention because there were no
  white ambulances available.}
\item
  \emph{Zwerg was the only white male in the group.}
\item
  \emph{Zwerg recalls, "There was nothing particularly heroic in what I
  did.}
\item
  \emph{Meanwhile, at an SNCC meeting in Tennessee, Lewis, Zwerg and 11
  other volunteers decided to be reinforcements.}
\item
  \emph{"Mr. Zwerg was hit with his own suitcase in the face.}
\end{itemize}

In 1961, the Congress of Racial Equality (CORE) began to organize
Freedom Rides. The first departed from Washington, D.C. and involved 13
black and white riders who rode into the South challenging white only
lunch counters and restaurants. When they reached Anniston, Alabama one
of the buses was ambushed and attacked. Meanwhile, at an SNCC meeting in
Tennessee, Lewis, Zwerg and 11 other volunteers decided to be
reinforcements. Zwerg was the only white male in the group. Although
scared for his life, Zwerg never had second thoughts. He recalled, "My
faith was never so strong as during that time. I knew I was doing what I
should be doing."\\
The group traveled by bus to Birmingham, where Zwerg was first arrested
for not moving to the back of the bus with his black seating companion,
Paul Brooks. Three days later, the riders regrouped and headed to
Montgomery. At first the bus station there was quiet and eerie, but the
scene turned into an ambush, with the riders attacked from all
directions. "Mr. Zwerg was hit with his own suitcase in the face. Then
he was knocked down and a group pummeled him" (qtd. in Loory 577). The
prostrate activist was beaten into unconsciousness somewhere around the
time a man took Zwerg's head between his knees while others took turns
pounding and clawing at his face. At one point while Zwerg was
unconscious, three men held him up while a woman kicked him in the
groin. After it seemed that the worst of the onslaught was over, Zwerg
gained semi-consciousness and tried to use the handrails to the loading
platform to pull himself to his feet. As he struggled to get upright, a
white man came and threw Zwerg over the rail. He crashed to the ground
below, landing on his head. He was only the first to be beaten that day,
but the attack on him may have been the most ruthless (Loory 573-79).
Zwerg recalls, "There was nothing particularly heroic in what I did. If
you want to talk about heroism, consider the black man who probably
saved my life. This man in coveralls, just off of work, happened to walk
by as my beating was going on and said 'Stop beating that kid. If you
want to beat someone, beat me.' And they did. He was still unconscious
when I left the hospital. I don't know if he lived or died."

Zwerg was denied prompt medical attention because there were no white
ambulances available. "I suppose a person has to be dead before anyone
will call an ambulance in Montgomery" were Jim's words as he lay in the
hospital bed after being brutally beaten. He remained unconscious for
two days and stayed in the hospital for five days. His post-riot photos
were published in many newspapers and magazines across the country.
After his beating, Zwerg claimed he had had an incredible religious
experience and God helped him to not fight back. In a 2013 interview
recalling the incident, he said, "In that instant, I had the most
incredible religious experience of my life. I felt a presence with me. A
peace. Calmness. It was just like I was surrounded by kindness, love. I
knew in that instance that whether I lived or died, I would be OK." In a
famous moving speech from his hospital room, Zwerg stated, "Segregation
must be stopped. It must be broken down. Those of us on the Freedom Ride
will continue.... We're dedicated to this, we'll take hitting, we'll
take beating. We're willing to accept death. But we're going to keep
coming until we can ride from anywhere in the South to any place else in
the South without anybody making any comments, just as American
citizens."

\section{Post-Freedom Rides}\label{post-freedom-rides}

\begin{itemize}
\item
  \emph{After a conversation with King, Zwerg decided to enroll at
  Garrett Theological Seminary.}
\item
  \emph{Later in 1961, Martin Luther King presented Zwerg with the
  Southern Christian Leadership Conference Freedom Award.}
\item
  \emph{Zwerg continues to spread awareness to this day about the trials
  and tribulations of the Freedom Rides and how love is what is most
  important.}
\end{itemize}

Later in 1961, Martin Luther King presented Zwerg with the Southern
Christian Leadership Conference Freedom Award. After a conversation with
King, Zwerg decided to enroll at Garrett Theological Seminary. He met
his future wife Carrie. Zwerg was ordained a minister, serving for five
years in three rural Wisconsin communities. The Zwergs settled in
Tucson, Arizona in 1970 and had three children. He changed his career
several times, including charity organization work and a stint in
community relations at IBM. Zwerg retired in 1993 after which the couple
built a cabin in rural New Mexico about 50 miles (80~km) from the
nearest grocery store. Zwerg continues to spread awareness to this day
about the trials and tribulations of the Freedom Rides and how love is
what is most important. He recently did a speech on May 18, 2011 at Troy
University Rosa Parks Museum. He spoke about the effect the Freedom
Rides had on his life. In a recent interview with Lisa Simeone Jim
talked about how blessed he was to have been a part of the Movement.
"Everywhere we've stopped people have been so gracious and so kind and
one of the things that has certainly been rewarding to me has been to
see how many people brought their children; seeing a little
eight-year-old boy come up to me and talk to me and say, `May I please
have your autograph? Thank you for what you did.' That was pretty
special. I appreciated that." Zwerg remains a devoted loving Christian
to this day and what is most important to him is love. "I think the
thing I would add is love is still the most powerful force in the
universe. Hatred will never beat it. Violence will never beat it."

There is a reference of Zwerg in a scene of the 1961 Soviet film Kogda
derevya byli bolshimi (English: When the Trees Were Tall) where Inna
Gulaya is reading an article from a U.S. newspaper describing a Freedom
Riders demonstration where Zwerg: "was thrown off the bus and his face
was smashed against the hot concrete of the road."

\section{References}\label{references}

\section{External links}\label{external-links}

\begin{itemize}
\item
  \emph{Blake, John (16 May 2011).}
\end{itemize}

Blake, John (16 May 2011). "Shocking photo created a hero, but not to
his family". CNN. Retrieved 16 May 2011.

\section{Further reading}\label{further-reading}

\begin{itemize}
\item
  \emph{Freedom Riders: 1961 and the Struggle for Racial Justice.}
\end{itemize}

Arsenault, Raymond (2006). Freedom Riders: 1961 and the Struggle for
Racial Justice. Oxford University Press. ISBN~9780199755813.
