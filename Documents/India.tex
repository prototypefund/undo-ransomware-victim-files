\textbf{From Wikipedia, the free encyclopedia}

https://en.wikipedia.org/wiki/India\\
Licensed under CC BY-SA 3.0:\\
https://en.wikipedia.org/wiki/Wikipedia:Text\_of\_Creative\_Commons\_Attribution-ShareAlike\_3.0\_Unported\_License

\section{India}\label{india}

\begin{itemize}
\item
  \emph{India's modern age was marked by British Crown rule and a
  nationalist movement which, under Mahatma Gandhi, was noted for
  nonviolence and led to India's independence in 1947.}
\item
  \emph{North India fell to the Delhi Sultanate; south India was united
  under the Vijayanagara Empire.}
\item
  \emph{India (ISO: Bhārat), also known as the Republic of India (ISO:
  Bhārat Gaṇarājya), is a country in South Asia.}
\end{itemize}

India (ISO: Bhārat), also known as the Republic of India (ISO: Bhārat
Gaṇarājya), is a country in South Asia. It is the seventh-largest
country by area, the second-most populous country, and the most populous
democracy in the world. Bounded by the Indian Ocean on the south, the
Arabian Sea on the southwest, and the Bay of Bengal on the southeast, it
shares land borders with Pakistan to the west; China, Nepal, and Bhutan
to the northeast; and Bangladesh and Myanmar to the east. In the Indian
Ocean, India is in the vicinity of Sri Lanka and the Maldives; its
Andaman and Nicobar Islands share a maritime border with Thailand and
Indonesia.

The Indian subcontinent was home to the Indus Valley Civilisation of the
bronze age. In the next two millennia, the oldest scriptures of Hinduism
were composed, social stratification based on caste emerged, and
Buddhism and Jainism arose. Political consolidations took place under
the Maurya and Gupta Empires. The peninsular Middle Kingdoms influenced
the cultures of Southeast Asia. In India's medieval era, Judaism,
Zoroastrianism, Christianity, and Islam arrived, and Sikhism emerged,
adding to a diverse culture. North India fell to the Delhi Sultanate;
south India was united under the Vijayanagara Empire. In the early
modern era, the expansive Mughal Empire was followed by British East
India Company rule. India's modern age was marked by British Crown rule
and a nationalist movement which, under Mahatma Gandhi, was noted for
nonviolence and led to India's independence in 1947.

Economic liberalisation, begun in 1991, has caused India to become a
fast growing major economy and a newly industrialised country. Its gross
domestic product ranks sixth in the world in market exchange rates and
third in purchasing power parity. Its per capita income ranks 133rd and
116th in the two measures. India faces challenges of poverty,
corruption, malnutrition, and inadequate public healthcare. A nuclear
weapons state and regional power, it has the second largest active
military in the world and ranks high in military expenditure. India is a
secular, federal republic, governed in a democratic parliamentary
system, and administered in 29 states and seven union territories. A
pluralistic, multilingual and multi-ethnic society, India is home to 1.3
billion people. It is also home to a high diversity of wildlife in a
variety of protected habitats.

\section{Etymology}\label{etymology}

\begin{itemize}
\item
  \emph{Currently, the name may refer to either the northern part of
  India or the entire country.}
\item
  \emph{It was introduced into India by the Mughals and widely used
  since then.}
\item
  \emph{Hindustan ({[}ɦɪndʊˈstaːn{]} (listen)) is a Middle Persian name
  for India.}
\item
  \emph{Its meaning varied, referring to a region that encompassed
  northern India and Pakistan or India in its entirety.}
\end{itemize}

The name India is derived from Indus, which originates from the Old
Persian word Hindush, equivalent to the Sanskrit word Sindhu, which was
the historical local appellation for the Indus River. The ancient Greeks
referred to the Indians as Indoi (Ἰνδοί), which translates as "The
people of the Indus".

The geographical term Bharat (Bhārat; pronounced~{[}ˈbʱaːɾət{]}
(listen)), which is recognised by the Constitution of India as an
official name for the country, is used by many Indian languages in its
variations. It is a modernisation of the historical name Bharatavarsha,
which traditionally referred to the Indian subcontinent and gained
increasing currency from the mid-19th century as a native name for
India.

Hindustan ({[}ɦɪndʊˈstaːn{]} (listen)) is a Middle Persian name for
India. It was introduced into India by the Mughals and widely used since
then. Its meaning varied, referring to a region that encompassed
northern India and Pakistan or India in its entirety. Currently, the
name may refer to either the northern part of India or the entire
country.

\section{History}\label{history}

\includegraphics[width=5.32400in,height=5.50000in]{media/image1.jpg}\\
\emph{Paintings at the Ajanta Caves in Aurangabad, Maharashtra, 6th
century}

\section{Ancient India}\label{ancient-india}

\begin{itemize}
\item
  \emph{In South India, a progression to sedentary life is indicated by
  the large number of megalithic monuments dating from this period, as
  well as by nearby traces of agriculture, irrigation tanks, and craft
  traditions.}
\item
  \emph{In North India, Hinduism asserted patriarchal control within the
  family, leading to increased subordination of women.}
\item
  \emph{These gradually developed into the Indus Valley Civilisation,
  the first urban culture in South Asia, which flourished during
  2500--1900~BCE in what is now Pakistan and western India.}
\end{itemize}

The earliest known human remains in South Asia date to about 30,000
years ago. Nearly contemporaneous human rock art sites have been found
in many parts of the Indian subcontinent, including at the Bhimbetka
rock shelters in Madhya Pradesh. After 6500 BCE, evidence for
domestication of food crops and animals, construction of permanent
structures, and storage of agricultural surplus, appeared in Mehrgarh
and other sites in what is now Balochistan. These gradually developed
into the Indus Valley Civilisation, the first urban culture in South
Asia, which flourished during 2500--1900~BCE in what is now Pakistan and
western India. Centred around cities such as Mohenjo-daro, Harappa,
Dholavira, and Kalibangan, and relying on varied forms of subsistence,
the civilization engaged robustly in crafts production and wide-ranging
trade.

During the period 2000--500 BCE, many regions of the subcontinent
transitioned from the Chalcolithic cultures to the Iron Age ones. The
Vedas, the oldest scriptures associated with Hinduism, were composed
during this period, and historians have analysed these to posit a Vedic
culture in the Punjab region and the upper Gangetic Plain. Most
historians also consider this period to have encompassed several waves
of Indo-Aryan migration into the subcontinent from the north-west. The
caste system, which created a hierarchy of priests, warriors, and free
peasants, but which excluded indigenous peoples by labeling their
occupations impure, arose during this period. On the Deccan Plateau,
archaeological evidence from this period suggests the existence of a
chiefdom stage of political organisation. In South India, a progression
to sedentary life is indicated by the large number of megalithic
monuments dating from this period, as well as by nearby traces of
agriculture, irrigation tanks, and craft traditions.

In the late Vedic period, around the 6th century BCE, the small states
and chiefdoms of the Ganges Plain and the north-western regions had
consolidated into 16 major oligarchies and monarchies that were known as
the mahajanapadas. The emerging urbanisation gave rise to non-Vedic
religious movements, two of which became independent religions. Jainism
came into prominence during the life of its exemplar, Mahavira.
Buddhism, based on the teachings of Gautama Buddha, attracted followers
from all social classes excepting the middle class; chronicling the life
of the Buddha was central to the beginnings of recorded history in
India. In an age of increasing urban wealth, both religions held up
renunciation as an ideal, and both established long-lasting monastic
traditions. Politically, by the 3rd century BCE, the kingdom of Magadha
had annexed or reduced other states to emerge as the Mauryan Empire. The
empire was once thought to have controlled most of the subcontinent
excepting the far south, but its core regions are now thought to have
been separated by large autonomous areas. The Mauryan kings are known as
much for their empire-building and determined management of public life
as for Ashoka's renunciation of militarism and far-flung advocacy of the
Buddhist dhamma.

The Sangam literature of the Tamil language reveals that, between 200
BCE and 200 CE, the southern peninsula was being ruled by the Cheras,
the Cholas, and the Pandyas, dynasties that traded extensively with the
Roman Empire and with West and South-East Asia. In North India, Hinduism
asserted patriarchal control within the family, leading to increased
subordination of women. By the 4th and 5th centuries, the Gupta Empire
had created in the greater Ganges Plain a complex system of
administration and taxation that became a model for later Indian
kingdoms. Under the Guptas, a renewed Hinduism based on devotion rather
than the management of ritual began to assert itself. The renewal was
reflected in a flowering of sculpture and architecture, which found
patrons among an urban elite. Classical Sanskrit literature flowered as
well, and Indian science, astronomy, medicine, and mathematics made
significant advances.

\section{Medieval India}\label{medieval-india}

\begin{itemize}
\item
  \emph{They were imitated all over India and led to both the resurgence
  of Hinduism and the development of all modern languages of the
  subcontinent.}
\item
  \emph{The sultanate was to control much of North India and to make
  many forays into South India.}
\item
  \emph{The sultanate's raiding and weakening of the regional kingdoms
  of South India paved the way for the indigenous Vijayanagara Empire.}
\end{itemize}

The Indian early medieval age, 600 CE to 1200 CE, is defined by regional
kingdoms and cultural diversity. When Harsha of Kannauj, who ruled much
of the Indo-Gangetic Plain from 606 to 647 CE, attempted to expand
southwards, he was defeated by the Chalukya ruler of the Deccan. When
his successor attempted to expand eastwards, he was defeated by the Pala
king of Bengal. When the Chalukyas attempted to expand southwards, they
were defeated by the Pallavas from farther south, who in turn were
opposed by the Pandyas and the Cholas from still farther south. No ruler
of this period was able to create an empire and consistently control
lands much beyond his core region. During this time, pastoral peoples
whose land had been cleared to make way for the growing agricultural
economy were accommodated within caste society, as were new
non-traditional ruling classes. The caste system consequently began to
show regional differences.

In the 6th and 7th centuries, the first devotional hymns were created in
the Tamil language. They were imitated all over India and led to both
the resurgence of Hinduism and the development of all modern languages
of the subcontinent. Indian royalty, big and small, and the temples they
patronised drew citizens in great numbers to the capital cities, which
became economic hubs as well. Temple towns of various sizes began to
appear everywhere as India underwent another urbanisation. By the 8th
and 9th centuries, the effects were felt in South-East Asia, as South
Indian culture and political systems were exported to lands that became
part of modern-day Myanmar, Thailand, Laos, Cambodia, Vietnam,
Philippines, Malaysia, and Java. Indian merchants, scholars, and
sometimes armies were involved in this transmission; South-East Asians
took the initiative as well, with many sojourning in Indian seminaries
and translating Buddhist and Hindu texts into their languages.

After the 10th century, Muslim Central Asian nomadic clans, using
swift-horse cavalry and raising vast armies united by ethnicity and
religion, repeatedly overran South Asia's north-western plains, leading
eventually to the establishment of the Islamic Delhi Sultanate in 1206.
The sultanate was to control much of North India and to make many forays
into South India. Although at first disruptive for the Indian elites,
the sultanate largely left its vast non-Muslim subject population to its
own laws and customs. By repeatedly repulsing Mongol raiders in the 13th
century, the sultanate saved India from the devastation visited on West
and Central Asia, setting the scene for centuries of migration of
fleeing soldiers, learned men, mystics, traders, artists, and artisans
from that region into the subcontinent, thereby creating a syncretic
Indo-Islamic culture in the north. The sultanate's raiding and weakening
of the regional kingdoms of South India paved the way for the indigenous
Vijayanagara Empire. Embracing a strong Shaivite tradition and building
upon the military technology of the sultanate, the empire came to
control much of peninsular India, and was to influence South Indian
society for long afterwards.

\includegraphics[width=3.05067in,height=5.50000in]{media/image2.jpg}\\
\emph{Writing the will and testament of the Mughal king in Persian,
1590--1595}

\section{Early modern India}\label{early-modern-india}

\begin{itemize}
\item
  \emph{Expanding commerce during Mughal rule gave rise to new Indian
  commercial and political elites along the coasts of southern and
  eastern India.}
\item
  \emph{India was then no longer exporting manufactured goods as it long
  had, but was instead supplying the British Empire with raw materials,
  and many historians consider this to be the onset of India's colonial
  period.}
\end{itemize}

In the early 16th century, northern India, being then under mainly
Muslim rulers, fell again to the superior mobility and firepower of a
new generation of Central Asian warriors. The resulting Mughal Empire
did not stamp out the local societies it came to rule, but rather
balanced and pacified them through new administrative practices and
diverse and inclusive ruling elites, leading to more systematic,
centralised, and uniform rule. Eschewing tribal bonds and Islamic
identity, especially under Akbar, the Mughals united their far-flung
realms through loyalty, expressed through a Persianised culture, to an
emperor who had near-divine status. The Mughal state's economic
policies, deriving most revenues from agriculture and mandating that
taxes be paid in the well-regulated silver currency, caused peasants and
artisans to enter larger markets. The relative peace maintained by the
empire during much of the 17th century was a factor in India's economic
expansion, resulting in greater patronage of painting, literary forms,
textiles, and architecture. Newly coherent social groups in northern and
western India, such as the Marathas, the Rajputs, and the Sikhs, gained
military and governing ambitions during Mughal rule, which, through
collaboration or adversity, gave them both recognition and military
experience. Expanding commerce during Mughal rule gave rise to new
Indian commercial and political elites along the coasts of southern and
eastern India. As the empire disintegrated, many among these elites were
able to seek and control their own affairs.

By the early 18th century, with the lines between commercial and
political dominance being increasingly blurred, a number of European
trading companies, including the English East India Company, had
established coastal outposts. The East India Company's control of the
seas, greater resources, and more advanced military training and
technology led it to increasingly flex its military muscle and caused it
to become attractive to a portion of the Indian elite; these factors
were crucial in allowing the company to gain control over the Bengal
region by 1765 and sideline the other European companies. Its further
access to the riches of Bengal and the subsequent increased strength and
size of its army enabled it to annex or subdue most of India by the
1820s. India was then no longer exporting manufactured goods as it long
had, but was instead supplying the British Empire with raw materials,
and many historians consider this to be the onset of India's colonial
period. By this time, with its economic power severely curtailed by the
British parliament and effectively having been made an arm of British
administration, the company began to more consciously enter non-economic
arenas such as education, social reform, and culture.

\includegraphics[width=5.50000in,height=4.48858in]{media/image3.jpg}\\
\emph{The British Indian Empire, from the 1909 edition of The Imperial
Gazetteer of India. Areas directly governed by the British are shaded
pink; the princely states under British suzerainty are in yellow.}

\includegraphics[width=5.50000in,height=4.15094in]{media/image4.jpg}\\
\emph{Jawaharlal Nehru (left) became India's first prime minister in
1947. Mahatma Gandhi (right) led the independence movement.}

\section{Modern India}\label{modern-india}

\begin{itemize}
\item
  \emph{All were capped by the advent of independence in 1947, but
  tempered by the partition of India into two states: India and
  Pakistan.}
\item
  \emph{Yet, India is also shaped by seemingly unyielding poverty, both
  rural and urban; by religious and caste-related violence; by
  Maoist-inspired Naxalite insurgencies; and by separatism in Jammu and
  Kashmir and in Northeast India.}
\item
  \emph{Although the rebellion was suppressed by 1858, it led to the
  dissolution of the East India Company and the direct administration of
  India by the British government.}
\end{itemize}

Historians consider India's modern age to have begun sometime between
1848 and 1885. The appointment in 1848 of Lord Dalhousie as Governor
General of the East India Company set the stage for changes essential to
a modern state. These included the consolidation and demarcation of
sovereignty, the surveillance of the population, and the education of
citizens. Technological changes---among them, railways, canals, and the
telegraph---were introduced not long after their introduction in Europe.
However, disaffection with the company also grew during this time, and
set off the Indian Rebellion of 1857. Fed by diverse resentments and
perceptions, including invasive British-style social reforms, harsh land
taxes, and summary treatment of some rich landowners and princes, the
rebellion rocked many regions of northern and central India and shook
the foundations of Company rule. Although the rebellion was suppressed
by 1858, it led to the dissolution of the East India Company and the
direct administration of India by the British government. Proclaiming a
unitary state and a gradual but limited British-style parliamentary
system, the new rulers also protected princes and landed gentry as a
feudal safeguard against future unrest. In the decades following, public
life gradually emerged all over India, leading eventually to the
founding of the Indian National Congress in 1885.

The rush of technology and the commercialisation of agriculture in the
second half of the 19th century was marked by economic setbacks---many
small farmers became dependent on the whims of far-away markets. There
was an increase in the number of large-scale famines, and, despite the
risks of infrastructure development borne by Indian taxpayers, little
industrial employment was generated for Indians. There were also
salutary effects: commercial cropping, especially in the newly canalled
Punjab, led to increased food production for internal consumption. The
railway network provided critical famine relief, notably reduced the
cost of moving goods, and helped the nascent Indian-owned industry.

After World War I, in which approximately one million Indians served, a
new period began. It was marked by British reforms but also repressive
legislation, by more strident Indian calls for self-rule, and by the
beginnings of a nonviolent movement of non-co-operation, of which
Mohandas Karamchand Gandhi would become the leader and enduring symbol.
During the 1930s, slow legislative reform was enacted by the British;
the Indian National Congress won victories in the resulting elections.
The next decade was beset with crises: Indian participation in World War
II, the Congress's final push for non-co-operation, and an upsurge of
Muslim nationalism. All were capped by the advent of independence in
1947, but tempered by the partition of India into two states: India and
Pakistan.

Vital to India's self-image as an independent nation was its
constitution, completed in 1950, which put in place a secular and
democratic republic. It has remained a democracy with civil liberties,
an active Supreme Court, and a largely independent press. Economic
liberalisation, which was begun in the 1990s, has created a large urban
middle class, transformed India into one of the world's fastest-growing
economies, and increased its geopolitical clout. Indian movies, music,
and spiritual teachings play an increasing role in global culture. Yet,
India is also shaped by seemingly unyielding poverty, both rural and
urban; by religious and caste-related violence; by Maoist-inspired
Naxalite insurgencies; and by separatism in Jammu and Kashmir and in
Northeast India. It has unresolved territorial disputes with China and
with Pakistan. The India--Pakistan nuclear rivalry came to a head in
1998. India's sustained democratic freedoms are unique among the world's
newer nations; however, in spite of its recent economic successes,
freedom from want for its disadvantaged population remains a goal yet to
be achieved.

\section{Geography}\label{geography}

\begin{itemize}
\item
  \emph{The original Indian plate survives as peninsular India, the
  oldest and geologically most stable part of India.}
\item
  \emph{India has two archipelagos: the Lakshadweep, coral atolls off
  India's south-western coast; and the Andaman and Nicobar Islands, a
  volcanic chain in the Andaman Sea.}
\item
  \emph{Coastal features include the marshy Rann of Kutch of western
  India and the alluvial Sundarbans delta of eastern India; the latter
  is shared with Bangladesh.}
\end{itemize}

India comprises the bulk of the Indian subcontinent, lying atop the
Indian tectonic plate, a part of the Indo-Australian Plate. India's
defining geological processes began 75 million years ago when the Indian
plate, then part of the southern supercontinent Gondwana, began a
north-eastward drift caused by seafloor spreading to its south-west and,
later, south and south-east. Simultaneously, the vast Tethyn oceanic
crust, to its northeast, began to subduct under the Eurasian plate.
These dual processes, driven by convection in the Earth's mantle, both
created the Indian Ocean and caused the Indian continental crust
eventually to under-thrust Eurasia and to uplift the Himalayas.
Immediately south of the emerging Himalayas, plate movement created a
vast trough that rapidly filled with river-borne sediment and now
constitutes the Indo-Gangetic Plain. Cut off from the plain by the
ancient Aravalli Range lies the Thar Desert.

The original Indian plate survives as peninsular India, the oldest and
geologically most stable part of India. It extends as far north as the
Satpura and Vindhya ranges in central India. These parallel chains run
from the Arabian Sea coast in Gujarat in the west to the coal-rich Chota
Nagpur Plateau in Jharkhand in the east. To the south, the remaining
peninsular landmass, the Deccan Plateau, is flanked on the west and east
by coastal ranges known as the Western and Eastern Ghats; the plateau
contains the country's oldest rock formations, some over one billion
years old. Constituted in such fashion, India lies to the north of the
equator between 6° 44' and 35° 30' north latitude and 68° 7' and 97° 25'
east longitude.

India's coastline measures 7,517 kilometres (4,700~mi) in length; of
this distance, 5,423 kilometres (3,400~mi) belong to peninsular India
and 2,094 kilometres (1,300~mi) to the Andaman, Nicobar, and Lakshadweep
island chains. According to the Indian naval hydrographic charts, the
mainland coastline consists of the following: 43\% sandy beaches; 11\%
rocky shores, including cliffs; and 46\% mudflats or marshy shores.

Major Himalayan-origin rivers that substantially flow through India
include the Ganges and the Brahmaputra, both of which drain into the Bay
of Bengal. Important tributaries of the Ganges include the Yamuna and
the Kosi; the latter's extremely low gradient often leads to severe
floods and course changes. Major peninsular rivers, whose steeper
gradients prevent their waters from flooding, include the Godavari, the
Mahanadi, the Kaveri, and the Krishna, which also drain into the Bay of
Bengal; and the Narmada and the Tapti, which drain into the Arabian Sea.
Coastal features include the marshy Rann of Kutch of western India and
the alluvial Sundarbans delta of eastern India; the latter is shared
with Bangladesh. India has two archipelagos: the Lakshadweep, coral
atolls off India's south-western coast; and the Andaman and Nicobar
Islands, a volcanic chain in the Andaman Sea.

The Indian climate is strongly influenced by the Himalayas and the Thar
Desert, both of which drive the economically and culturally pivotal
summer and winter monsoons. The Himalayas prevent cold Central Asian
katabatic winds from blowing in, keeping the bulk of the Indian
subcontinent warmer than most locations at similar latitudes. The Thar
Desert plays a crucial role in attracting the moisture-laden south-west
summer monsoon winds that, between June and October, provide the
majority of India's rainfall. Four major climatic groupings predominate
in India: tropical wet, tropical dry, subtropical humid, and montane.

\section{Biodiversity}\label{biodiversity}

\begin{itemize}
\item
  \emph{India lies within the Indomalaya ecozone and contains three
  biodiversity hotspots.}
\item
  \emph{Between these extremes lie the moist deciduous sal forest of
  eastern India; the dry deciduous teak forest of central and southern
  India; and the babul-dominated thorn forest of the central Deccan and
  western Gangetic plain.}
\end{itemize}

India lies within the Indomalaya ecozone and contains three biodiversity
hotspots. One of 17 megadiverse countries, it hosts 8.6\% of all
mammalian, 13.7\% of all avian, 7.9\% of all reptilian, 6\% of all
amphibian, 12.2\% of all piscine, and 6.0\% of all flowering plant
species. About 21.2\% of the country's landmass is covered by forests
(tree canopy density \textgreater{}10\%), of which 12.2\% comprises
moderately or very dense forests (tree canopy density
\textgreater{}40\%). Endemism is high among plants, 33\%, and among
ecoregions such as the shola forests. Habitat ranges from the tropical
rainforest of the Andaman Islands, Western Ghats, and North-East India
to the coniferous forest of the Himalaya. Between these extremes lie the
moist deciduous sal forest of eastern India; the dry deciduous teak
forest of central and southern India; and the babul-dominated thorn
forest of the central Deccan and western Gangetic plain. The medicinal
neem, widely used in rural Indian herbal remedies, is a key Indian tree.
The luxuriant pipal fig tree, shown on the seals of Mohenjo-daro, shaded
Gautama Buddha as he sought enlightenment.

Many Indian species descend from taxa originating in Gondwana, from
which the Indian plate separated more than 105 million years Before
Present. Peninsular India's subsequent movement towards and collision
with the Laurasian landmass set off a mass exchange of species. Epochal
volcanism and climatic changes 20 million years ago forced a mass
extinction. Mammals then entered India from Asia through two
zoogeographical passes flanking the rising Himalaya. Thus, while 45.8\%
of reptiles and 55.8\% of amphibians are endemic, only 12.6\% of mammals
and 4.5\% of birds are. Among them are the Nilgiri leaf monkey and
Beddome's toad of the Western Ghats. India contains 172 IUCN-designated
threatened animal species, or 2.9\% of endangered forms.

The pervasive and ecologically devastating human encroachment of recent
decades has critically endangered Indian wildlife. In response, the
system of national parks and protected areas, first established in 1935,
was substantially expanded. In 1972, India enacted the Wildlife
Protection Act and Project Tiger to safeguard crucial wilderness; the
Forest Conservation Act was enacted in 1980 and amendments added in
1988. India hosts more than five hundred wildlife sanctuaries and
thirteen~biosphere reserves, four of which are part of the World Network
of Biosphere Reserves; twenty-five wetlands are registered under the
Ramsar Convention.

\section{Politics and government}\label{politics-and-government}

\includegraphics[width=5.50000in,height=3.66667in]{media/image5.jpg}\\
\emph{A parliamentary joint session being held in the Sansad Bhavan.}

\section{Politics}\label{politics}

\begin{itemize}
\item
  \emph{India is the world's most populous democracy.}
\item
  \emph{On 20 July 2017, Ram Nath Kovind was elected India's 14th
  president and took the oath of office on 25 July 2017.}
\end{itemize}

India is the world's most populous democracy. A parliamentary republic
with a multi-party system, it has seven~recognised national parties,
including the Indian National Congress and the Bharatiya Janata Party
(BJP), and more than 40~regional parties. The Congress is considered
centre-left in Indian political culture, and the BJP right-wing. For
most of the period between 1950---when India first became a
republic---and the late 1980s, the Congress held a majority in the
parliament. Since then, however, it has increasingly shared the
political stage with the BJP, as well as with powerful regional parties
which have often forced the creation of multi-party coalition
governments at the centre.

In the Republic of India's first three general elections, in 1951, 1957,
and 1962, the Jawaharlal Nehru-led Congress won easy victories. On
Nehru's death in 1964, Lal Bahadur Shastri briefly became prime
minister; he was succeeded, after his own unexpected death in 1966, by
Indira Gandhi, who went on to lead the Congress to election victories in
1967 and 1971. Following public discontent with the state of emergency
she declared in 1975, the Congress was voted out of power in 1977; the
then-new Janata Party, which had opposed the emergency, was voted in.
Its government lasted just over two years. Voted back into power in
1980, the Congress saw a change in leadership in 1984, when Indira
Gandhi was assassinated; she was succeeded by her son Rajiv Gandhi, who
won an easy victory in the general elections later that year. The
Congress was voted out again in 1989 when a National Front coalition,
led by the newly formed Janata Dal in alliance with the Left Front, won
the elections; that government too proved relatively short-lived,
lasting just under two years. Elections were held again in 1991; no
party won an absolute majority. The Congress, as the largest single
party, was able to form a minority government led by P. V. Narasimha
Rao.

A two-year period of political turmoil followed the general election of
1996. Several short-lived alliances shared power at the centre. The BJP
formed a government briefly in 1996; it was followed by two
comparatively long-lasting United Front coalitions, which depended on
external support. In 1998, the BJP was able to form a successful
coalition, the National Democratic Alliance (NDA). Led by Atal Bihari
Vajpayee, the NDA became the first non-Congress, coalition government to
complete a five-year term. In the 2004 Indian general elections, again
no party won an absolute majority, but the Congress emerged as the
largest single party, forming another successful coalition: the United
Progressive Alliance (UPA). It had the support of left-leaning parties
and MPs who opposed the BJP. The UPA returned to power in the 2009
general election with increased numbers, and it no longer required
external support from India's communist parties. That year, Manmohan
Singh became the first prime minister since Jawaharlal Nehru in 1957 and
1962 to be re-elected to a consecutive five-year term. In the 2014
general election, the BJP became the first political party since 1984 to
win a majority and govern without the support of other parties. The
incumbent Indian prime minister is Narendra Modi, a former chief
minister of Gujarat. On 20 July 2017, Ram Nath Kovind was elected
India's 14th president and took the oath of office on 25 July 2017.

\section{Government}\label{government}

\begin{itemize}
\item
  \emph{The Constitution of India, which came into effect on 26 January
  1950, states in its preamble that India is a sovereign, socialist,
  secular, democratic republic.}
\item
  \emph{Legislature: The legislature of India is the bicameral
  parliament.}
\item
  \emph{India is a federation with a parliamentary system governed under
  the Constitution of India, which serves as the country's supreme legal
  document.}
\item
  \emph{The Government of India comprises three branches:}
\end{itemize}

India is a federation with a parliamentary system governed under the
Constitution of India, which serves as the country's supreme legal
document. It is a constitutional republic and representative democracy,
in which "majority rule is tempered by minority rights protected by
law". Federalism in India defines the power distribution between the
union, or central, government and the states. The government abides by
constitutional checks and balances. The Constitution of India, which
came into effect on 26 January 1950, states in its preamble that India
is a sovereign, socialist, secular, democratic republic. India's form of
government, traditionally described as "quasi-federal" with a strong
centre and weak states, has grown increasingly federal since the late
1990s as a result of political, economic, and social changes.

The Government of India comprises three branches:

Executive: The President of India is the ceremonial head of state, who
is elected indirectly for a five-year term by an electoral college
comprising members of national and state legislatures. The Prime
Minister of India is the head of government and exercises most executive
power. Appointed by the president, the prime minister is by convention
supported by the party or political alliance having a majority of seats
in the lower house of parliament. The executive of the Indian government
consists of the president, the vice president, and the Union Council of
Ministers---with the cabinet being its executive committee---headed by
the prime minister. Any minister holding a portfolio must be a member of
one of the houses of parliament. In the Indian parliamentary system, the
executive is subordinate to the legislature; the prime minister and his
or her council are directly responsible to the lower house of the
parliament. The civil servants act as permanent executives and all
decisions of the executive are implemented by them.

Legislature: The legislature of India is the bicameral parliament.
Operating under a Westminster-style parliamentary system, it comprises
an upper house called the Rajya Sabha (Council of States) and a lower
house called the Lok Sabha (House of the People). The Rajya Sabha is a
permanent body of 245~members who serve staggered six-year~terms. Most
are elected indirectly by the state and union territorial legislatures
in numbers proportional to their state's share of the national
population. All but two of the Lok Sabha's 545~members are directly
elected by popular vote; they represent single-member constituencies for
five-year~terms. The remaining two~members are nominated by the
president from among the Anglo-Indian community, in case the president
decides that they are not adequately represented.

Judiciary: India has a three-tier~unitary independent judiciary
comprising the supreme court, headed by the Chief Justice of India,
24~high courts, and a large number of trial courts. The supreme court
has original jurisdiction over cases involving fundamental rights and
over disputes between states and the centre and has appellate
jurisdiction over the high courts. It has the power to both strike down
union or state laws which contravene the constitution, and invalidate
any government action it deems unconstitutional.

\section{Administrative divisions}\label{administrative-divisions}

\begin{itemize}
\item
  \emph{India is a federal union comprising 29 states and 7 union
  territories.}
\end{itemize}

India is a federal union comprising 29 states and 7 union territories.
All states, as well in addition to the union territories of Puducherry
and the National Capital Territory of Delhi, have elected legislatures
and governments following on the Westminster system of governance. The
remaining five union territories are directly ruled by the centre
through appointed administrators. In 1956, under the States
Reorganisation Act, states were reorganised on a linguistic basis. Since
then, their structure has remained largely unchanged. Each state or
union territory is further divided into administrative districts. The
districts are further divided into tehsils and ultimately into villages.

\section{Foreign, economic and strategic
relations}\label{foreign-economic-and-strategic-relations}

\begin{itemize}
\item
  \emph{It comprises the Indian Army, the Indian Navy, the Indian Air
  Force, and the Indian Coast Guard.}
\item
  \emph{In 2008, a civilian nuclear agreement was signed between India
  and the United States.}
\item
  \emph{As of 2012{[}update{]}, India is the world's largest arms
  importer; between 2007 and 2011, it accounted for 10\% of funds spent
  on international arms purchases.}
\item
  \emph{Since its independence in 1947, India has maintained cordial
  relations with most nations.}
\end{itemize}

Since its independence in 1947, India has maintained cordial relations
with most nations. In the 1950s, it strongly supported decolonisation in
Africa and Asia and played a lead role in the Non-Aligned Movement. In
the late 1980s, the Indian military twice intervened abroad at the
invitation of neighbouring countries: a peace-keeping operation in Sri
Lanka between 1987 and 1990; and an armed intervention to prevent a 1988
coup d'état attempt in the Maldives. India has tense relations with
neighbouring Pakistan; the two nations have gone to war four times: in
1947, 1965, 1971, and 1999. Three of these wars were fought over the
disputed territory of Kashmir, while the fourth, the 1971 war, followed
from India's support for the independence of Bangladesh. After waging
the 1962 Sino-Indian War and the 1965 war with Pakistan, India pursued
close military and economic ties with the Soviet Union; by the late
1960s, the Soviet Union was its largest arms supplier.

Aside from ongoing special relationship with Russia, India has
wide-ranging defence relations with Israel and France. In recent years,
it has played key roles in the South Asian Association for Regional
Cooperation and the World Trade Organization. The nation has provided
100,000 military and police personnel to serve in 35 UN peacekeeping
operations across four continents. It participates in the East Asia
Summit, the G8+5, and other multilateral forums. India has close
economic ties with South America, Asia, and Africa; it pursues a "Look
East" policy that seeks to strengthen partnerships with the ASEAN
nations, Japan, and South Korea that revolve around many issues, but
especially those involving economic investment and regional security.

China's nuclear test of 1964, as well as its repeated threats to
intervene in support of Pakistan in the 1965 war, convinced India to
develop nuclear weapons. India conducted its first nuclear weapons test
in 1974 and carried out further underground testing in 1998. Despite
criticism and military sanctions, India has signed neither the
Comprehensive Nuclear-Test-Ban Treaty nor the Nuclear Non-Proliferation
Treaty, considering both to be flawed and discriminatory. India
maintains a "no first use" nuclear policy and is developing a nuclear
triad capability as a part of its "Minimum Credible Deterrence"
doctrine. It is developing a ballistic missile defence shield and, in
collaboration with Russia, a fifth-generation fighter jet. Other
indigenous military projects involve the design and implementation of
Vikrant-class aircraft carriers and Arihant-class nuclear submarines.

Since the end of the Cold War, India has increased its economic,
strategic, and military co-operation with the United States and the
European Union. In 2008, a civilian nuclear agreement was signed between
India and the United States. Although India possessed nuclear weapons at
the time and was not party to the Nuclear Non-Proliferation Treaty, it
received waivers from the International Atomic Energy Agency and the
Nuclear Suppliers Group, ending earlier restrictions on India's nuclear
technology and commerce. As a consequence, India became the sixth de
facto nuclear weapons state. India subsequently signed co-operation
agreements involving civilian nuclear energy with Russia, France, the
United Kingdom, and Canada.

The President of India is the supreme commander of the nation's armed
forces; with 1.395~million active troops, they compose the world's
second-largest military. It comprises the Indian Army, the Indian Navy,
the Indian Air Force, and the Indian Coast Guard. The official Indian
defence budget for 2011 was US\$36.03~billion, or 1.83\% of GDP. For the
fiscal year spanning 2012--2013, US\$40.44~billion was budgeted.
According to a 2008 SIPRI report, India's annual military expenditure in
terms of purchasing power stood at US\$72.7~billion. In 2011, the annual
defence budget increased by 11.6\%, although this does not include funds
that reach the military through other branches of government. As of
2012{[}update{]}, India is the world's largest arms importer; between
2007 and 2011, it accounted for 10\% of funds spent on international
arms purchases. Much of the military expenditure was focused on defence
against Pakistan and countering growing Chinese influence in the Indian
Ocean. In May 2017, the Indian Space Research Organisation launched the
South Asia Satellite, a gift from India to its neighbouring SAARC
countries. In October 2018, India signed a US\$5.43~billion (over Rs
400~billion) agreement with Russia to procure four S-400 Triumf
surface-to-air missile defence systems, Russia's most advanced
long-range missile defence system.

\section{Economy}\label{economy}

\begin{itemize}
\item
  \emph{In 2008, India's share of world trade was 1.68\%; In 2011, India
  was the world's tenth-largest importer and the nineteenth-largest
  exporter.}
\item
  \emph{India was the second largest textile exporter after China in the
  world in the calendar year 2013.}
\item
  \emph{India has been a member of WTO since 1 January 1995.}
\item
  \emph{India's consumer market, the world's eleventh-largest, is
  expected to become fifth-largest by 2030.}
\end{itemize}

According to the International Monetary Fund (IMF), the Indian economy
in 2017 was nominally worth US\$2.611~trillion; it is the sixth-largest
economy by market exchange rates, and is, at US\$9.459~trillion, the
third-largest by purchasing power parity, or PPP. With its average
annual GDP growth rate of 5.8\% over the past two decades, and reaching
6.1\% during 2011--12, India is one of the world's fastest-growing
economies. However, the country ranks 140th in the world in nominal GDP
per capita and 129th in GDP per capita at PPP. Until 1991, all Indian
governments followed protectionist policies that were influenced by
socialist economics. Widespread state intervention and regulation
largely walled the economy off from the outside world. An acute balance
of payments crisis in 1991 forced the nation to liberalise its economy;
since then it has slowly moved towards a free-market system by
emphasising both foreign trade and direct investment inflows. India has
been a member of WTO since 1 January 1995.

The 513.7-million-worker Indian labour force is the world's
second-largest, as of 2016{[}update{]}. The service sector makes up
55.6\% of GDP, the industrial sector 26.3\% and the agricultural sector
18.1\%. India's foreign exchange remittances of US\$70~billion in 2014,
the largest in the world, contributed to its economy by 25 million
Indians working in foreign countries. Major agricultural products
include rice, wheat, oilseed, cotton, jute, tea, sugarcane, and
potatoes. Major industries include textiles, telecommunications,
chemicals, pharmaceuticals, biotechnology, food processing, steel,
transport equipment, cement, mining, petroleum, machinery, and software.
In 2006, the share of external trade in India's GDP stood at 24\%, up
from 6\% in 1985. In 2008, India's share of world trade was 1.68\%; In
2011, India was the world's tenth-largest importer and the
nineteenth-largest exporter. Major exports include petroleum products,
textile goods, jewellery, software, engineering goods, chemicals, and
leather manufactures. Major imports include crude oil, machinery, gems,
fertiliser, and chemicals. Between 2001 and 2011, the contribution of
petrochemical and engineering goods to total exports grew from 14\% to
42\%. India was the second largest textile exporter after China in the
world in the calendar year 2013.

Averaging an economic growth rate of 7.5\% for several years prior to
2007, India has more than doubled its hourly wage rates during the first
decade of the 21st century. Some 431 million Indians have left poverty
since 1985; India's middle classes are projected to number around
580~million by 2030. Though ranking 51st in global competitiveness,
India ranks 17th in financial market sophistication, 24th in the banking
sector, 44th in business sophistication, and 39th in innovation, ahead
of several advanced economies, as of 2010{[}update{]}. With 7 of the
world's top 15 information technology outsourcing companies based in
India, the country is viewed as the second-most favourable outsourcing
destination after the United States, as of 2009{[}update{]}. India's
consumer market, the world's eleventh-largest, is expected to become
fifth-largest by 2030. However, hardly 2\% of Indians pay income taxes.

Driven by growth, India's nominal GDP per capita has steadily increased
from US\$329 in 1991, when economic liberalisation began, to US\$1,265
in 2010, to an estimated US\$1,723 in 2016, and is expected to grow to
US\$2,358 by 2020; however, it has remained lower than those of other
Asian developing countries such as Indonesia, Malaysia, Philippines, Sri
Lanka, and Thailand, and is expected to remain so in the near future.
However, it is higher than Pakistan, Nepal, Afghanistan, Bangladesh and
others.

According to a 2011 PricewaterhouseCoopers (PwC) report, India's GDP at
purchasing power parity could overtake that of the United States by
2045. During the next four decades, Indian GDP is expected to grow at an
annualised average of 8\%, making it potentially the world's
fastest-growing major economy until 2050. The report highlights key
growth factors: a young and rapidly growing working-age population;
growth in the manufacturing sector because of rising education and
engineering skill levels; and sustained growth of the consumer market
driven by a rapidly growing middle-class. The World Bank cautions that,
for India to achieve its economic potential, it must continue to focus
on public sector reform, transport infrastructure, agricultural and
rural development, removal of labour regulations, education, energy
security, and public health and nutrition.

According to the Worldwide Cost of Living Report 2017 released by the
Economist Intelligence Unit (EIU) which was created by comparing more
than 400 individual prices across 160 products and services, four of the
cheapest cities were in India: Bangalore (3rd), Mumbai (5th), Chennai
(5th) and New Delhi (8th).

\section{Industries}\label{industries}

\begin{itemize}
\item
  \emph{India's telecommunication industry, the world's fastest-growing,
  added 227 million subscribers during the period 2010--11, and after
  the third quarter of 2017, India surpassed the US to become the second
  largest smartphone market in the world after China.}
\item
  \emph{India is among the top 12 biotech destinations in the world.}
\item
  \emph{India's R \& D spending constitutes 60\% of the
  biopharmaceutical industry.}
\end{itemize}

India's telecommunication industry, the world's fastest-growing, added
227 million subscribers during the period 2010--11, and after the third
quarter of 2017, India surpassed the US to become the second largest
smartphone market in the world after China.

The Indian automotive industry, the world's second-fastest growing,
increased domestic sales by 26\% during 2009--10, and exports by 36\%
during 2008--09. India's capacity to generate electrical power is 300
gigawatts, of which 42 gigawatts is renewable. At the end of 2011, the
Indian IT industry employed 2.8~million professionals, generated
revenues close to US\$100~billion equalling 7.5\% of Indian GDP and
contributed 26\% of India's merchandise exports.

The pharmaceutical industry in India is among the significant emerging
markets for the global pharmaceutical industry. The Indian
pharmaceutical market is expected to reach \$48.5~billion by 2020.
India's R \& D spending constitutes 60\% of the biopharmaceutical
industry. India is among the top 12 biotech destinations in the world.
The Indian biotech industry grew by 15.1\% in 2012--13, increasing its
revenues from 204.4~billion INR (Indian rupees) to 235.24~billion INR
(3.94 B US\$ -- exchange rate June 2013: 1 US\$ approx. 60 INR).

\section{Socio-economic challenges}\label{socio-economic-challenges}

\begin{itemize}
\item
  \emph{Despite economic growth during recent decades, India continues
  to face socio-economic challenges.}
\item
  \emph{According to Corruption Perceptions Index, India ranked 76th out
  of 176 countries in 2016, from 85th in 2014.}
\item
  \emph{Corruption in India is perceived to have decreased.}
\item
  \emph{30.7\% of India's children under the age of five are
  underweight.}
\end{itemize}

Despite economic growth during recent decades, India continues to face
socio-economic challenges. In 2006, India contained the largest number
of people living below the World Bank's international poverty line of
US\$1.25 per day, the proportion having decreased from 60\% in 1981 to
42\% in 2005; under its later revised poverty line, it was 21\% in 2011.
30.7\% of India's children under the age of five are underweight.
According to a Food and Agriculture Organization report in 2015, 15\% of
the population is undernourished. The Mid-Day Meal Scheme attempts to
lower these rates.

According to a Walk Free Foundation report in 2016, there were an
estimated 18.3~million people in India, or 1.4\% of the population,
living in the forms of modern slavery, such as bonded labour, child
labour, human trafficking, and forced begging, among others. According
to the 2011 census, there were 10.1~million child labourers in the
country, a decline of 2.6~million from 12.6~million child labourers in
2001.

Since 1991, economic inequality between India's states has consistently
grown: the per-capita net state domestic product of the richest states
in 2007 was 3.2 times that of the poorest. Corruption in India is
perceived to have decreased. According to Corruption Perceptions Index,
India ranked 76th out of 176 countries in 2016, from 85th in 2014.

\includegraphics[width=5.50000in,height=3.83197in]{media/image6.png}\\
\emph{Population pyramid 2016}

\section{Demographics}\label{demographics}

\begin{itemize}
\item
  \emph{India continues to face several public health-related
  challenges.}
\item
  \emph{With 1,210,193,422 residents reported in the 2011 provisional
  census report, India is the world's second-most populous country.}
\item
  \emph{Life expectancy in India is at 68 years, with life expectancy
  for women being 69.6 years and for men being 67.3.}
\item
  \emph{Migration from rural to urban areas has been an important
  dynamic in the recent history of India.}
\end{itemize}

With 1,210,193,422 residents reported in the 2011 provisional census
report, India is the world's second-most populous country. Its
population grew by 17.64\% during 2001--2011, compared to 21.54\% growth
in the previous decade (1991--2001). The human sex ratio, according to
the 2011 census, is 940 females per 1,000 males. The median age was 27.6
as of 2016{[}update{]}. The first post-colonial census, conducted in
1951, counted 361.1~million people. Medical advances made in the last 50
years as well as increased agricultural productivity brought about by
the "Green Revolution" have caused India's population to grow rapidly.
India continues to face several public health-related challenges.

Life expectancy in India is at 68 years, with life expectancy for women
being 69.6 years and for men being 67.3. There are around 50 physicians
per 100,000 Indians. Migration from rural to urban areas has been an
important dynamic in the recent history of India. The number of Indians
living in urban areas grew by 31.2\% between 1991 and 2001. Yet, in
2001, over 70\% still lived in rural areas. The level of urbanisation
increased further from 27.81\% in the 2001 Census to 31.16\% in the 2011
Census. The slowing down of the overall growth rate of population was
due to the sharp decline in the growth rate in rural areas since 1991.
According to the 2011 census, there are 53 million-plus urban
agglomerations in India; among them Mumbai, Delhi, Kolkata, Chennai,
Bangalore, Hyderabad and Ahmedabad, in decreasing order by population.
The literacy rate in 2011 was 74.04\%: 65.46\% among females and 82.14\%
among males. The rural-urban literacy gap, which was 21.2 percentage
points in 2001, dropped to 16.1 percentage points in 2011. The
improvement in literacy rate in rural area is two times that in urban
areas. Kerala is the most literate state with 93.91\% literacy; while
Bihar the least with 63.82\%.

\section{Languages}\label{languages}

\begin{itemize}
\item
  \emph{India has no national language.}
\item
  \emph{Other languages spoken in India come from the Austroasiatic and
  Sino-Tibetan language families.}
\end{itemize}

India is home to two major language families: Indo-Aryan (spoken by
about 74\% of the population) and Dravidian (spoken by 24\% of the
population). Other languages spoken in India come from the Austroasiatic
and Sino-Tibetan language families. India has no national language.
Hindi, with the largest number of speakers, is the official language of
the government. English is used extensively in business and
administration and has the status of a "subsidiary official language";
it is important in education, especially as a medium of higher
education. Each state and union territory has one or more official
languages, and the constitution recognises in particular 22 "scheduled
languages".

\section{Religions}\label{religions}

\begin{itemize}
\item
  \emph{The 2011 census reported that the religion in India with the
  largest number of followers was Hinduism (79.80\% of the population),
  followed by Islam (14.23\%); the remaining were Christianity (2.30\%),
  Sikhism (1.72\%), Buddhism (0.70\%), Jainism (0.36\%) and others
  (0.9\%).}
\item
  \emph{India has the world's largest Hindu, Sikh, Jain, Zoroastrian,
  and Bahá'í populations, and has the third-largest Muslim
  population---the largest for a non-Muslim majority country.}
\end{itemize}

The 2011 census reported that the religion in India with the largest
number of followers was Hinduism (79.80\% of the population), followed
by Islam (14.23\%); the remaining were Christianity (2.30\%), Sikhism
(1.72\%), Buddhism (0.70\%), Jainism (0.36\%) and others (0.9\%). India
has the world's largest Hindu, Sikh, Jain, Zoroastrian, and Bahá'í
populations, and has the third-largest Muslim population---the largest
for a non-Muslim majority country.

\section{Culture}\label{culture}

\begin{itemize}
\item
  \emph{India is notable for its religious diversity, with Hinduism,
  Buddhism, Sikhism, Islam, Christianity, and Jainism among the nation's
  major religions.}
\item
  \emph{Indian cultural history spans more than 4,500~years.}
\end{itemize}

Indian cultural history spans more than 4,500~years. During the Vedic
period (c. 1700~-- c. 500~BCE), the foundations of Hindu philosophy,
mythology, theology and literature were laid, and many beliefs and
practices which still exist today, such as dhárma, kárma, yóga, and
mokṣa, were established. India is notable for its religious diversity,
with Hinduism, Buddhism, Sikhism, Islam, Christianity, and Jainism among
the nation's major religions. The predominant religion, Hinduism, has
been shaped by various historical schools of thought, including those of
the Upanishads, the Yoga Sutras, the Bhakti movement, and by Buddhist
philosophy.

\section{Art and architecture}\label{art-and-architecture}

\begin{itemize}
\item
  \emph{The Taj Mahal, built in Agra between 1631 and 1648 by orders of
  Emperor Shah Jahan in memory of his wife, has been described in the
  UNESCO World Heritage List as "the jewel of Muslim art in India and
  one of the universally admired masterpieces of the world's heritage".}
\item
  \emph{Much of Indian architecture, including the Taj Mahal, other
  works of Mughal architecture, and South Indian architecture, blends
  ancient local traditions with imported styles.}
\end{itemize}

Much of Indian architecture, including the Taj Mahal, other works of
Mughal architecture, and South Indian architecture, blends ancient local
traditions with imported styles. Vernacular architecture is also highly
regional in it flavours. Vastu shastra, literally "science of
construction" or "architecture" and ascribed to Mamuni Mayan, explores
how the laws of nature affect human dwellings; it employs precise
geometry and directional alignments to reflect perceived cosmic
constructs. As applied in Hindu temple architecture, it is influenced by
the Shilpa Shastras, a series of foundational texts whose basic
mythological form is the Vastu-Purusha mandala, a square that embodied
the "absolute". The Taj Mahal, built in Agra between 1631 and 1648 by
orders of Emperor Shah Jahan in memory of his wife, has been described
in the UNESCO World Heritage List as "the jewel of Muslim art in India
and one of the universally admired masterpieces of the world's
heritage". Indo-Saracenic Revival architecture, developed by the British
in the late 19th century, drew on Indo-Islamic architecture.

\section{Literature}\label{literature}

\begin{itemize}
\item
  \emph{From the 14th to the 18th centuries, India's literary traditions
  went through a period of drastic change because of the emergence of
  devotional poets such as Kabīr, Tulsīdās, and Guru Nānak.}
\item
  \emph{The earliest literature in India, composed between 1500 BCE and
  1200 CE, was in the Sanskrit language.}
\end{itemize}

The earliest literature in India, composed between 1500 BCE and 1200 CE,
was in the Sanskrit language. Major works of Sanskrit literature include
the Rigveda (c. 1500 BCE--1200 BCE), the epics:Mahābhārata (c. 400
BCE--400 CE) and the Ramayana (c. 300 BCE and later);
Abhijñānaśākuntalam (The Recognition of Śakuntalā, and other dramas of
Kālidāsa (c. 5th century CE) and Mahākāvya poetry. In Tamil literature,
Sangam Literature (c 600 BCE--300 BCE) consisting of 2,381 poems,
composed by 473 poets, is the earliest work. From the 14th to the 18th
centuries, India's literary traditions went through a period of drastic
change because of the emergence of devotional poets such as Kabīr,
Tulsīdās, and Guru Nānak. This period was characterised by a varied and
wide spectrum of thought and expression; as a consequence, medieval
Indian literary works differed significantly from classical traditions.
In the 19th century, Indian writers took a new interest in social
questions and psychological descriptions. In the 20th century, Indian
literature was influenced by the works of Bengali poet and novelist
Rabindranath Tagore, who was a recipient of the Nobel Prize in
Literature.

\section{Performing arts}\label{performing-arts}

\begin{itemize}
\item
  \emph{Theatre in India melds music, dance, and improvised or written
  dialogue.}
\item
  \emph{Eight dance forms, many with narrative forms and mythological
  elements, have been accorded classical dance status by India's
  National Academy of Music, Dance, and Drama.}
\item
  \emph{India has a theatre training institute NSD that is situated at
  New Delhi It is an autonomous organisation under the Ministry of
  Culture, Government of India.}
\end{itemize}

Indian music ranges over various traditions and regional styles.
Classical music encompasses two genres and their various folk offshoots:
the northern Hindustani and southern Carnatic schools. Regionalised
popular forms include filmi and folk music; the syncretic tradition of
the bauls is a well-known form of the latter. Indian dance also features
diverse folk and classical forms. Among the better-known folk dances are
the bhangra of Punjab, the bihu of Assam, the Jhumair and chhau of
Jharkhand, Odisha and West Bengal, garba and dandiya of Gujarat, ghoomar
of Rajasthan, and the lavani of Maharashtra. Eight dance forms, many
with narrative forms and mythological elements, have been accorded
classical dance status by India's National Academy of Music, Dance, and
Drama. These are: bharatanatyam of the state of Tamil Nadu, kathak of
Uttar Pradesh, kathakali and mohiniyattam of Kerala, kuchipudi of Andhra
Pradesh, manipuri of Manipur, odissi of Odisha, and the sattriya of
Assam. Theatre in India melds music, dance, and improvised or written
dialogue. Often based on Hindu mythology, but also borrowing from
medieval romances or social and political events, Indian theatre
includes the bhavai of Gujarat, the jatra of West Bengal, the nautanki
and ramlila of North India, tamasha of Maharashtra, burrakatha of Andhra
Pradesh, terukkuttu of Tamil Nadu, and the yakshagana of Karnataka.
India has a theatre training institute NSD that is situated at New Delhi
It is an autonomous organisation under the Ministry of Culture,
Government of India.

\section{Motion pictures, television}\label{motion-pictures-television}

\begin{itemize}
\item
  \emph{Today, television is the most penetrative media in India;
  industry estimates indicate that as of 2012{[}update{]} there are over
  554 million TV consumers, 462~million with satellite and/or cable
  connections, compared to other forms of mass media such as press
  (350~million), radio (156~million) or internet (37~million).}
\item
  \emph{The Indian film industry produces the world's most-watched
  cinema.}
\item
  \emph{Television broadcasting began in India in 1959 as a state-run
  medium of communication and had slow expansion for more than two
  decades.}
\item
  \emph{South Indian cinema attracts more than 75\% of national film
  revenue.}
\end{itemize}

The Indian film industry produces the world's most-watched cinema.
Established regional cinematic traditions exist in the Assamese,
Bengali, Bhojpuri, Hindi, Kannada, Malayalam, Punjabi, Gujarati,
Marathi, Odia, Tamil, and Telugu languages. South Indian cinema attracts
more than 75\% of national film revenue.

Television broadcasting began in India in 1959 as a state-run medium of
communication and had slow expansion for more than two decades. The
state monopoly on television broadcast ended in the 1990s and, since
then, satellite channels have increasingly shaped the popular culture of
Indian society. Today, television is the most penetrative media in
India; industry estimates indicate that as of 2012{[}update{]} there are
over 554 million TV consumers, 462~million with satellite and/or cable
connections, compared to other forms of mass media such as press
(350~million), radio (156~million) or internet (37~million).

\section{Cuisine}\label{cuisine}

\begin{itemize}
\item
  \emph{The spice trade between India and Europe is often cited by
  historians as the primary catalyst for Europe's Age of Discovery.}
\end{itemize}

Indian cuisine encompasses a wide variety of regional and traditional
cuisines, often depending on a particular state (such as Maharashtrian
cuisine). Staple foods of Indian cuisine include pearl millet (ISO:
bājra), rice, whole-wheat flour (aṭṭa), and a variety of lentils, such
as masoor (most often red lentils), toor (pigeon peas), urad (black
gram), and mong (mung beans). Lentils may be used whole, dehusked---for
example, dhuli moong or dhuli urad---or split. Split lentils, or dal,
are used extensively. The spice trade between India and Europe is often
cited by historians as the primary catalyst for Europe's Age of
Discovery.

\section{Society}\label{society}

\begin{itemize}
\item
  \emph{Family values are important in the Indian tradition, and
  multi-generational patriarchal joint families have been the norm in
  India, though nuclear families are becoming common in urban areas.}
\item
  \emph{India declared untouchability to be illegal in 1947 and has
  since enacted other anti-discriminatory laws and social welfare
  initiatives.}
\item
  \emph{At the workplace in urban India and in international or leading
  Indian companies, the caste related identification has pretty much
  lost its importance.}
\end{itemize}

Traditional Indian society is sometimes defined by social hierarchy. The
Indian caste system embodies much of the social stratification and many
of the social restrictions found in the Indian subcontinent. Social
classes are defined by thousands of endogamous hereditary groups, often
termed as jātis, or "castes". India declared untouchability to be
illegal in 1947 and has since enacted other anti-discriminatory laws and
social welfare initiatives. At the workplace in urban India and in
international or leading Indian companies, the caste related
identification has pretty much lost its importance.

Family values are important in the Indian tradition, and
multi-generational patriarchal joint families have been the norm in
India, though nuclear families are becoming common in urban areas. An
overwhelming majority of Indians, with their consent, have their
marriages arranged by their parents or other elders in the family.
Marriage is thought to be for life, and the divorce rate is extremely
low. As of 2001{[}update{]}, just 1.6 percent of Indian women were
divorced, but this figure was rising due to their education and economic
independence. Child marriages are common, especially in rural areas;
many women wed before reaching 18, which is their legal marriageable
age. Female infanticide and female foeticide in the country have caused
a discrepancy in the sex ratio, as of 2005{[}update{]} it was estimated
that there were 50 million more males than females in the nation.
However, a report from 2011 has shown improvement in the gender ratio.
The payment of dowry, although illegal, remains widespread across class
lines. Deaths resulting from dowry, mostly from bride burning, are on
the rise, despite stringent anti-dowry laws.

Many Indian festivals are religious in origin. The best known include
Diwali, Ganesh Chaturthi, Thai Pongal, Holi, Durga Puja, Eid ul-Fitr,
Bakr-Id, Christmas, and Vaisakhi.

\section{Clothing}\label{clothing}

\begin{itemize}
\item
  \emph{Cotton was domesticated in India by 4000 BCE.}
\item
  \emph{Use of delicate jewellery, modelled on real flowers worn in
  ancient India, is part of a tradition dating back some 5,000 years;
  gemstones are also worn in India as talismans.}
\end{itemize}

Cotton was domesticated in India by 4000 BCE. Traditional Indian dress
varies in colour and style across regions and depends on various
factors, including climate and faith. Popular styles of dress include
draped garments such as the sari for women and the dhoti or lungi for
men. Stitched clothes, such as the shalwar kameez for women and
kurta--pyjama combinations or European-style trousers and shirts for
men, are also popular. Use of delicate jewellery, modelled on real
flowers worn in ancient India, is part of a tradition dating back some
5,000 years; gemstones are also worn in India as talismans.

\section{Sports}\label{sports}

\begin{itemize}
\item
  \emph{India has a comparatively strong presence in shooting sports,
  and has won several medals at the Olympics, the World Shooting
  Championships, and the Commonwealth Games.}
\item
  \emph{Cricket is the most popular sport in India.}
\item
  \emph{India has traditionally been the dominant country at the South
  Asian Games.}
\item
  \emph{Chess, commonly held to have originated in India as chaturaṅga,
  is regaining widespread popularity with the rise in the number of
  Indian grandmasters.}
\end{itemize}

In India, several traditional indigenous sports remain fairly popular,
such as kabaddi, kho kho, pehlwani and gilli-danda. Some of the earliest
forms of Asian martial arts, such as kalarippayattu, musti yuddha,
silambam, and marma adi, originated in India. Chess, commonly held to
have originated in India as chaturaṅga, is regaining widespread
popularity with the rise in the number of Indian grandmasters. Pachisi,
from which parcheesi derives, was played on a giant marble court by
Akbar.

The improved results garnered by the Indian Davis Cup team and other
Indian tennis players in the early 2010s have made tennis increasingly
popular in the country. India has a comparatively strong presence in
shooting sports, and has won several medals at the Olympics, the World
Shooting Championships, and the Commonwealth Games. Other sports in
which Indians have succeeded internationally include badminton (Saina
Nehwal and P V Sindhu are two of the top-ranked female badminton players
in the world), boxing, and wrestling. Football is popular in West
Bengal, Goa, Tamil Nadu, Kerala, and the north-eastern states.

Cricket is the most popular sport in India. Major domestic competitions
include the Indian Premier League, which is the most-watched cricket
league in the world and ranks sixth among all sports leagues.

India has hosted or co-hosted several international sporting events: the
1951 and 1982 Asian Games; the 1987, 1996, and 2011 Cricket World Cup
tournaments; the 2003 Afro-Asian Games; the 2006 ICC Champions Trophy;
the 2010 Hockey World Cup; the 2010 Commonwealth Games; and the 2017
FIFA U-17 World Cup. Major international sporting events held annually
in India include the Chennai Open, the Mumbai Marathon, the Delhi Half
Marathon, and the Indian Masters. The first Formula 1 Indian Grand Prix
featured in late 2011 but has been discontinued from the F1 season
calendar since 2014. India has traditionally been the dominant country
at the South Asian Games. An example of this dominance is the basketball
competition where the Indian team won three out of four tournaments to
date.

\section{See also}\label{see-also}

\begin{itemize}
\item
  \emph{Outline of India}
\end{itemize}

Outline of India

Index of India-related articles

\section{Notes}\label{notes}

\section{References}\label{references}

\section{Bibliography}\label{bibliography}

\begin{itemize}
\item
  \emph{Biodiversity}
\item
  \emph{Overview}
\item
  \emph{Foreign relations and military}
\item
  \emph{Etymology}
\item
  \emph{Politics}
\item
  \emph{History}
\item
  \emph{Geography}
\end{itemize}

Overview

Etymology

History

Geography

Biodiversity

Politics

Foreign relations and military

Economy

Demographics

Culture

\section{External links}\label{external-links}

\begin{itemize}
\item
  \emph{India from the BBC News}
\item
  \emph{Wikimedia Atlas of India}
\item
  \emph{Official website of Government of India}
\item
  \emph{"India".}
\item
  \emph{India from UCB Libraries GovPubs}
\item
  \emph{Government of India Web Directory}
\item
  \emph{India at Curlie}
\end{itemize}

Government

Official website of Government of India

Government of India Web Directory

General information

"India". The World Factbook. Central Intelligence Agency.

India at Curlie

India from UCB Libraries GovPubs

India from the BBC News

Indian State district block village website

Wikimedia Atlas of India

Geographic data related to India at OpenStreetMap

Key Development Forecasts for India from International Futures

Coordinates: 21°N 78°E / 21°N 78°E / 21; 78
