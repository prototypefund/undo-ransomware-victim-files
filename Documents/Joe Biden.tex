\textbf{From Wikipedia, the free encyclopedia}

https://en.wikipedia.org/wiki/Joe\%20Biden\\
Licensed under CC BY-SA 3.0:\\
https://en.wikipedia.org/wiki/Wikipedia:Text\_of\_Creative\_Commons\_Attribution-ShareAlike\_3.0\_Unported\_License

\section{Joe Biden}\label{joe-biden}

\begin{itemize}
\item
  \emph{Obama and Biden were re-elected in 2012.}
\item
  \emph{In 2008, Biden was the running mate of Democratic presidential
  nominee Barack Obama.}
\item
  \emph{Biden also represented Delaware in the U.S. Senate from 1973 to
  2009.}
\item
  \emph{Biden unsuccessfully sought the Democratic presidential
  nomination in 1988 and in 2008.}
\item
  \emph{A member of the Democratic Party, Biden is a candidate for
  President in the 2020 election.}
\end{itemize}

Joseph Robinette Biden Jr. (/ˌrɒbɪˈnɛt ˈbaɪdən/; born November 20, 1942)
is an American politician who served as the 47th vice president of the
United States from 2009 to 2017. Biden also represented Delaware in the
U.S. Senate from 1973 to 2009. A member of the Democratic Party, Biden
is a candidate for President in the 2020 election.

Biden was born in Scranton, Pennsylvania, and lived there for ten years
before moving with his family to Delaware. He became a lawyer in 1969
and was elected to the New Castle County Council in 1970. He was first
elected to the U.S. Senate in 1972, when he became the sixth-youngest
senator in American history. Biden was re-elected six times and was the
fourth most senior senator when he resigned to assume the vice
presidency in 2009. Biden was a long-time member and former chairman of
the Foreign Relations Committee. He opposed the Gulf War in 1991, but
advocated U.S. and NATO intervention in the Bosnian War in 1994 and
1995. He voted in favor of the resolution authorizing the Iraq War in
2002 but opposed the surge of U.S. troops in 2007. He has also served as
chairman of the Senate Judiciary Committee, dealing with issues related
to drug policy, crime prevention, and civil liberties. Biden led the
efforts to pass the Violent Crime Control and Law Enforcement Act, and
the Violence Against Women Act. He also chaired the Judiciary Committee
during the contentious U.S. Supreme Court nominations of Robert Bork and
Clarence Thomas. Biden unsuccessfully sought the Democratic presidential
nomination in 1988 and in 2008.

In 2008, Biden was the running mate of Democratic presidential nominee
Barack Obama. As vice president, Biden oversaw infrastructure spending
aimed at counteracting the Great Recession and helped formulate U.S.
policy toward Iraq through the withdrawal of U.S. troops in 2011. His
ability to negotiate with congressional Republicans helped the Obama
administration pass legislation such as the Tax Relief, Unemployment
Insurance Reauthorization, and Job Creation Act of 2010, which resolved
a taxation deadlock; the Budget Control Act of 2011, which resolved that
year's debt ceiling crisis; and the American Taxpayer Relief Act of
2012, which addressed the impending fiscal cliff. Obama and Biden were
re-elected in 2012.

In October 2015, after months of speculation, Biden announced he would
not seek the presidency in the 2016 elections. In January 2017, Obama
awarded Biden the Presidential Medal of Freedom with distinction. After
completing his second term as vice president, Biden joined the faculty
of the University of Pennsylvania, where he was named the Benjamin
Franklin Professor of Presidential Practice. He announced his 2020 run
for president on April 25, 2019.

\includegraphics[width=4.25885in,height=5.50000in]{media/image1.jpg}\\
\emph{Biden while a student at Archmere Academy in the 1950s.}

\section{Early life (1942--1965)}\label{early-life-19421965}

\begin{itemize}
\item
  \emph{Biden was admitted to the Delaware bar in 1969.}
\item
  \emph{Biden was born on November 20, 1942, at St. Mary's Hospital in
  Scranton, Pennsylvania, to Catherine Eugenia Biden (née Finnegan) and
  Joseph Robinette Biden Sr.}
\item
  \emph{Negative impressions of drinking alcohol in the Biden and
  Finnegan families and in the neighborhood led to Biden becoming a
  teetotaler.}
\end{itemize}

Biden was born on November 20, 1942, at St. Mary's Hospital in Scranton,
Pennsylvania, to Catherine Eugenia Biden (née Finnegan) and Joseph
Robinette Biden Sr. He was the first of four siblings in a Catholic
family, with a sister and two brothers. His mother was of Irish descent,
with roots variously attributed to County Louth or County Londonderry.
His paternal grandparents, Mary Elizabeth (Robinette) and Joseph H.
Biden, an oil businessman from Baltimore, Maryland, were of English,
French, and Irish ancestry. His paternal great-great-great grandfather,
William Biden, was born in Sussex, England, and immigrated to the United
States. His maternal great-grandfather, Edward Francis Blewitt, was a
member of the Pennsylvania State Senate.

Biden's father had been wealthy earlier in his life but suffered several
financial setbacks by the time his son was born. For several years, the
family had to live with Biden's maternal grandparents, the Finnegans.
When the Scranton area fell into economic decline during the 1950s,
Biden's father could not find sustained work. In 1953, the Biden family
moved into an apartment in Claymont, Delaware, where they lived for
several years before again moving to a house in Wilmington, Delaware.
Joe Biden Sr. became a successful used car salesman, and the family's
circumstances were middle class.

Biden attended the Archmere Academy in Claymont where he was a standout
halfback/wide receiver on the high school football team; he helped lead
a perennially losing team to an undefeated season in his senior year. He
played on the baseball team as well. During these years, he participated
in an anti-segregation sit-in at a Wilmington theatre. Academically, he
was an above-average student, was considered a natural leader among the
students, and was elected class president during his junior and senior
years. He graduated in 1961.

He earned his bachelor's in 1965 from the University of Delaware, with a
double major in history and political science, graduating with a class
rank of 506 out of 688. His classmates were impressed by his cramming
abilities, and he played halfback with the Blue Hens freshman football
team. In 1964, while on spring break in the Bahamas, he met and began
dating Neilia Hunter, who was from an affluent background in
Skaneateles, New York, and attended Syracuse University. He told her
that he aimed to become a senator by the age of 30 and then President.
He dropped a junior year plan to play for the varsity football team as a
defensive back, enabling him to spend more time visiting out of state
with her.

He then entered Syracuse University College of Law, receiving a half
scholarship based on financial need with some additional assistance
based on academics. By his own description, he found law school to be
"the biggest bore in the world" and pulled many all-nighters to get by.
During his first year there, he was accused of having plagiarized five
of 15 pages of a law review article. Biden said it was inadvertent due
to his not knowing the proper rules of citation, and he was permitted to
retake the course after receiving an "F" grade, which was subsequently
dropped from his record. This incident would later attract attention
when further plagiarism accusations emerged in 1987. He received his
Juris Doctor in 1968, graduating 76th of 85 in his class. Biden was
admitted to the Delaware bar in 1969.

Biden received student draft deferments during this period, at the peak
of the Vietnam War, and in 1968, he was reclassified by the Selective
Service System as not available for service due to having had asthma as
a teenager. He never took part in anti-war demonstrations, later saying
that at the time he was preoccupied with marriage and law school, and
"wore sports coats ... not tie-dyed".

Negative impressions of drinking alcohol in the Biden and Finnegan
families and in the neighborhood led to Biden becoming a teetotaler. He
suffered from stuttering through much of his childhood and into his
twenties, and says he overcame it by spending many hours reciting poetry
in front of a mirror.

\section{Early political career and family life
(1966--1972)}\label{early-political-career-and-family-life-19661972}

\begin{itemize}
\item
  \emph{On August 27, 1966, while Biden was still a law student, he
  married Neilia Hunter.}
\item
  \emph{He also started his own firm, Biden and Walsh.}
\item
  \emph{Balick named him to the Democratic Forum, a group trying to
  reform and revitalize the state party, and Biden switched his
  registration to Democrat.}
\item
  \emph{They had three children, Joseph R. "Beau" Biden III in 1969,
  Robert Hunter in 1970, and Naomi Christina in 1971.}
\item
  \emph{Biden represented the 4th district on the county council.}
\end{itemize}

On August 27, 1966, while Biden was still a law student, he married
Neilia Hunter. They overcame her parents' initial reluctance for her to
wed a Roman Catholic, and the ceremony was held in a Catholic church in
Skaneateles. They had three children, Joseph R. "Beau" Biden III in
1969, Robert Hunter in 1970, and Naomi Christina in 1971.

During 1968, Biden clerked for six months at a Wilmington law firm
headed by prominent local Republican William Prickett and, as he later
said, "thought of myself as a Republican". He disliked the conservative
racial politics of incumbent Democratic Governor of Delaware Charles L.
Terry and supported a more liberal Republican, Russell W. Peterson, who
defeated Terry in 1968. The local Republicans tried to recruit him, but
he resisted due to his distaste for Republican presidential candidate
Richard Nixon, and registered as an Independent instead.

In 1969, Biden resumed practicing law in Wilmington, first as a public
defender and then at a firm headed by Sid Balick, a locally active
Democrat. Balick named him to the Democratic Forum, a group trying to
reform and revitalize the state party, and Biden switched his
registration to Democrat. He also started his own firm, Biden and Walsh.
Corporate law, however, did not appeal to him and criminal law did not
pay well. He supplemented his income by managing properties.

Later in 1969, Biden ran as a Democrat for the New Castle County Council
on a liberal platform that included support for public housing in the
suburban area. He won by a solid, two-thousand vote margin in the
usually Republican district and in a bad year for Democrats in the
state. Even before taking his seat, he was already talking about running
for the U.S. Senate in a couple of years. He served on the County
Council from 1970 to 1972 while continuing his private law practice.
Biden represented the 4th district on the county council. Among issues
he addressed on the council was his opposition to large highway projects
that might disrupt Wilmington neighborhoods, including those related to
Interstate 95.

\section{1972 U.S. Senate campaign}\label{u.s.-senate-campaign}

\begin{itemize}
\item
  \emph{Biden's entry into the 1972 U.S. Senate election in Delaware
  presented a unique circumstance.}
\item
  \emph{Biden's campaign had virtually no money and was given no chance
  of winning.}
\end{itemize}

Biden's entry into the 1972 U.S. Senate election in Delaware presented a
unique circumstance. Longtime Delaware political figure and Republican
incumbent Senator J. Caleb Boggs was considering retirement, which would
likely have left U.S. Representative Pete du Pont and Wilmington Mayor
Harry G. Haskell Jr. in a divisive primary fight. To avoid that, U.S.
President Richard M. Nixon helped convince Boggs to run again with full
party support. No other Democrat wanted to run against Boggs. Biden's
campaign had virtually no money and was given no chance of winning. It
was managed by his sister Valerie Biden Owens (who would go on to manage
his future campaigns) and staffed by other family members, and relied
upon handed-out newsprint position papers and meeting voters
face-to-face; the small size of the state and lack of a major media
market made the approach feasible. He did receive some assistance from
the AFL--CIO and Democratic pollster Patrick Caddell. His campaign
issues focused on withdrawal from Vietnam, the environment, civil
rights, mass transit, more equitable taxation, health care, the public's
dissatisfaction with politics-as-usual, and "change". During the summer,
he trailed by almost 30~percentage points, but his energy level, his
attractive young family, and his ability to connect with voters'
emotions gave the surging Biden an advantage over the ready-to-retire
Boggs. He won the November 7, 1972 election in an upset by a margin of
3,162 votes.

\section{Family deaths}\label{family-deaths}

\begin{itemize}
\item
  \emph{Subsequent to the accident, Biden commented that the truck
  driver had been drinking alcohol before the collision, but these
  allegations were denied by the driver's family and were never
  substantiated by the police.}
\item
  \emph{Biden considered resigning to care for them, but was persuaded
  not to by Senate Majority Leader Mike Mansfield.}
\item
  \emph{Neilia Biden's station wagon was hit by a tractor-trailer as she
  pulled out from an intersection; the truck driver was cleared of any
  wrongdoing.}
\end{itemize}

On December 18, 1972, a few weeks after the election, Biden's wife and
one-year-old daughter Naomi were killed in an automobile accident while
Christmas shopping in Hockessin, Delaware. Neilia Biden's station wagon
was hit by a tractor-trailer as she pulled out from an intersection; the
truck driver was cleared of any wrongdoing. Biden's sons Beau and Hunter
survived the accident and were taken to the hospital in fair condition,
Beau with a broken leg and other wounds, and Hunter with a minor skull
fracture and other head injuries. Doctors soon said both would make full
recoveries. Biden considered resigning to care for them, but was
persuaded not to by Senate Majority Leader Mike Mansfield. Subsequent to
the accident, Biden commented that the truck driver had been drinking
alcohol before the collision, but these allegations were denied by the
driver's family and were never substantiated by the police.

\section{United States Senate
(1973--2009)}\label{united-states-senate-19732009}

\section{Recovery and new family}\label{recovery-and-new-family}

\begin{itemize}
\item
  \emph{In his memoirs, Biden notes that staffers were taking bets on
  how long he would last.}
\item
  \emph{Biden would credit her with renewing his interest in both
  politics and life.}
\item
  \emph{They had met on a blind date arranged by Biden's brother,
  although it turned out that Biden had already noticed a photograph of
  her in an advertisement for a local park in Wilmington, Delaware.}
\end{itemize}

Biden was sworn into office on January 5, 1973, by Francis R. Valeo, the
Secretary of the Senate in a small chapel at the Delaware Division of
the Wilmington Medical Center. Beau was wheeled in with his leg still in
traction; Hunter, who had already been released, was also there, as were
other members of the extended family. Witnesses and television cameras
were also present and the event received national attention.

At age 30 (the minimum age required to hold the office), Biden became
the sixth-youngest senator in U.S. history, and one of only 18 senators
who took office before reaching the age of 31. But the accident left him
filled with both anger and religious doubt: "I liked to {[}walk around
seedy neighborhoods{]} at night when I thought there was a better chance
of finding a fight ... I had not known I was capable of such rage ... I
felt God had played a horrible trick on me." To be at home every day for
his young sons, Biden began the practice of commuting every day by
Amtrak train for 90 minutes each way from his home in the Wilmington
suburbs to Washington, D.C., which he continued to do throughout his
Senate career. In the aftermath of the accident, he had trouble focusing
on work and appeared to just go through the motions of being a senator.
In his memoirs, Biden notes that staffers were taking bets on how long
he would last. A single father for five years, he left standing orders
that he be interrupted in the Senate at any time if his sons called. In
remembrance of his wife and daughter, Biden does not work on December
18, the anniversary of the accident.

Biden's elder son, Beau, became Delaware Attorney General and an Army
Judge Advocate who served in Iraq; his younger son, Hunter, became a
Washington attorney and lobbyist. On May 30, 2015, Beau died at the age
of 46 after a two-year battle with brain cancer. At the time of his
death, Beau had been widely seen as the frontrunner to be the Democratic
nominee for Governor of Delaware in 2016.

In 1975, Biden met Jill Tracy Jacobs, who grew up in Willow Grove,
Pennsylvania, and would become a teacher in Delaware. They had met on a
blind date arranged by Biden's brother, although it turned out that
Biden had already noticed a photograph of her in an advertisement for a
local park in Wilmington, Delaware. Biden would credit her with renewing
his interest in both politics and life. On June 17, 1977, Biden and
Jacobs were married by a Catholic priest at the Chapel at the United
Nations in New York. Jill Biden has a bachelor's degree from the
University of Delaware; two master's degrees, one from West Chester
University and the other from Villanova University; and a doctorate in
education from the University of Delaware. They have one daughter
together, Ashley Blazer (born 1981), who became a social worker and
staffer at the Delaware Department of Services for Children, Youth, and
Their Families. Biden and his wife are Roman Catholics and regularly
attend Mass at St. Joseph's on the Brandywine in Greenville, Delaware.

\includegraphics[width=5.50000in,height=3.09375in]{media/image2.jpg}\\
\emph{Biden with President Jimmy Carter in the Oval Office}

\includegraphics[width=5.50000in,height=3.09375in]{media/image3.jpg}\\
\emph{Senator Biden, Senator Frank Church and President of Egypt Anwar
el-Sadat after signing the Egyptian--Israeli Peace Treaty, 1979}

\section{Early Senate activities}\label{early-senate-activities}

\begin{itemize}
\item
  \emph{Biden became ranking minority member of the U.S. Senate
  Committee on the Judiciary in 1981.}
\item
  \emph{In mid-1974, freshman Senator Biden was named one of the 200
  Faces for the Future by Time magazine, in a profile that mentioned
  what had happened to his family and characterized Biden as
  "self-confident" and "compulsively ambitious".}
\end{itemize}

During his first years in the Senate, Biden focused on legislation
regarding consumer-protection and environmental issues and called for
greater accountability on the part of government. In mid-1974, freshman
Senator Biden was named one of the 200 Faces for the Future by Time
magazine, in a profile that mentioned what had happened to his family
and characterized Biden as "self-confident" and "compulsively
ambitious".

In 1974, Biden was one of the Senate's leading opponents of
desegregation busing, a then frequently court-ordered practice of
racially integrating schools. Biden said that he supported the aim of
school desegregation, but not the practice of busing, which his white
constituents bitterly opposed. Such opposition would later lead his
party to mostly abandon school desegregation policies.

Biden became ranking minority member of the U.S. Senate Committee on the
Judiciary in 1981. In 1984, he was Democratic floor manager for the
successful passage of the Comprehensive Crime Control Act. Over time,
the tough-on-crime provisions of the legislation became controversial on
the left and among criminal justice reform proponents, and in 2019,
Biden characterized his role in passing the legislation as a "big
mistake". Biden and his supporters praised him for modifying some of the
Act's worst provisions, and it was his most important legislative
accomplishment at that point in time. He first considered running for
president in that year, after he gained notice for giving speeches to
party audiences that simultaneously scolded and encouraged Democrats.

Regarding foreign policy, during his first decade in the Senate, Biden
focused on arms control issues. In response to the refusal of the U.S.
Congress to ratify the SALT II Treaty signed in 1979 by Soviet leader
Leonid Brezhnev and President Jimmy Carter, he took the initiative to
meet the Soviet Foreign Minister Andrei Gromyko, educated him about
American concerns and interests, and secured several changes to address
objections of the Foreign Relations Committee. When the Reagan
administration wanted to interpret the 1972 SALT I Treaty loosely in
order to allow the Strategic Defense Initiative to proceed, Biden argued
for strict adherence to the treaty's terms. He clashed again with the
Reagan administration in 1986 over economic sanctions against South
Africa; he received considerable attention when he excoriated Secretary
of State George P. Shultz at a Senate hearing because of the
administration's support of that country, which continued to practice
the apartheid system.

\section{1988 presidential campaign}\label{presidential-campaign}

\begin{itemize}
\item
  \emph{Biden had in fact cited Kinnock as the source for the
  formulation on previous occasions.}
\item
  \emph{Later in 1987, the Delaware Supreme Court's Board of
  Professional Responsibility cleared Biden of the law school plagiarism
  charges regarding his standing as a lawyer, saying Biden had "not
  violated any rules".}
\item
  \emph{Biden has had no recurrences or effects from the aneurysms since
  then.}
\item
  \emph{The hospitalization and recovery kept Biden from his duties in
  the U.S. Senate for seven months.}
\item
  \emph{While Biden's speech included the lines:}
\end{itemize}

Biden ran for the 1988 Democratic presidential nomination, formally
declaring his candidacy at the Wilmington train station on June 9, 1987.
He was attempting to become the youngest president since John F.
Kennedy. When the campaign began, he was considered a potentially strong
candidate because of his moderate image, his speaking ability on the
stump, his appeal to Baby Boomers, his high-profile position as chair of
the Senate Judiciary Committee at the upcoming Robert Bork Supreme Court
nomination hearings, and his fundraising appeal. He raised \$1.7~million
in the first quarter of 1987, more than any other candidate.

By August 1987, Biden's campaign, whose messaging was confused due to
staff rivalries, had begun to lag behind those of Michael Dukakis and
Dick Gephardt, although he had still raised more funds than all
candidates but Dukakis, and was seeing an upturn in Iowa polls. In
September 1987, the campaign ran into trouble when he was accused of
plagiarizing a speech that had been made earlier that year by Neil
Kinnock, leader of the British Labour Party. Kinnock's speech included
the lines:

While Biden's speech included the lines:

Biden had in fact cited Kinnock as the source for the formulation on
previous occasions. But he made no reference to the original source at
the August 23 Democratic debate at the Iowa State Fair being reported
on, nor in an August 26 interview for the National Education
Association. Moreover, while political speeches often appropriate ideas
and language from each other, Biden's use came under more scrutiny
because he fabricated aspects of his own family's background in order to
match Kinnock's. Biden was soon found to have earlier that year lifted
passages from a 1967 speech by Robert F. Kennedy (for which his aides
took the blame), and a short phrase from the 1961 inaugural address of
John F. Kennedy; and in two prior years to have done the same with a
1976 passage from Hubert H. Humphrey.

A few days later, Biden's plagiarism incident in law school came to
public light. Video was also released showing that when earlier
questioned by a New Hampshire resident about his grades in law school,
he had stated that he had graduated in the "top half" of his class, that
he had attended law school on a full scholarship, and that he had
received three degrees in college, each of which was untrue or
exaggerations of his actual record. Advisers and reporters pointed out
that he falsely claimed to have marched in the Civil Rights Movement.

The Kinnock and school revelations were magnified by the limited amount
of other news about the nomination race at the time, when most of the
public were not yet paying attention to any of the campaigns; Biden thus
fell into what The Washington Post writer Paul Taylor described as that
year's trend, a "trial by media ordeal". He lacked a strong demographic
or political group of support to help him survive the crisis. He
withdrew from the nomination race on September 23, 1987, saying his
candidacy had been overrun by "the exaggerated shadow" of his past
mistakes.

After Biden withdrew from the race, it was revealed that the Dukakis
campaign had secretly made a video highlighting the Biden--Kinnock
comparison and distributed it to news outlets. Later in 1987, the
Delaware Supreme Court's Board of Professional Responsibility cleared
Biden of the law school plagiarism charges regarding his standing as a
lawyer, saying Biden had "not violated any rules".

In February 1988, after suffering from several episodes of increasingly
severe neck pain, Biden was taken by long-distance ambulance to Walter
Reed Army Medical Center and given lifesaving surgery to correct an
intracranial berry aneurysm that had begun leaking; the situation was
serious enough that a priest had administered last rites at the
hospital. While recuperating, he suffered a pulmonary embolism, which
represented a major complication. Another operation to repair a second
aneurysm, which had caused no symptoms but was also at risk from
bursting, was performed in May 1988. The hospitalization and recovery
kept Biden from his duties in the U.S. Senate for seven months. Biden
has had no recurrences or effects from the aneurysms since then. In
retrospect, Biden's family came to believe that the early end to his
presidential campaign had been a blessing in disguise, for had he still
been campaigning in the midst of the primaries in early 1988, he might
well have not have stopped to seek medical attention and the condition
might have become unsurvivable.

\includegraphics[width=4.31106in,height=5.50000in]{media/image4.jpg}\\
\emph{Joe Biden, U.S. Senate photo}

\section{Senate Judiciary Committee}\label{senate-judiciary-committee}

\begin{itemize}
\item
  \emph{Biden was a long-time member of the U.S. Senate Committee on the
  Judiciary.}
\item
  \emph{Viewers of the high-profile hearings were often annoyed by
  Biden's style.}
\item
  \emph{Biden voted to acquit on both charges during the impeachment of
  President Clinton.}
\item
  \emph{The nomination was approved by a 52--48 vote in the full Senate,
  with Biden again opposed.}
\end{itemize}

Biden was a long-time member of the U.S. Senate Committee on the
Judiciary. He chaired it from 1987 until 1995 and he served as ranking
minority member on it from 1981 until 1987 and again from 1995 until
1997.

While chairman, Biden presided over two of the most contentious U.S.
Supreme Court confirmation hearings in history, those for Robert Bork in
1987 and Clarence Thomas in 1991. In the Bork hearings, he stated his
opposition to Bork soon after the nomination, reversing an approval in
an interview of a hypothetical Bork nomination he had made the previous
year and angering conservatives who thought he could not conduct the
hearings dispassionately. At the close, he won praise for conducting the
proceedings fairly and with good humor and courage, as his 1988
presidential campaign collapsed in the middle of the hearings. Rejecting
some of the less intellectually honest arguments that other Bork
opponents were making, Biden framed his discussion around the belief
that the U.S. Constitution provides rights to liberty and privacy that
extend beyond those explicitly enumerated in the text, and that Bork's
strong originalism was ideologically incompatible with that view. Bork's
nomination was rejected in the committee by a 9--5 vote, and then
rejected in the full Senate by a 58--42 margin.

In the Thomas hearings, Biden's questions on constitutional issues were
often long and convoluted, sometimes such that Thomas forgot the
question being asked. Viewers of the high-profile hearings were often
annoyed by Biden's style. Thomas later wrote that despite earlier
private assurances from the senator, Biden's questions had been akin to
a beanball. The nomination came out of the committee without a
recommendation, with Biden opposed. In part due to his own bad
experiences in 1987 with his presidential campaign, Biden was reluctant
to let personal matters enter into the hearings. Biden initially shared
with the committee, but not the public, Anita Hill's sexual harassment
charges, on the grounds she was not yet willing to testify. After she
did, Biden did not permit other witnesses to testify further on her
behalf, such as Angela Wright (who made a similar charge) and experts on
harassment. Biden said he was striving to preserve Thomas's right to
privacy and the decency of the hearings. The nomination was approved by
a 52--48 vote in the full Senate, with Biden again opposed. During and
afterward, Biden was strongly criticized by liberal legal groups and
women's groups for having mishandled the hearings and having not done
enough to support Hill. Biden subsequently sought out women to serve on
the Judiciary Committee and emphasized women's issues in the committee's
legislative agenda. In April 2019, Biden called Hill to express regret
over his treatment of her; after the conversation, Hill said that she
remained deeply unsatisfied.

Biden was involved in crafting many federal crime laws. He spearheaded
the Violent Crime Control and Law Enforcement Act of 1994, also known as
the Biden Crime Law, which included the Federal Assault Weapons Ban,
which expired in 2004 after its ten-year sunset period and was not
renewed. It also included the landmark Violence Against Women Act
(VAWA), which contains a broad array of measures to combat domestic
violence. In 2000, the Supreme Court ruled in United States v. Morrison
that the section of VAWA allowing a federal civil remedy for victims of
gender-motivated violence exceeded Congress's authority and therefore
was unconstitutional. Congress reauthorized VAWA in 2000 and 2005. Biden
has said, "I consider the Violence Against Women Act the single most
significant legislation that I've crafted during my 35-year tenure in
the Senate." In 2004 and 2005, Biden enlisted major American technology
companies in diagnosing the problems of the Austin, Texas-based National
Domestic Violence Hotline, and to donate equipment and expertise to it
in a successful effort to improve its services.

Biden was critical of the actions of Independent Counsel Kenneth Starr
during the 1990s Whitewater controversy and Lewinsky scandal
investigations, and said "it's going to be a cold day in hell" before
another Independent Counsel is granted the same powers. Biden voted to
acquit on both charges during the impeachment of President Clinton.

As chairman of the International Narcotics Control Caucus, Biden wrote
the laws that created the U.S. "Drug Czar", who oversees and coordinates
national drug control policy. In April 2003, he introduced the
controversial Reducing Americans' Vulnerability to Ecstasy Act, also
known as the RAVE Act. He continued to work to stop the spread of "date
rape drugs" such as flunitrazepam, and drugs such as Ecstasy and
Ketamine. In 2004, he worked to pass a bill outlawing steroids like
androstenedione, the drug used by many baseball players.

Biden's "Kids 2000" legislation established a public/private partnership
to provide computer centers, teachers, Internet access, and technical
training to young people, particularly to low-income and at-risk youth.

\includegraphics[width=5.50000in,height=3.09375in]{media/image5.jpg}\\
\emph{Senator Biden travels with President Clinton and other officials
to Bosnia in 1997}

\includegraphics[width=5.50000in,height=4.12500in]{media/image6.jpg}\\
\emph{Biden gives an opening statement and questions at a Senate Foreign
Relations Committee hearing on Iraq in 2007}

\section{Senate Foreign Relations
Committee}\label{senate-foreign-relations-committee}

\begin{itemize}
\item
  \emph{Biden stated: "This is bullshit.}
\item
  \emph{Biden later apologized for using the expletive.}
\item
  \emph{Biden was generally a liberal internationalist in foreign
  policy.}
\item
  \emph{Biden was also co-chairman of the NATO Observer Group in the
  Senate.}
\item
  \emph{Biden was a long-time member of the U.S. Senate Committee on
  Foreign Relations.}
\item
  \emph{Biden was a strong supporter of the 2001 war in Afghanistan,
  saying "Whatever it takes, we should do it."}
\end{itemize}

Biden was a long-time member of the U.S. Senate Committee on Foreign
Relations. In 1997, he became the ranking minority member and chaired
the committee in January 2001 and from June 2001 through 2003. When
Democrats re-took control of the Senate following the 2006 elections,
Biden again assumed the top spot on the committee in 2007. Biden was
generally a liberal internationalist in foreign policy. He collaborated
effectively with important Republican Senate figures such as Richard
Lugar and Jesse Helms and sometimes went against elements of his own
party. Biden was also co-chairman of the NATO Observer Group in the
Senate. A partial list covering this time showed Biden meeting with some
150 leaders from nearly 60 countries and international organizations.
Biden held frequent hearings as chairman of the committee, as well as
holding many subcommittee hearings during the three times he chaired the
Subcommittee on European Affairs.

Biden became interested in the Yugoslav Wars after hearing about Serbian
abuses during the Croatian War of Independence in 1991. Once the Bosnian
War broke out, Biden was among the first to call for the "lift and
strike" policy of lifting the arms embargo, training Bosnian Muslims and
supporting them with NATO air strikes, and investigating war crimes.
Both the George H. W. Bush administration and Clinton administration
were reluctant to implement the policy, fearing Balkan entanglement. In
April 1993, Biden spent a week in the Balkans and held a tense
three-hour meeting with Serbian leader Slobodan Milošević. Biden related
that he told Milošević, "I think you're a damn war criminal and you
should be tried as one." Biden wrote an amendment in 1992 to compel the
Bush administration to arm the Bosnians, but deferred in 1994 to a
somewhat softer stance preferred by the Clinton administration, before
signing on the following year to a stronger measure sponsored by Bob
Dole and Joe Lieberman. The engagement led to a successful NATO
peacekeeping effort. Biden has called his role in affecting Balkans
policy in the mid-1990s his "proudest moment in public life" that
related to foreign policy. In 1999, during the Kosovo War, Biden
supported the NATO bombing campaign against Serbia and Montenegro, and
co-sponsored with his friend John McCain the McCain-Biden Kosovo
Resolution, which called on President Clinton to use all necessary
force, including ground troops, to confront Milosevic over Serbian
actions in Kosovo. In 1998, Congressional Quarterly named Biden one of
"Twelve Who Made a Difference" for playing a lead role in several
foreign policy matters, including NATO enlargement and the successful
passage of bills to streamline foreign affairs agencies and punish
religious persecution overseas.

Biden had voted against authorization for the Gulf War in 1991, siding
with 45 of the 55 Democratic senators; he said the U.S. was bearing
almost all the burden in the anti-Iraq coalition. Biden was a strong
supporter of the 2001 war in Afghanistan, saying "Whatever it takes, we
should do it." Regarding Iraq, Biden stated in 2002 that Saddam Hussein
was a threat to national security, and that there was no option but to
eliminate that threat. In October 2002, Biden voted in favor of the
Authorization for Use of Military Force Against Iraq, justifying the
Iraq War. While he soon became a critic of the war and viewed his vote
as a "mistake", he did not push to require a U.S. withdrawal. He
supported the appropriations to pay for the occupation, but argued
repeatedly that the war should be internationalized, that more soldiers
were needed, and that the Bush administration should "level with the
American people" about the cost and length of the conflict.

By late 2006, Biden's stance had shifted, and he opposed the troop surge
of 2007, saying General David Petraeus was "dead, flat wrong" in
believing the surge could work. Biden was instead a leading advocate for
dividing Iraq into a loose federation of three ethnic states. In
November 2006, Biden and Leslie H. Gelb, President Emeritus of the
Council on Foreign Relations, released a comprehensive strategy to end
sectarian violence in Iraq. Rather than continuing the present approach
or withdrawing, the plan called for "a third way": federalizing Iraq and
giving Kurds, Shiites, and Sunnis "breathing room" in their own regions.
In September 2007, a non-binding resolution passed the Senate endorsing
such a scheme. However, the idea was unfamiliar, had no political
constituency, and failed to gain traction. Iraq's political leadership
united in denouncing the resolution as a de facto partitioning of the
country, and the U.S. Embassy in Baghdad issued a statement distancing
itself.

In March 2004, Biden secured the brief release of Libyan democracy
activist and political prisoner Fathi Eljahmi, after meeting with leader
Muammar Gaddafi in Tripoli. In May 2008, Biden sharply criticized
President George W. Bush for his speech to Israel's Knesset in which he
suggested that some Democrats were acting in the same way some Western
leaders did when they appeased Hitler in the runup to World War II.
Biden stated: "This is bullshit. This is malarkey. This is outrageous.
Outrageous for the president of the United States to go to a foreign
country, sit in the Knesset ... and make this kind of ridiculous
statement." Biden later apologized for using the expletive. Biden
further stated, "Since when does this administration think that if you
sit down, you have to eliminate the word 'no' from your vocabulary?"

\includegraphics[width=5.26533in,height=5.50000in]{media/image7.jpg}\\
\emph{Biden receiving a 1997 tour of a new facility at Delaware's Dover
Air Force Base}

\section{Delaware matters}\label{delaware-matters}

\begin{itemize}
\item
  \emph{In 2007, Biden requested and gained \$67~million worth of
  projects for his constituents through congressional earmarks.}
\item
  \emph{Biden held up trade agreements with Russia when that country
  stopped importing U.S. chickens.}
\item
  \emph{In 1975, Biden broke from liberal orthodoxy when he took
  legislative action to limit desegregation busing.}
\end{itemize}

Biden was a familiar figure to his Delaware constituency, by virtue of
his daily train commuting from there, and generally sought to attend to
state needs. Biden was a strong supporter of increased Amtrak funding
and rail security; he hosted barbecues and an annual Christmas dinner
for the Amtrak crews, and they would sometimes hold the last train of
the night a few minutes so he could catch it. He earned the nickname
"Amtrak Joe" as a result (and in 2011, Amtrak's Wilmington Station was
named the Joseph R. Biden Jr. Railroad Station, in honor of the over
7,000 trips he made from there). He was an advocate for Delaware
military installations, including Dover Air Force Base and New Castle
Air National Guard Base.

In 1975, Biden broke from liberal orthodoxy when he took legislative
action to limit desegregation busing. In doing so, he said busing was a
"bankrupt idea {[}that violated{]} the cardinal rule of common sense,"
and that his opposition would make it easier for other liberals to
follow suit. Three years later, Wilmington's federally mandated
cross-district busing plan generated much turmoil, and in trying to
legislate a compromise solution, Biden found himself alienating both
black and white voters for a while.

Beginning in 1991, Biden served as an adjunct professor at the Widener
University School of Law, Delaware's only law school, teaching a seminar
on constitutional law. The seminar was one of Widener's most popular,
often with a waiting list for enrollment. Biden typically co-taught the
course with another professor, taking on at least half the course
minutes and sometimes flying back from overseas to make one of the
classes.

Biden was a sponsor of bankruptcy legislation during the 2000s, which
was sought by MBNA, one of Delaware's largest companies, and other
credit card issuers. Biden allowed an amendment to the bill to increase
the homestead exemption for homeowners declaring bankruptcy and fought
for an amendment to forbid anti-abortion felons from using bankruptcy to
discharge fines; the overall bill was vetoed by Bill Clinton in 2000 but
then finally passed as the Bankruptcy Abuse Prevention and Consumer
Protection Act in 2005, with Biden supporting it.

Biden held up trade agreements with Russia when that country stopped
importing U.S. chickens. The downstate Sussex County region is the
nation's top chicken-producing area.

In 2007, Biden requested and gained \$67~million worth of projects for
his constituents through congressional earmarks.

\includegraphics[width=3.64467in,height=5.50000in]{media/image8.jpg}\\
\emph{Biden's official Senate photo (2005)}

\section{Reputation}\label{reputation}

\begin{itemize}
\item
  \emph{In May 1999, Biden became the youngest senator to cast 10,000
  votes.}
\item
  \emph{During his years as a senator, Biden amassed a reputation for
  loquaciousness, with his questions and remarks during Senate hearings
  being known as long-winded.}
\item
  \emph{Traub concludes that "Biden is the kind of fundamentally happy
  person who can be as generous toward others as he is to himself."}
\end{itemize}

Following his initial election in 1972, Biden was re-elected to six
additional terms, in the elections of 1978, 1984, 1990, 1996, 2002, and
2008, usually getting about 60~percent of the vote. He did not face
strong opposition; Pete du Pont, then governor, chose not to run against
him in 1984. Biden spent 28 years as a junior senator due to the
two-year seniority of his Republican colleague William V. Roth Jr. After
Roth was defeated for re-election by Tom Carper in 2000, Biden became
Delaware's senior senator. He then became the longest-serving senator in
Delaware history and, as of 2018{[}update{]}, was the 18th longest
serving senator in the history of the United States. In May 1999, Biden
became the youngest senator to cast 10,000 votes.

With a net worth between \$59,000 and \$366,000, and almost no outside
income or investment income, Biden was consistently ranked as one of the
least wealthy members of the Senate. Biden stated that he was listed as
the second-poorest member in Congress; he was not proud of the
distinction, but attributed it to having been elected early in his
career. Biden realized early in his senatorial career how vulnerable
poorer public officials are to offers of financial contributions in
exchange for policy support, and he pushed campaign finance reform
measures during his first term.

During his years as a senator, Biden amassed a reputation for
loquaciousness, with his questions and remarks during Senate hearings
being known as long-winded. He has been a strong speaker and debater and
a frequent and effective guest on Sunday morning talk shows. In public
appearances, he is known to deviate from prepared remarks at will.
According to political analyst Mark Halperin, he has shown "a persistent
tendency to say silly, offensive, and off-putting things"; The New York
Times writes that Biden's "weak filters make him capable of blurting out
pretty much anything". Journalist James Traub has written that "Biden's
vanity and his regard for his own gifts seem considerable even by the
rarefied standards of the U.S. Senate."

The political writer Howard Fineman has said, "Biden is not an academic,
he's not a theoretical thinker, he's a great street pol. He comes from a
long line of working people in Scranton---auto salesmen, car dealers,
people who know how to make a sale. He has that great Irish gift."
Political columnist David S. Broder has viewed Biden as having grown
since he came to Washington and since his failed 1988 presidential bid:
"He responds to real people---that's been consistent throughout. And his
ability to understand himself and deal with other politicians has gotten
much much better." Traub concludes that "Biden is the kind of
fundamentally happy person who can be as generous toward others as he is
to himself."

\section{2008 presidential campaign}\label{presidential-campaign-1}

\begin{itemize}
\item
  \emph{Biden made a formal announcement to Tim Russert on Meet the
  Press, stating he would "be the best Biden I can be".}
\item
  \emph{Biden made remarks during the campaign that attracted
  controversy.}
\item
  \emph{In particular, it changed the relationship between Biden and
  Obama.}
\item
  \emph{Now, having gotten to know each other during 2007, Obama
  appreciated Biden's campaigning style and appeal to working class
  voters, and Biden was convinced that Obama was "the real deal".}
\end{itemize}

Biden had thought about running for president again ever since his
failed 1988 bid.

Biden declared his candidacy for President on January 31, 2007, after
having discussed running for months prior. Biden made a formal
announcement to Tim Russert on Meet the Press, stating he would "be the
best Biden I can be". In January 2006, Delaware newspaper columnist
Harry F. Themal wrote that Biden "occupies the sensible center of the
Democratic Party". Themal concludes that this is the position Biden
desires, and that in a campaign "he plans to stress the dangers to the
security of the average American, not just from the terrorist threat,
but from the lack of health assistance, crime, and energy dependence on
unstable parts of the world".

During his campaign, Biden focused on the war in Iraq and his support
for the implementation of the Biden-Gelb plan to achieve political
success. He touted his record in the Senate as the head of major
congressional committees and his experience on foreign policy. Despite
speculation to the contrary, Biden rejected the notion of accepting the
position of Secretary of State, focusing only on the presidency. At a
2007 campaign event, Biden said, "I know a lot of my opponents out there
say I'd be a great Secretary of State. Seriously, every one of them. Do
you watch any of the debates? 'Joe's right, Joe's right, Joe's right.'"
Other candidates' comments that "Joe is right" in the Democratic debates
were converted into a Biden campaign theme and ad. In mid-2007, Biden
stressed his foreign policy expertise compared to Obama's, saying of the
latter, "I think he can be ready, but right now I don't believe he is.
The presidency is not something that lends itself to on-the-job
training." Biden also said that Obama was copying some of his foreign
policy ideas. Biden was noted for his one-liners on the campaign trail,
saying of Republican then-frontrunner Rudy Giuliani at the debate on
October 30, 2007, in Philadelphia, "There's only three things he
mentions in a sentence: a noun, and a verb and 9/11." Overall, Biden's
debate performances were an effective mixture of humor, and sharp and
surprisingly disciplined comments.

Biden made remarks during the campaign that attracted controversy. On
the day of his January 2007 announcement, he spoke of fellow Democratic
candidate and Senator Barack Obama: "I mean, you got the first
mainstream African-American who is articulate and bright and clean and a
nice-looking guy, I mean, that's a storybook, man." This comment
undermined his campaign as soon as it began and significantly damaged
his fund-raising capabilities; it later took second place on Time
magazine's list of Top 10 Campaign Gaffes for 2007. Biden had earlier
been criticized in July 2006 for a remark he made about his support
among Indian Americans: "I've had a great relationship. In Delaware, the
largest growth in population is Indian-Americans moving from India. You
cannot go to a 7-Eleven or a Dunkin' Donuts unless you have a slight
Indian accent. I'm not joking." Biden later said the remark was not
intended to be derogatory.

Overall, Biden had difficulty raising funds, struggled to draw people to
his rallies, and failed to gain traction against the high-profile
candidacies of Obama and Senator Hillary Clinton; he never rose above
single digits in the national polls of the Democratic candidates. In the
initial contest on January 3, 2008, Biden placed fifth in the Iowa
caucuses, garnering slightly less than one percent of the state
delegates. Biden withdrew from the race that evening, saying "There is
nothing sad about tonight. ... I feel no regret."

Despite the lack of success, Biden's stature in the political world rose
as the result of his 2008 campaign. In particular, it changed the
relationship between Biden and Obama. Although the two had served
together on the Senate Foreign Relations Committee, they had not been
close, with Biden having resented Obama's quick rise to political
stardom, and Obama having viewed Biden as garrulous and patronizing.
Now, having gotten to know each other during 2007, Obama appreciated
Biden's campaigning style and appeal to working class voters, and Biden
was convinced that Obama was "the real deal".

\section{2008 vice presidential
campaign}\label{vice-presidential-campaign}

\begin{itemize}
\item
  \emph{On November 4, 2008, Obama was elected President and Biden was
  elected Vice President of the United States.}
\item
  \emph{Privately, Obama was frustrated by Biden's remarks, saying "How
  many times is Biden gonna say something stupid?"}
\item
  \emph{Obama campaign staffers referred to Biden blunders as "Joe
  bombs" and kept Biden uninformed about strategy discussions, which in
  turn irked Biden.}
\end{itemize}

Since shortly following Biden's withdrawal from the presidential race,
Obama had been privately telling Biden that he was interested in finding
an important place for him in a possible Obama administration. Biden
declined Obama's first request to vet him for the vice presidential
slot, fearing the vice presidency would represent a loss in status and
voice from his Senate position, but subsequently changed his mind. In a
June 22, 2008, interview on NBC's Meet the Press, Biden confirmed that,
although he was not actively seeking a spot on the ticket, he would
accept the vice presidential nomination if offered. In early August,
Obama and Biden met in secret to discuss a possible vice-presidential
relationship, and the two developed a strong personal rapport. On August
22, 2008, Barack Obama announced that Biden would be his running mate.
The New York Times reported that the strategy behind the choice
reflected a desire to fill out the ticket with someone who has foreign
policy and national security experience---and not to help the ticket win
a swing state or to emphasize Obama's "change" message. Other observers
pointed out Biden's appeal to middle class and blue-collar voters, as
well as his willingness to aggressively challenge Republican nominee
John McCain in a way that Obama seemed uncomfortable doing at times. In
accepting Obama's offer, Biden ruled out to him the possibility of
running for president again in 2016 (although comments by Biden in
subsequent years seemed to back off that stance, with Biden not wanting
to diminish his political power by appearing uninterested in
advancement). Biden was officially nominated for vice president on
August 27 by voice vote at the 2008 Democratic National Convention in
Denver, Colorado.

After his selection as a vice presidential candidate, Biden was
criticized by his own Roman Catholic Diocese of Wilmington Bishop
Michael Saltarelli for not opposing abortion. The diocese confirmed that
even if elected vice president, Biden would not be allowed to speak at
Catholic schools. Biden was soon barred from receiving Holy Communion by
the bishop of his original hometown of Scranton, Pennsylvania, because
of his support for abortion rights; however, Biden did continue to
receive Communion at his local Delaware parish. Scranton became a flash
point in the competition for swing state Catholic voters between the
Democratic campaign and liberal Catholic groups, who stressed that other
social issues should be considered as much or more than abortion, and
many bishops and conservative Catholics, who maintained abortion was
paramount. Biden said he believed that life began at conception but that
he would not impose his personal religious views on others. Bishop
Saltarelli had previously stated regarding stances similar to Biden's:
"No one today would accept this statement from any public servant: 'I am
personally opposed to human slavery and racism but will not impose my
personal conviction in the legislative arena.' Likewise, none of us
should accept this statement from any public servant: 'I am personally
opposed to abortion but will not impose my personal conviction in the
legislative arena.'"

Biden's vice presidential campaigning gained little media visibility, as
far greater press attention was focused on the Republican running mate,
Alaskan Governor Sarah Palin. During one week in September 2008, for
instance, the Pew Research Center's Project for Excellence in Journalism
found that Biden was included in only five percent of the news coverage
of the race, far less than for the other three candidates on the
tickets. Biden nevertheless focused on campaigning in economically
challenged areas of swing states and trying to win over blue-collar
Democrats, especially those who had supported Hillary Clinton. Biden
attacked McCain heavily, despite a long-standing personal friendship; he
would say, "That guy I used to know, he's gone. It literally saddens
me." As the financial crisis of 2007--2010 reached a peak with the
liquidity crisis of September 2008 and the proposed bailout of the
United States financial system became a major factor in the campaign,
Biden voted in favor of the \$700~billion Emergency Economic
Stabilization Act of 2008, which passed the Senate 74--25.

On October 2, 2008, Biden participated in the campaign's one vice
presidential debate with Palin. Post-debate polls found that while Palin
exceeded many voters' expectations, Biden had won the debate overall. On
October 5, Biden suspended campaign events for a few days after the
death of his mother-in-law. During the final days of the campaign, Biden
focused on less-populated, older, less well-off areas of battleground
states, especially in Florida, Ohio, and Pennsylvania, where polling
indicated he was popular and where Obama had not campaigned or performed
well in the Democratic primaries. He also campaigned in some normally
Republican states, as well as in areas with large Catholic populations.

Under instructions from the Obama campaign, Biden kept his speeches
succinct and tried to avoid off-hand remarks, such as one about Obama's
being tested by a foreign power soon after taking office, which had
attracted negative attention. Privately, Obama was frustrated by Biden's
remarks, saying "How many times is Biden gonna say something stupid?"
Obama campaign staffers referred to Biden blunders as "Joe bombs" and
kept Biden uninformed about strategy discussions, which in turn irked
Biden. Relations between the two campaigns became strained for a month,
until Biden apologized on a call to Obama and the two built a stronger
partnership. Publicly, Obama strategist David Axelrod said that any
unexpected comments had been outweighed by Biden's high popularity
ratings. Nationally, Biden had a 60~percent favorability rating in a Pew
Research Center poll, compared to Palin's 44~percent.

On November 4, 2008, Obama was elected President and Biden was elected
Vice President of the United States. The Obama--Biden ticket won 365
Electoral College votes to McCain--Palin's 173, and had a 53--46~percent
edge in the nationwide popular vote.

Biden had continued to run for his Senate seat as well as for Vice
President, as permitted by Delaware law. On November 4, Biden was also
re-elected as senator, defeating Republican Christine O'Donnell.\\
Having won both races, Biden made a point of holding off his resignation
from the Senate so that he could be sworn in for his seventh term on
January 6, 2009. He became the youngest senator ever to start a seventh
full term, and said, "In all my life, the greatest honor bestowed upon
me has been serving the people of Delaware as their United States
senator." Biden cast his last Senate vote on January 15, supporting the
release of the second \$350~billion for the Troubled Asset Relief
Program. Biden resigned from the Senate later that day; in emotional
farewell remarks on the Senate floor, where he had spent most of his
adult life, Biden said, "Every good thing I have seen happen here, every
bold step taken in the 36-plus years I have been here, came not from the
application of pressure by interest groups, but through the maturation
of personal relationships."

\section{Vice Presidency (2009--2017)}\label{vice-presidency-20092017}

\includegraphics[width=5.50000in,height=3.84144in]{media/image9.jpg}\\
\emph{Vice President-elect Biden meets with Vice President Dick Cheney
at Number One Observatory Circle on November 13, 2008}

\section{Post-election transition}\label{post-election-transition}

\begin{itemize}
\item
  \emph{On November 4, 2008, Biden was elected Vice President of the
  United States as Obama's running mate.}
\item
  \emph{During the transition phase of the Obama administration, Biden
  said he was in daily meetings with Obama and that McCain was still his
  friend.}
\item
  \emph{Biden said he was closely involved in all the cabinet
  appointments that were made during the transition.}
\end{itemize}

On November 4, 2008, Biden was elected Vice President of the United
States as Obama's running mate.

Soon after the election, he was appointed chairman of President-elect
Obama's transition team. During the transition phase of the Obama
administration, Biden said he was in daily meetings with Obama and that
McCain was still his friend. The U.S. Secret Service codename given to
Biden is "Celtic", referencing his Irish roots.

Biden chose veteran Democratic lawyer and aide Ron Klain to be his chief
of staff, and Time Washington bureau chief Jay Carney to be his director
of communications. Biden intended to eliminate some of the explicit
roles assumed by the vice presidency of his predecessor, Dick Cheney,
who had established himself as an autonomous power center. Otherwise,
Biden said he would not emulate any previous vice presidency, but would
instead seek to provide advice and counsel on every critical decision
Obama would make. Biden said he was closely involved in all the cabinet
appointments that were made during the transition. Biden was also named
to head the new White House Task Force on Working Families, an
initiative aimed at improving the economic well being of the middle
class. In his last act as Chairman of the Foreign Relations Committee,
Biden went on a trip to Iraq, Afghanistan and Pakistan during the second
week of January 2009, meeting with the leadership of those countries.

\includegraphics[width=5.50000in,height=3.66667in]{media/image10.jpg}\\
\emph{Biden is sworn into office by Associate Justice John Paul Stevens,
January 20, 2009}

\includegraphics[width=5.50000in,height=3.66341in]{media/image11.jpg}\\
\emph{President Obama walking with Vice President Biden at the White
House, February 2009}

\includegraphics[width=5.50000in,height=3.68304in]{media/image12.jpg}\\
\emph{Biden speaks to Navy SEAL trainees, NAB Coronado, California, May
2009}

\includegraphics[width=5.50000in,height=3.66667in]{media/image13.jpg}\\
\emph{President Obama congratulates Biden for his role in shaping the
debt ceiling deal that led to the Budget Control Act of 2011.}

\includegraphics[width=5.50000in,height=3.66667in]{media/image14.jpg}\\
\emph{Biden, Obama and the U.S. national security team gathered in the
White House Situation Room to monitor the progress of the May 2011 U.S.
mission to kill Osama bin Laden.}

\section{First term (2009--2013)}\label{first-term-20092013}

\begin{itemize}
\item
  \emph{In the early months of the Obama administration, Biden assumed
  the role of an important behind-the-scenes counselor.}
\item
  \emph{More generally, overseeing Iraq policy became Biden's
  responsibility: the president is said to have put it as "Joe, you do
  Iraq".}
\item
  \emph{Biden has supported closer economic ties with Russia.}
\item
  \emph{Biden said Iraq "could be one of the great achievements of this
  administration".}
\end{itemize}

On January 20, 2009, at noon, Biden became the 47th Vice President of
the United States, sworn into the office by Supreme Court Justice John
Paul Stevens. Biden is the first United States Vice President from
Delaware and the first Roman Catholic to attain that office.

In the early months of the Obama administration, Biden assumed the role
of an important behind-the-scenes counselor. One role was to adjudicate
disputes between Obama's "team of rivals". The president compared
Biden's efforts to a basketball player "who does a bunch of things that
don't show up in the stat sheet". Biden played a key role in gaining
Senate support for several major pieces of Obama legislation, and was a
main factor in convincing Senator Arlen Specter to switch from the
Republican to the Democratic party. Biden lost an internal debate to
Secretary of State Hillary Clinton regarding his opposition to sending
21,000 new troops to the war in Afghanistan. His skeptical voice was
still considered valuable within the administration, however, and later
in 2009 Biden's views achieved more prominence within the White House as
Obama reconsidered his Afghanistan strategy.

Biden made visits to Iraq about once every two months, including trips
to Baghdad in August and September 2009 to listen to Prime Minister
Nouri al-Maliki and reiterate U.S. stances on Iraq's future; by this
time he had become the administration's point man in delivering messages
to Iraqi leadership about expected progress in the country. More
generally, overseeing Iraq policy became Biden's responsibility: the
president is said to have put it as "Joe, you do Iraq". Biden said Iraq
"could be one of the great achievements of this administration". Biden's
January 2010 visit to Iraq in the midst of turmoil over banned
candidates from the upcoming Iraqi parliamentary election resulted in 59
of the several hundred candidates being reinstated by the Iraqi
government two days later. By 2012, Biden had made eight trips there,
but his oversight of U.S. policy in Iraq receded with the exit in 2011
of U.S. troops.

Biden was also in charge of the oversight role for infrastructure
spending from the Obama stimulus package intended to help counteract the
ongoing recession, and stressed that only worthy projects should get
funding. He talked with hundreds of governors, mayors, and other local
officials in this role. During this period, Biden was satisfied that no
major instances of waste or corruption had occurred, and when he
completed that role in February 2011, he said that the number of fraud
incidents with stimulus monies had been less than one percent.

It took some time for the cautious Obama and the blunt, rambling Biden
to work out ways of dealing with each other. In late April 2009, Biden's
off-message response to a question during the beginning of the swine flu
outbreak, that he would advise family members against travelling on
airplanes or subways, led to a swift retraction from the White House.
The remark revived Biden's reputation for gaffes, and led to a spate of
late-night television jokes themed on him being a loose-talking buffoon.
In the face of persistently rising unemployment through July 2009, Biden
acknowledged that the administration had "misread how bad the economy
was" but maintained confidence that the stimulus package would create
many more jobs once the pace of expenditures picked up. The same month,
Secretary of State Clinton quickly disavowed Biden's remarks disparaging
Russia as a power, but despite any missteps, Biden still retained
Obama's confidence and was increasingly influential within the
administration. On March 23, 2010, a microphone picked up Biden telling
the president that his signing of the Patient Protection and Affordable
Care Act was "a big ... deal", using an adjective beginning with "f",
during live national news telecasts. White House press secretary Robert
Gibbs replied via Twitter "And yes Mr. Vice President, you're right ..."
Despite their different personalities, Obama and Biden formed a
friendship, partly based around Obama's daughter Sasha and Biden's
granddaughter Maisy, who attended Sidwell Friends School together.

Biden's most important role within the administration was to question
assumptions, playing a contrarian role. Obama said that "The best thing
about Joe is that when we get everybody together, he really forces
people to think and defend their positions, to look at things from every
angle, and that is very valuable for me." Another senior Obama advisor
said Biden "is always prepared to be the skunk at the family picnic to
make sure we are as intellectually honest as possible". On June 11,
2010, Biden represented the United States at the opening ceremony of the
World Cup, attended the England v. U.S. game which was tied 1--1, and
visited Egypt, Kenya, and South Africa.\\
Throughout, Joe and Jill Biden maintained a relaxed atmosphere at their
official residence in Washington, often entertaining some of their
grandchildren, and regularly returned to their home in Delaware.

Biden campaigned heavily for Democrats in the 2010 midterm elections,
maintaining an attitude of optimism in the face of general predictions
of large-scale losses for the party. Following large-scale Republican
gains in the elections and the departure of White House Chief of Staff
Rahm Emanuel, Biden's past relationships with Republicans in Congress
became more important. He led the successful administration effort to
gain Senate approval for the New START treaty. In December 2010, Biden's
advocacy within the White House for a middle ground, followed by his
direct negotiations with Senate Minority Leader Mitch McConnell, were
instrumental in producing the administration's compromise tax package
that revolved around a temporary extension of the Bush tax cuts. Biden
then took the lead in trying to sell the agreement to a reluctant
Democratic caucus in Congress, which was passed as the Tax Relief,
Unemployment Insurance Reauthorization, and Job Creation Act of 2010.

In foreign policy, Biden supported the NATO-led military intervention in
Libya in 2011. Biden has supported closer economic ties with Russia.

In March 2011, Obama detailed Biden to lead negotiations between both
houses of Congress and the White House in resolving federal spending
levels for the rest of the year, and avoiding a government shutdown. By
May 2011, a "Biden panel" with six congressional members was trying to
reach a bipartisan deal on raising the U.S. debt ceiling as part of an
overall deficit reduction plan. The U.S. debt ceiling crisis developed
over the next couple of months, but it was again Biden's relationship
with McConnell that proved to be a key factor in breaking a deadlock and
finally bringing about a bipartisan deal to resolve it, in the form of
the Budget Control Act of 2011, signed on August 2, 2011, the same day
that an unprecedented U.S. default had loomed. Biden had spent the most
time bargaining with Congress on the debt question of anyone in the
administration, and one Republican staffer said, "Biden's the only guy
with real negotiating authority, and {[}McConnell{]} knows that his word
is good. He was a key to the deal."

It has been reported that Biden was opposed to going forward with the
May 2011 U.S. mission to kill Osama bin Laden, lest failure adversely
affect Obama's chances for a second term. He took the lead in notifying
Congressional leaders of the successful outcome.

\includegraphics[width=3.83533in,height=5.50000in]{media/image15.jpg}\\
\emph{Biden with President Barack Obama, July 2012}

\section{2012 re-election campaign}\label{re-election-campaign}

\begin{itemize}
\item
  \emph{The incident showed that Biden still struggled at times with
  message discipline; as Time wrote, "everyone knows {[}that{]} Biden's
  greatest strength is also his greatest weakness."}
\item
  \emph{Biden apologized to Obama in private for having spoken out,
  while Obama acknowledged publicly it had been done from the heart.}
\end{itemize}

In October 2010, Biden stated that Obama had asked him to remain as his
running mate for the 2012 presidential election. With Obama's popularity
on the decline, however, in late 2011 White House Chief of Staff William
M. Daley conducted some secret polling and focus group research into the
idea of Secretary of State Clinton replacing Biden on the ticket. The
notion was dropped when the results showed no appreciable improvement
for Obama, and White House officials later said that Obama had never
entertained the idea.

Biden's May 2012 statement that he was "absolutely comfortable" with
same-sex marriage gained considerable public attention in comparison to
President Obama's position, which had been described as "evolving".
Biden made his statement without administration consent, and Obama and
his aides were quite irked, since Obama had planned to shift position
several months later, in the build-up to the party convention, and since
Biden had previously counseled the president to avoid the issue lest key
Catholic voters be offended. Gay rights advocates seized upon the Biden
stance, and within days, Obama announced that he too supported same-sex
marriage, an action in part forced by Biden's unexpected remarks. Biden
apologized to Obama in private for having spoken out, while Obama
acknowledged publicly it had been done from the heart. The incident
showed that Biden still struggled at times with message discipline; as
Time wrote, "everyone knows {[}that{]} Biden's greatest strength is also
his greatest weakness." Relations were also strained between the
campaigns when Biden appeared to use his to bolster fundraising contacts
for a possible run on his own in the 2016 presidential election, and the
vice president ended up being excluded from Obama campaign strategy
meetings.

The Obama campaign nevertheless still valued Biden as a retail-level
politician who could connect with disaffected, blue collar workers and
rural residents, and he had a heavy schedule of appearances in swing
states as the Obama re-election campaign began in earnest in spring
2012. An August 2012 remark before a mixed-race audience that proposed
Republican relaxation of Wall Street regulations would "put y'all back
in chains" led to a similar analysis of Biden's face-to-face campaigning
abilities versus tendency to go off track. The Los Angeles Times wrote,
"Most candidates give the same stump speech over and over, putting
reporters if not the audience to sleep. But during any Biden speech,
there might be a dozen moments to make press handlers cringe, and prompt
reporters to turn to each other with amusement and confusion." Time
magazine wrote that Biden often goes too far and that "Along with the
familiar Washington mix of neediness and overconfidence, Biden's brain
is wired for more than the usual amount of goofiness."

Biden was officially nominated for a second term as vice president on
September 6 by voice vote at the 2012 Democratic National Convention in
Charlotte, North Carolina. He faced his Republican counterpart,
Representative Paul Ryan, in the lone 2012 vice presidential debate on
October 11 in Danville, Kentucky. There he made a feisty, emotional
defense of the Obama administration's record and energetically attacked
the Republican ticket, in an effort to regain campaign momentum lost by
Obama's unfocused debate performance against Republican nominee Mitt
Romney the week before.

On November 6, 2012, the president and vice president were elected to
second terms. The Obama--Biden ticket won 332 Electoral College votes to
Romney--Ryan's 206 and had a 51--47~percent edge in the nationwide
popular vote.

\includegraphics[width=5.50000in,height=3.66667in]{media/image16.jpg}\\
\emph{Biden speaks during the U.S.--China Strategic and Economic
Dialogue, in Washington, D.C., July 2013}

\section{Post-election}\label{post-election}

\begin{itemize}
\item
  \emph{Later that month, during the final days before the country fell
  off the "fiscal cliff", Biden's relationship with McConnell once more
  proved important as the two negotiated a deal that led to the American
  Taxpayer Relief Act of 2012 being passed at the start of 2013.}
\item
  \emph{In December 2012, Biden was named by Obama to head the Gun
  Violence Task Force, created to address the causes of gun violence in
  the United States in the aftermath of the Sandy Hook Elementary School
  shooting.}
\end{itemize}

In December 2012, Biden was named by Obama to head the Gun Violence Task
Force, created to address the causes of gun violence in the United
States in the aftermath of the Sandy Hook Elementary School shooting.
Later that month, during the final days before the country fell off the
"fiscal cliff", Biden's relationship with McConnell once more proved
important as the two negotiated a deal that led to the American Taxpayer
Relief Act of 2012 being passed at the start of 2013. It made permanent
much of the Bush tax cuts but raised rates on upper income levels.

\section{Second term (2013--2017)}\label{second-term-20132017}

\begin{itemize}
\item
  \emph{Biden has a strong stance on sexual assault.}
\item
  \emph{Biden himself said that the U.S. would follow ISIL "to the gates
  of hell".}
\item
  \emph{Biden issued federal guidelines while presenting a speech at the
  University of New Hampshire.}
\item
  \emph{Biden's Violence Against Women Act was reauthorized again in
  2013.}
\item
  \emph{Biden favored arming Syria's rebel fighters.}
\end{itemize}

Biden was inaugurated to a second term in the early morning of January
20, 2013, at a small ceremony in his official residence with Justice
Sonia Sotomayor presiding (a public ceremony took place on January 21).
He continued to be in the forefront as, in the wake of the Sandy Hook
Elementary School shooting, the Obama administration put forth executive
orders and proposed legislation towards new gun control measures (the
legislation failed to pass).

During the discussions that led to the October 2013 passage of the
Continuing Appropriations Act, 2014, which resolved the U.S. federal
government shutdown of 2013 and the U.S. debt-ceiling crisis of 2013,
Biden played little role. This was due to Senate Majority Leader Harry
Reid and other Democratic leaders cutting the vice president out of any
direct talks with Congress, feeling that Biden had given too much away
during previous negotiations.

Biden's Violence Against Women Act was reauthorized again in 2013. The
act led to further related developments in the creation of the White
House Council on Women and Girls, begun in the first term, as well as
the White House Task Force to Protect Students from Sexual Assault,
begun in January 2014 with Biden as co-chair along with Jarrett. Biden
has a strong stance on sexual assault. For example, Biden stated to a
victim of sexual assault at Stanford University, "you did it ... in the
hope that your strength might prevent this crime from happening to
someone else. Your bravery is breathtaking." He has also taken legality
into the situation. Biden issued federal guidelines while presenting a
speech at the University of New Hampshire. He stated that "No means no,
if you're drunk or you're sober. No means no if you're in bed, in a dorm
or on the street. No means no even if you said yes at first and you
changed your mind. No means no."

Biden favored arming Syria's rebel fighters. As Iraq fell apart during
2014, renewed attention was paid to the Biden-Gelb Iraqi federalization
plan of 2006, with some observers suggesting that Biden had been right
all along. Biden himself said that the U.S. would follow ISIL "to the
gates of hell". In October 2014, Biden said that Turkey, Saudi Arabia
and the United Arab Emirates had "poured hundreds of millions of dollars
and tens of thousands of tons of weapons into anyone who would fight
against Al-Assad, except that the people who were being supplied were
al-Nusra, and al Qaeda, and the extremist elements of jihadis coming
from other parts of the world."

By 2015, a series of swearings-in and other events where Biden placed
his hands on women and girls and talked closely to them had attracted
the attention of both the press and social media. In one case, a senator
issued a statement afterward saying about his daughter, "No, she doesn't
think the vice president is creepy." On January 17, 2015, Secret Service
agents heard shots were fired as a vehicle drove near Biden's Delaware
residence at 8:28~p.m. outside the security perimeter, but the vice
president and his wife Jill were not home. A vehicle was observed by an
agent speeding away.

On December 8, 2015, Biden spoke in Ukraine's parliament in Kiev in one
of his many visits to set USA aid and policy stance for Ukraine.\\
On February 28, 2016, Biden gave a speech at the 88th Academy Awards to
do with awareness for sexual assault; he also introduced Lady Gaga.

On December 8, 2016, Biden went to Ottawa to meet with the Prime
Minister of Canada, Justin Trudeau.

During his two full terms, Joe Biden never cast a tie-breaking vote in
the Senate, making him the longest serving Vice President with this
distinction.

\section{Death of Beau Biden}\label{death-of-beau-biden}

\begin{itemize}
\item
  \emph{At the time of his death, Beau Biden had been widely seen as the
  frontrunner to be the Democratic nominee for Governor of Delaware in
  2016.}
\item
  \emph{In a statement, the Vice President's office said, "The entire
  Biden family is saddened beyond words."}
\item
  \emph{On May 30, 2015, Biden's son, Beau Biden, died at age 46 after
  having battled brain cancer for several years.}
\end{itemize}

On May 30, 2015, Biden's son, Beau Biden, died at age 46 after having
battled brain cancer for several years. In a statement, the Vice
President's office said, "The entire Biden family is saddened beyond
words." The nature and seriousness of the illness had not been
previously disclosed to the public, and Biden had quietly reduced his
public schedule in order to spend more time with his son. At the time of
his death, Beau Biden had been widely seen as the frontrunner to be the
Democratic nominee for Governor of Delaware in 2016.

\includegraphics[width=5.50000in,height=3.66667in]{media/image17.jpg}\\
\emph{Biden meeting with Vice President--elect Mike Pence on November
10, 2016}

\section{Role in the 2016 presidential
campaign}\label{role-in-the-2016-presidential-campaign}

\begin{itemize}
\item
  \emph{As of September~11, 2015{[}update{]}, Biden was still uncertain
  whether or not to run.}
\item
  \emph{After Obama endorsed Hillary Clinton on June 9, 2016, Biden
  endorsed her later the same day.}
\item
  \emph{As of the end of January 2016, neither Biden nor President
  Barack Obama had endorsed any candidate in the 2016 presidential
  election.}
\end{itemize}

During much of his second term, Biden was said to be preparing for a
possible bid for the 2016 Democratic presidential nomination. At age 74
on Inauguration Day in January 2017, he would have been the oldest
president on inauguration in history. With his family, many friends, and
donors encouraging him in mid-2015 to enter the race, and with Hillary
Clinton's favorability ratings in decline at that time, Biden was
reported to again be seriously considering the prospect and a "Draft
Biden 2016" PAC was established.

As of September~11, 2015{[}update{]}, Biden was still uncertain whether
or not to run. Biden cited the recent death of his son being a large
drain on his emotional energy, and that "nobody has a right ... to seek
that office unless they're willing to give it 110\% of who they are".

On October 21, speaking from a podium in the Rose Garden with his wife
and President Obama by his side, Biden announced his decision not to
enter the race for the Democratic presidential nomination in the 2016
election. In January 2016, Biden affirmed that not running was the right
decision, but admitted to regretting not running for President "every
day."

As of the end of January 2016, neither Biden nor President Barack Obama
had endorsed any candidate in the 2016 presidential election. Biden did
miss his annual Thanksgiving tradition of going to Nantucket, opting
instead to travel abroad and meet with several European leaders. He took
time to meet with Martin O'Malley, having previously met with Bernie
Sanders, both 2016 candidates. Neither of these meetings was considered
an endorsement, as Biden had said that he would meet with any candidate
who asked.

After Obama endorsed Hillary Clinton on June 9, 2016, Biden endorsed her
later the same day. Though Biden and Clinton were scheduled to campaign
together in Scranton on July 8, the appearance was canceled by Clinton
in light of the shooting of Dallas police officers the previous day.

Following his endorsement of Clinton, Biden publicly displayed his
disagreements with the policies of Republican presidential nominee
Donald Trump. On June 20, Biden critiqued Trump's proposal to
temporarily ban Muslims from entering the country as well as his stated
intent to build a wall between the United States and Mexico border,
furthering that Trump's suggestion to either torture and or kill family
members of terrorists was both damaging to American values and "deeply
damaging to our security". During an interview with George
Stephanopoulos at the 2016 Democratic National Convention on July 26,
Biden asserted that "moral and centered" voters would not vote for
Trump. On October 21, the anniversary of his choice to not run, Biden
said he wished he was still in high school so he could take Trump
"behind the gym". On October 24, Biden clarified he would only have
fought Trump if he was still in high school, and the following day,
October 25, Trump responded that he would "love that".

\section{Post--Vice Presidency
(2017--present)}\label{postvice-presidency-2017present}

\begin{itemize}
\item
  \emph{He opened with "My name's Joe Biden.}
\item
  \emph{In 2018, Sen. McCain died at the age of 81 after dealing with
  the same cancer that Joe Biden's late son Beau Biden died of.}
\item
  \emph{Biden gave the eulogy at McCain's funeral service in Phoenix,
  Arizona.}
\item
  \emph{Biden had been close friends with Sen. John McCain for over 30
  years.}
\end{itemize}

In 2017, Biden was named the Benjamin Franklin Presidential Practice
professor at the University of Pennsylvania, where he intended to focus
on foreign policy, diplomacy, and national security while leading the
Penn Biden Center for Diplomacy and Global Engagement. He also wanted to
pursue his "cancer moonshot" agenda, calling the fight against cancer
"the only bipartisan thing left in America" in March 2017.

Biden had been close friends with Sen. John McCain for over 30 years. In
2018, Sen. McCain died at the age of 81 after dealing with the same
cancer that Joe Biden's late son Beau Biden died of. Biden gave the
eulogy at McCain's funeral service in Phoenix, Arizona. He opened with
"My name's Joe Biden. I'm a Democrat. And I loved John McCain.", he also
called him a "brother". Biden also served as a pallbearer at Sen.
McCain's memorial service at the Washington National Cathedral alongside
Warren Beatty, and Michael Bloomberg.

\section{Comments on President Trump}\label{comments-on-president-trump}

\begin{itemize}
\item
  \emph{In October 2018, Biden said if Democrats retake the House of
  Representatives, "I hope they don't {[}impeach Trump{]}.}
\item
  \emph{While attending the launch of the Penn Biden Center for
  Diplomacy and Global Engagement on March 30, 2017, a student asked
  Biden what "piece of advice" he would give to President Trump, Biden
  responding that the president should grow up and cease his tweeting so
  he could focus on the office.}
\end{itemize}

While attending the launch of the Penn Biden Center for Diplomacy and
Global Engagement on March 30, 2017, a student asked Biden what "piece
of advice" he would give to President Trump, Biden responding that the
president should grow up and cease his tweeting so he could focus on the
office. During a speech at a May 29, 2017 gathering of Philip D. Murphy
supporters at a community center gymnasium, Biden said, "There are a lot
of people out there who are frightened. Trump played on their fears.
What we haven't done, in my view---and this is a criticism of all
us---we haven't spoken enough to the fears and aspirations of the people
we come from." On June 17, 2017, Biden predicted the "state the nation
is today will not be sustained by the American people" while speaking at
a Florida Democratic Party fundraiser in Hollywood. Biden told CBS This
Morning that Trump's administration "seems to feel the need to coddle
autocrats and dictators" like Saudi Arabian leaders, Russian President
Putin, North Korean leader Kim Jong-un or Philippine President Rodrigo
Duterte. In October 2018, Biden said if Democrats retake the House of
Representatives, "I hope they don't {[}impeach Trump{]}. I don't think
there's a basis for doing that right now."

\section{Climate change}\label{climate-change}

\begin{itemize}
\item
  \emph{The following day, after President Trump announced his
  withdrawal of the US from the Paris Agreement, Biden tweeted that the
  choice "imperils US security and our ability to own the clean energy
  future."}
\item
  \emph{During an appearance at the Brainstorm Health Conference in San
  Diego, California on May 2, 2017, Biden said the public "has moved
  ahead of the administration {[}on science{]}".}
\end{itemize}

During an appearance at the Brainstorm Health Conference in San Diego,
California on May 2, 2017, Biden said the public "has moved ahead of the
administration {[}on science{]}". On May 31, Biden tweeted that climate
change was an "existential threat to our future" and remaining in the
Paris Agreement was the "best way to protect our children and global
leadership." The following day, after President Trump announced his
withdrawal of the US from the Paris Agreement, Biden tweeted that the
choice "imperils US security and our ability to own the clean energy
future." While appearing at the Concordia Europe Summit in Athens,
Greece on June 7, Biden said, referring to the Paris Agreement, "The
vast majority of the American people do not agree with the decision the
president made."

\section{Healthcare}\label{healthcare}

\begin{itemize}
\item
  \emph{On May 4, after the House of Representatives narrowly voted for
  the American Health Care Act, Biden tweeted that it was a "Day of
  shame for Congress", lamenting the loss of pre-existing condition
  protections.}
\item
  \emph{On July 28, in response to the Republican Senate healthcare bill
  falling through, Biden tweeted, "Thank you to everyone who tirelessly
  worked to protect the healthcare of millions."}
\end{itemize}

On March 22, 2017, Biden referred to the Republican healthcare bill as a
"tax bill" meant to transfer nearly US\$1 trillion used for health
benefits for the lower classes to wealthy Americans during his first
appearance on Capitol Hill since Trump's inauguration.\\
On May 4, after the House of Representatives narrowly voted for the
American Health Care Act, Biden tweeted that it was a "Day of shame for
Congress", lamenting the loss of pre-existing condition protections. On
June 24, in response to Senate Republicans revealing an American Health
Care Act draft the previous day, Biden tweeted that the bill "isn't
about health care at all---it's a wealth transfer: slashes care to fund
tax cuts for the wealthy \& corporations". On July 28, in response to
the Republican Senate healthcare bill falling through, Biden tweeted,
"Thank you to everyone who tirelessly worked to protect the healthcare
of millions."

\section{Immigration}\label{immigration}

\begin{itemize}
\item
  \emph{On September 5, 2017, after Attorney General Jeff Sessions
  announced that the Trump Administration is rescinding the Deferred
  Action for Childhood Arrivals, Biden tweeted, "Brought by parents,
  these children had no choice in coming here.}
\end{itemize}

On September 5, 2017, after Attorney General Jeff Sessions announced
that the Trump Administration is rescinding the Deferred Action for
Childhood Arrivals, Biden tweeted, "Brought by parents, these children
had no choice in coming here. Now they'll be sent to countries they've
never known. Cruel. Not America."

\section{LGBT rights}\label{lgbt-rights}

\begin{itemize}
\item
  \emph{On June 21, during a speech at a Democratic National Committee
  LGBT gala in New York City, Biden said, "Hold President Trump
  accountable for his pledge to be your friend."}
\item
  \emph{On July 26, 2017, after Trump announced a ban of transgender
  people serving in the military, Biden tweeted, "Every patriotic
  American who is qualified to serve in our military should be able to
  serve.}
\end{itemize}

On April 14, 2017, Biden released a statement both denouncing Chechnya
authorities for their rounding up, torturing, and murdering of
"individuals who are believed to be gay" and stating his hope that the
Trump administration honor a prior pledge to advance human rights by
confronting Chechen leader Ramzan Kadyrov and Russian leaders over
"these egregious violations of human rights". On June 21, during a
speech at a Democratic National Committee LGBT gala in New York City,
Biden said, "Hold President Trump accountable for his pledge to be your
friend."

On July 26, 2017, after Trump announced a ban of transgender people
serving in the military, Biden tweeted, "Every patriotic American who is
qualified to serve in our military should be able to serve. Full stop."

\section{2020 presidential campaign}\label{presidential-campaign-2}

\begin{itemize}
\item
  \emph{In May 2019, Biden chose Philadelphia to be his 2020 U.S.
  presidential campaign headquarters.}
\item
  \emph{Between 2016 and 2019, Biden was mentioned by various media
  outlets as a potential candidate.}
\item
  \emph{A political action committee known as Time for Biden was formed
  in January 2018, seeking Biden's entry into the race.}
\end{itemize}

During a tour of the U.S. Senate with reporters before leaving office,
on December 5, 2016, Biden refused to rule out a bid for the presidency
in the 2020 presidential election, after leaving office as Vice
President. If he were to run in 2020, Biden would be 77 years old on
election day and 78 on inauguration day in 2021. He reasserted his
ambivalence about running on an appearance of The Late Show with Stephen
Colbert on December 7, in which he stated "never say never" about
running for President in 2020, while also acknowledging that he did not
see a scenario in which he would run for office again. He seemingly
announced on January 13, 2017, exactly one week prior to the expiration
of his vice presidential term, that he would not run. He then appeared
to backtrack four days later, on January 17, stating "I'll run if I can
walk." A political action committee known as Time for Biden was formed
in January 2018, seeking Biden's entry into the race.

Between 2016 and 2019, Biden was mentioned by various media outlets as a
potential candidate. He told a forum held in Bogota, Colombia, on July
17, 2018, that he would decide whether or not to formally declare as a
candidate by January 2019. On February 4, 2019, with no decision having
been forthcoming from Biden, Edward-Isaac Dovere of The Atlantic wrote
that Biden was "very close to saying yes" but that some close to him are
worried he would have a last-minute change of heart, as he did in 2016.
Dovere reported that Biden was concerned about the effect another
presidential run could have on his family and reputation, as well as
fundraising struggles and perceptions about his age and relative
centrism, compared to other declared and potential candidates.
Conversely, his "sense of duty," offense at the Trump presidency, the
lack of foreign policy experience amongst other Democratic hopefuls and
his desire to foster "bridge-building progressivism" in the Party, were
said to be factors prompting him to run. In March 2019, he indicated he
may run, and ultimately launched his campaign on April 25, 2019. In May
2019, Biden chose Philadelphia to be his 2020 U.S. presidential campaign
headquarters.

While at a fundraiser on June 18, 2019, Biden said that one of his
greatest strengths was "bringing people together" and pointed to his
relationships with Senators James Eastland and Herman Talmadge, two
segregationists as examples. While imitating a Southern drawl, Biden
remarked "I was in a caucus with James O. Eastland. He never called me
'boy,' he always called me 'son.''' New Jersey Senator Cory Booker was
one of many Democrats to criticize Biden for the remarks, issuing a
statement that said "You don't joke about calling black men 'boys.' Men
like James O. Eastland used words like that, and the racist policies
that accompanied them, to perpetuate white supremacy and strip black
Americans of our very humanity".

\section{Allegations of inappropriate physical
contact}\label{allegations-of-inappropriate-physical-contact}

\begin{itemize}
\item
  \emph{Biden has described himself as a "tactile politician" and
  admitted that this behavior has caused trouble for him in the past.}
\item
  \emph{Biden's spokesman stated that Biden did not recall the behavior
  described.}
\item
  \emph{Carter defended Biden's depicted behavior in a 2019 interview.}
\end{itemize}

There have been multiple photographs and videos of Biden engaged in what
commentators considered to be inappropriate proximity to women and
children, including kissing and or touching. Biden has described himself
as a "tactile politician" and admitted that this behavior has caused
trouble for him in the past. An image of Biden in close proximity to
Stephanie Carter during her husband's swearing in as Secretary of
Defense in 2015 resulted in a mocking epithet that was widely repeated.
Carter defended Biden's depicted behavior in a 2019 interview.

In March 2019, former Nevada assemblywoman Lucy Flores alleged that
Biden kissed her without her consent at a 2014 campaign rally in Las
Vegas. In a New York magazine op-ed for The Cut, Flores wrote that Biden
had walked up behind her, put his hands on her shoulders, smelled her
hair, and kissed the back of her head. Adding that the way he touched
her was "an intimate way reserved for close friends, family, or romantic
partners -- and I felt powerless to do anything about it." In an
interview with HuffPost, Flores stated she believed Biden's behavior to
be disqualifying for a 2020 presidential run. Biden's spokesman stated
that Biden did not recall the behavior described. Two days after Flores,
Amy Lappos, a former congressional aide to Jim Himes, said Biden crossed
a line of decency and respect when he touched her in a non-sexual, but
inappropriate way by holding her head to rub noses with her at a
political fundraiser in Greenwich in 2009. The next day, two additional
women came forward with allegations of inappropriate conduct. One woman
said that Biden placed his hand on her thigh, and the other said he ran
his hand from her shoulder down her back.

By early April 2019, a total of seven women had made allegations of
inappropriate physical contact regarding Biden. At a conference on April
5, Biden apologized for not understanding how individuals would react to
his actions, but stated that his intentions were honorable; he went on
to say that he was not sorry for anything that he had ever done, which
led critics to accuse him of sending a mixed message. He also
proclaimed---with each public embrace he gave during the event---that he
had received permission for it. Some critics interpreted this as Biden
jokingly deflecting criticism, while other observers considered his
change in tone responsive to the criticisms received.

\section{Political positions}\label{political-positions}

\begin{itemize}
\item
  \emph{Biden has been characterized as a moderate Democrat.}
\item
  \emph{Biden believes action must be taken on global warming.}
\item
  \emph{Various advocacy groups have given Biden scores or grades as to
  how well his votes align with the positions of each group.}
\item
  \emph{Biden has a lifetime liberal 72 percent score from the ADA
  through 2004, while the ACU awarded Biden a lifetime conservative
  rating of 13 percent through 2008.}
\end{itemize}

Biden has been characterized as a moderate Democrat. He has supported
deficit spending for fiscal stimulus in the American Recovery and
Reinvestment Act of 2009; the increased infrastructure spending proposed
by the Obama administration; mass transit, including Amtrak, bus, and
subway subsidies; same-sex marriage; and the reduced military spending
proposed in the Obama Administration's fiscal year 2014 budget.

A method that political scientists use for gauging ideology is to
compare the annual ratings by the Americans for Democratic Action (ADA)
with the ratings by the American Conservative Union (ACU). Biden has a
lifetime liberal 72 percent score from the ADA through 2004, while the
ACU awarded Biden a lifetime conservative rating of 13 percent through
2008. Using another metric, Biden has a lifetime average liberal score
of 77.5 percent, according to a National Journal analysis that places
him ideologically among the center of Senate Democrats as of 2008. The
Almanac of American Politics rates congressional votes as liberal or
conservative on the political spectrum, in three policy areas: economic,
social, and foreign. For 2005--2006, Biden's average ratings were as
follows: the economic rating was 80~percent liberal and 13~percent
conservative, the social rating was 78~percent liberal and 18~percent
conservative, and the foreign rating was 71~percent liberal and
25~percent conservative. This has not changed much over time; his
liberal ratings in the mid-1980s were also in the 70--80~percent range.

Various advocacy groups have given Biden scores or grades as to how well
his votes align with the positions of each group. The American Civil
Liberties Union gives him an 80 percent lifetime score, with a 91
percent score for the 110th Congress. Biden opposes drilling for oil in
the Arctic National Wildlife Refuge and supports governmental funding to
find new energy sources. Biden believes action must be taken on global
warming. He co-sponsored the Sense of the Senate resolution calling on
the United States to be a part of the United Nations climate
negotiations and the Boxer--Sanders Global Warming Pollution Reduction
Act, the most stringent climate bill in the United States Senate. Biden
was given an 85~percent lifetime approval rating from the AFL--CIO, and
he voted for the North American Free Trade Agreement (NAFTA).

\includegraphics[width=5.50000in,height=3.66341in]{media/image18.jpg}\\
\emph{President Barack Obama presents Vice President Joe Biden with the
Presidential Medal of Freedom with Distinction during a tribute to the
Vice President in the State Dining Room of the White House, January 12,
2017.}

\section{Distinctions}\label{distinctions}

\begin{itemize}
\item
  \emph{On January 12, 2017, Obama surprised Biden by awarding him the
  Presidential Medal of Freedom with Distinction during a farewell press
  conference at the White House honoring Biden and his wife.}
\item
  \emph{On December 11, 2018, the University of Delaware renamed their
  School of Public Policy and Administration after Biden, naming it the
  Joseph R. Biden, Jr. School of Public Policy and Administration, which
  also houses the Biden Institute.}
\end{itemize}

Biden has received honorary degrees from the University of Scranton
(1976), Saint Joseph's University (LL.D 1981), Widener University School
of Law (2000), Emerson College (2003), his alma mater the University of
Delaware (2004), Suffolk University Law School (2005), and his other
alma mater Syracuse University (LL.D 2009) University of Pennsylvania
(LL.D 2013) Miami Dade College (2014) Trinity College, Dublin (LL.D
2016) Colby College (LL.D 2017) Morgan State University (DPS 2017)
University of South Carolina (DPA 2017)

Biden also received the Chancellor Medal from his alma mater, Syracuse
University, in 1980, and in 2005, he received the George Arents Pioneer
Medal---Syracuse's highest alumni award---"for excellence in public
affairs."

In 2008, Biden received the Best of Congress Award, for "improving the
American quality of life through family-friendly work policies", from
Working Mother magazine. Also in 2008, Biden shared with fellow Senator
Richard Lugar the Hilal-i-Pakistan award from the Government of Pakistan
"in recognition of their consistent support for Pakistan". In 2009,
Biden received the Golden Medal of Freedom award from Kosovo, that
region's highest award, for his vocal support for their independence in
the late 1990s.

Biden is an inductee of the Delaware Volunteer Firemen's Association
Hall of Fame. He was named to the Little League Hall of Excellence in
2009.

On June 25, 2016, Joe Biden received the freedom of County Louth in the
Republic of Ireland.

On January 12, 2017, Obama surprised Biden by awarding him the
Presidential Medal of Freedom with Distinction during a farewell press
conference at the White House honoring Biden and his wife. Obama said he
was awarding the Medal of Freedom to Biden for "faith in your fellow
Americans, for your love of country and a lifetime of service that will
endure through the generations". It was the first and only time Obama
awarded the Medal of Freedom with the additional honor of distinction,
an honor which his three predecessors had reserved only for President
Ronald Reagan, Colin Powell and Pope John Paul II, respectively.

On December 11, 2018, the University of Delaware renamed their School of
Public Policy and Administration after Biden, naming it the Joseph R.
Biden, Jr. School of Public Policy and Administration, which also houses
the Biden Institute.

\section{Electoral history}\label{electoral-history}

\section{Writings by Biden}\label{writings-by-biden}

\section{Notes}\label{notes}

\section{References}\label{references}

\section{Footnotes}\label{footnotes}

\section{Books}\label{books}

\section{External links}\label{external-links}

\begin{itemize}
\item
  \emph{Joe Biden at Curlie}
\item
  \emph{Joe Biden on IMDb}
\item
  \emph{Joe Biden President campaign website}
\end{itemize}

Joe Biden President campaign website

Archive of Obama White House official biography

Appearances on C-SPAN

Joe Biden at Curlie

Senate campaign website (archived)

Biography at the Biographical Directory of the United States Congress

Financial information (federal office) at the Federal Election
Commission

Joe Biden on IMDb

In the Words of Joe Biden---slideshow by Life magazine
