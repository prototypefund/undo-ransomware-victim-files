\textbf{From Wikipedia, the free encyclopedia}

https://en.wikipedia.org/wiki/The\%20Cincinnati\%20Enquirer\\
Licensed under CC BY-SA 3.0:\\
https://en.wikipedia.org/wiki/Wikipedia:Text\_of\_Creative\_Commons\_Attribution-ShareAlike\_3.0\_Unported\_License

\section{The Cincinnati Enquirer}\label{the-cincinnati-enquirer}

\begin{itemize}
\item
  \emph{The Cincinnati Enquirer is a morning daily newspaper published
  by Gannett Company in Cincinnati, Ohio, United States.}
\item
  \emph{First published in 1841, the Enquirer is the last remaining
  daily newspaper in Greater Cincinnati and Northern Kentucky, although
  the daily Journal-News competes with the Enquirer in the northern
  suburbs.}
\end{itemize}

The Cincinnati Enquirer is a morning daily newspaper published by
Gannett Company in Cincinnati, Ohio, United States. First published in
1841, the Enquirer is the last remaining daily newspaper in Greater
Cincinnati and Northern Kentucky, although the daily Journal-News
competes with the Enquirer in the northern suburbs. The Enquirer has the
highest circulation of any print publication in the Cincinnati
metropolitan area. A daily local edition for Northern Kentucky is
published as The Kentucky Enquirer.

The Enquirer won the 2018 Pulitzer Prize for local reporting for its
project titled "Seven Days of Heroin."

In addition to the Cincinnati Enquirer and Kentucky Enquirer, Gannett
publishes a variety of print and electronic periodicals in the
Cincinnati area, including 16 Community Press weekly newspapers, 10
Community Recorder weekly newspapers, and OurTown magazine. The Enquirer
is available online at the Cincinnati.com website.

\section{Content}\label{content}

\begin{itemize}
\item
  \emph{The Enquirer is regarded as a conservative, Republican-leaning
  newspaper, in contrast to The Cincinnati Post, a former competing
  daily.}
\item
  \emph{The Kentucky Enquirer consists of an additional section wrapped
  around the Cincinnati Enquirer and a remade Local section.}
\item
  \emph{In 2016, the Enquirer launched a true crime podcast called
  Accused that reached the top of iTunes' podcasts chart.}
\end{itemize}

The Enquirer is regarded as a conservative, Republican-leaning
newspaper, in contrast to The Cincinnati Post, a former competing daily.
From 1920 to 2012, the editorial board endorsed every Republican
candidate for United States president. By contrast, the current
editorial board claims to take a pragmatic editorial stance. According
to editor Peter Bhatia, "It is made up of pragmatic, solution-driven
members who, frankly, don't have much use for extreme ideologies from
the right or the left. ... The board's mantra in our editorials has been
about problem-solving and improving the quality of life for everyone in
greater Cincinnati." On September 24, 2016, the Enquirer endorsed
Hillary Clinton for president, its first endorsement of a Democrat for
president since Woodrow Wilson in 1916.

The Kentucky Enquirer consists of an additional section wrapped around
the Cincinnati Enquirer and a remade Local section. The front page is
remade from the Ohio edition, although it may contain similar elements.

Reader-submitted content is featured in six zoned editions of Your
HomeTown Enquirer, a local news insert published twice-weekly on
Thursdays and Saturdays in Hamilton, Butler, Warren, and Clermont
counties.

Since September 2015, the Enquirer and local Fox affiliate WXIX-TV have
partnered on news gathering and have shared news coverage and video
among the paper, broadcasts, and online media. In 2016, the Enquirer
launched a true crime podcast called Accused that reached the top of
iTunes' podcasts chart.

Under then-editor Peter Bhatia, the Enquirer became the first newsroom
in the nation to dedicate a reporter to covering the heroin epidemic
full time. That reporter, Terry DeMio, and reporter Dan Horn helped lead
a staff of about 60 journalists to report the heroin project that won
the newspaper its second Pulitzer Prize. The award was the first the
newsroom won for its reporting, but its second win overall. The first
Pulitzer win was awarded to Jim Borgman for editorial cartoons in 1991.

\section{History}\label{history}

\includegraphics[width=4.12133in,height=5.50000in]{media/image1.jpg}\\
\emph{The first issue of the Daily Cincinnati Enquirer.}

\section{Early years}\label{early-years}

\begin{itemize}
\item
  \emph{In January 1845, the paper dropped the Message name, becoming
  The Cincinnati Daily Enquirer.}
\item
  \emph{In November 1843, the Enquirer merged with the Daily Morning
  Message to become the Enquirer and Message (the Daily Enquirer and
  Message beginning in May 1844).}
\item
  \emph{Finally, in May 1849, the paper became The Cincinnati Enquirer.}
\end{itemize}

The Enquirer's predecessor was the Phoenix, edited by Moses Dawson as
early as 1828. It later became the Commercial Advertiser and in 1838 the
Cincinnati Advertiser and Journal. By the time John and Charles Brough
purchased it and renamed it the Daily Cincinnati Enquirer, it was
considered a newspaper of record for the city. The Enquirer's first
issue, on April 10, 1841, consisted of "just four pages of
squint-inducing text that was, at times, as ugly in tone as it was in
appearance". It declared its staunch support for the Democratic Party,
in contrast to the three Whig papers and two ostensibly independent
papers then in circulation. A weekly digest edition for regional
farmers, the Weekly Cincinnati Enquirer, began publishing on April 14
and would continue until November 25, 1843, as The Cincinnati Weekly
Enquirer.

In November 1843, the Enquirer merged with the Daily Morning Message to
become the Enquirer and Message (the Daily Enquirer and Message
beginning in May 1844). In January 1845, the paper dropped the Message
name, becoming The Cincinnati Daily Enquirer. Finally, in May 1849, the
paper became The Cincinnati Enquirer. On April 20, 1848, the Enquirer
became one of the first newspapers in the United States to publish a
Sunday edition.

\section{McLean ownership and Washington
trust}\label{mclean-ownership-and-washington-trust}

\begin{itemize}
\item
  \emph{A competitor, the Cincinnati Daily Times, allowed the Enquirer
  to print on its presses in the wake of the disaster.}
\item
  \emph{By the late 1940s, sales of the Enquirer, Cincinnati's last
  remaining morning daily, had increased dramatically, fueled in part by
  the success of its Sunday morning monopoly; meanwhile, The Cincinnati
  Post and especially The Cincinnati Times-Star faced a declining
  afternoon market.}
\end{itemize}

In 1844, James J. Faran took an interest in the Enquirer. In 1848,
Washington McLean and his brother S.~B. Wiley McLean acquired an
interest in the Enquirer.

On March 22, 1866, a gas leak caused Pike's Opera House to explode,
taking with it the Enquirer offices next door. A competitor, the
Cincinnati Daily Times, allowed the Enquirer to print on its presses in
the wake of the disaster. As a result, the Enquirer missed only one day
of publication. However, archives of the paper's first 25 years were
lost.

Washington McLean was a leading Copperhead whose editorial policies led
to the suppression of the paper by the United States government during
the Civil War. After the war, McLean pursued an anti-Republican stance.
One of his star writers was Lafcadio Hearn, who wrote for the paper from
1872 to 1875. James W. Faulkner served as the paper's political
correspondent, covering the Ohio State Legislature and Statehouse, from
1887 until his death in 1923. The Faulkner Letter was a well-known
column often carried in regional newspapers.

In the 1860s, Washington McLean bought out Faran's interest in the
Enquirer. In 1872, he sold a half interest in the newspaper to his son,
John Roll McLean, who assumed full ownership of the paper in 1881. He
owned the paper until his death in 1916. Having little faith in his only
child, Ned, John Roll McLean put the Enquirer and another paper he
owned, The Washington Post, in trust with the American Security and
Trust Company of Washington, D.C., as trustee. Ned successfully broke
the trust regarding The Post, an action that led to its bankruptcy and
eventual sale to Eugene Meyer in 1933. The Enquirer, however, continued
to be held in trust until 1952.

In the 1910s, the Enquirer was known for an attention-getting style of
headline in which individual words or phrases cascaded vertically,
beginning with a single word in large type. According to a 1912 college
textbook on newspaper making, "The Enquirer has printed some
masterpieces replete with a majesty of diction that is most artistic;
but there are few papers that can imitate it successfully." During the
1930s and 1940s, the Enquirer was widely regarded among newspapers for
its innovative and distinctive typography.

In the 1920s, the Enquirer ran a promotion that offered a free plot of
land near Loveland, Ohio, along the Little Miami River, after paying for
a one-year subscription to the daily. The Loveland Castle was built on
two such plots. The surrounding community is now known as Loveland Park.

By the late 1940s, sales of the Enquirer, Cincinnati's last remaining
morning daily, had increased dramatically, fueled in part by the success
of its Sunday morning monopoly; meanwhile, The Cincinnati Post and
especially The Cincinnati Times-Star faced a declining afternoon market.

\section{Employee ownership}\label{employee-ownership}

\begin{itemize}
\item
  \emph{In February 1952, The Cincinnati Times-Star offered to buy the
  Enquirer from the American Security and Trust Company for
  \$7.5~million.}
\item
  \emph{In its first year under employee ownership, the Enquirer
  reported a net earnings of \$349,421.}
\end{itemize}

In February 1952, The Cincinnati Times-Star offered to buy the Enquirer
from the American Security and Trust Company for \$7.5~million. In
response, the 845 employees of the paper pooled their assets, formed a
committee, and obtained loans to successfully outbid the Times-Star with
an offer of \$7.6~million, with the Portsmouth Steel Company as their
agent. The deal closed on June 6, 1952. In its first year under employee
ownership, the Enquirer reported a net earnings of \$349,421.

\section{Scripps ownership}\label{scripps-ownership}

\begin{itemize}
\item
  \emph{Beset by financial problems and internal strife, they sold the
  paper to The E. W. Scripps Company, owner of The Cincinnati Post, on
  April 26, 1956.}
\item
  \emph{The E. W. Scripps Company operated the Enquirer at arm's length,
  even omitting the Scripps lighthouse logo from the Enquirer's
  nameplate.}
\item
  \emph{With the Times-Star and Enquirer acquisitions, the Scripps
  family owned all of Cincinnati's dailies, along with WCPO-AM, WCPO-FM,
  and WCPO-TV.}
\end{itemize}

The employees lacked sufficient capital and managerial expertise to run
the paper. City editor John F. Cronin led a revolt against management on
November 25, 1955; he was fired the following month. Beset by financial
problems and internal strife, they sold the paper to The E. W. Scripps
Company, owner of The Cincinnati Post, on April 26, 1956. Scripps
purchased a 36.5\% controlling interest in the Enquirer for \$4,059,000,
beating out The Times-Star Company's \$2,380,051 and Tribune
Publishing's \$15 per share, or \$2,238,000. Two years later, Scripps
also acquired the Times-Star, merging the afternoon paper with the Post.

With the Times-Star and Enquirer acquisitions, the Scripps family owned
all of Cincinnati's dailies, along with WCPO-AM, WCPO-FM, and WCPO-TV.
The E. W. Scripps Company operated the Enquirer at arm's length, even
omitting the Scripps lighthouse logo from the Enquirer's nameplate.
Nevertheless, the United States Department of Justice filed an antitrust
suit against the company in 1964.

\section{Gannett ownership and joint operating
agreement}\label{gannett-ownership-and-joint-operating-agreement}

\begin{itemize}
\item
  \emph{In April 2006, The Enquirer was cited by The Associated Press
  with the news cooperative's General Excellence Award, naming The
  Enquirer as the best major daily newspaper in Ohio.}
\item
  \emph{Following the Post's closure, the Enquirer made efforts to
  appeal to The Kentucky Post's former readership, for example referring
  to the Cincinnati metropolitan area as "Greater Cincinnati and
  Northern Kentucky" rather than simply "Greater Cincinnati".}
\item
  \emph{On September 22, 1977, the Enquirer signed a joint operating
  agreement (JOA) with The Cincinnati Post.}
\end{itemize}

In 1968, Scripps entered into a consent decree to sell the Enquirer. It
was sold to influential Cincinnati millionaire Carl Lindner Jr.'s
American Financial Corporation on February 20, 1971. In turn, Lindner
sold the Enquirer to a Phoenix-based company of his, Combined
Communications, in 1969, for \$30~million plus 500,000 shares of common
stock and 750,000 shares of common stock warrants in Combined
Communications. Combined Communications merged with Gannett Company in
1979.

On September 22, 1977, the Enquirer signed a joint operating agreement
(JOA) with The Cincinnati Post. For two years, the Enquirer had secretly
negotiated the terms of the JOA with the Post while securing concessions
from labor unions. The two papers petitioned the Justice Department for
an antitrust exemption under the Newspaper Preservation Act of 1970.
This was the second JOA application under the Newspaper Preservation
Act; the first, involving the Anchorage Daily News and Anchorage Times,
was summarily approved but already seen as a failure.

The Enquirer--Post agreement was approved on November 26, 1979, taking
effect after negotiations and legal battles with unions. As the more
financially sound paper, the Enquirer received an 80\% stake in the
business and handled all business functions of both papers, including
printing, distribution, and selling advertising. Gannett opened a new
printing press off Western Avenue in the West End to print both papers.

In August 1980, William J. Keating appointed George Blake to serve as
the Enquirer's first new editor since the Gannett acquisition. Blake,
who was previously editor at The News-Press of Fort Myers, Florida, had
a tendency to delegate that contrasted with the hands-on style of his
predecessor, Luke Feck. The Enquirer underwent a staff reorganization
and introduced a new format in September 1982.

Under Blake, the Enquirer had a reputation for friendliness to corporate
interests, exemplified in its weak coverage of the savings and loan
crisis that engulfed financier Charles Keating, brother of Enquirer
publisher William J. Keating. The paper's approach changed dramatically
in January 1993 with the arrival of president and publisher Harry
Whipple and editor Lawrence Beaupre from Gannett Suburban Newspapers in
White Plains, New York. Beaupre emphasized investigative reporting,
beginning with aggressive coverage of Charles Keating's conviction. By
1995, he had brought his team of aggressive investigative reporters from
White Plains to the Enquirer. The paper won awards for Michael
Gallagher's 1996 investigation into Fluor Daniel's cleanup of the
uranium processing plant at Fernald Feed Materials Production Center.

On May 3, 1998, the Enquirer published a special 18-page section, titled
"Chiquita Secrets Revealed", that accused the Cincinnati-based fruit
company of labor abuses, polluting, bribery, and other misdeeds.
Chiquita, owned by former Enquirer owner Lindner, denied all of the
allegations. Gallagher was charged and convicted for illegally obtaining
some of the evidence through voicemail hacking, and the Enquirer fired
him for lying about his sources. Faced with a potential lawsuit over the
voicemail hacking, the Enquirer settled with Chiquita out of court,
paying the company \$14~million. Under the terms of the agreement, the
paper published an unprecedented three-day-long, front-page retraction
of the entire series, destroyed any evidence they had gathered against
Chiquita, and transferred Beaupre to Gannett headquarters. The paper
largely reverted to its former approach to business coverage.

On April 10, 2000, the Enquirer and Post downsized from a traditional
12~5⁄16-inch-wide (313~mm) broadsheet format to an 11~5⁄8-inch-wide
(300~mm) format similar to Berliner. They also began publishing in color
every day of the week. Gannett promoted the narrower format as being
"easier to handle, hold, and read" but also cited reduced newsprint
costs.

In May 2003, Gannett replaced Harry Whipple with Cincinnati native
Margaret E. Buchanan as president and publisher. Buchanan, previously
publisher of the Idaho Statesman, was the newspaper's first woman
publisher. The same year, Tom Callinan became editor of the Enquirer
after stints as editor of The Arizona Republic, the Democrat and
Chronicle of Rochester, New York, and the Lansing State Journal. One of
his first moves was to reassign media critics to reporting positions.

Callinan originally attempted to address declining circulation by
focusing on lifestyle content aimed at younger readers; however, this
approach alienated the paper's older core audience. The paper responded
by reemphasizing national news in the newspaper and creating niche,
crowsourced products online for younger audiences. In October 2003, The
Enquirer began publishing and distributing CiN Weekly, a free lifestyle
magazine aimed at younger readers, to compete against Cincinnati
CityBeat. In 2004, Gannett purchased local magazines Design and Inspire
and increased coverage in The Kentucky Enquirer. In November 2004,
Gannett purchased HomeTown Communications Network, publisher of a daily
newspaper and 62 weekly and biweekly newspapers branded The Community
Press in Ohio and The Community Recorder in Kentucky. The Department of
Justice cleared the purchase the following March.

In January 2004, the Enquirer informed the Post of its intention to let
the JOA expire. The Post published its final print edition upon the
JOA's expiration on December 31, 2007, leaving the Enquirer as the only
daily newspaper in Greater Cincinnati and Northern Kentucky. Following
the Post's closure, the Enquirer made efforts to appeal to The Kentucky
Post's former readership, for example referring to the Cincinnati
metropolitan area as "Greater Cincinnati and Northern Kentucky" rather
than simply "Greater Cincinnati".

In April 2006, The Enquirer was cited by The Associated Press with the
news cooperative's General Excellence Award, naming The Enquirer as the
best major daily newspaper in Ohio. Earlier that year, parent Gannett
Co. named The Enquirer the most improved of the more than 100 newspapers
in the chain.{[}citation needed{]}

In December 2010, Callinan left for a professorship at the University of
Cincinnati and was succeeded by Carolyn Washburn as editor.

In October 2012, the online version of the Enquirer went behind a
metered paywall.

In March 2013, Gannett closed its West End printing facility and
contracted with The Columbus Dispatch to print the Enquirer in Columbus.
Shortly after, the Enquirer began publishing in a smaller compact
format. Former Post and Enquirer pressman Al Bamberger purchased the
former Enquirer facility that June and sold it to Wegman Company, an
office furniture installation company.

Buchanan retired in March 2015. Gannett named Rick Green, the editor of
The Des Moines Register and a former Enquirer assistant editor, as
president and publisher. In August 2016, Gannett eliminated the
Enquirer's Publisher position, transferring Green to the North Jersey
Media Group in New Jersey.

\section{Facilities}\label{facilities}

\begin{itemize}
\item
  \emph{From 1916 to 1928, the newspaper constructed a new headquarters
  and printing plant, the Cincinnati Enquirer Building, on this
  property.}
\item
  \emph{Since March 2013, Gannett has contracted with The Columbus
  Dispatch in Columbus to print all its Cincinnati publications,
  including the Enquirer.}
\item
  \emph{The Enquirer has published from many downtown Cincinnati
  locations.}
\end{itemize}

The Enquirer has published from many downtown Cincinnati locations. From
Fifth Street between Main and Sycamore, it moved to Third Street, then
to the corner of Third and Main, then to Main between Third and Pearl.
In 1866, the Enquirer began publishing from offices in the 600 block of
Vine Street, near Baker Street. From 1916 to 1928, the newspaper
constructed a new headquarters and printing plant, the Cincinnati
Enquirer Building, on this property. In 1992, the newspaper moved to its
present Elm Street headquarters.

The Enquirer operated two news bureaus until July 2013. The Northern
Kentucky bureau produced The Kentucky Enquirer and The Community
Recorder, while the West Chester bureau covered Butler and Warren
counties for The Cincinnati Enquirer's northern zones and produced some
editions of The Community Press.

From 1977 to 2013, the Enquirer was printed from a 130,000-square-foot
(12,000~m2) press off Western Avenue in the West End. Until 2007, this
facility also printed The Cincinnati Post under a joint operating
agreement. Since March 2013, Gannett has contracted with The Columbus
Dispatch in Columbus to print all its Cincinnati publications, including
the Enquirer. Similarly, Gannett has contracted with the Lafayette,
Indiana, Journal \& Courier to print Community Press and Community
Recorder editions since 2007.

\includegraphics[width=5.50000in,height=1.82573in]{media/image2.png}\\
\emph{Former Cincinnati.com logo}

\section{Online presence}\label{online-presence}

\begin{itemize}
\item
  \emph{The CiN Weekly, Community Press, and Community Recorder weekly
  newspapers have also been online partners with the Enquirer.}
\item
  \emph{Archives of Enquirer articles can be found in online
  subscription databases.}
\item
  \emph{Due to a joint operating agreement with The Cincinnati Post, it
  launched concurrently with the Post's site, @The Post.}
\item
  \emph{The Enquirer launched its first website, Enquirer.com, on
  November 1, 1996.}
\end{itemize}

The Enquirer launched its first website, Enquirer.com, on November 1,
1996. Due to a joint operating agreement with The Cincinnati Post, it
launched concurrently with the Post's site, @The Post. A shared website,
GoCincinnati!, located at gocinci.net, displayed classified advertising
and offered dial-up Internet access subscriptions. Local access numbers
were available in cities throughout the country through a network of
Gannett publications. Both papers' home pages moved to a more memorable
domain, Cincinnati.com, on November 1, 1998. The new brand encompassed
about 300 local commercial sites and some community organizations.

From May 2002 to March 2007, Cincinnati.com also included WCPO.com, the
website of Post sister company WCPO-TV. The Post closed at the end of
2007, ending Scripps' involvement in Cincinnati.com. The CiN Weekly,
Community Press, and Community Recorder weekly newspapers have also been
online partners with the Enquirer.

In October 2005, the Enquirer launched NKY.com, a website covering news
from Boone, Campbell, and Kenton counties in Northern Kentucky. NKY.com
was one of the first newspaper-published websites to make extensive use
of user-created content, which it featured prominently on 38 community
pages. In August 2006, Cincinnati.com launched 186 community pages
covering towns and neighborhoods in Ohio and Indiana and began
soliciting and publishing stories and articles from readers, which
appear in Your Hometown Enquirer inserts.

Since October 2012, Cincinnati.com has operated behind a metered paywall
that allows readers to view 10 stories a month before paying a
subscription fee. As a Gannett property, Cincinnati.com is branded as
"part of the USA Today Network". Its primary competitor in the market is
WCPO-TV's website, WCPO.com.

Archives of Enquirer articles can be found in online subscription
databases. ProQuest contains full text of articles from 1841 to 1922 and
from 1999 to present, as well as "digital microfilm" of articles from
2010 to 2012. As of September~2016{[}update{]}, Newspapers.com has scans
of 4.2 million pages from 1841 to present.

\section{Notable people}\label{notable-people}

\begin{itemize}
\item
  \emph{Former Enquirer owners and publishers:}
\item
  \emph{Rudolph K. Hynicka~-- Cincinnati politician affiliated with Boss
  Cox}
\item
  \emph{Carolyn Washburn~-- Enquirer editor}
\end{itemize}

Current employees:

Amber Hunt~-- crime author

Former employees and contributors:

Lee Allen~-- baseball historian

Peter Bhatia ~-- newspaper editor

Roy Beck~-- anti--illegal immigration activist

Jim Borgman~-- Pulitzer Prize--winning editorial cartoonist

O. P. Caylor~-- baseball columnist

George Randolph Chester~-- writer

James M. Cox~-- Governor of Ohio, U.S. Representative, and U.S.
presidential candidate

Harry M. Daugherty~-- U.S. Attorney General

Timothy C. Day~-- U.S. Representative

Jerry Dowling~-- cartoonist

James W. Faulkner~-- political journalist

Suzanne Fournier~-- Chief of Public Affairs for the U.S. Army Corps of
Engineers

Michael Gallagher~-- investigative journalist

Sloane Gordon~-- political writer

Murat Halstead~-- newspaper editor

Lafcadio Hearn~-- writer

Rudolph K. Hynicka~-- Cincinnati politician affiliated with Boss Cox

Peter King~-- sportswriter

Winsor McCay~-- cartoonist and animator

Robert D. McFadden~-- journalist

John McIntyre~-- copyeditor

Charles Murphy~-- owner of the Chicago Cubs

Terence Moore~-- sports journalist

David Philipson~-- Reform rabbi and orator

Jacob J. Rosenthal~-- theater manager

Frederick Bushnell "Jack" Ryder~-- football coach and sportswriter

Al Schottelkotte~-- WCPO-TV news anchor

Bill Thomas~-- author

Whitney Tower~-- horse racing reporter

Lawson Wulsin~-- professor of psychiatry and family medicine

Former Enquirer owners and publishers:

Francis L. Dale~-- publisher

James J. Faran~-- owner and associate editor; U.S. Representative

William J. Keating~-- CEO and publisher; U.S. Representative

Carl Lindner Jr.~-- owner

John Roll McLean~-- publisher

Washington McLean~-- owner

Carolyn Washburn~-- Enquirer editor

\section{References}\label{references}

\section{Further reading}\label{further-reading}

\begin{itemize}
\item
  \emph{Cincinnati CityBeat.}
\item
  \emph{Cincinnati: The Cincinnati Enquirer, 1991.}
\end{itemize}

Nicholas Bender. "Banana Report." Columbia Journalism Review. May/June
2001.

Graydon Decamp. The Grand Old Lady of Vine Street. Cincinnati: The
Cincinnati Enquirer, 1991. (Official history).

Douglas Frantz. "After Apology, Issues Raised In Chiquita Articles
Remain." The New York Times. July 17, 1998. p. A1, A14

Douglas Frantz. "Mysteries Behind Story's Publication." The New York
Times. July 17, 1998. p. A14.

Lew Moores. "Media, Myself \& I". Cincinnati CityBeat. January 7, 2004.

Lew Moores. "The Day the Music Critic Died." Cincinnati CityBeat.
February 11, 2004.

Randolph Reddick. The Old Lady of Vine Street. Ohio University Ph. D.
dissertation, 1991. (A study of the four years of employee ownership).

Nicholas Stein. "Banana Peel." Columbia Journalism Review.
September/October 1998.

Taft, Robert, Jr. (October 1960). "Epilogue For a Lady: The Passing of
the Times-Star" (PDF). Bulletin of the Historical and Philosophical
Society of Ohio. 18 (4): 260--277. OCLC~52305709.

\section{External links}\label{external-links}

\begin{itemize}
\item
  \emph{Gannett Co. Inc. profile of The Cincinnati Enquirer}
\item
  \emph{Public Library of Cincinnati and Hamilton County, Newsdex (an
  index to historical newspapers in the Cincinnati area),
  http://newsdex.cincinnatilibrary.org/uhtbin/cgisirsi/x/0/0/49.}
\end{itemize}

Cincinnati.Com (official site)

Cincinnati.Com (official mobile site)

(official iPhone site)

NKY.com (official site)

Enquirer.com (official site)

Cinweekly.com (official site)

Gannett Co. Inc. official site

Gannett Co. Inc. profile of The Cincinnati Enquirer

Public Library of Cincinnati and Hamilton County, Newsdex (an index to
historical newspapers in the Cincinnati area),
http://newsdex.cincinnatilibrary.org/uhtbin/cgisirsi/x/0/0/49.
