\textbf{From Wikipedia, the free encyclopedia}

https://en.wikipedia.org/wiki/Michael\%20Chertoff\\
Licensed under CC BY-SA 3.0:\\
https://en.wikipedia.org/wiki/Wikipedia:Text\_of\_Creative\_Commons\_Attribution-ShareAlike\_3.0\_Unported\_License

\section{Michael Chertoff}\label{michael-chertoff}

\begin{itemize}
\item
  \emph{Chertoff co-chairs the Bipartisan Policy Center's Immigration
  Task Force.}
\item
  \emph{Michael Chertoff (born November 28, 1953) is an American
  attorney who was the second United States Secretary of Homeland
  Security, serving under President George W. Bush.}
\item
  \emph{He also co-founded the Chertoff Group, a risk-management and
  security consulting company, which employs several former senior
  political appointees.}
\end{itemize}

Michael Chertoff (born November 28, 1953) is an American attorney who
was the second United States Secretary of Homeland Security, serving
under President George W. Bush. He was the co-author of the USA PATRIOT
Act. He previously served as a United States Circuit Judge of the United
States Court of Appeals for the Third Circuit, as a federal prosecutor,
and as Assistant U.S. Attorney General. He succeeded Tom Ridge as U.S.
Secretary of Homeland Security on February 15, 2005.

Since leaving government service, Chertoff has worked as senior of
counsel at the Washington, D.C. law firm of Covington \& Burling. He
also co-founded the Chertoff Group, a risk-management and security
consulting company, which employs several former senior political
appointees. Chertoff was also elected as Chairman of BAE Systems for a
three-year term, beginning May 1, 2012.

Chertoff co-chairs the Bipartisan Policy Center's Immigration Task
Force.

\section{Early life}\label{early-life}

\begin{itemize}
\item
  \emph{Michael Chertoff was born on November 28, 1953 in Elizabeth, New
  Jersey.}
\item
  \emph{His paternal grandparents are Rabbi Paul Chertoff and Esther
  Barish Chertoff.}
\item
  \emph{Chertoff worked on Mafia and political corruption--related
  cases.}
\item
  \emph{In the mid-1990s, Chertoff returned to Latham \& Watkins for a
  brief period, founding the firm's office in Newark, New Jersey.}
\item
  \emph{Chertoff has been a resident of Westfield, New Jersey.}
\end{itemize}

Michael Chertoff was born on November 28, 1953 in Elizabeth, New Jersey.
His father was Rabbi Gershon Baruch Chertoff (1915--96), a Talmud
scholar and the former leader of the Congregation B'nai Israel in
Elizabeth. His mother is Livia Chertoff (née Eisen), an Israeli citizen
and the first flight attendant for El Al. His paternal grandparents are
Rabbi Paul Chertoff and Esther Barish Chertoff.

Chertoff attended the Jewish Educational Center in Elizabeth as well as
the Pingry School. He graduated from Harvard College with a Bachelor of
Arts degree in 1975. During his sophomore year, he studied abroad at the
London School of Economics and Political Science. He then attended
Harvard Law School, where he worked as a research assistant for John
Hart Ely on his book Democracy and Distrust. After receiving a Juris
Doctor magna cum laude in 1978, Chertoff served as a law clerk to Judge
Murray Gurfein of the United States Court of Appeals for the Second
Circuit. He then clerked for United States Supreme Court Justice William
J. Brennan, Jr. from 1979 to 1980.

He worked in private practice with Latham \& Watkins from 1980 to 1983
before being hired as a prosecutor by Rudolph Giuliani, then the United
States Attorney for the Southern District of New York. Chertoff worked
on Mafia and political corruption--related cases. In the mid-1990s,
Chertoff returned to Latham \& Watkins for a brief period, founding the
firm's office in Newark, New Jersey.

Chertoff has been a resident of Westfield, New Jersey.

\section{Public service}\label{public-service}

\begin{itemize}
\item
  \emph{In 1990, Chertoff was appointed by President George H. W. Bush
  as United States Attorney for the District of New Jersey.}
\item
  \emph{On March 5, 2003, Chertoff was nominated by President Bush to a
  seat on the United States Court of Appeals for the Third Circuit
  vacated by Morton I. Greenberg.}
\item
  \emph{In 2000, Chertoff worked as special counsel to the New Jersey
  Senate Judiciary Committee, investigating racial profiling in New
  Jersey.}
\end{itemize}

In September 1986, together with United States Attorney for the Southern
District of New York Rudolph Giuliani, Chertoff was instrumental in the
crackdown on organized crime in the Mafia Commission Trial.

In 1990, Chertoff was appointed by President George H. W. Bush as United
States Attorney for the District of New Jersey. Among his most important
cases, in 1992 Chertoff achieved conviction of second-term Jersey City
mayor Gerald McCann on charges of defrauding money from a savings and
loan scam. McCann served two years in federal prison.

In 1993, he was a prosecutor in the fraud case against Eddie Antar,
founder of the Crazy Eddie's electronics store chain.

Chertoff was asked to stay in his position when the Clinton
administration took office in 1993, at the request of Democratic Senator
Bill Bradley. He was the only United States Attorney who was not
replaced due to the change in administrations. He continued to work with
the U.S. Attorney's office until 1994, when he entered private practice,
returning to Latham \& Watkins as a partner.

Despite his friendly relationship with some Democrats, Chertoff was
appointed as the special counsel for the Senate Whitewater Committee
studying allegations against President Clinton and his wife in what was
known as the Whitewater investigation. No charges were brought against
the Clintons.

In 2000, Chertoff worked as special counsel to the New Jersey Senate
Judiciary Committee, investigating racial profiling in New Jersey. He
also did some fundraising for George W. Bush and other
Republicans{[}citation needed{]} during the 2000 election cycle. He
advised Bush's presidential campaign on criminal justice issues.
Chertoff was appointed by Bush to head the criminal division of the
Department of Justice, serving from 2001 to 2003. He led the federal
prosecution's case against suspected terrorist Zacarias Moussaoui.

Chertoff also led the prosecution's case against accounting firm Arthur
Andersen for destroying documents relating to the Enron collapse. The
prosecution of Arthur Andersen was controversial, as the firm was
effectively dissolved, resulting in the loss of 26,000 jobs. The United
States Supreme Court overturned the conviction, and the case has not
been retried. Chertoff has been criticized for his role at DOJ in
detaining hundreds of Middle Eastern immigrants.

On March 5, 2003, Chertoff was nominated by President Bush to a seat on
the United States Court of Appeals for the Third Circuit vacated by
Morton I. Greenberg. He was confirmed by the Senate 88--1 on June 9,
2003, with Senator Hillary Clinton of New York casting the lone
dissenting vote; he received his commission the following day. Senator
Clinton said that she had dissented to register her protest for the way
Chertoff's staff mistreated junior White House staffers during the
Whitewater investigation.

\includegraphics[width=5.50000in,height=3.84144in]{media/image1.jpg}\\
\emph{President Bush discussing border security with Chertoff near El
Paso, Texas, November 2005}

\section{Secretary of Homeland Security and subsequent
career}\label{secretary-of-homeland-security-and-subsequent-career}

\begin{itemize}
\item
  \emph{The firm is also led by Chad Sweet; he served as the Chief of
  Staff of Homeland Security while Chertoff was Secretary and also had a
  two-year stint at the Directorate of Operations for the CIA.}
\item
  \emph{Hurricane Katrina occurred while Chertoff was Secretary of
  Homeland Security.}
\end{itemize}

In late 2004, Bernard Kerik was forced to decline President Bush's offer
to replace Tom Ridge, the outgoing Secretary of Homeland Security. After
a lengthy search to find a suitable replacement, Bush nominated Chertoff
to the post in January 2005, citing his experience with post-9/11 terror
legislation. He was unanimously approved for the position by the United
States Senate on February 15, 2005.

Hurricane Katrina occurred while Chertoff was Secretary of Homeland
Security. The Department was criticized for its lack of preparation in
advance of the well-forecast hurricane; most criticism was directed
toward the Federal Emergency Management Agency. DHS in general, and
Chertoff in particular, were criticized for responding poorly to the
disaster, ignoring crucial information about the catastrophic nature of
the storm and devoting little attention to the federal response to what
became the most costly disaster in American history.

Chertoff was the Bush administration's point man for pushing the
comprehensive immigration reform bill, a measure that stalled in the
Senate in June 2007.

Chertoff was asked by the Obama administration to stay in his post until
9 a.m. on January 21, 2009, (one day after President Obama's
inauguration) "to ensure a smooth transition".

He formed The Chertoff Group (TCG) on February 2, 2009 to work on crisis
and risk management. The firm is also led by Chad Sweet; he served as
the Chief of Staff of Homeland Security while Chertoff was Secretary and
also had a two-year stint at the Directorate of Operations for the CIA.
They also employ Charles E. Allen, Larry Castro, Jay M. Cohen, General
Michael V. Hayden and other former high-ranking government employees and
appointees.

\section{Views}\label{views}

\section{Construction of border
fence}\label{construction-of-border-fence}

\begin{itemize}
\item
  \emph{According to The New York Times columnist Adam Liptak, Chertoff
  had excluded the Department of Homeland Security from having to follow
  laws "protecting the environment, endangered species, migratory birds,
  the bald eagle, antiquities, farms, deserts, forests, Native American
  graves and religious freedom."}
\item
  \emph{In April 2008, Chertoff was criticized in The New York Times
  editorial for waiving the Endangered Species Act, the Clean Water Act,
  and other environmental protection legislation to construct a 700-mile
  (1,100~km) fence along the Mexico--United States border.}
\end{itemize}

In April 2008, Chertoff was criticized in The New York Times editorial
for waiving the Endangered Species Act, the Clean Water Act, and other
environmental protection legislation to construct a 700-mile (1,100~km)
fence along the Mexico--United States border. The Times wrote: "To the
long list of things the Bush administration is willing to trash in its
rush to appease immigration hard-liners, you can now add dozens of
important environmental laws and hundreds of thousands of acres of
fragile habitat on the southern border."

According to The New York Times columnist Adam Liptak, Chertoff had
excluded the Department of Homeland Security from having to follow laws
"protecting the environment, endangered species, migratory birds, the
bald eagle, antiquities, farms, deserts, forests, Native American graves
and religious freedom."

A report issued by the Congressional Research Service, the non-partisan
research division of the Library of Congress, said that the unchecked
delegation of powers to Chertoff was unprecedented:

\section{Actions regarding illegal
immigration}\label{actions-regarding-illegal-immigration}

\begin{itemize}
\item
  \emph{In September 2007, Chertoff told a House committee that the DHS
  would not tolerate interference by sanctuary cities that would block
  the "Basic Pilot Program," which requires some types of employers to
  validate the legal status of their workers.}
\item
  \emph{In 2008 it was reported that the residential housekeeping
  company Chertoff had hired to clean his house employed illegal
  immigrants.}
\end{itemize}

In September 2007, Chertoff told a House committee that the DHS would
not tolerate interference by sanctuary cities that would block the
"Basic Pilot Program," which requires some types of employers to
validate the legal status of their workers. He said that the DHS was
exploring its legal options and intended to take action to prevent any
interference with the law.

In 2008 it was reported that the residential housekeeping company
Chertoff had hired to clean his house employed illegal immigrants.

\section{Globalization}\label{globalization}

\begin{itemize}
\item
  \emph{At the Global Creative Leadership Summit in 2009, Chertoff
  described globalization as a double-edged sword.}
\item
  \emph{Although globalization may help raise the standard of living for
  people around the world, Chertoff claims that it can also enable
  terrorists and transnational criminals.}
\end{itemize}

At the Global Creative Leadership Summit in 2009, Chertoff described
globalization as a double-edged sword. Although globalization may help
raise the standard of living for people around the world, Chertoff
claims that it can also enable terrorists and transnational criminals.

\section{Body scanners}\label{body-scanners}

\begin{itemize}
\item
  \emph{His lobbying firm Chertoff Group (founded 2009) represents
  manufacturers of the scanners.}
\item
  \emph{Chertoff has been an advocate of enhanced technologies, such as
  full body scanners.}
\end{itemize}

Chertoff has been an advocate of enhanced technologies, such as full
body scanners. His lobbying firm Chertoff Group (founded 2009)
represents manufacturers of the scanners.

\section{Climate change}\label{climate-change}

\begin{itemize}
\item
  \emph{Chertoff co-signed the preface to the report "National Security
  and the Accelerating Risks of Climate Change" published in 2014 where
  he stated that "projected climate change is a complex multi-decade
  challenge.}
\item
  \emph{Without action to build resilience, it will increase security
  risks over much of the planet.}
\end{itemize}

Chertoff co-signed the preface to the report "National Security and the
Accelerating Risks of Climate Change" published in 2014 where he stated
that "projected climate change is a complex multi-decade challenge.
Without action to build resilience, it will increase security risks over
much of the planet. It will not only increase threats to developing
nations in resource-challenged parts of the world, but it will also test
the security of nations with robust capability, including significant
elements of our National Power here at home."

\section{Political endorsements}\label{political-endorsements}

\begin{itemize}
\item
  \emph{For the 2016 presidential election, Chertoff endorsed Hillary
  Clinton.}
\end{itemize}

For the 2016 presidential election, Chertoff endorsed Hillary Clinton.

\section{References}\label{references}

\section{External links}\label{external-links}

\begin{itemize}
\item
  \emph{Michael Chertoff at the Biographical Directory of Federal
  Judges, a public domain publication of the Federal Judicial Center.}
\end{itemize}

Michael Chertoff at the Biographical Directory of Federal Judges, a
public domain publication of the Federal Judicial Center.

Appearances on C-SPAN
