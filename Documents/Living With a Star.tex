\textbf{From Wikipedia, the free encyclopedia}

https://en.wikipedia.org/wiki/Living\%20With\%20a\%20Star\\
Licensed under CC BY-SA 3.0:\\
https://en.wikipedia.org/wiki/Wikipedia:Text\_of\_Creative\_Commons\_Attribution-ShareAlike\_3.0\_Unported\_License

\includegraphics[width=5.50000in,height=5.50000in]{media/image1.png}\\
\emph{Living With a Star program logo}

\section{Living With a Star}\label{living-with-a-star}

\begin{itemize}
\item
  \emph{LWS is composed of three major components: scientific
  investigations on spaceflight platforms study different regions of the
  Sun, interplanetary space, and geospace; an applied science Space
  Environment Testbeds program where protocols and components are
  tested; and a Targeted Research and Technology Program.}
\item
  \emph{The program is managed by the Heliophysics Science Division of
  NASA's Science Mission Directorate.}
\item
  \emph{Major spacecraft include the Van Allen Probes, the Solar
  Dynamics Observatory, and the Parker Solar Probe}
\end{itemize}

Living With a Star (LWS) is a NASA scientific program to study those
aspects of the connected Sun-Earth system that directly affect life and
society. LWS is a crosscutting initiative with goals and objectives
relevant to NASA's Exploration Initiative, as well as to NASA's
Strategic Enterprises. The program is managed by the Heliophysics
Science Division of NASA's Science Mission Directorate.

LWS is composed of three major components: scientific investigations on
spaceflight platforms study different regions of the Sun, interplanetary
space, and geospace; an applied science Space Environment Testbeds
program where protocols and components are tested; and a Targeted
Research and Technology Program. Major spacecraft include the Van Allen
Probes, the Solar Dynamics Observatory, and the Parker Solar Probe

LWS was started in 2001 and is still active in the late 2010s.

\includegraphics[width=5.50000in,height=3.09375in]{media/image2.png}\\
\emph{Solar prominence as recorded by SDO. a space Solar observatory}

\section{Overview and history}\label{overview-and-history}

\begin{itemize}
\item
  \emph{Living With a Star was proposed in 2000 and established with
  funding in the fall of 2001.}
\item
  \emph{Solar eclipses}
\item
  \emph{At the same time an international collaboration was also
  established, International Living With a Star program that is
  conducted with the Interagency Consultative Group task force.}
\item
  \emph{Space weather}
\item
  \emph{Explorers Program}
\item
  \emph{Living with a Star}
\item
  \emph{Another component of LWS, is the Targeted Research \& Technology
  program.}
\end{itemize}

Living With a Star was proposed in 2000 and established with funding in
the fall of 2001. At the same time an international collaboration was
also established, International Living With a Star program that is
conducted with the Interagency Consultative Group task force. The first
IWLS meeting was held in 2002.

Another component of LWS, is the Targeted Research \& Technology
program.

Areas of study for LWS include:

The Sun

Space weather

Explorers Program

Magnetosphere

Heliosphere

Living with a Star

Solar eclipses

\section{Objectives}\label{objectives}

\begin{itemize}
\item
  \emph{Aeronautics and Space Transportation}
\item
  \emph{Space Science}
\item
  \emph{Earth Science}
\end{itemize}

The program is focused on understanding the relationship between the Sun
and the Earth, a goal that is tackled across several disciplines and
areas of study,

Summary:

Space Science

Earth Science

Human Exploration and Development

Aeronautics and Space Transportation

\includegraphics[width=5.50000in,height=5.43284in]{media/image3.png}\\
\emph{Solar Dynamics Observatory spacecraft}

\includegraphics[width=5.50000in,height=3.97016in]{media/image4.png}\\
\emph{Van Allen Probes logo}

\section{Spaceflight segment}\label{spaceflight-segment}

\begin{itemize}
\item
  \emph{The first two science missions have launched: Solar Dynamics
  Observatory (SDO) and Van Allen Probes.}
\item
  \emph{Mission have included Balloon Array for Radiation-belt
  Relativistic Electron Losses (BARREL), Space Environment Testbeds
  (SET) , Space Environment Testbeds, the Solar Orbiter Collaboration
  (SOC), and Parker Solar Probe (PSP).}
\end{itemize}

The first two science missions have launched: Solar Dynamics Observatory
(SDO) and Van Allen Probes. The SDO mission launched on February 11,
2010, and the Van Allen Probes mission launched on August 30, 2012.
Mission have included Balloon Array for Radiation-belt Relativistic
Electron Losses (BARREL), Space Environment Testbeds (SET) , Space
Environment Testbeds, the Solar Orbiter Collaboration (SOC), and Parker
Solar Probe (PSP).

Science requirements and conceptual mission implementation have been
defined for the Ionosphere-Thermosphere Storm Probes (ITSP) and the
Solar Sentinels.

The Parker Solar Probe (Formerly Solar Probe+) was launched by a Delta
IV Heavy on August 12, 2018 from Florida, USA.

\section{Space Environment Testbeds}\label{space-environment-testbeds}

\begin{itemize}
\item
  \emph{The goal of this spacecraft is to understand the nature of space
  environment and how that environment impacts spacecraft.}
\end{itemize}

SET uses existing data and new data from low-cost SET missions to
achieve the following:\\
Define the mechanisms for induced space environment and effects;\\
reduce uncertainties in the definitions of the induced environment and
effects on spacecraft and their payloads; and to\\
improve design and operations guidelines and test protocols so that
spacecraft anomalies and failures due to environmental effects during
operations are reduced.

The goal of this spacecraft is to understand the nature of space
environment and how that environment impacts spacecraft.

\section{Targeted Research and
Technology}\label{targeted-research-and-technology}

\begin{itemize}
\item
  \emph{The focused science topic panels are a novel approach to
  collaborative science, and initial results appear promising.}
\item
  \emph{With the 2001 inception of the LWS Program, new opportunities
  were created for a systematic, goal-oriented research program
  targeting those aspects of the Sun-Earth system that affect life and
  society.}
\end{itemize}

With the 2001 inception of the LWS Program, new opportunities were
created for a systematic, goal-oriented research program targeting those
aspects of the Sun-Earth system that affect life and society. To provide
immediate progress toward achieving the LWS goals, the Targeted Research
and Technology (TR\&T) component of the program was developed. The TR\&T
element has solicited five rounds of proposals seeking quantitative
understanding and predictive capability throughout the system. TR\&T has
funded independent research awards, focused science topic panels, and
strategic capability challenges to enable a cross-disciplinary,
integrated, system-wide understanding of how the Sun varies, and how
Earth and planets respond. The focused science topic panels are a novel
approach to collaborative science, and initial results appear promising.

\includegraphics[width=5.50000in,height=3.09375in]{media/image5.jpg}\\
\emph{illustration of the twin Van Allen Probes deploying in Earth
Orbit}

\section{Missions}\label{missions}

\begin{itemize}
\item
  \emph{January 2018 -- Space Environment Testbeds (SET)}
\item
  \emph{The following missions are associated with the Living With a
  Star program, with launch dates indicated.}
\item
  \emph{July 2018 -- Parker Solar Probe - Previously known as Solar
  Probe Plus}
\item
  \emph{February 2010 -- Solar Dynamics Observatory (SDO)}
\item
  \emph{2020 -- Solar Orbiter}
\end{itemize}

The following missions are associated with the Living With a Star
program, with launch dates indicated.

February 2010 -- Solar Dynamics Observatory (SDO)

August 2012 -- Van Allen Probes

December 2012/December 2013 -- Balloon Array for Radiation-belt
Relativistic Electron Losses (BARREL)

January 2018 -- Space Environment Testbeds (SET)

July 2018 -- Parker Solar Probe - Previously known as Solar Probe Plus

2020 -- Solar Orbiter

\section{See also}\label{see-also}

\begin{itemize}
\item
  \emph{List of heliophysics missions}
\item
  \emph{List of NASA missions}
\end{itemize}

List of heliophysics missions

List of NASA missions

\section{References}\label{references}

\section{External links}\label{external-links}

\begin{itemize}
\item
  \emph{Living With a Star website at NASA.gov}
\end{itemize}

Living With a Star website at NASA.gov
