\textbf{From Wikipedia, the free encyclopedia}

https://en.wikipedia.org/wiki/Opposition\%20to\%20multiculturalism\\
Licensed under CC BY-SA 3.0:\\
https://en.wikipedia.org/wiki/Wikipedia:Text\_of\_Creative\_Commons\_Attribution-ShareAlike\_3.0\_Unported\_License

\section{Criticism of
multiculturalism}\label{criticism-of-multiculturalism}

\begin{itemize}
\item
  \emph{Critics of multiculturalism may argue against cultural
  integration of different ethnic and cultural groups to the existing
  laws and values of the country.}
\item
  \emph{Criticism of multiculturalism questions the ideal of the
  maintenance of distinct ethnic cultures within a country.}
\item
  \emph{Multiculturalism is a particular subject of debate in certain
  European nations that are associated with the idea of a single nation
  within their country.}
\end{itemize}

Criticism of multiculturalism questions the ideal of the maintenance of
distinct ethnic cultures within a country. Multiculturalism is a
particular subject of debate in certain European nations that are
associated with the idea of a single nation within their country.
Critics of multiculturalism may argue against cultural integration of
different ethnic and cultural groups to the existing laws and values of
the country. Alternatively critics may argue for assimilation of
different ethnic and cultural groups to a single national identity.

\section{Australia}\label{australia}

\begin{itemize}
\item
  \emph{In January 2007 the Howard Government removed the word
  "multicultural" from the name of the Department of Immigration and
  Multicultural Affairs, changing its name to the Department of
  Immigration and Citizenship.}
\item
  \emph{The response to multiculturalism in Australia has been varied.}
\end{itemize}

Rifts within Australian society, right through history, whether between
the continent's Indigenous people and the European settler population
or, in recent times, inter-ethnic tension manifest in the form of riots,
street violence and ethnic gangs pose major challenges to
multiculturalism in the country.

The response to multiculturalism in Australia has been varied. A
nationalist, anti-mass immigration party, the One Nation Party, was
formed by Pauline Hanson in the late 1990s. The party enjoyed brief
electoral success, most notably in its home state of Queensland, but
became electorally marginalized until its resurgence in 2016. In the
late 1990s, One Nation called for the abolition of multiculturalism
alleging that it represented "a threat to the very basis of the
Australian culture, identity and shared values", arguing that there was
"no reason why migrant cultures should be maintained at the expense of
our shared, national culture."

An Australian Federal Government proposal in 2006 to introduce a
compulsory citizenship test, which would assess English skills and
knowledge of Australian values, sparked renewed debate over the future
of multiculturalism in Australia. Andrew Robb, then Parliamentary
Secretary for Immigration and Multicultural Affairs, told a conference
in November 2006 that some Australians worried the term "multicultural"
had been transformed by interest groups into a philosophy that put
"allegiances to original culture ahead of national loyalty, a philosophy
which fosters separate development, a federation of ethnic cultures, not
one community". He added: "A community of separate cultures fosters a
rights mentality, rather than a responsibilities mentality. It is
divisive. It works against quick and effective integration." The
Australian citizenship test commenced in October 2007 for all new
citizens between the ages of 18 and 60.

In January 2007 the Howard Government removed the word "multicultural"
from the name of the Department of Immigration and Multicultural
Affairs, changing its name to the Department of Immigration and
Citizenship.

\section{Intellectual critique}\label{intellectual-critique}

\begin{itemize}
\item
  \emph{A manifestation of this embrace of multiculturalism has been the
  creation of ethnic branches within the Labor Party and ethnic branch
  stacking.}
\item
  \emph{Blainey remained a persistent critic of multiculturalism into
  the 1990s, denouncing multiculturalism as "morally, intellectually and
  economically ... a sham".}
\end{itemize}

The earliest academic critics of multiculturalism in Australia were the
philosophers Lachlan Chipman and Frank Knopfelmacher, sociologist Tanya
Birrell and the political scientist Raymond Sestito. Chipman and
Knopfelmacher were concerned with threats to social cohesion, while
Birrell's concern was that multiculturalism obscures the social costs
associated with large scale immigration that fall most heavily on the
most recently arrived and unskilled immigrants. Sestito's arguments were
based on the role of political parties. He argued that political parties
were instrumental in pursuing multicultural policies, and that these
policies would put strain on the political system and would not promote
better understanding in the Australian community.

It was the high-profile historian Geoffrey Blainey, however, who first
achieved mainstream recognition for the anti-multiculturalist cause when
he wrote that multiculturalism threatened to transform Australia into a
"cluster of tribes". In his 1984 book All for Australia, Blainey
criticised multiculturalism for tending to "emphasise the rights of
ethnic minorities at the expense of the majority of Australians" and
also for tending to be "anti-British", even though "people from the
United Kingdom and Ireland form the dominant class of pre-war immigrants
and the largest single group of post-war immigrants."

According to Blainey, such a policy, with its "emphasis on what is
different and on the rights of the new minority rather than the old
majority," was unnecessarily creating division and threatened national
cohesion. He argued that "the evidence is clear that many multicultural
societies have failed and that the human cost of the failure has been
high" and warned that "we should think very carefully about the perils
of converting Australia into a giant multicultural laboratory for the
assumed benefit of the peoples of the world."

In one of his numerous criticisms of multiculturalism, Blainey wrote:

Blainey remained a persistent critic of multiculturalism into the 1990s,
denouncing multiculturalism as "morally, intellectually and economically
... a sham".

The late historian John Hirst was another intellectual critic of
multiculturalism. He has argued that while multiculturalism might serve
the needs of ethnic politics and the demands of certain ethnic groups
for government funding for the promotion of their separate ethnic
identity, it was a perilous concept on which to base national policy.

Critics associated with the Centre for Population and Urban Research at
Monash University have argued that both Right and Left factions in the
Australian Labor Party have adopted a multicultural stance for the
purposes of increasing their support within the party. A manifestation
of this embrace of multiculturalism has been the creation of ethnic
branches within the Labor Party and ethnic branch stacking.

Following the upsurge of support for the One Nation Party in 1996,
Lebanese-born Australian anthropologist Ghassan Hage published a
critique in 1997 of Australian multiculturalism in the book White
Nation.

\section{Canada}\label{canada}

\begin{itemize}
\item
  \emph{Canadian Daniel Stoffman's book Who Gets In questions the policy
  of Canadian multiculturalism.}
\item
  \emph{Foreign born Canadian, Neil Bissoondath in his book Selling
  Illusions: The Cult of Multiculturalism in Canada, argues that
  official multiculturalism limits the freedom of minority members, by
  confining them to cultural and geographic ethnic enclaves.}
\end{itemize}

Many Québécois, despite an official national bilingualism policy, insist
that multiculturalism threatens to reduce them to just another ethnic
group. Quebec's policy seeks to promote interculturalism, welcoming
people of all origins while insisting that they integrate into Quebec's
majority French-speaking society. In 2008, a Consultation Commission on
Accommodation Practices Related to Cultural Differences, headed by
sociologist Gerard Bouchard and philosopher Charles Taylor, recognized
that Quebec is a de facto pluralist society, but that the Canadian
multiculturalism model "does not appear well suited to conditions in
Quebec".

According to a study conducted by The University of Victoria, many
Canadians do not feel a strong sense of belonging in Canada, or cannot
integrate themselves into society as a result of ethnic enclaves. Many
immigrants to Canada choose to live in ethnic enclaves because it can be
much easier than fitting in with mainstream Canadian culture.

Foreign born Canadian, Neil Bissoondath in his book Selling Illusions:
The Cult of Multiculturalism in Canada, argues that official
multiculturalism limits the freedom of minority members, by confining
them to cultural and geographic ethnic enclaves. He also argues that
cultures are very complex, and must be transmitted through close family
and kin relations. To him, the government view of cultures as being
about festivals and cuisine is a crude oversimplification that leads to
easy stereotyping.

Canadian Daniel Stoffman's book Who Gets In questions the policy of
Canadian multiculturalism. Stoffman points out that many cultural
practices, such as allowing dog meat to be served in restaurants and
street cockfighting, are simply incompatible with Canadian and Western
culture. He also raises concern about the number of recent immigrants
who are not being linguistically integrated into Canada (i.e., not
learning either English or French). He stresses that multiculturalism
works better in theory than in practice and Canadians need to be far
more assertive about valuing the "national identity of English-speaking
Canada".

\section{Germany}\label{germany}

\begin{itemize}
\item
  \emph{This has added to a growing debate within Germany on the levels
  of immigration, its effect on the country and the degree to which
  Muslim immigrants have integrated into German society.}
\item
  \emph{In October 2010, amid a nationwide controversy about Thilo
  Sarrazin's bestselling book Deutschland schafft sich ab ("Germany is
  abolishing Itself"), chancellor Angela Merkel of the conservative
  Christian Democratic Union judged attempts to build a multicultural
  society in Germany to have "failed, utterly failed".}
\end{itemize}

Criticisms of parallel societies established by some immigrant
communities increasingly came to the fore in the German public discourse
during the 1990s, giving rise to the concept of the Leitkultur ("lead
culture"). In October 2010, amid a nationwide controversy about Thilo
Sarrazin's bestselling book Deutschland schafft sich ab ("Germany is
abolishing Itself"), chancellor Angela Merkel of the conservative
Christian Democratic Union judged attempts to build a multicultural
society in Germany to have "failed, utterly failed". She added: "The
concept that we are now living side by side and are happy about it does
not work". She continued to say that immigrants should integrate and
adopt Germany's culture and values. This has added to a growing debate
within Germany on the levels of immigration, its effect on the country
and the degree to which Muslim immigrants have integrated into German
society. According to one poll around the time, one-third of Germans
believed the country was "overrun by foreigners".

\section{Italy}\label{italy}

\begin{itemize}
\item
  \emph{Many intellectuals have opposed multiculturalism among those:}
\end{itemize}

Italy has recently seen a substantial rise in immigration and an influx
of African immigrants.

Many intellectuals have opposed multiculturalism among those:

Ida Magli, professor emeritus of cultural anthropology at the University
of Rome. She was a contributor to the weekly L'Espresso and was a
columnist for the daily La Repubblica. She expressed criticism of
multicultural societies.

Oriana Fallaci was an Italian journalist, author, and political
interviewer. A partisan during World War II, she had a long and
successful journalistic career. Fallaci became famous worldwide for her
coverage of war and revolution, and her interviews with many world
leaders during the 1960s, 1970s, and 1980s. After retirement, she
returned to the spotlight after writing a series of controversial
articles and books critical of Islam and immigration.

\section{Japan}\label{japan}

\begin{itemize}
\item
  \emph{Japanese society, with its homogeneity, has traditionally
  rejected any need to recognize ethnic differences in Japan, even as
  such claims have been rejected by such ethnic minorities as the Ainu
  and Ryukyuans.}
\end{itemize}

Japanese society, with its homogeneity, has traditionally rejected any
need to recognize ethnic differences in Japan, even as such claims have
been rejected by such ethnic minorities as the Ainu and Ryukyuans.
Former Japanese Prime Minister (Deputy Prime Minister as of 26 December
2012) Taro Aso has called Japan a "one race" nation.

\section{Malaysia}\label{malaysia}

\begin{itemize}
\item
  \emph{Criticisms of multiculturalism have been periodically sparked by
  the entrenched constitutional position the Malay ethnicity enjoys
  through, inter alia, the Malaysian social contract.}
\item
  \emph{Malaysia is a multicultural society with a Muslim Malay majority
  and substantial Malaysian Chinese and Malaysian Indian minorities.}
\end{itemize}

Malaysia is a multicultural society with a Muslim Malay majority and
substantial Malaysian Chinese and Malaysian Indian minorities.
Criticisms of multiculturalism have been periodically sparked by the
entrenched constitutional position the Malay ethnicity enjoys through,
inter alia, the Malaysian social contract. Contrary to other countries,
in Malaysia affirmative action are often tailored to the needs of the
Malay majority population. In 2006, the forced removal of Hindu temples
across the country has led to accusations of "an unofficial policy of
Hindu temple-cleansing in Malaysia".

\section{Netherlands}\label{netherlands}

\begin{itemize}
\item
  \emph{In 2000, Paul Scheffer---a member of the Labour Party and
  subsequently a professor of urban studies---published his essay "The
  multicultural tragedy", an essay critical of both immigration and
  multiculturalism.}
\item
  \emph{Legal philosopher Paul Cliteur attacked multiculturalism in his
  book The Philosophy of Human Rights.}
\end{itemize}

Legal philosopher Paul Cliteur attacked multiculturalism in his book The
Philosophy of Human Rights. Cliteur rejects all political correctness on
the issue: Western culture, the Rechtsstaat (rule of law), and human
rights are superior to non-Western culture and values. They are the
product of the Enlightenment. Cliteur sees non-Western cultures not as
merely different but as anachronistic. He sees multiculturalism
primarily as an unacceptable ideology of cultural relativism, which
would lead to acceptance of barbaric practices, including those brought
to the Western World by immigrants. Cliteur lists infanticide, torture,
slavery, oppression of women, homophobia, racism, anti-Semitism, gangs,
female genital cutting, discrimination by immigrants, suttee, and the
death penalty. Cliteur compares multiculturalism to the moral acceptance
of Auschwitz, Joseph Stalin, Pol Pot and the Ku Klux Klan.

In 2000, Paul Scheffer---a member of the Labour Party and subsequently a
professor of urban studies---published his essay "The multicultural
tragedy", an essay critical of both immigration and multiculturalism.
Scheffer is a committed supporter of the nation-state, assuming that
homogeneity and integration are necessary for a society: the presence of
immigrants undermines this. A society does have a finite "absorptive
capacity" for those from other cultures, he says, but this has been
exceeded in the Netherlands. He specifically cites failure to
assimilate, spontaneous ethnic segregation, adaptation problems such as
school drop-out, unemployment, and high crime rates (see immigration and
crime), and opposition to secularism among Muslim immigrants as the main
problems resulting from immigration.

\section{United Kingdom}\label{united-kingdom}

\begin{itemize}
\item
  \emph{Journalist Ed West argued in his 2013 book, The Diversity
  Illusion, that the British political establishment had uncritically
  embraced multiculturalism without proper consideration of the
  downsides of ethnic diversity.}
\item
  \emph{There is now a debate in the UK over whether explicit
  multiculturalism and "social cohesion and inclusion" are in fact
  mutually exclusive.}
\end{itemize}

With considerable immigration after the Second World War making the UK
an increasingly ethnically and racially diverse state, race relations
policies have been developed that broadly reflect the principles of
multiculturalism, although there is no official national commitment to
the concept. This model has faced criticism on the grounds that it has
failed to sufficiently promote social integration, although some
commentators have questioned the dichotomy between diversity and
integration that this critique presumes. It has been argued that the UK
government has since 2001, moved away from policy characterised by
multiculturalism and towards the assimilation of minority communities.

Opposition has grown to state sponsored multicultural policies, with
some believing that it has been a costly failure. Critics of the policy
come from many parts of British society. There is now a debate in the UK
over whether explicit multiculturalism and "social cohesion and
inclusion" are in fact mutually exclusive. In the wake of the 7 July
2005 London bombings David Davis, the opposition Conservative shadow
home secretary, called on the government to scrap its "outdated" policy
of multiculturalism.

The British columnist Leo McKinstry has persistently criticized
multiculturalism, stating that "Britain is now governed by a suicide
cult bent on wiping out any last vestige of nationhood" and called
multiculturalism a "profoundly disturbing social experiment".

McKinstry also wrote:

Trevor Phillips, the head of the Commission for Racial Equality, who has
called for an official end to multicultural policy, has criticised
"politically correct liberals for their "misguided" pandering to the
ethnic lobby".

Journalist Ed West argued in his 2013 book, The Diversity Illusion, that
the British political establishment had uncritically embraced
multiculturalism without proper consideration of the downsides of ethnic
diversity. He wrote:

West has also argued:

In the May 2004 edition of Prospect Magazine, the editor David Goodhart
temporarily couched the debate on multiculturalism in terms of whether a
modern welfare state and a "good society" is sustainable as its citizens
become increasingly diverse.

In November 2005 John Sentamu, the Archbishop of York, stated,
"Multiculturalism has seemed to imply, wrongly for me: let other
cultures be allowed to express themselves but do not let the majority
culture at all tell us its glories, its struggles, its joys, its pains."
The Bishop of Rochester Michael Nazir-Ali was also critical, calling for
the Church to regain a prominent position in public life and blaming the
"newfangled and insecurely founded doctrine of multiculturalism" for
entrenching the segregation of communities.

Whilst minority cultures are allowed to remain distinct, British culture
and traditions are sometimes perceived as exclusive and adapted
accordingly, often without the consent of the local
population.{[}citation needed{]} For instance, Birmingham City Council
was heavily criticised when it was alleged to have renamed Christmas as
"Winterval" in 1998, although in truth it had done no such thing.

In August 2006, the community and local government secretary Ruth Kelly
made a speech perceived as signalling the end of multiculturalism as
official policy. In November 2006, Prime Minister Tony Blair stated that
Britain has certain "essential values" and that these are a "duty". He
did not reject multiculturalism outright, but he included British
heritage among the essential values:

\section{New Labour and
multiculturalism}\label{new-labour-and-multiculturalism}

\begin{itemize}
\item
  \emph{The Conservative party demanded an independent inquiry into the
  issue and alleged that the document showed that Labour had overseen a
  deliberate open-door policy on immigration to boost multiculturalism
  for political ends.}
\item
  \emph{In February 2011, the then Prime Minister David Cameron stated
  that the "doctrine of state multiculturalism" (promoted by the
  previous Labour government) had failed and will no longer be state
  policy.}
\end{itemize}

Renewed controversy on the subject came to the fore when Andrew
Neather---a former adviser to Jack Straw, Tony Blair and David
Blunkett---said that Labour ministers had a hidden agenda in allowing
mass immigration into Britain, to "change the face of Britain forever".
This alleged conspiracy has become known by the sobriquet "Neathergate".

According to Neather, who was present at closed meetings in 2000, a
secret Government report called for mass immigration to change Britain's
cultural make-up, and that "mass immigration was the way that the
government was going to make the UK truly multicultural".\\
Neather went on to say that "the policy was intended---even if this
wasn't its main purpose --- to rub the right's nose in diversity and
render their arguments out of date".

This was later affirmed after a request through the freedom of
information act secured access to the full version of a 2000 government
report on immigration that had been heavily edited on a previous
release. The Conservative party demanded an independent inquiry into the
issue and alleged that the document showed that Labour had overseen a
deliberate open-door policy on immigration to boost multiculturalism for
political ends.

In February 2011, the then Prime Minister David Cameron stated that the
"doctrine of state multiculturalism" (promoted by the previous Labour
government) had failed and will no longer be state policy. He stated
that the UK needed a stronger national identity and signalled a tougher
stance on groups promoting Islamist extremism. However, official
statistics showed that EU and non-EU mass immigration, together with
asylum seeker applications, all increased substantially during Cameron's
term in office.

\section{United States}\label{united-states}

\begin{itemize}
\item
  \emph{Huntington outlined the risks he associated with
  multiculturalism in his 2004 book Who Are We?}
\item
  \emph{A prominent criticism in the US, later echoed in Europe, Canada
  and Australia, was that multiculturalism undermined national unity,
  hindered social integration and cultural assimilation, and led to the
  fragmentation of society into several ethnic factions
  (Balkanization).}
\end{itemize}

The U.S. Congress passed the Emergency Quota Act in 1921, followed by
the Immigration Act of 1924. The Immigration Act of 1924 was aimed at
further restricting the Southern and Eastern Europeans, especially
Italians and Slavs, who had begun to enter the country in large numbers
beginning in the 1890s.

In the 1980s and 1990s many criticisms were expressed, from both the
left and right. Criticisms come from a wide variety of perspectives, but
predominantly from the perspective of liberal individualism, from
American conservatives concerned about shared traditional values, and
from a national unity perspective.

A prominent criticism in the US, later echoed in Europe, Canada and
Australia, was that multiculturalism undermined national unity, hindered
social integration and cultural assimilation, and led to the
fragmentation of society into several ethnic factions (Balkanization).

In 1991, Arthur M. Schlesinger, Jr., a former advisor to the Kennedy and
other US administrations and Pulitzer Prize winner, published a book
critical of multiculturalism with the title The Disuniting of America:
Reflections on a Multicultural Society.

In his 1991 work, Illiberal Education, Dinesh D'Souza argues that the
entrenchment of multiculturalism in American universities undermined the
universalist values that liberal education once attempted to foster. In
particular, he was disturbed by the growth of ethnic studies programs
(e.g., black studies).

The late Samuel P. Huntington, political scientist and author, known for
his Clash of Civilizations theory, described multiculturalism as
"basically an anti-Western ideology." According to Huntington,
multiculturalism had "attacked the identification of the United States
with Western civilization, denied the existence of a common American
culture, and promoted racial, ethnic, and other subnational cultural
identities and groupings." Huntington outlined the risks he associated
with multiculturalism in his 2004 book Who Are We? The Challenges to
America's National Identity.

\section{Diversity and social trust}\label{diversity-and-social-trust}

\begin{itemize}
\item
  \emph{Harvard professor of political science Robert D. Putnam
  conducted a nearly decade long study on how diversity affects social
  trust.}
\item
  \emph{In the presence of such ethnic diversity, Putnam maintains that}
\end{itemize}

Harvard professor of political science Robert D. Putnam conducted a
nearly decade long study on how diversity affects social trust. He
surveyed 26,200 people in 40 American communities, finding that when the
data were adjusted for class, income and other factors, the more
racially diverse a community is, the greater the loss of trust. People
in diverse communities "don't trust the local mayor, they don't trust
the local paper, they don't trust other people and they don't trust
institutions," writes Putnam. In the presence of such ethnic diversity,
Putnam maintains that

\section{Yugoslavia}\label{yugoslavia}

\begin{itemize}
\item
  \emph{The conflict had its roots in various underlying political,
  economic and cultural problems, which provided justifications for
  political and religious leaders, and manifested itself through often
  provoked and artificial created ethnic and religious tensions.}
\end{itemize}

Before World War II, major tensions arose from the last, monarchist
Yugoslavia's multi-ethnic makeup and absolute political and demographic
domination of the Serbs. The Yugoslav wars that took place between 1991
and 2001 were characterized by bitter ethnic conflicts between the
peoples of the former Yugoslavia, mostly between Serbs on the one side
and Croats, Bosniaks or Albanians on the other; but also between
Bosniaks and Croats in Bosnia and Macedonians and Albanians in North
Macedonia.

The conflict had its roots in various underlying political, economic and
cultural problems, which provided justifications for political and
religious leaders, and manifested itself through often provoked and
artificial created ethnic and religious tensions.

\section{Multiculturalism and Islam}\label{multiculturalism-and-islam}

\begin{itemize}
\item
  \emph{The belief behind this backlash on multiculturalism is that it
  creates friction within society.}
\end{itemize}

In an article in the Hudson Review, Bruce Bawer writes about what he
sees as a developing distaste toward the idea and policies of
multiculturalism in Europe, especially, as stated earlier, in the
Netherlands, Denmark, United Kingdom, Norway, Sweden, Austria and
Germany. The belief behind this backlash on multiculturalism is that it
creates friction within society.

\section{See also}\label{see-also}

\begin{itemize}
\item
  \emph{Ethnic penalty}
\item
  \emph{cultural assimilation)}
\item
  \emph{Immigration reduction}
\item
  \emph{Ethnic nepotism}
\end{itemize}

Ethnic nepotism

Ethnic penalty

Immigrant criminality

Immigration reduction

National assimilation (a.k.a. cultural assimilation)

Divide and rule

\section{Assimilation}\label{assimilation}

\begin{itemize}
\item
  \emph{Londonistan: How Britain is Creating a Terror State Within}
\item
  \emph{Stop Islamisation of Europe}
\item
  \emph{English Defence League}
\item
  \emph{Race traitor}
\item
  \emph{Criticism of Islam}
\item
  \emph{Criticism of Islamism}
\item
  \emph{Undercover Mosque}
\end{itemize}

Race traitor

Criticism of Islam

Criticism of Islamism

Undercover Mosque

Stop Islamisation of Europe

English Defence League

Londonistan: How Britain is Creating a Terror State Within

Eurabia

\section{References}\label{references}

\section{Further reading}\label{further-reading}
