\textbf{From Wikipedia, the free encyclopedia}

https://en.wikipedia.org/wiki/Gomillion\%20v.\%20Lightfoot\\
Licensed under CC BY-SA 3.0:\\
https://en.wikipedia.org/wiki/Wikipedia:Text\_of\_Creative\_Commons\_Attribution-ShareAlike\_3.0\_Unported\_License

\section{Gomillion v. Lightfoot}\label{gomillion-v.-lightfoot}

\begin{itemize}
\item
  \emph{Gomillion v. Lightfoot, 364 U.S. 339 (1960), was a United States
  Supreme Court decision that found an electoral district with
  boundaries created to disenfranchise blacks violated the Fifteenth
  Amendment.}
\end{itemize}

Gomillion v. Lightfoot, 364 U.S. 339 (1960), was a United States Supreme
Court decision that found an electoral district with boundaries created
to disenfranchise blacks violated the Fifteenth Amendment.

\section{Background}\label{background}

\begin{itemize}
\item
  \emph{In the city of Tuskegee, Alabama, after passage of the Civil
  Rights Act of 1957, activists had been slowly making progress in
  registering African-American voters, whose numbers on the rolls began
  to approach those of white registered voters.}
\item
  \emph{Gomillion and his attorneys appealed the case to the US Supreme
  Court.}
\end{itemize}

In the city of Tuskegee, Alabama, after passage of the Civil Rights Act
of 1957, activists had been slowly making progress in registering
African-American voters, whose numbers on the rolls began to approach
those of white registered voters. The city was the location of the
Tuskegee Institute, a historically black college, and a large Veterans
Administration hospital, both staffed entirely by African Americans.

In terms of total population, African Americans outnumbered whites in
the city by a four-to-one margin, and whites were worried about being
governed by the majority. Local white residents lobbied the Alabama
legislature to redefine the boundaries of the city. Without debate in
1957 and ignoring African-American protests, the legislature enacted
Local Law 140, to form a 28-sided city boundary by which nearly all
African-American voters would be excluded and no whites would be. The
act was written by state legislator Engelhardt, who was executive
secretary of the White Citizens' Council of Alabama and an advocate of
white supremacy. Charles G. Gomillion, a professor at Tuskegee, and
other African Americans protested; community activists mounted a boycott
against white-owned businesses in the city. Gomillion and others filed
suit against the city mayor and other officials, claiming that the act
was discriminatory in purpose under the Fourteenth Amendment's due
process and equal protection clause.

The U.S. District Court for the Middle District of Alabama, located in
the capital of Montgomery, headed by Judge Frank M. Johnson, dismissed
the case, ruling that the state had the right to draw boundaries of
election districts and jurisdictions. This ruling was upheld by the
Court of Appeals for the Fifth Circuit in New Orleans.

As head of Tuskegee, Booker T. Washington had promoted blacks advancing
by education and self-improvement, with the expectation of being
accepted by whites when they showed they were "deserving." At the time
of the US Supreme Court hearing of this case, journalist Bernard Taper
wrote,

The state's redrawing of the city's boundaries had the "unintended
effect of uniting Tuskegee Institute's African American intellectuals
with the less educated African Americans living outside the sphere of
the school. Some members of the school's faculty realized that
possessing advanced degrees ultimately provided them no different status
among the city's white establishment."

Gomillion and his attorneys appealed the case to the US Supreme Court.
The case was argued by Fred Gray, an experienced Alabama civil rights
attorney, and Robert L. Carter, lead counsel for the National
Association for the Advancement of Colored People (NAACP), with
assistance from Arthur D. Shores, who provided additional legal counsel.
The defendant team was led by James J. Carter (no relation).

(As of the early 21st century, the Alabama legislature continues to
exert considerable control over local and county affairs; few counties
in the state have home rule.)

\section{Decision}\label{decision}

\begin{itemize}
\item
  \emph{This case was cited in the Court's ruling in the Tennessee
  malapportionment case of Baker v. Carr (1962), which required state
  legislatures (including both houses of bicameral legislatures) to
  redistrict based on population, in order to reflect demographic
  changes and enable representation of urban populations.}
\item
  \emph{In this landmark voting rights case, the Supreme Court ruled on
  whether Act 140 of the Alabama legislature violated the Fifteenth
  Amendment.}
\item
  \emph{Alabama passed Act 140 in 1957, which changed the boundaries of
  the city of Tuskegee, Alabama.}
\end{itemize}

In this landmark voting rights case, the Supreme Court ruled on whether
Act 140 of the Alabama legislature violated the Fifteenth Amendment.
Alabama passed Act 140 in 1957, which changed the boundaries of the city
of Tuskegee, Alabama. It had previously been a square but the
legislature redrew it as a 28-sided figure, excluding all but a handful
of potential African-American voters and no white voters. Among those
excluded were the entire educated, professional faculty of the Tuskegee
University and doctors and staff of the Tuskegee Veterans Administration
Hospital.

Justice Frankfurter issued the opinion of the Court, which held that the
Act did violate the provision of the 15th Amendment prohibiting states
from denying anyone their right to vote on account of race, color, or
previous condition of servitude. Justice Whitaker concurred but he said
in his opinion that he believed the law should have been struck down
under the Equal Protection Clause of the Fourteenth Amendment.

This case was cited in the Court's ruling in the Tennessee
malapportionment case of Baker v. Carr (1962), which required state
legislatures (including both houses of bicameral legislatures) to
redistrict based on population, in order to reflect demographic changes
and enable representation of urban populations. It established the
principle of "one man, one vote" under the Equal Protection Clause.

\section{Whittaker's Concurrence}\label{whittakers-concurrence}

\begin{itemize}
\item
  \emph{This case should be examined under the Equal Protection Clause,
  not the 15th Amendment.}
\item
  \emph{But in this case, completely fencing African-American citizens
  out of a district is an unlawful segregation of black citizens and a
  clear violation of the Equal Protection Clause.}
\item
  \emph{Just because someone has been redistricted to vote in another
  district does not automatically mean his rights have been denied.}
\end{itemize}

This case should be examined under the Equal Protection Clause, not the
15th Amendment.

Just because someone has been redistricted to vote in another district
does not automatically mean his rights have been denied. It is not a
right to vote in a particular jurisdiction. But in this case, completely
fencing African-American citizens out of a district is an unlawful
segregation of black citizens and a clear violation of the Equal
Protection Clause.

\section{Subsequent history}\label{subsequent-history}

\begin{itemize}
\item
  \emph{Congress effectively negated Bolden in 1982 when it amended
  Section 2 of the Voting Rights Act, 42 U.S.C.}
\item
  \emph{In the 1980 case Mobile v. Bolden, the court limited its holding
  in Gomillion, ruling that racially discriminatory effect and intent
  would be necessary to prompt intervention by federal courts for
  violations of Section 2 of the Voting Rights Act.}
\end{itemize}

"The case showed that all state powers were subject to limitations
imposed by the U.S. Constitution; therefore, states were not insulated
from federal judicial review when they jeopardized federally protected
rights." The case was returned to the lower court; in 1961, under the
direction of Judge Johnson, the gerrymandering was reversed and the
original map of the city was reinstituted.

In the 1980 case Mobile v. Bolden, the court limited its holding in
Gomillion, ruling that racially discriminatory effect and intent would
be necessary to prompt intervention by federal courts for violations of
Section 2 of the Voting Rights Act.

Congress effectively negated Bolden in 1982 when it amended Section 2 of
the Voting Rights Act, 42 U.S.C. § 1973. Congress' amendments returned
the law to the pre-Bolden interpretation, under which violations of
Section 2 did not require a showing of racially discriminatory intent,
but it was sufficient to show discriminatory effect. This legislation
was important for the many subsequent cases challenging political and
electoral systems that resulted in dilution of voting or other effects
that deprived citizens of their ability to elect a candidate of their
choice.

\section{See also}\label{see-also}

\begin{itemize}
\item
  \emph{Baker v. Carr 369 U.S. 186 (1962)}
\item
  \emph{Brown v. Board of Education of Topeka (347 U.S. 483 (1954))}
\item
  \emph{Timeline of the civil rights movement}
\item
  \emph{Civil Rights Cases}
\item
  \emph{Hunt v. Cromartie 526 U.S. 541 (1999)}
\item
  \emph{List of United States Supreme Court cases, volume 364}
\end{itemize}

Gerrymandering

Hunt v. Cromartie 526 U.S. 541 (1999)

Baker v. Carr 369 U.S. 186 (1962)

List of United States Supreme Court cases, volume 364

Civil Rights Cases

Brown v. Board of Education of Topeka (347 U.S. 483 (1954))

Timeline of the civil rights movement

\section{References}\label{references}

\section{Further reading}\label{further-reading}

\begin{itemize}
\item
  \emph{Gomillion."}
\item
  \emph{Reaping the Whirlwind: The Civil Rights Movement in Tuskegee,
  New York: Alfred A. Knopf, 1985.}
\item
  \emph{'Gomillion versus Lightfoot:' The Tuskegee Gerrymander Case, New
  York: McGraw-Hill, 1962.}
\item
  \emph{Gomillion v. Lightfoot, 364 U.S. 339 (1960).}
\end{itemize}

Elwood, William A. "An Interview with Charles G. Gomillion." Callaloo 40
(Summer 1989): 576-99.

Gomillion, C. G. "The Negro Voter in the South." Journal of Negro
Education 26(3): 281-86.

Gomillion v. Lightfoot, 364 U.S. 339 (1960).

Norrell, Robert J. Reaping the Whirlwind: The Civil Rights Movement in
Tuskegee, New York: Alfred A. Knopf, 1985.

Taper, Bernard. 'Gomillion versus Lightfoot:' The Tuskegee Gerrymander
Case, New York: McGraw-Hill, 1962.

\section{External links}\label{external-links}

\begin{itemize}
\item
  \emph{\^{} 364 U.S. 339 (1960)}
\end{itemize}

\^{} 364 U.S. 339 (1960)
