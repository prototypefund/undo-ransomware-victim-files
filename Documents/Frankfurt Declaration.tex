\textbf{From Wikipedia, the free encyclopedia}

https://en.wikipedia.org/wiki/Frankfurt\%20Declaration\\
Licensed under CC BY-SA 3.0:\\
https://en.wikipedia.org/wiki/Wikipedia:Text\_of\_Creative\_Commons\_Attribution-ShareAlike\_3.0\_Unported\_License

\section{Frankfurt Declaration}\label{frankfurt-declaration}

\begin{itemize}
\item
  \emph{However, the Declaration stated that true socialism could only
  be achieved through democracy.}
\item
  \emph{The Frankfurt Declaration is the general name that refers to the
  set of principles titled Aims and Tasks of Democratic Socialism issued
  on 3 July 1951 by the Socialist International in Frankfurt, West
  Germany.}
\end{itemize}

The Frankfurt Declaration is the general name that refers to the set of
principles titled Aims and Tasks of Democratic Socialism issued on 3
July 1951 by the Socialist International in Frankfurt, West Germany. The
Declaration condemned capitalism for placing the "rights of ownership
before the rights of man", for allowing economic inequality and for its
historical support of imperialism and fascism.

The Frankfurt Declaration was updated at the 18th Congress of the
Socialist International in Stockholm in June 1989.

It declared that capitalism has coincided with "devastating crises and
mass unemployment". It praised the development of the welfare state as
challenging capitalism and declared its opposition to Bolshevik
communism.

It declared that socialism was an international movement that was plural
in nature that required different approaches in different circumstances.
However, the Declaration stated that true socialism could only be
achieved through democracy. According to the Declaration, the economic
goals of socialism include full employment, the welfare state and
achievement of public ownership through a variety of means, including
nationalization, creation of cooperatives to counter capitalist private
enterprise and/or securing rights for trade unions.

The Declaration stated that economic and social planning did not
necessarily have to be achieved in a centralized form, but could also be
achieved in decentralized forms. The Declaration denounced that all
forms of discrimination whether economic, legal, or political must be
abolished, including discrimination against women, races, regions and
other social groups. The Declaration denounced all forms of colonialism
and imperialism.

\section{Articles of principles of the Frankfurt
Declaration}\label{articles-of-principles-of-the-frankfurt-declaration}

\section{See also}\label{see-also}

\begin{itemize}
\item
  \emph{Social democracy}
\end{itemize}

Social democracy

\section{References}\label{references}
