\textbf{From Wikipedia, the free encyclopedia}

https://en.wikipedia.org/wiki/Alessandra\%20Stanley\\
Licensed under CC BY-SA 3.0:\\
https://en.wikipedia.org/wiki/Wikipedia:Text\_of\_Creative\_Commons\_Attribution-ShareAlike\_3.0\_Unported\_License

\section{Alessandra Stanley}\label{alessandra-stanley}

\begin{itemize}
\item
  \emph{Alessandra Stanley (born 1955) is an American journalist.}
\end{itemize}

Alessandra Stanley (born 1955) is an American journalist. As of 2019,
she is the co-founder of a weekly newsletter "for worldly cosmopolitans"
called Air Mail, alongside former Vanity Fair editor-in-chief Graydon
Carter.

\section{Biography}\label{biography}

\begin{itemize}
\item
  \emph{In 2003 she became the chief television critic for The New York
  Times.}
\item
  \emph{She has also written for The New York Times Magazine, The New
  Republic, GQ and Vogue.}
\item
  \emph{Stanley then moved to the New York Times as a foreign
  correspondent, first as co-chief of their Moscow bureau, and then Rome
  bureau chief.}
\item
  \emph{As of 2017, Stanley is no longer employed by the Times.}
\item
  \emph{Stanley lives in New York City with her daughter.}
\end{itemize}

She was born in Boston, MA and grew up in Washington, D.C. and Europe.
She is the daughter of defense adviser to NATO Timothy W. Stanley. She
studied literature at Harvard University and then became a correspondent
for Time, working overseas as well as in Los Angeles and in Washington,
D.C., where she covered the White House. Stanley then moved to the New
York Times as a foreign correspondent, first as co-chief of their Moscow
bureau, and then Rome bureau chief. In 2003 she became the chief
television critic for The New York Times. She has also written for The
New York Times Magazine, The New Republic, GQ and Vogue. Stanley lives
in New York City with her daughter.

In 1993, Alessandra Stanley received the Matrix Award from Women in
Communications, and in 1998, she received the Weintal Prize for
Diplomatic Reporting.

Among Stanley's notable columns are her critical take on the series
finale of The Sopranos, her assessment of Jerry Sandusky's denial of
charges of pedophilia to NBC and her coverage of Russian television on
the eve of the 2012 presidential election.

In the fall of 2011, Stanley taught a semester at Princeton University
entitled "Investigative Viewing: The Art of Television Criticism," an
"intensive introduction to criticism as it is undertaken at the highest
level of a cultural institution."

Several news and media organizations, including the Times, have
criticized the accuracy of Stanley's reporting. Among the articles that
they have criticized are a September 5, 2005, piece on Hurricane
Katrina, a 2005 article that mistakenly called the sitcom Everybody
Loves Raymond "All About Raymond," and a July 18, 2009, retrospective on
the career of Walter Cronkite that contained errors. In an August 2009
article examining the mistakes in the Cronkite piece, Clark Hoyt, the
Times's public editor, described Stanley as "much admired by editors for
the intellectual heft of her coverage of television" but "with a history
of errors." Then executive editor Bill Keller defended Stanley, saying
"She is --- in my opinion, among others --- a brilliant critic." In
April 2012, Salon contributor Glenn Greenwald described her New York
Times review of Julian Assange's television debut as "revealing,
reckless snideness" and "cowardly."

Stanley, who is white, wrote a Times article in September 2014 entitled
"Wrought in Rhimes's Image: Viola Davis Plays Shonda Rhimes's Latest
Tough Heroine" about television series How to Get Away with Murder and
the career of its African-American producer, Shonda Rhimes. Stanley
wrote, "When Shonda Rhimes writes her autobiography, it should be called
'How to Get Away With Being an Angry Black Woman'" and made comments
about African-Americans that were seen as offensive. Stanley's piece,
wrote the Times's Public Editor, Margaret Sullivan, "struck many readers
as completely off-base. Many called it offensive. Some went further,
saying it was racist." Stanley defended her piece, writing in an email
message to Talking Points Memo, "{[}t{]}he whole point of the piece -\/-
once you read past the first 140 characters -\/- is to praise Shonda
Rhimes for pushing back so successfully on a tiresome but insidious
stereotype." The organization Color of Change called for a retraction
from the Times.

As of 2017, Stanley is no longer employed by the Times.

\section{Personal life}\label{personal-life}

\begin{itemize}
\item
  \emph{Stanley was previously married to Michael Specter.}
\end{itemize}

Stanley was previously married to Michael Specter.

\section{References}\label{references}
