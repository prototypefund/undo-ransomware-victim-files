\textbf{From Wikipedia, the free encyclopedia}

https://en.wikipedia.org/wiki/Flowers\_in\_the\_Mirror\\
Licensed under CC BY-SA 3.0:\\
https://en.wikipedia.org/wiki/Wikipedia:Text\_of\_Creative\_Commons\_Attribution-ShareAlike\_3.0\_Unported\_License

\section{Flowers in the Mirror}\label{flowers-in-the-mirror}

\begin{itemize}
\item
  \emph{Flowers in the Mirror (simplified Chinese: 镜花缘; traditional
  Chinese: 鏡花緣; pinyin: Jìnghuāyuán), also translated as The Marriage
  of Flowers in the Mirror, or Romance of the Flowers in the Mirror, is
  a fantasy novel written by Li Ruzhen (Li Ju-chen), completed in the
  year of 1827 during the Qing dynasty.}
\end{itemize}

Flowers in the Mirror (simplified Chinese: 镜花缘; traditional Chinese:
鏡花緣; pinyin: Jìnghuāyuán), also translated as The Marriage of Flowers
in the Mirror, or Romance of the Flowers in the Mirror, is a fantasy
novel written by Li Ruzhen (Li Ju-chen), completed in the year of 1827
during the Qing dynasty.

The 100-chapter novel is well-known today for its various kinds of
fantasy stories and humorous writing style.

\section{Plot summary}\label{plot-summary}

\begin{itemize}
\item
  \emph{Flowers in the Mirror is set in the reign of the Empress Wu
  Zetian (684--705) in the Tang dynasty.}
\item
  \emph{During his journey, Tang Ao travels to the Country of Gentlemen,
  the Country of Women, the Country of Intestineless People, the Country
  of Sexless People, and the Country of Two-faced People, as well as
  many other countries.}
\item
  \emph{Tang Ao is the father of the incarnation of the Fairy of a
  Hundred Flowers.}
\end{itemize}

Flowers in the Mirror is set in the reign of the Empress Wu Zetian
(684--705) in the Tang dynasty. She took the throne from her own son,
Emperor Zhongzong of Tang. Empress Wu lets the power she is given go to
her head, and she demands that all of the flowers on the earth be in
bloom by the next morning. The flower-spirits fear her and follow her
orders, but they are then punished by the gods for doing so. Their
punishment is to live on earth. Once their penance is complete, they
will be allowed to go back to heaven again.

Tang Ao is the father of the incarnation of the Fairy of a Hundred
Flowers. The Empress suspects him of having had a part in plotting
rebellion against her and so she takes away his high scholarly rank and
leaves him with the lowest rank that one can attain. Tang Ao responds to
this by freeing himself from the coil of the mortal strife which binds
the soul to the body and resolves to become an immortal by cultivating
Tao.

Tang Ao is then told by a dream spirit that his destiny lies in foreign
parts; and he decides to go overseas by junk with his brother-in-law,
Merchant Lin. Tang Ao finds twelve of the incarnated flower-spirits
during his journey and helps them all with the difficulties that they
are having. Doing so enables him to become an immortal and, at the fair
mountain of Little Penglai, he disappears.

During his journey, Tang Ao travels to the Country of Gentlemen, the
Country of Women, the Country of Intestineless People, the Country of
Sexless People, and the Country of Two-faced People, as well as many
other countries. In the second half of the book Tang Ao's daughter goes
to Little Penglai to look for him after his disappearance. Also, the
incarnated flower-spirits take part in the "Imperial Examinations for
Women", and along with their husbands and brothers they rise up and
overthrow Empress Wu's rule, so restoring Emperor Tang Chung-tsung to
the throne.

\section{Theme}\label{theme}

\begin{itemize}
\item
  \emph{The real world during Li Ruzhen's time was rather a nation of
  contention that is opposed to the idea the "Comity of Nation" which
  suggests: "rather yielding than arguing" (好让不争/好讓不爭).}
\item
  \emph{The author's idea is reflected through his creation of ``The
  Nation of Girls'' (女儿国/女兒國), where a woman is the center of
  all.}
\end{itemize}

The novel is known for its contribution to the idea of feminism. It
eulogizes woman's talents, fully acknowledges their social status and
breaks the old concept of gender roles. The author's idea is reflected
through his creation of ``The Nation of Girls'' (女儿国/女兒國), where a
woman is the center of all. Men are made to dress like woman and stay at
home and look after the children, while women wear men's clothes, own
businesses and are involved in politics. From the emperor to officials
and servants, all are women. The author reinforces the idea that a
woman's capability is not any less than a man's and that they can do
just about anything that a man is capable of.

The author's ideal society is expressed through another model, the
"Comity of Nation" (君子国/君子國). The name itself in Chinese, junzi,
and is originally from Confucius tradition which means a person of noble
character and integrity. The real world during Li Ruzhen's time was
rather a nation of contention that is opposed to the idea the "Comity of
Nation" which suggests: "rather yielding than arguing"
(好让不争/好讓不爭). Under the author's imagination, there is no
putridity; no bribes between officials; citizens are all amicable and
live a joyful life under a steady and rich government.

\section{Background}\label{background}

\begin{itemize}
\item
  \emph{In writing Flowers in the Mirror, he was attempting to ask for
  equality of men and women.}
\item
  \emph{The author, Li Ruzhen, apparently believed in equal rights for
  women, a revolutionary idea in the feudal society of his day.}
\item
  \emph{Li had planned on writing a sequel to the Flowers in the Mirror
  but never did.}
\item
  \emph{He spent fifteen years writing the one hundred chapters of
  Flowers in the Mirror, and three years after it was published, he
  died.}
\end{itemize}

The author, Li Ruzhen, apparently believed in equal rights for women, a
revolutionary idea in the feudal society of his day. In writing Flowers
in the Mirror, he was attempting to ask for equality of men and women.
When Li writes about the different countries he is giving his
aspirations for an ideal society. The novel is full of fantasy which
shows the author's unique vision of life.

Li had planned on writing a sequel to the Flowers in the Mirror but
never did. He spent fifteen years writing the one hundred chapters of
Flowers in the Mirror, and three years after it was published, he died.
Li said that those of his friends who had been having trouble
emotionally laughed when they read the first one hundred chapters and
insisted that he get them published without writing any more.

\section{See also}\label{see-also}

\begin{itemize}
\item
  \emph{Strange Stories from a Chinese Studio by Pu Songling}
\item
  \emph{Other late-imperial fantasy novels, whose content and style
  comparable:}
\end{itemize}

The Peach Blossom Spring: Another journey to a flower-related utopia.

Other late-imperial fantasy novels, whose content and style comparable:

Journey to the West by Wu Cheng'en

Romance of Gods by Xu Zhonglin

Strange Stories from a Chinese Studio by Pu Songling

\section{References}\label{references}

\begin{itemize}
\item
  \emph{California: University of California Press..mw-parser-output
  cite.citation\{font-style:inherit\}.mw-parser-output .citation
  q\{quotes:"\textbackslash{}"""\textbackslash{}"""'""'"\}.mw-parser-output
  .citation .cs1-lock-free
  a\{background:url("//upload.wikimedia.org/wikipedia/commons/thumb/6/65/Lock-green.svg/9px-Lock-green.svg.png")no-repeat;background-position:right
  .1em center\}.mw-parser-output .citation .cs1-lock-limited
  a,.mw-parser-output .citation .cs1-lock-registration
  a\{background:url("//upload.wikimedia.org/wikipedia/commons/thumb/d/d6/Lock-gray-alt-2.svg/9px-Lock-gray-alt-2.svg.png")no-repeat;background-position:right
  .1em center\}.mw-parser-output .citation .cs1-lock-subscription
  a\{background:url("//upload.wikimedia.org/wikipedia/commons/thumb/a/aa/Lock-red-alt-2.svg/9px-Lock-red-alt-2.svg.png")no-repeat;background-position:right
  .1em center\}.mw-parser-output .cs1-subscription,.mw-parser-output
  .cs1-registration\{color:\#555\}.mw-parser-output .cs1-subscription
  span,.mw-parser-output .cs1-registration span\{border-bottom:1px
  dotted;cursor:help\}.mw-parser-output .cs1-ws-icon
  a\{background:url("//upload.wikimedia.org/wikipedia/commons/thumb/4/4c/Wikisource-logo.svg/12px-Wikisource-logo.svg.png")no-repeat;background-position:right
  .1em center\}.mw-parser-output
  code.cs1-code\{color:inherit;background:inherit;border:inherit;padding:inherit\}.mw-parser-output
  .cs1-hidden-error\{display:none;font-size:100\%\}.mw-parser-output
  .cs1-visible-error\{font-size:100\%\}.mw-parser-output
  .cs1-maint\{display:none;color:\#33aa33;margin-left:0.3em\}.mw-parser-output
  .cs1-subscription,.mw-parser-output
  .cs1-registration,.mw-parser-output
  .cs1-format\{font-size:95\%\}.mw-parser-output
  .cs1-kern-left,.mw-parser-output
  .cs1-kern-wl-left\{padding-left:0.2em\}.mw-parser-output
  .cs1-kern-right,.mw-parser-output
  .cs1-kern-wl-right\{padding-right:0.2em\}}
\item
  \emph{Zhu Meishu 朱眉叔 (1992): Li Ruzhen yu Jing hua yuan
  (李汝珍与镜花缘/李汝珍與鏡花緣 "Li Ruzhen and Flowers in the
  Mirror").}
\item
  \emph{Flowers in the Mirror.}
\item
  \emph{Li, Ju-chen (1965).}
\end{itemize}

Zhu Meishu 朱眉叔 (1992): Li Ruzhen yu Jing hua yuan
(李汝珍与镜花缘/李汝珍與鏡花緣 "Li Ruzhen and Flowers in the Mirror").
Shenyang: Liaoning jiaoyu chubanshe.

Li, Ju-chen (1965). Flowers in the Mirror. Translated by Lin Tai-yi.
California: University of California Press..mw-parser-output
cite.citation\{font-style:inherit\}.mw-parser-output .citation
q\{quotes:"\textbackslash{}"""\textbackslash{}"""'""'"\}.mw-parser-output
.citation .cs1-lock-free
a\{background:url("//upload.wikimedia.org/wikipedia/commons/thumb/6/65/Lock-green.svg/9px-Lock-green.svg.png")no-repeat;background-position:right
.1em center\}.mw-parser-output .citation .cs1-lock-limited
a,.mw-parser-output .citation .cs1-lock-registration
a\{background:url("//upload.wikimedia.org/wikipedia/commons/thumb/d/d6/Lock-gray-alt-2.svg/9px-Lock-gray-alt-2.svg.png")no-repeat;background-position:right
.1em center\}.mw-parser-output .citation .cs1-lock-subscription
a\{background:url("//upload.wikimedia.org/wikipedia/commons/thumb/a/aa/Lock-red-alt-2.svg/9px-Lock-red-alt-2.svg.png")no-repeat;background-position:right
.1em center\}.mw-parser-output .cs1-subscription,.mw-parser-output
.cs1-registration\{color:\#555\}.mw-parser-output .cs1-subscription
span,.mw-parser-output .cs1-registration span\{border-bottom:1px
dotted;cursor:help\}.mw-parser-output .cs1-ws-icon
a\{background:url("//upload.wikimedia.org/wikipedia/commons/thumb/4/4c/Wikisource-logo.svg/12px-Wikisource-logo.svg.png")no-repeat;background-position:right
.1em center\}.mw-parser-output
code.cs1-code\{color:inherit;background:inherit;border:inherit;padding:inherit\}.mw-parser-output
.cs1-hidden-error\{display:none;font-size:100\%\}.mw-parser-output
.cs1-visible-error\{font-size:100\%\}.mw-parser-output
.cs1-maint\{display:none;color:\#33aa33;margin-left:0.3em\}.mw-parser-output
.cs1-subscription,.mw-parser-output .cs1-registration,.mw-parser-output
.cs1-format\{font-size:95\%\}.mw-parser-output
.cs1-kern-left,.mw-parser-output
.cs1-kern-wl-left\{padding-left:0.2em\}.mw-parser-output
.cs1-kern-right,.mw-parser-output
.cs1-kern-wl-right\{padding-right:0.2em\}
