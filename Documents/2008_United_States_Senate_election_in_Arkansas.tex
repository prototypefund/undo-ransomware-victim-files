\textbf{From Wikipedia, the free encyclopedia}

https://en.wikipedia.org/wiki/2008\_United\_States\_Senate\_election\_in\_Arkansas\\
Licensed under CC BY-SA 3.0:\\
https://en.wikipedia.org/wiki/Wikipedia:Text\_of\_Creative\_Commons\_Attribution-ShareAlike\_3.0\_Unported\_License

\section{2008 United States Senate election in
Arkansas}\label{united-states-senate-election-in-arkansas}

\begin{itemize}
\item
  \emph{Kennedy received the highest ever vote share of any Green Party
  candidate running for U.S. Senate.}
\item
  \emph{The 2008 United States Senate election in Arkansas was held on
  November 4, 2008.}
\item
  \emph{As of 2019, this is the last U.S. Senate election in Arkansas
  won by a Democrat.}
\end{itemize}

The 2008 United States Senate election in Arkansas was held on November
4, 2008. Incumbent Democratic U.S. Senator Mark Pryor decided to run for
a second term. No Republican filed to challenge him. His only opponent
was Green Party candidate Rebekah Kennedy. He won re-election with
almost 80\% of the vote.

Kennedy received the highest ever vote share of any Green Party
candidate running for U.S. Senate.

As of 2019, this is the last U.S. Senate election in Arkansas won by a
Democrat.

\section{Candidates}\label{candidates}

\section{Democratic}\label{democratic}

\begin{itemize}
\item
  \emph{Mark Pryor, incumbent U.S.}
\end{itemize}

Mark Pryor, incumbent U.S. Senator

\section{Green}\label{green}

\begin{itemize}
\item
  \emph{Rebekah Kennedy, attorney and nominee for Attorney General in
  2006 \& 2010}
\end{itemize}

Rebekah Kennedy, attorney and nominee for Attorney General in 2006 \&
2010

\section{Campaign}\label{campaign}

\begin{itemize}
\item
  \emph{Formicola lost the GOP primaries for the Senate in 2004 and the
  U.S. House of Representatives in 2006.}
\item
  \emph{This is the highest percentage of the vote for any Green Party
  candidate running for U.S. Senate ever, and her 206,504 votes is the
  second most total votes received by a Green Party candidate for U.S.
  Senate after Medea Susan Benjamin's 326,828 votes in the 2000
  California Senate race.}
\item
  \emph{Kennedy's campaign, in addition to being record breaking for the
  Green Party, was also the strongest showing of any independent or
  third party candidate running for the U.S. Senate in 2008.}
\end{itemize}

On March 10, the state Republican Party announced it has no plans to
field a candidate against Pryor. The only Republican to express interest
in the race, health care executive Tom Formicola, decided not to run the
weekend before filing began. Formicola lost the GOP primaries for the
Senate in 2004 and the U.S. House of Representatives in 2006. As a
result, Pryor was one of the few freshman Senators to face no
major-party opposition in a reelection bid.

There had been speculation that former Governor Mike Huckabee would run
against Pryor if his presidential bid were unsuccessful, but on March 8,
Huckabee said he would not contest the race.

Pryor's biggest challenger was Green Party nominee Rebekah Kennedy, who
entered the race in April 2007. Kennedy became the sole challenger of
Democratic Senator Mark Pryor in his first race as an incumbent due to
the GOP declining to nominate anyone for the race. Kennedy received a
206,504 votes (20.54\%). This is the highest percentage of the vote for
any Green Party candidate running for U.S. Senate ever, and her 206,504
votes is the second most total votes received by a Green Party candidate
for U.S. Senate after Medea Susan Benjamin's 326,828 votes in the 2000
California Senate race. Kennedy's campaign, in addition to being record
breaking for the Green Party, was also the strongest showing of any
independent or third party candidate running for the U.S. Senate in
2008.

\section{Polling}\label{polling}

\begin{itemize}
\item
  \emph{Pryor was polled at 90\% in a poll without a challenger in
  March.}
\end{itemize}

Pryor was polled at 90\% in a poll without a challenger in March.

\section{Results}\label{results}

\section{See also}\label{see-also}

\begin{itemize}
\item
  \emph{2002 United States Senate election in Arkansas}
\end{itemize}

2002 United States Senate election in Arkansas

\section{References}\label{references}

\section{External links}\label{external-links}

\begin{itemize}
\item
  \emph{Arkansas U.S. Senate from OurCampaigns.com}
\item
  \emph{U.S. Congress candidates for Arkansas at Project Vote Smart}
\item
  \emph{Arkansas U.S. Senate race from 2008 Race Tracker}
\item
  \emph{Arkansas, U.S. Senate from CQ Politics}
\end{itemize}

Elections from the Arkansas Secretary of State

U.S. Congress candidates for Arkansas at Project Vote Smart

Arkansas, U.S. Senate from CQ Politics

Arkansas U.S. Senate from OurCampaigns.com

Arkansas U.S. Senate race from 2008 Race Tracker

Campaign contributions from OpenSecrets.org

official campaign websites (Archived)\\
Rebekah Kennedy, Green Party candidate\\
Mark Pryor, Democratic incumbent
