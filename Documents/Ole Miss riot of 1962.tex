\textbf{From Wikipedia, the free encyclopedia}

https://en.wikipedia.org/wiki/Ole\%20Miss\%20riot\%20of\%201962\\
Licensed under CC BY-SA 3.0:\\
https://en.wikipedia.org/wiki/Wikipedia:Text\_of\_Creative\_Commons\_Attribution-ShareAlike\_3.0\_Unported\_License

\section{Ole Miss riot of 1962}\label{ole-miss-riot-of-1962}

\begin{itemize}
\item
  \emph{Segregationists were protesting the enrollment of James
  Meredith, a black US military veteran, at the University of
  Mississippi (known affectionately as Ole Miss) at Oxford,
  Mississippi.}
\item
  \emph{The Ole Miss riot of 1962, or Battle of Oxford, was fought
  between Southern segregationists and federal and state forces
  beginning the night of September 30, 1962.}
\end{itemize}

The Ole Miss riot of 1962, or Battle of Oxford, was fought between
Southern segregationists and federal and state forces beginning the
night of September 30, 1962. Segregationists were protesting the
enrollment of James Meredith, a black US military veteran, at the
University of Mississippi (known affectionately as Ole Miss) at Oxford,
Mississippi. Two civilians, one a French journalist, were killed during
the night, and over 300 people were injured, including one-third of the
US Marshals deployed.

\section{Background}\label{background}

\begin{itemize}
\item
  \emph{In late September 1962, the administration of President John F.
  Kennedy had extensive discussions with Governor Barnett and his staff
  about protecting Meredith, but Barnett publicly vowed to keep the
  university segregated.}
\item
  \emph{Barnett was committed to maintaining civil order and reluctantly
  agreed to allow Meredith to register in exchange for a scripted
  face-saving event.}
\end{itemize}

In 1954 the U.S. Supreme Court had ruled in Brown v. Board of Education
that segregation in public schools was unconstitutional. Meredith
applied as a legitimate student with strong experience as an Air Force
veteran and good grades in completed coursework at Jackson State
University. Despite this, his entrance was barred first by university
officials and later by segregationist Governor Ross Barnett, who
nominated himself as registrar. On September 13, on television Barnett
stated:

In late September 1962, the administration of President John F. Kennedy
had extensive discussions with Governor Barnett and his staff about
protecting Meredith, but Barnett publicly vowed to keep the university
segregated. The President and Attorney General Robert F. Kennedy wanted
to avoid bringing in federal forces for several reasons. Robert Kennedy
hoped that legal means, along with an escort of U.S. Marshals, would be
enough to force the Governor to comply. He also was very concerned that
there might be a "mini-civil war" between the U.S. Army troops and armed
protesters.

Governor Barnett, under pressure from the courts, conducted secret
backdoor discussions in response to calls from the Kennedy
administration between Thursday, September 27 and Sunday, September 30.

Barnett was committed to maintaining civil order and reluctantly agreed
to allow Meredith to register in exchange for a scripted face-saving
event. Robert F. Kennedy ordered 500 U.S. Marshals to accompany Meredith
during his arrival and registration.

During the days preceding the riot, bands of racists drove cars through
Oxford and the campus with stickers stating that "The South shall rise
again," waving Confederate flags, and assaulting blacks.

\section{Events}\label{events}

\section{Start of the riot}\label{start-of-the-riot}

\begin{itemize}
\item
  \emph{In accordance with Barnett and Kennedy's plan, on Sunday
  evening, September 30, the day before the anticipated showdown,
  Meredith was flown to Oxford.}
\item
  \emph{He was quietly escorted by Mississippi Highway Patrol as he
  moved into a dorm room.}
\item
  \emph{At 7:30 p.m., Barnett announced on radio that Meredith had been
  brought to Mississippi by force.}
\end{itemize}

In accordance with Barnett and Kennedy's plan, on Sunday evening,
September 30, the day before the anticipated showdown, Meredith was
flown to Oxford. He was quietly escorted by Mississippi Highway Patrol
as he moved into a dorm room.

The federal marshals assembled on campus, supported by the 70th Army
Engineer Combat Battalion from Fort Campbell, Kentucky.

Responding to the federal presence, a crowd of a thousand, mostly
students--led by Edwin Walker--quickly crowded onto campus. Four days
earlier, Walker had said the following on radio:

As the scene grew more out of control, the highway patrol initially
helped hold off the crowds but, despite Barnett's renewed commitment,
those police were withdrawn by State Senator George Yarbrough starting
at about 7:25 p.m. local time. The student demonstration, joined by an
increasing number of other agitators, started to break out into a full
riot on the Oxford campus.

At 7:30 p.m., Barnett announced on radio that Meredith had been brought
to Mississippi by force. After signing his enrollment, Barnett said to
the Federal troops:

\section{Violence on the campus}\label{violence-on-the-campus}

\begin{itemize}
\item
  \emph{President Kennedy reluctantly decided to call in reinforcements
  in the middle of the night under the command of Brigadier General
  Charles Billingslea.}
\item
  \emph{Naval Hospital in Millington, Tennessee, were also sent to the
  university.}
\item
  \emph{Before they arrived, white rioters roaming the campus discovered
  Meredith was in Baxter Hall and started to assault it.}
\end{itemize}

The crowd eventually swelled to about three thousand. As its behavior
turned increasingly violent, including the death of a journalist, the
marshals ran out of tear gas defending the officials in the Lyceum.
President Kennedy reluctantly decided to call in reinforcements in the
middle of the night under the command of Brigadier General Charles
Billingslea. He ordered in U.S. Army military police from the 503rd and
716th Military Police Battalions, which had previously been readied for
deployment under cover of the nuclear war Exercise Spade Fork, plus the
U.S. Border Patrol and the federalized Mississippi National Guard. U.S.
Navy medical personnel (physicians and hospital corpsmen) attached to
the U.S. Naval Hospital in Millington, Tennessee, were also sent to the
university.

Before they arrived, white rioters roaming the campus discovered
Meredith was in Baxter Hall and started to assault it. Early in the
morning, as Gen. Billingslea's party entered the university gate, a
white mob attacked his staff car and set it on fire. Billingslea, the
Deputy Commanding General John Corley, and aide, Capt Harold Lyon, were
trapped inside the burning car, but they forced the door open, then
crawled 200 yards under gunfire from the mob to the University Lyceum
Building. The army did not return this fire.

To keep control, Gen Billingslea had established a series of escalating
secret code words for issuing ammunition down to the platoons, a second
one for issuing it to squads, and the third one for loading, none of
which could take place without the General confirming the secret codes.

By the end, one-third of the US Marshals, a total of 166 men, were
injured in the melee, and 40 soldiers and National Guardsmen were
wounded.

\includegraphics[width=5.50000in,height=3.92110in]{media/image1.jpg}\\
\emph{US Army trucks loaded with steel-helmeted US Marshals roll across
the University of Mississippi campus on October 3, 1962.}

\section{Aftermath}\label{aftermath}

\begin{itemize}
\item
  \emph{Finally, on October 1, 1962, Meredith became the first
  African-American student to be enrolled at the University of
  Mississippi, and attended his first class, in American History.}
\item
  \emph{Governor Barnett was fined \$10,000 and sentenced to jail for
  contempt.}
\item
  \emph{Meredith graduated from the university on August 18, 1963 with a
  degree in political science.}
\end{itemize}

Two men were murdered during the first night of the riots: French
journalist Paul Guihard, on assignment for Agence France-Presse (AFP),
who was found behind the Lyceum building with a gunshot wound to the
back; and 23-year-old Ray Gunter, a white jukebox repairman who had
visited the campus out of curiosity. Gunter was found with a bullet
wound in his forehead. Law enforcement officials described these as
execution-style killings.

Finally, on October 1, 1962, Meredith became the first African-American
student to be enrolled at the University of Mississippi, and attended
his first class, in American History. Meredith graduated from the
university on August 18, 1963 with a degree in political science. At
that time, there were still hundreds of troops guarding him 24 hours a
day although, in order to appease the local sensitivities, 4,000 Black
soldiers were removed from the Federal troops under Robert Kennedy's
secret orders.

Governor Barnett was fined \$10,000 and sentenced to jail for contempt.
The charges were later dismissed by the 5th Circuit Court of Appeals.

\section{Representation in other
media}\label{representation-in-other-media}

\section{Television}\label{television}

\begin{itemize}
\item
  \emph{Sports journalist Wright Thompson wrote an article "Ghosts of
  Mississippi" (2010), that described the riot and the football team's
  season that year.}
\item
  \emph{It was adapted as a documentary film for the ESPN 30 for 30
  series, entitled The Ghosts of Ole Miss (2012), about the 1962
  football team's perfect season and the early violence in the fall over
  integration of the historic university.}
\end{itemize}

Sports journalist Wright Thompson wrote an article "Ghosts of
Mississippi" (2010), that described the riot and the football team's
season that year. It was adapted as a documentary film for the ESPN 30
for 30 series, entitled The Ghosts of Ole Miss (2012), about the 1962
football team's perfect season and the early violence in the fall over
integration of the historic university.

\section{Music}\label{music}

\begin{itemize}
\item
  \emph{Gene Greenblath did "Talking Ole Miss"}
\end{itemize}

Several singers made songs about this event:

Bob Dylan wrote and sang about the events in his song "Oxford Town"

Phil Ochs wrote "The Ballad of Oxford"

Gene Greenblath did "Talking Ole Miss"

The Chad Mitchell Trio did the song "Alma Mater"

\section{Legacy}\label{legacy}

\begin{itemize}
\item
  \emph{Charles W. Eagle described Meredith's achievement by the
  following:}
\item
  \emph{The event is regarded as a pivotal moment in the history of
  civil rights in the United States.}
\item
  \emph{Because of the civil rights significance of Meredith's
  admission, the Lyceum-The Circle Historic District where the riot took
  place has been designated as a National Historic Landmark and state
  historic district.}
\end{itemize}

The event is regarded as a pivotal moment in the history of civil rights
in the United States.

Charles W. Eagle described Meredith's achievement by the following:

Because of the civil rights significance of Meredith's admission, the
Lyceum-The Circle Historic District where the riot took place has been
designated as a National Historic Landmark and state historic district.

A statue of Meredith has been erected on the campus to commemorate his
historic role.

The university conducted a series of programs for a year beginning in
2002 to mark the 40th anniversary of its integration. In 2012, it
initiated a yearlong series of programs to mark its 50th anniversary of
integration.

\section{See also}\label{see-also}

\section{References}\label{references}

\section{Bibliography}\label{bibliography}

\begin{itemize}
\item
  \emph{James Meredith and the Ole Miss riot~: A Soldier's story.}
\item
  \emph{"'The Fight for Men's Minds': The Aftermath of the Ole Miss Riot
  of 1962" (PDF).}
\item
  \emph{The Journal of Mississippi History.}
\item
  \emph{Jackson: University Press of Mississippi.}
\item
  \emph{71 (1): 1--53., reprinted at Mississippi Department of Archives
  and History website}
\end{itemize}

Branch, Taylor (1988). Parting the Waters: America in the King Years,
1954-63. New York: Simon \& Schuster. ISBN~978-0-671-68742-7.

Bryant, Nick (Autumn 2006). "Black Man Who Was Crazy Enough to Apply to
Ole Miss". The Journal of Blacks in Higher Education (53): 60--71.
Archived from the original on 2014-08-08. (restricted access -
subscription)

Eagles, Charles W. (Spring 2009). "'The Fight for Men's Minds': The
Aftermath of the Ole Miss Riot of 1962" (PDF). The Journal of
Mississippi History. 71 (1): 1--53., reprinted at Mississippi Department
of Archives and History website

Gallagher, Henry (2012). James Meredith and the Ole Miss riot~: A
Soldier's story. Jackson: University Press of Mississippi.
ISBN~978-1-61703-653-8.

Schlesinger, Arthur (2002) {[}1978{]}. Robert Kennedy and His Times. New
York: First Mariner Books. ISBN~0-618-21928-5.

\section{Further reading}\label{further-reading}

\begin{itemize}
\item
  \emph{The Price of Defiance: James Meredith and the Integration of Ole
  Miss.}
\end{itemize}

Eagles, Charles W. (2009). The Price of Defiance: James Meredith and the
Integration of Ole Miss. Chapel Hill, North Carolina: University of
North Carolina Press. ISBN~9780807895597.

Scheips, Paul J. (2005). "The Riot at Oxford". The Role of Federal
Military Forces in Domestic Disorders, 1945--1992 (PDF). Washington,
D.C.: Center of Military History, U.S. Army. pp.~101--136.
ISBN~9780160723612.

\section{External links}\label{external-links}

\begin{itemize}
\item
  \emph{"Integrating Ole Miss: A Civil Rights Milestone".}
\end{itemize}

"Integrating Ole Miss: A Civil Rights Milestone". John F. Kennedy
Presidential Library and Museum. Retrieved 31 May 2017.
