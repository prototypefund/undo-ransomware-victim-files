\textbf{From Wikipedia, the free encyclopedia}

https://en.wikipedia.org/wiki/Saku\_District\%2C\_Shinano\\
Licensed under CC BY-SA 3.0:\\
https://en.wikipedia.org/wiki/Wikipedia:Text\_of\_Creative\_Commons\_Attribution-ShareAlike\_3.0\_Unported\_License

\section{Saku District, Shinano}\label{saku-district-shinano}

\begin{itemize}
\item
  \emph{Saku (佐久郡, -gun) was a district located in Shinano Province
  (now Nagano Prefecture).}
\item
  \emph{The former Gunga(ancient district office) (郡衙) is estimated to
  be located at Nagatoro in the city of Saku.}
\item
  \emph{Due to land reforms, Saku District split into Minamisaku
  (南佐久郡) and Kitasaku (北佐久郡) Districts on January 14, 1879.}
\end{itemize}

Saku (佐久郡, -gun) was a district located in Shinano Province (now
Nagano Prefecture).\\
Due to land reforms, Saku District split into Minamisaku (南佐久郡) and
Kitasaku (北佐久郡) Districts on January 14, 1879.

The former Gunga(ancient district office) (郡衙) is estimated to be
located at Nagatoro in the city of Saku.

\section{Pesticide problems}\label{pesticide-problems}

\begin{itemize}
\item
  \emph{Cases of chronic and solar dermatitis gradually decreased, which
  may be explained by reductions in the use of allergenic or
  photosensitive sulfur agents and organophosphates.}
\item
  \emph{Cases of dermatitis caused by pesticide exposures, tabulated by
  the Division of Dermatology, Saku Central Hospital, Japan, from 1975
  to 2000 are described.}
\item
  \emph{Chemical burns from calcium polysulfide were responsible for
  most of the severe cases.}
\end{itemize}

Cases of dermatitis caused by pesticide exposures, tabulated by the
Division of Dermatology, Saku Central Hospital, Japan, from 1975 to 2000
are described. Dermatitis cases gradually decreased from 1975 to 2000,
presumably accelerated by the phase-out of dermatitis-causing
pesticides, including difolatan fungicide and salithion, an
organophosphate insecticide. Cases of chronic and solar dermatitis
gradually decreased, which may be explained by reductions in the use of
allergenic or photosensitive sulfur agents and organophosphates.
However, the ratios of chemical burns from irritant pesticides---calcium
polysulfide, dazomet, methyl bromide, chlorpicrin, paraquat/diquat,
organophosphorus, quintozene, and glyphosate---rose in those years.
Chemical burns from calcium polysulfide were responsible for most of the
severe cases.

\section{References}\label{references}

\section{External links}\label{external-links}

\begin{itemize}
\item
  \emph{History of Saku Region}
\end{itemize}

History of Saku Region

36°17′06″N 138°27′21″E / 36.284879°N 138.45575°E / 36.284879;
138.45575Coordinates: 36°17′06″N 138°27′21″E / 36.284879°N
138.45575°E / 36.284879; 138.45575
