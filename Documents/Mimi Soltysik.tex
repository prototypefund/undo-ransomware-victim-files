\textbf{From Wikipedia, the free encyclopedia}

https://en.wikipedia.org/wiki/Mimi\%20Soltysik\\
Licensed under CC BY-SA 3.0:\\
https://en.wikipedia.org/wiki/Wikipedia:Text\_of\_Creative\_Commons\_Attribution-ShareAlike\_3.0\_Unported\_License

\section{Mimi Soltysik}\label{mimi-soltysik}

\begin{itemize}
\item
  \emph{Soltysik also served as State Chair of the Socialist Party of
  California from its chartering in 2011 to 2017.}
\item
  \emph{He was the party's nominee for President of the United States in
  the 2016 election.}
\item
  \emph{Emidio "Mimi" Soltysik (born October 30, 1974) is an American
  democratic socialist political activist for the Socialist Party USA.}
\item
  \emph{From October 2013 to October 2015, Soltysik served as National
  Co-Chair of the Socialist Party USA.}
\end{itemize}

Emidio "Mimi" Soltysik (born October 30, 1974) is an American democratic
socialist political activist for the Socialist Party USA. He was the
party's nominee for President of the United States in the 2016 election.

From October 2013 to October 2015, Soltysik served as National Co-Chair
of the Socialist Party USA. Prior to that, he served as National Vice
Chair from 2011 to 2013. Soltysik also served as State Chair of the
Socialist Party of California from its chartering in 2011 to 2017.

\section{Background}\label{background}

\begin{itemize}
\item
  \emph{Soltysik graduated from Troy University and went on to study for
  an MPA degree at California State University, Northridge.}
\end{itemize}

Soltysik graduated from Troy University and went on to study for an MPA
degree at California State University, Northridge.

\section{Political career}\label{political-career}

\begin{itemize}
\item
  \emph{Soltysik was elected as the male co-chair of the Socialist Party
  USA for the 2013-2015 term at the party's 2013 national convention.}
\item
  \emph{In October 2015, Soltysik filed a Federal lawsuit against the
  California Secretary of State because he had been required to list
  "Party Preference: None" on the 2014 primary ballot; the California
  election law allows candidates to list only a party preference of a
  qualified party or "None" on the ballot, and the Socialist Party was
  not a qualified party in California.}
\item
  \emph{He is also the chair of the Socialist Party of California.}
\end{itemize}

Soltysik was elected as the male co-chair of the Socialist Party USA for
the 2013-2015 term at the party's 2013 national convention. He is also
the chair of the Socialist Party of California.

In 2014, Soltysik was one of eight candidates in the primary for
California's 62nd State Assembly district. Under California's
nonpartisan blanket primary system, the top two candidates from the
primary, regardless of party, advance to the general election. Soltysik
finished in 7th place with 2.5\% of the vote.

In October 2015, Soltysik filed a Federal lawsuit against the California
Secretary of State because he had been required to list "Party
Preference: None" on the 2014 primary ballot; the California election
law allows candidates to list only a party preference of a qualified
party or "None" on the ballot, and the Socialist Party was not a
qualified party in California.

\section{2016 presidential campaign}\label{presidential-campaign}

\begin{itemize}
\item
  \emph{The party did not have automatic ballot access in any state.}
\item
  \emph{Soltysik has been interviewed about socialist issues by web
  sites including MTV Hive and Bloomberg Politics.}
\item
  \emph{On October 17, 2015, the Socialist Party USA's national
  convention nominated Soltysik for president and Angela Nicole Walker
  for vice president.}
\end{itemize}

On October 17, 2015, the Socialist Party USA's national convention
nominated Soltysik for president and Angela Nicole Walker for vice
president. The party did not have automatic ballot access in any state.

Soltysik has been interviewed about socialist issues by web sites
including MTV Hive and Bloomberg Politics.

Soltysik has been interviewed regarding his presidential bid in a number
of outlets, including the Independent Voter Network, The North Star and
the Hampton Institute.

In April 2016, Soltysik was interviewed on CNBC regarding growing
support for socialism in the United States.

The Soltysik/Walker ticket received 2,704 total votes including
write-ins. He also won 4.2\% of the vote in Guam, though Guam has no
representation in the Electoral College.

\section{References}\label{references}

\section{External links}\label{external-links}

\begin{itemize}
\item
  \emph{2016 campaign web site}
\end{itemize}

2016 campaign web site
