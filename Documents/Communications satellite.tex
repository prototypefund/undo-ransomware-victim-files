\textbf{From Wikipedia, the free encyclopedia}

https://en.wikipedia.org/wiki/Communications\%20satellite\\
Licensed under CC BY-SA 3.0:\\
https://en.wikipedia.org/wiki/Wikipedia:Text\_of\_Creative\_Commons\_Attribution-ShareAlike\_3.0\_Unported\_License

\includegraphics[width=5.50000in,height=3.92308in]{media/image1.jpg}\\
\emph{An Advanced Extremely High Frequency communications orion
satellite centre relays secure communications for the United States\\
and other allied countries.}

\section{Communications satellite}\label{communications-satellite}

\begin{itemize}
\item
  \emph{There are 2,134 communications satellites in Earth's orbit, used
  by both private and government organizations.}
\item
  \emph{A communications satellite is an artificial satellite that
  relays and amplifies radio telecommunications signals via a
  transponder; it creates a communication channel between a source
  transmitter and a receiver at different locations on Earth.}
\end{itemize}

A communications satellite is an artificial satellite that relays and
amplifies radio telecommunications signals via a transponder; it creates
a communication channel between a source transmitter and a receiver at
different locations on Earth. Communications satellites are used for
television, telephone, radio, internet, and military applications. There
are 2,134 communications satellites in Earth's orbit, used by both
private and government organizations. Many are in geostationary orbit
22,200 miles (35,700~km) above the equator, so that the satellite
appears stationary at the same point in the sky, so the satellite dish
antennas of ground stations can be aimed permanently at that spot and do
not have to move to track it.

The high frequency radio waves used for telecommunications links travel
by line of sight and so are obstructed by the curve of the Earth. The
purpose of communications satellites is to relay the signal around the
curve of the Earth allowing communication between widely separated
geographical points. Communications satellites use a wide range of radio
and microwave frequencies. To avoid signal interference, international
organizations have regulations for which frequency ranges or "bands"
certain organizations are allowed to use. This allocation of bands
minimizes the risk of signal interference.

\section{History}\label{history}

\begin{itemize}
\item
  \emph{The first artificial Earth satellite was Sputnik 1.}
\item
  \emph{The world's first inflatable satellite --- or "satelloon", as
  they were informally known --- helped lay the foundation of today's
  satellite communications.}
\item
  \emph{Syncom 2 was the first communications satellite in a
  geosynchronous orbit.}
\item
  \emph{Its successor, Syncom 3 was the first geostationary
  communications satellite.}
\item
  \emph{Passive satellites were the first communications satellites, but
  are little used now.}
\end{itemize}

The concept of the geostationary communications satellite was first
proposed by Arthur C. Clarke, along with Vahid K. Sanadi building on
work by Konstantin Tsiolkovsky. In October 1945, Clarke published an
article titled "Extraterrestrial Relays" in the British magazine
Wireless World. The article described the fundamentals behind the
deployment of artificial satellites in geostationary orbits for the
purpose of relaying radio signals. Thus, Arthur C. Clarke is often
quoted as being the inventor of the communications satellite and the
term 'Clarke Belt' employed as a description of the orbit.

Decades later a project named Communication Moon Relay was a
telecommunication project carried out by the United States Navy. Its
objective was to develop a secure and reliable method of wireless
communication by using the Moon as a passive reflector and a natural
communications satellite.

The first artificial Earth satellite was Sputnik 1. Put into orbit by
the Soviet Union on October 4, 1957, it was equipped with an on-board
radio-transmitter that worked on two frequencies: 20.005 and 40.002~
MHz. Sputnik 1 was launched as a major step in the exploration of space
and rocket development. However, it was not placed in orbit for the
purpose of sending data from one point on earth to another.

The first satellite to relay communications was Pioneer 1, an intended
lunar probe. Though the spacecraft only made it about halfway to the
moon, it flew high enough to carry out the proof of concept relay of
telemetry across the world, first from Cape Canaveral to Manchester,
England; then from Hawaii to Cape Canaveral; and finally, across the
world from Hawaii to Manchester.

The first satellite purpose-built to relay communications was NASA's
Project SCORE in 1958, which used a tape recorder to store and forward
voice messages. It was used to send a Christmas greeting to the world
from U.S. President Dwight D. Eisenhower. Courier 1B, built by Philco,
launched in 1960, was the world's first active repeater satellite.

The first artificial satellite used solely to further advances in global
communications was a balloon named Echo 1. Echo 1 was the world's first
artificial communications satellite capable of relaying signals to other
points on Earth. It soared 1,600 kilometres (1,000~mi) above the planet
after its Aug. 12, 1960 launch, yet relied on humanity's oldest flight
technology --- ballooning. Launched by NASA, Echo 1 was a 30-metre
(100~ft) aluminized PET film balloon that served as a passive reflector
for radio communications. The world's first inflatable satellite --- or
"satelloon", as they were informally known --- helped lay the foundation
of today's satellite communications. The idea behind a communications
satellite is simple: Send data up into space and beam it back down to
another spot on the globe. Echo 1 accomplished this by essentially
serving as an enormous mirror, 10 stories tall, that could be used to
reflect communications signals.

There are two major classes of communications satellites, passive and
active. Passive satellites only reflect the signal coming from the
source, toward the direction of the receiver. With passive satellites,
the reflected signal is not amplified at the satellite, and only a very
small amount of the transmitted energy actually reaches the receiver.
Since the satellite is so far above Earth, the radio signal is
attenuated due to free-space path loss, so the signal received on Earth
is very, very weak. Active satellites, on the other hand, amplify the
received signal before retransmitting it to the receiver on the ground.
Passive satellites were the first communications satellites, but are
little used now.\\
Telstar was the second active, direct relay communications satellite.
Belonging to AT\&T as part of a multi-national agreement between AT\&T,
Bell Telephone Laboratories, NASA, the British General Post Office, and
the French National PTT (Post Office) to develop satellite
communications, it was launched by NASA from Cape Canaveral on July 10,
1962, in the first privately sponsored space launch. Relay 1 was
launched on December 13, 1962, and it became the first satellite to
transmit across the Pacific Ocean on November 22, 1963.

An immediate antecedent of the geostationary satellites was the Hughes
Aircraft Company's Syncom 2, launched on July 26, 1963. Syncom 2 was the
first communications satellite in a geosynchronous orbit. It revolved
around the earth once per day at constant speed, but because it still
had north-south motion, special equipment was needed to track it. Its
successor, Syncom 3 was the first geostationary communications
satellite. Syncom 3 obtained a geosynchronous orbit, without a
north-south motion, making it appear from the ground as a stationary
object in the sky.

Beginning with the Mars Exploration Rovers, landers on the surface of
Mars have used orbiting spacecraft as communications satellites for
relaying their data to Earth. The landers use UHF transmitters to send
their data to the orbiters, which then relay the data to Earth using
either X band or Ka band frequencies. These higher frequencies, along
with more powerful transmitters and larger antennas, permit the orbiters
to send the data much faster than the landers could manage transmitting
directly to Earth, which conserves valuable time on the NASA Deep Space
Network.

\section{Satellite orbits}\label{satellite-orbits}

\begin{itemize}
\item
  \emph{The advantage of this orbit is that ground antennas do not have
  to track the satellite across the sky, they can be fixed to point at
  the location in the sky the satellite appears.}
\item
  \emph{This is because the satellite's orbital period is the same as
  the rotation rate of the Earth.}
\item
  \emph{Medium Earth orbit (MEO) satellites are closer to Earth.}
\end{itemize}

Communications satellites usually have one of three primary types of
orbit, while other orbital classifications are used to further specify
orbital details.

Geostationary satellites have a geostationary orbit (GEO), which is
36,000 kilometres (22,000~mi) from Earth's surface. This orbit has the
special characteristic that the apparent position of the satellite in
the sky when viewed by a ground observer does not change, the satellite
appears to "stand still" in the sky. This is because the satellite's
orbital period is the same as the rotation rate of the Earth. The
advantage of this orbit is that ground antennas do not have to track the
satellite across the sky, they can be fixed to point at the location in
the sky the satellite appears.

Medium Earth orbit (MEO) satellites are closer to Earth. Orbital
altitudes range from 2,000 to 36,000 kilometres (1,200 to 22,400~mi)
above Earth.

The region below medium orbits is referred to as low Earth orbit (LEO),
and is about 160 to 2,000 kilometres (99 to 1,243~mi) above Earth.

As satellites in MEO and LEO orbit the Earth faster, they do not remain
visible in the sky to a fixed point on Earth continually like a
geostationary satellite, but appear to a ground observer to cross the
sky and "set" when they go behind the Earth. Therefore, to provide
continuous communications capability with these lower orbits requires a
larger number of satellites, so one will always be in the sky for
transmission of communication signals. However, due to their relatively
small distance to the Earth their signals are stronger.{[}clarification
needed{]}

\section{Low Earth orbit (LEO)}\label{low-earth-orbit-leo}

\begin{itemize}
\item
  \emph{In addition, satellites in low earth orbit change their position
  relative to the ground position quickly.}
\item
  \emph{Low-Earth-orbiting satellites are less expensive to launch into
  orbit than geostationary satellites and, due to proximity to the
  ground, do not require as high signal strength (Recall that signal
  strength falls off as the square of the distance from the source, so
  the effect is dramatic).}
\end{itemize}

A low Earth orbit (LEO) typically is a circular orbit about 160 to 2,000
kilometres (99 to 1,243~mi) above the earth's surface and,
correspondingly, a period (time to revolve around the earth) of about 90
minutes.

Because of their low altitude, these satellites are only visible from
within a radius of roughly 1,000 kilometres (620~mi) from the
sub-satellite point. In addition, satellites in low earth orbit change
their position relative to the ground position quickly. So even for
local applications, a large number of satellites are needed if the
mission requires uninterrupted connectivity.

Low-Earth-orbiting satellites are less expensive to launch into orbit
than geostationary satellites and, due to proximity to the ground, do
not require as high signal strength (Recall that signal strength falls
off as the square of the distance from the source, so the effect is
dramatic). Thus there is a trade off between the number of satellites
and their cost.

In addition, there are important differences in the onboard and ground
equipment needed to support the two types of missions.

\section{Satellite constellation}\label{satellite-constellation}

\begin{itemize}
\item
  \emph{The Iridium system has 66 satellites.}
\item
  \emph{A group of satellites working in concert is known as a satellite
  constellation.}
\item
  \emph{Two such constellations, intended to provide satellite phone
  services, primarily to remote areas, are the Iridium and Globalstar
  systems.}
\item
  \emph{This will be the case with the CASCADE system of Canada's
  CASSIOPE communications satellite.}
\end{itemize}

A group of satellites working in concert is known as a satellite
constellation. Two such constellations, intended to provide satellite
phone services, primarily to remote areas, are the Iridium and
Globalstar systems. The Iridium system has 66 satellites.

It is also possible to offer discontinuous coverage using a
low-Earth-orbit satellite capable of storing data received while passing
over one part of Earth and transmitting it later while passing over
another part. This will be the case with the CASCADE system of Canada's
CASSIOPE communications satellite. Another system using this store and
forward method is Orbcomm.

\section{Medium Earth orbit (MEO)}\label{medium-earth-orbit-meo}

\begin{itemize}
\item
  \emph{MEO satellites are similar to LEO satellites in functionality.}
\item
  \emph{One disadvantage is that a MEO satellite's distance gives it a
  longer time delay and weaker signal than a LEO satellite, although
  these limitations are not as severe as those of a GEO satellite.}
\item
  \emph{Typically the orbit of a medium earth orbit satellite is about
  16,000 kilometres (10,000~mi) above earth.}
\end{itemize}

A MEO is a satellite in orbit somewhere between 2,000 and 35,786
kilometres (1,243 and 22,236~mi) above the earth's surface. MEO
satellites are similar to LEO satellites in functionality. MEO
satellites are visible for much longer periods of time than LEO
satellites, usually between 2 and 8 hours. MEO satellites have a larger
coverage area than LEO satellites. A MEO satellite's longer duration of
visibility and wider footprint means fewer satellites are needed in a
MEO network than a LEO network. One disadvantage is that a MEO
satellite's distance gives it a longer time delay and weaker signal than
a LEO satellite, although these limitations are not as severe as those
of a GEO satellite.

Like LEOs, these satellites don't maintain a stationary distance from
the earth. This is in contrast to the geostationary orbit, where
satellites are always 35,786 kilometres (22,236~mi) from the earth.

Typically the orbit of a medium earth orbit satellite is about 16,000
kilometres (10,000~mi) above earth. In various patterns, these
satellites make the trip around earth in anywhere from 2 to 8 hours.

\section{Example}\label{example}

\begin{itemize}
\item
  \emph{In 1962, the first communications satellite, Telstar, was
  launched.}
\item
  \emph{It was a medium earth orbit satellite designed to help
  facilitate high-speed telephone signals.}
\end{itemize}

In 1962, the first communications satellite, Telstar, was launched. It
was a medium earth orbit satellite designed to help facilitate
high-speed telephone signals. Although it was the first practical way to
transmit signals over the horizon, its major drawback was soon realized.
Because its orbital period of about 2.5 hours did not match the Earth's
rotational period of 24 hours, continuous coverage was impossible. It
was apparent that multiple MEOs needed to be used in order to provide
continuous coverage.

\section{Geostationary orbit (GEO)}\label{geostationary-orbit-geo}

\begin{itemize}
\item
  \emph{To an observer on Earth, a satellite in a geostationary orbit
  appears motionless, in a fixed position in the sky.}
\item
  \emph{A geostationary orbit is useful for communications because
  ground antennas can be aimed at the satellite without their having to
  track the satellite's motion.}
\end{itemize}

To an observer on Earth, a satellite in a geostationary orbit appears
motionless, in a fixed position in the sky. This is because it revolves
around the Earth at Earth's own angular velocity (one revolution per
sidereal day, in an equatorial orbit).

A geostationary orbit is useful for communications because ground
antennas can be aimed at the satellite without their having to track the
satellite's motion. This is relatively inexpensive.

In applications that require a large number of ground antennas, such as
DirecTV distribution, the savings in ground equipment can more than
outweigh the cost and complexity of placing a satellite into orbit.

\section{Examples}\label{examples}

\begin{itemize}
\item
  \emph{It was the first geostationary satellite for telecommunications
  over the Atlantic Ocean.}
\item
  \emph{By 2000, Hughes Space and Communications (now Boeing Satellite
  Development Center) had built nearly 40 percent of the more than one
  hundred satellites in service worldwide.}
\item
  \emph{On May 30, 1977, the first geostationary communications
  satellite in the world to be three-axis stabilized was launched: the
  experimental satellite ATS-6 built for NASA.}
\end{itemize}

The first geostationary satellite was Syncom 3, launched on August 19,
1964, and used for communication across the Pacific starting with
television coverage of the 1964 Summer Olympics. Shortly after Syncom 3,
Intelsat I, aka Early Bird, was launched on April 6, 1965, and placed in
orbit at 28° west longitude. It was the first geostationary satellite
for telecommunications over the Atlantic Ocean.

On November 9, 1966, Canada's first geostationary satellite serving the
continent, Anik A1, was launched by Telesat Canada, with the United
States following suit with the launch of Westar 1 by Western Union on
April 13, 1967.

On May 30, 1977, the first geostationary communications satellite in the
world to be three-axis stabilized was launched: the experimental
satellite ATS-6 built for NASA.

After the launches of the Telstar through Westar 1 satellites, RCA
Americom (later GE Americom, now SES) launched Satcom 1 in 1975. It was
Satcom 1 that was instrumental in helping early cable TV channels such
as WTBS (now TBS), HBO, CBN (now Freeform) and The Weather Channel
become successful, because these channels distributed their programming
to all of the local cable TV headends using the satellite. Additionally,
it was the first satellite used by broadcast television networks in the
United States, like ABC, NBC, and CBS, to distribute programming to
their local affiliate stations. Satcom 1 was widely used because it had
twice the communications capacity of the competing Westar 1 in America
(24 transponders as opposed to the 12 of Westar 1), resulting in lower
transponder-usage costs. Satellites in later decades tended to have even
higher transponder numbers.

By 2000, Hughes Space and Communications (now Boeing Satellite
Development Center) had built nearly 40 percent of the more than one
hundred satellites in service worldwide. Other major satellite
manufacturers include Space Systems/Loral, Orbital Sciences Corporation
with the Star Bus series, Indian Space Research Organisation, Lockheed
Martin (owns the former RCA Astro Electronics/GE Astro Space business),
Northrop Grumman, Alcatel Space, now Thales Alenia Space, with the
Spacebus series, and Astrium.

\section{Molniya orbit}\label{molniya-orbit}

\begin{itemize}
\item
  \emph{In November 1967 Soviet engineers created a unique system of
  national TV network of satellite television, called Orbita, that was
  based on Molniya satellites.}
\item
  \emph{(Elevation is the extent of the satellite's position above the
  horizon.}
\item
  \emph{Thus, a satellite at the horizon has zero elevation and a
  satellite directly overhead has elevation of 90 degrees.)}
\end{itemize}

Geostationary satellites must operate above the equator and therefore
appear lower on the horizon as the receiver gets farther from the
equator. This will cause problems for extreme northerly latitudes,
affecting connectivity and causing multipath interference (caused by
signals reflecting off the ground and into the ground antenna).

Thus, for areas close to the North (and South) Pole, a geostationary
satellite may appear below the horizon. Therefore, Molniya orbit
satellites have been launched, mainly in Russia, to alleviate this
problem.

Molniya orbits can be an appealing alternative in such cases. The
Molniya orbit is highly inclined, guaranteeing good elevation over
selected positions during the northern portion of the orbit. (Elevation
is the extent of the satellite's position above the horizon. Thus, a
satellite at the horizon has zero elevation and a satellite directly
overhead has elevation of 90 degrees.)

The Molniya orbit is designed so that the satellite spends the great
majority of its time over the far northern latitudes, during which its
ground footprint moves only slightly. Its period is one half day, so
that the satellite is available for operation over the targeted region
for six to nine hours every second revolution. In this way a
constellation of three Molniya satellites (plus in-orbit spares) can
provide uninterrupted coverage.

The first satellite of the Molniya series was launched on April 23, 1965
and was used for experimental transmission of TV signals from a Moscow
uplink station to downlink stations located in Siberia and the Russian
Far East, in Norilsk, Khabarovsk, Magadan and Vladivostok. In November
1967 Soviet engineers created a unique system of national TV network of
satellite television, called Orbita, that was based on Molniya
satellites.

\section{Polar orbit}\label{polar-orbit}

\begin{itemize}
\item
  \emph{In the United States, the National Polar-orbiting Operational
  Environmental Satellite System (NPOESS) was established in 1994 to
  consolidate the polar satellite operations of\\
  NASA (National Aeronautics and Space Administration)\\
  NOAA (National Oceanic and Atmospheric Administration).}
\item
  \emph{NPOESS manages a number of satellites for various purposes; for
  example, METSAT for meteorological satellite, EUMETSAT for the
  European branch of the program, and METOP for meteorological
  operations.}
\end{itemize}

In the United States, the National Polar-orbiting Operational
Environmental Satellite System (NPOESS) was established in 1994 to
consolidate the polar satellite operations of\\
NASA (National Aeronautics and Space Administration)\\
NOAA (National Oceanic and Atmospheric Administration). NPOESS manages a
number of satellites for various purposes; for example, METSAT for
meteorological satellite, EUMETSAT for the European branch of the
program, and METOP for meteorological operations.

These orbits are sun synchronous, meaning that they cross the equator at
the same local time each day. For example, the satellites in the NPOESS
(civilian) orbit will cross the equator, going from south to north, at
times 1:30 P.M., 5:30 P.M., and 9:30 P.M.

\section{Structure}\label{structure}

\begin{itemize}
\item
  \emph{The bandwidth available from a satellite depends upon the number
  of transponders provided by the satellite.}
\item
  \emph{Engines used to bring the satellite to its desired orbit}
\item
  \emph{The ground control Earth stations monitor the satellite
  performance and control its functionality during various phases of its
  life-cycle.}
\end{itemize}

Communications Satellites are usually composed of the following
subsystems:

Communication Payload, normally composed of transponders, antennas, and
switching systems

Engines used to bring the satellite to its desired orbit

A station keeping tracking and stabilization subsystem used to keep the
satellite in the right orbit, with its antennas pointed in the right
direction, and its power system pointed towards the sun

Power subsystem, used to power the Satellite systems, normally composed
of solar cells, and batteries that maintain power during solar eclipse

Command and Control subsystem, which maintains communications with
ground control stations. The ground control Earth stations monitor the
satellite performance and control its functionality during various
phases of its life-cycle.

The bandwidth available from a satellite depends upon the number of
transponders provided by the satellite. Each service (TV, Voice,
Internet, radio) requires a different amount of bandwidth for
transmission. This is typically known as link budgeting and a network
simulator can be used to arrive at the exact value.

\section{Frequency Allocation for satellite
systems}\label{frequency-allocation-for-satellite-systems}

\begin{itemize}
\item
  \emph{Allocating frequencies to satellite services is a complicated
  process which requires international coordination and planning.}
\item
  \emph{Some of the services provided by satellites are:}
\item
  \emph{Fixed satellite service (FSS)}
\item
  \emph{Broadcasting satellite service (BSS)}
\end{itemize}

Allocating frequencies to satellite services is a complicated process
which requires international coordination and planning. This is carried
out under the auspices of the International Telecommunication Union
(ITU).\\
To facilitate frequency planning, the world is divided into three
regions:\\
Region 1: Europe, Africa, what was formerly the Soviet Union, and
Mongolia\\
Region 2: North and South America and Greenland\\
Region 3: Asia (excluding region 1 areas), Australia, and the southwest
Pacific

Within these regions, frequency bands are allocated to various satellite
services, although a given service may be allocated different frequency
bands in different regions. Some of the services provided by satellites
are:

Fixed satellite service (FSS)

Broadcasting satellite service (BSS)

Mobile-satellite service

Radionavigation-satellite service

Meteorological-satellite service

\section{Applications}\label{applications}

\section{Telephone}\label{telephone}

\begin{itemize}
\item
  \emph{Satellite communications are still used in many applications
  today.}
\item
  \emph{Satellite phones connect directly to a constellation of either
  geostationary or low-Earth-orbit satellites.}
\item
  \emph{In this example, almost any type of satellite can be used.}
\item
  \emph{Satellite communications also provide connection to the edges of
  Antarctica and Greenland.}
\end{itemize}

The first and historically most important application for communication
satellites was in intercontinental long distance telephony. The fixed
Public Switched Telephone Network relays telephone calls from land line
telephones to an earth station, where they are then transmitted to a
geostationary satellite. The downlink follows an analogous path.
Improvements in submarine communications cables through the use of
fiber-optics caused some decline in the use of satellites for fixed
telephony in the late 20th century.

Satellite communications are still used in many applications today.
Remote islands such as Ascension Island, Saint Helena, Diego Garcia, and
Easter Island, where no submarine cables are in service, need satellite
telephones. There are also regions of some continents and countries
where landline telecommunications are rare to nonexistent, for example
large regions of South America, Africa, Canada, China, Russia, and
Australia. Satellite communications also provide connection to the edges
of Antarctica and Greenland. Other land use for satellite phones are
rigs at sea, a back up for hospitals, military, and recreation. Ships at
sea, as well as planes, often use satellite phones.

Satellite phone systems can be accomplished by a number of means. On a
large scale, often there will be a local telephone system in an isolated
area with a link to the telephone system in a main land area. There are
also services that will patch a radio signal to a telephone system. In
this example, almost any type of satellite can be used. Satellite phones
connect directly to a constellation of either geostationary or
low-Earth-orbit satellites. Calls are then forwarded to a satellite
teleport connected to the Public Switched Telephone Network .

\section{Television}\label{television}

\begin{itemize}
\item
  \emph{Some satellites have been launched that have transponders in the
  Ka band, such as DirecTV's SPACEWAY-1 satellite, and Anik F2.}
\item
  \emph{Two satellite types are used for North American television and
  radio: Direct broadcast satellite (DBS), and Fixed Service Satellite
  (FSS).}
\item
  \emph{A direct broadcast satellite is a communications satellite that
  transmits to small DBS satellite dishes (usually 18 to 24 inches or 45
  to 60~cm in diameter).}
\end{itemize}

As television became the main market, its demand for simultaneous
delivery of relatively few signals of large bandwidth to many receivers
being a more precise match for the capabilities of geosynchronous
comsats. Two satellite types are used for North American television and
radio: Direct broadcast satellite (DBS), and Fixed Service Satellite
(FSS).

The definitions of FSS and DBS satellites outside of North America,
especially in Europe, are a bit more ambiguous. Most satellites used for
direct-to-home television in Europe have the same high power output as
DBS-class satellites in North America, but use the same linear
polarization as FSS-class satellites. Examples of these are the Astra,
Eutelsat, and Hotbird spacecraft in orbit over the European continent.
Because of this, the terms FSS and DBS are more so used throughout the
North American continent, and are uncommon in Europe.

Fixed Service Satellites use the C band, and the lower portions of the
Ku band. They are normally used for broadcast feeds to and from
television networks and local affiliate stations (such as program feeds
for network and syndicated programming, live shots, and backhauls), as
well as being used for distance learning by schools and universities,
business television (BTV), Videoconferencing, and general commercial
telecommunications. FSS satellites are also used to distribute national
cable channels to cable television headends.

Free-to-air satellite TV channels are also usually distributed on FSS
satellites in the Ku band. The Intelsat Americas 5, Galaxy 10R and AMC 3
satellites over North America provide a quite large amount of FTA
channels on their Ku band transponders.

The American Dish Network DBS service has also recently used FSS
technology as well for their programming packages requiring their
SuperDish antenna, due to Dish Network needing more capacity to carry
local television stations per the FCC's "must-carry" regulations, and
for more bandwidth to carry HDTV channels.

A direct broadcast satellite is a communications satellite that
transmits to small DBS satellite dishes (usually 18 to 24 inches or 45
to 60~cm in diameter). Direct broadcast satellites generally operate in
the upper portion of the microwave Ku band. DBS technology is used for
DTH-oriented (Direct-To-Home) satellite TV services, such as DirecTV and
DISH Network in the United States, Bell TV and Shaw Direct in Canada,
Freesat and Sky in the UK, Ireland, and New Zealand and DSTV in South
Africa.

Operating at lower frequency and lower power than DBS, FSS satellites
require a much larger dish for reception (3 to 8 feet (1 to 2.5 m) in
diameter for Ku band, and 12 feet (3.6 m) or larger for C band). They
use linear polarization for each of the transponders' RF input and
output (as opposed to circular polarization used by DBS satellites), but
this is a minor technical difference that users do not notice. FSS
satellite technology was also originally used for DTH satellite TV from
the late 1970s to the early 1990s in the United States in the form of
TVRO (TeleVision Receive Only) receivers and dishes. It was also used in
its Ku band form for the now-defunct Primestar satellite TV service.

Some satellites have been launched that have transponders in the Ka
band, such as DirecTV's SPACEWAY-1 satellite, and Anik F2. NASA and ISRO
have also launched experimental satellites carrying Ka band beacons
recently.

Some manufacturers have also introduced special antennas for mobile
reception of DBS television. Using Global Positioning System (GPS)
technology as a reference, these antennas automatically re-aim to the
satellite no matter where or how the vehicle (on which the antenna is
mounted) is situated. These mobile satellite antennas are popular with
some recreational vehicle owners. Such mobile DBS antennas are also used
by JetBlue Airways for DirecTV (supplied by LiveTV, a subsidiary of
JetBlue), which passengers can view on-board on LCD screens mounted in
the seats.

\section{Radio broadcasting}\label{radio-broadcasting}

\begin{itemize}
\item
  \emph{A satellite radio or subscription radio (SR) is a digital radio
  signal that is broadcast by a communications satellite, which covers a
  much wider geographical range than terrestrial radio signals.}
\item
  \emph{Satellite radio offers a meaningful alternative to ground-based
  radio services in some countries, notably the United States.}
\end{itemize}

Satellite radio offers audio broadcast services in some countries,
notably the United States. Mobile services allow listeners to roam a
continent, listening to the same audio programming anywhere.

A satellite radio or subscription radio (SR) is a digital radio signal
that is broadcast by a communications satellite, which covers a much
wider geographical range than terrestrial radio signals.

Satellite radio offers a meaningful alternative to ground-based radio
services in some countries, notably the United States. Mobile services,
such as SiriusXM, and Worldspace, allow listeners to roam across an
entire continent, listening to the same audio programming anywhere they
go. Other services, such as Music Choice or Muzak's satellite-delivered
content, require a fixed-location receiver and a dish antenna. In all
cases, the antenna must have a clear view to the satellites. In areas
where tall buildings, bridges, or even parking garages obscure the
signal, repeaters can be placed to make the signal available to
listeners.

Initially available for broadcast to stationary TV receivers, by 2004
popular mobile direct broadcast applications made their appearance with
the arrival of two satellite radio systems in the United States: Sirius
and XM Satellite Radio Holdings. Later they merged to become the
conglomerate SiriusXM.

Radio services are usually provided by commercial ventures and are
subscription-based. The various services are proprietary signals,
requiring specialized hardware for decoding and playback. Providers
usually carry a variety of news, weather, sports, and music channels,
with the music channels generally being commercial-free.

In areas with a relatively high population density, it is easier and
less expensive to reach the bulk of the population with terrestrial
broadcasts. Thus in the UK and some other countries, the contemporary
evolution of radio services is focused on Digital Audio Broadcasting
(DAB) services or HD Radio, rather than satellite radio.

\section{Amateur radio}\label{amateur-radio}

\begin{itemize}
\item
  \emph{Amateur radio operators have access to amateur satellites, which
  have been designed specifically to carry amateur radio traffic.}
\item
  \emph{Some satellites also provide data-forwarding services using the
  X.25 or similar protocols.}
\item
  \emph{Due to launch costs, most current amateur satellites are
  launched into fairly low Earth orbits, and are designed to deal with
  only a limited number of brief contacts at any given time.}
\end{itemize}

Amateur radio operators have access to amateur satellites, which have
been designed specifically to carry amateur radio traffic. Most such
satellites operate as spaceborne repeaters, and are generally accessed
by amateurs equipped with UHF or VHF radio equipment and highly
directional antennas such as Yagis or dish antennas. Due to launch
costs, most current amateur satellites are launched into fairly low
Earth orbits, and are designed to deal with only a limited number of
brief contacts at any given time. Some satellites also provide
data-forwarding services using the X.25 or similar protocols.

\section{Internet access}\label{internet-access}

\begin{itemize}
\item
  \emph{After the 1990s, satellite communication technology has been
  used as a means to connect to the Internet via broadband data
  connections.}
\end{itemize}

After the 1990s, satellite communication technology has been used as a
means to connect to the Internet via broadband data connections. This
can be very useful for users who are located in remote areas, and cannot
access a broadband connection, or require high availability of services.

\section{Military}\label{military}

\begin{itemize}
\item
  \emph{Examples of military systems that use communication satellites
  are the MILSTAR, the DSCS, and the FLTSATCOM of the United States,
  NATO satellites, United Kingdom satellites (for instance Skynet), and
  satellites of the former Soviet Union.}
\item
  \emph{Communications satellites are used for military communications
  applications, such as Global Command and Control Systems.}
\end{itemize}

Communications satellites are used for military communications
applications, such as Global Command and Control Systems. Examples of
military systems that use communication satellites are the MILSTAR, the
DSCS, and the FLTSATCOM of the United States, NATO satellites, United
Kingdom satellites (for instance Skynet), and satellites of the former
Soviet Union. India has launched its first Military Communication
satellite GSAT-7, its transponders operate in UHF, F, C and Ku band
bands. Typically military satellites operate in the UHF, SHF (also known
as X-band) or EHF (also known as Ka band) frequency bands.

\section{See also}\label{see-also}

\begin{itemize}
\item
  \emph{List of communication satellite companies}
\item
  \emph{Satellite space segment}
\item
  \emph{List of communications satellite firsts}
\item
  \emph{Reconnaissance satellite}
\end{itemize}

Commercialization of space

Space pollution

List of communication satellite companies

List of communications satellite firsts

NewSpace

Reconnaissance satellite

Satcom On The Move

Satellite space segment

History of telecommunication

\section{References}\label{references}

\section{External links}\label{external-links}

\begin{itemize}
\item
  \emph{Beyond The Ionosphere: Fifty Years of Satellite Communication
  (NASA SP-4217, 1997)}
\item
  \emph{http://prmt.com/glossary-of-terms/ Satellite Glossary{]}}
\item
  \emph{Satellite Industry Association}
\item
  \emph{European Satellite Operators Association}
\item
  \emph{Communications satellites short history by David J. Whalen}
\item
  \emph{The future of communication satellite business}
\end{itemize}

Satellite Industry Association

European Satellite Operators Association

http://prmt.com/glossary-of-terms/ Satellite Glossary{]}

SatMagazine

SatNews

The future of communication satellite business

Communications satellites short history by David J. Whalen

Beyond The Ionosphere: Fifty Years of Satellite Communication (NASA
SP-4217, 1997)
