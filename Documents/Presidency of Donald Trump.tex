\textbf{From Wikipedia, the free encyclopedia}

https://en.wikipedia.org/wiki/Presidency\%20of\%20Donald\%20Trump\\
Licensed under CC BY-SA 3.0:\\
https://en.wikipedia.org/wiki/Wikipedia:Text\_of\_Creative\_Commons\_Attribution-ShareAlike\_3.0\_Unported\_License

\section{Presidency of Donald Trump}\label{presidency-of-donald-trump}

\begin{itemize}
\item
  \emph{The presidency of Donald Trump began at noon EST on January 20,
  2017, when Donald Trump was inaugurated as the 45th president of the
  United States, succeeding Barack Obama.}
\item
  \emph{The Mueller team refrained from charging Trump because
  investigators abided by an Office of Legal Counsel (OLC) opinion that
  a sitting president cannot stand trial, and did not exonerate Trump on
  this issue.}
\end{itemize}

The presidency of Donald Trump began at noon EST on January 20, 2017,
when Donald Trump was inaugurated as the 45th president of the United
States, succeeding Barack Obama. A Republican, Trump was a businessman
and reality television personality from New York City at the time of his
2016 presidential election victory over Democratic nominee Hillary
Clinton. While Trump lost the popular vote by nearly 3 million votes, he
won the Electoral College vote, 304 to 227, in a presidential contest
that American intelligence agencies concluded was targeted by a Russian
interference campaign. Trump has made many false or misleading
statements during his campaign and presidency. The statements have been
documented by fact-checkers, with political scientists and historians
widely describing the phenomenon as unprecedented in modern American
politics. Trump's approval rating has been stable, hovering in the
high-30 percent to mid-40 percent range throughout his presidency.

Trump repealed environmental protections intended to address
anthropogenic climate change. He ended the Clean Power Plan, withdrew
from the Paris Agreement on climate change mitigation, and urged for
subsidies to increase fossil fuel production, calling man-made climate
change a hoax. Trump failed in his efforts to repeal the Affordable Care
Act, but signed legislation eliminating the individual mandate
provision. He enacted a partial repeal of the Dodd-Frank Act that had
previously imposed stricter constraints on banks in the aftermath of the
2008 financial crisis and withdrew from the Trans-Pacific Partnership.
Trump signed the Tax Cuts and Jobs Act of 2017, which lowered corporate
and estate taxes, and most individual income tax rates on a temporary
basis. He enacted tariffs on steel and aluminum imports and other goods,
triggering retaliatory tariffs from Canada, Mexico and the European
Union, and a trade war with China.

Trump's interventionist and unilateralist foreign policy drew the United
States closer to Saudi Arabia and Israel. He agreed to sell 110 billion
dollars of arms to Saudi Arabia, recognized Jerusalem as the capital of
Israel, withdrew the United States from the Iran Deal, and issued a
controversial executive order denying entry into the U.S. to citizens
from several Muslim-majority countries. Trump's demand for federal
funding of a U.S.--Mexico border wall resulted in the 2018--2019
government shutdown (the longest in American history) and followed with
Trump's declaration of a national emergency regarding the U.S. southern
border. He ended the Deferred Action for Childhood Arrivals (DACA)
program. Trump also appointed Neil Gorsuch and Brett Kavanaugh to the
Supreme Court. The Trump administration enforced a "zero tolerance"
policy of detaining families entering the U.S at the U.S.--Mexico border
and controversially separating parents from their children, resulting in
national and international outcry.

After Trump dismissed FBI Director James Comey in May 2017, a former FBI
director, Robert Mueller, was appointed as special counsel to take over
a prior FBI investigation into Russian interference in the 2016
elections and related matters, including coordination or links between
the Trump campaign and the Russian government. Of Trump campaign
advisors and staff, six were indicted and five pled guilty to criminal
charges. Trump has repeatedly denied collusion or obstruction of
justice, and has often criticized the investigation, calling it a
politically motivated "witch hunt".

Mueller concluded his investigation on March 22, 2019 by submitting his
report to Attorney General William Barr. On March 24, Barr sent a letter
to Congress with his conclusions and on April 18 released a partially
redacted version of the full Mueller Report. In the report's first
volume, Mueller confirmed that Russia interfered to favor Trump's
candidacy and hinder Clinton's, and concluded that the prevailing
evidence "did not establish that members of the Trump campaign conspired
or coordinated with the Russian government". In the report's second
volume, the investigators documented ten actions by the Trump presidency
that could be construed as obstruction of justice. The Mueller team
refrained from charging Trump because investigators abided by an Office
of Legal Counsel (OLC) opinion that a sitting president cannot stand
trial, and did not exonerate Trump on this issue. On March 24, Barr and
Deputy Attorney General Rod Rosenstein decided that the evidence was not
sufficient to demonstrate a criminal offense of obstruction. The Trump
administration has rebuffed requests for testimony and documents,
including by subpoena, in several ongoing investigations by committees
of both house of Congress.

\section{2016 presidential election}\label{presidential-election}

\begin{itemize}
\item
  \emph{Trump won 304 electoral votes compared to Clinton's 227, though
  Clinton won a plurality of the popular vote, receiving nearly 2.9
  million more votes than Trump.}
\item
  \emph{On November 9, 2016, Republicans Donald Trump of New York and
  Governor Mike Pence of Indiana won the 2016 election, defeating
  Democrats former Secretary of State Hillary Clinton of New York and
  Senator Tim Kaine of Virginia.}
\item
  \emph{Trump thus became the fifth person to win the presidency while
  losing the popular vote.}
\end{itemize}

On November 9, 2016, Republicans Donald Trump of New York and Governor
Mike Pence of Indiana won the 2016 election, defeating Democrats former
Secretary of State Hillary Clinton of New York and Senator Tim Kaine of
Virginia. Trump won 304 electoral votes compared to Clinton's 227,
though Clinton won a plurality of the popular vote, receiving nearly 2.9
million more votes than Trump. Trump thus became the fifth person to win
the presidency while losing the popular vote. In the concurrent
congressional elections, Republicans maintained majorities in both the
House of Representatives and the Senate.

Trump made false claims that massive amounts of voter fraud -- up to 5
million illegal votes -- in Clinton's favor occurred during the
election, and he called for a major investigation after taking office.
Numerous studies have found no evidence of widespread voter fraud.

\includegraphics[width=5.50000in,height=3.66341in]{media/image1.jpg}\\
\emph{Outgoing President Barack Obama and President-elect Donald Trump
in the Oval Office on November 10, 2016}

\section{Transition period and
inauguration}\label{transition-period-and-inauguration}

\begin{itemize}
\item
  \emph{Prior to the election, Trump named Chris Christie as the leader
  of his transition team.}
\item
  \emph{Trump was inaugurated on January 20, 2017.}
\item
  \emph{Accompanied by his wife, Melania Trump, he was sworn in by Chief
  Justice John Roberts.}
\end{itemize}

Prior to the election, Trump named Chris Christie as the leader of his
transition team. After the election, Vice President-elect Mike Pence
replaced Christie as chairman of the transition team, while Christie
became a vice-chairman alongside Senator Jeff Sessions of Alabama,
retired Army Lt. Gen. Michael T. Flynn, former New York City Mayor Rudy
Giuliani, former presidential candidate Ben Carson, and former House
Speaker Newt Gingrich.

Trump was inaugurated on January 20, 2017. Accompanied by his wife,
Melania Trump, he was sworn in by Chief Justice John Roberts. In his
seventeen-minute inaugural address, Trump made a broad condemnation of
contemporary America, pledging to end "American carnage" and saying that
America's "wealth, strength and confidence has dissipated". He repeated
the "America First" slogan that he had used in the campaign and promised
that "{[}e{]}very decision on trade, on taxes, on immigration, on
foreign affairs, will be made to benefit American workers and American
factories". At the age of seventy, Trump surpassed Ronald Reagan and
became the oldest person to assume the presidency, and the first without
any prior government or military experience. The largest single-day
protest in the history of the United States was against Trump's
Presidency the day after his inauguration.

\section{Personnel}\label{personnel}

\begin{itemize}
\item
  \emph{The Trump administration has been characterized by record
  turnover, particularly among White House staff.}
\item
  \emph{On September 5, 2018, The New York Times published an article
  entitled "I Am Part of the Resistance Inside the Trump
  Administration", written by an anonymous senior official in the Trump
  administration.}
\end{itemize}

The Trump administration has been characterized by record turnover,
particularly among White House staff. By the end of his first year in
office, 34 percent of Trump's original staff had resigned, been fired,
or been reassigned. As of early March~2018{[}update{]}, 43 percent of
senior White House positions had turned over.

On September 5, 2018, The New York Times published an article entitled
"I Am Part of the Resistance Inside the Trump Administration", written
by an anonymous senior official in the Trump administration. The author
asserted that "many of the senior officials in his own administration
are working diligently from within to frustrate parts of his agenda and
his worst inclinations."

\section{Cabinet}\label{cabinet}

\begin{itemize}
\item
  \emph{Since taking office, Trump has made two unsuccessful cabinet
  nominations.}
\item
  \emph{At the time of Pruitt's resignation, he is the fifth member of
  Trump's cabinet to resign or be fired since Trump took office.}
\item
  \emph{In February 2017, Trump formally announced his cabinet
  structure, elevating the Director of National Intelligence and
  Director of the CIA to cabinet level.}
\end{itemize}

Days after the presidential election, Trump selected RNC Chairman Reince
Priebus as his Chief of Staff. Priebus and Senior Counselor Steve Bannon
were named as "equal partners" within the White House power structure,
although Bannon was not an official member of the Cabinet. On November
18, Trump announced that he had chosen Alabama Senator Jeff Sessions for
the position of Attorney General.

In February 2017, Trump formally announced his cabinet structure,
elevating the Director of National Intelligence and Director of the CIA
to cabinet level. The Chair of the Council of Economic Advisers, which
had been added to the cabinet by Obama in 2009, was removed from the
cabinet. Trump's cabinet consists of 24 members, more than Barack Obama
at 23 or George W. Bush at 21. His final initial Cabinet-level nominee,
U.S. Trade Representative Robert Lighthizer, was confirmed on May 12,
2017.

In July 2017, John F. Kelly, who had served as Secretary of Homeland
Security, replaced Priebus as Chief of Staff. Bannon was fired in August
2017, leaving Kelly as one of the most powerful individuals in the White
House. In September 2017, Tom Price resigned as Secretary of Health and
Human Services amid criticism over his use of private charter jets for
his personal travel. Don J. Wright replaced Price as acting Secretary of
Health and Human Services. Kirstjen Nielsen succeeded Kelly as Secretary
in December 2017. Secretary of State Rex Tillerson was fired via a tweet
in March 2018; Trump appointed Mike Pompeo to replace Tillerson and Gina
Haspel to succeed Pompeo as the Director of the CIA. In the wake of a
series of controversies, Scott Pruitt resigned as Administrator of the
Environmental Protection Agency in July 2018. Deputy Administrator
Andrew Wheeler is slated to serve as acting administrator beginning July
9, 2018. At the time of Pruitt's resignation, he is the fifth member of
Trump's cabinet to resign or be fired since Trump took office.

Since taking office, Trump has made two unsuccessful cabinet
nominations. Andrew Puzder was nominated for the position of Secretary
of Labor in 2017, while Ronny Jackson, who had previously served as the
President's physician, was nominated as Secretary of Veterans Affairs in
2018. Each withdrew their name from consideration after facing
opposition in the Senate.

\section{Notable departures}\label{notable-departures}

\begin{itemize}
\item
  \emph{By March 2018, White House staff turnover was estimated at
  43\%.}
\item
  \emph{In the first 13 months of the administration of Donald Trump,
  the White House staff had a higher turnover than the previous four
  presidents had in the first two years of their respective
  administrations.}
\end{itemize}

In the first 13 months of the administration of Donald Trump, the White
House staff had a higher turnover than the previous four presidents had
in the first two years of their respective administrations. By March
2018, White House staff turnover was estimated at 43\%.

\section{Firing of Michael Flynn}\label{firing-of-michael-flynn}

\begin{itemize}
\item
  \emph{Yates was fired by Donald Trump on January 30 because "she
  defiantly refused to defend his executive order closing the nation's
  borders to refugees and people from predominantly Muslim countries".}
\item
  \emph{On February 13, 2017, Trump fired Michael Flynn from the post of
  National Security Adviser on grounds that he had lied to Vice
  President Pence about his communications with the Russian Ambassador
  to the United States, Sergey Kislyak.}
\end{itemize}

On February 13, 2017, Trump fired Michael Flynn from the post of
National Security Adviser on grounds that he had lied to Vice President
Pence about his communications with the Russian Ambassador to the United
States, Sergey Kislyak. Flynn was fired amidst the ongoing controversy
concerning Russian interference in the 2016 United States elections and
accusations that Trump's electoral team colluded with Russian agents. In
May 2017, Sally Yates testified before the Senate Judiciary's
Subcommittee on Crime and Terrorism that she had told White House
Counsel Don McGahn in late January 2017 that Flynn had misled Vice
President Mike Pence and other administration officials and warned that
Flynn was potentially compromised by Russia. Flynn remained in his post
for another two weeks and was fired after The Washington Post broke the
story. Yates was fired by Donald Trump on January 30 because "she
defiantly refused to defend his executive order closing the nation's
borders to refugees and people from predominantly Muslim countries".

\section{Firing of James Comey}\label{firing-of-james-comey}

\begin{itemize}
\item
  \emph{Comey later told the Senate Intelligence Committee that he
  created written records immediately after his conversations with Trump
  because he "was honestly concerned that he {[}Trump{]} might lie about
  the nature of our meeting".}
\item
  \emph{On May 31, Trump wrote on Twitter: "I never fired James Comey
  because of Russia!"}
\end{itemize}

On May 9, 2017, Trump dismissed FBI director James Comey, stating that
he had accepted the recommendations of the U.S. Attorney General Jeff
Sessions and Deputy Attorney General Rod Rosenstein to dismiss Comey.
Sessions' recommendation was based on Rosenstein's, while Rosenstein
wrote that Comey should be dismissed for his handling of the conclusion
of the FBI investigation into the Hillary Clinton email controversy.

On May 10, Trump met Russian Foreign Minister Sergey Lavrov and Russian
Ambassador Sergey Kislyak. Based on White House notes of the meeting,
Trump told the Russians: "I just fired the head of the FBI. He was
crazy, a real nut job ... I faced great pressure because of Russia.
That's taken off." On May 11, Trump said in a videoed interview: "...
regardless of recommendation, I was going to fire Comey ... in fact,
when I decided to just do it, I said to myself, I said, you know, this
Russia thing with Trump and Russia is a made-up story." On May 18,
Rosenstein told members of the U.S. Senate that he recommended Comey's
dismissal while knowing that Trump had already decided to fire Comey. On
May 31, Trump wrote on Twitter: "I never fired James Comey because of
Russia!"

In the aftermath of Comey's firing, the events were compared with those
of the "Saturday Night Massacre" during Richard Nixon's administration,
and there was debate over whether Trump had provoked a constitutional
crisis as he had dismissed the man leading an investigation into Trump's
associates.

Comey had previously prepared seven detailed memos, four of which
contained classified information, documenting most of his meetings and
telephone conversations with President Trump. He provided some of the
memos to his friend Daniel Richman, who then released the substance of
the memos to the press. Comey later told the Senate Intelligence
Committee that he created written records immediately after his
conversations with Trump because he "was honestly concerned that he
{[}Trump{]} might lie about the nature of our meeting". In his memo
about a February 14, 2017, Oval Office meeting, Comey says Trump
attempted to persuade him to abort the investigation into General Flynn.

\section{Resignation of Jim Mattis}\label{resignation-of-jim-mattis}

\begin{itemize}
\item
  \emph{Secretary of Defense Jim Mattis informed Trump of his
  resignation following Trump's abrupt December 19, 2018 announcement
  that the remaining 2,000 American troops in Syria would be withdrawn,
  against the recommendations of his military and civilian advisors.}
\item
  \emph{In his resignation letter, Mattis appeared to criticize Trump's
  worldview, praising NATO, which Trump has often derided, as well as
  the Defeat-ISIS coalition that Trump had just decided to abandon.}
\end{itemize}

Secretary of Defense Jim Mattis informed Trump of his resignation
following Trump's abrupt December 19, 2018 announcement that the
remaining 2,000 American troops in Syria would be withdrawn, against the
recommendations of his military and civilian advisors. In his
resignation letter, Mattis appeared to criticize Trump's worldview,
praising NATO, which Trump has often derided, as well as the Defeat-ISIS
coalition that Trump had just decided to abandon. Mattis' resignation
became effective on February 28, 2019

\section{Judicial nominees}\label{judicial-nominees}

\begin{itemize}
\item
  \emph{On July 9, 2018, Trump nominated Brett Kavanaugh, a judge on the
  D.C.}
\item
  \emph{By November 2018, Trump had appointed 29 judges to the United
  States courts of appeals, more than any other president in the first
  two years of a presidential term.}
\item
  \emph{Compared to President Obama, Trump has nominated fewer non-white
  and female judges.}
\end{itemize}

On January 31, 2017, Trump nominated federal appellate judge Neil
Gorsuch to the Supreme Court to fill the vacancy which arose after the
February 2016 death of Antonin Scalia and which had not been filled
under the then-president Obama because of Republican obstruction.
Gorsuch's appointment was confirmed on April 7, 2017, in a 54--45 vote.
Prior to this nomination, the support of three-fifths of the Senate had
effectively been required for the confirmation of Supreme Court nominees
due to the Senate filibuster. However, the Senate's Republican majority
changed the rules for the filibuster via the so-called "nuclear option,"
and the confirmation of Supreme Court justices now requires only a
simple majority vote.

In June 2018, Associate Justice Anthony Kennedy, widely considered to be
the key swing vote on the Supreme Court, announced his retirement. On
July 9, 2018, Trump nominated Brett Kavanaugh, a judge on the D.C.
Circuit Court of Appeals, to fill the vacancy caused by Associate
Justice Anthony Kennedy impending retirement. During the confirmation
process, Kavanaugh was accused of sexually assaulting Christine Blasey
Ford, currently a professor in clinical psychology at Palo Alto
University, while they were both in high school. On October 6, the
Senate voted 50--48 to confirm Kavanaugh's nomination to the Supreme
Court.

By November 2018, Trump had appointed 29 judges to the United States
courts of appeals, more than any other president in the first two years
of a presidential term. Compared to President Obama, Trump has nominated
fewer non-white and female judges. Bloomberg News noted that Trump's
judicial nominees tended to be young and favored by the conservative
Federalist Society.

\section{First year}\label{first-year}

\begin{itemize}
\item
  \emph{Gorsuch was confirmed by the Senate on April 7, 2017.}
\item
  \emph{None of Trump's bills are considered to be "major bills"---based
  on a "longstanding political-science standard for 'major bills'."}
\item
  \emph{Trump signed 24 executive orders in his first 100 days, the most
  executive orders of any president since World War II.}
\end{itemize}

One of Trump's major first year accomplishments, made as part of a
"100-day pledge", was the confirmation of Neil Gorsuch as an Associate
Justice of the Supreme Court of the United States. Structurally,
President Trump had the advantage of a Republican Party majority in the
U.S. House of Representatives and the Senate, but was unable to fulfill
his major pledges in his first 100 days and had an approval rating of
between 40 and 42 percent, "the lowest for any first-term president at
this point in his tenure".{[}citation needed{]} Although he tried to
make progress on one of his key economic policies---the dismantling of
the Dodd--Frank Wall Street Reform and Consumer Protection Act---his
failure to repeal the Patient Protection and Affordable Care Act (ACA)
in the first 100 days was a major setback. He reversed his position on a
number of issues including labeling China as a currency manipulator,
NATO, launching the 2017 Shayrat missile strike without congressional
approval, the North American Free Trade Agreement (NAFTA), renomination
of Janet Yellen as Chair of the Federal Reserve, and the nomination of
Export-Import Bank directors. Supporters claimed that as the first
person in history to have been elected president who has never held any
military, political, or government office of any type, he therefore
faced a steep learning curve.

Trump signed 24 executive orders in his first 100 days, the most
executive orders of any president since World War II. He also signed 22
presidential memoranda, 20 presidential proclamations, and 28 bills.
About a dozen of those bills roll-back regulations finalized during the
last months of his immediate predecessor Barack Obama's presidency using
the Congressional Review Act. Most of the other bills are "small-scale
measures that appoint personnel, name federal facilities or modify
existing programs." None of Trump's bills are considered to be "major
bills"---based on a "longstanding political-science standard for 'major
bills'." Presidential historian Michael Beschloss said that "based on a
legislative standard"---which is what the first 100 days has been judged
on since the tenure of President Franklin D. Roosevelt, who enacted 76
laws in 100 days including nine that were "major"---"Trump is really
pretty low down on the list."

On January 31, Trump nominated U.S. Appeals Court judge Neil Gorsuch to
fill the seat on the Supreme Court previous held by Justice Antonin
Scalia until his death in 2016. Gorsuch was confirmed by the Senate on
April 7, 2017.

In December 2017, Trump signed the Tax Cuts and Jobs Act of 2017, which
cut the corporate tax rate to 21 percent, lowered personal tax brackets,
increased child tax credit, doubled the estate tax threshold to
\$11.2~million, and limited the state and local tax deduction to
\$10,000. The reduction in individual tax rates ends in 2025. While
people would generally get a tax cut, those with higher incomes would
see the most benefit. Households in the lower or middle class would also
see a small tax increase after the tax cuts expire. The bill is
estimated to increase deficits by \$1.5~trillion over 10 years.

\section{Leadership style and
philosophy}\label{leadership-style-and-philosophy}

\begin{itemize}
\item
  \emph{In 2019, Axios published Trump's schedule from November 7, 2018
  to February 1, 2019, and calculated that around 60\% of the time
  between 8am to 5pm was "executive time".}
\item
  \emph{Trump reportedly expresses anger if intelligence analysis
  contradicts his beliefs or public statements, with two briefers
  stating they have been instructed by superiors to not provide Trump
  with information that contradicts his public statements.}
\end{itemize}

Trump reportedly eschews reading detailed briefing documents, including
the President's Daily Brief, in favor of receiving oral briefings.
Intelligence briefers reportedly repeat the President's name and title
in order to keep his attention. He is also known to acquire information
by watching up to eight hours of television each day, most notably Fox
News programs such as Fox \& Friends and Hannity, whose broadcast
talking points Trump sometimes repeats in public statements,
particularly in early morning tweets. Trump reportedly expresses anger
if intelligence analysis contradicts his beliefs or public statements,
with two briefers stating they have been instructed by superiors to not
provide Trump with information that contradicts his public statements.

Trump has reportedly fostered chaos as a management technique, resulting
in low morale and policy confusion among his staff, although he has
maintained his White House runs like a "well-oiled machine." Jeffrey
Pfeffer, professor of organizational behavior at Stanford, noted that
Trump possesses several management qualities that are prevalent among
many leaders, including narcissism and dishonesty, but added, "With a
modicum of management skill he could have gotten his wall, and he would
probably be on the path to re-election. But he has very few
accomplishments to his credit." Trump proved unable to effectively
compromise during the 115th United States Congress, which led to
significant governmental gridlock and few notable legislative
accomplishments despite Republican control of both houses the House and
the Senate. Presidential historian Doris Kearns Goodwin found Trump
lacks several traits of an effective leader, including "humility,
acknowledging errors, shouldering blame and learning from mistakes,
empathy, resilience, collaboration, connecting with people and
controlling unproductive emotions." The New York Times reported that
"before taking office, Mr. Trump told top aides to think of each
presidential day as an episode in a television show in which he
vanquishes rivals."

In January 2018, Axios reported that Trump's working hours were
typically around 11am to 6pm (a later start and an earlier end compared
to the beginning of his presidency) and that he was holding fewer
meetings during his working hours, in order to accommodate Trump's
desire for more unstructured free time (labelled as "executive time").
Later that year, Politico reported Trump's schedule for October 22--26
that he never started work earlier than 11am, had large amounts of
"executive time" and only a total of two hours of policy briefings in
five days. In 2019, Axios published Trump's schedule from November 7,
2018 to February 1, 2019, and calculated that around 60\% of the time
between 8am to 5pm was "executive time".

\section{False and misleading
statements}\label{false-and-misleading-statements}

\begin{itemize}
\item
  \emph{As of May 2019, Trump had made more than 10,000 false or
  misleading claims.}
\item
  \emph{During the first year of Trump's presidency, The Washington
  Post's fact-checker wrote, "President Trump is the most
  fact-challenged politician that The Fact Checker has ever
  encountered... the pace and volume of the president's misstatements
  means that we cannot possibly keep up."}
\item
  \emph{Trump's presidency started out with a series of falsehoods
  initiated by Trump himself.}
\item
  \emph{The Washington Post found that Trump averaged 15 false
  statements per day during 2018.}
\end{itemize}

As president, Trump has made so many false statements in public
speeches, remarks, and in tweets, that media commentators and
fact-checkers have described the rate of his falsehoods as unprecedented
for an American president or even unprecedented in politics. This trait
of his was similarly observed when he was a presidential candidate. His
falsehoods have become a distinctive part of his political identity, and
they have also described as part of a gaslighting tactic. His White
House has dismissed the idea of objective truth, and his campaign and
presidency have been described as being "post-truth" and
hyper-Orwellian, Trump's rhetorical signature includes disregarding data
from federal institutions which are incompatible to his arguments,
quoting hearsay, anecdotal evidence and questionable claims in partisan
media, denying reality (including his own statements), and distracting
when falsehoods are exposed.

During the first year of Trump's presidency, The Washington Post's
fact-checker wrote, "President Trump is the most fact-challenged
politician that The Fact Checker has ever encountered... the pace and
volume of the president's misstatements means that we cannot possibly
keep up." By August 2018, the pace of the false statements increased
substantially. In June and July alone 968 new incidences had been noted,
and a total of 4,229 "false or misleading" statements had by then been
recorded in his tenure. Immigration issues led the subject list at that
point, with 538 recorded mendacities.

Senior administration officials have also regularly given false,
misleading or tortured statements~to the media. By May 2017, Politico
reported that the repeated untruths by senior officials made it
difficult for the media to take official statements seriously.

Trump's presidency started out with a series of falsehoods initiated by
Trump himself. The day after his inauguration, he falsely accused the
media of lying about the size of the inauguration crowd. Then he
proceeded to exaggerate the size, and Sean Spicer backed up his claims.
When Spicer was accused of intentionally misstating the figures,
Kellyanne Conway, in an interview with NBC's Chuck Todd, defended Spicer
by stating that he merely presented "alternative facts". Other notable
claims by Trump which fact checkers rated false include the claim that
his electoral college victory was a "landslide" and that Hillary Clinton
received 3-5 million illegal votes.

In the seven weeks leading up to the midterm elections---it had risen to
an average of 30 per day from 4.9 during his first 100 days in office.
The Washington Post found that Trump averaged 15 false statements per
day during 2018. As of May 2019, Trump had made more than 10,000 false
or misleading claims.

\section{Rule of law}\label{rule-of-law}

\begin{itemize}
\item
  \emph{In May 2018 Trump demanded the DOJ to investigate "whether or
  not the FBI/DOJ infiltrated or surveilled the Trump Campaign for
  Political Purposes," which the DOJ referred to its inspector general.}
\item
  \emph{The Times reported in November 2018 that more than a dozen
  members of the conservative-libertarian Federalist Society --- which
  had been instrumental in selecting Trump's appointments to federal
  courts --- "are urging their fellow conservatives to speak up about
  what they say are the Trump administration's betrayals of bedrock
  legal norms."}
\end{itemize}

Shortly before Trump secured the 2016 Republican nomination, The New
York Times reported that "legal experts across the political spectrum
say" Trump's rhetoric reflected "a constitutional worldview that shows
contempt for the First Amendment, the separation of powers and the rule
of law," adding "many conservative and libertarian legal scholars warn
that electing Mr. Trump is a recipe for a constitutional crisis." A
group named "Originalists Against Trump" declared in October 2016,
"Trump's long record of statements and conduct have shown him
indifferent or hostile to the Constitution's basic features." As the
Trump presidency unfolded, numerous prominent conservative Republicans
expressed similar concerns that Trump's perceived disregard for the rule
of law betrayed conservative principles. The Times reported in November
2018 that more than a dozen members of the conservative-libertarian
Federalist Society --- which had been instrumental in selecting Trump's
appointments to federal courts --- "are urging their fellow
conservatives to speak up about what they say are the Trump
administration's betrayals of bedrock legal norms."

During the first two years of his presidency, Trump repeatedly sought to
influence the Justice Department to investigate those he saw as his
political adversaries --- including Hillary Clinton, the Democratic
National Committee, James Comey and the FBI --- regarding a variety of
persistent allegations, at least some of which had already been
investigated or debunked. In spring 2018, Trump told White House counsel
Don McGahn that he wanted to order the DOJ to prosecute Clinton and
Comey, but McGahn advised Trump that such action would constitute abuse
of power and invite possible impeachment. In May 2018 Trump demanded the
DOJ to investigate "whether or not the FBI/DOJ infiltrated or surveilled
the Trump Campaign for Political Purposes," which the DOJ referred to
its inspector general. Although it is not unlawful for a president to
exert influence on the DOJ to open an investigation, presidents have
assiduously avoided doing so to prevent perceptions of political
interference. Some of Trump's congressional allies asked Attorney
General Jeff Sessions to appoint a special counsel to investigate the
FBI and an alleged Uranium One controversy involving Clinton; Sessions
instead appointed in May 2018 federal prosecutor John Huber to examine
the matters and make a recommendation as to whether a special counsel
was justified. Sessions otherwise resisted demands by Trump and his
allies for investigations, causing Trump to repeatedly express
frustration, saying at one point, "I don't have an attorney general."
Matthew Whitaker, a Trump loyalist whom the President appointed to
succeed Sessions as Acting Attorney General in November 2018, had in
2017 reportedly provided private advice to Trump on how the White House
might pressure the Justice Department to investigate the President's
adversaries, including appointing a special counsel to investigate the
FBI and Hillary Clinton. In 2014, Whitaker criticized Marbury v.
Madison, the 1803 landmark Supreme Court decision that established the
bedrock principle of judicial review that empowered courts to strike
down statutes and government actions that contravene the Constitution.

In an extraordinary rebuke of a sitting president, in November 2018
Supreme Court Chief Justice John Roberts responded to Trump's
characterization of a judge who had ruled against his policies as an
"Obama judge," adding "That's not law." Roberts wrote, "We do not have
Obama judges or Trump judges, Bush judges or Clinton judges. What we
have is an extraordinary group of dedicated judges doing their level
best to do equal right to those appearing before them. That independent
judiciary is something we should all be thankful for."

\includegraphics[width=5.50000in,height=3.66667in]{media/image2.jpg}\\
\emph{Trump talking to the press, March 21, 2017, before signing S.422,
the National Aeronautics and Space Administration Transition
Authorization Act, in the Oval Office}

\includegraphics[width=5.50000in,height=3.66667in]{media/image3.jpg}\\
\emph{Trump speaks to reporters on the White House South Lawn in June
2019}

\section{Relationship with the press}\label{relationship-with-the-press}

\begin{itemize}
\item
  \emph{In May 2018, Trump tweeted that "91\% of the Network News about
  me is negative (Fake)."}
\item
  \emph{In August 2018 Trump tweeted that "Holt got caught fudging my
  tape on Russia," followed by his attorney Jay Sekulow asserting in
  September 2018 that NBC had edited the Trump interview.}
\item
  \emph{In May 2018 Trump denied firing Comey because of the Russia
  investigation.}
\item
  \emph{Trump's first press conference was also the last (as of January
  2019).}
\end{itemize}

Early into his presidency, the administration developed a highly
contentious relationship with the media, repeatedly describing it as the
"fake news media" and "the enemy of the people". Through August 2018, at
least three journalists received threatening phone calls from men
calling them the enemy of the people, with one suspect being arrested by
the FBI for making death threats. Trump both privately and publicly
mused about taking away critical reporters' White House press
credentials (despite, during his campaign, promising not to do so once
he became president). At the same time, the Trump White House gave
temporary press passes to far-right pro-Trump fringe outlets, such as
InfoWars and The Gateway Pundit, which are known for publishing hoaxes
and conspiracy theories.

On his first day in office, Trump falsely accused journalists of
understating the size of the crowd at his inauguration, and called the
media "among the most dishonest human beings on earth". Trump's claims
were notably defended by Press Secretary Sean Spicer, who claimed that
the inauguration crowd had been the biggest in history, a claim
disproven by photographs. Trump's senior adviser Kellyanne Conway then
defended Spicer when asked about the falsehood, saying that it was an
"alternative fact", not a falsehood.

Less than a month into his presidency, Trump held a press conference
claiming that the media was not speaking for the people, but for special
interests. He claimed that they were dishonest, out of control and doing
a disservice to the American people. On February 17, 2017, Trump
tweeted, "The FAKE NEWS media (failing @nytimes, @NBCNews, @ABC, @CBS,
@CNN) is not my enemy, it is the enemy of the American People!" Trump's
first press conference was also the last (as of January 2019). For
comparison, Barack Obama had held 11 solo press conferences by the end
of his first year, George W. Bush held five, and Bill Clinton held 12.
Later in the month, the administration blocked reporters from The New
York Times, BuzzFeed News, CNN, Los Angeles Times and Politico from
attending an off-camera briefing with Press Secretary Sean Spicer.
Reporters from Time magazine and The Associated Press chose not to
attend the briefing in protest at the White House's actions. The New
York Times described the move as "a highly unusual breach of relations
between the White House and its press corps", and the White House
Correspondents' Association issued a statement of protest.

In March 2017, all major U.S. television networks declined to air a paid
campaign ad placed by the 2020 Trump campaign which included a graphic
claiming that mainstream media is "fake news". In a statement, CNN said
that they "requested that the advertiser remove the false graphic that
the mainstream media is 'fake news'". Lara Trump, daughter-in-law to
Trump and adviser for his campaign, called the rejection a "chilling
precedent against free speech rights", and "an unprecedented act of
censorship in America that should concern every freedom-loving citizen".

The relationship between Trump, the media, and fake news has been
studied. One study found that between October 7 and November 14, 2016,
while 1 in 4 Americans visited a fake news website, "Trump supporters
visited the most fake news websites, which were overwhelmingly
pro-Trump" and "almost 6 in 10 visits to fake news websites came from
the 10\% of people with the most conservative online information diets".
Brendan Nyhan, one of the authors of the study by researchers from
Princeton University, Dartmouth College, and the University of Exeter,
stated in an interview: "People got vastly more misinformation from
Donald Trump than they did from fake news websites".

In May 2018, Trump tweeted that "91\% of the Network News about me is
negative (Fake)." The Washington Post described this Trump making it
"explicit" that negative coverage on him has to be fake. Also that
month, journalist Lesley Stahl recounted that after Trump won the 2016
presidential election, he had told her that he kept attacking the media
to "demean" and "discredit", "so when you write negative stories about
me no one will believe you". Trump also attacked The New York Times on
their coverage of a White House briefing on the 2018 North Korea--United
States summit. Trump claimed that the anonymous "senior White House
official" that the newspaper quoted "doesn't exist", instructing: "Use
real people, not phony sources". Following Trump's claim, journalists
provided audio evidence of the official being introduced as Matt
Pottinger of the National Security Council, with White House deputy
press secretary Raj Shah insisting that Pottinger's anonymity was
required. The White House's invitation for the briefing to journalists
also surfaced.

The Boston Globe called for a nationwide refutation of Trump's "dirty
war" against the media, with the hashtag \#EnemyOfNone. Over 300 news
outlets joined the campaign. The New York Times called Trump's attacks
"dangerous to the lifeblood of democracy" and published excerpts from
dozens of further publications. The New York Post wrote, "It may be
frustrating to argue that just because we print inconvenient truths
doesn't mean that we're fake news, but being a journalist isn't a
popularity contest. All we can do is to keep reporting." The
Philadelphia Inquirer wrote, "If the press is not free from reprisal,
punishment or suspicion for unpopular views or information, neither is
the country. Neither are its people"

On August 16, 2018 the Senate unanimously passed a resolution affirming
that "the press is not the enemy of the people," marking the second time
the Senate had unanimously rebuked Trump within a month.

In October 2018, Trump praised US representative Greg Gianforte for
assaulting political reporter Ben Jacobs in 2017. According to analysts,
the incident marked the first time the President has "openly and
directly praised a violent act against a journalist on American soil".
Later that month, as CNN and prominent Democrats were targeted with
bombs, Trump initially condemned the bomb attempts but shortly
thereafter blamed the "Mainstream Media that I refer to as Fake News"
for causing "a very big part of the Anger we see today in our society."

In a May 2017 interview with NBC News anchorman Lester Holt, Trump
stated he was thinking of "this Russia thing" when he decided to fire
FBI Director James Comey. Trump's statement raised concerns of potential
obstruction of justice. In May 2018 Trump denied firing Comey because of
the Russia investigation. In August 2018 Trump tweeted that "Holt got
caught fudging my tape on Russia," followed by his attorney Jay Sekulow
asserting in September 2018 that NBC had edited the Trump interview.
Neither Trump nor Sekulow produced evidence that the tape had been
modified.

Following a contentious Trump press conference on November 7, 2018 in
which CNN reporter Jim Acosta was criticized by Trump and press
secretary Sarah Sanders for perceived disruptive behavior, including an
alleged assault on a White House intern as she attempted to take a
microphone from Acosta, the White House revoked Acosta's press pass. CNN
sued Trump and several of his aides six days later, asserting that
Acosta's due process and First Amendment rights had been violated. The
case was heard by Timothy Kelly, a Trump appointee to the District Court
for the District of Columbia who ruled on November 16 that Acosta's due
process rights had been violated and his press pass must be restored.
Kelly made no ruling on the First Amendment issue. Sanders faced
criticism for tweeting a video clip that originated from Paul Joseph
Watson of Infowars purporting to prove the alleged assault, which
differed from original video of the incident, resulting in a misleading
impression that Acosta had aggressively thrust his hand at the intern.
Later asked if the video had been altered, Trump aide Kellyanne Conway
replied, "That's not altered, that's sped up," likening it to a
television replay of a sporting event.

\section{Use of Twitter}\label{use-of-twitter}

\begin{itemize}
\item
  \emph{The Trump administration has described Trump's tweets as
  "official statements by the President of the United States".}
\item
  \emph{A federal judge ruled in May 2018 that Trump's blocking of other
  Twitter users due to opposing political views violated the First
  Amendment to the United States Constitution and that he must unblock
  them; however, according to a plaintiff, Trump has yet to comply with
  the unblocking order.}
\item
  \emph{Trump continued the use of Twitter from the presidential
  campaign.}
\item
  \emph{Tillerson was eventually fired via a tweet by Trump.}
\end{itemize}

Trump continued the use of Twitter from the presidential campaign. Trump
has continued to personally tweet from @realDonaldTrump, his personal
account, while his staff tweet on his behalf using the official @POTUS
account. His use of Twitter has been unconventional for a president,
initiating controversy and becoming news in their own right. The Trump
administration has described Trump's tweets as "official statements by
the President of the United States". A federal judge ruled in May 2018
that Trump's blocking of other Twitter users due to opposing political
views violated the First Amendment to the United States Constitution and
that he must unblock them; however, according to a plaintiff, Trump has
yet to comply with the unblocking order. The administration has appealed
the court's ruling.

His tweets have been reported as ill-considered, impulsive, vengeful,
and bullying, often being made late at night or in the early hours of
the morning. His tweets about a Muslim ban were successfully turned
against his administration to halt two versions of travel restrictions
from some Muslim-majority countries. He has used Twitter to threaten and
intimidate his political opponents and potential political allies needed
to pass bills.{[}citation needed{]}

Many tweets appear to be based on stories that Trump has seen in the
media, including far-right news websites such as Breitbart, and
television shows such as Fox \& Friends. One notable example is the
Trump Tower wiretapping allegations which appeared to originate in an
unsubstantiated claim by Andrew Napolitano on Fox News. Despite a lack
of evidence for the claims, Trump continued to push the claim in media
and through Twitter.

Trump has used Twitter to attack federal judges who have ruled against
him in court cases and to criticize officials within his own
administration, including then-Secretary of State Rex Tillerson,
then-National Security Advisor H. R. McMaster, Deputy Attorney General
Rod Rosenstein, and at various times Attorney General Jeff Sessions.
Tillerson was eventually fired via a tweet by Trump. Trump has also
tweeted that his Justice Department is part of the American "deep
state"; that "there was tremendous leaking, lying and corruption at the
highest levels of the FBI, Justice \& State" Departments; and that the
special counsel investigation is a "WITCH HUNT!" In August 2018, Trump
used Twitter to write that Attorney General Jeff Sessions "should stop"
the special counsel investigation immediately; he also referred to it as
"rigged" and its investigators as biased.

\section{Domestic policy}\label{domestic-policy}

\section{Abortion and fetal tissue
research}\label{abortion-and-fetal-tissue-research}

\begin{itemize}
\item
  \emph{At an April 2019 rally he falsely contended, "The baby is born.}
\item
  \emph{Trump, in his first few days in office, signed an executive
  order reinstating the Mexico City policy that requires all foreign
  non-governmental organizations that receive federal funding to refrain
  from performing or promoting abortion as a method of family planning
  in other countries.}
\item
  \emph{Criticizing late-term abortion, in his January 2019 State of the
  Union Address Trump falsely asserted that a New York law would allow
  "a baby to be ripped from the mother's womb moments before birth."}
\end{itemize}

Trump, in his first few days in office, signed an executive order
reinstating the Mexico City policy that requires all foreign
non-governmental organizations that receive federal funding to refrain
from performing or promoting abortion as a method of family planning in
other countries. In 2018, the United States was the only country to
oppose a nonbinding draft resolution in the United Nations designed to
combat violence against women; the administration raised concern about a
reference to "sexual and reproductive health", which it believed could
be interpreted as support for abortion rights. The administration also
argued that resolution might conflate "physical violence against women
with sexual harassment." In 2018, the administration prohibited
scientists at the National Institutes of Health (NIH) from acquiring new
fetal tissue for research, and a year later stopped all medical research
by government scientists that used fetal tissue.

The administration geared HHS funding towards abstinence education
programs for teens rather than the comprehensive sexual education
programs that the Obama administration funded.

Criticizing late-term abortion, in his January 2019 State of the Union
Address Trump falsely asserted that a New York law would allow "a baby
to be ripped from the mother's womb moments before birth." At an April
2019 rally he falsely contended, "The baby is born. The mother meets
with the doctor. They take care of the baby. They wrap the baby
beautifully. And then the doctor and the mother determine whether or not
they will execute the baby.''

\section{Agriculture}\label{agriculture}

\begin{itemize}
\item
  \emph{The New York Times reported in April 2019 that Wisconsin dairy
  farmers were facing "extinction" because Trump trade policies had
  exacerbated years of depressed prices caused by a production glut.}
\item
  \emph{Trump provided farmers \$12 billion in direct payments in July
  2018 to mitigate the negative impacts of his tariffs, increasing the
  payments by \$14.5 billion in May 2019 after trade talks with China
  ended without agreement.}
\item
  \emph{Farmers have been a large component of Trump's political base.}
\end{itemize}

Farmers have been a large component of Trump's political base. Trump's
fiscal 2020 budget proposed a 15\% funding cut for the Agriculture
Department, calling farm subsidies "overly generous," at a time when
American farmers were in their worst crisis in decades because of
Trump's trade tariffs combined with depressed commodities prices.
Richard Guebert, president of the Illinois Farm Bureau, which describes
itself as "the voice of agricultural producers at all levels," remarked,
"We are very concerned and disappointed with the administration's
proposed budget."

The New York Times reported in April 2019 that Wisconsin dairy farmers
were facing "extinction" because Trump trade policies had exacerbated
years of depressed prices caused by a production glut. Although the
United States--Mexico--Canada Agreement (USMCA) Trump had negotiated
would have given dairy farmers better access to Canadian markets, the
expected benefits had not materialized because the treaty had not been
approved by the Senate. Republican Senator Chuck Grassley of Iowa, a
farm state, wrote an April 2019 Wall Street Journal op-ed entitled
"Trump's Tariffs End or His Trade Deal Dies," stating "Congress won't
approve USMCA while constituents pay the price for Mexican and Canadian
retaliation." Trump announced three weeks later that the steel and
aluminum tariffs on Mexico and Canada would be lifted.

Politico reported in May 2019 that some economists in the Economic
Research Service of the Agriculture Department felt they were being
punished for presenting analysis showing that farmers were being harmed
by Trump's trade and tax policies, with six economists having more than
50 years of combined experience at the Service resigning on the same day
of April 2019.

Trump provided farmers \$12 billion in direct payments in July 2018 to
mitigate the negative impacts of his tariffs, increasing the payments by
\$14.5 billion in May 2019 after trade talks with China ended without
agreement.

\section{Consumer protections}\label{consumer-protections}

\begin{itemize}
\item
  \emph{In December 2017, Trump scrapped a proposed rule from the Obama
  administration that airlines disclose baggage fees.}
\item
  \emph{The Trump administration reduced enforcement of regulations
  against airlines; the fines levied by the administration in 2017 were
  less than half of what the Obama administration did the year before.}
\end{itemize}

In October 2017, Vice President Pence cast the tie-breaking vote in the
Senate to reverse a Consumer Financial Protection Bureau (CFPB) rule
that placed limits on mandatory arbitration and made it easier for
aggrieved consumers to pursue class actions against banks. Financial
firms lobbied for years against the rule; the Associated Press
characterized the reversal as a victory for Wall Street banks.

In December 2017, Trump scrapped a proposed rule from the Obama
administration that airlines disclose baggage fees. The Trump
administration reduced enforcement of regulations against airlines; the
fines levied by the administration in 2017 were less than half of what
the Obama administration did the year before.

Under Mick Mulvaney's tenure the CFPB reduced enforcement of rules that
protected consumers from predatory payday lenders. Center for Responsive
Politics data showed Mulvaney ranked ninth among all members of Congress
in campaign donations from the payday lending industry during the
2015-16 election cycle.

\includegraphics[width=5.50000in,height=3.66309in]{media/image4.jpg}\\
\emph{Trump signed new anti-sex-trafficking legislation on April 16,
2018}

\includegraphics[width=5.50000in,height=3.66016in]{media/image5.jpg}\\
\emph{Trump pays tribute to fallen police officers, May 15, 2017}

\section{Criminal justice}\label{criminal-justice}

\begin{itemize}
\item
  \emph{Trump appeared to advocate police brutality in a July 2017
  speech to police officers, prompting criticism from law enforcement
  agencies.}
\item
  \emph{Previously, in February 2017, when a sheriff complained about a
  state senator who proposed legislation to end asset forfeiture, Trump
  responded, "Who is the state senator?}
\end{itemize}

In November 2017, the New York Times summarized the Trump
administration's "general approach to law enforcement" as "cracking down
on violent crime", "not regulating the police departments that fight
it", and overhauling "programs that the Obama administration used to
ease tensions between communities and the police."

In July 2017, the Department of Justice reinstated the use of asset
forfeiture, which is the practice of seizing the property of crime
suspects who have not been charged or convicted with a crime. This meant
that local authorities in the 24 states that banned the practice or
limited its use so that it could now seize property from individuals who
have not even been charged with a crime if the property is forwarded to
the federal government. Previously, in February 2017, when a sheriff
complained about a state senator who proposed legislation to end asset
forfeiture, Trump responded, "Who is the state senator? Do you want to
give his name?~We'll destroy his career."

Trump appeared to advocate police brutality in a July 2017 speech to
police officers, prompting criticism from law enforcement agencies.

In December 2017, Trump signed into law the bipartisan First Step Act,
as part of several government efforts to overhaul the criminal justice
system. The act seeks help with prisoner reinsertion and reduce
recidivism, notably by expanding job training and early-release
programs, and lowering mandatory minimum sentences for nonviolent drug
offenders.

Special Counsel Robert Mueller's April 2019 report documented that Trump
pressured Attorney General Jeff Sessions and the Justice Department to
re-open the investigation into the Hillary Clinton email controversy.

Trump's 2019 budget plan proposed nearly 50\% cuts to COPS Hiring
Program which provides funding to state and local law enforcement
agencies to help hire community policing officers.

\section{Presidential pardons and
commutations}\label{presidential-pardons-and-commutations}

\begin{itemize}
\item
  \emph{In May 2019, Trump pardoned Conrad Black, a billionaire
  fraudster who had written a fawning book about Trump, and Pat Nolan, a
  Republican politician who pleaded guilty to public corruption.}
\item
  \emph{In June 2018, Trump commuted the sentence of Alice Johnson, a
  63-year old who was serving a life sentence for a nonviolent drug
  offense, after Kim Kardashian met Trump to lobby for her cause.}
\item
  \emph{Trump has issued a number of presidential pardons.}
\end{itemize}

Trump has issued a number of presidential pardons. In August 2017, he
pardoned Sheriff Joe Arpaio, who had been convicted of contempt of court
for failing to comply with court orders to stop racially profiling
Hispanics. In March 2018 he pardoned Kristian Saucier, a sailor
convicted for taking pictures aboard a nuclear submarine. In April 2018,
Trump pardoned Lewis "Scooter" Libby, chief of staff to former Vice
President Dick Cheney, who was convicted of obstruction of justice and
perjury in the investigation of the leak of the covert identity of
Central Intelligence Agency officer Valerie Plame Wilson. In May 2018,
Trump granted a posthumous pardon to black heavyweight boxer Jack
Johnson, who had been convicted in 1913 for taking his white girlfriend
across state lines, per the "moral purity" Mann Act of 1910. That same
month, Trump pardoned conservative pundit Dinesh D'Souza, who was
convicted of illegal campaign contributions in a 2012 Senate race. The
New York Times remarked that Trump took no action on more than 10,000
pending applications and that he solely used his pardon power on "public
figures whose cases resonated with him given his own grievances with
investigators." In June 2018, Trump commuted the sentence of Alice
Johnson, a 63-year old who was serving a life sentence for a nonviolent
drug offense, after Kim Kardashian met Trump to lobby for her cause. In
July 2018, Trump pardoned two Oregon ranchers who were convicted of
intentionally setting fires on public land in Oregon, and whose prison
sentences prompted armed protestors led by Clive Bundy to violently
seize the Malheur National Wildlife Refuge in Oregon for 41 days in
2016. In May 2019, Trump pardoned Conrad Black, a billionaire fraudster
who had written a fawning book about Trump, and Pat Nolan, a Republican
politician who pleaded guilty to public corruption.

\section{Defense}\label{defense}

\begin{itemize}
\item
  \emph{In December 2018, Trump complained about the amount America
  spends on an "uncontrollable arms race" with Russia and China.}
\item
  \emph{During 2018, Trump asserted he had secured the largest defense
  budget authorization ever, the first military pay raise in ten years,
  and that military spending was at least 4.0\% of GDP, "which got a lot
  bigger since I became your president."}
\item
  \emph{As a candidate and as president, Trump called for a major
  build-up of American military capabilities, including increasing the
  nuclear arsenal tenfold.}
\end{itemize}

During 2018, Trump asserted he had secured the largest defense budget
authorization ever, the first military pay raise in ten years, and that
military spending was at least 4.0\% of GDP, "which got a lot bigger
since I became your president." These statements were false.

As a candidate and as president, Trump called for a major build-up of
American military capabilities, including increasing the nuclear arsenal
tenfold. He stated he was open to allowing Japan and South Korea to
acquire nuclear weapons. He announced in October 2018 that America would
withdraw from the Intermediate-Range Nuclear Forces Treaty with Russia
to enable America to counter increasing Chinese intermediate nuclear
missile capabilities in the Pacific. In December 2018, Trump complained
about the amount America spends on an "uncontrollable arms race" with
Russia and China. Trump stated the \$716 billion America is now spending
on the "arms race" was "Crazy!," after praising his increased defense
spending five months earlier. The total fiscal 2019 defense budget
authorization was \$716 billion, although missile defense and nuclear
programs comprised about \$10 billion of the total.

\section{Drug policy}\label{drug-policy}

\begin{itemize}
\item
  \emph{The Trump administration's decision contradicted then-candidate
  Trump's statement that marijuana legalization should be "up to the
  states".}
\end{itemize}

In a May 2017 departure from the Obama DOJ's policy to reduce long jail
sentencing for minor drug offenses, Sessions ordered federal prosecutors
to seek maximum sentencing for drug offenses. According to The New York
Times, the action ran "contrary to the growing bipartisan consensus
coursing through Washington and many state capitals in recent years ---
a view that America was guilty of excessive incarceration and that large
prison populations were too costly in tax dollars and the toll on
families and communities."

In a January 2018 move that created uncertainty regarding the legality
of recreational and medical marijuana, Sessions rescinded federal policy
that had barred federal law enforcement officials from aggressively
enforcing federal cannabis law in states where the drug is legal. The
Trump administration's decision contradicted then-candidate Trump's
statement that marijuana legalization should be "up to the states". That
same month, the Department of Veterans Affairs said that it would not
research cannabis as a potential treatment against PTSD and chronic
pain; veterans organizations had pushed for such a study.

\includegraphics[width=5.50000in,height=3.09375in]{media/image6.jpg}\\
\emph{Trump and Vice-President Pence met with key automobile industry
leaders, January 24, 2017}

\includegraphics[width=5.50000in,height=3.66667in]{media/image7.jpg}\\
\emph{During President Trump's first foreign trip to Saudi Arabia, Trump
announced an arms deal with Saudi Arabia, May 2017}

\section{Economy}\label{economy}

\begin{itemize}
\item
  \emph{The deficit in goods, Trump's preferred trade balance measure,
  increased 8\% in 2017 and 10\% in 2018, reaching a record high of
  \$891 billion in 2018.}
\item
  \emph{In May 2018, Trump initiated a trade conflict with the EU by
  imposing tariffs on steel and aluminum, for which the EU retaliated in
  June with tariffs of their own, with Trump threatening to escalate the
  conflict with additional tariffs.}
\item
  \emph{Trump linked his criticism of Amazon to Amazon's ownership by
  Jeff Bezos, who also owns The Washington Post, which Trump has derided
  as "fake news".}
\end{itemize}

Prior to Trump's election, the American economy had been expanding for
over seven years, with steady growth in employment, a declining
unemployment rate, and steadily rising home values, stock values and
household income and wealth. Trump's economic policies centered on tax
cuts, deregulation, trade protectionism and immigration reduction. As
candidate and president, Trump claimed his policies would spur much
higher GDP growth, stating in December 2017, "I see no reason why we
don't go to 4, 5, even 6 percent," figures that economists generally
consider impossible to achieve on a sustained basis. In July 2018, Trump
stated, "We have added 3.7 million new jobs since the election, a number
that is unthinkable if you go back to the campaign. Nobody would have
said it." During the 19 months since the election, nearly 3.7 million
jobs were created; during the 19 months prior to the election, 4.3
million jobs were created. Trump also stated, "More than 10 million
additional Americans had been added to food stamps, past years. But
we've turned it all around." SNAP participation had been steadily
declining since December 2012.

During the 2016 campaign, Trump proposed \$1 trillion in infrastructure
investments. In February 2018, Trump released a \$1.5 trillion federal
infrastructure plan, but left the details of the plan for Congress to
solve, including how to pay for it. Congress showed little enthusiasm
for the plan.

One of the administration's first actions was to indefinitely suspend a
cut in fee rates for HUD mortgages implemented by the Obama
administration. The cut in fee rates would have saved individuals with
lower credit scores around \$500 per year on a typical loan.

In September 2017, the DOJ announced it would not defend in courts a
mandate that would have extended overtime benefits to more than 4
million workers.

In September 2017, the administration proposed a tax overhaul, which
became the Tax Cuts and Jobs Act of 2017. The proposal was to reduce the
corporate tax rate to 20\% (from 35\%) and eliminate the estate tax. On
individual tax returns it would change the number of tax brackets from
seven to three, with tax rates of 12\%, 25\%, and 35\%; apply a 25\% tax
rate to business income reported on a personal tax return; eliminate the
alternative minimum tax; eliminate personal exemptions; double the
standard deduction; and eliminate many itemized deductions (specifically
retaining the deductions for mortgage interest and charitable
contributions).

According to The New York Times, the plan would result in a "huge
windfall" for the very wealthy but would not benefit those in the bottom
third of the income distribution. The nonpartisan Tax Policy Center
estimated that the richest 0.1\% and 1\% would benefit the most in raw
dollar amounts and percentage terms from the tax plan, earning 10.2\%
and 8.5\% more income after taxes respectively. Middle-class households
would on average earn 1.2\% more after tax, but 13.5\% of middle class
households would see their tax burden increase. The poorest fifth of
Americans would earn 0.5\% more. Treasury Secretary Steven Mnuchin
argued that the corporate income tax cut will benefit workers the most;
however, the nonpartisan Joint Committee on Taxation, Congressional
Budget Office and many economists estimated that owners of capital would
benefit vastly more than workers. A preliminary estimate by the
Committee for a Responsible Federal Budget found that the tax plan would
add more than \$2 trillion over the next decade to the federal debt,
while the Tax Policy Center found that it would add \$2.4 trillion to
the debt.

In January 2018, ProPublica analyzed specific claims made by President
Trump about job creation in companies during the first year of his
presidency; Trump claimed that 2.4 million jobs had been or would be
created as a result of his policies. ProPublica found that only 136,000
new jobs were created, and that only 63,000 of those jobs could be
potentially attributed to Trump's policies.

For the first year when the Trump administration was fully in charge of
the budget, the fiscal year of 2018, the federal government was on track
to borrow nearly a trillion dollars; "this is the first time borrowing
has jumped this much (as a share of GDP) in a non-recession time since
Ronald Reagan was president." The budget shortfall was primarily due to
the GOP's 2017 tax reform.

During his tenure, Trump repeatedly sought to intervene in the economy
in ways to determine corporate winners and losers. This was a shift from
Republican orthodoxy. Trump, for example, sought to compel power grid
operators to buy coal and nuclear energy, and sought tariffs on metals
to protect domestic metal producers. Trump also publicly attacked Boeing
and Lockheed Martin, sending their stocks tumbling. Trump repeatedly
singled out Amazon for criticism and advocated steps that would harm the
company, such as ending a mutually lucrative arrangement between Amazon
and the US Postal Service and raising taxes on Amazon. Trump linked his
criticism of Amazon to Amazon's ownership by Jeff Bezos, who also owns
The Washington Post, which Trump has derided as "fake news". Trump
expressed vociferous opposition to the merger between Time Warner (the
parent company of CNN) and AT\&T. After the merger was completed, Trump
in June 2019 suggested boycotting AT\&T to force changes at CNN.

In March 2018, Trump announced tariffs on steel and aluminum imports,
triggering a series of tit for tat tariffs and threatened additional
tariffs from multiple nations, which by June 2018 had escalated into
what some characterized as a trade war The trade dispute disrupted
global commerce, with the New York Times noting that "shipments are
slowing at ports and airfreight terminals around the world. Prices for
crucial raw materials are rising. At factories from Germany to Mexico,
orders are being cut and investments delayed. American farmers are
losing sales as trading partners hit back with duties of their own." By
June 2018, negative effects of the Trump tariffs policy had begun to
ripple through the American economy, in particular the agriculture
sector. In July 2018, China retaliated with a \$34~billion in tariffs on
U.S. goods. Trump had signaled that he might impose an additional
\$200~billion in tariffs if China imposed their own tariffs, with the
potential for a further \$200~billion, in an escalating trade war. China
and the United States entered into negotiations in the summer of 2018 to
seek a comprehensive solution to their trade conflict, but the talks
ended without agreement on May 10, 2019, and that day Trump carried out
his earlier threat to impose additional tariffs on \$200 billion in
Chinese goods. Some economists estimated that the cumulative effect of
the continuing trade conflict could raise costs for the average American
household by several hundreds of dollars per year.

Upon taking office, Trump halted trade negotiations with the European
Union on the Transatlantic Trade and Investment Partnership (TTIP),
which had been under way since 2013. In May 2018, Trump initiated a
trade conflict with the EU by imposing tariffs on steel and aluminum,
for which the EU retaliated in June with tariffs of their own, with
Trump threatening to escalate the conflict with additional tariffs. In
July 2018, Trump and the EU declared a truce of sorts, announcing they
would enter into negotiations for an agreement similar to the TTIP.

From June 2018, Trump has repeatedly and falsely characterized the
economy during his presidency as the best in American history; he has
made some version of this claim over 130 times.

The New York Times reported on August 5, 2018 that two major American
steel companies with close ties to senior Trump administration officials
had succeeded in blocking requests from 1,600 American manufacturing
companies for waivers of the steel tariffs, compelling them to purchase
more expensive American steel. Nucor had financed a documentary made by
Peter Navarro, Trump's Director of the White House National Trade
Council, and US Steel had previously been represented in legal matters
by Trump's trade representative Robert Lighthizer and his deputy Jeffrey
Gerrish.

A July 2018 paper, which used the synthetic control method, found no
evidence that Trump had an impact on the US economy during his 18 months
in office.

Analysis conducted by Bloomberg News at the end of Trump's second year
in office found that his economy ranked sixth among the last seven
presidents, based on fourteen metrics of economic activity and financial
performance.

During his February 2019 State of the Union Address, Trump asserted,
"Wages are rising at the fastest pace in decades, and growing for blue
collar workers, who I promised to fight for, faster than anyone else."
However, nominal wage growth for production and nonsupervisory workers
averaged 3.0\% during 2018, the highest rate since 2009. Adjusted for
inflation, the 2018 average growth rate for such workers was 0.5\%, the
highest rate since 2016, when real wages rose 1.2\%. Real wage growth
was lower during both of Trump's first two years in office than during
any of the preceding four years.

The Trump campaign economic policy, as expressed by Peter Navarro and
Wilbur Ross in a September 2016 white paper, included a priority of
"eliminating America's chronic trade deficit," particularly with China.
However, the overall trade deficit increased in both of Trump's first
two years in office, up 10\% in 2017 and 13\% in 2018, compared to
single-digit increases during each of the preceding three years. The
deficit in goods, Trump's preferred trade balance measure, increased 8\%
in 2017 and 10\% in 2018, reaching a record high of \$891 billion in
2018. The goods deficit with China reached a record high for the second
consecutive year in 2018, up 12\% from 2017. The overall deficits were
mitigated somewhat by surpluses in services, continuing a trend of many
years.

According to two 2019 studies, Trump's trade war harmed the U.S.
economy, with U.S. consumers bearing the brunt of the cost.

Shortly after being elected, Trump cited a handful of anecdotes to
assert that foreign investment had begun pouring into America because of
his election. However, aggregate statistical data showed that foreign
direct investment---the total flow of investment capital into the United
States from the rest of the world---declined sharply during Trump's
first two years in office, down 40\% compared to the two years
immediately preceding his presidency.

In April 2019, one week after asserting, "the economy is roaring" and
"our country has never done better economically," Trump called for the
Federal Reserve to cut interest rates and renew quantitative easing to
stimulate economic growth. Trump has been repeatedly critical of the
Fed's use of low interest rates and quantitative easing to boost the
economy in the aftermath of the Great Recession during the Obama
presidency.

Analysis conducted by CNBC in May 2019 found that Trump "enacted tariffs
equivalent to one of the largest tax increases in decades," while Tax
Foundation and Tax Policy Center analyses found the tariffs could wipe
out the benefits of the Tax Cuts and Jobs Act of 2017 for many
households. While Trump has repeatedly asserted that his tariffs
contribute to GDP growth, the consensus among analysts --- including
Trump's top economic advisor, Larry Kudlow --- is that the Trump tariffs
have had a small to moderately negative effect on GDP growth. Kudlow was
also quoted as being in support of the administrations efforts to
renegotiate tariffs with China, stating that this is, ``a risk we should
and can take without damaging our economy in any appreciable way'' in
order ``to correct 20 years plus of unfair trading practices with
China.'' Kudlow went on to state that, ``We have had unfair trading
practices all these years, and so in my judgment, the economic
consequences are so small that the possible improvement in trade and
exports and open markets for the United States, this is worthwhile
doing,''.

When first quarter 2019 GDP growth reached 3.2\%, Trump falsely asserted
it was ``a number that they haven't hit in 14 years.''

Three weeks after Republican Senator Chuck Grassley, chairman of the
Senate Finance Committee, wrote an April 2019 Wall Street Journal op-ed
entitled "Trump's Tariffs End or His Trade Deal Dies," stating "Congress
won't approve USMCA while constituents pay the price for Mexican and
Canadian retaliation," Trump lifted steel and aluminum tariffs on Mexico
and Canada. Two weeks later, Trump unexpectedly announced that he would
impose a 5\% tariff on all imports from Mexico on June 10, increasing to
10\% on July 1, and by another 5\% each month for three months, ``until
such time as illegal migrants coming through Mexico, and into our
Country, STOP.'' Hours later, Grassley commented, ``This is a misuse of
presidential tariff authority and counter to congressional intent.
Following through on this threat would seriously jeopardize passage of
USMCA, a central campaign pledge of President Trump's and what could be
a big victory for the country." That same day, the Trump administration
formally initiated the process to seek congressional approval of USMCA.
Trump's top trade advisor, US Trade Representative Robert Lighthizer,
opposed the new Mexican tariffs on concerns it would jeopardize passage
of USMCA. Treasury secretary Steven Mnuchin and Trump senior advisor
Jared Kushner also opposed the action. Grassley, whose committee is
instrumental in passing USMCA, was not informed in advance of Trump's
surprise announcement. An array of lawmakers and business groups
expressed consternation about the proposed tariffs. With 2018 imports of
Mexican goods totaling \$346.5 billion, a 5\% tariff constitutes a tax
increase of over \$17 billion. On June 7, Trump announced the tariffs
would be "indefinitely suspended" after Mexico agreed to take actions,
including deploying its National Guard throughout the country and along
its southern border. The New York Times reported the following day that
Mexico had actually agreed to most of the actions months earlier. Also
that day, Trump tweeted, "MEXICO HAS AGREED TO IMMEDIATELY BEGIN BUYING
LARGE QUANTITIES OF AGRICULTURAL PRODUCT FROM OUR GREAT PATRIOT
FARMERS!," although the communique between the countries did not mention
any such deal and Mexican officials were not aware of such discussions,
while American officials declined comment.

Analysis conducted by Deutsche Bank estimated that Trump's trade actions
had resulted in foregone American stock market capitalization of \$5
trillion through May 2019.

Analysis of Trump's 2017 tax cut released by the Congressional Research
Service in May 2019 found that "On the whole, the growth effects tend to
show a relatively small (if any) first-year effect on the economy."

By late 2018 and early 2019, the national average unemployment continued
to decline to the lowest level since 1969, with occasional record lows
set for 19 states,, for the Hispanic ethnicity, and for the black and
Asian American races.

\includegraphics[width=5.50000in,height=3.66667in]{media/image8.jpg}\\
\emph{U.S. Secretary of Education Betsy DeVos and President Trump visit
Saint Andrew's Catholic School in Orlando, Florida, March 3, 2017}

\section{Education}\label{education}

\begin{itemize}
\item
  \emph{In March 2017, the Trump administration revoked an Obama
  administration memo which provided protections for people in default
  on student loans.}
\item
  \emph{In September 2017, the Trump administration scrapped an Obama
  administration guidance on how schools and universities should combat
  sexual harassment and sexual violence.}
\end{itemize}

In March 2017, the Trump administration revoked an Obama administration
memo which provided protections for people in default on student loans.
In September 2017, the Department of Education announced that it would
cancel agreements with the CFPB to police student loan fraud. In August
2018, Seth Frotman, the CFPB student loan ombudsman, resigned, accusing
the Trump administration of undermining the CFPB's work on protecting
student borrowers.

In September 2017, the Trump administration scrapped an Obama
administration guidance on how schools and universities should combat
sexual harassment and sexual violence. DeVos criticized the guidance for
undermining the rights of those accused of sexual harassment.

In May 2018, a New York Times investigation found that DeVos had
marginalized an investigative unit within the Department of Education
which under Obama investigated predatory activities by for-profit
colleges. The unit had been scaled down from a dozen employees to three,
and had been repurposed to process student loan forgiveness applications
and focus on smaller compliance cases. An investigation started under
Obama into the practices of DeVry Education Group, which operates
for-profit colleges, was halted in early 2017, and the former dean at
DeVry was made into the supervisor for the investigative unit later that
summer. DeVry paid a \$100 million fine in 2016 for defrauding students.
In August 2018, the administration rescinded a regulation that
restricted federal funding and financial aid to for-profit colleges
unable to demonstrate that college graduates had a reasonable
debt-to-earnings ratio after entering the job market.

\section{Election integrity}\label{election-integrity}

\begin{itemize}
\item
  \emph{On the eve of the 2018 mid-term elections, Politico described
  the Trump administration's efforts to combat election propaganda as
  "rudderless".}
\end{itemize}

On the eve of the 2018 mid-term elections, Politico described the Trump
administration's efforts to combat election propaganda as "rudderless".
At the same time, U.S. intelligence agencies warned about "on-going
campaigns" by Russia, China and Iran to influence American elections.

\includegraphics[width=5.46333in,height=5.50000in]{media/image9.png}\\
\emph{2017 Trump rally in Harrisburg, Pennsylvania. Trump hold a placard
that reads "TRUMP DIGS COAL"}

\includegraphics[width=3.80600in,height=5.50000in]{media/image10.jpg}\\
\emph{The official portrait of Scott Pruitt as EPA Administrator.}

\includegraphics[width=5.50000in,height=3.16250in]{media/image11.jpg}\\
\emph{Trump signing the presidential memoranda to advance the
construction of the Keystone XL and Dakota Access pipelines. January 24,
2017}

\section{Environment and energy}\label{environment-and-energy}

\begin{itemize}
\item
  \emph{The move raised ethical questions because Trump owns a resort in
  Florida and because Florida is a swing state that Trump would like to
  win in the 2020 presidential election.}
\item
  \emph{In March 2017, Trump issued an executive order reversing
  multiple Obama administration policies meant to tackle climate
  change.}
\item
  \emph{In 2018, the Trump administration referred the lowest number of
  pollution cases for criminal prosecution in 30 years.}
\end{itemize}

By December 2018, the administration had overturned or was in process of
rolling back 78 environmental regulations. A 2018 study in the American
Journal of Public Health found that that in the first six months of
Pruitt's tenure as EPA head the agency adopted a pro-business attitude
unlike that of any previous administration. The study argued that the
EPA "moved away from the public interest and explicitly favored the
interests of the regulated industries." The study found that the agency
was vulnerable to regulatory capture and that the consequences for
public and environmental health could be far-reaching. The Washington
Post summarized Pruitt's leadership of the EPA in 2017 as follows, "In
legal maneuvers and executive actions, in public speeches and
closed-door meetings with industry groups, he has moved to shrink the
agency's reach, alter its focus, and pause or reverse numerous
environmental rules. The effect has been to steer the EPA in the
direction sought by those being regulated. Along the way, Pruitt has
begun to dismantle former president Barack Obama's environmental legacy,
halting the agency's efforts to combat climate change and to shift the
nation away from its reliance on fossil fuels." In December 2017, a New
York Times analysis of EPA enforcement data found that the Trump
administration had adopted a far more lenient approach to enforcing
federal pollution laws than the Obama and Bush administrations. The
Trump administration brought fewer cases against polluters, sought a
lower total of civil penalties and made fewer requests of companies to
retrofit facilities to curb pollution. According to the New York Times,
"confidential internal E.P.A. documents show that the enforcement
slowdown coincides with major policy changes ordered by Mr. Pruitt's
team after pleas from oil and gas industry executives." In 2018, the
Trump administration referred the lowest number of pollution cases for
criminal prosecution in 30 years. Two years into Trump's presidency, The
New York Times wrote he had "unleashed a regulatory rollback, lobbied
for and cheered on by industry, with little parallel in the past
half-century."

Moments after Trump's inauguration, the White House removed all
references to climate change on its website, with the sole exception of
mentioning Trump's intention to eliminate the Obama administration's
climate change policies. By April, the EPA had removed climate change
material on its website, including detailed climate data and scientific
information. Anticipating political interference that could result in
loss of government data on climate, scientists had already sourced links
and copied the data into independent servers.

In January 2017, the administration instituted a temporary media
blackout for the EPA, saying that this was to make sure the messages
going out reflected the new administration's priorities. In February
2017, the administration ended its earlier freeze on EPA contract and
grant approvals, and the appearance of some EPA press releases that week
indicated the media blackout was partially lifted. The EPA hired an
opposition research firm associated with the Republican Party for
\$120,000 in a no-bid contract to investigate EPA employees who had
expressed criticism of the management of the EPA under Pruitt's tenure.
In March 2018, leaked memos showed that EPA employees had been issued
guidelines to use climate change denial talking points in official
communications about climate change. In October 2018, the EPA disbanded
a 20-expert panel on pollution which advised the EPA on the appropriate
threshold levels to set for air quality standards.

In February 2017, Trump and Congress removed a rule that required oil,
gas and mining firms to disclose how much they paid foreign governments.
The industries claimed the rule gave global rivals a competitive edge,
although EU, Canadian, Russian, Chinese and Brazilian energy firms are
bound by similar requirements. The administration withdrew from the
international Extractive Industries Transparency Initiative (EITI). EITI
was aimed at fighting corruption by requiring the disclosure of payments
and donations made by oil, gas and mining companies to governments.

The administration invalidated the Stream Protection Rule, a regulation
intended to prevent coal mining debris from being dumped into nearby
streams, and to lessen the impact of coal mining on groundwater and
surface waters. The administration rolled back regulations which limited
dumping by power plants of toxic wastewater containing metals like
arsenic and mercury into public waterways, Obama-era regulations on coal
ash (carcinogenic leftover waste produced by coal plants), and an
Obama-era executive order on protections for oceans, coastlines and
lakes which was enacted after the Deepwater Horizon oil spill. The EPA
sought to repeal a regulation which required oil and gas companies to
restrict emissions of methane, a potent greenhouse gas. In July 2018,
the EPA granted a loophole allowing a small set of trucking companies to
skirt emissions rules, allowing the firms to produce trucks that emit 40
to 55 times the air pollutants of other new trucks.

In March 2017, Trump issued an executive order reversing multiple Obama
administration policies meant to tackle climate change. Trump said he
was "putting an end to the war on coal", removing "job-killing
regulations" and "restrictions on American energy" to make "America
wealthy again". Trump ended the moratorium on federal coal leasing,
revoked several Obama executive orders including the Presidential
Climate Action Plan, and also removed guidance for federal agencies on
taking climate change into account during National Environmental Policy
Act action reviews. Trump also ordered reviews and possibly
modifications to several directives, such as the Clean Power Plan (CPP),
the estimate for the "social cost of carbon" emissions, carbon dioxide
emission standards for new coal plants, methane emissions standards from
oil and natural gas extraction, as well as any regulations inhibiting
domestic energy production. A 2019 projection by the Energy Information
Administration estimated that coal production without CPP would decline
over coming decades at a faster rate than indicated in the agency's 2017
projection, which had assumed the CPP was in effect.

That same month, the EPA rejected a ban on the toxic pesticide
chlorpyrifos, which the EPA's own agency staff had recommended banning
due to extensive research showing adverse health effects on children. In
August 2018, a federal court ordered the EPA to ban the pesticide,
because EPA heads had ignored conclusions of the EPA's own scientists.

In June 2017, Trump announced U.S. withdrawal from the Paris Agreement,
a 2015 climate change accord reached by 200 nations to cut greenhouse
gas emissions, defying broad global backing for the plan.

The administration suspended a number of large research programs, such
as a \$1 million National Academy of Sciences (NAS) study on the public
health effects of mountaintop removal coal-mining, a \$580,000 NAS study
intended to make offshore drilling safer, a multimillion-dollar program
that distributed grants for research the effects of chemical exposure on
children, and \$10-million-a-year research line for NASA's Carbon
Monitoring System. The administration unsuccessfully sought to kill
aspects of NASA's climate science program.

In August 2017, the administration rolled back regulations requiring the
federal government to account for climate change and sea-level rise when
building infrastructure.

By October 2017, the EPA expedited the process for approving new
chemicals and made the process of evaluating the safety of those
chemicals less stringent. Officials and longtime scientists at the EPA
expressed concerns that the agency's ability to stop hazardous chemicals
was being compromised. Internal emails showed that Pruitt's aides in
early 2018 prevented the publication of a health study showing that some
toxic chemicals endanger humans at far lower levels than the EPA
previously characterized as safe. The aides said that the study would be
a "potential public relations nightmare" and would attract the attention
of the public, media and Congress. The chemical in question was present
in high quantities around a number of military bases, including in the
ground water. The non-disclosure of the study and the delay in public
knowledge of the findings may have prevented the government from
updating the infrastructure at the bases and individuals who lived near
the bases from avoiding the tap water. In June 2018, the EPA scaled back
its health and safety risk assessments of chemicals.

In December 2017, the administration sharply reduced the size of two
national monuments in Utah by approximately two million acres, making it
the largest reduction of public land protections in American history.
Shortly afterwards, Interior Secretary Zinke advocated for downsizing
four additional national monuments and change the way that six
additional monuments were managed.

In December 2017, President Trump - who had repeatedly called scientific
consensus on climate a "hoax" before becoming president - for the first
time as president called into question climate change by falsely
implying that cold weather at the end of December meant that climate
change was not occurring.

In January 2018, the administration singled out the state of Florida as
an exemption from the administration's offshore drilling plan. The move
stirred controversy because it came after the Governor of Florida,
Republican Rick Scott (who was considering a 2018 Senate run),
complained about the offshore drilling plan. The move raised ethical
questions because Trump owns a resort in Florida and because Florida is
a swing state that Trump would like to win in the 2020 presidential
election. NBC News said that the decision had the appearance of
"transactional favoritism" and that it was likely to lead to lawsuits.

That same month, the administration enacted 30\% tariffs on solar
panels. The American solar energy industry is highly reliant on foreign
parts (80\% of parts are made abroad); as a result, the tariffs could
raise the costs of solar energy, reduce innovation and reduce jobs in
the industry --- which in 2017 employed nearly four times as many
American workers as the coal industry. Bloomberg News described the move
as the Trump administration "most targeted strike on the
{[}renewables{]} industry" in a series of actions taken to undermine
renewables.

In April 2018, Pruitt announced a policy change within the EPA whereby
EPA regulators would be prohibited from considering scientific research
unless the raw data of the research was made publicly available. This
would limit EPA regulators' use of much environmental research, given
that participants in many such studies provide personal health
information which is kept confidential. The EPA cited two bipartisan
reports and various nonpartisan studies about the use of science in
government to defend the decision. However, the authors of those reports
dismissed that the EPA followed their instructions, with one author
saying, "They don't adopt any of our recommendations, and they go in a
direction that's completely opposite, completely different. They don't
adopt any of the recommendations of any of the sources they cite."

In June 2018, David Cutler and Francesca Dominici of Harvard University
estimated conservatively that the Trump administration's modifications
to environmental rules could result in over 80 000 additional U.S.
deaths and widespread respiratory ailments. In August 2018, the Trump
administration's own analysis showed that the administration's loosening
of restrictions on coal plants could cause up to 1,400 premature deaths
and 15,000 new cases of respiratory problems.

In July 2018, the administration proposed to change the Endangered
Species Act to eliminate automatic protections for threatened plant and
animal species, and make it easier to remove species from the list.

The day after Thanksgiving 2018, the administration released the Fourth
National Climate Assessment (NCA4), a long-awaited study conducted by
numerous federal agencies that found "the evidence of human-caused
climate change is overwhelming and continues to strengthen, that the
impacts of climate change are intensifying across the country, and that
climate-related threats to Americans' physical, social, and economic
well-being are rising." Two days earlier, Trump had repeated his many
previous assertions that a current cold weather spell calls global
warming science into question, a notion that has been repeatedly
debunked by climate scientists. Steven Milloy, a climate-change denier
who served on Trump's EPA transition team, called the report a product
of the so-called deep state, adding "We don't care. In our view, this is
made-up hysteria anyway." He noted that the Administration did not alter
the report's findings but rather chose to release it the day after
Thanksgiving "on a day when nobody cares, and hope it gets swept away by
the next day's news."

In December 2018, the administration joined Russia and the gulf states
of Saudi Arabia and Kuwait in stopping the Katowice climate change
conference from welcoming an October 2018 IPCC report on the dangers of
climate change.

That same month, the administration rolled back major Clean Water Act
protections. Studies by the Obama-era EPA suggest that up to two-thirds
of California's inland freshwater streams would lose protections under
the rule change.

In March 2019, the administration scaled back the ban on the use of a
lethal chemical, methylene chloride, which is used in paint stripping.

In June 2009, the Trump White House tried to prevent a State Department
intelligence analyst from testifying to Congress about "possibly
catastrophic" effects of human-caused climate change, and prevented his
written testimony containing science from NASA and NOAA from being
included in the official Congressional Record because it was not
consistent with administration positions.

\section{Government size and
regulations}\label{government-size-and-regulations}

\begin{itemize}
\item
  \emph{The hiring freeze was lifted in April 2017.}
\item
  \emph{In early 2017, Trump signed an executive order directing federal
  agencies to slash two existing regulations for every new one (without
  spending on regulations going up).}
\item
  \emph{In the first six weeks of his tenure, Trump suspended --- or in
  a few cases, revoked --- over 90 regulations.}
\item
  \emph{The Trump OMB released an analysis in February 2018 indicating
  that the economic benefits of regulations significantly outweigh the
  economic costs.}
\end{itemize}

The New York Times found in November 2018 that the administration had
"presided over a sharp decline in financial penalties against banks and
big companies accused of malfeasance," relative to the Obama
administration.

In the first six weeks of his tenure, Trump suspended --- or in a few
cases, revoked --- over 90 regulations. In January 2017, Trump ordered a
temporary government-wide hiring freeze of the civilian work force
(excluding staff in the military, national security, public safety and
offices of new presidential appointees). In February 2017, said he did
not intend to fill many of the governmental positions that were still
vacant, as he considered them unnecessary; there were nearly 2,000
vacant government positions. The hiring freeze was lifted in April 2017.

In early 2017, Trump signed an executive order directing federal
agencies to slash two existing regulations for every new one (without
spending on regulations going up). A September 2017 Bloomberg BNA review
found that due to unclear wording in the order and the large proportion
of regulations that it exempts, the order had had little effect since it
was signed. The Trump OMB released an analysis in February 2018
indicating that the economic benefits of regulations significantly
outweigh the economic costs.

In July 2018, the administration ended the requirement that nonprofits,
including political advocacy groups who collect so-called "dark money",
disclose the names of large donors to the Internal Revenue Service.
Later that year, the Senate voted to overturn the administration's rule
change.

\section{Guns}\label{guns}

\begin{itemize}
\item
  \emph{In March 2018, Trump instructed the DOJ to ban bump stocks.}
\item
  \emph{A White House statement quoted Trump saying, "We will never
  surrender America's sovereignty to an unelected, unaccountable, global
  bureaucracy."}
\end{itemize}

In February 2017, the administration rolled back an Obama-era regulation
prohibiting gun ownership among the approximately 75,000 individuals who
received Social Security checks due to mental illness and who were
deemed unfit to handle their financial affairs.

In March 2018, Trump instructed the DOJ to ban bump stocks. On March 26,
2019, 'bump stocks' were banned under federal law.

During an address before the National Rifle Association on April 26,
2019, Trump signed a letter that ended American involvement in the Arms
Trade Treaty, a United Nations agreement to curb the international trade
of conventional arms, including guns, with countries having poor human
rights records. America had been abiding by the treaty since 2014,
although it had not yet been ratified. A White House statement quoted
Trump saying, "We will never surrender America's sovereignty to an
unelected, unaccountable, global bureaucracy." The Arms Trade Treaty
explicitly preserves "the right of States to regulate internal transfers
of arms and national ownership, including through national
constitutional protections on private ownership, exclusively within
their territory."

\includegraphics[width=5.50000in,height=3.08917in]{media/image12.png}\\
\emph{CBO estimated in May 2017 that the Republican AHCA would reduce
the number of persons with health insurance by 23 million during 2026,
relative to current law.}

\includegraphics[width=3.84267in,height=5.50000in]{media/image13.jpg}\\
\emph{HHS Secretary Alex Azar}

\includegraphics[width=5.50000in,height=4.12500in]{media/image14.jpg}\\
\emph{Drug overdoses killed 70,200 in the United States in 2017.}

\section{Health care}\label{health-care}

\begin{itemize}
\item
  \emph{Trump repeatedly expressed a desire to "let Obamacare fail", and
  the Trump administration has been accused of trying to "sabotage
  Obamacare" by various actions.}
\item
  \emph{Trump had previously vowed to protect Medicare and Medicaid.}
\item
  \emph{First, in March 2017, Trump endorsed the American Health Care
  Act (AHCA), a Republican bill to repeal and replace the ACA.}
\end{itemize}

The 2010 Patient Protection and Affordable Care Act (also known as
"Obamacare" or the ACA) elicited major opposition from the Republican
Party from its inception, and Trump called for a repeal of the law
during the 2016 election campaign. On taking office, Trump promised to
pass a healthcare bill that would cover everyone and result in better
and less expensive insurance.

Congressional Republicans made two serious efforts to repeal the ACA.
First, in March 2017, Trump endorsed the American Health Care Act
(AHCA), a Republican bill to repeal and replace the ACA. Opposition from
several House Republicans, including both moderate and conservatives,
led to the defeat of this version of the bill on March 24, 2017. At the
time, Trump stated that the "best thing politically is to let Obamacare
explode". Several weeks later on May 4, the House narrowly voted in
favor of a new version of the AHCA to repeal the ACA, sending the bill
to the Senate for deliberation. Over the next weeks the Senate made
several attempts to create a repeal bill; however, all the proposals
were ultimately rejected in a series of Senate votes in late July. Trump
reacted by alternately urging Congress to keep trying and threatening to
"let Obamacare implode". The individual mandate was ultimately repealed
in December 2017 by the Tax Cuts and Jobs Act. The CBO estimated in May
2018 that the repeal of the individual mandate would increase the number
of uninsured by 8 million and that individual healthcare insurance
premiums increased by 10\% between 2017 and 2018.

Trump repeatedly expressed a desire to "let Obamacare fail", and the
Trump administration has been accused of trying to "sabotage Obamacare"
by various actions. The open enrollment period was cut from 12 weeks to
6, the advertising budget for enrollment was cut by 90\%, and
organizations helping people shop for coverage got 39\% less money. In
September 2017, the administration ordered HHS regional directors not to
participate in state open enrollment events, as they had in previous
years. A September 2017 report by the (CBO) found that ACA enrollment at
health care exchanges would be lower in 2018 and future years than its
previous forecasts due to the Trump administration's cuts to
advertisement spending for enrollment, a smaller enrollment window, and
less outreach. The CBO also found that insurance premiums would rise
sharply in 2018 due to the Trump administration's refusal to commit to
continuing paying ACA subsidies, which added uncertainty to the
insurance market and led insurers to raise premiums for fear they will
not get subsidized. In June 2018, the administration sided with a
lawsuit to overturn the ACA, including protections for individuals with
pre-existing conditions. In October 2018, Trump stated "Some of the
Democrats have been talking about ending (coverage for) pre-existing
conditions," and 20 days later he tweeted "Republicans will totally
protect people with Pre-Existing Conditions, Democrats will not!"
Politifact rated Trump's statement about Democrats as "Pants on Fire,"
also noting that eighteen attorneys general and two governors, all
Republicans, filed a federal lawsuit in February 2018 that would end
coverage for pre-existing conditions, and that the Trump administration
had chosen to not challenge the suit.

In October 2017, the administration ended subsidy payments to health
insurance companies, saying that they are "moving toward lower costs and
more options in the health care market". The decision was expected to
raise premiums in 2018 for middle-class families by an average of about
20\% nationwide and cost the federal government nearly \$200 billion
more than it saved over a ten-year period. People with lower incomes
would be unaffected because the ACA provides tax credits that ensure
their out-of-pocket insurance costs remain stable. The administration
made it easier for businesses to use health insurance plans that are not
covered by several of the ACA's protections, such as to protect
individuals with preexisting conditions. During the 2018 mid-term
election campaign, Trump said that "all Republicans", him included,
supported protections for individuals with preexisting conditions; at
the time, the administration had supported attempts both in Congress and
in the courts to roll back the ACA (and its protections for preexisting
conditions).

In October 2017, the administration modified a requirement that
employer-provided health insurance policies had to cover birth control
methods free of charge to women so that any company or nonprofit could
opt out of the requirement if they had religious or moral objections to
birth control. Survey results indicate that more than 10\% of companies
with more than 200 employees would opt out of birth control coverage if
they had the option to whereas the administration said that no more than
120,000 women~would be affected. In justifying the action, the
administration said that contraceptive use caused harms, such as risky
sex behavior, cited the potential side effects of contraceptives, and
asserted that the relationship between contraceptive use and unintended
pregnancy was uncertain and complex. Indiana University professor of
pediatrics Aaron E. Carroll noted "there is ample evidence that
contraception works, that reducing its expense leads to more women who
use it appropriately, and that using it doesn't lead to riskier sexual
behavior."

In December 2017, the administration reduced enforcement of penalties
against nursing homes that harm residents.

In February 2018, the Centers for Disease Control and Prevention (CDC)
announced that it would cut 80\% of its efforts to stop
infectious-disease epidemics worldwide due to budget cuts.

As a candidate and throughout his presidency, Trump said he would cut
the costs of pharmaceuticals. During his first seven months in office,
there were 96 price hikes for every drug price cut. In May 2018, Trump
announced that he would not allow Medicare to use its bargaining power
to negotiate lower drug prices from pharmaceutical companies, abandoning
a promise he made as candidate. Shortly after the 2018 mid-term
elections, the large pharmaceutical company Pfizer announced a price
increase for dozens of drugs, after it had reportedly bowed to pressure
not to do so earlier by the Trump administration.

The administration's proposed March 2019 budget called for substantial
spending cuts to Medicare, Medicaid and Social Security Disability
Insurance. Trump had previously vowed to protect Medicare and Medicaid.

In March 2019, the Justice Department changed its position on the ACA
(previously arguing that the individual mandate provision was
unconstitutional, but could be severed from the whole healthcare law) to
argue that the entire law is unconstitutional. In May 2019, the
department conducted a filing with the United States Court of Appeals
for the Fifth Circuit to nullify the entire ACA, arguing that the
removal of the provision on individual mandate results in the entire law
becoming unconstitutional. As of that day, President Donald Trump has
promised to produce a replacement health insurance plan only after he
wins reelection in 2020.

\includegraphics[width=5.50000in,height=2.06250in]{media/image15.jpg}\\
\emph{Donald Trump at the 15th Annual Opioid Takeback Day}

\section{Opioid epidemic}\label{opioid-epidemic}

\begin{itemize}
\item
  \emph{In September 2017, Trump nominated Tom Marino to lead the Office
  of National Drug Control Policy and become the nation's drug czar.}
\item
  \emph{In January 2018, The Washington Post reported that one of the
  top officials at the Office of National Drug Control Policy, which is
  tasked with multibillion-dollar anti-drug initiatives and curbing the
  opioid epidemic, was a 24-year old campaign staffer from the Trump
  2016 campaign who lied on his CV and whose stepfather went to jail for
  manufacturing illegal drugs; after the administration was contacted
  about the official's qualifications and CV, the administration gave
  him a job with different tasks in the ONDCP.}
\end{itemize}

In September 2017, Trump nominated Tom Marino to lead the Office of
National Drug Control Policy and become the nation's drug czar. In
October 2017, Marino withdrew his name from consideration after a joint
Washington Post and 60 Minutes investigation found that Marino had been
the chief architect of a bill that crippled the enforcement powers of
the DEA and worsened the opioid crisis in the United States.

In November 2017, it was announced that Kellyanne Conway would lead
White House efforts to combat the opioid epidemic; Conway had no
experience or expertise on matters of public health, substance abuse, or
law enforcement. Conway sidelined drug experts and opted instead for the
use of political staff. In February 2018, Politico wrote that the
administration's "main response" to the opioid crisis had "so far has
been to call for a border wall and to promise a "just say no" campaign."

In October 2017, the administration declared a 90-day public health
emergency over the opioid epidemic and pledged to urgently mobilize the
federal government in response to the crisis. On January 11, 2018, 12
days before the declaration ran out, Politico noted that "beyond drawing
more attention to the crisis, virtually nothing of consequence has been
done." The administration had not proposed any new resources or
spending, had not started the promised advertising campaign to spread
awareness about addiction, and had yet to fill key public health and
drug positions in the administration. In January 2018, The Washington
Post reported that one of the top officials at the Office of National
Drug Control Policy, which is tasked with multibillion-dollar anti-drug
initiatives and curbing the opioid epidemic, was a 24-year old campaign
staffer from the Trump 2016 campaign who lied on his CV and whose
stepfather went to jail for manufacturing illegal drugs; after the
administration was contacted about the official's qualifications and CV,
the administration gave him a job with different tasks in the ONDCP.

\includegraphics[width=5.50000in,height=3.66016in]{media/image16.jpg}\\
\emph{United States Secretary of Housing and Urban Development, Ben
Carson, on the first day of the job.}

\section{Housing and urban policy}\label{housing-and-urban-policy}

\begin{itemize}
\item
  \emph{In March 2018, HUD removed the words "inclusive" and "free from
  discrimination" from its mission statement.}
\item
  \emph{In June 2017, the administration designated Lynne Patton, an
  event planner who had worked on the Trump campaign and planned Eric
  Trump's wedding, to lead HUD's New York and New Jersey office (which
  oversees billions of federal dollars).}
\end{itemize}

In December 2017, The Economist described the Department of Housing and
Urban Development (HUD), led by Ben Carson, as "directionless". Most of
the top HUD positions were unfilled and Carson's leadership was
"inconspicuous and inscrutable". Of the policies that HUD was enacting,
The Economist wrote, "it is hard not to conclude that the governing
principle at HUD is to take whatever the Obama administration was doing,
and do the opposite." Under Carson's tenure, HUD scaled back the
enforcement of fair housing laws, and halted several fair housing
investigations started by the Obama administration. In March 2018, HUD
removed the words "inclusive" and "free from discrimination" from its
mission statement.

In June 2017, the administration designated Lynne Patton, an event
planner who had worked on the Trump campaign and planned Eric Trump's
wedding, to lead HUD's New York and New Jersey office (which oversees
billions of federal dollars).

\includegraphics[width=5.50000in,height=3.66016in]{media/image17.jpg}\\
\emph{President Trump signs the Hurricane Harvey relief bill at Camp
David, September 8, 2017}

\section{Disaster relief}\label{disaster-relief}

\section{Hurricanes Harvey and Irma}\label{hurricanes-harvey-and-irma}

\begin{itemize}
\item
  \emph{The next day, Trump visited Corpus Christi near where Harvey
  made landfall, and the Austin, Texas Emergency Operations Center.}
\item
  \emph{Trump visited the impacted area and reviewed relief efforts,
  promising full financial backing for the state's recovery.}
\item
  \emph{On September 8, Trump signed into law H.R.}
\end{itemize}

On August 28, 2017, the Category 4 Hurricane Harvey made landfall in
southeastern Texas, caused 40-60 inches of rainfall and massive flooding
in the Houston area. The next day, Trump visited Corpus Christi near
where Harvey made landfall, and the Austin, Texas Emergency Operations
Center. On September 8, Trump signed into law H.R. 601, which among
other spending actions designated \$15 billion for Hurricane Harvey
relief. Two weeks later, on September 10, Category 4 Hurricane Irma
struck the southwestern tip of Florida and then moved up Florida Gulf
coast causing extensive damage and prolonged power outages. Trump
visited the impacted area and reviewed relief efforts, promising full
financial backing for the state's recovery.

\section{Hurricane Maria}\label{hurricane-maria}

\begin{itemize}
\item
  \emph{A March 2018 Politico analysis of the administration's response
  indicated that the administration and Trump himself showed far more
  attention to Hurricane Harvey in Texas and that the response to
  Hurricane Maria in Puerto Rico was slower and weaker.}
\item
  \emph{The Trump administration came under criticism for its delayed
  response to the humanitarian crisis on the island.}
\item
  \emph{Trump did not comment on Puerto Rico for several days while the
  crisis was unfolding.}
\end{itemize}

On September 20, 2017, Puerto Rico was struck by Category 4 Hurricane
Maria, causing widespread devastation, knocking out the power system and
phone towers, destroying buildings, and causing widespread flooding. The
Trump administration came under criticism for its delayed response to
the humanitarian crisis on the island. Politicians of both parties had
called for immediate aid for Puerto Rico, and criticized Trump for
focusing on a feud with the NFL instead. Trump did not comment on Puerto
Rico for several days while the crisis was unfolding. According to The
Washington Post, the White House did not feel a sense of urgency until
"images of the utter destruction and desperation --- and criticism of
the administration's response --- began to appear on television." Trump
later dismissed the criticism, saying he was "very proud" of an
"amazing" response and that efforts to distribute necessary supplies and
services were "doing well". The Washington Post noted, "on the ground in
Puerto Rico, nothing could be further from the truth." Carmen Yulín
Cruz, the mayor of Puerto Rico's capital San Juan, repeatedly criticized
US relief efforts, saying that they were not reaching the people who
needed the aid; on September 29 she made a desperate plea for help,
saying that people are "dying, starving, thirsty". Trump responded by
criticizing Puerto Rico officials, saying that they had "poor leadership
ability" and "want everything to be done for them", and repeatedly
pointing out Puerto Rico's debt crisis. On September 28 the Army
dispatched Lt. Gen. Jeffrey S. Buchanan to Puerto Rico to assess the
situation and see how the military could be more effective in helping.

In January 2018, FEMA officially ended its humanitarian mission in
Puerto Rico. At the time of FEMA's departure, one third of Puerto Rico
residents still lacked electricity and some places lacked running water.
A March 2018 Politico analysis of the administration's response
indicated that the administration and Trump himself showed far more
attention to Hurricane Harvey in Texas and that the response to
Hurricane Maria in Puerto Rico was slower and weaker. An academic study
by the New England Journal of Medicine estimated that the number of
hurricane-related deaths during the period September 20, 2017 to
December 31, 2017 was around 4,600 (range 793-8,498) The official death
rate due to Maria reported by the Commonwealth of Puerto Rico is 2,975;
the figure was based on an independent investigation by George
Washington University commissioned by the governor of Puerto Rico. Trump
falsely claimed that the official death rate was wrong, and said that
the Democrats were trying to make him "look as bad as possible".

\section{California wildfires}\label{california-wildfires}

\begin{itemize}
\item
  \emph{In November 2018, while California was experiencing was one of
  its most destructive wildfires, Trump blamed the fires on "gross" and
  "poor" "mismanagement" of forests by California, saying that there was
  no other reason for these wildfires.}
\item
  \emph{The New York Times described Trump's claims as misleading,
  noting that the fires in question were not "forest fires", that most
  of the forest was owned by federal agencies, and that climate change
  in part contributed to the fires.}
\end{itemize}

In November 2018, while California was experiencing was one of its most
destructive wildfires, Trump blamed the fires on "gross" and "poor"
"mismanagement" of forests by California, saying that there was no other
reason for these wildfires. The New York Times described Trump's claims
as misleading, noting that the fires in question were not "forest
fires", that most of the forest was owned by federal agencies, and that
climate change in part contributed to the fires.

\includegraphics[width=3.74000in,height=5.50000in]{media/image18.jpg}\\
\emph{Kevin McAleenan is the acting Secretary of the Department of
Homeland Security}

\section{Immigration}\label{immigration}

\begin{itemize}
\item
  \emph{In January 2017, Trump signed an executive order directing the
  DHS Secretary to begin work on a wall.}
\item
  \emph{By February 2018, arrests of undocumented immigrants by ICE
  increased by 40\% during Trump's tenure.}
\item
  \emph{In January 2018, Trump was widely criticized after referring to
  Haiti, El Salvador, and African nations in general as "shithole
  countries" at a bipartisan meeting on immigration.}
\end{itemize}

Trump has repeatedly characterized illegal immigrants as criminals,
although multiple studies have found they have lower crime and
incarceration rates than native-born Americans. Prior to taking office,
Trump promised to deport the 11 million illegal immigrants living in the
United States and to build a wall along the Mexico--United States
border. In January 2017, Trump signed an executive order directing the
DHS Secretary to begin work on a wall. An internal DHS report estimated
that Trump's wall would cost \$21.6 billion and take 3.5 years to build
(far higher than the Trump 2016 campaign's estimate (\$12 billion) and
the \$15 billion estimate from Republican congressional leaders). Other
analyses estimated a total cost of up to \$25 billion, with the cost of
private land acquisitions and fence maintenance pushing the total cost
up further.

In August 2017, the transcript of the January 2017 phone call between
Trump and Mexican president Enrique Peña Nieto was leaked; in the phone
call, Trump conceded that he would fund the border wall, not by charging
Mexico as he promised during the campaign, and implored the Mexican
president to stop saying publicly that the Mexican government would not
pay for the border wall. In January 2018, the administration proposed
spending \$18 billion over the next 10 years on the wall, more than half
of the \$33 billion spending blueprint for border security. Trump's plan
would reduce funding for border surveillance, radar technology, patrol
boats and customs agents; experts and officials say that these are more
effective at curbing illegal immigration and preventing terrorism and
smuggling than a border wall.

In February 2017, Trump stated, "According to data provided by the
Department of Justice, the vast majority of individuals convicted of
terrorism and terrorism-related offenses since 9/11 came here from
outside of our country." Fact-checkers found that most of those
convictions were for cases of terrorism that did not occur in America
but were prosecuted in America, and many other cases involved
non-violent offenses like fraud or immigration violations. Moreover, an
April 2017 analysis by the Government Accountability Office found that
between September 12, 2001 and December 31, 2016, 73\% of violent
extremist incidents resulting in deaths were perpetrated by far right
wing violent extremist groups, while 27\% were perpetrated by radical
Islamist violent extremists.

The administration embraced the Reforming American Immigration for a
Strong Economy (RAISE) Act in August 2017. The RAISE Act sought to
reduce legal immigration levels to the U.S. by 50\% by halving the
number of green cards issued, capping refugee admissions at 50,000 a
year and ending the visa diversity lottery.

In August 2017, the administration terminated a program that granted
temporary legal residence to unaccompanied Central American minors.
2,714 individuals would need to renew their legal residence status
through other more difficult immigrant channels. In November 2017, the
Temporary Protected Status (TPS) granted to 60,000 Haitians following
the 2010 Haiti earthquake was revoked. In January 2018, the
administration announced that approximately 200,000 Salvadorans, who
were given Temporary Protected Status in the U.S. after a series of
devastating earthquakes in 2001, would have their residency permits
revoked; which means that they will have to leave the country, seek new
permits or stay as undocumented immigrants. The Salvadorans are parents
to an estimated 190,000 U.S.-born children. In October 2018, a federal
judge blocked the administration's attempt to deport the TPS recipients,
citing what the judge said was Trump's racial "animus against non-white,
non-European immigrants."

An analysis released by Trump's Department of Health and Human Services
in September 2017 was found to have removed earlier findings that
refugees entering America had a \$63 billion net positive effect on tax
revenues between 2005 and 2014, with the final report counting only the
costs that refugees incur. In July 2018, Sessions rescinded a DOJ
guidance on refugees and asylum seekers' right to work, thus prohibiting
them from working in the United States.

In October 2017, Secretary of Defense Mattis added additional background
checks for non-citizens who served in the military and extended the time
that the service members had to serve before they could receive
necessary paperwork to pursue US citizenship. As a result of these
changes, the number of service members who applied for citizenship
through their service declined by 65\% in the first quarter of fiscal
year 2018.

In December 2017, the administration announced that it would make it
illegal for spouses of H-1B visa holders to work in the United States.

In January 2018, Trump was widely criticized after referring to Haiti,
El Salvador, and African nations in general as "shithole countries" at a
bipartisan meeting on immigration. Multiple international leaders
condemned his remarks as racist.

By February 2018, arrests of undocumented immigrants by ICE increased by
40\% during Trump's tenure. Arrests of noncriminal undocumented
immigrants were twice as high as during Obama's final year in office.
Arrests of undocumented immigrants with criminal convictions only
increased slightly.

In March 2018, the Commerce Department announced that it would add a
citizenship question to the 2020 census. Experts noted that the
inclusion of such a question would likely result in severe undercounting
of the population and faulty data, as undocumented immigrants would be
less likely to respond to the census. Blue states, especially
California, are therefore likely to get less congressional apportionment
and fund apportionment than they would otherwise get, because they have
larger undocumented populations. In response, Xavier Becerra,
California's attorney general, announced his attention to sue the
administration over the decision. Similar suits were filed in New York,
Washington D.C., and several cities. The American Civil Liberties Union
(ACLU) and immigrants' rights organizations sued in June 2018. Federal
District Court judge Jesse Furman blocked the administration plan on
January 15, 2019. Documents released in May 2019 showed that Thomas B.
Hofeller, an architect of Republican gerrymandering, had found that
adding the census question would help to gerrymander maps that "would be
advantageous to Republicans and non-Hispanic whites." Hofeller later
wrote the DOJ letter which justified the policy by claiming it was
needed to enforce the 1965 Voting Rights Act.

In July 2018, experts noted that due to the administration's strict
border security policy, there was an increase in criminality and
lawlessness along the US-Mexico border. There was a strengthening of
ties between human smugglers, organized crime and corrupt local law
enforcement along the US-Mexico border, and that organized crime groups
were preying on asylum seekers who were prevented by US authorities from
filing for asylum.

During the 2018 mid-term election campaign, Trump sent nearly 5,600
troops to the U.S.-Mexico border for the stated purpose of protecting
the United States against a caravan of Central American migrants. The
Pentagon had previously concluded that the caravan posed no threat to
the U.S. The border deployment was estimated to cost as much as \$220
million by the end of the year. With daily warnings from Trump about the
dangers of the caravan during the mid-terms, the frequency and intensity
of the caravan rhetoric nearly stopped after election day.

During his February 2019 State of the Union address, Trump asserted that
the border city of El Paso, Texas once had one of the highest violent
crime rates in the nation but became one of America's safest cities
after a border barrier was built there in 2009. He repeated a variation
of that assertion at a rally in El Paso days later. Those assertions
were false: El Paso never had one of the country's highest violent crime
rates, its rate had long been lower than the average of comparably-sized
cities, and the rate had been declining steadily since 1993. Also at
that rally, large banners appeared over the stage declaring "Finish the
Wall," in contrast to the previous slogan of "Build the Wall." The
following day, Trump stated "we're building a lot of wall" and tweeted
"it is being built as we speak!" No new barrier construction had yet
begun during his presidency. Starting in February 2019 14 miles of new
barriers are slated to be built.

\section{Family separation policy}\label{family-separation-policy}

\begin{itemize}
\item
  \emph{On June 20, 2018, amid worldwide outrage and enormous political
  pressure to roll back his policy, Trump signed an executive order to
  end family separations at the U.S. border, unilaterally reversing his
  policy.}
\item
  \emph{Later that month, Trump falsely accused Democrats of creating
  that policy, despite it originating from his own administration, and
  urged Congress to "get together" and pass an immigration bill.}
\end{itemize}

In May 2018, the administration announced it would separate children
from parents caught unlawfully crossing the southern border into the
United States. Parents were routinely charged with a misdemeanor and
jailed; their children were placed in separate detention centers with no
established procedure to track them or reunite them with their parent
after they had served time for their offence, generally only a few hours
or days. Later that month, Trump falsely accused Democrats of creating
that policy, despite it originating from his own administration, and
urged Congress to "get together" and pass an immigration bill. Members
of Congress from both parties condemned the practice and pointed out
that the White House could end the separations on its own; Republican
Senator Lindsey Graham said, "President Trump could stop this policy
with a phone call." The Washington Post quoted a White House official as
saying that Trump's decision to separate migrant families was to gain
political leverage to force Democrats and moderate Republicans to accept
hardline immigration legislation.

Six weeks into the implementation of the "zero tolerance" policy, at
least 2,300 migrant children had been separated from their families. The
American Academy of Pediatrics, the American College of Physicians and
the American Psychiatric Association condemned the policy, with the
American Academy of Pediatrics saying that the policy was causing
"irreparable harm" to the children. The policy was extremely unpopular,
more so than any major piece of legislation in recent memory. Images of
children held in cage-like detention centers, interviews of sobbing
mothers who had no idea where their children were and had not heard from
them for weeks and months, and an audio of sobbing children resulted in
an outrage calling the practice "inhumane," "cruel" and "evil." On June
30, a national protest\\
drew hundreds of thousands of protesters from all 50 states to
demonstrate in more than 600 towns and cities. All four living former
First Ladies of the United States---Rosalynn Carter, Hillary Clinton,
Laura Bush, and Michelle Obama---condemned the policy of separating
children from their parents. Amidst the growing outrage, DHS secretary
Kirstjen Nielsen falsely claimed on June 17, "We do not have a policy of
separating families at the border. Period."

On June 20, 2018, amid worldwide outrage and enormous political pressure
to roll back his policy, Trump signed an executive order to end family
separations at the U.S. border, unilaterally reversing his policy. He
had earlier said that "you can't do it through an executive order." As
the result of a class-action lawsuit filed by the American Civil
Liberties Union, on June 26, U.S. District Judge Dana Sabraw issued a
nationwide preliminary injunction against the family-separation policy.
In his opinion, Sabraw wrote that the federal government "readily keeps
track of personal property of detainees in criminal and immigration
proceedings", yet "has no system in place to keep track of, provide
effective communication with, and promptly produce alien children." The
injunction required the government to reunite separated families within
30 days except where not appropriate.

On July 26, the administration said that 1,442 children had been
reunited with their parents while 711 remain in government shelters
because their cases are still under review, their parents have criminal
records, or they are no longer in the United States. Administration
officials state that 431 parents of those children have already been
deported without their children. Officials said they will work with the
court to return the remaining children, including the children whose
parents have been deported.

\includegraphics[width=5.50000in,height=4.12500in]{media/image19.jpg}\\
\emph{Trump signing Executive Order 13769 at the Pentagon as Vice
President Mike Pence and Secretary of Defense James Mattis look on,
January 27, 2017}

\section{Immigration order}\label{immigration-order}

\begin{itemize}
\item
  \emph{On September 24, 2017, Trump signed a proclamation that placed
  limits on the six countries in the second executive order and added
  Chad, North Korea, and Venezuela.}
\item
  \emph{During his first nine months in office, Trump issued several
  directives aimed at restricting entry of certain people into the
  United States.}
\end{itemize}

During his first nine months in office, Trump issued several directives
aimed at restricting entry of certain people into the United States.
Each directive was challenged in court.

On January 27, 2017, Trump signed an executive order which indefinitely
suspended admission of asylum seekers fleeing the Syrian Civil War,
suspended admission of all other refugees for 120 days, and denied entry
to citizens of Iraq, Iran, Libya, Somalia, Sudan, Syria and Yemen for 90
days. The order also established a religious test for refugees from
Muslim nations by giving priority to refugees of other religions over
Muslim refugees. Later, the administration seemed to reverse a portion
of part of the order, effectively exempting visitors with a green card.
After the order was challenged in the federal courts, several federal
judges issued rulings enjoining the government from enforcing the order.
On January 30, Trump fired acting Attorney General Sally Yates after she
stated she would not defend the order in court; Yates was replaced by
Dana Boente, who stated the DOJ would defend the order.

A new executive order was signed in March which places limits on travel
to the U.S. from six different countries for 90 days, and by all
refugees who do not possess either a visa or valid travel documents for
120 days. The new executive order revoked and replaced the former
Executive Order 13769 issued in January.

On June 26, the Supreme Court partially stayed certain injunctions that
were put on the order by two federal appeals courts earlier, allowing
the executive order to mostly go into effect. On October 10 the Court
dismissed the case, saying that the orders had been replaced by a new
proclamation, so challenges to the previous executive orders are moot.

On September 24, 2017, Trump signed a proclamation that placed limits on
the six countries in the second executive order and added Chad, North
Korea, and Venezuela. On October 17, 2017, Judge Derrick Watson, of the
United States District Court for the District of Hawaii issued another
temporary restraining order in response to a petition by the state of
Hawaii. On December 4, 2017, the Supreme Court allowed the September
2017 travel restrictions to go into effect while legal challenges in
Hawaii and Maryland are heard. The decision effectively bars most
citizens of Iran, Libya, Syria, Yemen, Somalia, Chad and North Korea
from entry into the United States along with some government officials
from Venezuela and their families.

\section{2018--2019 federal government
shutdown}\label{federal-government-shutdown}

\begin{itemize}
\item
  \emph{On December 22, 2018 the federal government was partially shut
  down after Trump demanded \$5.7~billion in federal funds for a
  U.S.--Mexico border wall to partly fulfill his campaign promise.}
\item
  \emph{In negotiations with Democrat leaders leading up to the
  shutdown, Trump commented that he would be "proud to shut down the
  government for border security."}
\end{itemize}

On December 22, 2018 the federal government was partially shut down
after Trump demanded \$5.7~billion in federal funds for a U.S.--Mexico
border wall to partly fulfill his campaign promise. The House and Senate
lacked votes necessary to support his funding demand and to overcome
Trump's refusal to sign the appropriations last passed by Congress into
law. In negotiations with Democrat leaders leading up to the shutdown,
Trump commented that he would be "proud to shut down the government for
border security." The shutdown, which was ended on January 25, 2019, was
the longest shutdown in U.S. history, and at over one month in duration
it has surpassed the previously longest shutdown of the 21-day shutdown
of 1995--96. By mid-January 2019, the White House Council of Economic
Advisors estimated that each week of the shutdown reduced GDP by 0.1
percentage points, the equivalent of 1.2 points per quarter.

\section{Reorganization of Department of Homeland
Security}\label{reorganization-of-department-of-homeland-security}

\begin{itemize}
\item
  \emph{The reorganization was reported to be on the recommendation of
  Trump advisor Stephen Miller, an anti-immigration hardliner.}
\item
  \emph{Trump denied reports he planned to renew and expand his family
  separation policy, asserting "President Obama had child separation.}
\item
  \emph{Saying he wanted to go in a "tougher direction," Trump began a
  major reorganization of the DHS on April 5, 2019, first by withdrawing
  his nomination of Ron Vitiello to head Immigration and Customs
  Enforcement.}
\end{itemize}

Saying he wanted to go in a "tougher direction," Trump began a major
reorganization of the DHS on April 5, 2019, first by withdrawing his
nomination of Ron Vitiello to head Immigration and Customs Enforcement.
Two days later he forced the resignation of DHS secretary Kirstjen
Nielsen. Trump named Customs and Border Protection commissioner Kevin
McAleenan to succeed Nielsen, although by law Under Secretary of
Homeland Security for Management Claire Grady was in line to succeed
Nielsen; Grady was reported to be leaving the administration. Also
leaving were Director of Citizenship and Immigration Lee Cissna and DHS
general counsel John Mitnick. The reorganization was reported to be on
the recommendation of Trump advisor Stephen Miller, an anti-immigration
hardliner. CNN reported that during March 2019 meetings Trump demanded
that asylum seekers be denied entry into the country, which he was
advised was contrary to law and could expose border agents to personal
legal liability. He also demanded that the port of El Paso be closed by
noon the next day. CNN quoted a senior administration official as
saying, "At the end of the day, the President refuses to understand that
the Department of Homeland Security is constrained by the laws." Trump
denied reports he planned to renew and expand his family separation
policy, asserting "President Obama had child separation. Take a look ---
the press knows it, you know it, we all know it. I'm the one that
stopped it." In contrast to the Trump systematic family separation
policy to deter migrants from entering the US, the Obama separation
policy was used only in instances when the child's safety was in
question or the adult had a prior criminal conviction.

\section{LGBT rights}\label{lgbt-rights}

\begin{itemize}
\item
  \emph{Asked about the administration's campaign, Trump appeared to be
  unaware of it.}
\item
  \emph{In July 2017, Trump said he would not allow "transgender
  individuals to serve in any capacity in the U.S. Military," citing
  disruptions and medical costs.}
\item
  \emph{In March 2018, President Trump signed Presidential Memorandum
  for the Secretary of Defense and the Secretary of Homeland Security
  Regarding Military Service by Transgender Individuals.}
\end{itemize}

In February 2017, the United States Departments of Justice and Education
rescinded a federal bathroom policy for transgender students. In March
2017, the DOJ declined to appeal a nationwide preliminary court
injunction temporarily halting enforcement of the Affordable Care Act's
nondiscrimination protections on the basis of gender identity. That same
month, the DOJ abandoned its request for a preliminary injunction
against North Carolina's bathroom law. That same month, HUD announced it
would withdraw purposed agency policies designed to protect LGBT
homelessness people. The same month, HHS announced that its national
survey of older adults would no longer collect information on LGBT
participants, and the Census Bureau announced it would remove "sexual
orientation" and "gender identity" as proposed subjects for possible
inclusion on the Decennial Census and/or American Community Survey in
the future.

In April 2017, the DOJ and the Labor Department cancelled quarterly
conference calls with LGBT organizations. In May 2017, HHS announced a
plan to roll back regulations interpreting the Affordable Care Act's
nondiscrimination provisions to protect gender identity status. In June
2017, the Department of Education withdrew its finding that an Ohio
school district discriminated against a transgender girl. In July 2017,
Trump said he would not allow "transgender individuals to serve in any
capacity in the U.S. Military," citing disruptions and medical costs.
That same day, the DOJ filed a legal brief on behalf of the United
States in the U.S. Court of Appeals for the Second Circuit, arguing that
the 1964 Civil Rights Act does not prohibit discrimination based on
sexual orientation or, implicitly, gender identity.

In September 2017, the DOJ filed a legal brief on behalf of the United
States in the Supreme Court, arguing for a constitutional right for
businesses to discriminate on the basis of sexual orientation and,
implicitly, gender identity. In October 2017, the DOJ reversed an
Obama-era policy explicitly that defines gender identity as protected
under employment discrimination laws due to what qualifies as employment
discrimination under Title VII of the Civil Rights Act. In October 2017,
the DOJ released a memo allowing federal agencies, government
contractors, government grantees, and private businesses to engage in
discrimination, as long as they can cite religious reasons for doing so.
In December 2017, Press Secretary Sarah Sanders responded that President
Trump agrees that it should be possible for a baker to put a sign in
their window saying, "We don't bake cakes for gay weddings.". In
December 2017, the staff at the Centers for Disease Control and
Prevention were, allegedly, instructed not to use the words
"transgender", "vulnerable," "entitlement," "diversity," "fetus,"
"evidence-based," and "science-based" in official documents. In December
2017, the Treasury Department imposed sanctions under the Magnitsky Act
on Chechen President Ramzan Kadyrov and a Chechen law enforcement
official, Ayub Katayev, citing "gross violations of internationally
recognized human rights" in connection with anti-gay purges in Chechnya.

In January 2018, HHS's Office of Civil Rights opened a "Conscience and
Religious Freedom Division" that will promote discrimination by health
care providers who can cite religious or moral reasons for denying care.
Also in January 2018, HHS proposed a rule that encourages medical
providers to use religious grounds to deny treatment to transgender
people. In February 2018, the United States Department of Education
announced it will dismiss complaints from transgender students involving
exclusion from school facilities and any other gender identity
discrimination claims. In March 2018, President Trump signed
Presidential Memorandum for the Secretary of Defense and the Secretary
of Homeland Security Regarding Military Service by Transgender
Individuals. If it takes effect, it would prohibit transgender persons,
whether transitioned or not, with a history or diagnosis of gender
dysphoria are disqualified from military service, except for individuals
who have had 36 consecutive months of stability "in their biological sex
prior to accession" and currently serving transgender persons in
military service.

In May 2018, the Bureau of Prisons in the DOJ adopted a policy of
housing almost all transgender people in federal prison facilities that
match their sex assigned at birth, rolling back existing protections for
transgender people in federal prisons. In August 2018, the United States
Department of Labor released a new directive for Office of Federal
Contract Compliance Programs staff encouraging religious exemptions to
federal contractors with religious-based objections to complying with
nondiscrimination laws. It also deleted material from an Office of
Federal Contract Compliance Programs FAQ on nondiscrimination
protections for sexual orientation and gender identity that previously
clarified the limited scope of allowable religious exemptions. In
November 2018, the United States Office of Personnel Management erased
guidance that helped federal agency managers understand how to support
transgender federal workers and their rights, replacing with a guidance
hostile to transgender workers.

In February 2019, the administration launched a global campaign to end
the criminalization of homosexuality, an initiative pushed by Richard
Grenell, the U.S. Ambassador to Germany. The campaign was spurred by
reports of the death of a gay man in Iran, one of the administration's
main geopolitical enemies. Asked about the administration's campaign,
Trump appeared to be unaware of it.

In May 2019, the administration proposed a rollback of regulations
prohibiting discrimination by health care providers against LGBTQ
patients.

Trump has nominated two LGBT persons to the federal judiciary (Mary M.
Rowland to the U.S. District Court for the Northern District of Illinois
and Patrick J. Bumatay to the U.S. Court of Appeals for the Ninth
Circuit). Other high-profile appointments of LGBT persons made by Trump
are Richard Grenell as ambassador to Germany, James T. Abbott as a
member of Federal Labor Relations Authority, and David Glawe as Under
Secretary of Homeland Security for Intelligence and Analysis.

\section{Science}\label{science}

\begin{itemize}
\item
  \emph{In June 2018, the administration began restricting the funding
  of research involving human fetal tissue.}
\item
  \emph{In December 2017, the administration sent a list to the Center
  for Disease Control (CDC) on words that the agency that was prohibited
  from using in its official communications.}
\item
  \emph{In March 2017, the United States Department of Energy prohibited
  the use of the term "climate change".}
\end{itemize}

The administration marginalized the role of science in policymaking. It
was the first administration since 1941 not to name a Science Advisor to
the President. While preparing for talks with Kim Jong-un, the White
House did not do so with the assistance of a White House science adviser
or senior counselor trained in nuclear physics. The position of chief
scientist in the State Department or the Department of Agriculture was
not filled. The administration nominated Sam Clovis to be chief
scientist in the United States Department of Agriculture, but he had no
scientific background and the White House later withdrew the nomination.
The administration successfully nominated Jim Bridenstine, who had no
background in science and rejected the scientific consensus on climate
change, to lead NASA. The United States Department of the Interior, the
National Oceanic and Atmospheric Administration, and the Food and Drug
Administration disbanded advisory committees.

In March 2017, the United States Department of Energy prohibited the use
of the term "climate change". In December 2017, the administration sent
a list to the Center for Disease Control (CDC) on words that the agency
that was prohibited from using in its official communications. These
terms included "transgender," "fetus," "evidence-based,"
"science-based," "vulnerable," "entitlement," and "diversity." The
Director of the CDC denied these reports.

In June 2018, the administration began restricting the funding of
research involving human fetal tissue. This immediately ended the
decades-long partnership between the National Institutes of Health and
the University of California, San Francisco, which have been researching
on HIV/AIDS disease.

\section{Veterans affairs}\label{veterans-affairs}

\begin{itemize}
\item
  \emph{During the 2018 mid-term election campaign, Trump repeatedly and
  falsely took credit for the Veterans Choice Program, saying it was his
  "greatest idea" and that it took 44 years for the law to pass.}
\item
  \emph{Prior to David Shulkin's firing in April 2018, The New York
  Times described the United States Department of Veterans Affairs (VA)
  as a "rare spot of calm in the Trump administration".}
\end{itemize}

Prior to David Shulkin's firing in April 2018, The New York Times
described the United States Department of Veterans Affairs (VA) as a
"rare spot of calm in the Trump administration". Shulkin built upon
changes started under the Obama administration to do a long-term
overhaul of the VA system. In May 2018, legislation to increase
veterans' access to private care was stalled, as was a VA overhaul which
sought to synchronize medical records. In May 2018, there were reports
of a large number of resignations of senior staffers and a major
re-shuffling.

In August 2018, ProPublica reported that a group of three wealthy
Mar-a-Lago patrons, who had no experience in the military or the
government, formed an "informal council" that strongly shaped VA
decision-making, including involving a \$10 billion contract to
modernize veterans' health records. The trio, which VA staff referred to
as "the Mar-a-Lago Crowd", spoke to VA staff daily, and provided
instructions on policy and personnel decisions at the agency. The
Government Accountability Office announced on November 19, 2018 that it
would investigate the matter.

During the 2018 mid-term election campaign, Trump repeatedly and falsely
took credit for the Veterans Choice Program, saying it was his "greatest
idea" and that it took 44 years for the law to pass. The law in question
was signed by President Obama in 2014.

\section{Voting rights}\label{voting-rights}

\begin{itemize}
\item
  \emph{Trump has repeatedly, without evidence, alleged that there is
  widespread voter fraud, and did so specifically for both the 2016
  elections and the 2018 mid-term elections.}
\item
  \emph{In January 2018, the Trump administration disbanded the
  commission, and informed Dunlap that it would not obey the court order
  to provide the documents because the commission no longer existed.}
\end{itemize}

Under the first 18 months of the administration, the DOJ "launched no
new efforts to roll back state restrictions on the ability to vote, and
instead often sides with them."

Trump has repeatedly, without evidence, alleged that there is widespread
voter fraud, and did so specifically for both the 2016 elections and the
2018 mid-term elections. In May 2017, the administration created the
Presidential Advisory Commission on Election Integrity (commonly
referred to as the Voter Fraud Commission), with the stated purpose to
review the extent of voter fraud. The commission was created in the wake
of Trump's false claim that millions of unauthorized votes cost him the
popular vote in the 2016 United States presidential election. It was
chaired by Vice President Mike Pence, while the vice chair and
day-to-day administrator was Kansas Secretary of State Kris Kobach, best
known for promoting restrictions on access to voting. The commission
began its work by requesting each state to turn over detailed
information about all registered voters in their database. Most states
rejected the request, citing privacy concerns or state laws.

Multiple lawsuits were filed against the commission. In November 2017,
Maine Secretary of State Matthew Dunlap, a Democratic member of the
commission, said that Kobach was refusing to share working documents and
scheduling information with him and the other Democrats on the
commission. He filed suit, and in December a federal judge ordered the
commission to hand over the documents. In January 2018, the Trump
administration disbanded the commission, and informed Dunlap that it
would not obey the court order to provide the documents because the
commission no longer existed. In the announcement disbanding the
commission, Trump blamed states for not handing over requested voter
information to the commission, while still maintaining that there was
"substantial evidence of voter fraud", an assertion which is contrary to
existing research and expert assessments, which have shown voter fraud
to be extremely rare. Election integrity experts argued that the
commission was disbanded because of the lawsuits, which would have led
to greater transparency and accountability in the commission and thus
prevented the Republican members of the commission from producing a sham
report to justify restrictions on voting rights. In January 2018, it was
revealed that the Commission had, in its requests for Texas voter data,
specifically asked for data that identifies voters with Hispanic
surnames.

\section{White nationalists and Charlottesville
rally}\label{white-nationalists-and-charlottesville-rally}

\begin{itemize}
\item
  \emph{The New York Times reported that Trump "was the only national
  political figure to spread blame for the 'hatred, bigotry and
  violence' that resulted in the death of one person to 'many sides'",
  and said that Trump had "buoyed the white nationalist movement on
  Tuesday as no president has done in generations".}
\end{itemize}

On August 13, 2017, Trump condemned violence "on many sides" after a
gathering of hundreds of white nationalists in Charlottesville,
Virginia, the previous day (August 12) turned deadly. A white
supremacist drove a car into a crowd of counter-protesters, killing one
woman and injuring 19 others. According to Sessions, that action met the
definition of domestic terrorism. During the rally there had been other
violence, as some counter-protesters charged at the white nationalists
with swinging clubs and mace, throwing bottles, rocks, and paint. Trump
did not expressly mention Neo-Nazis, white supremacists, or the
alt-right movement in his remarks on August 13, but the following day
(August 14) he did denounce white supremacists. He condemned "the KKK,
neo-Nazis, white supremacists, and other hate groups". Then the next day
(August 15), he again blamed "both sides".

Many Republican and Democratic elected officials condemned the violence
and hatred of white nationalists, neo-Nazis and alt-right activists.
Trump came under criticism from world leaders and politicians, as well
as a variety of religious groups and anti-hate organizations for his
remarks, which were seen as muted and equivocal. The New York Times
reported that Trump "was the only national political figure to spread
blame for the 'hatred, bigotry and violence' that resulted in the death
of one person to 'many sides'", and said that Trump had "buoyed the
white nationalist movement on Tuesday as no president has done in
generations". White nationalist groups felt "emboldened" after the rally
and planned additional demonstrations.

\includegraphics[width=5.50000in,height=3.66667in]{media/image20.jpg}\\
\emph{Trump and Vietnam's Communist Party leader Nguyễn Phú Trọng in
front of a statue of Ho Chi Minh in Hanoi, February 27, 2019}

\includegraphics[width=5.50000in,height=3.66667in]{media/image21.jpg}\\
\emph{On February 18, 2019 Trump appealed to the Venezuelan military to
back Venezuela's opposition leader Juan Guaidó.}

\section{Foreign policy}\label{foreign-policy}

\begin{itemize}
\item
  \emph{The New York Times reported on January 14, 2019 that on several
  occasions during 2018 Trump privately stated he wanted America to
  withdraw from NATO.}
\item
  \emph{On February 5, 2019 the Senate voted overwhelmingly to rebuke
  Trump for his decisions to withdraw troops from Syria and
  Afghanistan.}
\item
  \emph{Both assessments directly contradicted core tenets of Trump's
  stated foreign policy.}
\end{itemize}

The stated aims of the foreign policy of the Donald Trump administration
include a focus on security, by fighting terrorists abroad and
strengthening border defenses and immigration controls; an expansion of
the U.S. military; an "America First" approach to trade; and diplomacy
whereby "old enemies become friends". The foreign policy positions
expressed by Trump during his presidential campaign changed frequently,
so that it was "difficult to glean a political agenda, or even a set of
clear, core policy values ahead of his presidency."

Trump has repeatedly praised authoritarian strongmen such as China's
president Xi Jinping, Philippines president Rodrigo Duterte, Egyptian
president Abdel Fattah el-Sisi, Turkey's president Recep Tayyip Erdoğan,
King Salman of Saudi Arabia, Italy's prime minister Giuseppe Conte,
Brazil's president Jair Bolsonaro and Hungarian prime minister Viktor
Orban. Trump also praised Poland under the EU-skeptic, anti-immigrant
Law and Justice party (PiS) as a defender of Western civilization.

The New York Times reported on January 14, 2019 that on several
occasions during 2018 Trump privately stated he wanted America to
withdraw from NATO. Top defense and national security officials such as
Jim Mattis and John R. Bolton reportedly "scrambled to keep American
strategy on track without mention of a withdrawal that would drastically
reduce Washington's influence in Europe and could embolden Russia for
decades."

A January 2019 intelligence community assessment found that Iran was not
pursuing nuclear weapons and North Korea was unlikely to relinquish its
nuclear arsenal. Both assessments directly contradicted core tenets of
Trump's stated foreign policy. The intelligence community also assessed
that Trump's trade policies and unilateralism had damaged traditional
alliances and induced foreign partners to seek new relationships. The
day after the heads of the intelligence community presented their
findings in public testimony before the Senate Intelligence Committee,
Trump referred to them as "extremely passive and naive" and "wrong" in
their assessments. The following day, Trump asserted the press had
misquoted the intelligence chiefs' testimony to fabricate a conflict,
claiming he and the intelligence community were "on the same page!" In a
subsequent interview with The New York Times, Trump falsely asserted
that the intelligence community had characterized Iran as "essentially,
a wonderful place."

On February 5, 2019 the Senate voted overwhelmingly to rebuke Trump for
his decisions to withdraw troops from Syria and Afghanistan. Drafted by
majority leader Mitch McConnell, the measure was supported by nearly all
Republicans and was the second time in two months that the
Republican-controlled Senate had criticized the President's foreign
policy.

After initially adopting a verbally hostile posture toward North Korea
and its leader, Kim Jong-un, Trump quickly pivoted to embrace the
regime, stating that he and Kim "fell in love". Trump engaged Kim by
meeting him at two summits, in June 2018 and February 2019, an
unprecedented move by an American president, as previous policy had been
that a president simply meeting with the North Korean leader would
legitimize the regime on the world stage. During the June 2018 summit,
the leaders signed a vague agreement to pursue denuclearization of the
Korean peninsula, with Trump immediately declaring "There is no longer a
Nuclear Threat from North Korea." Little progress was made toward that
goal during the months before the February 2019 summit, which ended
abruptly without an agreement, hours after the White House announced a
signing ceremony was imminent. During the months between the summits, a
growing body of evidence indicated that North Korea was continuing its
nuclear fuel, bomb and missile development, including by redeveloping an
ICBM site it was previously appearing to dismantle --- even while the
second summit was underway. In the aftermath of the February 2019 failed
summit, the Treasury department imposed additional sanctions on North
Korea. The following day, Trump tweeted, "It was announced today by the
U.S. Treasury that additional large scale Sanctions would be added to
those already existing Sanctions on North Korea. I have today ordered
the withdrawal of those additional Sanctions!"

In April 2019 the Trump administration threatened to veto a United
Nations Security Council resolution condemning the weaponization of rape
in war zones unless all references to "sexual and reproductive health"
were removed, because in the administration's view it suggested support
for abortion.

\includegraphics[width=5.50000in,height=3.86598in]{media/image22.jpg}\\
\emph{Robert Mueller in the Oval Office }

\section{Russia and related
investigations}\label{russia-and-related-investigations}

\begin{itemize}
\item
  \emph{Trump said in 2017, "I can tell you, speaking for myself, I own
  nothing in Russia.}
\item
  \emph{The indictments were made before Trump's meeting with Putin in
  Helsinki, in which Trump supported Putin's denial that Russia was
  involved and criticized American law enforcement and intelligence
  community (subsequently Trump partially walked back some of his
  comments).}
\end{itemize}

American intelligence sources have stated with "high confidence" that
the Russian government attempted to intervene in the 2016 presidential
election to favor the election of Trump, and that members of Trump's
campaign were in contact with Russian government officials both before
and after the presidential election. In May 2017, the United States
Department of Justice appointed Robert Mueller as special counsel to
investigate "any links and/or coordination between Russian government
and individuals associated with the campaign of President Donald Trump,
and any matters that arose or may arise directly from the
investigation". Because of the Russian interference and subsequent
investigation, many members of Trump's administration have come under
special scrutiny regarding past ties to Russia or actions during the
campaign. Several of Trump's top advisers, including Paul Manafort and
Michael T. Flynn, who had official positions before Trump replaced them,
have strong ties to Russia. Several others had meetings with Russians
during the campaign which they did not initially disclose.

Trump himself hosted the 2013 Miss Universe pageant in Moscow, in
partnership with Russian-Azerbaijani billionaire Aras Agalarov. On many
occasions since 1987, Trump and his children and other associates have
traveled to Moscow to explore potential business opportunities, such as
a failed attempt to build a Trump Tower Moscow. Between 1996 and 2008
Trump's company submitted at least eight trademark applications for
potential real estate development deals in Russia. However, as of 2017
he has no known investments or businesses in Russia. Trump said in 2017,
"I can tell you, speaking for myself, I own nothing in Russia. I have no
loans in Russia. I don't have any deals in Russia." In 2008, his son
Donald Trump Jr. said "Russians make up a pretty disproportionate
cross-section of a lot of our assets" and "we see a lot of money pouring
in from Russia".

During his January 2017 confirmation hearings as the attorney general
nominee before the Senate, then-Senator Jeff Sessions (R-AL) was asked
by Senator Patrick J. Leahy (D-VT) if he had been "in contact with
anyone connected to any part of the Russian government about the 2016
election, either before or after election day?" Sessions' single word
response was "No", which raised questions about what appeared to be
deliberate omission of two meetings he had in 2016 with Russian
Ambassador Sergey Kislyak. Sessions later amended his testimony saying
he "never met with any Russian officials to discuss issues of the
campaign". He said that in March 2016, he had twice met with Ambassador
Kislyak, and "stood by his earlier remarks as an honest and correct
answer to a question". Officials with the DOJ stated that when Sessions
met with Kislyak, it was not as a Trump campaign surrogate, rather it
was "in his capacity as a member of the armed services panel". Following
his amended statement, Sessions recused himself from any investigation
regarding connections between Trump and Russia.

In May 2017, Donald Trump discussed highly classified intelligence in an
Oval Office meeting with the Russian foreign minister Sergey Lavrov and
ambassador Sergey Kislyak, providing details that could expose the
source of the information and the manner in which it was collected.\\
The intelligence was about an ISIS plot. A Middle Eastern ally provided
the intelligence which had the highest level of classification and was
not intended to be shared widely. The New York Times reported that "Mr.
Trump's disclosure does not appear to have been illegal - the president
has the power to declassify almost anything. But sharing the information
without the express permission of the ally who provided it was a major
breach of espionage etiquette, and could jeopardize a crucial
intelligence-sharing relationship". The White House, through National
Security Advisor H. R. McMaster, issued a limited denial, saying that
the story "as reported" was not correct, and stated that no
"intelligence sources or methods" were discussed. McMaster did not deny
that information had been disclosed. The following day Trump stated on
Twitter that Russia is an important ally against terrorism and that he
had an "absolute right" to share classified information with Russia.

In October 2017, former Trump campaign advisor George Papadopoulos
pleaded guilty to one count of making false statements to the FBI
regarding his contacts with Russian agents. During the campaign he had
tried repeatedly but unsuccessfully to set up meetings in Russia between
Trump campaign representatives and Russian officials. The guilty plea
was part of a plea bargain whereby Papadopoulos cooperates with the
Mueller investigation.

In February 2018, when Special Counsel Mueller indicted more than a
dozen Russians and three entities for interference in the 2016 election,
Trump's response was to assert that the indictment was proof that his
campaign did not collude with the Russians. The New York Times noted
that Trump "voiced no concern that a foreign power had been trying for
nearly four years to upend American democracy, much less resolve to stop
it from continuing to do so this year." A day after the indictment,
Trump used the FBI's alleged failure to stop the Stoneman Douglas High
School shooter to call for the end to investigations of Russian
interference in the 2016 presidential election.

In July 2018, the special counsel's office indicted 12 Russian
intelligence operatives and accused them of conspiring to interfere in
the 2016 US elections, by hacking servers and emails of the Democratic
Party and the Hillary Clinton campaign. The indictments were made before
Trump's meeting with Putin in Helsinki, in which Trump supported Putin's
denial that Russia was involved and criticized American law enforcement
and intelligence community (subsequently Trump partially walked back
some of his comments). A few days later, it was reported that Trump had
actually been briefed on the veracity and extent of Russian
cyber-attacks two weeks before his inauguration, back in December 2016,
including the fact that these were ordered by Putin himself. The
evidence presented to him at the time included text and email
conversations between Russian military officers as well as information
from a source close to Putin. According to the report, at the time, in
the classified meeting, Trump "sounded grudgingly convinced".

The Washington Post reported on January 12, 2019 that Trump had gone to
"extraordinary lengths" to keep details of his private conversations
with Russian president Putin secret, including in one case by retaining
his interpreter's notes and instructing the linguist to not share the
contents of the discussions with anyone in the administration. As a
result, there were no detailed records, even in classified files, of
Trump's conversations with Putin on five occasions. According to The
Financial Times, there were no American aides present when Trump met
privately with Putin at the 2018 G20 Buenos Aires summit in November
2018.

Of Trump's campaign advisors and staff, six of them were indicted by the
special counsel's office; five of them (Michael Cohen, Michael Flynn,
Rick Gates, Paul Manafort, George Papadopoulos) pleaded guilty, while
one has pleaded not guilty (Roger Stone).

\section{Special Counsel's report}\label{special-counsels-report}

\begin{itemize}
\item
  \emph{According to the report, the investigation "identified numerous
  links between the Russian government and the Trump campaign," and
  found that Russia "perceived it would benefit from a Trump presidency"
  and the 2016 Trump presidential campaign "expected it would benefit
  electorally" from Russian hacking efforts.}
\item
  \emph{The report "does not conclude that the president committed a
  crime", but specifically did not exonerate Trump on obstruction of
  justice, because investigators were not confident that Trump was
  innocent after examining his intent and actions.}
\end{itemize}

On March 22, 2019, Special Counsel Robert Mueller submitted the final
report to Attorney General William Barr. On March 24, 2019, Attorney
General Barr sent Congress a four-page letter, describing what he said
were the special counsel's principal conclusions in the Mueller Report.
Barr added that since that the special counsel "did not draw a
conclusion" on obstruction, this "leaves it to the Attorney General to
determine whether the conduct described in the report constitutes a
crime". Barr continued: "Deputy Attorney General Rod Rosenstein and I
have concluded that the evidence developed during the Special Counsel's
investigation is not sufficient to establish that the President
committed an obstruction-of-justice offense".

On April 18, 2019, a two-volume redacted version of the Special
Counsel's report titled Report on the Investigation into Russian
Interference in the 2016 Presidential Election, was released to Congress
and the public. About one-eighth of the lines in the public version were
redacted.

Volume I discusses about Russian interference in the 2016 presidential
election, concluding that interference occurred "in sweeping and
systematic fashion" and "violated U.S. criminal law." The report
detailed activities by the Internet Research Agency, a Kremlin-linked
Russian troll farm, to create a "social media campaign that favored
presidential candidate Donald J. Trump and disparaged presidential
candidate Hillary Clinton," and to "provoke and amplify political and
social discord in the United States". The report also described how the
Russian intelligence service, the GRU, performed computer hacking and
strategic releasing of damaging material from the Clinton campaign and
Democratic Party organizations. To establish whether a crime was
committed by members of the Trump campaign with regard to Russian
interference, investigators used the legal standard for criminal
conspiracy rather than the popular concept of "collusion", because a
crime of "collusion" is not found in criminal law or the United States
Code.

According to the report, the investigation "identified numerous links
between the Russian government and the Trump campaign," and found that
Russia "perceived it would benefit from a Trump presidency" and the 2016
Trump presidential campaign "expected it would benefit electorally" from
Russian hacking efforts. Ultimately, "the investigation did not
establish that members of the Trump campaign conspired or coordinated
with the Russian government in its election interference activities."
However, investigators had an incomplete picture of what had really
occurred during the 2016 campaign, due to some associates of Trump
campaign providing either false, incomplete or declined testimony
(exercising the Fifth Amendment), as well as having deleted, unsaved or
encrypted communications. As such, the Mueller Report "cannot rule out
the possibility" that information then unavailable to investigators
would have presented different findings.

Volume II covered obstruction of justice. The report described ten
episodes where Trump may have obstructed justice as president, plus one
instance before he was elected. The report said that in addition to
Trump's public attacks on the investigation and its subjects, Trump also
privately tried to "control the investigation" in multiple ways, but
mostly failed to influence it because his subordinates or associates
refused to carry out his instructions. For that reason, no charges
against the Trump's aides and associates were recommended "beyond those
already filed". The Special Counsel refrained from charging Trump
himself because investigators abided by an Office of Legal Counsel (OLC)
opinion that a sitting president cannot stand trial, and they feared
that charges would affect Trump's governing and possibly preempt his
impeachment. In addition, investigators felt it would be unfair to
accuse Trump of a crime without charges and without a trial in which he
could clear his name, hence investigators "determined not to apply an
approach that could potentially result in a judgment that the President
committed crimes."

Since the Special Counsel's office had decided "not to make a
traditional prosecutorial judgment" on whether to "initiate or decline a
prosecution," they "did not draw ultimate conclusions about the
President's conduct." The report "does not conclude that the president
committed a crime", but specifically did not exonerate Trump on
obstruction of justice, because investigators were not confident that
Trump was innocent after examining his intent and actions. The report
concluded "that Congress has authority to prohibit a President's corrupt
use of his authority in order to protect the integrity of the
administration of justice" and "that Congress may apply the obstruction
laws to the president's corrupt exercise of the powers of office accords
with our constitutional system of checks and balances and the principle
that no person is above the law".

On May 1, 2019, following publication of the Special Counsel's report,
Barr testified before the Senate Judiciary Committee. He declined to
testify before the House Judiciary Committee the following day because
he objected to the committee's plan to use staff lawyers during
questioning. Barr also repeatedly failed to give the unredacted Special
Counsel's report to the Judiciary Committee by its deadline of May 6,
2019. On May 8, 2019, the committee voted to hold Barr in contempt of
Congress, which refers the matter to entire House for resolution.
Concurrently, Trump asserted executive privilege via the Department of
Justice in an effort to prevent the redacted portions of the Special
Counsel's report and the underlying evidence from being disclosed.
Committee chairman Jerry Nadler stated that the U.S. is in a
constitutional crisis, "because the President is disobeying the law, is
refusing all information to Congress." Speaker Nancy Pelosi agreed with
Nadler's characterization and told fellow Democrats that Trump was
``self-impeaching'' by stonewalling Congress, with some Democrats and
analysts noting that refusing to comply with subpoenas was the third
article of impeachment for Richard Nixon.

Following release of the Mueller Report, Trump and his allies turned
their attention toward "investigating the investigators." On May 23,
2019, Trump ordered the intelligence community to cooperate with Barr's
investigation of the origins of the investigation, granting Barr full
authority to declassify any intelligence information related to the
matter. Some analysts expressed concerns that the order could create a
conflict between the Justice Department and the intelligence community
over closely-guarded intelligence sources and methods, as well as open
the possibility that Barr could cherrypick intelligence for public
release to help Trump.

Upon announcing the formal closure of the investigation and his
resignation from the Justice Department on May 29, 2019, Mueller stated,
``If we had had confidence that the president clearly did not commit a
crime, we would have said so. We did not, however, make a determination
as to whether the president did commit a crime.''

\section{Ethics}\label{ethics}

\begin{itemize}
\item
  \emph{In keeping with this pledge, Trump donated the entirety of his
  first two quarterly salaries as president to government agencies.}
\item
  \emph{Trump's transition team also announced that registered lobbyists
  would be barred from serving in the Trump administration.}
\item
  \emph{One of Trump's campaign promises was that he would not accept a
  presidential salary.}
\end{itemize}

During the 2016 campaign, Trump promised to "drain the swamp in
Washington D.C." - a phrase that usually refers to entrenched corruption
and lobbying in D.C. - and he proposed a series of ethics reforms.
However, according to federal records and interviews, there has been a
dramatic increase in lobbying by corporations and hired interests during
Trump's tenure, particularly through the office of the Vice-President
Mike Pence. About twice as many lobbying firms contacted Pence, compared
to previous presidencies, among them representatives of major energy
firms and drug companies. In many cases, the lobbyists have charged
their clients millions of dollars for access to the vice president, and
then have turned around and donated the money to Pence's political
causes.

Among his proposals was a five-year ban on serving as a lobbyist after
working in the executive branch. Trump's transition team also announced
that registered lobbyists would be barred from serving in the Trump
administration. However, an Obama era ban on lobbyists taking
administrative jobs was lifted and at least nine transition officials
became lobbyists within the first 100 days.

One of Trump's campaign promises was that he would not accept a
presidential salary. In keeping with this pledge, Trump donated the
entirety of his first two quarterly salaries as president to government
agencies.

\section{Potential conflicts of
interest}\label{potential-conflicts-of-interest}

\begin{itemize}
\item
  \emph{In February 2017, Trump senior advisor Kellyanne Conway promoted
  the clothing line of Ivanka Trump in a TV appearance from the White
  House briefing room.}
\item
  \emph{Trump placed his sons Eric Trump and Donald Trump Jr. at the
  head of his businesses claiming that they would not communicate with
  him regarding his interests.}
\end{itemize}

Trump's presidency has been marked by significant public concern about
conflict of interest stemming from his diverse business ventures. In the
lead up to his inauguration, Trump promised to remove himself from the
day-to-day operations of his businesses. Trump placed his sons Eric
Trump and Donald Trump Jr. at the head of his businesses claiming that
they would not communicate with him regarding his interests. However
critics noted that this would not prevent him from having input into his
businesses and knowing how to benefit himself, and Trump continued to
receive quarterly updates on his businesses. As his presidency
progressed, he failed to take steps or show interest in further
distancing himself from his business interests resulting in numerous
potential conflicts.

Many ethics experts found Trump's plan to address conflicts of interest
between his position as president and his private business interests to
be entirely inadequate; Norman L. Eisen and Richard Painter, who served
as the chief White House ethics lawyers for Barack Obama and George W.
Bush, respectively, stated that the plan "falls short in every respect".
Unlike every other president in the last 40 years, Trump did not put his
business interests in a blind trust or equivalent arrangement "to
cleanly sever himself from his business interests". Eisen stated that
Trump's case is "an even more problematic situation because he's
receiving foreign government payments and other benefits and things of
value that's expressly prohibited by the Constitution of the United
States" in the Foreign Emoluments Clause.

In January 2018, a year into his presidency, a survey found that he
"continues to own stakes in hundreds of businesses, both in this country
and abroad."

After Trump took office, the watchdog group Citizens for Responsibility
and Ethics in Washington, represented by a number of constitutional
scholars, sued him for violations of the Foreign Emoluments Clause (a
constitutional provision that bars the president or any other federal
official from taking gifts or payments from foreign governments),
because his hotels and other businesses accept payment from foreign
governments. CREW separately filed a complaint with the General Services
Administration (GSA) over Trump International Hotel Washington, D.C.;
the 2013 lease that Trump and the GSA signed "explicitly forbids any
elected government official from holding the lease or benefiting from
it". The GSA said that it was "reviewing the situation". By May 2017,
the CREW v. Trump lawsuit had grown with additional plaintiffs and
alleged violations of the Domestic Emoluments Clause. In June 2017,
attorneys from the Department of Justice filed a motion to dismiss the
case on the grounds that the plaintiffs had no right to sue and that the
described conduct was not illegal. Also in June 2017, two more lawsuits
were filed based on the Foreign Emoluments Clause: D.C. and Maryland v.
Trump, and Blumenthal v. Trump, which was signed by more than one-third
of the voting members of Congress. United States District Judge George
B. Daniels dismissed the CREW case on December 21, 2017, holding that
plaintiffs lacked standing. D.C. and Maryland v. Trump cleared three
judicial hurdles to proceed to the discovery phase during 2018, with
prosecutors issuing 38 subpoenas to Trump's businesses and cabinet
departments in December before the Fourth Circuit Court of Appeals
issued a stay days later at the behest of the Justice Department,
pending hearings in March 2019. NBC News reported that by June 2019
representatives of 22 governments had spent money at Trump properties.

In February 2017, Trump senior advisor Kellyanne Conway promoted the
clothing line of Ivanka Trump in a TV appearance from the White House
briefing room. Office of Government Ethics director Walter Shaub
requested disciplinary action in a letter to the White House Counsel's
office. Under federal ethics regulations, federal employees are barred
from using their public office to endorse products.

\section{Saudi Arabia}\label{saudi-arabia}

\begin{itemize}
\item
  \emph{Trump also praised his relationship with Saudi Arabia's powerful
  Crown Prince Mohammad bin Salman.}
\item
  \emph{Trump actively supported the Saudi Arabian-led intervention in
  Yemen against the Houthis.}
\item
  \emph{In October 2018, amid widespread condemnation of Saudi Arabia
  for the murder of prominent Saudi journalist and dissident Jamal
  Khashoggi, the Trump administration pushed back on the condemnation.}
\end{itemize}

In March 2018, The New York Times reported that George Nader turned
Trump's major fundraiser Elliott Broidy "into an instrument of influence
at the White House for the rulers of Saudi Arabia and the United Arab
Emirates...High on the agenda of the two men...was pushing the White
House to remove Secretary of State Rex W. Tillerson", a top defender of
the Iran nuclear deal in Donald Trump's administration, and "backing
confrontational approaches to Iran and Qatar".

Trump actively supported the Saudi Arabian-led intervention in Yemen
against the Houthis. Trump also praised his relationship with Saudi
Arabia's powerful Crown Prince Mohammad bin Salman.\\
On May 20, 2017, Trump and Saudi Arabia's King Salman bin Abdulaziz Al
Saud signed a series of letters of intent for the Kingdom of Saudi
Arabia to purchase arms from the United States totaling US\$110 billion
immediately, and \$350 billion over 10 years. The transfer was widely
seen as a counterbalance against the influence of Iran in the region and
a "significant" and "historic" expansion of United States relations with
Saudi Arabia.

In October 2018, amid widespread condemnation of Saudi Arabia for the
murder of prominent Saudi journalist and dissident Jamal Khashoggi, the
Trump administration pushed back on the condemnation. After the CIA
assessed that Saudi Crown Prince Mohammad bin Salman ordered the murder
of Khashoggi, Trump rejected the assessment and said that the CIA only
had "feelings" on the matter.

\section{Transparency and data
availability}\label{transparency-and-data-availability}

\begin{itemize}
\item
  \emph{Nathan Cortez of the Southern Methodist University's Dedman
  School of Law, who studies the handling of public data, said that the
  Trump administration, unlike the Obama administration, was taking
  transparency "in the opposite direction".}
\item
  \emph{The Trump administration stopped the Obama administration policy
  of logging visitors to the White House, making it difficult to tell
  who has visited the White House.}
\end{itemize}

The Washington Post reported in May 2017, "a wide variety of information
that until recently was provided to the public, limiting access, for
instance, to disclosures about workplace violations, energy efficiency,
and~animal welfare abuses" had been removed or tucked away. The Obama
administration had used the publication of enforcement actions taken by
federal agencies against companies as a way to name and shame companies
that engaged in unethical and illegal behaviors.

The Trump administration stopped the Obama administration policy of
logging visitors to the White House, making it difficult to tell who has
visited the White House. Nathan Cortez of the Southern Methodist
University's Dedman School of Law, who studies the handling of public
data, said that the Trump administration, unlike the Obama
administration, was taking transparency "in the opposite direction".

\section{Cost of trips}\label{cost-of-trips}

\begin{itemize}
\item
  \emph{When Obama was president, Trump frequently criticized him for
  taking vacations which were paid for with public funds.}
\item
  \emph{The Washington Post reported that Trump's atypically lavish
  lifestyle is far more expensive to the taxpayers than what was typical
  of former presidents and could end up in the hundreds of millions of
  dollars over the whole of Trump's term.}
\end{itemize}

According to several reports, Trump's and his family's trips in the
first month of his presidency cost the US taxpayers nearly as much as
former president Obama's travel expenses for an entire year. When Obama
was president, Trump frequently criticized him for taking vacations
which were paid for with public funds. The Washington Post reported that
Trump's atypically lavish lifestyle is far more expensive to the
taxpayers than what was typical of former presidents and could end up in
the hundreds of millions of dollars over the whole of Trump's term.

\section{Security clearances}\label{security-clearances}

\begin{itemize}
\item
  \emph{In March 2019, Tricia Newbold, a White House employee working on
  security clearances, spoke privately to the House Oversight Committee,
  claiming that at least 25 Trump administration officials were granted
  security clearances over the objections of career staffers.}
\item
  \emph{Kline eventually appeared before the committee on May 1.}
\end{itemize}

In March 2019, Tricia Newbold, a White House employee working on
security clearances, spoke privately to the House Oversight Committee,
claiming that at least 25 Trump administration officials were granted
security clearances over the objections of career staffers. Newbold also
asserted that some of these officials had previously had their
applications rejected for "disqualifying issues", only for those
rejections to be overturned with inadequate explanation.

The House Oversight Committee subpoenaed Carl Kline, the former head of
White House security clearances to testify on April 23. However White
House deputy counsel Michael Purpura instructed Kline not to appear at
the deposition, citing constitutional concerns. Kline eventually
appeared before the committee on May 1.

\section{Accepting political information from foreign
powers}\label{accepting-political-information-from-foreign-powers}

\begin{itemize}
\item
  \emph{Responding to a reporter who told him that FBI director
  Christopher Wray had stated that such activities should be reported to
  the FBI, Trump stated, "the FBI director is wrong."}
\item
  \emph{On June 12, 2019, Trump asserted he saw nothing wrong in
  accepting intelligence on his political adversaries from foreign
  powers, such as Russia, and he would see no reason to contact the FBI
  about it.}
\item
  \emph{Trump elaborated, "there's nothing wrong with listening.}
\end{itemize}

On June 12, 2019, Trump asserted he saw nothing wrong in accepting
intelligence on his political adversaries from foreign powers, such as
Russia, and he would see no reason to contact the FBI about it.
Responding to a reporter who told him that FBI director Christopher Wray
had stated that such activities should be reported to the FBI, Trump
stated, "the FBI director is wrong." Trump elaborated, "there's nothing
wrong with listening. If somebody called from a country, Norway, `we
have information on your opponent' --- oh, I think I'd want to hear it."
Both Democrats and Republicans repudiated the remarks.

\section{Elections during the Trump
presidency}\label{elections-during-the-trump-presidency}

\section{2018 mid-term elections}\label{mid-term-elections}

\begin{itemize}
\item
  \emph{In the 2018 mid-term elections, Democrats won control of the
  House of Representatives, while Republicans expanded their majority in
  the Senate.}
\end{itemize}

In the 2018 mid-term elections, Democrats won control of the House of
Representatives, while Republicans expanded their majority in the
Senate.

\section{Historical evaluations and public
opinion}\label{historical-evaluations-and-public-opinion}

\section{Popular polling}\label{popular-polling}

\begin{itemize}
\item
  \emph{Trump's approval rating during his first term has been
  "incredibly stable (and also historically low)" within a band from
  about 36\% to 44\%.}
\item
  \emph{By January 20, 2017, Inauguration Day, Trump's approval rating
  average was 42\%, the lowest rating average for an incoming president
  in the history of modern polling.}
\end{itemize}

At the time of the 2016 election, polls by Gallup found Trump had a
favorable rating around 35\% and an unfavorable rating around 60\%,
while Democratic nominee Hillary Clinton held a favorable rating of 40\%
and an unfavorable rating of 57\%. 2016 was the first election cycle in
modern presidential polling in which both major-party candidates were
viewed so unfavorably.

By January 20, 2017, Inauguration Day, Trump's approval rating average
was 42\%, the lowest rating average for an incoming president in the
history of modern polling. Trump's approval rating during his first term
has been "incredibly stable (and also historically low)" within a band
from about 36\% to 44\%.

\section{Historians and political
scientists}\label{historians-and-political-scientists}

\begin{itemize}
\item
  \emph{Political scientist Norman Ornstein states that many writers
  have tried to put Trump in perspective:}
\item
  \emph{Not all are strictly about Trump --- the fact is the conditions
  and dynamics that brought us Trump long preceded him, and the changes
  in the fabric of our Republic are paralleled by changes in other
  longstanding democracies around the globe.}
\end{itemize}

Political scientist Norman Ornstein states that many writers have tried
to put Trump in perspective:

Among President Trump's major accomplishments is the booming industry in
books about him, his administration, the state of democracy in America,
the rise of autocracy in America and abroad, the reasons for his rise,
the bases of his support, the state of the Republican Party, the state
of his mental health or lack thereof, the chaos in his White House and
so on.

Not all are strictly about Trump --- the fact is the conditions and
dynamics that brought us Trump long preceded him, and the changes in the
fabric of our Republic are paralleled by changes in other longstanding
democracies around the globe.

\section{Democratic backsliding}\label{democratic-backsliding}

\begin{itemize}
\item
  \emph{Since the beginning of the presidency of Donald Trump, ratings
  of U.S. democracy sharply plunged in the United States.}
\item
  \emph{According to the 2018 Varieties of Democracy Annual Democracy
  Report, there has been "a significant democratic backsliding in the
  United States {[}since the Inauguration of Donald Trump{]} ...
  attributable to weakening constraints on the executive."}
\end{itemize}

Since the beginning of the presidency of Donald Trump, ratings of U.S.
democracy sharply plunged in the United States.

According to the 2018 Varieties of Democracy Annual Democracy Report,
there has been "a significant democratic backsliding in the United
States {[}since the Inauguration of Donald Trump{]} ... attributable to
weakening constraints on the executive." Independent assessments by
Freedom House and Bright Line Watch found a similar significant decline
in overall democratic functioning.

\section{Historical rankings}\label{historical-rankings}

\begin{itemize}
\item
  \emph{Siena College Research Institute's 6th presidential expert poll,
  released in February 2019, placed Donald Trump 42nd out of 44th ---
  ahead of Andrew Johnson and James Buchanan.}
\item
  \emph{A 2018 poll administered by the American Political Science
  Association (APSA) among political scientists specializing in the
  American presidency ranked Donald Trump in last place.}
\end{itemize}

A 2018 poll administered by the American Political Science Association
(APSA) among political scientists specializing in the American
presidency ranked Donald Trump in last place. Republican survey
respondents rated him 40th out of 44th, Independents/Other respondents
rated him 43rd out of 44th, while Democratic historians rated him 44th
out of 44th. Siena College Research Institute's 6th presidential expert
poll, released in February 2019, placed Donald Trump 42nd out of 44th
--- ahead of Andrew Johnson and James Buchanan.

\section{See also}\label{see-also}

\section{Notes}\label{notes}

\begin{itemize}
\item
  \emph{\^{} A small portion of the 115th Congress (January 3, 2017 --
  January 19, 2017) took place under President Obama.}
\end{itemize}

\^{} A small portion of the 115th Congress (January 3, 2017 -- January
19, 2017) took place under President Obama.

\^{} In 1824, there were six states in which electors were legislatively
appointed, rather than popularly elected, so it is uncertain what the
national popular vote would have been if all presidential electors had
been popularly elected.

\section{References}\label{references}
