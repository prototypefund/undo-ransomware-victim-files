\textbf{From Wikipedia, the free encyclopedia}

https://en.wikipedia.org/wiki/American\%20Ethnologist\\
Licensed under CC BY-SA 3.0:\\
https://en.wikipedia.org/wiki/Wikipedia:Text\_of\_Creative\_Commons\_Attribution-ShareAlike\_3.0\_Unported\_License

\section{American Ethnological
Society}\label{american-ethnological-society}

\begin{itemize}
\item
  \emph{The American Ethnological Society (AES) is the oldest
  professional anthropological association in the United States.}
\end{itemize}

The American Ethnological Society (AES) is the oldest professional
anthropological association in the United States.

\section{History of the American Ethnological
Society}\label{history-of-the-american-ethnological-society}

\begin{itemize}
\item
  \emph{In the early 1980s the American Ethnological Society became
  incorporated into the American Anthropological Association as a
  sub-section.}
\item
  \emph{In 1916, the AES became the American Ethnological Society, Inc.
  During this time, it also became associated with Columbia University
  and linked to the American Anthropological Association.}
\end{itemize}

Albert Gallatin and John Russell Bartlett founded the American
Ethnological Society in New York City in 1842. Their goal was to promote
research in ethnology and all inquiries involving humans. The early
meetings of the AES took place in the homes of the members, where they
discussed all aspects of human life, from history and geography to
philology and anthropology. The AES was a scholarly institution, in
which papers were presented that were later published.

In the late 19th century, the AES's focus changed from the evolutionary
concerns of ethnology to the academic discipline of anthropology. The
AES remained small, due to financial difficulties until the 1920s. In
1916, the AES became the American Ethnological Society, Inc. During this
time, it also became associated with Columbia University and linked to
the American Anthropological Association. In the 1930s, the AES and AAA
jointly published the American Anthropologist, which concerned itself
with all four fields of anthropology.

In 1950, the AES went nationwide and started having biannual meetings
across North America. In 1972, the new American Ethnologist journal was
created to focus on the expanding field of socio-cultural anthropology.
In the early 1980s the American Ethnological Society became incorporated
into the American Anthropological Association as a sub-section.

Since 2003, the American Ethnological Society has awarded three awards
biennially. These are the Sharon Stephens Prize (for junior scholars)
and the Senior Book Prize (for senior scholars), which are each awarded
for a book "that speaks to contemporary social issues with relevance
beyond the discipline and beyond the academy," and the Elsie Clews
Parsons Prize, which is awarded to a graduate student for a stand-alone
paper based on an ethnography.

\section{American Anthropologist}\label{american-anthropologist}

\begin{itemize}
\item
  \emph{The American Anthropologist is the quarterly journal of the
  American Anthropological Association.}
\end{itemize}

The American Anthropologist is the quarterly journal of the American
Anthropological Association. The journal advances the Association's
mission through publishing articles that add to, integrate, synthesize,
and interpret anthropological knowledge; commentaries and essays on
issues of importance to the discipline; and reviews of books, films,
sound recordings and exhibits."

\section{American Ethnologist}\label{american-ethnologist}

\begin{itemize}
\item
  \emph{American Ethnologist is a quarterly journal concerned with
  ethnology in the broadest sense of the term.}
\item
  \emph{American Ethnologist is published in partnership with
  Wiley-Blackwell.}
\end{itemize}

American Ethnologist is a quarterly journal concerned with ethnology in
the broadest sense of the term. The editor welcomes manuscripts that
creatively demonstrate the connections between ethnographic specificity
and theoretical originality, as well as the ongoing relevance of the
ethnographic imagination to the contemporary world." The editor-in-chief
is Niko Besnier (University of Amsterdam). According to the Journal
Citation Reports, the journal has a 2016 impact factor of 1.956, ranking
it 14th out of 82 journals in the category "Anthropology". American
Ethnologist is published in partnership with Wiley-Blackwell.

\section{References}\label{references}

\section{External links}\label{external-links}

\begin{itemize}
\item
  \emph{Register to the Records of the American Ethnological Society,
  National Anthropological Archives, Smithsonian Institution}
\item
  \emph{American Ethnological Society}
\item
  \emph{American Anthropological Association}
\end{itemize}

American Ethnological Society

American Anthropological Association

Register to the Records of the American Ethnological Society, National
Anthropological Archives, Smithsonian Institution
