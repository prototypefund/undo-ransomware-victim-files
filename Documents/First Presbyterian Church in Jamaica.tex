\textbf{From Wikipedia, the free encyclopedia}

https://en.wikipedia.org/wiki/First\%20Presbyterian\%20Church\%20in\%20Jamaica\\
Licensed under CC BY-SA 3.0:\\
https://en.wikipedia.org/wiki/Wikipedia:Text\_of\_Creative\_Commons\_Attribution-ShareAlike\_3.0\_Unported\_License

\section{First Presbyterian Church in
Jamaica}\label{first-presbyterian-church-in-jamaica}

\begin{itemize}
\item
  \emph{The First Presbyterian Church in Jamaica is located in Jamaica,
  Queens, a neighbourhood of New York City.}
\item
  \emph{Organized in 1662, it is the oldest continuously serving
  Presbyterian church in the United States.}
\end{itemize}

The First Presbyterian Church in Jamaica is located in Jamaica, Queens,
a neighbourhood of New York City. Organized in 1662, it is the oldest
continuously serving Presbyterian church in the United States.

\section{History}\label{history}

\begin{itemize}
\item
  \emph{In 1699, a stone church was built on what is now Jamaica Avenue,
  paid for by tax dollars.}
\item
  \emph{The church was first organized in 1662.}
\item
  \emph{Later that year, the church turned down landmark recognition by
  the New York City Landmarks Preservation Commission.}
\item
  \emph{In April 2008, the church celebrated its 345th anniversary.}
\item
  \emph{A new church was constructed in 1813 near what is now 163rd
  Street.}
\end{itemize}

The church was first organized in 1662. Most of its founders came from
Halifax, West Yorkshire, England. Though there are older churches on
Long Island, this congregation has never stopped service. In 1699, a
stone church was built on what is now Jamaica Avenue, paid for by tax
dollars. In 1702, the congregations of Grace Episcopal and First
Reformed split off. A new church was constructed in 1813 near what is
now 163rd Street. It was moved, along with a manse built in 1824, in
1920 to the present location at 89-60 164th Street. Two years later, a
bronze tablet was erected to mark the 260th anniversary. In 1925, a
church house, now known as the Magill Memorial Building, was erected to
accommodate the growing church's need for office and classroom space. In
1959, Donald Trump was confirmed here and attended services in his
formative years within a then all-white congregation. In April 2008, the
church celebrated its 345th anniversary. Later that year, the church
turned down landmark recognition by the New York City Landmarks
Preservation Commission.

\section{Ministries and community
outreach}\label{ministries-and-community-outreach}

\begin{itemize}
\item
  \emph{Internal ministries include communications within the church
  (e.g.}
\item
  \emph{); drama within the church (e.g.}
\item
  \emph{The church also partners with YouthWorks, which allows young
  people from all over the world to experience the wonderful diversity
  of culture and worship styles found in Jamaica.}
\item
  \emph{It also partners with many organizations, including Alcoholics
  Anonymous, Blood Services of New York City, the Cornell University
  Extension Program, and the New York City Department of Youth and
  Community Development.}
\end{itemize}

Internal ministries include communications within the church (e.g.
newsletter, bulletin, etc.); drama within the church (e.g. Christmas
pageants); small group ministries centering on personal interests;
Brothers on the Move, the men's ministry; Presbyterian Women; and the
Alice Horn Gift Shop, including the Pastor's Book of the Month Club.

The church has many community outreach programs, including blood drives,
prostate cancer screenings, sponsoring Girl Scouts and Cub Scouts, and
running the Bread of Life Food Pantry, which serves the nearly 39\% of
homes in Queens afflicted by hunger. It also partners with many
organizations, including Alcoholics Anonymous, Blood Services of New
York City, the Cornell University Extension Program, and the New York
City Department of Youth and Community Development.

The church also partners with YouthWorks, which allows young people from
all over the world to experience the wonderful diversity of culture and
worship styles found in Jamaica.

\section{Tree of Life Center}\label{tree-of-life-center}

\begin{itemize}
\item
  \emph{The Tree of Life Center is an outreach program to economically
  empower Jamaica, and make individuals in the community
  self-sufficient.}
\end{itemize}

The Tree of Life Center is an outreach program to economically empower
Jamaica, and make individuals in the community self-sufficient. Programs
at the Center include GED, ESL, and SAT preparation, financial literacy
and job readiness classes, and hunger relief. The Center also
complements existing social service providers.

\section{References}\label{references}

\section{External links}\label{external-links}

\begin{itemize}
\item
  \emph{First Presbyterian Church in Jamaica website}
\end{itemize}

First Presbyterian Church in Jamaica website
