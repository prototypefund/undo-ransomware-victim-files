\textbf{From Wikipedia, the free encyclopedia}

https://en.wikipedia.org/wiki/Claudette\%20Colvin\\
Licensed under CC BY-SA 3.0:\\
https://en.wikipedia.org/wiki/Wikipedia:Text\_of\_Creative\_Commons\_Attribution-ShareAlike\_3.0\_Unported\_License

\section{Claudette Colvin}\label{claudette-colvin}

\begin{itemize}
\item
  \emph{Claudette Colvin (born September 5, 1939) is an American nurse
  and was a pioneer of the Civil Rights Movement.}
\item
  \emph{Colvin has said, "Young people think Rosa Parks just sat down on
  a bus and ended segregation, but that wasn't the case at all."}
\item
  \emph{Three days later, the Supreme Court issued an order to
  Montgomery and the state of Alabama to end bus segregation, and the
  Montgomery Bus Boycott was called off.}
\item
  \emph{Colvin was the last witness to testify.}
\end{itemize}

Claudette Colvin (born September 5, 1939) is an American nurse and was a
pioneer of the Civil Rights Movement. On March 2, 1955, she was arrested
at the age of 15 in Montgomery, Alabama for refusing to give up her seat
to a white woman on a crowded, segregated bus. Colvin acted nine months
before the more widely known incident in which Rosa Parks, secretary of
the local chapter of the NAACP, played the lead role, sparking the
Montgomery Bus Boycott that began that year.

Colvin was among the five plaintiffs originally included in the federal
court case filed by civil rights attorney Fred Gray on February 1, 1956,
as Browder v. Gayle, to challenge bus segregation in the city. She
testified before the three-judge panel that heard the case in the United
States District Court. On June 13, 1956, the judges determined that the
state and local laws requiring bus segregation in Alabama were
unconstitutional. The case went to the United States Supreme Court on
appeal by the state, and it upheld the District Court ruling on December
17, 1956. Colvin was the last witness to testify. Three days later, the
Supreme Court issued an order to Montgomery and the state of Alabama to
end bus segregation, and the Montgomery Bus Boycott was called off.

For many years, Montgomery's black leaders did not publicize Colvin's
pioneering effort. She was an unmarried teenager at the time, and was
reportedly pregnant by a married man. Colvin has said, "Young people
think Rosa Parks just sat down on a bus and ended segregation, but that
wasn't the case at all." Her case did help the cause, however.

\section{Early life}\label{early-life}

\begin{itemize}
\item
  \emph{Colvin was born September 5, 1939, and was adopted by C.P Colvin
  and Mary Anne Colvin.}
\end{itemize}

Colvin was born September 5, 1939, and was adopted by C.P Colvin and
Mary Anne Colvin. She grew up in a poor black neighborhood of
Montgomery, Alabama. In 1943, at the age of four, Colvin was at a retail
store with her mother when a couple of white boys entered. They asked
her to touch hands in order to compare their colors. Seeing this, her
mother slapped her in the face and told her that she was not allowed to
touch the white boys.

\section{Bus incident}\label{bus-incident}

\begin{itemize}
\item
  \emph{Colvin was a member of the NAACP Youth Council, and had been
  learning about the Civil Rights Movement in school.}
\item
  \emph{Claudette Colvin: "My mother told me to be quiet about what I
  did.}
\item
  \emph{"The bus was getting crowded, and I remember the bus driver
  looking through the rear view mirror asking her {[}Colvin{]} to get up
  for the white woman, which she didn't," said Annie Larkins Price, a
  classmate of Colvin.}
\item
  \emph{Colvin was handcuffed, arrested, and forcibly removed from the
  bus.}
\end{itemize}

In 1955, Colvin was a student at the segregated Booker T. Washington
High School in the city. She relied on the city's buses to get to and
from school, because her parents did not own a car. The majority of
customers on the bus system were African American, but they were
discriminated against by its custom of segregated seating. She said that
she aspired to be President one day. Colvin was a member of the NAACP
Youth Council, and had been learning about the Civil Rights Movement in
school. On March 2, 1955, she was returning home from school. She sat in
the colored section about two seats away from an emergency exit, in a
Capitol Heights bus.

If the bus became so crowded that all the so-called "white seats" in
front of the bus were filled till white people were standing, any
African Americans were supposed to get up from nearby seats to make room
for whites, move further to the back, and stand in the aisle if there
were no free seats in that section. When a white woman who got on the
bus was left standing in the front, the bus driver, Robert W. Cleere,
commanded Colvin and three other black women in her row to move to the
back. The other three moved, but another pregnant black woman, Ruth
Hamilton, got on and sat next to Colvin.

The driver looked at them in his mirror. "He asked us both to get up.
{[}Mrs. Hamilton{]} said she was not going to get up and that she had
paid her fare and that she didn't feel like standing," recalls Colvin.
"So I told him I was not going to get up either. So he said, 'If you are
not going to get up, I will get a policeman.'" The police arrived and
convinced a black man sitting behind the two women to move so that Mrs.
Hamilton could move back, but Colvin still refused to move. She was
forcibly removed from the bus and arrested by the two policemen, Thomas
J. Ward and Paul Headley. This event took place nine months before the
NAACP secretary Rosa Parks was famously arrested for the same offense.
Claudette Colvin: "My mother told me to be quiet about what I did. She
told me to let Rosa be the one: white people aren't going to bother
Rosa, they like her".

When Colvin refused to get up, she was thinking about a school paper she
had written that day about the local custom that prohibited blacks from
using the dressing rooms in order to try on clothes in department
stores. In a later interview, she said: "We couldn't try on clothes. You
had to take a brown paper bag and draw a diagram of your foot {[}...{]}
and take it to the store''. Referring to the segregation on the bus and
the white woman: "She couldn't sit in the same row as us because that
would mean we were as good as her".

"The bus was getting crowded, and I remember the bus driver looking
through the rear view mirror asking her {[}Colvin{]} to get up for the
white woman, which she didn't," said Annie Larkins Price, a classmate of
Colvin. "She had been yelling, 'It's my constitutional right!'. She
decided on that day that she wasn't going to move." Colvin recalled,
"History kept me stuck to my seat. I felt the hand of Harriet Tubman
pushing down on one shoulder and Soujourner Truth pushing down on the
other." Colvin was handcuffed, arrested, and forcibly removed from the
bus. She shouted that her constitutional rights were being violated.
Claudette Colvin said "But I made a personal statement, too, one that
{[}Parks{]} didn't make and probably couldn't have made. Mine was the
first cry for justice, and a loud one."

Price testified for Colvin, who was tried in juvenile court. Colvin was
initially charged with disturbing the peace, violating the segregation
laws, and assault. "There was no assault," Price said. She was bailed
out by her minister, who told her that she had brought the revolution to
Montgomery.

Through the trial Colvin was represented by Fred Gray, a lawyer for the
Montgomery Improvement Association (MIA), which was organizing civil
rights actions. When Colvin's case was brought to the Montgomery Circuit
Court on May 6, 1955, the charges of disturbing the peace and violating
the segregation laws were dropped.

\section{Browder v. Gayle}\label{browder-v.-gayle}

\begin{itemize}
\item
  \emph{During the court case, Colvin described her arrest: "I kept
  saying, 'He has no civil right... this is my constitutional right...
  you have no right to do this.'}
\item
  \emph{The case, organized and filed in federal court by civil rights
  attorney Fred Gray, challenged city bus segregation in Montgomery,
  Alabama as unconstitutional.}
\end{itemize}

Together with Aurelia S. Browder, Susie McDonald, Mary Louise Smith, and
Jeanetta Reese, Colvin was one of the five plaintiffs in the court case
of Browder v. Gayle. Jeanetta Reese later resigned from the case. The
case, organized and filed in federal court by civil rights attorney Fred
Gray, challenged city bus segregation in Montgomery, Alabama as
unconstitutional. During the court case, Colvin described her arrest: "I
kept saying, 'He has no civil right... this is my constitutional
right... you have no right to do this.' And I just kept blabbing things
out, and I never stopped. That was worse than stealing, you know,
talking back to a white person." Momentum on the case started to slow
down until stopping after finding out that Claudette Colvin was several
months pregnant and has been prone to outbursts and cursing. Therefore
the case was dropped and a boycott and legal case never materialized.

Browder v. Gayle made its way through the courts. On June 5, 1956, the
United States District Court for the Middle District of Alabama issued a
ruling declaring the state of Alabama and Montgomery's laws mandating
public bus segregation as unconstitutional. State and local officials
appealed the case to the United States Supreme Court. The Supreme Court
summarily affirmed the District Court decision on November 13, 1956. One
month later, the Supreme Court declined to reconsider, and on December
20, 1956, the court ordered Montgomery and the state of Alabama to end
bus segregation permanently.

\section{Life after activism}\label{life-after-activism}

\begin{itemize}
\item
  \emph{In New York, the young Claudette Colvin and her son Raymond
  initially lived with her older sister, Velma Colvin.}
\item
  \emph{{[}citation needed{]} Colvin left Montgomery for New York City
  in 1958, because she had difficulty finding and keeping work following
  her participation in the federal court case that overturned bus
  segregation.}
\item
  \emph{Similarly, Rosa Parks left Montgomery for Detroit in 1957.}
\item
  \emph{Colvin said that after her actions on the bus, she was branded a
  troublemaker by many in her community.}
\end{itemize}

Colvin gave birth to a son, Raymond in March 1956. He was light-skinned
(like his father) and people frequently assumed his father was Elliot
Klein (a very prominent white male in the Montgomery community who
sympathized with blacks). Elliot likely had European ancestry, among
more distant ancestors. Elliot later admitted to being the father of the
child, but nobody believed him.{[}citation needed{]} Colvin left
Montgomery for New York City in 1958, because she had difficulty finding
and keeping work following her participation in the federal court case
that overturned bus segregation. Similarly, Rosa Parks left Montgomery
for Detroit in 1957. Colvin said that after her actions on the bus, she
was branded a troublemaker by many in her community. She had to drop out
of college and struggled in the local environment.

In New York, the young Claudette Colvin and her son Raymond initially
lived with her older sister, Velma Colvin. Claudette got a job as a
nurse's aide in a nursing home in Manhattan. She worked there for 35
years, from 1969 till retiring in 2004. While living in New York, she
had a second son. He gained an education and became an accountant in
Atlanta, where he also married and had his own family. Raymond Colvin
died in 1993 in New York of a heart attack, aged 37.

\section{Legacy}\label{legacy}

\begin{itemize}
\item
  \emph{Colvin has often said she is not angry that she did not get more
  recognition; rather, she is disappointed.}
\item
  \emph{Colvin was a predecessor to the Montgomery bus boycott movement
  of 1955, which gained national attention.}
\item
  \emph{In 2005, Colvin told the Montgomery Advertiser that she would
  not have changed her decision to remain seated on the bus: "I feel
  very, very proud of what I did," she said.}
\end{itemize}

Colvin was a predecessor to the Montgomery bus boycott movement of 1955,
which gained national attention. But she rarely told her story after
moving to New York City. The discussions in the black community began to
focus on black enterprise rather than integration, although national
civil rights legislation did not pass until 1964 and 1965. NPR's Margot
Adler has said that black organizations believed that Rosa Parks would
be a better figure for a test case for integration because she was an
adult, had a job, and had a middle-class appearance. They felt she had
the maturity to handle being at the center of potential controversy.

In 2005, Colvin told the Montgomery Advertiser that she would not have
changed her decision to remain seated on the bus: "I feel very, very
proud of what I did," she said. "I do feel like what I did was a spark
and it caught on." "I'm not disappointed. Let the people know Rosa Parks
was the right person for the boycott. But also let them know that the
attorneys took four other women to the Supreme Court to challenge the
law that led to the end of segregation."

Colvin has often said she is not angry that she did not get more
recognition; rather, she is disappointed. She said she felt as if she
was "getting her Christmas in January rather than the 25th."

\section{Seeking recognition}\label{seeking-recognition}

\begin{itemize}
\item
  \emph{March 2 was named Claudette Colvin day in Montgomery.}
\item
  \emph{In an interview, Colvin said,}
\item
  \emph{In 2016, the Smithsonian Institution and its National Museum of
  African-American History and Culture (NMAAHC) were challenged by
  Colvin and her family, who asked that Colvin be given a more prominent
  mention in the history of the civil rights movement.}
\item
  \emph{Councilman Larkin's sister was on the bus in 1955 when Colvin
  was arrested.}
\end{itemize}

In an interview, Colvin said,

Colvin and her family have been fighting for recognition for her action.
In 2016, the Smithsonian Institution and its National Museum of
African-American History and Culture (NMAAHC) were challenged by Colvin
and her family, who asked that Colvin be given a more prominent mention
in the history of the civil rights movement. The NMAAHC has a section
dedicated to Rosa Parks, which Colvin does not want taken away, but her
family's goal is to get the historical record right, and for officials
to include Colvin's part of history. Colvin was not invited officially
for the formal dedication of the museum, which opened to the public in
September 2016.

``All we want is the truth, why does history fail to get it right?''
Colvin's sister, Gloria Laster, said. ``Had it not been for Claudette
Colvin, Aurelia Browder, Susie McDonald, and Mary Louise Smith there may
not have been a Thurgood Marshall, a Martin Luther King or a Rosa
Parks.''

In 2000, Troy State University opened a Rosa Parks Museum in Montgomery
to honor the town's place in civil rights history. Roy White, who was in
charge of most of the project, asked Colvin if she would like to appear
in a video to tell her story, but Colvin refused. She said, "They've
already called it the Rosa Parks museum, so they've already made up
their minds what the story is."

Colvin's role has not gone completely unrecognized. Rev. Joseph Rembert
said, ``If nobody did anything for Claudette Colvin in the past why
don't we do something for her right now?'' He reached out to Montgomery
Councilmen Charles Jinright and Tracy Larkin to make it happen. In 2017,
the Montgomery Council passed a resolution for a proclamation honoring
Colvin. March 2 was named Claudette Colvin day in Montgomery. Mayor Todd
Strange presented the proclamation and, when speaking of Colvin, said,
``She was an early foot soldier in our civil rights, and we did not want
this opportunity to go by without declaring March 2 as Claudette Colvin
Day to thank her for her leadership in the modern day civil rights
movement.'' Rembert said, ``I know people have heard her name before,
but I just thought we should have a day to celebrate her.'' Colvin could
not attend the proclamation due to health concerns.

Councilman Larkin's sister was on the bus in 1955 when Colvin was
arrested. A few years ago, Larkin arranged for a streetcar to be named
after Colvin.\\
\hspace*{0.333em}

\section{In culture}\label{in-culture}

\begin{itemize}
\item
  \emph{Poet Laureate Rita Dove memorialized Colvin in her poem
  "Claudette Colvin Goes To Work", published in her 1999 book On the Bus
  with Rosa Parks; folk singer John McCutcheon turned this poem into a
  song, which was first publicly performed in Charlottesville,
  Virginia's Paramount Theater in 2006.}
\item
  \emph{In a 2014 episode of Drunk History about Montgomery, Alabama,
  Claudette Colvin's resistance on the bus was shown.}
\end{itemize}

Pulitzer Prize winner and former U.S. Poet Laureate Rita Dove
memorialized Colvin in her poem "Claudette Colvin Goes To Work",
published in her 1999 book On the Bus with Rosa Parks; folk singer John
McCutcheon turned this poem into a song, which was first publicly
performed in Charlottesville, Virginia's Paramount Theater in 2006.

In a 2014 episode of Drunk History about Montgomery, Alabama, Claudette
Colvin's resistance on the bus was shown. She was played by Mariah Iman
Wilson.

In the second season (2013) of the HBO drama The Newsroom, the lead
character, Will McAvoy (played by Jeff Daniels), uses Colvin's refusal
to comply with segregation as an example of how "one thing" can change
everything. He remarks that if the ACLU had used her act of civil
disobedience, rather than that of Rosa Parks' eight months later, to
highlight the injustice of segregation, a young preacher named Dr.
Martin Luther King Jr. may never have attracted national attention, and
America probably would not have had his voice for the Civil Rights
Movement.

\section{See also}\label{see-also}

\begin{itemize}
\item
  \emph{List of civil rights leaders}
\item
  \emph{Montgomery Bus Boycott}
\end{itemize}

List of civil rights leaders

Montgomery Bus Boycott

Mary Louise Smith

Aurelia Browder

Irene Morgan

E. D. Nixon

Sarah Keys

\section{References}\label{references}

\section{Further reading}\label{further-reading}

\begin{itemize}
\item
  \emph{Farrar, Straus and Giroux (BYR), Claudette Colvin, Twice Toward
  Justice.}
\end{itemize}

Phillip Hoose. Farrar, Straus and Giroux (BYR), Claudette Colvin, Twice
Toward Justice. (2009). ISBN~0-374-31322-9.

Taylor Branch. New York, Simon \& Schuster Paperbacks, Parting The
Waters - American in the King Years 1954-63. (1988). ISBN~0-671-68742-5.

\section{External links}\label{external-links}

\begin{itemize}
\item
  \emph{The Rebellious Life of Mrs. Rosa Parks, Rosa Parks Biography}
\item
  \emph{Daybreak of Freedom: The Montgomery Bus Boycott (Excerpt)}
\item
  \emph{Let us Look at Jim Crow for the Criminal he is; Claudette
  Colvin.}
\item
  \emph{Daybreak of Freedom: The Montgomery Bus Boycott (Preface)}
\item
  \emph{The Other Rosa Parks (Colvin interview with Democracy Now!)}
\end{itemize}

The Other Rosa Parks (Colvin interview with Democracy Now!)

She had a Dream

Daybreak of Freedom: The Montgomery Bus Boycott (Preface)

Daybreak of Freedom: The Montgomery Bus Boycott (Excerpt)

"Browder v. Gayle: The Women Before Rosa Parks", Tolerance

Vanessa de la Torre, "In The Shadow of Rosa Parks: 'Unsung Hero' of
Civil Rights Movement Speaks Out", The Cardinal Inquirer, January 20,
2005

"She Would not be Moved", The Guardian

"An asterisk, not a star, of black history", Pulsejournal

Let us Look at Jim Crow for the Criminal he is; Claudette Colvin. The
Rebellious Life of Mrs. Rosa Parks, Rosa Parks Biography
