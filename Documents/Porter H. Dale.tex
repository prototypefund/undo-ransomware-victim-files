\textbf{From Wikipedia, the free encyclopedia}

https://en.wikipedia.org/wiki/Porter\%20H.\%20Dale\\
Licensed under CC BY-SA 3.0:\\
https://en.wikipedia.org/wiki/Wikipedia:Text\_of\_Creative\_Commons\_Attribution-ShareAlike\_3.0\_Unported\_License

\section{Porter H. Dale}\label{porter-h.-dale}

\begin{itemize}
\item
  \emph{Porter Hinman Dale (March 1, 1867~-- October 6, 1933) was a
  member of both the United States House of Representatives and later
  the United States Senate from Vermont.}
\end{itemize}

Porter Hinman Dale (March 1, 1867~-- October 6, 1933) was a member of
both the United States House of Representatives and later the United
States Senate from Vermont.

\includegraphics[width=4.24698in,height=5.50000in]{media/image1.jpg}\\
\emph{Porter H. Dale as candidate for Congress, 1898}

\section{Early life and career}\label{early-life-and-career}

\begin{itemize}
\item
  \emph{After the death of his father, Dale practiced in partnership
  with Harry B. Amey.}
\item
  \emph{The son of Lieutenant Governor George N. Dale and Helen (Hinman)
  Dale, Porter Dale was born in Island Pond, Vermont on March 1, 1867.}
\item
  \emph{Dale was elected to the Vermont State Senate in 1910 and served
  two two-year terms.}
\end{itemize}

The son of Lieutenant Governor George N. Dale and Helen (Hinman) Dale,
Porter Dale was born in Island Pond, Vermont on March 1, 1867.

Dale attended public schools in his hometown and went on to study at
Eastman Business College. Later he studied in Philadelphia and Boston,
and he spent two years studying elocution and oratory with James Edward
Murdoch, a Shakespearean scholar and actor.

Upon completion of his education, he taught school at the Green Mountain
Seminary in Waterbury, Vermont, and at Bates College in Lewiston, Maine.
Dale then studied law with his father, was admitted to the bar in 1896,
and practiced in Island Pond. After the death of his father, Dale
practiced in partnership with Harry B. Amey.

Dale served as chief deputy collector of customs at Island Pond from
1897 to 1910, when he resigned and was appointed judge of the Brighton
municipal court. He also served in the state militia as colonel on the
staff Governor Josiah Grout, and he was also involved in the lumber,
electric, and banking businesses.

In 1900 he was an unsuccessful candidate for the Republican nomination
in the election for Vermont's Second District seat in the U.S. House.
Dale was elected to the Vermont State Senate in 1910 and served two
two-year terms.

\section{House of Representatives}\label{house-of-representatives}

\begin{itemize}
\item
  \emph{He served from March 4, 1915 to August 11, 1923, when he
  resigned to become a candidate for the United States Senate.}
\item
  \emph{Dale served as chairman of the Committee on Expenditures in the
  Department of the Treasury during the Sixty-Sixth and Sixty-Seventh
  Congresses.}
\item
  \emph{In 1914, Dunnett was a candidate for the Republican U.S. House
  nomination in Vermont's 2nd District.}
\end{itemize}

In 1914, Dunnett was a candidate for the Republican U.S. House
nomination in Vermont's 2nd District. He defeated Alexander Dunnett on
the 21st ballot at the state party convention, and went on to win the
general election. He served from March 4, 1915 to August 11, 1923, when
he resigned to become a candidate for the United States Senate. Dale
served as chairman of the Committee on Expenditures in the Department of
the Treasury during the Sixty-Sixth and Sixty-Seventh Congresses.

\section{Extraordinary inauguration of Calvin
Coolidge}\label{extraordinary-inauguration-of-calvin-coolidge}

\begin{itemize}
\item
  \emph{By most accounts, it was Dale who suggested persistently that
  Coolidge be sworn in immediately to ensure continuity in the
  presidency, and Dale witnessed Coolidge receiving the oath of office
  from John Coolidge early on the morning of August 3.}
\item
  \emph{Calvin Coolidge was staying at the home of his father John
  Calvin Coolidge Sr. in Plymouth, Vermont, and Dale traveled to the
  Coolidge home to ensure that Coolidge was informed and to offer his
  assistance.}
\end{itemize}

Dale was campaigning for the Senate on the night of August 2, 1923 when
he heard of the death of President Warren G. Harding. Calvin Coolidge
was staying at the home of his father John Calvin Coolidge Sr. in
Plymouth, Vermont, and Dale traveled to the Coolidge home to ensure that
Coolidge was informed and to offer his assistance. By most accounts, it
was Dale who suggested persistently that Coolidge be sworn in
immediately to ensure continuity in the presidency, and Dale witnessed
Coolidge receiving the oath of office from John Coolidge early on the
morning of August 3. Dale later wrote an account of this event, which
was published as a magazine article.

\section{U.S. Senate}\label{u.s.-senate}

\begin{itemize}
\item
  \emph{Dale was reelected in 1926 and 1932, and served from November 7,
  1923, until his death.}
\item
  \emph{Dale was elected to the United States Senate on November 6, 1923
  for the remainder of the term ending March 3, 1927, which had been
  made vacant by the death of William P. Dillingham.}
\end{itemize}

Dale was elected to the United States Senate on November 6, 1923 for the
remainder of the term ending March 3, 1927, which had been made vacant
by the death of William P. Dillingham. Dale was reelected in 1926 and
1932, and served from November 7, 1923, until his death. He was chairman
of the Committee on Civil Service (Sixty-ninth through Seventy-second
Congresses).

\section{Death and burial}\label{death-and-burial}

\begin{itemize}
\item
  \emph{Dale died at his summer home on Lake Willoughby in Westmore,
  Vermont On October 6, 1933.}
\end{itemize}

Dale died at his summer home on Lake Willoughby in Westmore, Vermont On
October 6, 1933. He was buried in Lakeside Cemetery in Island Pond.

\section{Family}\label{family}

\begin{itemize}
\item
  \emph{In 1891, Dale married Amy K. Bartlett (b.}
\item
  \emph{With his first wife, Dale was the father of Marian (1892-1975),
  Timothy (1894-1977), Amy (1895-1938), and George (1898-1962).}
\end{itemize}

In 1891, Dale married Amy K. Bartlett (b. 1861) of Island Pond. She died
on August 1, 1907, and in 1910 he married Augusta M. Wood (1876-1961) of
Boston. With his first wife, Dale was the father of Marian (1892-1975),
Timothy (1894-1977), Amy (1895-1938), and George (1898-1962).

\section{See also}\label{see-also}

\begin{itemize}
\item
  \emph{List of United States Congress members who died in office
  (1900--49)}
\end{itemize}

List of United States Congress members who died in office (1900--49)

\section{References}\label{references}

\section{External links}\label{external-links}

\begin{itemize}
\item
  \emph{Porter Hinman Dale at Find A Grave}
\item
  \emph{United States Congress.}
\item
  \emph{"Porter H. Dale (id: D000009)".}
\item
  \emph{Biographical Directory of the United States Congress.}
\end{itemize}

United States Congress. "Porter H. Dale (id: D000009)". Biographical
Directory of the United States Congress.

Porter Hinman Dale at Find A Grave

~This article incorporates~public domain material from the Biographical
Directory of the United States Congress website
http://bioguide.congress.gov.
