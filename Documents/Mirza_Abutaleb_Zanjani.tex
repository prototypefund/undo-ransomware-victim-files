\textbf{From Wikipedia, the free encyclopedia}

https://en.wikipedia.org/wiki/Mirza\_Abutaleb\_Zanjani\\
Licensed under CC BY-SA 3.0:\\
https://en.wikipedia.org/wiki/Wikipedia:Text\_of\_Creative\_Commons\_Attribution-ShareAlike\_3.0\_Unported\_License

\includegraphics[width=4.52571in,height=5.50000in]{media/image1.jpg}\\
\emph{Mirza Abutaleb Zanjani}

\section{Mirza Abutaleb Zanjani}\label{mirza-abutaleb-zanjani}

\begin{itemize}
\item
  \emph{He married a daughter of Bahram Mirza son of Abbas Mirza and had
  4 daughters.}
\item
  \emph{Arthur Henry Hardinge, the British ambassador to Iran, wrote
  about him:"As I have previously mentioned, the most intellectual and
  enlightened Shia scholar that I've met in Tehran was Mirza Abutaleb
  Zanjani with whom we usually had debates about religion and politics.}
\end{itemize}

Mirzā Abutāleb Zanjānī (Persian: میرزا ابوطالب زنجانی‎) also known as
Sayyid Fakhr al-Din Mohammad Abutāleb Mousavi al-Zanjānī (10 December
1843 -- 16 March 1911) Iranian Jurist and Shia scholar

He was born on 10 December 1843, to an educated family in Zanjan, Iran.
His paternal ancestors were all celebrated scholars. He started his
education in his birthplace and continued in Qazvin, Iran and Najaf,
Iraq, trained under Morteza Ansari, Sheikh Razi and Sayyid Hossein
Kooh-kamari. He returned to Iran at the age of 40 and stayed in Tehran,
where as a distinguished disciple of Koohkamari, became the centre of
clerical circles.

He spent most of his time on teaching his students and writing religious
books.

He advocated Persian Constitutional Revolution, but later adhered to
royalists.

He died at the age of 69 on 16 March 1911, Tehran and was buried in
Mashhad.

Zanjani was among the few scholars of Qajar period who used their Arabic
knowledge to translate Arabic texts into Persian. In addition to Persian
and Azari, he had acquaintance with French and Ottoman Turkish language.
He also had knowledge of ideas of his contemporary European intellects
such as Thomas Malthus and Charles Darwin, and used their views in his
essays.

Arthur Henry Hardinge, the British ambassador to Iran, wrote about
him:"As I have previously mentioned, the most intellectual and
enlightened Shia scholar that I've met in Tehran was Mirza Abutaleb
Zanjani with whom we usually had debates about religion and politics. I
personally think that Mirza Abutaleb worked on the same aspiration for
Islamic unity as Abdul Hamid II ... although he himself had less faith
to these principals".

He married a daughter of Bahram Mirza son of Abbas Mirza and had 4
daughters. He is maternal grandfather of Reza Zanjani.

\section{Notes}\label{notes}
