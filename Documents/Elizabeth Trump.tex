\textbf{From Wikipedia, the free encyclopedia}

https://en.wikipedia.org/wiki/Elizabeth\%20Trump\\
Licensed under CC BY-SA 3.0:\\
https://en.wikipedia.org/wiki/Wikipedia:Text\_of\_Creative\_Commons\_Attribution-ShareAlike\_3.0\_Unported\_License

\section{Elizabeth Christ Trump}\label{elizabeth-christ-trump}

\begin{itemize}
\item
  \emph{She founded the real estate development company Elizabeth Trump
  \& Son with her son Fred.}
\item
  \emph{The company, now known as the Trump Organization, is currently
  owned by her grandson, Donald Trump, the 45th President of the United
  States.}
\item
  \emph{She married Frederick Trump in 1902.}
\item
  \emph{Elizabeth Christ Trump (née Elisabeth Christ; October 10, 1880
  -- June 6, 1966) was a German-American businesswoman and is considered
  the matriarch of the Trump family.}
\end{itemize}

Elizabeth Christ Trump (née Elisabeth Christ; October 10, 1880 -- June
6, 1966) was a German-American businesswoman and is considered the
matriarch of the Trump family. She married Frederick Trump in 1902.
While raising their three children, the early death of her husband in
1918 required the 37-year-old widow to manage their properties. She
founded the real estate development company Elizabeth Trump \& Son with
her son Fred. The company, now known as the Trump Organization, is
currently owned by her grandson, Donald Trump, the 45th President of the
United States.

\section{Early life}\label{early-life}

\begin{itemize}
\item
  \emph{Elizabeth Trump was born as Elisabeth Christ in Kallstadt,
  Kingdom of Bavaria, the daughter of Philipp Christ by his wife Anna
  Maria Christ (née Anthon).}
\item
  \emph{He ran his trade from his house on Freinsheimer Straße in
  Kallstadt, which was just across the street from the home of the Trump
  family, where Katharina Trump, an elderly widow, lived with her six
  children.}
\end{itemize}

Elizabeth Trump was born as Elisabeth Christ in Kallstadt, Kingdom of
Bavaria, the daughter of Philipp Christ by his wife Anna Maria Christ
(née Anthon). While the family owned a little vineyard, the income from
that was not adequate to meet their needs, and Philipp Christ worked as
a tinker repairing and polishing old utensils and selling pots and pans.
He ran his trade from his house on Freinsheimer Straße in Kallstadt,
which was just across the street from the home of the Trump family,
where Katharina Trump, an elderly widow, lived with her six children.

\includegraphics[width=5.50000in,height=4.23511in]{media/image1.jpg}\\
\emph{Elizabeth Christ Trump \& Frederick Trump, circa 1918}

\includegraphics[width=5.50000in,height=3.94195in]{media/image2.jpg}\\
\emph{Trump family portrait, from left to right: Fred, Frederick,
Elizabeth, Elizabeth Christ, and John, 1918}

\section{Marriage}\label{marriage}

\begin{itemize}
\item
  \emph{Despite living in a German neighborhood, Elizabeth was homesick
  and the family returned to Kallstadt in 1904, selling their assets in
  America.}
\item
  \emph{Their first child, a daughter whom they named Elizabeth, was
  born on April 30, 1904.}
\item
  \emph{After Elizabeth gave birth to her third child, John, the family
  moved to Queens, where Frederick began to develop real estate.}
\end{itemize}

Katharina's son Friedrich had immigrated to America in 1885 at the age
of 16 and made his fortune with restaurants in the Klondike Gold Rush.
When he returned to Germany in 1901, he wooed Elisabeth over the
objections of his mother, who felt that her prosperous son could and
should find a bride from a wealthier and more refined family than
Elisabeth's. Nonetheless, Frederick proposed to Elisabeth, who accepted,
and they were married on 26 August 1902. He was 33 and she was
22-years-old. Frederick and Elisabeth moved to New York and they set up
house in an apartment in the predominantly German quarter of Morrisania
in the Bronx. Elizabeth (as her name was spelled in the US) kept house,
while Frederick worked as a restaurant and hotel manager. Their first
child, a daughter whom they named Elizabeth, was born on April 30, 1904.

Despite living in a German neighborhood, Elizabeth was homesick and the
family returned to Kallstadt in 1904, selling their assets in America.

Because the Bavarian authorities suspected that he had left Germany in
order to avoid the service in the Imperial Army, Frederick could not
remain in Germany, so the family returned to the United States in 1905.
Their second child, Fred, was born and they set up house on 177th Street
in the Bronx. After Elizabeth gave birth to her third child, John, the
family moved to Queens, where Frederick began to develop real estate. In
1918, he died of influenza during the 1918 flu pandemic, leaving an
estate valued at \$31,359 (\$492,016 in 2016 dollars).

\section{Founding Elizabeth Trump \&
Son}\label{founding-elizabeth-trump-son}

\begin{itemize}
\item
  \emph{Her vision was to have her three children continue the family
  business when they finished school, but Fred wanted to start earlier,
  so she founded the company Elizabeth Trump \& Son, the predecessor of
  The Trump Organization.}
\item
  \emph{Even in her 70s she would collect coins from the laundromats in
  Trump buildings.}
\item
  \emph{In 1927 when Fred was 22, Elizabeth Trump \& Son was formally
  incorporated.}
\end{itemize}

Elizabeth had a "remarkable talent" for keeping the real estate business
going. She hired a contractor to build houses on an empty piece of
property left by her husband, sold the houses, and lived off the
mortgages paid by the new owners. Her vision was to have her three
children continue the family business when they finished school, but
Fred wanted to start earlier, so she founded the company Elizabeth Trump
\& Son, the predecessor of The Trump Organization. Since Fred was still
a minor, she had to sign all legal documents on his behalf, and perform
the real estate closings. In 1927 when Fred was 22, Elizabeth Trump \&
Son was formally incorporated. Fred became quite successful with the
business but Elizabeth remained involved throughout her life. Even in
her 70s she would collect coins from the laundromats in Trump buildings.

\section{Personal life}\label{personal-life}

\begin{itemize}
\item
  \emph{Elizabeth was considered the matriarch of the Trump family.}
\end{itemize}

Elizabeth was considered the matriarch of the Trump family. She had
"extraordinary determination". She dressed very conservatively and
formally. She was hardworking and had a stern demeanor. She remained
close to her son Fred for her entire life.

\section{See also}\label{see-also}

\begin{itemize}
\item
  \emph{Married Women's Property Acts in the United States}
\item
  \emph{Women's property rights}
\end{itemize}

Married Women's Property Acts in the United States

Women's property rights

\section{References}\label{references}

\section{External links}\label{external-links}

\begin{itemize}
\item
  \emph{Elizabeth Christ Trump at Find a Grave}
\end{itemize}

Elizabeth Christ Trump at Find a Grave
