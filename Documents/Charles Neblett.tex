\textbf{From Wikipedia, the free encyclopedia}

https://en.wikipedia.org/wiki/Charles\%20Neblett\\
Licensed under CC BY-SA 3.0:\\
https://en.wikipedia.org/wiki/Wikipedia:Text\_of\_Creative\_Commons\_Attribution-ShareAlike\_3.0\_Unported\_License

\section{Charles Neblett}\label{charles-neblett}

\begin{itemize}
\item
  \emph{Charles "Chuck" Neblett (born 1941) is a civil rights activist
  best known for helping to found and being a member of The Freedom
  Singers.}
\end{itemize}

Charles "Chuck" Neblett (born 1941) is a civil rights activist best
known for helping to found and being a member of The Freedom Singers.

\section{Early life and activism}\label{early-life-and-activism}

\begin{itemize}
\item
  \emph{He took an interest in the Civil Rights Movement from a young
  age.}
\item
  \emph{Neblett hails from Cairo, Illinois.}
\item
  \emph{Neblett said.}
\item
  \emph{Neblett attended Southern Illinois University.}
\item
  \emph{There he had his first chances to be involved in fighting for
  Civil Rights when he was recruited by the Student Non-violent
  Coordinating Committee (SNCC).}
\end{itemize}

Neblett hails from Cairo, Illinois. He took an interest in the Civil
Rights Movement from a young age. His first awareness of the Movement
was noticing that the schools he and his fellow African Americans
attended received inferior funding to white schools. When Emmett Till
was murdered in 1955, the news profoundly affected Neblett. It was in
light of this tragedy that he realized as a black American he "had no
rights that white people would respect." He was the same age as Till at
the time: fourteen. He knew then that he had to be a part of the
Movement. "It was like I got religion." Neblett said. Neblett attended
Southern Illinois University. There he had his first chances to be
involved in fighting for Civil Rights when he was recruited by the
Student Non-violent Coordinating Committee (SNCC). He met with success
when he protested the discrimination in housing at the university. He
took his complaints to the University President, and the President made
changes the very next semester. Neblett said it was after this "I
realized we could make a difference."

\section{The Freedom Singers}\label{the-freedom-singers}

\begin{itemize}
\item
  \emph{The Freedom Singers were a creation of SNCC, and the group's
  goals were the same as its parent organization's.}
\item
  \emph{The Freedom Singers were valuable to SNCC as one of their most
  successful fundraisers, but being a member was not always safe.}
\item
  \emph{As a member of the Freedom Singers, Neblett traveled through
  more than forty states and 100,000 miles, traveling mostly by station
  wagon.}
\end{itemize}

The Freedom Singers were a creation of SNCC, and the group's goals were
the same as its parent organization's. They were formed in Albany,
Georgia two years after SNCC, in 1962, with four original members.
Neblett sang bass, performing with soprano Rutha Mae Harris, alto
Bernice Johnson (now Dr. Bernice Johnson Reagon), and tenor Cordell
Reagon. The first tour was planned by SNCC and lasted from December 1962
to August 1963. The group's schedule was a busy one, and they sometimes
sang as many as three concerts a day. Their venues included parties,
churches, protest marches, universities, and even jails all over the
nation. The Freedom Singers were valuable to SNCC as one of their most
successful fundraisers, but being a member was not always safe. Even in
the north they sometimes ran into violent opposition, including Klan
demonstrations during concerts.

The group's repertoire consisted of freedom songs that had been written
or adapted for the movement, including "We Shall Overcome", "We Shall
Not be Moved", and "Keep Your Eyes on the Prize". After the tour, the
original group disbanded and was carried on by others. Beyond the 1980s
the original four reunited to sing several times. The singers remained
lifelong friends.

As a member of the Freedom Singers, Neblett traveled through more than
forty states and 100,000 miles, traveling mostly by station
wagon.{[}citation needed{]} In 1963, the group performed at the March on
Washington for Jobs and Freedom.

\section{Other civil rights
involvement}\label{other-civil-rights-involvement}

\begin{itemize}
\item
  \emph{In all Neblett was arrested 27 times for his involvement.}
\item
  \emph{Neblett was a SNCC field secretary 1961--1966.}
\item
  \emph{Charles Neblett, Carol Ableman, and Matt Jones were separated
  from the rest of the group and surrounded.Neblett attempted to escape
  by climbing over the fence, but the crowd reached him began hitting
  him with their metal chairs.}
\end{itemize}

Neblett was a SNCC field secretary 1961--1966. In 1964, he was part of a
delegation that an Atlanta conference to which Alabama governor George
Wallace and Mississippi Governor Ross Barnett had come to renew their
commitment to preserving segregation with other southern leaders. Upon
entering the stadium the group realized that the "conference" was
actually a meeting of Klan leaders. Charles Neblett, Carol Ableman, and
Matt Jones were separated from the rest of the group and
surrounded.Neblett attempted to escape by climbing over the fence, but
the crowd reached him began hitting him with their metal chairs. Police
officers refused to put a stop to the violence. Ableman, who was white,
escaped without injury, but Neblett and Jones were injured and they were
taken to the emergency room in a police van.

In all Neblett was arrested 27 times for his involvement. In jail he
suffered much inhumane treatment, putting up with rotten food, beatings,
and uncomfortably high temperatures. During this time he found strength
in singing, and even composed while he was incarcerated.

He has worked in the so-called "Black Bottom" neighborhood in
Russellville Kentucky, preserving homes of black Civil War veterans, and
helping young people to research their Civil War ancestors. He served as
the first black elected magistrate in Logan County, Kentucky.

\section{Family}\label{family}

\begin{itemize}
\item
  \emph{Charles and his wife Marvinia have four children, Khary, Kwesi,
  Komero and Kesi.}
\item
  \emph{Charles' brother, Chico Neblett, was also involved in
  non-violent protest in Illinois.}
\end{itemize}

Charles and his wife Marvinia have four children, Khary, Kwesi, Komero
and Kesi. Charles' brother, Chico Neblett, was also involved in
non-violent protest in Illinois.

\section{Contributions in later life}\label{contributions-in-later-life}

\begin{itemize}
\item
  \emph{He was inducted into the Kentucky Civil Rights Hall of Fame in
  2010.}
\item
  \emph{Neblett was impressed with his reception at the White House,
  saying that he "realized the work done in the past was actually
  respected."}
\item
  \emph{Neblett was among many notable performers and sang with Rutha
  Mae Harris, Dr. Bernice Johnson Reagon, and Bernice and Cordell
  Reagon's daughter, Toshi.}
\end{itemize}

He was inducted into the Kentucky Civil Rights Hall of Fame in 2010.
That same year he was present at the 44th Annual Folklife festival at
the Smithsonian. In 2014, he was a guest of President Barack Obama at
the White House. Neblett was among many notable performers and sang with
Rutha Mae Harris, Dr. Bernice Johnson Reagon, and Bernice and Cordell
Reagon's daughter, Toshi. By this time, Cordell Reagon had died. Neblett
also helped Michelle Obama run a workshop for approximately 200
children, among whom were Sasha and Malia Obama. Neblett was impressed
with his reception at the White House, saying that he "realized the work
done in the past was actually respected."

\section{References}\label{references}

\section{External links}\label{external-links}

\begin{itemize}
\item
  \emph{Charles Neblett at the Jodi F. Solomon Speakers Bureau}
\item
  \emph{"Civil-rights activist sings for freedom".}
\item
  \emph{SNCC Digital Gateway: Charles Neblett, Documentary website
  created by the SNCC Legacy Project and Duke University, telling the
  story of the Student Nonviolent Coordinating Committee \& grassroots
  organizing from the inside-out}
\end{itemize}

SNCC Digital Gateway: Charles Neblett, Documentary website created by
the SNCC Legacy Project and Duke University, telling the story of the
Student Nonviolent Coordinating Committee \& grassroots organizing from
the inside-out

Charles Neblett at the Jodi F. Solomon Speakers Bureau

Stansbury, Amy (2013-02-14). "Civil-rights activist sings for freedom".
Evening Sun. Archived from the original on 2014-05-05. Retrieved
2014-05-05.
