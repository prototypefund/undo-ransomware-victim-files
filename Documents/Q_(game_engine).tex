\textbf{From Wikipedia, the free encyclopedia}

https://en.wikipedia.org/wiki/Q\_\%28game\_engine\%29\\
Licensed under CC BY-SA 3.0:\\
https://en.wikipedia.org/wiki/Wikipedia:Text\_of\_Creative\_Commons\_Attribution-ShareAlike\_3.0\_Unported\_License

\section{Q (game engine)}\label{q-game-engine}

\begin{itemize}
\item
  \emph{Q is a 3D engine / tech development platform / interoperability
  standard developed by the London-based developer Qube Software.}
\end{itemize}

Q is a 3D engine / tech development platform / interoperability standard
developed by the London-based developer Qube Software.

\section{Overview}\label{overview}

\begin{itemize}
\item
  \emph{Q is configured as a framework into which all the supplied
  components plug in modular form.}
\item
  \emph{Qube also claims to have developed Q as an interoperability
  standard for 3D, providing the same degree of coherence for the 3D
  products across both gaming and non gaming environments that Flash or
  HTML provide for web applications.}
\item
  \emph{Qube has made considerable claims for Q.}
\end{itemize}

Qube has made considerable claims for Q. Its lead designers, Servan
Keondjian and Doug Rabson, have pointed to Q's architecture as being its
key innovation.

Q is configured as a framework into which all the supplied components
plug in modular form. The framework's common APIs are designed to make
adding and removing components a trivial task and one that can be done
neatly. The key idea is that this makes it simple for studios licensing
the platform to develop and add whatever elements their project requires
and to license original components amongst one another.

The claim has had customer endorsements: ``If we develop a plug-in
during the course of one project its easy to use it or build on it for
another; so our development work is cumulative. We can build a library
of plug-ins. Nothing is wasted.''

Qube also claims to have developed Q as an interoperability standard for
3D, providing the same degree of coherence for the 3D products across
both gaming and non gaming environments that Flash or HTML provide for
web applications.

The claim is predicated on Q's supposed ability to accommodate any
platform (albeit floating point technology is required and it thus fails
to cater for handheld consoles such as the Nintendo DS and Game Boy
Advance), scripting language, or genre of game, or 3D application.

Licensees have already reported titles in production or shipped for the
PC, PS2, Wii and PS3. Keondjian said early in 2008 that an Xbox 360 port
would follow: "we know it's the easiest.'' The company has also
indicated that Mac and Linux versions of Q are available and that the
platform would be compatible with the PSP, iPhone and next generation
mobiles.

\section{Features}\label{features}

\begin{itemize}
\item
  \emph{According to Qube, Q ships with a range of features including:
  arbitrary scene rendering algorithm support, arbitrary shader program
  support (HLSL 2 -- 4, GLSL, Cg, shader states), keyframe animation,
  simultaneous n-dimensional animation blending, animation state
  machines, multi-gigabyte texture manager, background data streaming,
  hierarchical LOD and scene management schemes, collision detection,
  network-enabled media pipeline, live editing of game content,
  scripting across all core and custom components, cross-platform data
  formats and APIs, platform-specific extended data formats and APIs, 2D
  and 3D audio with effects, background texture compression /
  decompression, user input, hardware accelerated math, Max and Maya
  exporters, application framework, command line tool framework, and
  cross-platform build.}
\end{itemize}

According to Qube, Q ships with a range of features including: arbitrary
scene rendering algorithm support, arbitrary shader program support
(HLSL 2 -- 4, GLSL, Cg, shader states), keyframe animation, simultaneous
n-dimensional animation blending, animation state machines,
multi-gigabyte texture manager, background data streaming, hierarchical
LOD and scene management schemes, collision detection, network-enabled
media pipeline, live editing of game content, scripting across all core
and custom components, cross-platform data formats and APIs,
platform-specific extended data formats and APIs, 2D and 3D audio with
effects, background texture compression / decompression, user input,
hardware accelerated math, Max and Maya exporters, application
framework, command line tool framework, and cross-platform build.

\section{Virtual Worlds and MMOGs}\label{virtual-worlds-and-mmogs}

\begin{itemize}
\item
  \emph{Early in 2009, Qube and Brighton-based server solution company
  RedBedlam announced that they would bring their technologies together
  to produce a `one stop shop' for online environments.}
\item
  \emph{Messiah has been adopted by NearGlobal for the NearLondon
  virtual world.}
\end{itemize}

Early in 2009, Qube and Brighton-based server solution company RedBedlam
announced that they would bring their technologies together to produce a
`one stop shop' for online environments. The project was given the
codename '"Messiah". Messiah has been adopted by NearGlobal for the
NearLondon virtual world.

\section{Customers}\label{customers}

\begin{itemize}
\item
  \emph{The developer has hinted that other studios are using Q on
  projects that have not yet been made public.}
\item
  \emph{Take up of Q 2.0 has been steady if unspectacular to
  date.{[}when?{]}}
\end{itemize}

Take up of Q 2.0 has been steady if unspectacular to date.{[}when?{]}
Clients announced include Candella Software, Asylum Entertainment, EC-I
Interactive, NearGlobal, Airo Wireless, and Beyond the Void. The
developer has hinted that other studios are using Q on projects that
have not yet been made public.

\section{History}\label{history}

\begin{itemize}
\item
  \emph{"Basically," Keondjian told the website Gamasutra in 2008, "when
  we left Microsoft after we'd done Direct3D, we wanted to build a
  middleware solution.}
\item
  \emph{Q 1.0 was released in 2001.}
\item
  \emph{Work on Q started in 1998 after Qube founder Servan Keondjian
  left Microsoft.}
\item
  \emph{Q 2.0 was released in February 2008.}
\item
  \emph{Q 2.1 was announced in July 2008 and included script debugging
  and new shader and scene rendering plugins.}
\end{itemize}

Work on Q started in 1998 after Qube founder Servan Keondjian left
Microsoft. There, he had led the team that turned his own Reality Lab
API into Direct3D. According to Qube's website, Keondjian and his
Reality Lab coding partner Doug Rabson believed: ``Microsoft was a great
place to ship products but not a place for innovation and new ideas.''

"Basically," Keondjian told the website Gamasutra in 2008, "when we left
Microsoft after we'd done Direct3D, we wanted to build a middleware
solution. I didn't just want to make another middleware solution, I felt
there was a problem with middleware in the game industry, and I wanted
to really understand that problem and see if we could crack it. That was
the mission."

Q 1.0 was released in 2001. In effect, a prototype for the version that
was to follow it was first used on the BBC's Dinosaur World (June 2001),
LEGO Creator Harry Potter and the Chamber of Secrets (Sept 2002) and
projects for Microsoft and Virgin Interactive.

Q 2.0 was released in February 2008.

Q 2.1 was announced in July 2008 and included script debugging and new
shader and scene rendering plugins.

\section{References}\label{references}

\section{External links}\label{external-links}

\begin{itemize}
\item
  \emph{Qube official website}
\end{itemize}

Qube official website
