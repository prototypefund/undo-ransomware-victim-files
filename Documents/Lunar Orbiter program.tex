\textbf{From Wikipedia, the free encyclopedia}

https://en.wikipedia.org/wiki/Lunar\%20Orbiter\%20program\\
Licensed under CC BY-SA 3.0:\\
https://en.wikipedia.org/wiki/Wikipedia:Text\_of\_Creative\_Commons\_Attribution-ShareAlike\_3.0\_Unported\_License

\includegraphics[width=5.50000in,height=3.05067in]{media/image1.jpg}\\
\emph{Lunar orbiter spacecraft (NASA)}

\section{Lunar Orbiter program}\label{lunar-orbiter-program}

\begin{itemize}
\item
  \emph{The Lunar Orbiter program was a series of five unmanned lunar
  orbiter missions launched by the United States from 1966 through
  1967.}
\item
  \emph{During the Lunar Orbiter missions, the first pictures of Earth
  as a whole were taken, beginning with Earth-rise over the lunar
  surface by Lunar Orbiter 1 in August, 1966.}
\end{itemize}

The Lunar Orbiter program was a series of five unmanned lunar orbiter
missions launched by the United States from 1966 through 1967. Intended
to help select Apollo landing sites by mapping the Moon's surface, they
provided the first photographs from lunar orbit and photographed both
the Moon and Earth.

All five missions were successful, and 99 percent of surface of the Moon
was mapped from photographs taken with a resolution of 60 meters
(200~ft) or better. The first three missions were dedicated to imaging
20 potential manned lunar landing sites, selected based on Earth-based
observations. These were flown at low-inclination orbits. The fourth and
fifth missions were devoted to broader scientific objectives and were
flown in high-altitude polar orbits. Lunar Orbiter 4 photographed the
entire nearside and nine percent of the far side, and Lunar Orbiter 5
completed the far side coverage and acquired medium (20~m (66~ft)) and
high (2~m (6~ft 7~in)) resolution images of 36 preselected areas. All of
the Lunar Orbiter spacecraft were launched by Atlas-Agena-D launch
vehicles.

The Lunar Orbiters had an ingenious imaging system, which consisted of a
dual-lens camera, a film processing unit, a readout scanner, and a film
handling apparatus. Both lenses, a 610~mm (24~in) narrow angle high
resolution (HR) lens and an 80~mm (3.1~in) wide angle medium resolution
(MR) lens, placed their frame exposures on a single roll of 70 mm film.
The axes of the two cameras were coincident so the area imaged in the HR
frames were centered within the MR frame areas. The film was moved
during exposure to compensate for the spacecraft velocity, which was
estimated by an electro-optical sensor. The film was then processed,
scanned, and the images transmitted back to Earth.

During the Lunar Orbiter missions, the first pictures of Earth as a
whole were taken, beginning with Earth-rise over the lunar surface by
Lunar Orbiter 1 in August, 1966. The first full picture of the whole
Earth was taken by Lunar Orbiter 5 on 8 August 1967. A second photo of
the whole Earth was taken by Lunar Orbiter 5 on 10 November 1967.

\includegraphics[width=5.50000in,height=3.74787in]{media/image2.png}\\
\emph{Lunar Orbiter diagram (NASA)}

\section{Spacecraft and subsystems}\label{spacecraft-and-subsystems}

\begin{itemize}
\item
  \emph{The main bus of the Lunar Orbiter had the general shape of a
  truncated cone, 1.65~m (5~ft 5~in) tall and 1.5~m (4~ft 11~in) in
  diameter at the base.}
\item
  \emph{Some proposals were made that NASA not publish the orbital
  parameters of the Lunar Orbiter probes so that the resolution of the
  images could not be calculated through their altitude.}
\end{itemize}

The Boeing-Eastman Kodak proposal was announced by NASA on 20 December
1963. The main bus of the Lunar Orbiter had the general shape of a
truncated cone, 1.65~m (5~ft 5~in) tall and 1.5~m (4~ft 11~in) in
diameter at the base. The spacecraft was composed of three decks
supported by trusses and an arch. The equipment deck at the base of the
craft held the battery, transponder, flight programmer, inertial
reference unit (IRU), Canopus star tracker, command decoder, multiplex
encoder, traveling-wave tube amplifier (TWTA), and the photographic
system. Four solar panels were mounted to extend out from this deck with
a total span across of 3.72~m (12.2~ft). Also extending out from the
base of the spacecraft were a high gain antenna on a 1.32~m (4~ft 4~in)
boom and a low-gain antenna on a 2.08~m (6~ft 10~in) boom. Above the
equipment deck, the middle deck held the velocity control engine,
propellant, oxidizer, and pressurization tanks, Sun sensors, and
micrometeoroid detectors. The third deck consisted of a heat shield to
protect the spacecraft from the firing of the velocity control engine.
The nozzle of the engine protruded through the center of the shield.
Mounted on the perimeter of the top deck were four attitude control
thrusters.

Power of 375 W was provided by the four solar arrays containing 10,856
n/p solar cells which would directly run the spacecraft and also charge
the 12 A·h nickel-cadmium battery. The batteries were used during the
brief periods of occultation when no solar power was available.
Propulsion for major maneuvers was provided by the gimballed velocity
control engine, a hypergolic 100 pound-force (445 N) thrust Marquardt
Corp. rocket motor. Three axis stabilization and attitude control were
provided by four one lb-force (four newton) nitrogen gas jets.
Navigational knowledge was provided by five sun sensors, the Canopus
star sensor, and the inertial navigation system. Communications were via
a 10~W transmitter and the directional one meter diameter high-gain
antenna for transmission of photographs, and a 0.5~W transmitter and
omnidirectional low-gain antenna for other communications. Both
transmitters operated in the S band at about 2295~MHz. Thermal control
was maintained by a multilayer aluminized Mylar and Dacron thermal
blanket which enshrouded the main bus, special paint, insulation, and
small heaters.

The photographic system was provided by Eastman Kodak and it was derived
from a system, provided by the National Reconnaissance Office (NRO),
designed for the U-2 and SR-71 reconnaissance planes.

The camera used two lenses to simultaneously expose a wide-angle and a
high-resolution image on the same film. The wide-angle, medium
resolution mode used an 80~mm F 2.8 Xenotar lens manufactured by
Schneider Kreuznach of West Germany. The high-resolution mode used a
610~mm F 5.6 Panoramic lens manufactured by the Pacific Optical Company.

The photographic film was developed in-orbit with a semidry process, and
then it was scanned by a photomultiplier for transmission to Earth. This
system was adapted under permission of the NRO from the SAMOS E-1
reconnaissance camera, built by Kodak for a short-lived USAF
near-realtime satellite imaging project.

Originally, the Air Force had offered NASA several spare cameras from
the KH-7 GAMBIT program, but then authorities became concerned over
security surrounding the classified cameras, including the possibility
of images of the Moon giving away their resolution. Some proposals were
made that NASA not publish the orbital parameters of the Lunar Orbiter
probes so that the resolution of the images could not be calculated
through their altitude. In the end, NASA's existing camera systems,
while lower resolution, proved to be adequate for the needs of the
mission.

\section{Potential backup}\label{potential-backup}

\begin{itemize}
\item
  \emph{NASA canceled the project in the summer of 1967 after the
  complete success of the Lunar Orbiters.}
\item
  \emph{As a backup for Lunar Orbiter program, NASA and the NRO
  cooperated on the Lunar Mapping and Survey System (LM\&SS), based on
  the KH-7 reconnaissance satellite.}
\item
  \emph{Replacing the Lunar Module in the Saturn V, Apollo astronauts
  would operate LM\&SS remotely in lunar orbit.}
\end{itemize}

As a backup for Lunar Orbiter program, NASA and the NRO cooperated on
the Lunar Mapping and Survey System (LM\&SS), based on the KH-7
reconnaissance satellite. Replacing the Lunar Module in the Saturn V,
Apollo astronauts would operate LM\&SS remotely in lunar orbit. NASA
canceled the project in the summer of 1967 after the complete success of
the Lunar Orbiters.

\includegraphics[width=5.50000in,height=4.32552in]{media/image3.jpg}\\
\emph{Lunar Orbiter camera (NASA)}

\section{Results}\label{results}

\begin{itemize}
\item
  \emph{The Lunar Orbiter program was managed by NASA Langley Research
  Center at a total cost of roughly \$200 million.}
\item
  \emph{Lunar Orbiter 4\\
  Launched May 4, 1967\\
  Imaged Moon: May 11 to 26, 1967\\
  Impact with Moon: Approximately October 31, 1967\\
  Lunar mapping mission}
\item
  \emph{Below is the flight log information of the five Lunar Orbiter
  photographic missions:}
\end{itemize}

The Lunar Orbiter program consisted of five spacecraft which returned
photography of 99 percent of the surface of the Moon (near and far side)
with resolution down to 1 meter (3~ft 3~in). Altogether the Orbiters
returned 2180 high resolution and 882 medium resolution frames. The
micrometeoroid experiments recorded 22 impacts showing the average
micrometeoroid flux near the Moon was about two orders of magnitude
greater than in interplanetary space, but slightly less than in the
near-Earth environment. The radiation experiments confirmed that the
design of Apollo hardware would protect the astronauts from average and
greater than average short term exposure to solar particle events.

The use of Lunar Orbiters for tracking to evaluate the Manned Space
Flight Network tracking stations and Apollo Orbit Determination Program
was successful, with three of the Lunar Orbiters (2, 3, and 5) being
tracked simultaneously from August through October 1967. The Lunar
Orbiters were all eventually commanded to crash on the Moon before their
attitude control fuel ran out so they would not present navigational or
communications hazards to later Apollo flights. The Lunar Orbiter
program was managed by NASA Langley Research Center at a total cost of
roughly \$200 million.

Doppler tracking of the five orbiters allowed mapping of the
gravitational field of the Moon and discovery of mass concentrations
(mascons), or gravitational highs, which were located in the centers of
some (but not all) of the lunar maria.

Below is the flight log information of the five Lunar Orbiter
photographic missions:

Lunar Orbiter 1\\
Launched August 10, 1966\\
Imaged Moon: August 18 to 29, 1966\\
Impact with Moon: October 29, 1966\\
Apollo landing site survey mission

Lunar Orbiter 2\\
Launched November 6, 1966\\
Imaged Moon: November 18 to 25, 1966\\
Impact with Moon: October 11, 1967\\
Apollo landing site survey mission

Lunar Orbiter 3\\
Launched February 5, 1967\\
Imaged Moon: February 15 to 23, 1967\\
Impact with Moon: October 9, 1967\\
Apollo landing site survey mission

Lunar Orbiter 4\\
Launched May 4, 1967\\
Imaged Moon: May 11 to 26, 1967\\
Impact with Moon: Approximately October 31, 1967\\
Lunar mapping mission

Lunar Orbiter 5\\
Launched August 1, 1967\\
Imaged Moon: August 6 to 18, 1967\\
Impact with Moon: January 31, 1968\\
Lunar mapping and hi-res survey mission

\section{Data availability}\label{data-availability}

\begin{itemize}
\item
  \emph{Several atlases and books featuring Lunar Orbiter photographs
  have been published.}
\item
  \emph{In part because of high interest in the data and in part because
  that atlas is out of print, the task was undertaken at the Lunar and
  Planetary Institute to scan the large-format prints of Lunar Orbiter
  data.}
\item
  \emph{These were made available online as the Digital Lunar Orbiter
  Photographic Atlas of the Moon.}
\item
  \emph{The film data were used to create hand-made mosaics of Lunar
  Orbiter frames.}
\end{itemize}

The Lunar Orbiter orbital photographs were transmitted to Earth as
analog data after onboard scanning of the original film into a series of
strips. The data were written to magnetic tape and also to film. The
film data were used to create hand-made mosaics of Lunar Orbiter frames.
Each LO exposure resulted in two photographs: medium-resolution frames
recorded by the 80-mm focal-length lens and high-resolution frames
recorded by the 610-mm focal length lens. Due to their large size, HR
frames were divided into three sections, or sub-frames. Large-format
prints (16 by 20 inches (410~mm ×~510~mm)) from the mosaics were created
and several copies were distributed across the U.S. to NASA image and
data libraries known as Regional Planetary Information Facilities. The
resulting outstanding views were of generally very high spatial
resolution and covered a substantial portion of the lunar surface, but
they suffered from a "venetian blind" striping, missing or duplicated
data, and frequent saturation effects that hampered their use. For many
years these images have been the basis of much of lunar scientific
research. Because they were obtained at low to moderate Sun angles, the
Lunar Orbiter photographic mosaics are particularly useful for studying
the morphology of lunar topographic features.

Several atlases and books featuring Lunar Orbiter photographs have been
published. Perhaps the most definitive was that of Bowker and Hughes
(1971); it contained 675 photographic plates with approximately global
coverage of the Moon. In part because of high interest in the data and
in part because that atlas is out of print, the task was undertaken at
the Lunar and Planetary Institute to scan the large-format prints of
Lunar Orbiter data. These were made available online as the Digital
Lunar Orbiter Photographic Atlas of the Moon.

\includegraphics[width=5.50000in,height=5.50000in]{media/image4.jpg}\\
\emph{A detail of an original image at the top, compared to a
reprocessed version at the bottom created by LOIRP.}

\section{Data recovery and
digitization}\label{data-recovery-and-digitization}

\begin{itemize}
\item
  \emph{Data from Lunar Orbiter missions III, IV and V were included in
  the global mosaic.}
\item
  \emph{Because of its emphasis on construction of a global mosaic, this
  project only scanned about 15\% of the available Lunar Orbiter
  photographic frames.}
\item
  \emph{Almost all of the Lunar Orbiter images have been successfully
  recovered as of February 2014 and are undergoing digital processing
  before being submitted to NASA's Planetary Data System.}
\end{itemize}

In 2000, the Astrogeology Research Program of the US Geological Survey
in Flagstaff, Arizona was funded by NASA (as part of the Lunar Orbiter
Digitization Project) to scan at 25 micrometer resolution archival LO
positive film strips that were produced from the original data. The goal
was to produce a global mosaic of the Moon using the best available
Lunar Orbiter frames (largely the same coverage as that of Bowker and
Hughes, 1971). The frames were constructed from scanned film strips;
they were digitally constructed, geometrically controlled, and
map-projected without the stripes that had been noticeable in the
original photographic frames. Because of its emphasis on construction of
a global mosaic, this project only scanned about 15\% of the available
Lunar Orbiter photographic frames. Data from Lunar Orbiter missions III,
IV and V were included in the global mosaic.

In addition, the USGS digitization project created frames from very high
resolution Lunar Orbiter images for several 'sites of scientific
interest.' These sites had been identified in the 1960s when the Apollo
landing sites were being selected. Frames for sites such as the Apollo
12 landing site, the Marius Hills, and the Sulpicius Gallus rille have
been released.

In 2007, the Lunar Orbiter Image Recovery Project (LOIRP) began a
process to convert the Lunar Orbiter Images directly from the original
Ampex FR-900 analog video recordings of the spacecraft data to digital
image format, a change which provided vastly improved resolution over
the original images released in the 1960s. The first of these restored
images were released in late 2008. Almost all of the Lunar Orbiter
images have been successfully recovered as of February 2014 and are
undergoing digital processing before being submitted to NASA's Planetary
Data System.

\section{See also}\label{see-also}

\begin{itemize}
\item
  \emph{Lunar Orbiter Image Recovery Project}
\item
  \emph{Lunar Reconnaissance Orbiter}
\end{itemize}

Astrogeology Research Program

Lunar Orbiter Image Recovery Project

Lunar Reconnaissance Orbiter

Ranger program

Surveyor program

Apollo program

Luna programme

Exploration of the Moon

Robert J. Helberg

Norman L. Crabill

\section{References}\label{references}

\section{External links}\label{external-links}

\begin{itemize}
\item
  \emph{Astronautix on Lunar Orbiter}
\item
  \emph{Digital Lunar Orbiter Photographic Atlas of the Moon Lunar and
  Planetary Institute}
\item
  \emph{DESTINATION MOON: A history of the Lunar Orbiter Program (HTML)}
\item
  \emph{Lunar Orbiter to the Moon National Space Science Data Center
  (NSSDC)}
\item
  \emph{Exploring the Moon: Lunar Orbiter Program Lunar and Planetary
  Institute}
\end{itemize}

B.K. Byers, 1977, DESTINATION MOON: A history of the Lunar Orbiter
Program (PDF), NASA TM X-3487

DESTINATION MOON: A history of the Lunar Orbiter Program (PDF) 1977,
NASA TM X-3487

DESTINATION MOON: A history of the Lunar Orbiter Program (HTML)

The above links lead to a whole book on the Lunar Orbiter program. For
the HTML one, scroll down to see the table of contents link.

T.P. Hansen, 1970 Guide to Lunar Orbiter Photographs (PDF), NASA SP-242

Guide to Lunar Orbiter Photographs (PDF) 1970, NASA SP-242

Lunar Orbiter to the Moon National Space Science Data Center (NSSDC)

Astronautix on Lunar Orbiter

Exploring the Moon: Lunar Orbiter Program Lunar and Planetary Institute

Digital Lunar Orbiter Photographic Atlas of the Moon Lunar and Planetary
Institute

Lunar Orbiter Photo Gallery Over 2,600 high- and moderate-resolution
photographs

Lunar Orbiter Digitization Project USGS Astrogeology Science Center

Lunar Orbiter Digitized Film Project USGS Astrogeology Science Center,
NASA Planetary Data System Archive

Lunar Orbiter Image Recovery Project (LOIRP) Overview

Close Up of the Moon: A Look at Lunar Orbiter on YouTube NASA
Documentary from 1967.
