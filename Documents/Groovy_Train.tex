\textbf{From Wikipedia, the free encyclopedia}

https://en.wikipedia.org/wiki/Groovy\_Train\\
Licensed under CC BY-SA 3.0:\\
https://en.wikipedia.org/wiki/Wikipedia:Text\_of\_Creative\_Commons\_Attribution-ShareAlike\_3.0\_Unported\_License

\section{Groovy Train}\label{groovy-train}

\begin{itemize}
\item
  \emph{6 on the UK Singles Chart, no.}
\item
  \emph{15 on the US Billboard Modern Rock Tracks chart.}
\item
  \emph{41 on the US Billboard Hot 100, and no.}
\item
  \emph{"Groovy Train" was the second single released by Liverpool-based
  group The Farm.}
\item
  \emph{"Groovy Train" featured on the influential 1990 Madchester
  compilation album Happy Daze.}
\item
  \emph{The single reached no.}
\end{itemize}

"Groovy Train" was the second single released by Liverpool-based group
The Farm. It was released in August 1990 as the first single from their
debut album Spartacus (which would be released in April 1991), having
been produced by Graham "Suggs" McPherson of Madness and Terry Farley.
The single reached no. 6 on the UK Singles Chart, no. 41 on the US
Billboard Hot 100, and no. 15 on the US Billboard Modern Rock Tracks
chart.

It contains a distinctive guitar intro by Keith Mullin which was
possibly his most significant contribution to any one song. "Groovy
Train" featured on the influential 1990 Madchester compilation album
Happy Daze.

The video for the single was filmed at Pleasureland Southport and
features a cameo from actor Bill Dean, who at the time was in Liverpool
soap opera Brookside. His character, Harry Cross was a retired train
driver, and Dean is seen in the video driving a train with the band
aboard.

\section{Chart performance}\label{chart-performance}

\section{References}\label{references}

\section{External links}\label{external-links}

\begin{itemize}
\item
  \emph{Video for "Groovy Train"}
\end{itemize}

Lyrics

Cover art

Video for "Groovy Train"
