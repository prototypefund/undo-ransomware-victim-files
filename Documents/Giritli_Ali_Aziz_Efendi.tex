\textbf{From Wikipedia, the free encyclopedia}

https://en.wikipedia.org/wiki/Giritli\_Ali\_Aziz\_Efendi\\
Licensed under CC BY-SA 3.0:\\
https://en.wikipedia.org/wiki/Wikipedia:Text\_of\_Creative\_Commons\_Attribution-ShareAlike\_3.0\_Unported\_License

\includegraphics[width=4.49440in,height=5.50000in]{media/image1.jpg}\\
\emph{Giritli Ali Aziz Efendi}

\section{Giritli Ali Aziz Efendi}\label{giritli-ali-aziz-efendi}

\begin{itemize}
\item
  \emph{Giritli Ali Aziz Efendi (1749, in Kandiye (Heraklion) -- 29
  October 1798, in Berlin) was an Ottoman ambassador and an Ottoman
  author of the late-18th century and he is notable for his novel
  "Muhayyelât" (Imaginations), a unique work of fiction blending
  personal and fantastic themes, well in the current of the traditional
  Ottoman prose, but also exhibiting influences from Western
  literature.}
\end{itemize}

Giritli Ali Aziz Efendi (1749, in Kandiye (Heraklion) -- 29 October
1798, in Berlin) was an Ottoman ambassador and an Ottoman author of the
late-18th century and he is notable for his novel "Muhayyelât"
(Imaginations), a unique work of fiction blending personal and fantastic
themes, well in the current of the traditional Ottoman prose, but also
exhibiting influences from Western literature.

\section{Biography}\label{biography}

\begin{itemize}
\item
  \emph{He was born in Kandiye (Crete) as the son of Tahmisçi Mehmed
  Efendi, who was the defterdar of the Crete Eyalet, in 1749.}
\item
  \emph{He rose through the Ottoman hierarchy and was sent as ambassador
  to Prussia in 1796 and he died in Berlin in 1798.}
\item
  \emph{His burial marked also the opening of the first Turkish or
  Muslim cemetery in Berlin.}
\end{itemize}

He was born in Kandiye (Crete) as the son of Tahmisçi Mehmed Efendi, who
was the defterdar of the Crete Eyalet, in 1749. The details on his life
are rather sparse and scattered. He rose through the Ottoman hierarchy
and was sent as ambassador to Prussia in 1796 and he died in Berlin in
1798. His burial marked also the opening of the first Turkish or Muslim
cemetery in Berlin.

\section{Muhayyelât}\label{muhayyeluxe2t}

\begin{itemize}
\item
  \emph{Ali Aziz Efendi also wrote further and shorter works of prose,
  which present as complementary extensions to Muhayyelât, as well as
  some poetry, and kept a correspondence with a number of notable
  figures of his time, both Ottoman and Western.}
\item
  \emph{He is also cited for a short sefâretnâme he wrote relating his
  introduction to his mission as the ambassador of the Ottoman Empire in
  Prussia.}
\end{itemize}

Consisting in three parts and written in a laconical style contrasting
with its content, where djinns and fairies surge from within contexts
drawn from ordinary real life situations, Ali Aziz Efendi often pursues
by pulling the reader towards description of magic and to extraordinary
occurrences. Inspired by a much older story written both in Arabic and
Assyrian, the author also displays in his work his deep knowledge of
sufism, hurufism and Bektashi traditions. Muhayyelât is considered to be
an early precursor of the new Turkish literature to emerge in the
Tanzimat period of the 19th century. It also influenced Tanzimat
literature directly when the manuscript was printed in 1867 and became a
very popular book of the time. His work is re-discovered by Turkey's
reading public rather recently and is increasingly admitted as a
classic.

Ali Aziz Efendi also wrote further and shorter works of prose, which
present as complementary extensions to Muhayyelât, as well as some
poetry, and kept a correspondence with a number of notable figures of
his time, both Ottoman and Western.

He is also cited for a short sefâretnâme he wrote relating his
introduction to his mission as the ambassador of the Ottoman Empire in
Prussia.

\section{Works}\label{works}

\begin{itemize}
\item
  \emph{Muhayyelât (Imaginations), 312p., .mw-parser-output
  cite.citation\{font-style:inherit\}.mw-parser-output .citation
  q\{quotes:"\textbackslash{}"""\textbackslash{}"""'""'"\}.mw-parser-output
  .citation .cs1-lock-free
  a\{background:url("//upload.wikimedia.org/wikipedia/commons/thumb/6/65/Lock-green.svg/9px-Lock-green.svg.png")no-repeat;background-position:right
  .1em center\}.mw-parser-output .citation .cs1-lock-limited
  a,.mw-parser-output .citation .cs1-lock-registration
  a\{background:url("//upload.wikimedia.org/wikipedia/commons/thumb/d/d6/Lock-gray-alt-2.svg/9px-Lock-gray-alt-2.svg.png")no-repeat;background-position:right
  .1em center\}.mw-parser-output .citation .cs1-lock-subscription
  a\{background:url("//upload.wikimedia.org/wikipedia/commons/thumb/a/aa/Lock-red-alt-2.svg/9px-Lock-red-alt-2.svg.png")no-repeat;background-position:right
  .1em center\}.mw-parser-output .cs1-subscription,.mw-parser-output
  .cs1-registration\{color:\#555\}.mw-parser-output .cs1-subscription
  span,.mw-parser-output .cs1-registration span\{border-bottom:1px
  dotted;cursor:help\}.mw-parser-output .cs1-ws-icon
  a\{background:url("//upload.wikimedia.org/wikipedia/commons/thumb/4/4c/Wikisource-logo.svg/12px-Wikisource-logo.svg.png")no-repeat;background-position:right
  .1em center\}.mw-parser-output
  code.cs1-code\{color:inherit;background:inherit;border:inherit;padding:inherit\}.mw-parser-output
  .cs1-hidden-error\{display:none;font-size:100\%\}.mw-parser-output
  .cs1-visible-error\{font-size:100\%\}.mw-parser-output
  .cs1-maint\{display:none;color:\#33aa33;margin-left:0.3em\}.mw-parser-output
  .cs1-subscription,.mw-parser-output
  .cs1-registration,.mw-parser-output
  .cs1-format\{font-size:95\%\}.mw-parser-output
  .cs1-kern-left,.mw-parser-output
  .cs1-kern-wl-left\{padding-left:0.2em\}.mw-parser-output
  .cs1-kern-right,.mw-parser-output
  .cs1-kern-wl-right\{padding-right:0.2em\}ISBN~975-6295-42-2}
\end{itemize}

Muhayyelât (Imaginations), 312p., .mw-parser-output
cite.citation\{font-style:inherit\}.mw-parser-output .citation
q\{quotes:"\textbackslash{}"""\textbackslash{}"""'""'"\}.mw-parser-output
.citation .cs1-lock-free
a\{background:url("//upload.wikimedia.org/wikipedia/commons/thumb/6/65/Lock-green.svg/9px-Lock-green.svg.png")no-repeat;background-position:right
.1em center\}.mw-parser-output .citation .cs1-lock-limited
a,.mw-parser-output .citation .cs1-lock-registration
a\{background:url("//upload.wikimedia.org/wikipedia/commons/thumb/d/d6/Lock-gray-alt-2.svg/9px-Lock-gray-alt-2.svg.png")no-repeat;background-position:right
.1em center\}.mw-parser-output .citation .cs1-lock-subscription
a\{background:url("//upload.wikimedia.org/wikipedia/commons/thumb/a/aa/Lock-red-alt-2.svg/9px-Lock-red-alt-2.svg.png")no-repeat;background-position:right
.1em center\}.mw-parser-output .cs1-subscription,.mw-parser-output
.cs1-registration\{color:\#555\}.mw-parser-output .cs1-subscription
span,.mw-parser-output .cs1-registration span\{border-bottom:1px
dotted;cursor:help\}.mw-parser-output .cs1-ws-icon
a\{background:url("//upload.wikimedia.org/wikipedia/commons/thumb/4/4c/Wikisource-logo.svg/12px-Wikisource-logo.svg.png")no-repeat;background-position:right
.1em center\}.mw-parser-output
code.cs1-code\{color:inherit;background:inherit;border:inherit;padding:inherit\}.mw-parser-output
.cs1-hidden-error\{display:none;font-size:100\%\}.mw-parser-output
.cs1-visible-error\{font-size:100\%\}.mw-parser-output
.cs1-maint\{display:none;color:\#33aa33;margin-left:0.3em\}.mw-parser-output
.cs1-subscription,.mw-parser-output .cs1-registration,.mw-parser-output
.cs1-format\{font-size:95\%\}.mw-parser-output
.cs1-kern-left,.mw-parser-output
.cs1-kern-wl-left\{padding-left:0.2em\}.mw-parser-output
.cs1-kern-right,.mw-parser-output
.cs1-kern-wl-right\{padding-right:0.2em\}ISBN~975-6295-42-2

\section{See also}\label{see-also}

\begin{itemize}
\item
  \emph{Ahmed Resmî Efendi}
\end{itemize}

Ahmed Resmî Efendi

Sefâretnâme

Cretan Turks

\section{Sources}\label{sources}

\section{References}\label{references}
