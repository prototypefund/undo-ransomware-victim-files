\textbf{From Wikipedia, the free encyclopedia}

https://en.wikipedia.org/wiki/Mahatma\%20Gandhi\\
Licensed under CC BY-SA 3.0:\\
https://en.wikipedia.org/wiki/Wikipedia:Text\_of\_Creative\_Commons\_Attribution-ShareAlike\_3.0\_Unported\_License

\section{Mahatma Gandhi}\label{mahatma-gandhi}

\begin{itemize}
\item
  \emph{Mohandas Karamchand Gandhi (/ˈɡɑːndi, ˈɡændi/;
  Hindustani:~{[}ˈmoːɦəndaːs ˈkərəmtʃənd ˈɡaːndʱi{]} (listen); 2 October
  1869~-- 30 January 1948) was an Indian activist who was the leader of
  the Indian independence movement against British colonial rule.}
\item
  \emph{Gandhi's birthday, 2 October, is commemorated in India as Gandhi
  Jayanti, a national holiday, and worldwide as the International Day of
  Nonviolence.}
\item
  \emph{Some Indians thought Gandhi was too accommodating.}
\end{itemize}

Mohandas Karamchand Gandhi (/ˈɡɑːndi, ˈɡændi/;
Hindustani:~{[}ˈmoːɦəndaːs ˈkərəmtʃənd ˈɡaːndʱi{]} (listen); 2 October
1869~-- 30 January 1948) was an Indian activist who was the leader of
the Indian independence movement against British colonial rule.
Employing nonviolent civil disobedience, Gandhi led India to
independence and inspired movements for civil rights and freedom across
the world. The honorific Mahātmā (Sanskrit: "high-souled", "venerable")
was applied to him first in 1914 in South Africa and is now used
worldwide. In India, he was also called Bapu, a term that he preferred
(Gujarati: endearment for father, papa), and Gandhi ji, and is known as
the Father of the Nation.

Born and raised in a Hindu merchant caste family in coastal Gujarat,
India, and trained in law at the Inner Temple, London, Gandhi first
employed nonviolent civil disobedience as an expatriate lawyer in South
Africa, in the resident Indian community's struggle for civil rights.
After his return to India in 1915, he set about organising peasants,
farmers, and urban labourers to protest against excessive land-tax and
discrimination. Assuming leadership of the Indian National Congress in
1921, Gandhi led nationwide campaigns for various social causes and for
achieving Swaraj or self-rule.

Gandhi led Indians in challenging the British-imposed salt tax with the
400~km (250~mi) Dandi Salt March in 1930, and later in calling for the
British to Quit India in 1942. He was imprisoned for many years, upon
many occasions, in both South Africa and India. He lived modestly in a
self-sufficient residential community and wore the traditional Indian
dhoti and shawl, woven with yarn hand-spun on a charkha. He ate simple
vegetarian food, and also undertook long fasts as a means of both
self-purification and political protest.

Gandhi's vision of an independent India based on religious pluralism was
challenged in the early 1940s by a new Muslim nationalism which was
demanding a separate Muslim homeland carved out of India. In August
1947, Britain granted independence, but the British Indian Empire was
partitioned into two dominions, a Hindu-majority India and
Muslim-majority Pakistan. As many displaced Hindus, Muslims, and Sikhs
made their way to their new lands, religious violence broke out,
especially in the Punjab and Bengal. Eschewing the official celebration
of independence in Delhi, Gandhi visited the affected areas, attempting
to provide solace. In the months following, he undertook several fasts
unto death to stop religious violence. The last of these, undertaken on
12 January 1948 when he was 78, also had the indirect goal of pressuring
India to pay out some cash assets owed to Pakistan. Some Indians thought
Gandhi was too accommodating. Among them was Nathuram Godse, a Hindu
nationalist, who assassinated Gandhi on 30 January 1948 by firing three
bullets into his chest. Captured along with many of his co-conspirators
and collaborators, Godse and his co-conspirator Narayan Apte were tried,
convicted and executed while many of their other accomplices were given
prison sentences.

Gandhi's birthday, 2 October, is commemorated in India as Gandhi
Jayanti, a national holiday, and worldwide as the International Day of
Nonviolence.

\section{Biography}\label{biography}

\includegraphics[width=4.10667in,height=5.50000in]{media/image1.jpg}\\
\emph{Gandhi (right) with his eldest brother Laxmidas in 1886.}

\section{Early life and background}\label{early-life-and-background}

\begin{itemize}
\item
  \emph{Gandhi's father was of Modh Baniya caste in the varna of
  Vaishya.}
\item
  \emph{In late 1885, Gandhi's father Karamchand died.}
\item
  \emph{Gandhi's father Karamchand was Hindu and his mother Putlibai was
  from a Pranami Vaishnava Hindu family.}
\item
  \emph{In November 1887, the 18-year-old Gandhi graduated from high
  school in Ahmedabad.}
\item
  \emph{The two deaths anguished Gandhi.}
\end{itemize}

Mohandas Karamchand Gandhi was born on 2 October 1869 into a Gujarati
Hindu Modh Baniya family in Porbandar (also known as Sudamapuri), a
coastal town on the Kathiawar Peninsula and then part of the small
princely state of Porbandar in the Kathiawar Agency of the Indian
Empire. His father, Karamchand Uttamchand Gandhi (1822--1885), served as
the diwan (chief minister) of Porbandar state.

Although he only had an elementary education and had previously been a
clerk in the state administration, Karamchand proved a capable chief
minister. During his tenure, Karamchand married four times. His first
two wives died young, after each had given birth to a daughter, and his
third marriage was childless. In 1857, Karamchand sought his third
wife's permission to remarry; that year, he married Putlibai
(1844--1891), who also came from Junagadh, and was from a Pranami
Vaishnava family. Karamchand and Putlibai had three children over the
ensuing decade: a son, Laxmidas (c. 1860--1914); a daughter, Raliatbehn
(1862--1960); and another son, Karsandas (c. 1866--1913).

On 2 October 1869, Putlibai gave birth to her last child, Mohandas, in a
dark, windowless ground-floor room of the Gandhi family residence in
Porbandar city. As a child, Gandhi was described by his sister Raliat as
"restless as mercury, either playing or roaming about. One of his
favourite pastimes was twisting dogs' ears." The Indian classics,
especially the stories of Shravana and king Harishchandra, had a great
impact on Gandhi in his childhood. In his autobiography, he admits that
they left an indelible impression on his mind. He writes: "It haunted me
and I must have acted Harishchandra to myself times without number."
Gandhi's early self-identification with truth and love as supreme values
is traceable to these epic characters.

The family's religious background was eclectic. Gandhi's father
Karamchand was Hindu and his mother Putlibai was from a Pranami
Vaishnava Hindu family. Gandhi's father was of Modh Baniya caste in the
varna of Vaishya. His mother came from the medieval Krishna bhakti-based
Pranami tradition, whose religious texts include the Bhagavad Gita, the
Bhagavata Purana, and a collection of 14 texts with teachings that the
tradition believes to include the essence of the Vedas, the Quran and
the Bible. Gandhi was deeply influenced by his mother, an extremely
pious lady who "would not think of taking her meals without her daily
prayers...she would take the hardest vows and keep them without
flinching. To keep two or three consecutive fasts was nothing to her."

In 1874, Gandhi's father Karamchand left Porbandar for the smaller state
of Rajkot, where he became a counsellor to its ruler, the Thakur Sahib;
though Rajkot was a less prestigious state than Porbandar, the British
regional political agency was located there, which gave the state's
diwan a measure of security. In 1876, Karamchand became diwan of Rajkot
and was succeeded as diwan of Porbandar by his brother Tulsidas. His
family then rejoined him in Rajkot.

At age 9, Gandhi entered the local school in Rajkot, near his home.
There he studied the rudiments of arithmetic, history, the Gujarati
language and geography. At age 11, he joined the High School in Rajkot.
He was an average student, won some prizes, but was a shy and tongue
tied student, with no interest in games; his only companions were books
and school lessons.

While at high school, Gandhi's elder brother introduced him to a Muslim
friend named Sheikh Mehtab. Mehtab was older in age, taller and
encouraged the strictly vegetarian boy to eat meat to gain height. He
also took Mohandas to a brothel one day, though Mohandas "was struck
blind and dumb in this den of vice," rebuffed the prostitutes' advances
and was promptly sent out of the brothel. The experience caused Mohandas
mental anguish, and he abandoned the company of Mehtab.

In May 1883, the 13-year-old Mohandas was married to 14-year-old
Kasturbai Makhanji Kapadia (her first name was usually shortened to
"Kasturba", and affectionately to "Ba") in an arranged marriage,
according to the custom of the region at that time. In the process, he
lost a year at school, but was later allowed to make up by accelerating
his studies. His wedding was a joint event, where his brother and cousin
were also married. Recalling the day of their marriage, he once said,
"As we didn't know much about marriage, for us it meant only wearing new
clothes, eating sweets and playing with relatives." As was prevailing
tradition, the adolescent bride was to spend much time at her parents'
house, and away from her husband.

Writing many years later, Mohandas described with regret the lustful
feelings he felt for his young bride, "even at school I used to think of
her, and the thought of nightfall and our subsequent meeting was ever
haunting me." He later recalled feeling jealous and possessive of her,
such as when she would visit a temple with her girlfriends, and being
sexually lustful in his feelings for her.

In late 1885, Gandhi's father Karamchand died. Gandhi, then 16 years
old, and his wife of age 17 had their first baby, who survived only a
few days. The two deaths anguished Gandhi. The Gandhi couple had four
more children, all sons: Harilal, born in 1888; Manilal, born in 1892;
Ramdas, born in 1897; and Devdas, born in 1900.

In November 1887, the 18-year-old Gandhi graduated from high school in
Ahmedabad. In January 1888, he enrolled at Samaldas College in Bhavnagar
State, then the sole degree-granting institution of higher education in
the region. But he dropped out and returned to his family in Porbandar.

\includegraphics[width=4.16461in,height=5.50000in]{media/image2.jpg}\\
\emph{Gandhi in London as a law student}

\section{English barrister}\label{english-barrister}

\begin{itemize}
\item
  \emph{Gandhi inquired about his pay for the work.}
\item
  \emph{The head of the community knew Gandhi's father.}
\item
  \emph{Putlibai gave Gandhi her permission and blessing.}
\item
  \emph{Gandhi wanted to go.}
\item
  \emph{Gandhi informed them of his promise to his mother and her
  blessings.}
\item
  \emph{Gandhi's brother Laxmidas, who was already a lawyer, cheered
  Gandhi's London studies plan and offered to support him.}
\end{itemize}

Gandhi came from a poor family, and he had dropped out of the cheapest
college he could afford. Mavji Dave Joshiji, a Brahmin priest and family
friend, advised Gandhi and his family that he should consider law
studies in London. In July 1888, his wife Kasturba gave birth to their
first surviving son, Harilal. His mother was not comfortable about
Gandhi leaving his wife and family, and going so far from home. Gandhi's
uncle Tulsidas also tried to dissuade his nephew. Gandhi wanted to go.
To persuade his wife and mother, Gandhi made a vow in front of his
mother that he would abstain from meat, alcohol and women. Gandhi's
brother Laxmidas, who was already a lawyer, cheered Gandhi's London
studies plan and offered to support him. Putlibai gave Gandhi her
permission and blessing.

On 10 August 1888, Gandhi aged 18, left Porbandar for Mumbai, then known
as Bombay. Upon arrival, he stayed with the local Modh Bania community
while waiting for the ship travel arrangements. The head of the
community knew Gandhi's father. After learning Gandhi's plans, he and
other elders warned Gandhi that England would tempt him to compromise
his religion, and eat and drink in Western ways. Gandhi informed them of
his promise to his mother and her blessings. The local chief disregarded
it, and excommunicated him from his caste. But Gandhi ignored this, and
on 4 September, he sailed from Bombay to London. His brother saw him
off.\\
Gandhi attended University College, London which is a constituent
college of University of London.

At UCL, he studied law and jurisprudence and was invited to enroll at
Inner Temple with the intention of becoming a barrister. His childhood
shyness and self withdrawal had continued through his teens, and he
remained so when he arrived in London, but he joined a public speaking
practice group and overcame this handicap to practise law.

His time in London was influenced by the vow he had made to his mother.
He tried to adopt "English" customs, including taking dancing lessons.
However, he could not appreciate the bland vegetarian food offered by
his landlady and was frequently hungry until he found one of London's
few vegetarian restaurants. Influenced by Henry Salt's writing, he
joined the Vegetarian Society, was elected to its executive committee,
and started a local Bayswater chapter. Some of the vegetarians he met
were members of the Theosophical Society, which had been founded in 1875
to further universal brotherhood, and which was devoted to the study of
Buddhist and Hindu literature. They encouraged Gandhi to join them in
reading the Bhagavad Gita both in translation as well as in the
original.

Gandhi, at age 22, was called to the bar in June 1891 and then left
London for India, where he learned that his mother had died while he was
in London and that his family had kept the news from him. His attempts
at establishing a law practice in Bombay failed because he was
psychologically unable to cross-examine witnesses. He returned to Rajkot
to make a modest living drafting petitions for litigants, but he was
forced to stop when he ran afoul of a British officer Sam Sunny.

In 1893, a Muslim merchant in Kathiawar named Dada Abdullah contacted
Gandhi. Abdullah owned a large successful shipping business in South
Africa. His distant cousin in Johannesburg needed a lawyer, and they
preferred someone with Kathiawari heritage. Gandhi inquired about his
pay for the work. They offered a total salary of £105 plus travel
expenses. He accepted it, knowing that it would be at least one-year
commitment in the Colony of Natal, South Africa, also a part of the
British Empire.

\includegraphics[width=5.50000in,height=3.22208in]{media/image3.jpg}\\
\emph{Gandhi with the stretcher-bearers of the Indian Ambulance Corps
during the Boer War.}

\includegraphics[width=5.50000in,height=3.58823in]{media/image4.jpg}\\
\emph{Gandhi and his wife Kasturba (1902)}

\section{Civil rights activist in South Africa
(1893--1914)}\label{civil-rights-activist-in-south-africa-18931914}

\begin{itemize}
\item
  \emph{Gandhi urged Indians to defy the new law and to suffer the
  punishments for doing so.}
\item
  \emph{Gandhi raised eleven hundred Indian volunteers, to support
  British combat troops against the Boers.}
\item
  \emph{Gandhi and thirty-seven other Indians received the Queen's South
  Africa Medal.}
\item
  \emph{Gandhi began to question his people's standing in the British
  Empire.}
\end{itemize}

In April 1893, Gandhi aged 23, set sail for South Africa to be the
lawyer for Abdullah's cousin. He spent 21 years in South Africa, where
he developed his political views, ethics and politics.

Immediately upon arriving in South Africa, Gandhi faced discrimination
because of his skin colour and heritage, like all people of colour. He
was not allowed to sit with European passengers in the stagecoach and
told to sit on the floor near the driver, then beaten when he refused;
elsewhere he was kicked into a gutter for daring to walk near a house,
in another instance thrown off a train at Pietermaritzburg after
refusing to leave the first-class. He sat in the train station,
shivering all night and pondering if he should return to India or
protest for his rights. He chose to protest and was allowed to board the
train the next day. In another incident, the magistrate of a Durban
court ordered Gandhi to remove his turban, which he refused to do.
Indians were not allowed to walk on public footpaths in South Africa.
Gandhi was kicked by a police officer out of the footpath onto the
street without warning.

When Gandhi arrived in South Africa, according to Herman, he thought of
himself as "a Briton first, and an Indian second". However, the
prejudice against him and his fellow Indians from British people that
Gandhi experienced and observed deeply bothered him. He found it
humiliating, struggling to understand how some people can feel honour or
superiority or pleasure in such inhumane practices. Gandhi began to
question his people's standing in the British Empire.

The Abdullah case that had brought him to South Africa concluded in May
1894, and the Indian community organised a farewell party for Gandhi as
he prepared to return to India. However, a new Natal government
discriminatory proposal led to Gandhi extending his original period of
stay in South Africa. He planned to assist Indians in opposing a bill to
deny them the right to vote, a right then proposed to be an exclusive
European right. He asked Joseph Chamberlain, the British Colonial
Secretary, to reconsider his position on this bill. Though unable to
halt the bill's passage, his campaign was successful in drawing
attention to the grievances of Indians in South Africa. He helped found
the Natal Indian Congress in 1894, and through this organisation, he
moulded the Indian community of South Africa into a unified political
force. In January 1897, when Gandhi landed in Durban, a mob of white
settlers attacked him and he escaped only through the efforts of the
wife of the police superintendent. However, he refused to press charges
against any member of the mob.

During the Boer War, Gandhi volunteered in 1900 to form a group of
stretcher-bearers as the Natal Indian Ambulance Corps. According to
Arthur Herman, Gandhi wanted to disprove the imperial British stereotype
that Hindus were not fit for "manly" activities involving danger and
exertion, unlike the Muslim "martial races". Gandhi raised eleven
hundred Indian volunteers, to support British combat troops against the
Boers. They were trained and medically certified to serve on the front
lines. They were auxiliaries at the Battle of Colenso to a White
volunteer ambulance corps. At the battle of Spion Kop Gandhi and his
bearers moved to the front line and had to carry wounded soldiers for
miles to a field hospital because the terrain was too rough for the
ambulances. Gandhi and thirty-seven other Indians received the Queen's
South Africa Medal.

In 1906, the Transvaal government promulgated a new Act compelling
registration of the colony's Indian and Chinese populations. At a mass
protest meeting held in Johannesburg on 11 September that year, Gandhi
adopted his still evolving methodology of Satyagraha (devotion to the
truth), or nonviolent protest, for the first time. According to Anthony
Parel, Gandhi was also influenced by the Tamil text Tirukkuṛaḷ because
Leo Tolstoy mentioned it in their correspondence that began with "A
Letter to a Hindu". Gandhi urged Indians to defy the new law and to
suffer the punishments for doing so. Gandhi's ideas of protests,
persuasion skills and public relations had emerged. He took these back
to India in 1915.

\includegraphics[width=4.30528in,height=5.50000in]{media/image5.jpg}\\
\emph{Gandhi photographed in South Africa (1909)}

\section{Europeans, Indians and
Africans}\label{europeans-indians-and-africans}

\begin{itemize}
\item
  \emph{The medical team commanded by Gandhi operated for less than two
  months.}
\item
  \emph{Gandhi suffered persecution from the beginning in South Africa.}
\item
  \emph{Gandhi focused his attention on Indians while in South Africa.}
\item
  \emph{White soldiers stopped Gandhi and team from treating the injured
  Zulu, and some African stretcher-bearers with Gandhi were shot dead by
  the British.}
\end{itemize}

Gandhi focused his attention on Indians while in South Africa. He was
not interested in politics. This changed after he was discriminated
against and bullied, such as by being thrown out of a train coach
because of his skin colour by a white train official. After several such
incidents with Whites in South Africa, Gandhi's thinking and focus
changed, and he felt he must resist this and fight for rights. He
entered politics by forming the Natal Indian Congress. According to
Ashwin Desai and Goolam Vahed, Gandhi's views on racism are contentious,
and in some cases, distressing to those who admire him. Gandhi suffered
persecution from the beginning in South Africa. Like with other coloured
people, white officials denied him his rights, and the press and those
in the streets bullied and called him a "parasite", "semi-barbarous",
"canker", "squalid coolie", "yellow man", and other epithets. People
would spit on him as an expression of racial hate.

While in South Africa, Gandhi focused on racial persecution of Indians,
but ignored those of Africans. In some cases, state Desai and Vahed, his
behaviour was one of being a willing part of racial stereotyping and
African exploitation. During a speech in September 1896, Gandhi
complained that the whites in the British colony of South Africa were
degrading Indian Hindus and Muslims to "a level of Kaffir". Scholars
cite it as an example of evidence that Gandhi at that time thought of
Indians and black South Africans differently. As another example given
by Herman, Gandhi, at age 24, prepared a legal brief for the Natal
Assembly in 1895, seeking voting rights for Indians. Gandhi cited race
history and European Orientalists' opinions that "Anglo-Saxons and
Indians are sprung from the same Aryan stock or rather the Indo-European
peoples", and argued that Indians should not be grouped with the
Africans.

Years later, Gandhi and his colleagues served and helped Africans as
nurses and by opposing racism, according to the Nobel Peace Prize winner
Nelson Mandela. The general image of Gandhi, state Desai and Vahed, has
been reinvented since his assassination as if he was always a saint,
when in reality his life was more complex, contained inconvenient truths
and was one that evolved over time. In contrast, other Africa scholars
state the evidence points to a rich history of co-operation and efforts
by Gandhi and Indian people with nonwhite South Africans against
persecution of Africans and the Apartheid.

In 1906, when the British declared war against the Zulu Kingdom in
Natal, Gandhi at age 36, sympathised with the Zulus, and encouraged the
Indian volunteers to help as an ambulance unit. He argued that Indians
should participate in the war efforts to change attitudes and
perceptions of the British people against the coloured people. Gandhi, a
group of 20 Indians and black people of South Africa volunteered as a
stretcher-bearer corps to treat wounded British soldiers and the
opposite side of the war: Zulu victims.

White soldiers stopped Gandhi and team from treating the injured Zulu,
and some African stretcher-bearers with Gandhi were shot dead by the
British. The medical team commanded by Gandhi operated for less than two
months. Gandhi volunteering to help as a "staunch loyalist" during the
Zulu and other wars made no difference in the British attitude, states
Herman, and the African experience was a part of his great
disillusionment with the West, transforming him into an "uncompromising
non-cooperator".

In 1910, Gandhi established, with the help of his friend Hermann
Kallenbach, an idealistic community they named 'Tolstoy Farm' near
Johannesburg. There he nurtured his policy of peaceful resistance.

In the years after black South Africans gained the right to vote in
South Africa (1994), Gandhi was proclaimed a national hero with numerous
monuments.

\section{Struggle for Indian independence
(1915--1947)}\label{struggle-for-indian-independence-19151947}

\begin{itemize}
\item
  \emph{Gandhi joined the Indian National Congress and was introduced to
  Indian issues, politics and the Indian people primarily by Gokhale.}
\item
  \emph{In August 1947 the British partitioned the land with India and
  Pakistan each achieving independence on terms that Gandhi
  disapproved.}
\item
  \emph{Gandhi took Gokhale's liberal approach based on British Whiggish
  traditions and transformed it to make it look Indian.}
\end{itemize}

At the request of Gopal Krishna Gokhale, conveyed to him by C. F.
Andrews, Gandhi returned to India in 1915. He brought an international
reputation as a leading Indian nationalist, theorist and community
organiser.

Gandhi joined the Indian National Congress and was introduced to Indian
issues, politics and the Indian people primarily by Gokhale. Gokhale was
a key leader of the Congress Party best known for his restraint and
moderation, and his insistence on working inside the system. Gandhi took
Gokhale's liberal approach based on British Whiggish traditions and
transformed it to make it look Indian.

Gandhi took leadership of the Congress in 1920 and began escalating
demands until on 26 January 1930 the Indian National Congress declared
the independence of India. The British did not recognise the declaration
but negotiations ensued, with the Congress taking a role in provincial
government in the late 1930s. Gandhi and the Congress withdrew their
support of the Raj when the Viceroy declared war on Germany in September
1939 without consultation. Tensions escalated until Gandhi demanded
immediate independence in 1942 and the British responded by imprisoning
him and tens of thousands of Congress leaders. Meanwhile, the Muslim
League did co-operate with Britain and moved, against Gandhi's strong
opposition, to demands for a totally separate Muslim state of Pakistan.
In August 1947 the British partitioned the land with India and Pakistan
each achieving independence on terms that Gandhi disapproved.

\section{Role in World War I}\label{role-in-world-war-i}

\begin{itemize}
\item
  \emph{In April 1918, during the latter part of World War I, the
  Viceroy invited Gandhi to a War Conference in Delhi.}
\item
  \emph{Gandhi's war recruitment campaign brought into question his
  consistency on nonviolence.}
\item
  \emph{Gandhi agreed to actively recruit Indians for the war effort.}
\item
  \emph{Gandhi's private secretary noted that "The question of the
  consistency between his creed of 'Ahimsa' (nonviolence) and his
  recruiting campaign was raised not only then but has been discussed
  ever since."}
\end{itemize}

In April 1918, during the latter part of World War I, the Viceroy
invited Gandhi to a War Conference in Delhi. Gandhi agreed to actively
recruit Indians for the war effort. In contrast to the Zulu War of 1906
and the outbreak of World War I in 1914, when he recruited volunteers
for the Ambulance Corps, this time Gandhi attempted to recruit
combatants. In a June 1918 leaflet entitled "Appeal for Enlistment",
Gandhi wrote "To bring about such a state of things we should have the
ability to defend ourselves, that is, the ability to bear arms and to
use them...If we want to learn the use of arms with the greatest
possible despatch, it is our duty to enlist ourselves in the army." He
did, however, stipulate in a letter to the Viceroy's private secretary
that he "personally will not kill or injure anybody, friend or foe."

Gandhi's war recruitment campaign brought into question his consistency
on nonviolence. Gandhi's private secretary noted that "The question of
the consistency between his creed of 'Ahimsa' (nonviolence) and his
recruiting campaign was raised not only then but has been discussed ever
since."

\section{Champaran and Kheda}\label{champaran-and-kheda}

\includegraphics[width=3.48857in,height=5.50000in]{media/image6.jpg}\\
\emph{Gandhi in 1918, at the time of the Kheda and Champaran
Satyagrahas}

\section{Champaran agitations}\label{champaran-agitations}

\begin{itemize}
\item
  \emph{Pursuing a strategy of nonviolent protest, Gandhi took the
  administration by surprise and won concessions from the authorities.}
\item
  \emph{Gandhi's first major achievement came in 1917 with the Champaran
  agitation in Bihar.}
\item
  \emph{Unhappy with this, the peasantry appealed to Gandhi at his
  ashram in Ahmedabad.}
\end{itemize}

Gandhi's first major achievement came in 1917 with the Champaran
agitation in Bihar. The Champaran agitation pitted the local peasantry
against their largely British landlords who were backed by the local
administration. The peasantry was forced to grow Indigo, a cash crop
whose demand had been declining over two decades, and were forced to
sell their crops to the planters at a fixed price. Unhappy with this,
the peasantry appealed to Gandhi at his ashram in Ahmedabad. Pursuing a
strategy of nonviolent protest, Gandhi took the administration by
surprise and won concessions from the authorities.

\section{Kheda agitations}\label{kheda-agitations}

\begin{itemize}
\item
  \emph{Gandhi moved his headquarters to Nadiad, organising scores of
  supporters and fresh volunteers from the region, the most notable
  being Vallabhbhai Patel.}
\item
  \emph{Gandhi worked hard to win public support for the agitation
  across the country.}
\item
  \emph{Using non-co-operation as a technique, Gandhi initiated a
  signature campaign where peasants pledged non-payment of revenue even
  under the threat of confiscation of land.}
\end{itemize}

In 1918, Kheda was hit by floods and famine and the peasantry was
demanding relief from taxes. Gandhi moved his headquarters to Nadiad,
organising scores of supporters and fresh volunteers from the region,
the most notable being Vallabhbhai Patel. Using non-co-operation as a
technique, Gandhi initiated a signature campaign where peasants pledged
non-payment of revenue even under the threat of confiscation of land. A
social boycott of mamlatdars and talatdars (revenue officials within the
district) accompanied the agitation. Gandhi worked hard to win public
support for the agitation across the country. For five months, the
administration refused but finally in end-May 1918, the Government gave
way on important provisions and relaxed the conditions of payment of
revenue tax until the famine ended. In Kheda, Vallabhbhai Patel
represented the farmers in negotiations with the British, who suspended
revenue collection and released all the prisoners.

\section{Khilafat movement}\label{khilafat-movement}

\begin{itemize}
\item
  \emph{Gandhi felt that Hindu-Muslim co-operation was necessary for
  political progress against the British.}
\item
  \emph{Gandhi had already supported the British crown with resources
  and by recruiting Indian soldiers to fight the war in Europe on the
  British side.}
\item
  \emph{Muslim leaders and delegates abandoned Gandhi and his Congress.}
\item
  \emph{Gandhi announced his satyagraha (civil disobedience)
  intentions.}
\item
  \emph{It initially led to a strong Muslim support for Gandhi.}
\end{itemize}

In 1919 after the World War I was over, Gandhi (aged 49) sought
political co-operation from Muslims in his fight against British
imperialism by supporting the Ottoman Empire that had been defeated in
the World War. Before this initiative of Gandhi, communal disputes and
religious riots between Hindus and Muslims were common in British India,
such as the riots of 1917--18. Gandhi had already supported the British
crown with resources and by recruiting Indian soldiers to fight the war
in Europe on the British side. This effort of Gandhi was in part
motivated by the British promise to reciprocate the help with swaraj
(self-government) to Indians after the end of World War I. The British
government, instead of self government, had offered minor reforms
instead, disappointing Gandhi. Gandhi announced his satyagraha (civil
disobedience) intentions. The British colonial officials made their
counter move by passing the Rowlatt Act, to block Gandhi's movement. The
Act allowed the British government to treat civil disobedience
participants as criminals and gave it the legal basis to arrest anyone
for "preventive indefinite detention, incarceration without judicial
review or any need for a trial".

Gandhi felt that Hindu-Muslim co-operation was necessary for political
progress against the British. He leveraged the Khilafat movement,
wherein Sunni Muslims in India, their leaders such as the sultans of
princely states in India and Ali brothers championed the Turkish Caliph
as a solidarity symbol of Sunni Islamic community (ummah). They saw the
Caliph as their means to support Islam and the Islamic law after the
defeat of Ottoman Empire in World War I. Gandhi's support to the
Khilafat movement led to mixed results. It initially led to a strong
Muslim support for Gandhi. However, the Hindu leaders including
Rabindranath Tagore questioned Gandhi's leadership because they were
largely against recognising or supporting the Sunni Islamic Caliph in
Turkey.

The increasing Muslim support for Gandhi, after he championed the
Caliph's cause, temporarily stopped the Hindu-Muslim communal violence.
It offered evidence of inter-communal harmony in joint Rowlatt
satyagraha demonstration rallies, raising Gandhi's stature as the
political leader to the British. His support for the Khilafat movement
also helped him sideline Muhammad Ali Jinnah, who had announced his
opposition to the satyagraha non-cooperation movement approach of
Gandhi. Jinnah began creating his independent support, and later went on
to lead the demand for West and East Pakistan.

By the end of 1922 the Khilafat movement had collapsed. Turkey's Ataturk
had ended the Caliphate, Khilafat movement ended, and Muslim support for
Gandhi largely evaporated. Muslim leaders and delegates abandoned Gandhi
and his Congress. Hindu-Muslim communal conflicts reignited. Deadly
religious riots re-appeared in numerous cities, with 91 in United
Provinces of Agra and Oudh alone.

\includegraphics[width=5.50000in,height=4.09633in]{media/image7.jpg}\\
\emph{Gandhi spinning yarn, in the late 1920s}

\section{Non-co-operation}\label{non-co-operation}

\begin{itemize}
\item
  \emph{The political base behind Gandhi had broken into factions.}
\item
  \emph{On 9 April, Gandhi was arrested.}
\item
  \emph{Investigation committees were formed by the British, which
  Gandhi asked Indians to boycott.}
\item
  \emph{Gandhi thus began his journey aimed at crippling the British
  India government economically, politically and administratively.}
\item
  \emph{Gandhi defied the order.}
\item
  \emph{In 1921, Gandhi was the leader of the Indian National Congress.}
\end{itemize}

With his book Hind Swaraj (1909) Gandhi, aged 40, declared that British
rule was established in India with the co-operation of Indians and had
survived only because of this co-operation. If Indians refused to
co-operate, British rule would collapse and swaraj would come.

In February 1919, Gandhi cautioned the Viceroy of India with a cable
communication that if the British were to pass the Rowlatt Act, he would
appeal to Indians to start civil disobedience. The British government
ignored him and passed the law, stating it would not yield to threats.
The satyagraha civil disobedience followed, with people assembling to
protest the Rowlatt Act. On 30 March 1919, British law officers opened
fire on an assembly of unarmed people, peacefully gathered,
participating in satyagraha in Delhi.

People rioted in retaliation. On 6 April 1919, a Hindu festival day, he
asked a crowd to remember not to injure or kill British people, but
express their frustration with peace, to boycott British goods and burn
any British clothing they own. He emphasised the use of non-violence to
the British and towards each other, even if the other side uses
violence. Communities across India announced plans to gather in greater
numbers to protest. Government warned him to not enter Delhi. Gandhi
defied the order. On 9 April, Gandhi was arrested.

People rioted. On 13 April 1919, people including women with children
gathered in an Amritsar park, and a British officer named Reginald Dyer
surrounded them and ordered his troops to fire on them. The resulting
Jallianwala Bagh massacre (or Amritsar massacre) of hundreds of Sikh and
Hindu civilians enraged the subcontinent, but was cheered by some
Britons and parts of the British media as an appropriate response.
Gandhi in Ahmedabad, on the day after the massacre in Amritsar, did not
criticise the British and instead criticised his fellow countrymen for
not exclusively using love to deal with the hate of the British
government. Gandhi demanded that people stop all violence, stop all
property destruction, and went on fast-to-death to pressure Indians to
stop their rioting.

The massacre and Gandhi's non-violent response to it moved many, but
also made some Sikhs and Hindus upset that Dyer was getting away with
murder. Investigation committees were formed by the British, which
Gandhi asked Indians to boycott. The unfolding events, the massacre and
the British response, led Gandhi to the belief that Indians will never
get a fair equal treatment under British rulers, and he shifted his
attention to Swaraj or self rule and political independence for India.
In 1921, Gandhi was the leader of the Indian National Congress. He
reorganised the Congress. With Congress now behind him, and Muslim
support triggered by his backing the Khilafat movement to restore the
Caliph in Turkey, Gandhi had the political support and the attention of
the British Raj.

Gandhi expanded his nonviolent non-co-operation platform to include the
swadeshi policy -- the boycott of foreign-made goods, especially British
goods. Linked to this was his advocacy that khadi (homespun cloth) be
worn by all Indians instead of British-made textiles. Gandhi exhorted
Indian men and women, rich or poor, to spend time each day spinning
khadi in support of the independence movement. In addition to boycotting
British products, Gandhi urged the people to boycott British
institutions and law courts, to resign from government employment, and
to forsake British titles and honours. Gandhi thus began his journey
aimed at crippling the British India government economically,
politically and administratively.

The appeal of "Non-cooperation" grew, its social popularity drew
participation from all strata of Indian society. Gandhi was arrested on
10 March 1922, tried for sedition, and sentenced to six years'
imprisonment. He began his sentence on 18 March 1922. With Gandhi
isolated in prison, the Indian National Congress split into two
factions, one led by Chitta Ranjan Das and Motilal Nehru favouring party
participation in the legislatures, and the other led by Chakravarti
Rajagopalachari and Sardar Vallabhbhai Patel, opposing this move.
Furthermore, co-operation among Hindus and Muslims ended as Khilafat
movement collapsed with the rise of Ataturk in Turkey. Muslim leaders
left the Congress and began forming Muslim organisations. The political
base behind Gandhi had broken into factions. Gandhi was released in
February 1924 for an appendicitis operation, having served only two
years.

\section{Salt Satyagraha (Salt March)}\label{salt-satyagraha-salt-march}

\begin{itemize}
\item
  \emph{The British did not respond favourably to Gandhi's proposal.}
\item
  \emph{British violence, Gandhi promised, was going to be defeated by
  Indian non-violence.}
\item
  \emph{After Gandhi's arrest, the women marched and picketed shops on
  their own, accepting violence and verbal abuse from British
  authorities for the cause in a manner Gandhi inspired.}
\end{itemize}

After his early release from prison for political crimes in 1924, over
the second half of the 1920s, Gandhi continued to pursue swaraj. He
pushed through a resolution at the Calcutta Congress in December 1928
calling on the British government to grant India dominion status or face
a new campaign of non-co-operation with complete independence for the
country as its goal. After his support for the World War I with Indian
combat troops, and the failure of Khilafat movement in preserving the
rule of Caliph in Turkey, followed by a collapse in Muslim support for
his leadership, some such as Subhas Chandra Bose and Bhagat Singh
questioned his values and non-violent approach. While many Hindu leaders
championed a demand for immediate independence, Gandhi revised his own
call to a one-year wait, instead of two.

The British did not respond favourably to Gandhi's proposal. British
political leaders such as Lord Birkenhead and Winston Churchill
announced opposition to "the appeasers of Gandhi", in their discussions
with European diplomats who sympathised with Indian demands. On 31
December 1929, the flag of India was unfurled in Lahore. Gandhi led
Congress celebrated 26 January 1930 as India's Independence Day in
Lahore. This day was commemorated by almost every other Indian
organisation. Gandhi then launched a new Satyagraha against the tax on
salt in March 1930. Gandhi sent an ultimatum in the form of a polite
letter to the viceroy of India, Lord Irwin, on 2 March. A young left
wing British Quaker by the name of Reg Reynolds delivered the letter.
Gandhi condemned British rule in the letter, describing it as "a curse"
that "has impoverished the dumb millions by a system of progressive
exploitation and by a ruinously expensive military and civil
administration... It has reduced us politically to serfdom." Gandhi also
mentioned in the letter that the viceroy received a salary "over five
thousand times India's average income." British violence, Gandhi
promised, was going to be defeated by Indian non-violence.

This was highlighted by the famous Salt March to Dandi from 12 March to
6 April, where, together with 78 volunteers, he marched 388 kilometres
(241~mi) from Ahmedabad to Dandi, Gujarat to make salt himself, with the
declared intention of breaking the salt laws. Thousands of Indians
joined him on this march to the sea. The march took 25 days to cover 240
miles with Gandhi speaking to often huge crowds along the way. On 5 May
he was interned under a regulation dating from 1827 in anticipation of a
protest that he had planned. The protest at Dharasana salt works on 21
May went ahead without its leader, Gandhi. A horrified American
journalist, Webb Miller, described the British response thus:

This went on for hours until some 300 or more protesters had been
beaten, many seriously injured and two killed. At no time did they offer
any resistance.

This campaign was one of his most successful at upsetting British hold
on India; Britain responded by imprisoning over 60,000~people. Congress
estimates, however, put the figure at 90,000. Among them was one of
Gandhi's lieutenants, Jawaharlal Nehru.

According to Sarma, Gandhi recruited women to participate in the salt
tax campaigns and the boycott of foreign products, which gave many women
a new self-confidence and dignity in the mainstream of Indian public
life. However, other scholars such as Marilyn French state that Gandhi
barred women from joining his civil disobedience movement because he
feared he would be accused of using women as political shield. When
women insisted that they join the movement and public demonstrations,
according to Thapar-Bjorkert, Gandhi asked the volunteers to get
permissions of their guardians and only those women who can arrange
child-care should join him. Regardless of Gandhi's apprehensions and
views, Indian women joined the Salt March by the thousands to defy the
British salt taxes and monopoly on salt mining. After Gandhi's arrest,
the women marched and picketed shops on their own, accepting violence
and verbal abuse from British authorities for the cause in a manner
Gandhi inspired.

\includegraphics[width=5.50000in,height=4.43698in]{media/image8.jpg}\\
\emph{Indian workers on strike in support of Gandhi in 1930.}

\section{Gandhi as folk hero}\label{gandhi-as-folk-hero}

\begin{itemize}
\item
  \emph{Gandhi also campaigned hard going from one rural corner of the
  Indian subcontinent to another.}
\item
  \emph{Gandhi captured the imagination of the people of his heritage
  with his ideas about winning "hate with love".}
\item
  \emph{According to Atlury Murali, Indian Congress in the 1920s
  appealed to Andhra Pradesh peasants by creating Telugu language plays
  that combined Indian mythology and legends, linked them to Gandhi's
  ideas, and portrayed Gandhi as a messiah, a reincarnation of ancient
  and medieval Indian nationalist leaders and saints.}
\end{itemize}

According to Atlury Murali, Indian Congress in the 1920s appealed to
Andhra Pradesh peasants by creating Telugu language plays that combined
Indian mythology and legends, linked them to Gandhi's ideas, and
portrayed Gandhi as a messiah, a reincarnation of ancient and medieval
Indian nationalist leaders and saints. The plays built support among
peasants steeped in traditional Hindu culture, according to Murali, and
this effort made Gandhi a folk hero in Telugu speaking villages, a
sacred messiah-like figure.

According to Dennis Dalton, it was the ideas that were responsible for
his wide following. Gandhi criticised Western civilisation as one driven
by "brute force and immorality", contrasting it with his categorisation
of Indian civilisation as one driven by "soul force and morality".
Gandhi captured the imagination of the people of his heritage with his
ideas about winning "hate with love". These ideas are evidenced in his
pamphlets from the 1890s, in South Africa, where too he was popular
among the Indian indentured workers. After he returned to India, people
flocked to him because he reflected their values.

Gandhi also campaigned hard going from one rural corner of the Indian
subcontinent to another. He used terminology and phrases such as
Rama-rajya from Ramayana, Prahlada as a paradigmatic icon, and such
cultural symbols as another facet of swaraj and satyagraha. These ideas
sounded strange outside India, during his lifetime, but they readily and
deeply resonated with the culture and historic values of his people.

\section{Negotiations}\label{negotiations}

\begin{itemize}
\item
  \emph{The conference was a disappointment to Gandhi and the
  nationalists.}
\item
  \emph{Churchill attempted to isolate Gandhi, and his criticism of
  Gandhi was widely covered by European and American press.}
\item
  \emph{Churchill's bitterness against Gandhi grew in the 1930s.}
\item
  \emph{The government, represented by Lord Irwin, decided to negotiate
  with Gandhi.}
\end{itemize}

The government, represented by Lord Irwin, decided to negotiate with
Gandhi. The Gandhi--Irwin Pact was signed in March 1931. The British
Government agreed to free all political prisoners, in return for the
suspension of the civil disobedience movement. According to the pact,
Gandhi was invited to attend the Round Table Conference in London for
discussions and as the sole representative of the Indian National
Congress. The conference was a disappointment to Gandhi and the
nationalists. Gandhi expected to discuss India's independence, while the
British side focused on the Indian princes and Indian minorities rather
than on a transfer of power. Lord Irwin's successor, Lord Willingdon,
took a hard line against India as an independent nation, began a new
campaign of controlling and subduing the nationalist movement. Gandhi
was again arrested, and the government tried and failed to negate his
influence by completely isolating him from his followers.

In Britain, Winston Churchill, a prominent Conservative politician who
was then out of office but later became its prime minister, became a
vigorous and articulate critic of Gandhi and opponent of his long-term
plans. Churchill often ridiculed Gandhi, saying in a widely reported
1931 speech:

Churchill's bitterness against Gandhi grew in the 1930s. He called
Gandhi as the one who was "seditious in aim" whose evil genius and
multiform menace was attacking the British empire. Churchill called him
a dictator, a "Hindu Mussolini", fomenting a race war, trying to replace
the Raj with Brahmin cronies, playing on the ignorance of Indian masses,
all for selfish gain. Churchill attempted to isolate Gandhi, and his
criticism of Gandhi was widely covered by European and American press.
It gained Churchill sympathetic support, but it also increased support
for Gandhi among Europeans. The developments heightened Churchill's
anxiety that the "British themselves would give up out of pacifism and
misplaced conscience".

\includegraphics[width=4.24531in,height=5.50000in]{media/image9.jpg}\\
\emph{Mahadev Desai (left) was Gandhi's personal assistant, both at
Birla House, Bombay, 7 April 1939}

\section{Round Table Conferences}\label{round-table-conferences}

\begin{itemize}
\item
  \emph{The British questioned the Congress party and Gandhi's authority
  to speak for all of India.}
\item
  \emph{In protest, Gandhi started a fast-unto-death, while he was held
  in prison.}
\item
  \emph{During the discussions between Gandhi and the British government
  over 1931--32 at the Round Table Conferences, Gandhi, now aged about
  62, sought constitutional reforms as a preparation to the end of
  colonial British rule, and begin the self-rule by Indians.}
\end{itemize}

During the discussions between Gandhi and the British government over
1931--32 at the Round Table Conferences, Gandhi, now aged about 62,
sought constitutional reforms as a preparation to the end of colonial
British rule, and begin the self-rule by Indians. The British side
sought reforms that would keep Indian subcontinent as a colony. The
British negotiators proposed constitutional reforms on a British
Dominion model that established separate electorates based on religious
and social divisions. The British questioned the Congress party and
Gandhi's authority to speak for all of India. They invited Indian
religious leaders, such as Muslims and Sikhs, to press their demands
along religious lines, as well as B. R. Ambedkar as the representative
leader of the untouchables. Gandhi vehemently opposed a constitution
that enshrined rights or representations based on communal divisions,
because he feared that it would not bring people together but divide
them, perpetuate their status and divert the attention from India's
struggle to end the colonial rule.

After Gandhi returned from the Second Round Table conference, he started
a new satyagraha. He was arrested and imprisoned at the Yerwada Jail,
Pune. While he was in prison, the British government enacted a new law
that granted untouchables a separate electorate. It came to be known as
the Communal Award. In protest, Gandhi started a fast-unto-death, while
he was held in prison. The resulting public outcry forced the
government, in consultations with Ambedkar, to replace the Communal
Award with a compromise Poona Pact.

\section{Congress politics}\label{congress-politics}

\begin{itemize}
\item
  \emph{Despite Gandhi's opposition, Bose won a second term as Congress
  President, against Gandhi's nominee, Dr. Pattabhi Sitaramayya; but
  left the Congress when the All-India leaders resigned en masse in
  protest of his abandonment of the principles introduced by Gandhi.}
\item
  \emph{In 1934 Gandhi resigned from Congress party membership.}
\item
  \emph{Gandhi declared that Sitaramayya's defeat was his defeat.}
\end{itemize}

In 1934 Gandhi resigned from Congress party membership. He did not
disagree with the party's position but felt that if he resigned, his
popularity with Indians would cease to stifle the party's membership,
which actually varied, including communists, socialists, trade
unionists, students, religious conservatives, and those with
pro-business convictions, and that these various voices would get a
chance to make themselves heard. Gandhi also wanted to avoid being a
target for Raj propaganda by leading a party that had temporarily
accepted political accommodation with the Raj.

Gandhi returned to active politics again in 1936, with the Nehru
presidency and the Lucknow session of the Congress. Although Gandhi
wanted a total focus on the task of winning independence and not
speculation about India's future, he did not restrain the Congress from
adopting socialism as its goal. Gandhi had a clash with Subhas Chandra
Bose, who had been elected president in 1938, and who had previously
expressed a lack of faith in nonviolence as a means of protest. Despite
Gandhi's opposition, Bose won a second term as Congress President,
against Gandhi's nominee, Dr. Pattabhi Sitaramayya; but left the
Congress when the All-India leaders resigned en masse in protest of his
abandonment of the principles introduced by Gandhi. Gandhi declared that
Sitaramayya's defeat was his defeat.

\includegraphics[width=4.34867in,height=5.50000in]{media/image10.jpg}\\
\emph{Gandhi in 1942, the year he launched the Quit India Movement}

\section{World War II and Quit India
movement}\label{world-war-ii-and-quit-india-movement}

\begin{itemize}
\item
  \emph{This was Gandhi's and the Congress Party's most definitive
  revolt aimed at securing the British exit from India.}
\item
  \emph{The British government responded quickly to the Quit India
  speech, and within hours after Gandhi's speech arrested Gandhi and all
  the members of the Congress Working Committee.}
\item
  \emph{Gelder then composed and released an interview summary, cabled
  it to the mainstream press, that announced sudden concessions Gandhi
  was willing to make, comments that shocked his countrymen, the
  Congress workers and even Gandhi.}
\end{itemize}

Gandhi opposed providing any help to the British war effort and he
campaigned against any Indian participation in the World War II.
Gandhi's campaign did not enjoy the support of Indian masses and many
Indian leaders such as Sardar Patel and Rajendra Prasad. His campaign
was a failure. Over 2.5~million Indians ignored Gandhi, volunteered and
joined the British military to fight on various fronts of the allied
forces.

Gandhi opposition to the Indian participation in the World War II was
motivated by his belief that India could not be party to a war
ostensibly being fought for democratic freedom while that freedom was
denied to India itself. He also condemned Nazism and Fascism, a view
which won endorsement of other Indian leaders. As the war progressed,
Gandhi intensified his demand for independence, calling for the British
to Quit India in a 1942 speech in Mumbai. This was Gandhi's and the
Congress Party's most definitive revolt aimed at securing the British
exit from India. The British government responded quickly to the Quit
India speech, and within hours after Gandhi's speech arrested Gandhi and
all the members of the Congress Working Committee. His countrymen
retaliated the arrests by damaging or burning down hundreds of
government owned railway stations, police stations, and cutting down
telegraph wires.

In 1942, Gandhi now nearing age 73, urged his people to completely stop
co-operating with the imperial government. In this effort, he urged that
they neither kill nor injure British people, but be willing to suffer
and die if violence is initiated by the British officials. He clarified
that the movement would not be stopped because of any individual acts of
violence, saying that the "ordered anarchy" of "the present system of
administration" was "worse than real anarchy." He urged Indians to Karo
ya maro ("Do or die") in the cause of their rights and freedoms.

Gandhi's arrest lasted two years, as he was held in the Aga Khan Palace
in Pune. During this period, his long time secretary Mahadev Desai died
of a heart attack, his wife Kasturba died after 18 months' imprisonment
on 22 February 1944; and Gandhi suffered a severe malaria attack. While
in jail, he agreed to an interview with Stuart Gelder, a British
journalist. Gelder then composed and released an interview summary,
cabled it to the mainstream press, that announced sudden concessions
Gandhi was willing to make, comments that shocked his countrymen, the
Congress workers and even Gandhi. The latter two claimed that it
distorted what Gandhi actually said on a range of topics and falsely
repudiated the Quit India movement.

Gandhi was released before the end of the war on 6 May 1944 because of
his failing health and necessary surgery; the Raj did not want him to
die in prison and enrage the nation. He came out of detention to an
altered political scene -- the Muslim League for example, which a few
years earlier had appeared marginal, "now occupied the centre of the
political stage" and the topic of Muhammad Ali Jinnah's campaign for
Pakistan was a major talking point. Gandhi and Jinnah had extensive
correspondence and the two men met several times over a period of two
weeks in September 1944, where Gandhi insisted on a united religiously
plural and independent India which included Muslims and non-Muslims of
the Indian subcontinent coexisting. Jinnah rejected this proposal and
insisted instead for partitioning the subcontinent on religious lines to
create a separate Muslim India (later Pakistan). These discussions
continued through 1947.

While the leaders of Congress languished in jail, the other parties
supported the war and gained organizational strength. Underground
publications flailed at the ruthless suppression of Congress, but it had
little control over events. At the end of the war, the British gave
clear indications that power would be transferred to Indian hands. At
this point Gandhi called off the struggle, and around 100,000 political
prisoners were released, including the Congress's leadership.

\includegraphics[width=5.50000in,height=3.58953in]{media/image11.jpg}\\
\emph{Gandhi with Muhammad Ali Jinnah in 1944}

\includegraphics[width=5.50000in,height=4.22535in]{media/image12.jpg}\\
\emph{Gandhi in 1947, with Lord Louis Mountbatten, Britain's last
Viceroy of India, and his wife Edwina Mountbatten}

\section{Partition and independence}\label{partition-and-independence}

\begin{itemize}
\item
  \emph{The Indian National Congress and Gandhi called for the British
  to Quit India.}
\item
  \emph{Gandhi opposed partition of the Indian subcontinent along
  religious lines.}
\item
  \emph{Gandhi was involved in the final negotiations, but Stanley
  Wolpert states the "plan to carve up British India was never approved
  of or accepted by Gandhi".}
\end{itemize}

Gandhi opposed partition of the Indian subcontinent along religious
lines. The Indian National Congress and Gandhi called for the British to
Quit India. However, the Muslim League demanded "Divide and Quit India".
Gandhi suggested an agreement which required the Congress and the Muslim
League to co-operate and attain independence under a provisional
government, thereafter, the question of partition could be resolved by a
plebiscite in the districts with a Muslim majority.

Jinnah rejected Gandhi's proposal and called for Direct Action Day, on
16 August 1946, to press Muslims to publicly gather in cities and
support his proposal for partition of Indian subcontinent into a Muslim
state and non-Muslim state. Huseyn Shaheed Suhrawardy, the Muslim League
Chief Minister of Bengal -- now Bangladesh and West Bengal, gave
Calcutta's police special holiday to celebrate the Direct Action Day.
The Direct Action Day triggered a mass murder of Calcutta Hindus and the
torching of their property, and holidaying police were missing to
contain or stop the conflict. The British government did not order its
army to move in to contain the violence. The violence on Direct Action
Day led to retaliatory violence against Muslims across India. Thousands
of Hindus and Muslims were murdered, and tens of thousands were injured
in the cycle of violence in the days that followed. Gandhi visited the
most riot-prone areas to appeal a stop to the massacres.

Archibald Wavell, the Viceroy and Governor-General of British India for
three years through February 1947, had worked with Gandhi and Jinnah to
find a common ground, before and after accepting Indian independence in
principle. Wavell condemned Gandhi's character and motives as well as
his ideas. Wavell accused Gandhi of harbouring the single minded idea to
"overthrow British rule and influence and to establish a Hindu raj", and
called Gandhi a "malignant, malevolent, exceedingly shrewd" politician.
Wavell feared a civil war on the Indian subcontinent, and doubted Gandhi
would be able to stop it.

The British reluctantly agreed to grant independence to the people of
the Indian subcontinent, but accepted Jinnah's proposal of partitioning
the land into Pakistan and India. Gandhi was involved in the final
negotiations, but Stanley Wolpert states the "plan to carve up British
India was never approved of or accepted by Gandhi".

The partition was controversial and violently disputed. More than half a
million were killed in religious riots as 10 million to 12~million
non-Muslims (Hindus, Sikhs mostly) migrated from Pakistan into India,
and Muslims migrated from India into Pakistan, across the newly created
borders of India, West Pakistan and East Pakistan.

Gandhi spent the day of independence not celebrating the end of the
British rule but appealing for peace among his countrymen by fasting and
spinning in Calcutta on 15 August 1947. The partition had gripped the
Indian subcontinent with religious violence and the streets were filled
with corpses. Some writers credit Gandhi's fasting and protests for
stopping the religious riots and communal violence. Others do not.
Archibald Wavell, for example, upon learning of Gandhi's assassination,
commented, "I always thought he {[}Gandhi{]} had more of malevolence
than benevolence in him, but who am I to judge, and how can an
Englishman estimate a Hindu?"

\includegraphics[width=5.50000in,height=3.80886in]{media/image13.jpg}\\
\emph{Gandhi's funeral was marked by millions of Indians.}

\section{Assassination}\label{assassination}

\begin{itemize}
\item
  \emph{According to some accounts, Gandhi died instantly.}
\item
  \emph{The government quelled any opposition to its economic and social
  policies, despite they being contrary to Gandhi's ideas, by
  reconstructing Gandhi's image and ideals.}
\item
  \emph{Gandhi's death was mourned nationwide.}
\item
  \emph{Gandhi's assassination dramatically changed the political
  landscape.}
\end{itemize}

At 5:17~pm on 30 January 1948, Gandhi was with his grandnieces in the
garden of the former Birla House (now Gandhi Smriti), on his way to
address a prayer meeting, when Nathuram Godse fired three bullets from a
Beretta M1934 9mm Corto pistol into his chest at point-blank range.
According to some accounts, Gandhi died instantly. In other accounts,
such as one prepared by an eyewitness journalist, Gandhi was carried
into the Birla House, into a bedroom. There he died about 30 minutes
later as one of Gandhi's family members read verses from Hindu
scriptures.

Prime Minister Jawaharlal Nehru addressed his countrymen over the
All-India Radio saying:

Gandhi's assassin Godse made no attempt to escape and was seized by the
witnesses. He was arrested. In the weeks that followed, his
collaborators were arrested as well. Godse was a Hindu nationalist with
links to the extremist Hindu Mahasabha. They were tried in court at
Delhi's Red Fort. At his trial, Godse did not deny the charges nor
express any remorse. According to Claude Markovits, a French historian
noted for his studies of colonial India, Godse stated that he killed
Gandhi because of his complacence towards Muslims, holding Gandhi
responsible for the frenzy of violence and sufferings during the
subcontinent's partition into Pakistan and India. Godse accused Gandhi
of subjectivism and of acting as if only he had a monopoly of the truth.
Godse was found guilty and executed in 1949.

Gandhi's death was mourned nationwide. Over two million people joined
the five-mile long funeral procession that took over five hours to reach
Raj Ghat from Birla house, where he was assassinated. Gandhi's body was
transported on a weapons carrier, whose chassis was dismantled overnight
to allow a high-floor to be installed so that people could catch a
glimpse of his body. The engine of the vehicle was not used; instead
four drag-ropes manned by 50 people each pulled the vehicle. All
Indian-owned establishments in London remained closed in mourning as
thousands of people from all faiths and denominations and Indians from
all over Britain converged at India House in London.

Gandhi's assassination dramatically changed the political landscape.
Nehru became his political heir. According to Markovits, while Gandhi
was alive, Pakistan's declaration that it was a "Muslim state" had led
Indian groups to demand that it be declared a "Hindu state". Nehru used
Gandhi's martyrdom as a political weapon to silence all advocates of
Hindu nationalism as well as his political challengers. He linked
Gandhi's assassination to politics of hatred and ill-will.

According to Guha, Nehru and his Congress colleagues called on Indians
to honour Gandhi's memory and even more his ideals. Nehru used the
assassination to consolidate the authority of the new Indian state.
Gandhi's death helped marshal support for the new government and
legitimise the Congress Party's control, leveraged by the massive
outpouring of Hindu expressions of grief for a man who had inspired them
for decades. The government suppressed the RSS, the Muslim National
Guards, and the Khaksars, with some 200,000 arrests.

For years after the assassination, states Markovits, "Gandhi's shadow
loomed large over the political life of the new Indian Republic". The
government quelled any opposition to its economic and social policies,
despite they being contrary to Gandhi's ideas, by reconstructing
Gandhi's image and ideals.

\section{Funeral and memorials}\label{funeral-and-memorials}

\begin{itemize}
\item
  \emph{The Birla House site where Gandhi was assassinated is now a
  memorial called Gandhi Smriti.}
\item
  \emph{Gandhi's ashes were poured into urns which were sent across
  India for memorial services.}
\item
  \emph{Gandhi was cremated in accordance with Hindu tradition.}
\item
  \emph{These are widely believed to be Gandhi's last words after he was
  shot, though the veracity of this statement has been disputed.}
\end{itemize}

Gandhi was cremated in accordance with Hindu tradition. Gandhi's ashes
were poured into urns which were sent across India for memorial
services. Most of the ashes were immersed at the Sangam at Allahabad on
12 February 1948, but some were secretly taken away. In 1997, Tushar
Gandhi immersed the contents of one urn, found in a bank vault and
reclaimed through the courts, at the Sangam at Allahabad. Some of
Gandhi's ashes were scattered at the source of the Nile River near
Jinja, Uganda, and a memorial plaque marks the event. On 30 January
2008, the contents of another urn were immersed at Girgaum Chowpatty.
Another urn is at the palace of the Aga Khan in Pune (where Gandhi was
held as a political prisoner from 1942 to 1944) and another in the
Self-Realization Fellowship Lake Shrine in Los Angeles.

The Birla House site where Gandhi was assassinated is now a memorial
called Gandhi Smriti. The place near Yamuna river where he was cremated
is the Rāj Ghāt memorial in New Delhi. A black marble platform, it bears
the epigraph "Hē Rāma" (Devanagari: हे ! राम or, Hey Raam). These are
widely believed to be Gandhi's last words after he was shot, though the
veracity of this statement has been disputed.

\section{Principles, practices and
beliefs}\label{principles-practices-and-beliefs}

\begin{itemize}
\item
  \emph{Gandhi's statements, letters and life have attracted much
  political and scholarly analysis of his principles, practices and
  beliefs, including what influenced him.}
\end{itemize}

Gandhi's statements, letters and life have attracted much political and
scholarly analysis of his principles, practices and beliefs, including
what influenced him. Some writers present him as a paragon of ethical
living and pacifism, while others present him as a more complex,
contradictory and evolving character influenced by his culture and
circumstances.

\includegraphics[width=5.50000in,height=4.89905in]{media/image14.jpg}\\
\emph{Gandhi with poet Rabindranath Tagore, 1940}

\section{Influences}\label{influences}

\begin{itemize}
\item
  \emph{Gandhi's London lifestyle incorporated the values he had grown
  up with.}
\item
  \emph{Historian Howard states the culture of Gujarat influenced Gandhi
  and his methods.}
\item
  \emph{According to Indira Carr and others, Gandhi was influenced by
  Vaishnavism, Jainism and Advaita Vedanta.}
\item
  \emph{Additional theories of possible influences on Gandhi have been
  proposed.}
\end{itemize}

Gandhi grew up in a Hindu and Jain religious atmosphere in his native
Gujarat, which were his primary influences, but he was also influenced
by his personal reflections and literature of Hindu Bhakti saints,
Advaita Vedanta, Islam, Buddhism, Christianity, and thinkers such as
Tolstoy, Ruskin and Thoreau. At age 57 he declared himself to be
Advaitist Hindu in his religious persuasion, but added that he supported
Dvaitist viewpoints and religious pluralism.

Gandhi was influenced by his devout Vaishnava Hindu mother, the regional
Hindu temples and saint tradition which co-existed with Jain tradition
in Gujarat. Historian R.B. Cribb states that Gandhi's thought evolved
over time, with his early ideas becoming the core or scaffolding for his
mature philosophy. He committed himself early to truthfulness,
temperance, chastity, and vegetarianism.

Gandhi's London lifestyle incorporated the values he had grown up with.
When he returned to India in 1891, his outlook was parochial and he
could not make a living as a lawyer. This challenged his belief that
practicality and morality necessarily coincided. By moving in 1893 to
South Africa he found a solution to this problem and developed the
central concepts of his mature philosophy.

According to Bhikhu Parekh, three books that influenced Gandhi most in
South Africa were William Salter's Ethical Religion (1889); Henry David
Thoreau's On the Duty of Civil Disobedience (1849); and Leo Tolstoy's
The Kingdom of God Is Within You (1894). Ruskin inspired his decision to
live an austere life on a commune, at first on the Phoenix Farm in Natal
and then on the Tolstoy Farm just outside Johannesburg, South Africa.
The most profound influence on Gandhi were those from Hinduism,
Christianity and Jainism, states Parekh, with his thoughts "in harmony
with the classical Indian traditions, specially the Advaita or monistic
tradition".

According to Indira Carr and others, Gandhi was influenced by
Vaishnavism, Jainism and Advaita Vedanta. Balkrishna Gokhale states that
Gandhi was influenced by Hinduism and Jainism, and his studies of Sermon
on the Mount of Christianity, Ruskin and Tolstoy.

Additional theories of possible influences on Gandhi have been proposed.
For example, in 1935, N. A. Toothi stated that Gandhi was influenced by
the reforms and teachings of the Swaminarayan tradition of Hinduism.
According to Raymond Williams, Toothi may have overlooked the influence
of the Jain community, and adds close parallels do exist in programs of
social reform in the Swaminarayan tradition and those of Gandhi, based
on "nonviolence, truth-telling, cleanliness, temperance and upliftment
of the masses." Historian Howard states the culture of Gujarat
influenced Gandhi and his methods.

\includegraphics[width=5.50000in,height=3.44468in]{media/image15.jpg}\\
\emph{Mohandas K. Gandhi and other residents of Tolstoy Farm, South
Africa, 1910}

\section{Tolstoy}\label{tolstoy}

\begin{itemize}
\item
  \emph{It was at Tolstoy Farm where Gandhi and Hermann Kallenbach
  systematically trained their disciples in the philosophy of
  nonviolence.}
\item
  \emph{In 1909, Gandhi wrote to Tolstoy seeking advice and permission
  to republish A Letter to a Hindu in Gujarati.}
\item
  \emph{Gandhi called for political involvement; he was a nationalist
  and was prepared to use nonviolent force.}
\end{itemize}

Along with the book mentioned above, in 1908 Leo Tolstoy wrote A Letter
to a Hindu, which said that only by using love as a weapon through
passive resistance could the Indian people overthrow colonial rule. In
1909, Gandhi wrote to Tolstoy seeking advice and permission to republish
A Letter to a Hindu in Gujarati. Tolstoy responded and the two continued
a correspondence until Tolstoy's death in 1910 (Tolstoy's last letter
was to Gandhi). The letters concern practical and theological
applications of nonviolence. Gandhi saw himself a disciple of Tolstoy,
for they agreed regarding opposition to state authority and colonialism;
both hated violence and preached non-resistance. However, they differed
sharply on political strategy. Gandhi called for political involvement;
he was a nationalist and was prepared to use nonviolent force. He was
also willing to compromise. It was at Tolstoy Farm where Gandhi and
Hermann Kallenbach systematically trained their disciples in the
philosophy of nonviolence.

\section{Shrimad Rajchandra}\label{shrimad-rajchandra}

\begin{itemize}
\item
  \emph{Gandhi exchanged letters with Rajchandra when he was in South
  Africa, referring to him as Kavi (literally, "poet").}
\item
  \emph{He had advised Gandhi to be patient and to study Hinduism
  deeply.}
\item
  \emph{Gandhi credited Shrimad Rajchandra, a poet and Jain philosopher,
  as his influential counsellor.}
\item
  \emph{Gandhi, in his autobiography, called Rajchandra his "guide and
  helper" and his "refuge {[}...{]} in moments of spiritual crisis".}
\end{itemize}

Gandhi credited Shrimad Rajchandra, a poet and Jain philosopher, as his
influential counsellor. In Modern Review, June 1930, Gandhi wrote about
their first encounter in 1891 at Dr. P.J. Mehta's residence in Bombay.
Gandhi exchanged letters with Rajchandra when he was in South Africa,
referring to him as Kavi (literally, "poet"). In 1930, Gandhi wrote,
"Such was the man who captivated my heart in religious matters as no
other man ever has till now." 'I have said elsewhere that in moulding my
inner life Tolstoy and Ruskin vied with Kavi. But Kavi's influence was
undoubtedly deeper if only because I had come in closest personal touch
with him.'

Gandhi, in his autobiography, called Rajchandra his "guide and helper"
and his "refuge {[}...{]} in moments of spiritual crisis". He had
advised Gandhi to be patient and to study Hinduism deeply.

\section{Religious texts}\label{religious-texts}

\begin{itemize}
\item
  \emph{Gandhi joined them in their prayers and debated Christian
  theology with them, but refused conversion stating he did not accept
  the theology therein or that Christ was the only son of God.}
\item
  \emph{Gandhi grew fond of Hinduism, and referred to the Bhagavad Gita
  as his spiritual dictionary and greatest single influence on his
  life.}
\end{itemize}

During his stay in South Africa, along with scriptures and philosophical
texts of Hinduism and other Indian religions, Gandhi read translated
texts of Christianity such as the Bible, and Islam such as the Quran. A
Quaker mission in South Africa attempted to convert him to Christianity.
Gandhi joined them in their prayers and debated Christian theology with
them, but refused conversion stating he did not accept the theology
therein or that Christ was the only son of God.

His comparative studies of religions and interaction with scholars, led
him to respect all religions as well as become concerned about
imperfections in all of them and frequent misinterpretations. Gandhi
grew fond of Hinduism, and referred to the Bhagavad Gita as his
spiritual dictionary and greatest single influence on his life.

\section{Sufism}\label{sufism}

\begin{itemize}
\item
  \emph{Winston Churchill also compared Gandhi to a Sufi fakir.}
\item
  \emph{According to Margaret Chatterjee, Gandhi as a Vaishnava Hindu
  shared values such as humility, devotion and brotherhood for the poor
  that is also found in Sufism.}
\item
  \emph{Gandhi was acquainted with Sufi Islam's Chishti Order during his
  stay in South Africa.}
\end{itemize}

Gandhi was acquainted with Sufi Islam's Chishti Order during his stay in
South Africa. He attended Khanqah gatherings there at Riverside.
According to Margaret Chatterjee, Gandhi as a Vaishnava Hindu shared
values such as humility, devotion and brotherhood for the poor that is
also found in Sufism. Winston Churchill also compared Gandhi to a Sufi
fakir.

\section{On wars and nonviolence}\label{on-wars-and-nonviolence}

\section{Support for Wars}\label{support-for-wars}

\begin{itemize}
\item
  \emph{Rahul Sagar interprets Gandhi's efforts to recruit for the
  British military during the War, as Gandhi's belief that, at that
  time, it would demonstrate that Indians were willing to fight.}
\item
  \emph{Gandhi participated in South African war against the Boers, on
  the British side in 1899.}
\item
  \emph{After the war, the British government offered minor reforms
  instead, which disappointed Gandhi.}
\end{itemize}

Gandhi participated in South African war against the Boers, on the
British side in 1899. Both the Dutch settlers called Boers and the
imperial British at that time discriminated against the coloured races
they considered as inferior, and Gandhi later wrote about his conflicted
beliefs during the Boer war. He stated that "when the war was declared,
my personal sympathies were all with the Boers, but my loyalty to the
British rule drove me to participation with the British in that war".
According to Gandhi, he felt that since he was demanding his rights as a
British citizen, it was also his duty to serve the British forces in the
defence of the British Empire.

During World War I (1914--1918), nearing the age of 50, Gandhi supported
the British and its allied forces by recruiting Indians to join the
British army, expanding the Indian contingent from about 100,000 to over
1.1~million. He encouraged his people to fight on one side of the war in
Europe and Africa at the cost of their lives. Pacifists criticised and
questioned Gandhi, who defended these practices by stating, according to
Sankar Ghose, "it would be madness for me to sever my connection with
the society to which I belong". According to Keith Robbins, the
recruitment effort was in part motivated by the British promise to
reciprocate the help with swaraj (self-government) to Indians after the
end of World War I. After the war, the British government offered minor
reforms instead, which disappointed Gandhi. He launched his satyagraha
movement in 1919. In parallel, Gandhi's fellowmen became sceptical of
his pacifist ideas and were inspired by the ideas of nationalism and
anti-imperialism.

In a 1920 essay, after the World War I, Gandhi wrote, "where there is
only a choice between cowardice and violence, I would advise violence."
Rahul Sagar interprets Gandhi's efforts to recruit for the British
military during the War, as Gandhi's belief that, at that time, it would
demonstrate that Indians were willing to fight. Further, it would also
show the British that his fellow Indians were "their subjects by choice
rather than out of cowardice." In 1922, Gandhi wrote that abstinence
from violence is effective and true forgiveness only when one has the
power to punish, not when one decides not to do anything because one is
helpless.

After World War II engulfed Britain, Gandhi actively campaigned to
oppose any help to the British war effort and any Indian participation
in the war. According to Arthur Herman, Gandhi believed that his
campaign would strike a blow to imperialism. Gandhi's position was not
supported by many Indian leaders, and his campaign against the British
war effort was a failure. The Hindu leader, Tej Bahadur Sapru declared
in 1941, states Herman, "A good many Congress leaders are fed up with
the barren program of the Mahatma". Over 2.5~million Indians ignored
Gandhi, volunteered and joined on the British side. They fought and died
as a part of the allied forces in Europe, North Africa and various
fronts of the World War II.

\includegraphics[width=5.50000in,height=3.94359in]{media/image16.jpg}\\
\emph{"God is truth. The way to truth lies through ahimsa (nonviolence)"
-- Sabarmati, 13 March 1927}

\includegraphics[width=3.76933in,height=5.50000in]{media/image17.jpg}\\
\emph{Gandhi picking salt during Salt Satyagraha to defy colonial law
giving salt collection monopoly to the British. His satyagraha attracted
vast numbers of Indian men and women.}

\section{Truth and Satyagraha}\label{truth-and-satyagraha}

\begin{itemize}
\item
  \emph{Gandhi summarised his beliefs first when he said "God is
  Truth".}
\item
  \emph{Thus, satya (truth) in Gandhi's philosophy is "God".}
\item
  \emph{Gandhi stated that the most important battle to fight was
  overcoming his own demons, fears, and insecurities.}
\item
  \emph{Gandhi wrote: "There must be no impatience, no barbarity, no
  insolence, no undue pressure.}
\end{itemize}

Gandhi dedicated his life to discovering and pursuing truth, or Satya,
and called his movement as satyagraha, which means "appeal to,
insistence on, or reliance on the Truth". The first formulation of the
satyagraha as a political movement and principle occurred in 1920, which
he tabled as "Resolution on Non-cooperation" in September that year
before a session of the Indian Congress. It was the satyagraha
formulation and step, states Dennis Dalton, that deeply resonated with
beliefs and culture of his people, embedded him into the popular
consciousness, transforming him quickly into Mahatma.

Gandhi based Satyagraha on the Vedantic ideal of self-realization,
ahimsa (nonviolence), vegetarianism, and universal love. William Borman
states that the key to his satyagraha is rooted in the Hindu Upanishadic
texts. According to Indira Carr, Gandhi's ideas on ahimsa and satyagraha
were founded on the philosophical foundations of Advaita Vedanta. I.
Bruce Watson states that some of these ideas are found not only in
traditions within Hinduism, but also in Jainism or Buddhism,
particularly those about non-violence, vegetarianism and universal love,
but Gandhi's synthesis was to politicise these ideas. Gandhi's concept
of satya as a civil movement, states Glyn Richards, are best understood
in the context of the Hindu terminology of Dharma and Ṛta.

Gandhi stated that the most important battle to fight was overcoming his
own demons, fears, and insecurities. Gandhi summarised his beliefs first
when he said "God is Truth". He would later change this statement to
"Truth is God". Thus, satya (truth) in Gandhi's philosophy is "God".
Gandhi, states Richards, described the term "God" not as a separate
power, but as the Being (Brahman, Atman) of the Advaita Vedanta
tradition, a nondual universal that pervades in all things, in each
person and all life. According to Nicholas Gier, this to Gandhi meant
the unity of God and humans, that all beings have the same one soul and
therefore equality, that atman exists and is same as everything in the
universe, ahimsa (non-violence) is the very nature of this atman.

The essence of Satyagraha is "soul force" as a political means, refusing
to use brute force against the oppressor, seeking to eliminate
antagonisms between the oppressor and the oppressed, aiming to transform
or "purify" the oppressor. It is not inaction but determined passive
resistance and non-co-operation where, states Arthur Herman, "love
conquers hate". A euphemism sometimes used for Satyagraha is that it is
a "silent force" or a "soul force" (a term also used by Martin Luther
King Jr. during his famous "I Have a Dream" speech). It arms the
individual with moral power rather than physical power. Satyagraha is
also termed a "universal force", as it essentially "makes no distinction
between kinsmen and strangers, young and old, man and woman, friend and
foe."

Gandhi wrote: "There must be no impatience, no barbarity, no insolence,
no undue pressure. If we want to cultivate a true spirit of democracy,
we cannot afford to be intolerant. Intolerance betrays want of faith in
one's cause." Civil disobedience and non-co-operation as practised under
Satyagraha are based on the "law of suffering", a doctrine that the
endurance of suffering is a means to an end. This end usually implies a
moral upliftment or progress of an individual or society. Therefore,
non-co-operation in Satyagraha is in fact a means to secure the
co-operation of the opponent consistently with truth and justice.

While Gandhi's idea of satyagraha as a political means attracted a
widespread following among Indians, the support was not universal. For
example, Muslim leaders such as Jinnah opposed the satyagraha idea,
accused Gandhi to be reviving Hinduism through political activism, and
began effort to counter Gandhi with Muslim nationalism and a demand for
Muslim homeland. The untouchability leader Ambedkar, in June 1945, after
his decision to convert to Buddhism and a key architect of the
Constitution of modern India, dismissed Gandhi's ideas as loved by
"blind Hindu devotees", primitive, influenced by spurious brew of
Tolstoy and Ruskin, and "there is always some simpleton to preach them".
Winston Churchill caricatured Gandhi as a "cunning huckster" seeking
selfish gain, an "aspiring dictator", and an "atavistic spokesman of a
pagan Hinduism". Churchill stated that the civil disobedience movement
spectacle of Gandhi only increased "the danger to which white people
there {[}British India{]} are exposed".

\includegraphics[width=4.12867in,height=5.50000in]{media/image18.jpg}\\
\emph{Gandhi with textile workers at Darwen, Lancashire, 26 September
1931}

\section{Nonviolence}\label{nonviolence}

\begin{itemize}
\item
  \emph{Although Gandhi was not the originator of the principle of
  nonviolence, he was the first to apply it in the political field on a
  large scale.}
\item
  \emph{Gandhi believed this act of "collective suicide", in response to
  the Holocaust, "would have been heroism".}
\item
  \emph{Gandhi explains his philosophy and ideas about ahimsa as a
  political means in his autobiography The Story of My Experiments with
  Truth.}
\end{itemize}

Although Gandhi was not the originator of the principle of nonviolence,
he was the first to apply it in the political field on a large scale.
The concept of nonviolence (ahimsa) has a long history in Indian
religious thought, with it being considered the highest dharma (ethical
value virtue), a precept to be observed towards all living beings
(sarvbhuta), at all times (sarvada), in all respects (sarvatha), in
action, words and thought. Gandhi explains his philosophy and ideas
about ahimsa as a political means in his autobiography The Story of My
Experiments with Truth.

Gandhi was criticised for refusing to protest the hanging of Bhagat
Singh, Sukhdev, Udham Singh and Rajguru. He was accused of accepting a
deal with the King's representative Irwin that released civil
disobedience leaders from prison and accepted the death sentence against
the highly popular revolutionary Bhagat Singh, who at his trial had
replied, "Revolution is the inalienable right of mankind".

Gandhi's views came under heavy criticism in Britain when it was under
attack from Nazi Germany, and later when the Holocaust was revealed. He
told the British people in 1940, "I would like you to lay down the arms
you have as being useless for saving you or humanity. You will invite
Herr Hitler and Signor Mussolini to take what they want of the countries
you call your possessions... If these gentlemen choose to occupy your
homes, you will vacate them. If they do not give you free passage out,
you will allow yourselves, man, woman, and child, to be slaughtered, but
you will refuse to owe allegiance to them." George Orwell remarked that
Gandhi's methods confronted 'an old-fashioned and rather shaky despotism
which treated him in a fairly chivalrous way', not a totalitarian Power,
'where political opponents simply disappear.'

In a post-war interview in 1946, he said, "Hitler killed five million
Jews. It is the greatest crime of our time. But the Jews should have
offered themselves to the butcher's knife. They should have thrown
themselves into the sea from cliffs... It would have aroused the world
and the people of Germany... As it is they succumbed anyway in their
millions." Gandhi believed this act of "collective suicide", in response
to the Holocaust, "would have been heroism".

\section{On inter-religious
relations}\label{on-inter-religious-relations}

\section{Buddhists, Jains and Sikhs}\label{buddhists-jains-and-sikhs}

\begin{itemize}
\item
  \emph{Sikh and Buddhist leaders disagreed with Gandhi, a disagreement
  Gandhi respected as a difference of opinion.}
\item
  \emph{Gandhi believed that Buddhism, Jainism and Sikhism were
  traditions of Hinduism, with shared history, rites and ideas.}
\item
  \emph{Sikhism, to Gandhi, was an integral part of Hinduism, in the
  form of another reform movement.}
\end{itemize}

Gandhi believed that Buddhism, Jainism and Sikhism were traditions of
Hinduism, with shared history, rites and ideas. At other times, he
acknowledged that he knew little about Buddhism other than his reading
of Edwin Arnold's book on it. Based on that book, he considered Buddhism
to be a reform movement and the Buddha to be a Hindu. He stated he knew
Jainism much more, and he credited Jains to have profoundly influenced
him. Sikhism, to Gandhi, was an integral part of Hinduism, in the form
of another reform movement. Sikh and Buddhist leaders disagreed with
Gandhi, a disagreement Gandhi respected as a difference of opinion.

\section{Muslims}\label{muslims}

\begin{itemize}
\item
  \emph{To Gandhi, Islam has "nothing to fear from criticism even if it
  be unreasonable".}
\item
  \emph{Gandhi believed that numerous interpreters have interpreted it
  to fit their preconceived notions.}
\item
  \emph{One of the strategies Gandhi adopted was to work with Muslim
  leaders of pre-partition India, to oppose the British imperialism in
  and outside the Indian subcontinent.}
\end{itemize}

Gandhi had generally positive and empathetic views of Islam, and he
extensively studied the Quran. He viewed Islam as a faith that
proactively promoted peace, and felt that non-violence had a predominant
place in the Quran. He also read the Islamic prophet Muhammad's
biography, and argued that it was "not the sword that won a place for
Islam in those days in the scheme of life. It was the rigid simplicity,
the utter self-effacement of the Prophet, the scrupulous regard for
pledges, his intense devotion to his friends and followers, his
intrepidity, his fearlessness, his absolute trust in God and in his own
mission." Gandhi had a large Indian Muslim following, who he encouraged
to join him in a mutual nonviolent jihad against the social oppression
of their time. Prominent Muslim allies in his nonviolent resistance
movement included Maulana Abul Kalam Azad and Abdul Ghaffar Khan.
However, Gandhi's empathy towards Islam, and his eager willingness to
valorize peaceful Muslim social activists, was viewed by many Hindus as
an appeasement of Muslims and later became a leading cause for his
assassination at the hands of intolerant Hindu extremists.

While Gandhi expressed mostly positive views of Islam, he did
occasionally criticize Muslims. He stated in 1925 that he did not
criticise the teachings of the Quran, but he did criticise the
interpreters of the Quran. Gandhi believed that numerous interpreters
have interpreted it to fit their preconceived notions. He believed
Muslims should welcome criticism of the Quran, because "every true
scripture only gains from criticism". Gandhi criticised Muslims who
"betray intolerance of criticism by a non-Muslim of anything related to
Islam", such as the penalty of stoning to death under Islamic law. To
Gandhi, Islam has "nothing to fear from criticism even if it be
unreasonable". He also believed there were material contradictions
between Hinduism and Islam, and he criticised Muslims along with
communists that were quick to resort to violence.

One of the strategies Gandhi adopted was to work with Muslim leaders of
pre-partition India, to oppose the British imperialism in and outside
the Indian subcontinent. After the World War I, in 1919--22, he won
Muslim leadership support of Ali Brothers by backing the Khilafat
Movement in favour the Islamic Caliph and his historic Ottoman
Caliphate, and opposing the secular Islam supporting Mustafa Kemal
Atatürk. By 1924, Ataturk had ended the Caliphate, the Khilafat Movement
was over, and Muslim support for Gandhi had largely evaporated.

In 1925, Gandhi gave another reason to why he got involved in the
Khilafat movement and the Middle East affairs between Britain and the
Ottoman Empire. Gandhi explained to his co-religionists (Hindu) that he
sympathised and campaigned for the Islamic cause, not because he cared
for the Sultan, but because "I wanted to enlist the Mussalman's sympathy
in the matter of cow protection". According to the historian M. Naeem
Qureshi, like the then Indian Muslim leaders who had combined religion
and politics, Gandhi too imported his religion into his political
strategy during the Khilafat movement.

In the 1940s, Gandhi pooled ideas with some Muslim leaders who sought
religious harmony like him, and opposed the proposed partition of
British India into India and Pakistan. For example, his close friend
Badshah Khan suggested that they should work towards opening Hindu
temples for Muslim prayers, and Islamic mosques for Hindu prayers, to
bring the two religious groups closer. Gandhi accepted this and began
having Muslim prayers read in Hindu temples to play his part, but was
unable to get Hindu prayers read in mosques. The Hindu nationalist
groups objected and began confronting Gandhi for this one-sided
practice, by shouting and demonstrating inside the Hindu temples, in the
last years of his life.

\section{Christians}\label{christians}

\begin{itemize}
\item
  \emph{According to Gandhi, "no religious tradition could claim a
  monopoly over truth or salvation".}
\item
  \emph{Some colonial era Christian preachers and faithfuls considered
  Gandhi as a saint.}
\item
  \emph{Biographers from France and Britain have drawn parallels between
  Gandhi and Christian saints.}
\item
  \emph{Gandhi criticised as well as praised Christianity.}
\end{itemize}

Gandhi criticised as well as praised Christianity. He was critical of
Christian missionary efforts in British India, because they mixed
medical or education assistance with demands that the beneficiary
convert to Christianity. According to Gandhi, this was not true
"service" but one driven by ulterior motive of luring people into
religious conversion and exploiting the economically or medically
desperate. It did not lead to inner transformation or moral advance or
to the Christian teaching of "love", but was based on false one-sided
criticisms of other religions, when Christian societies faced similar
problems in South Africa and Europe. It led to the converted person
hating his neighbours and other religions, it divided people rather than
bringing them closer in compassion. According to Gandhi, "no religious
tradition could claim a monopoly over truth or salvation". Gandhi did
not support laws to prohibit missionary activity, but demanded that
Christians should first understand the message of Jesus, and then strive
to live without stereotyping and misrepresenting other religions.
According to Gandhi, the message of Jesus wasn't to humiliate and
imperialistically rule over other people considering them inferior or
second class or slaves, but that "when the hungry are fed and peace
comes to our individual and collective life, then Christ is born".

Gandhi believed that his long acquaintance with Christianity had made
him like it as well as find it imperfect. He asked Christians to stop
humiliating his country and his people as heathens, idolators and other
abusive language, and to change their negative views of India. He
believed that Christians should introspect on the "true meaning of
religion" and get a desire to study and learn from Indian religions in
the spirit of universal brotherhood. According to Eric Sharpe -- a
professor of Religious Studies, though Gandhi was born in a Hindu family
and later became Hindu by conviction, many Christians in time thought of
him as an "exemplary Christian and even as a saint".

Some colonial era Christian preachers and faithfuls considered Gandhi as
a saint. Biographers from France and Britain have drawn parallels
between Gandhi and Christian saints. Recent scholars question these
romantic biographies and state that Gandhi was neither a Christian
figure nor mirrored a Christian saint. Gandhi's life is better viewed as
exemplifying his belief in the "convergence of various spiritualities"
of a Christian and a Hindu, states Michael de Saint-Cheron.

\section{Jews}\label{jews}

\begin{itemize}
\item
  \emph{Gandhi discussed the persecution of the Jews in Germany and the
  emigration of Jews from Europe to Palestine through his lens of
  Satyagraha.}
\item
  \emph{According to Kumaraswamy, Gandhi initially supported Arab
  demands with respect to Palestine.}
\item
  \emph{In 1937, Gandhi discussed Zionism with his close Jewish friend
  Hermann Kallenbach.}
\item
  \emph{In 1938, Gandhi stated that his "sympathies are all with the
  Jews.}
\end{itemize}

According to Kumaraswamy, Gandhi initially supported Arab demands with
respect to Palestine. He justified this support by invoking Islam,
stating that "non-Muslims cannot acquire sovereign jurisdiction" in
Jazirat al-Arab (the Arabian Peninsula). These arguments, states
Kumaraswamy, were a part of his political strategy to win Muslim support
during the Khilafat movement. In the post-Khilafat period, Gandhi
neither negated Jewish demands nor did he use Islamic texts or history
to support Muslim claims against Israel. Gandhi's silence after the
Khilafat period may represent an evolution in his understanding of the
conflicting religious claims over Palestine, according to Kumaraswamy.
In 1938, Gandhi spoke in favour of Jewish claims, and in March 1946, he
said to the Member of British Parliament Sidney Silverman, "if the Arabs
have a claim to Palestine, the Jews have a prior claim", a position very
different from his earlier stance.

Gandhi discussed the persecution of the Jews in Germany and the
emigration of Jews from Europe to Palestine through his lens of
Satyagraha. In 1937, Gandhi discussed Zionism with his close Jewish
friend Hermann Kallenbach. He said that Zionism was not the right answer
to the problems faced by Jews and instead recommended Satyagraha. Gandhi
thought the Zionists in Palestine represented European imperialism and
used violence to achieve their goals; he argued that "the Jews should
disclaim any intention of realizing their aspiration under the
protection of arms and should rely wholly on the goodwill of Arabs. No
exception can possibly be taken to the natural desire of the Jews to
find a home in Palestine. But they must wait for its fulfillment till
Arab opinion is ripe for it."

In 1938, Gandhi stated that his "sympathies are all with the Jews. I
have known them intimately in South Africa. Some of them became
life-long companions." Philosopher Martin Buber was highly critical of
Gandhi's approach and in 1939 wrote an open letter to him on the
subject. Gandhi reiterated his stance that "the Jews seek to convert the
Arab heart", and use "satyagraha in confronting the Arabs" in 1947.
According to Simone Panter-Brick, Gandhi's political position on
Jewish-Arab conflict evolved over the 1917-1947 period, shifting from a
support for the Arab position first, and for the Jewish position in the
1940s.

\section{On life, society and other application of his
ideas}\label{on-life-society-and-other-application-of-his-ideas}

\section{Vegetarianism, food, and
animals}\label{vegetarianism-food-and-animals}

\begin{itemize}
\item
  \emph{For some of these experiments, Gandhi combined his own ideas
  with those found on diet in Indian yoga texts.}
\item
  \emph{Gandhi's rationale for vegetarianism was largely along those
  found in Hindu and Jain texts.}
\item
  \emph{Gandhi championed animal rights in general.}
\item
  \emph{Gandhi was brought up as a vegetarian by his devout Hindu
  mother.}
\end{itemize}

Gandhi was brought up as a vegetarian by his devout Hindu mother. The
idea of vegetarianism is deeply ingrained in Hindu Vaishnavism and Jain
traditions in India, such as in his native Gujarat, where meat is
considered as a form of food obtained by violence to animals. Gandhi's
rationale for vegetarianism was largely along those found in Hindu and
Jain texts. Gandhi believed that any form of food inescapably harms some
form of living organism, but one should seek to understand and reduce
the violence in what one consumes because "there is essential unity of
all life".

Gandhi believed that some life forms are more capable of suffering, and
non-violence to him meant not having the intent as well as active
efforts to minimise hurt, injury or suffering to all life forms. Gandhi
explored food sources that reduced violence to various life forms in the
food chain. He believed that slaughtering animals is unnecessary, as
other sources of foods are available. He also consulted with
vegetarianism campaigners during his lifetime, such as with Henry
Stephens Salt. Food to Gandhi was not only a source of sustaining one's
body, but a source of his impact on other living beings, and one that
affected his mind, character and spiritual well being. He avoided not
only meat, but also eggs and milk. Gandhi wrote the book The Moral Basis
of Vegetarianism and wrote for the London Vegetarian Society's
publication.

Beyond his religious beliefs, Gandhi stated another motivation for his
experiments with diet. He attempted to find the most non-violent
vegetarian meal that the poorest human could afford, taking meticulous
notes on vegetables and fruits, and his observations with his own body
and his ashram in Gujarat. He tried fresh and dry fruits
(Fruitarianism), then just sun dried fruits, before resuming his prior
vegetarian diet on advice of his doctor and concerns of his friends. His
experiments with food began in the 1890s and continued for several
decades. For some of these experiments, Gandhi combined his own ideas
with those found on diet in Indian yoga texts. He believed that each
vegetarian should experiment with his or her diet because, in his
studies at his ashram he saw "one man's food may be poison for another".

Gandhi championed animal rights in general. Other than making vegetarian
choices, he actively campaigned against dissection studies and
experimentation on live animals (vivisection) in the name of science and
medical studies. He considered it a violence against animals, something
that inflicted pain and suffering. He wrote, "Vivisection in my opinion
is the blackest of all the blackest crimes that man is at present
committing against god and his fair creation."

\includegraphics[width=5.38267in,height=5.50000in]{media/image19.jpg}\\
\emph{Gandhi's last political protest using fasting, in January 1948}

\section{Fasting}\label{fasting}

\begin{itemize}
\item
  \emph{Gandhi's 1943 hunger strike took place during a two-year prison
  term for the anticolonial Quit India movement.}
\item
  \emph{These records indicate that despite being underweight at 46.7
  kgs Gandhi was generally healthy.}
\item
  \emph{Gandhi used fasting as a political device, often threatening
  suicide unless demands were met.}
\item
  \emph{Gandhi believed yoga offered health benefits.}
\end{itemize}

Gandhi used fasting as a political device, often threatening suicide
unless demands were met. Congress publicised the fasts as a political
action that generated widespread sympathy. In response the government
tried to manipulate news coverage to minimise his challenge to the Raj.
He fasted in 1932 to protest the voting scheme for separate political
representation for Dalits; Gandhi did not want them segregated. The
British government stopped the London press from showing photographs of
his emaciated body, because it would elicit sympathy. Gandhi's 1943
hunger strike took place during a two-year prison term for the
anticolonial Quit India movement. The government called on nutritional
experts to demystify his action, and again no photos were allowed.
However, his final fast in 1948, after the end of British rule in India,
his hunger strike was lauded by the British press and this time did
include full-length photos.

Alter states that Gandhi's fasting, vegetarianism and diet was more than
a political leverage, it was a part of his experiments with self
restraint and healthy living. He was "profoundly skeptical of
traditional Ayurveda", encouraging it to study the scientific method and
adopt its progressive learning approach. Gandhi believed yoga offered
health benefits. He believed that a healthy nutritional diet based on
regional foods and hygiene were essential to good health. Recently ICMR
made Gandhi's health records public in a book 'Gandhi and Health@150'.
These records indicate that despite being underweight at 46.7 kgs Gandhi
was generally healthy. He avoided modern medication and experiemented
extensively with water and earth healing. While his cardio records show
his heart was normal, there were several instances he suffered from
ailments like Malaria and was also operated twice for piles and
appendicts. Despite health challenges Gandhi was able to walk about
79000 kms in his lifetime which comes to an average of 18 kms per day
and is equivalent to walking around the earth twice.

\section{Women}\label{women}

\begin{itemize}
\item
  \emph{Women, to Gandhi, should be educated to be better in the
  domestic realm and educate the next generation.}
\item
  \emph{To Gandhi, the women of India were an important part of the
  "swadeshi movement" (Buy Indian), and his goal of decolonising the
  Indian economy.}
\item
  \emph{Gandhi strongly favoured the emancipation of women, and urged
  "the women to fight for their own self-development."}
\end{itemize}

Gandhi strongly favoured the emancipation of women, and urged "the women
to fight for their own self-development." He opposed purdah, child
marriage, dowry and sati. A wife is not a slave of the husband, stated
Gandhi, but his comrade, better half, colleague and friend, according to
Lyn Norvell. In his own life however, according to Suruchi
Thapar-Bjorkert, Gandhi's relationship with his wife were at odds with
some of these values.

At various occasions, Gandhi credited his orthodox Hindu mother, and his
wife, for first lessons in satyagraha. He used the legends of Hindu
goddess Sita to expound women's innate strength, autonomy and "lioness
in spirit" whose moral compass can make any demon "as helpless as a
goat". To Gandhi, the women of India were an important part of the
"swadeshi movement" (Buy Indian), and his goal of decolonising the
Indian economy.

Some historians such as Angela Woollacott and Kumari Jayawardena state
that even though Gandhi often and publicly expressed his belief in the
equality of sexes, yet his vision was one of gender difference and
complementarity between them. Women, to Gandhi, should be educated to be
better in the domestic realm and educate the next generation. His views
on women's rights were less liberal and more similar to
puritan-Victorian expectations of women, states Jayawardena, than other
Hindu leaders with him who supported economic independence and equal
gender rights in all aspects.

\section{Brahmacharya: abstinence from sex and
food}\label{brahmacharya-abstinence-from-sex-and-food}

\begin{itemize}
\item
  \emph{None of the women who participated in the brahmachari
  experiments of Gandhi indicated that they had sex or that Gandhi
  behaved in any sexual way.}
\item
  \emph{Nirmalkumar Bose, Gandhi's Bengali interpreter, for example
  criticised Gandhi, not because Gandhi did anything wrong, but because
  Bose was concerned about the psychological effect on the women who
  participated in his experiments.}
\end{itemize}

Along with many other texts, Gandhi studied Bhagavad Gita while in South
Africa. This Hindu scripture discusses jnana yoga, bhakti yoga and karma
yoga along with virtues such as non-violence, patience, integrity, lack
of hypocrisy, self restraint and abstinence. Gandhi began experiments
with these, and in 1906 at age 37, although married and a father, he
vowed to abstain from sexual relations.

Gandhi's experiment with abstinence went beyond sex, and extended to
food. He consulted the Jain scholar Rajchandra, whom he fondly called
Raychandbhai. Rajchandra advised him that milk stimulated sexual
passion. Gandhi began abstaining from cow's milk in 1912, and did so
even when doctors advised him to consume milk. According to Sankar
Ghose, Tagore described Gandhi as someone who did not abhor sex or
women, but considered sexual life as inconsistent with his moral goals.

Gandhi tried to test and prove to himself his brahmacharya. The
experiments began some time after the death of his wife in February
1944. At the start of his experiment he had women sleep in the same room
but in different beds. He later slept with women in the same bed but
clothed, and finally he slept naked with women. In April 1945, Gandhi
referenced being naked with several "women or girls" in a letter to
Birla as part of the experiments. According to the 1960s memoir of his
grandniece Manu, Gandhi feared in early 1947 that he and she may be
killed by Muslims in the run up to India's independence in August 1947,
and asked her when she was 18 years old if she wanted to help him with
his experiments to test their "purity", for which she readily accepted.
Gandhi slept naked in the same bed with Manu with the bedroom doors open
all night. Manu stated that the experiment had no "ill effect" on her.
Gandhi also shared his bed with 18-year-old Abha, wife of his
grandnephew Kanu. Gandhi would sleep with both Manu and Abha at the same
time. None of the women who participated in the brahmachari experiments
of Gandhi indicated that they had sex or that Gandhi behaved in any
sexual way. Those who went public said they felt as though they were
sleeping with their ageing mother.

According to Sean Scalmer, Gandhi in his final year of life was an
ascetic, looked ugly and a sickly skeletal figure, already caricatured
in the Western media. In February 1947, he asked his confidants such as
Birla and Ramakrishna if it was wrong for him to experiment his
brahmacharya oath. Gandhi's public experiments, as they progressed, were
widely discussed and criticised by his family members and leading
politicians. However, Gandhi said that if he would not let Manu sleep
with him, it would be a sign of weakness. Some of his staff resigned,
including two of his newspaper's editors who had refused to print some
of Gandhi's sermons dealing with his experiments. Nirmalkumar Bose,
Gandhi's Bengali interpreter, for example criticised Gandhi, not because
Gandhi did anything wrong, but because Bose was concerned about the
psychological effect on the women who participated in his experiments.
Veena Howard states Gandhi's views on brahmacharya and religious
renunciation experiments were a method to confront women issues in his
times.

\section{Untouchability and castes}\label{untouchability-and-castes}

\begin{itemize}
\item
  \emph{Gandhi spoke out against untouchability early in his life.}
\item
  \emph{The criticism of Gandhi by Ambedkar continued to influence the
  Dalit movement past Gandhi's death.}
\item
  \emph{According to Arthur Herman, Ambedkar's hate for Gandhi and
  Gandhi's ideas was so strong that after he heard the news of Gandhi's
  assassination, remarked after a momentary silence a sense of regret
  and then "my real enemy is gone; thank goodness the eclipse is over
  now".}
\end{itemize}

Gandhi spoke out against untouchability early in his life. Before 1932,
he and his colleagues used the term Antyaja for untouchables. One of the
major speeches he made on untouchability was at Nagpur in 1920, where he
called untouchability as a great evil in Hindu society. In his remarks,
he stated that the phenomena of untouchability is not unique to the
Hindu society, but has deeper roots because Europeans in South Africa
treat "all of us, Hindus and Muslims, as untouchables; we may not reside
in their midst, nor enjoy the rights which they do". He called it
intolerable. He stated this practice can be eradicated, Hinduism is
flexible to allow this, and a concerted effort is needed to persuade it
is wrong and by all to eradicate it.

According to Christophe Jaffrelot, while Gandhi considered
untouchability to be wrong and evil, he believed that caste or class are
based neither on inequality nor on inferiority. Gandhi believed that
individuals should freely intermarry whoever they want to, but no one
should expect everyone to befriend them. Every individual regardless of
his or her background, stated Gandhi, has a right to choose who they
welcome into their home, who they befriend and who they spend time with.

In 1932, Gandhi began a new campaign to improve the lives of the
untouchables, whom he started referring to as Harijans or "the children
of god". On 8 May 1933, Gandhi began a 21-day fast of self-purification
and launched a one-year campaign to help the Harijan movement. This new
campaign was not universally embraced within the Dalit community.
Ambedkar and his allies felt Gandhi was being paternalistic and was
undermining Dalit political rights. Ambedkar described him as "devious
and untrustworthy". He accused Gandhi as someone who wished to retain
the caste system. Ambedkar and Gandhi debated their ideas and concerns,
where both tried to persuade each other.

In 1935, Ambedkar announced his intentions to leave Hinduism and join
Buddhism. According to Sankar Ghose, the announcement shook Gandhi, who
reappraised his views and wrote many essays with his views on castes,
inter-marriage and what Hinduism says on the subject. These views
contrasted with those of Ambedkar. In actual elections of 1937, except
for some seats in Mumbai where Ambedkar's party won, India's
untouchables voted heavily in favour of Gandhi's campaign and his party,
the Congress.

Gandhi and his colleagues continued to consult Ambedkar, keeping him
influential. Ambedkar worked with other Congress leaders through the
1940s, wrote large parts of India's constitution in the late 1940s, and
converted to Buddhism in 1956. According to Jaffrelot, Gandhi's views
evolved between the 1920s and 1940s, when in 1946 he actively encouraged
inter-marriage across castes. However, Gandhi's approach to
untouchability was different from Ambedkar's, championing fusion, choice
and free intermixing. Ambedkar envisioned each segment of society
maintaining its identity group, and each group then separately advancing
the "politics of equality".

The criticism of Gandhi by Ambedkar continued to influence the Dalit
movement past Gandhi's death. According to Arthur Herman, Ambedkar's
hate for Gandhi and Gandhi's ideas was so strong that after he heard the
news of Gandhi's assassination, remarked after a momentary silence a
sense of regret and then "my real enemy is gone; thank goodness the
eclipse is over now". According to Ramachandra Guha, "ideologues have
carried these old rivalries into the present, with the demonization of
Gandhi now common among politicians who presume to speak in Ambedkar's
name."

\section{Nai Talim, basic education}\label{nai-talim-basic-education}

\begin{itemize}
\item
  \emph{In his autobiography, Gandhi wrote that he believed every Hindu
  boy and girl must learn Sanskrit because its historic and spiritual
  texts are in that language.}
\item
  \emph{Gandhi rejected the colonial Western format of education
  system.}
\item
  \emph{Gandhi called his ideas Nai Talim (literally, 'new education').}
\item
  \emph{Nehru government's vision of an industrialised, centrally
  planned economy after 1947 had scant place for Gandhi's
  village-oriented approach.}
\end{itemize}

Gandhi rejected the colonial Western format of education system. He
stated that it led to disdain for manual work, generally created an
elite administrative bureaucracy. Gandhi favoured an education system
with far greater emphasis on learning skills in practical and useful
work, one that included physical, mental and spiritual studies. His
methodology sought to treat all professions equal and pay everyone the
same.

Gandhi called his ideas Nai Talim (literally, 'new education'). He
believed that the Western style education violated and destroyed the
indigenous cultures. A different basic education model, he believed,
would lead to better self awareness, prepare people to treat all work
equally respectable and valued, and lead to a society with less social
diseases.

Nai Talim evolved out of his experiences at the Tolstoy Farm in South
Africa, and Gandhi attempted to formulate the new system at the Sevagram
ashram after 1937. Nehru government's vision of an industrialised,
centrally planned economy after 1947 had scant place for Gandhi's
village-oriented approach.

In his autobiography, Gandhi wrote that he believed every Hindu boy and
girl must learn Sanskrit because its historic and spiritual texts are in
that language.

\section{Swaraj, self-rule}\label{swaraj-self-rule}

\begin{itemize}
\item
  \emph{According to Gandhi, a non-violent state is like an "ordered
  anarchy".}
\item
  \emph{Gandhi believed that swaraj not only can be attained with
  non-violence, it can be run with non-violence.}
\item
  \emph{Tewari states that Gandhi saw democracy as more than a system of
  government; it meant promoting both individuality and the
  self-discipline of the community.}
\item
  \emph{"This is not the Swaraj I want", said Gandhi.}
\item
  \emph{For Gandhi, democracy was a way of life.}
\end{itemize}

Gandhi believed that swaraj not only can be attained with non-violence,
it can be run with non-violence. A military is unnecessary, because any
aggressor can be thrown out using the method of non-violent
non-co-operation. While military is unnecessary in a nation organised
under swaraj principle, Gandhi added that a police force is necessary
given human nature. However, the state would limit the use of weapons by
the police to the minimum, aiming for their use as a restraining force.

According to Gandhi, a non-violent state is like an "ordered anarchy".
In a society of mostly non-violent individuals, those who are violent
will sooner or later accept discipline or leave the community, stated
Gandhi. He emphasised a society where individuals believed more in
learning about their duties and responsibilities, not demanded rights
and privileges. On returning from South Africa, when Gandhi received a
letter asking for his participation in writing a world charter for human
rights, he responded saying, "in my experience, it is far more important
to have a charter for human duties."

Swaraj to Gandhi did not mean transferring colonial era British power
brokering system, favours-driven, bureaucratic, class exploitative
structure and mindset into Indian hands. He warned such a transfer would
still be English rule, just without the Englishman. "This is not the
Swaraj I want", said Gandhi. Tewari states that Gandhi saw democracy as
more than a system of government; it meant promoting both individuality
and the self-discipline of the community. Democracy meant settling
disputes in a nonviolent manner; it required freedom of thought and
expression. For Gandhi, democracy was a way of life.

\section{Hindu nationalism and
revivalism}\label{hindu-nationalism-and-revivalism}

\begin{itemize}
\item
  \emph{Some scholars state Gandhi supported a religiously diverse
  India, while others state that the Muslim leaders who championed the
  partition and creation of a separate Muslim Pakistan considered Gandhi
  to be Hindu nationalist or revivalist.}
\item
  \emph{For example, in his letters to Mohammad Iqbal, Jinnah accused
  Gandhi to be favouring a Hindu rule and revivalism, that Gandhi led
  Indian National Congress was a fascist party.}
\end{itemize}

Some scholars state Gandhi supported a religiously diverse India, while
others state that the Muslim leaders who championed the partition and
creation of a separate Muslim Pakistan considered Gandhi to be Hindu
nationalist or revivalist. For example, in his letters to Mohammad
Iqbal, Jinnah accused Gandhi to be favouring a Hindu rule and
revivalism, that Gandhi led Indian National Congress was a fascist
party.

In an interview with C.F. Andrews, Gandhi stated that if we believe all
religions teach the same message of love and peace between all human
beings, then there is neither any rationale nor need for proselytisation
or attempts to convert people from one religion to another. Gandhi
opposed missionary organisations who criticised Indian religions then
attempted to convert followers of Indian religions to Islam or
Christianity. In Gandhi's view, those who attempt to convert a Hindu,
"they must harbour in their breasts the belief that Hinduism is an
error" and that their own religion is "the only true religion". Gandhi
believed that people who demand religious respect and rights must also
show the same respect and grant the same rights to followers of other
religions. He stated that spiritual studies must encourage "a Hindu to
become a better Hindu, a Mussalman to become a better Mussalman, and a
Christian a better Christian."

According to Gandhi, religion is not about what a man believes, it is
about how a man lives, how he relates to other people, his conduct
towards others, and one's relationship to one's conception of god. It is
not important to convert or to join any religion, but it is important to
improve one's way of life and conduct by absorbing ideas from any source
and any religion, believed Gandhi.

\section{Gandhian economics}\label{gandhian-economics}

\begin{itemize}
\item
  \emph{Gandhi challenged Nehru and the modernizers in the late 1930s
  who called for rapid industrialisation on the Soviet model; Gandhi
  denounced that as dehumanising and contrary to the needs of the
  villages where the great majority of the people lived.}
\item
  \emph{Gandhi believed in sarvodaya economic model, which literally
  means "welfare, upliftment of all".}
\item
  \emph{Violence against any human being, born poor or rich, is wrong,
  believed Gandhi.}
\item
  \emph{After Gandhi's assassination, Nehru led India in accordance with
  his personal socialist convictions.}
\end{itemize}

Gandhi believed in sarvodaya economic model, which literally means
"welfare, upliftment of all". This, states Bhatt, was a very different
economic model than the socialism model championed and followed by free
India by Nehru -- India's first prime minister. To both, according to
Bhatt, removing poverty and unemployment were the objective, but the
Gandhian economic and development approach preferred adapting technology
and infrastructure to suit the local situation, in contrast to Nehru's
large scale, socialised state owned enterprises.

To Gandhi, the economic philosophy that aims at "greatest good for the
greatest number" was fundamentally flawed, and his alternative proposal
sarvodaya set its aim at the "greatest good for all". He believed that
the best economic system not only cared to lift the "poor, less skilled,
of impoverished background" but also empowered to lift the "rich, highly
skilled, of capital means and landlords". Violence against any human
being, born poor or rich, is wrong, believed Gandhi. He stated that the
mandate theory of majoritarian democracy should not be pushed to absurd
extremes, individual freedoms should never be denied, and no person
should ever be made a social or economic slave to the "resolutions of
majorities".

Gandhi challenged Nehru and the modernizers in the late 1930s who called
for rapid industrialisation on the Soviet model; Gandhi denounced that
as dehumanising and contrary to the needs of the villages where the
great majority of the people lived. After Gandhi's assassination, Nehru
led India in accordance with his personal socialist convictions.
Historian Kuruvilla Pandikattu says "it was Nehru's vision, not
Gandhi's, that was eventually preferred by the Indian State."

Gandhi called for ending poverty through improved agriculture and
small-scale cottage rural industries. Gandhi's economic thinking
disagreed with Marx, according to the political theory scholar and
economist Bhikhu Parekh. Gandhi refused to endorse the view that
economic forces are best understood as "antagonistic class interests".
He argued that no man can degrade or brutalise the other without
degrading and brutalising himself and that sustainable economic growth
comes from service, not from exploitation. Further, believed Gandhi, in
a free nation, victims exist only when they co-operate with their
oppressor, and an economic and political system that offered increasing
alternatives gave power of choice to the poorest man.

While disagreeing with Nehru about the socialist economic model, Gandhi
also critiqued capitalism that was driven by endless wants and a
materialistic view of man. This, he believed, created a vicious vested
system of materialism at the cost of other human needs such as
spirituality and social relationships. To Gandhi, states Parekh, both
communism and capitalism were wrong, in part because both focussed
exclusively on materialistic view of man, and because the former deified
the state with unlimited power of violence, while the latter deified
capital. He believed that a better economic system is one which does not
impoverish one's culture and spiritual pursuits.

\section{Gandhism}\label{gandhism}

\begin{itemize}
\item
  \emph{Gandhism designates the ideas and principles Gandhi promoted; of
  central importance is nonviolent resistance.}
\item
  \emph{However Gandhi himself did not approve of the notion of
  "Gandhism", as he explained in 1936:}
\end{itemize}

Gandhism designates the ideas and principles Gandhi promoted; of central
importance is nonviolent resistance. A Gandhian can mean either an
individual who follows, or a specific philosophy which is attributed to,
Gandhism. M. M. Sankhdher argues that Gandhism is not a systematic
position in metaphysics or in political philosophy. Rather, it is a
political creed, an economic doctrine, a religious outlook, a moral
precept, and especially, a humanitarian world view. It is an effort not
to systematise wisdom but to transform society and is based on an
undying faith in the goodness of human nature. However Gandhi himself
did not approve of the notion of "Gandhism", as he explained in 1936:

\includegraphics[width=3.97727in,height=5.50000in]{media/image20.png}\\
\emph{Young India, a weekly journal published by Gandhi from 1919 to
1932}

\section{Literary works}\label{literary-works}

\begin{itemize}
\item
  \emph{Gandhi was a prolific writer.}
\item
  \emph{One of Gandhi's earliest publications, Hind Swaraj, published in
  Gujarati in 1909, became "the intellectual blueprint" for India's
  independence movement.}
\item
  \emph{Gandhi usually wrote in Gujarati, though he also revised the
  Hindi and English translations of his books.}
\item
  \emph{Gandhi's complete works were published by the Indian government
  under the name The Collected Works of Mahatma Gandhi in the 1960s.}
\end{itemize}

Gandhi was a prolific writer. One of Gandhi's earliest publications,
Hind Swaraj, published in Gujarati in 1909, became "the intellectual
blueprint" for India's independence movement. The book was translated
into English the next year, with a copyright legend that read "No Rights
Reserved". For decades he edited several newspapers including Harijan in
Gujarati, in Hindi and in the English language; Indian Opinion while in
South Africa and, Young India, in English, and Navajivan, a Gujarati
monthly, on his return to India. Later, Navajivan was also published in
Hindi. In addition, he wrote letters almost every day to individuals and
newspapers.

Gandhi also wrote several books including his autobiography, The Story
of My Experiments with Truth (Gujarātī "સત્યના પ્રયોગો અથવા આત્મકથા"),
of which he bought the entire first edition to make sure it was
reprinted. His other autobiographies included: Satyagraha in South
Africa about his struggle there, Hind Swaraj or Indian Home Rule, a
political pamphlet, and a paraphrase in Gujarati of John Ruskin's Unto
This Last. This last essay can be considered his programme on economics.
He also wrote extensively on vegetarianism, diet and health, religion,
social reforms, etc. Gandhi usually wrote in Gujarati, though he also
revised the Hindi and English translations of his books.

Gandhi's complete works were published by the Indian government under
the name The Collected Works of Mahatma Gandhi in the 1960s. The
writings comprise about 50,000 pages published in about a hundred
volumes. In 2000, a revised edition of the complete works sparked a
controversy, as it contained a large number of errors and omissions. The
Indian government later withdrew the revised edition.

\section{Legacy and depictions in popular
culture}\label{legacy-and-depictions-in-popular-culture}

\begin{itemize}
\item
  \emph{Rabindranath Tagore is said to have accorded the title to
  Gandhi.}
\item
  \emph{These include M.G.Road (the main street of a number of Indian
  cities including Mumbai and Bangalore), Gandhi Market (near Sion,
  Mumbai) and Gandhinagar (the capital of the state of Gujarat, Gandhi's
  birthplace).}
\item
  \emph{In his autobiography, Gandhi nevertheless explains that he never
  valued the title, and was often pained by it.}
\end{itemize}

The word Mahatma, while often mistaken for Gandhi's given name in the
West, is taken from the Sanskrit words maha (meaning Great) and atma
(meaning Soul). Rabindranath Tagore is said to have accorded the title
to Gandhi. In his autobiography, Gandhi nevertheless explains that he
never valued the title, and was often pained by it.

Innumerable streets, roads and localities in India are named after
M.K.Gandhi. These include M.G.Road (the main street of a number of
Indian cities including Mumbai and Bangalore), Gandhi Market (near Sion,
Mumbai) and Gandhinagar (the capital of the state of Gujarat, Gandhi's
birthplace).

\includegraphics[width=4.02600in,height=5.50000in]{media/image21.jpg}\\
\emph{Mahatma Gandhi on a 1969 postage stamp of the Soviet Union}

\section{Followers and international
influence}\label{followers-and-international-influence}

\begin{itemize}
\item
  \emph{The Mahatma Gandhi District in Houston, Texas, United States, an
  ethnic Indian enclave, is officially named after Gandhi.}
\item
  \emph{Gandhi's life and teachings inspired many who specifically
  referred to Gandhi as their mentor or who dedicated their lives to
  spreading Gandhi's ideas.}
\item
  \emph{In Europe, Romain Rolland was the first to discuss Gandhi in his
  1924 book Mahatma Gandhi, and Brazilian anarchist and feminist Maria
  Lacerda de Moura wrote about Gandhi in her work on pacifism.}
\item
  \emph{Einstein said of Gandhi:}
\end{itemize}

Gandhi influenced important leaders and political movements. Leaders of
the civil rights movement in the United States, including Martin Luther
King Jr., James Lawson, and James Bevel, drew from the writings of
Gandhi in the development of their own theories about nonviolence. King
said "Christ gave us the goals and Mahatma Gandhi the tactics." King
sometimes referred to Gandhi as "the little brown saint." Anti-apartheid
activist and former President of South Africa, Nelson Mandela, was
inspired by Gandhi. Others include Khan Abdul Ghaffar Khan, Steve Biko,
and Aung San Suu Kyi.

In his early years, the former President of South Africa Nelson Mandela
was a follower of the nonviolent resistance philosophy of Gandhi. Bhana
and Vahed commented on these events as "Gandhi inspired succeeding
generations of South African activists seeking to end White rule. This
legacy connects him to Nelson Mandela...in a sense Mandela completed
what Gandhi started."

Gandhi's life and teachings inspired many who specifically referred to
Gandhi as their mentor or who dedicated their lives to spreading
Gandhi's ideas. In Europe, Romain Rolland was the first to discuss
Gandhi in his 1924 book Mahatma Gandhi, and Brazilian anarchist and
feminist Maria Lacerda de Moura wrote about Gandhi in her work on
pacifism. In 1931, notable European physicist Albert Einstein exchanged
written letters with Gandhi, and called him "a role model for the
generations to come" in a letter writing about him. Einstein said of
Gandhi:

Lanza del Vasto went to India in 1936 intending to live with Gandhi; he
later returned to Europe to spread Gandhi's philosophy and founded the
Community of the Ark in 1948 (modelled after Gandhi's ashrams).
Madeleine Slade (known as "Mirabehn") was the daughter of a British
admiral who spent much of her adult life in India as a devotee of
Gandhi.

In addition, the British musician John Lennon referred to Gandhi when
discussing his views on nonviolence. At the Cannes Lions International
Advertising Festival in 2007, former US Vice-President and
environmentalist Al Gore spoke of Gandhi's influence on him.

US President Barack Obama in a 2010 address to the Parliament of India
said that:

Obama in September 2009 said that his biggest inspiration came from
Gandhi. His reply was in response to the question 'Who was the one
person, dead or live, that you would choose to dine with?'. He continued
that "He's somebody I find a lot of inspiration in. He inspired Dr. King
with his message of nonviolence. He ended up doing so much and changed
the world just by the power of his ethics."

Time Magazine named The 14th Dalai Lama, Lech Wałęsa, Martin Luther
King, Cesar Chavez, Aung San Suu Kyi, Benigno Aquino, Jr., Desmond Tutu,
and Nelson Mandela as Children of Gandhi and his spiritual heirs to
nonviolence.\\
The Mahatma Gandhi District in Houston, Texas, United States, an ethnic
Indian enclave, is officially named after Gandhi.

On the basis of a petition, a statue of Gandhi at the University of
Ghana was removed on 15 December 2018, because it was viewed by the
petitioners as "an homage to a racist".

\section{Global days that celebrate
Gandhi}\label{global-days-that-celebrate-gandhi}

\begin{itemize}
\item
  \emph{In 2007, the United Nations General Assembly declared Gandhi's
  birthday 2 October as "the International Day of Nonviolence."}
\end{itemize}

In 2007, the United Nations General Assembly declared Gandhi's birthday
2 October as "the International Day of Nonviolence." First proposed by
UNESCO in 1948, as the School Day of Nonviolence and Peace (DENIP in
Spanish), 30 January is observed as the School Day of Nonviolence and
Peace in schools of many countries In countries with a Southern
Hemisphere school calendar, it is observed on 30 March.

\section{Awards}\label{awards}

\begin{itemize}
\item
  \emph{Gandhi could do without the Nobel Peace prize, whether Nobel
  committee can do without Gandhi is the question".}
\item
  \emph{Gandhi was also the runner-up to Albert Einstein as "Person of
  the Century" at the end of 1999.}
\item
  \emph{Gandhi was nominated in 1948 but was assassinated before
  nominations closed.}
\item
  \emph{Time magazine named Gandhi the Man of the Year in 1930.}
\end{itemize}

Time magazine named Gandhi the Man of the Year in 1930. The University
of Nagpur awarded him an LL.D. in 1937. Gandhi was also the runner-up to
Albert Einstein as "Person of the Century" at the end of 1999. The
Government of India awarded the annual Gandhi Peace Prize to
distinguished social workers, world leaders and citizens. Nelson
Mandela, the leader of South Africa's struggle to eradicate racial
discrimination and segregation, was a prominent non-Indian recipient. In
2011, Time magazine named Gandhi as one of the top 25 political icons of
all time.

Gandhi did not receive the Nobel Peace Prize, although he was nominated
five times between 1937 and 1948, including the first-ever nomination by
the American Friends Service Committee, though he made the short list
only twice, in 1937 and 1947. Decades later, the Nobel Committee
publicly declared its regret for the omission, and admitted to deeply
divided nationalistic opinion denying the award. Gandhi was nominated in
1948 but was assassinated before nominations closed. That year, the
committee chose not to award the peace prize stating that "there was no
suitable living candidate" and later research shows that the possibility
of awarding the prize posthumously to Gandhi was discussed and that the
reference to no suitable living candidate was to Gandhi. Geir Lundestad,
Secretary of Norwegian Nobel Committee in 2006 said, "The greatest
omission in our 106-year history is undoubtedly that Mahatma Gandhi
never received the Nobel Peace prize. Gandhi could do without the Nobel
Peace prize, whether Nobel committee can do without Gandhi is the
question". When the 14th Dalai Lama was awarded the Prize in 1989, the
chairman of the committee said that this was "in part a tribute to the
memory of Mahatma Gandhi". In the summer of 1995, the North American
Vegetarian Society inducted him posthumously into the Vegetarian Hall of
Fame.

\section{Father of the Nation}\label{father-of-the-nation}

\begin{itemize}
\item
  \emph{Indians widely describe Gandhi as the father of the nation.}
\item
  \emph{Origin of this title is traced back to a radio address (on
  Singapore radio) on 6 July 1944 by Subhash Chandra Bose where Bose
  addressed Gandhi as "The Father of the Nation".}
\item
  \emph{On 28 April 1947, Sarojini Naidu during a conference also
  referred Gandhi as "Father of the Nation".}
\end{itemize}

Indians widely describe Gandhi as the father of the nation. Origin of
this title is traced back to a radio address (on Singapore radio) on 6
July 1944 by Subhash Chandra Bose where Bose addressed Gandhi as "The
Father of the Nation". On 28 April 1947, Sarojini Naidu during a
conference also referred Gandhi as "Father of the Nation".

\section{Film, theatre and
literature}\label{film-theatre-and-literature}

\begin{itemize}
\item
  \emph{Life of Mohandas Karamchand Gandhi in eight volumes, Chaman
  Nahal's Gandhi Quartet, and Pyarelal and Sushila Nayyar with their
  Mahatma Gandhi in 10 volumes.}
\item
  \emph{The 1995 Marathi play Gandhi Virudh Gandhi explored the
  relationship between Gandhi and his son Harilal.}
\item
  \emph{The 2014 film Welcome Back Gandhi takes a fictionalised look at
  how Gandhi might react to modern day India.}
\end{itemize}

A 5 hour 9 minute long biographical documentary film, Mahatma: Life of
Gandhi, 1869--1948, made by Vithalbhai Jhaveri in 1968, quoting Gandhi's
words and using black \& white archival footage and photographs,
captures the history of those times. Ben Kingsley portrayed him in
Richard Attenborough's 1982 film Gandhi, which won the Academy Award for
Best Picture. It was based on the biography by Louis Fischer. The 1996
film The Making of the Mahatma documented Gandhi's time in South Africa
and his transformation from an inexperienced barrister to recognised
political leader. Gandhi was a central figure in the 2006 Bollywood
comedy film Lage Raho Munna Bhai. Jahnu Barua's Maine Gandhi Ko Nahin
Mara (I did not kill Gandhi), places contemporary society as a backdrop
with its vanishing memory of Gandhi's values as a metaphor for the
senile forgetfulness of the protagonist of his 2005 film, writes Vinay
Lal.

The 1979 opera Satyagraha by American composer Philip Glass is loosely
based on Gandhi's life. The opera's libretto, taken from the Bhagavad
Gita, is sung in the original Sanskrit.

Anti-Gandhi themes have also been showcased through films and plays. The
1995 Marathi play Gandhi Virudh Gandhi explored the relationship between
Gandhi and his son Harilal. The 2007 film, Gandhi, My Father was
inspired on the same theme. The 1989 Marathi play Me Nathuram Godse
Boltoy and the 1997 Hindi play Gandhi Ambedkar criticised Gandhi and his
principles.

Several biographers have undertaken the task of describing Gandhi's
life. Among them are D. G. Tendulkar with his Mahatma. Life of Mohandas
Karamchand Gandhi in eight volumes, Chaman Nahal's Gandhi Quartet, and
Pyarelal and Sushila Nayyar with their Mahatma Gandhi in 10 volumes. The
2010 biography, Great Soul: Mahatma Gandhi and His Struggle With India
by Joseph Lelyveld contained controversial material speculating about
Gandhi's sexual life. Lelyveld, however, stated that the press coverage
"grossly distort{[}s{]}" the overall message of the book. The 2014 film
Welcome Back Gandhi takes a fictionalised look at how Gandhi might react
to modern day India. The 2019 play Bharat Bhagya Vidhata, inspired by
Pujya Gurudevshri Rakeshbhai and produced by Sangeet Natak Akademi and
Shrimad Rajchandra Mission Dharampur takes a look at how Gandhi
cultivated the values of truth and non-violence.

"Mahatma Gandhi" is used by Cole Porter in his lyrics for the song
You're the Top which is included in the 1934 musical Anything Goes. In
the song Porter rhymes "Mahatma Gandhi' with "Napoleon Brandy."

\section{Current impact within India}\label{current-impact-within-india}

\begin{itemize}
\item
  \emph{There are three temples in India dedicated to Gandhi.}
\item
  \emph{Gandhi's date of death, 30 January, is commemorated as a
  Martyrs' Day in India.}
\item
  \emph{Gandhi's birthday, 2 October, is a national holiday in India,
  Gandhi Jayanti.}
\item
  \emph{The Gandhi Memorial in Kanyakumari resembles central Indian
  Hindu temples and the Tamukkam or Summer Palace in Madurai now houses
  the Mahatma Gandhi Museum.}
\end{itemize}

India, with its rapid economic modernisation and urbanisation, has
rejected Gandhi's economics but accepted much of his politics and
continues to revere his memory. Reporter Jim Yardley notes that, "modern
India is hardly a Gandhian nation, if it ever was one. His vision of a
village-dominated economy was shunted aside during his lifetime as rural
romanticism, and his call for a national ethos of personal austerity and
nonviolence has proved antithetical to the goals of an aspiring economic
and military power." By contrast Gandhi is "given full credit for
India's political identity as a tolerant, secular democracy."

Gandhi's birthday, 2 October, is a national holiday in India, Gandhi
Jayanti. Gandhi's image also appears on paper currency of all
denominations issued by Reserve Bank of India, except for the one rupee
note. Gandhi's date of death, 30 January, is commemorated as a Martyrs'
Day in India.

There are three temples in India dedicated to Gandhi. One is located at
Sambalpur in Orissa and the second at Nidaghatta village near Kadur in
Chikmagalur district of Karnataka and the third one at Chityal in the
district of Nalgonda, Telangana. The Gandhi Memorial in Kanyakumari
resembles central Indian Hindu temples and the Tamukkam or Summer Palace
in Madurai now houses the Mahatma Gandhi Museum.

\section{Descendants}\label{descendants}

\begin{itemize}
\item
  \emph{Gandhi's children and grandchildren live in India and other
  countries.}
\item
  \emph{Another grandson, Kanu Ramdas Gandhi (the son of Gandhi's third
  son Ramdas), was found living in an old age home in Delhi despite
  having taught earlier in the United States.}
\item
  \emph{Grandson Rajmohan Gandhi is a Professor in Illinois and an
  author of Gandhi's biography titled "Mohandas", while another, Tarun
  Gandhi, has authored several authoritative books on his grandfather.}
\end{itemize}

Gandhi's children and grandchildren live in India and other countries.
Grandson Rajmohan Gandhi is a Professor in Illinois and an author of
Gandhi's biography titled "Mohandas", while another, Tarun Gandhi, has
authored several authoritative books on his grandfather. Another
grandson, Kanu Ramdas Gandhi (the son of Gandhi's third son Ramdas), was
found living in an old age home in Delhi despite having taught earlier
in the United States.

\section{See also}\label{see-also}

\begin{itemize}
\item
  \emph{Gandhi Teerth -- Gandhi International Research Institute and
  Museum for Gandhian study, research on Mahatma Gandhi and dialogue.}
\item
  \emph{Gandhi cap}
\item
  \emph{Mahatma Gandhi and Theosophy}
\item
  \emph{Gandhi (bookstore)}
\end{itemize}

List of peace activists

List of civil rights leaders

Seven Social Sins (AKA Seven Blunders of the World)

Gandhi cap

Gandhi Teerth -- Gandhi International Research Institute and Museum for
Gandhian study, research on Mahatma Gandhi and dialogue.

Trikaranasuddhi

Gandhi (bookstore)

Mahatma Gandhi and Theosophy

\section{References}\label{references}

\section{Bibliography}\label{bibliography}

\section{Books}\label{books}

\section{Primary sources}\label{primary-sources}

\section{External links}\label{external-links}

\begin{itemize}
\item
  \emph{Mani Bhavan Gandhi Sangrahalaya Gandhi Museum \& Library}
\item
  \emph{Gandhi Ashram at Sabarmati}
\item
  \emph{About Mahatma Gandhi}
\item
  \emph{Works by Mahatma Gandhi at Project Gutenberg}
\item
  \emph{Gandhi's correspondence with the Indian government 1942--1944}
\item
  \emph{Mahatma Gandhi at Curlie}
\item
  \emph{Mahatma Gandhi at the Encyclopædia Britannica}
\end{itemize}

Wikilivres has original media or text related to this article: Mohandas
K. Gandhi (in the public domain in New Zealand)

Mahatma Gandhi at the Encyclopædia Britannica

Mahatma Gandhi at Curlie

Gandhi's correspondence with the Indian government 1942--1944

About Mahatma Gandhi

Gandhi Ashram at Sabarmati

Mani Bhavan Gandhi Sangrahalaya Gandhi Museum \& Library

Works by Mahatma Gandhi at Project Gutenberg

Works by or about Mahatma Gandhi at Internet Archive

Works by Mahatma Gandhi at LibriVox (public domain audiobooks)

Newspaper clippings about Mahatma Gandhi in the 20th Century Press
Archives of the German National Library of Economics (ZBW)
