\textbf{From Wikipedia, the free encyclopedia}

https://en.wikipedia.org/wiki/President\%20of\%20Syria\\
Licensed under CC BY-SA 3.0:\\
https://en.wikipedia.org/wiki/Wikipedia:Text\_of\_Creative\_Commons\_Attribution-ShareAlike\_3.0\_Unported\_License

\section{President of Syria}\label{president-of-syria}

\begin{itemize}
\item
  \emph{The President of Syria is the head of state of the Syrian Arab
  Republic.}
\end{itemize}

The President of Syria is the head of state of the Syrian Arab Republic.
He is vested with sweeping powers that may be delegated, at his sole
discretion, to his Vice Presidents. He appoints and dismisses the Prime
Minister and other members of the Council of Ministers (the cabinet) and
military officers.

\section{Term of office}\label{term-of-office}

\begin{itemize}
\item
  \emph{According to article 88 of the Syrian constitution, the
  president runs for a 7-year term after he is elected, and can only be
  reelected for one more term.}
\end{itemize}

According to article 88 of the Syrian constitution, the president runs
for a 7-year term after he is elected, and can only be reelected for one
more term.

\section{Eligibility criteria}\label{eligibility-criteria}

\begin{itemize}
\item
  \emph{Have lived continuously in Syria for 10 years before the
  election}
\item
  \emph{According to articles 84 and 85 of the Syrian constitution, the
  candidate for the office of President of the Republic must:}
\item
  \emph{Also, the Constitution states that "The religion of the
  President of the Republic is Islam"}
\item
  \emph{Be Syrian by birth, of parents who are Syrians by birth}
\end{itemize}

According to articles 84 and 85 of the Syrian constitution, the
candidate for the office of President of the Republic must:

Acquire the support of at least 35 members of the People's Assembly

Be above the age of 34 (as of a new law)

Have lived continuously in Syria for 10 years before the election

Be Syrian by birth, of parents who are Syrians by birth

Not be married to a non-Syrian spouse

Also, the Constitution states that "The religion of the President of the
Republic is Islam"

On 31 January 1973, Hafez al-Assad implemented the new Constitution,
which led to a national crisis. Unlike previous constitutions, this one
did not require that the president of Syria must be a Muslim, leading to
fierce demonstrations in Hama, Homs and Aleppo organized by the Muslim
Brotherhood and the ulama. They labeled Assad as the "enemy of God" and
called for a jihad against his rule. Robert D. Kaplan has compared
Assad's coming to power to "an untouchable becoming maharajah in India
or a Jew becoming tsar in Russia---an unprecedented development shocking
to the Sunni majority population which had monopolized power for so many
centuries."

\section{Powers}\label{powers}

\begin{itemize}
\item
  \emph{Apart from executive authority relating to a wide range of
  governmental functions including foreign affairs, the president has
  the right to dissolve the People's Council, in which case a new
  council must be elected within ninety days from the date of
  dissolution.}
\end{itemize}

Apart from executive authority relating to a wide range of governmental
functions including foreign affairs, the president has the right to
dissolve the People's Council, in which case a new council must be
elected within ninety days from the date of dissolution.

\section{List of Presidents}\label{list-of-presidents}

\section{Latest election}\label{latest-election}

\section{References}\label{references}

\section{External links}\label{external-links}

\begin{itemize}
\item
  \emph{President of Syria on Facebook}
\end{itemize}

President of Syria on Facebook
