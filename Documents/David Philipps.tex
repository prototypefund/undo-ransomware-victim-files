\textbf{From Wikipedia, the free encyclopedia}

https://en.wikipedia.org/wiki/David\%20Philipps\\
Licensed under CC BY-SA 3.0:\\
https://en.wikipedia.org/wiki/Wikipedia:Text\_of\_Creative\_Commons\_Attribution-ShareAlike\_3.0\_Unported\_License

\includegraphics[width=5.50000in,height=3.66016in]{media/image1.JPG}\\
\emph{Journalist David Philipps interviews medical personnel at Fort
Carson in 2013}

\section{David Philipps}\label{david-philipps}

\begin{itemize}
\item
  \emph{He is a national correspondent for the New York Times and is the
  author of two books.}
\item
  \emph{At the New York Times, Philipps covers the military, veterans
  and breaking news.}
\item
  \emph{The second was the 2017 Pulitzer Prize for breaking news, which
  he shared with the New York Times staff for coverage of a mass
  shooting in Las Vegas.}
\end{itemize}

David Nathaniel Philipps, born in 1977, is a Pulitzer Prize-winning
American journalist and author whose work has largely focused on the
human impact of the wars in Iraq and Afghanistan. He is a national
correspondent for the New York Times and is the author of two books. The
most recent is Wild Horse Country.

At the New York Times, Philipps covers the military, veterans and
breaking news.

Philipps won a Pulitzer Prize for National Reporting in 2014 for his
three-day series "Other Than Honorable" in The Gazette of Colorado
Springs on the treatment of injured American soldiers being discharged
without military benefits. He has also been named a finalist for the
Pulitzer Prize twice. The first was the 2010 Pulitzer Prize for local
reporting, which cited "his painstaking stories on the spike in violence
within the Band of Brothers, a battered combat brigade returning to Fort
Carson after bloody deployments to Iraq, leading to increased mental
health care for soldiers." The second was the 2017 Pulitzer Prize for
breaking news, which he shared with the New York Times staff for
coverage of a mass shooting in Las Vegas.

Philipps won the 2009 Livingston Award for his reporting on violence in
infantry troops returning from Iraq. His book, Lethal Warriors
chronicles how the 4th Brigade Combat Team of the 12th Infantry
Regiment, stationed at Fort Carson, Colorado, produced a high number of
murders after soldiers returned from unusually violent combat tours.
Philipps worked for eight years as an enterprise reporter at the
Colorado Springs Gazette.

Philipps has written extensively about wild horses in the West, and
gained attention in 2012 when U.S. Secretary of Interior Ken Salazar
threatened to punch him while Philipps was asking about troubles in the
department's wild horse program. Philipps' subsequent reporting led to
state and federal investigation of the wild horse program and its
largest horse buyer. His latest book, Wild Horse Country, traces the
culture and history that created modern wild horse management. Philipps'
writing on wild horse management has faced criticism as being based on
unsound science.

Philipps graduated from Middlebury College in 2000 and earned a master's
degree from the Columbia University Graduate School of Journalism in
2002.

\section{Notable work}\label{notable-work}

\begin{itemize}
\item
  \emph{"In unit stalked by suicide, members try to save one another" ,"
  The New York Times, Sept. 19, 2015}
\item
  \emph{"Wounded Warrior Project Spends Lavishly on Itself " The New
  York Times, January 27, 2016}
\item
  \emph{"A Marine Attacked an Iraqi Restaurant, But was it a Hate Crime
  or PTSD" ," The New York Times, Oct. 18, 2017}
\item
  \emph{"Other than Honorable," The Colorado Springs Gazette, May 19,
  2013}
\end{itemize}

"A Marine Attacked an Iraqi Restaurant, But was it a Hate Crime or PTSD"
," The New York Times, Oct. 18, 2017

"Wounded Warrior Project Spends Lavishly on Itself " The New York Times,
January 27, 2016

"In unit stalked by suicide, members try to save one another" ," The New
York Times, Sept. 19, 2015

"Other than Honorable," The Colorado Springs Gazette, May 19, 2013

"Casualties of War," The Colorado Springs Gazette, July 28, 2009.

"All the missing horses," ProPublica, Sept. 28, 2012

"Honor and Deception," The Colorado Springs Gazette, Dec. 1, 2013

\section{References}\label{references}
