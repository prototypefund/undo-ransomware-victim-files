\textbf{From Wikipedia, the free encyclopedia}

https://en.wikipedia.org/wiki/Tim\%20Scott\\
Licensed under CC BY-SA 3.0:\\
https://en.wikipedia.org/wiki/Wikipedia:Text\_of\_Creative\_Commons\_Attribution-ShareAlike\_3.0\_Unported\_License

\section{Tim Scott}\label{tim-scott}

\begin{itemize}
\item
  \emph{A member of the Republican Party, Scott was endorsed for the
  Senate by Tea Party groups.}
\item
  \emph{He was the first Republican African-American U.S. Representative
  from South Carolina since 1897.}
\item
  \emph{In 2010, Scott was elected to the United States House of
  Representatives for South Carolina's 1st congressional district, where
  he served from 2011 to 2013.}
\end{itemize}

Timothy Eugene Scott (born September 19, 1965) is an American politician
and businessman serving as the junior United States Senator from South
Carolina since 2013. Appointed by Governor Nikki Haley to replace the
retiring Jim DeMint, he later won a special election in 2014 and was
elected to a full term in 2016. A member of the Republican Party, Scott
was endorsed for the Senate by Tea Party groups.

In 2010, Scott was elected to the United States House of Representatives
for South Carolina's 1st congressional district, where he served from
2011 to 2013. Previously, Scott served one term (from 2009 to 2011) in
the South Carolina General Assembly and served on the Charleston County
council from 1996 to 2008.

Since January~2017,{[}update{]} Scott has been one of three
African-Americans in the U.S. Senate, and the first to serve in both
chambers of Congress. He is the first African-American senator from the
state of South Carolina, the first African-American senator to be
elected from the southern United States since 1881 (four years after the
end of the Reconstruction Era), and the first African-American
Republican to serve in the U.S. Senate since Edward Brooke departed in
1979. He was the first Republican African-American U.S. Representative
from South Carolina since 1897.

\section{Early life, education, and business
career}\label{early-life-education-and-business-career}

\begin{itemize}
\item
  \emph{Scott's younger brother is a U.S. Air Force colonel.}
\item
  \emph{Scott was born in North Charleston, South Carolina, a son of
  Frances, a nursing assistant, and Ben Scott, Sr. His parents were
  divorced when he was 7.}
\item
  \emph{In addition to his political career, Scott owns an insurance
  agency, Tim Scott Allstate, and worked as a financial adviser.}
\end{itemize}

Scott was born in North Charleston, South Carolina, a son of Frances, a
nursing assistant, and Ben Scott, Sr. His parents were divorced when he
was 7. He grew up in working class poverty with his mother working
16-hour days to support her family, including Tim's brothers. His older
brother is a sergeant major in the U.S. Army. Scott's younger brother is
a U.S. Air Force colonel.

Scott attended Presbyterian College from 1983 to 1984, on a partial
football scholarship. He graduated from Charleston Southern University
in 1988 with a B.S. in Political Science. Scott is also an alumnus of
South Carolina's Palmetto Boys State program, an experience which he
cites as an influential factor in his decision to enter public service.

In addition to his political career, Scott owns an insurance agency, Tim
Scott Allstate, and worked as a financial adviser.

\section{Charleston County Council
(1995--2008)}\label{charleston-county-council-19952008}

\section{Elections}\label{elections}

\begin{itemize}
\item
  \emph{In 1996, Scott challenged Democratic State Senator Robert Ford
  in South Carolina's 42nd Senate district, but lost 65\%--35\%.}
\item
  \emph{Scott was on the County Council for a time alongside Paul
  Thurmond, the son of the late Republican U.S.}
\item
  \emph{Scott won the seat as a Republican, receiving nearly 80\% of the
  vote in the white-majority district, which since the late 20th century
  has voted Republican.}
\end{itemize}

Scott ran in a February 1995 special election to the Charleston County
Council at-large seat vacated by Keith Summey, who resigned his seat
after being elected as Mayor of North Charleston. Scott won the seat as
a Republican, receiving nearly 80\% of the vote in the white-majority
district, which since the late 20th century has voted Republican. He
became the first black Republican elected to any office in South
Carolina since the late 19th century.

Scott was on the County Council for a time alongside Paul Thurmond, the
son of the late Republican U.S. Senator, Strom Thurmond, who switched
from the Democratic Party to the Republican Party in 1964.

In 1996, Scott challenged Democratic State Senator Robert Ford in South
Carolina's 42nd Senate district, but lost 65\%--35\%.

Scott won reelection to the County Council in 2000, again winning in
white-majority districts. In 2004, he was reelected again with 61\% of
the vote, defeating Democrat Elliot Summey (son of Mayor Keith Summey).

\section{Tenure}\label{tenure}

\begin{itemize}
\item
  \emph{Scott was on the council from 1995 until 2008, becoming chairman
  in 2007.}
\item
  \emph{The county council unanimously approved the display, and Scott
  nailed a King James version of the Commandments to the wall.}
\item
  \emph{Regarding the costs of the suit, Scott said, "Whatever it costs
  in the pursuit of this goal is worth it."}
\end{itemize}

Scott was on the council from 1995 until 2008, becoming chairman in
2007. In 1997, he supported posting the Ten Commandments outside the
council chambers, saying it would remind members of the absolute rules
they should follow. The county council unanimously approved the display,
and Scott nailed a King James version of the Commandments to the wall.
Shortly thereafter, the ACLU and Americans United for Separation of
Church and State challenged this in a federal suit. After an initial
court ruling that the display was unconstitutional, the council settled
out of court to avoid accruing more legal fees. Regarding the costs of
the suit, Scott said, "Whatever it costs in the pursuit of this goal is
worth it."

In January 2001, the U.S. Department of Justice sued Charleston County,
South Carolina for racial discrimination under the Voting Rights Act,
based on its having all its council seats elected by at-large districts.
DOJ had attempted to negotiate with county officials on this issue in
November 2000. Justice officials noted that at-large seats dilute the
voting strength of the significant African-American minority in the
county, who in 2000 made up 34.5\% of the population. They have been
unable to elect any "candidates of their choice" for years. Whites or
European Americans are 61.9 percent of the county population. Since the
late 20th century, the majority-white voters have elected Republican
candidates. County officials noted that the majority of voters in 1989
had approved electing members by at-large seats in a popular referendum.

Scott, the only African-American member of the county council, has said
about this case and the alternative of electing council members from
single-member districts,

The Department of Justice alleged that the voting preference issue was
not a question of ethnicity, stating that voters in black precincts in
the county had rejected Scott as a candidate for the council. The
lawsuit noted that because of the white majority, "white bloc voting
usually results in the defeat of candidates who are preferred by black
voters." The Department added that blacks live in compact areas of the
county, and could be a majority in three districts if the county seats
were apportioned as nine single-member districts.

Committee assignments

Economic Development Committee (Chair)

\section{South Carolina House of Representatives
(2009--2011)}\label{south-carolina-house-of-representatives-20092011}

\section{Elections}\label{elections-1}

\begin{itemize}
\item
  \emph{With support from advisors such as Nicolas Muzin, Scott decided
  to run for his seat in District 117 of the South Carolina House of
  Representatives and won the Republican primary with 53\% of the vote,
  defeating Bill Crosby and Wheeler Tillman.}
\item
  \emph{He won the general election unopposed, becoming the first
  Republican African American U.S. Representative from South Carolina in
  more than 100 years.}
\end{itemize}

In 2008, incumbent Republican State Representative Tom Dantzler decided
to retire. With support from advisors such as Nicolas Muzin, Scott
decided to run for his seat in District 117 of the South Carolina House
of Representatives and won the Republican primary with 53\% of the vote,
defeating Bill Crosby and Wheeler Tillman. He won the general election
unopposed, becoming the first Republican African American U.S.
Representative from South Carolina in more than 100 years.

\section{Tenure}\label{tenure-1}

\begin{itemize}
\item
  \emph{Scott supported South Carolina's right-to-work laws and argued
  that Boeing chose South Carolina as a site for manufacturing for that
  reason.}
\item
  \emph{In South Carolina Club for Growth's 2009--10 scorecard, Scott
  earned a B and a score of 80 out of 100.}
\end{itemize}

Scott supported South Carolina's right-to-work laws and argued that
Boeing chose South Carolina as a site for manufacturing for that reason.

In South Carolina Club for Growth's 2009--10 scorecard, Scott earned a B
and a score of 80 out of 100. He was praised by the South Carolina
Association of Taxpayers for his "diligent, principled and courageous
stands against higher taxes."

Committee assignments

Judiciary

Labor, Commerce and Industry

Ways and Means

\section{United States House of Representatives
(2011--2013)}\label{united-states-house-of-representatives-20112013}

\section{Elections}\label{elections-2}

\begin{itemize}
\item
  \emph{Scott finished first in the nine-candidate Republican primary of
  June 8, 2010, receiving a plurality of 32\% of the vote.}
\item
  \emph{Scott entered the election for lieutenant governor but switched
  to run for South Carolina's 1st congressional district following the
  retirement announcement of Republican incumbent Henry Brown.}
\item
  \emph{Scott also became the first African-American Republican elected
  to Congress from South Carolina in 114 years.}
\end{itemize}

2010

Scott entered the election for lieutenant governor but switched to run
for South Carolina's 1st congressional district following the retirement
announcement of Republican incumbent Henry Brown. The 1st district is
based in Charleston, and includes approximately the northern 3/4 of the
state's coastline (except for Beaufort and Hilton Head Island, which
have been included in the 2nd District since redistricting).

Scott finished first in the nine-candidate Republican primary of June 8,
2010, receiving a plurality of 32\% of the vote. Fellow Charleston
County Councilman Paul Thurmond, son of U.S. Senator Strom Thurmond, was
second with 16\%. Carroll A. Campbell III, the son of former Governor
Carroll A. Campbell, Jr., was third with 14\%. Charleston County School
Board member Larry Kobrovsky ranked fourth with 11\%. Five other
candidates had single-digit percentages.

Because no candidate had received 50\% or more of the vote, a runoff was
held on June 22 between Scott and Thurmond. Scott was endorsed by the
anti-tax Club for Growth, various Tea Party movement groups, former
Alaska Governor and Vice Presidential nominee Sarah Palin, Republican
House Whip Eric Cantor, former Arkansas governor Mike Huckabee, South
Carolina Senator Jim DeMint, and the founder of the Minuteman Project.
He defeated Thurmond 68\%--32\% and won every county in the
congressional district.

According to the Associated Press, Scott "swamped his opponents in
fundraising, spending almost \$725,000 during the election cycle to less
than \$20,000 for his November opponents". He won the general election
against Democrat Ben Frasier 65\%--29\%. With this election, Scott and
Allen West of Florida became the first African-American Republicans in
Congress since J. C. Watts retired in 2003. Scott also became the first
African-American Republican elected to Congress from South Carolina in
114 years. From 1895 to after 1965, most African-Americans had been
disenfranchised in the state, and they had comprised most of the
Republican Party when they were excluded from the political system.

2012

Scott was unopposed in the primary and won the general election against
Democrat Bobbie Rose, 62\%--36\%.

\includegraphics[width=4.40000in,height=5.50000in]{media/image1.jpg}\\
\emph{Scott's official 112th Congress portrait}

\section{Tenure}\label{tenure-2}

\begin{itemize}
\item
  \emph{The House Republican Steering Committee appointed Scott to the
  Committee on Transportation and the Committee on Small Business.}
\item
  \emph{Social issues~-- Scott describes himself as pro-life.}
\item
  \emph{Scott opposed the 2011 military intervention in Libya.}
\item
  \emph{Police body cameras~-- After the shooting of Walter Scott (no
  relation), Scott urged the Senate to hold hearings on police body
  cameras.}
\item
  \emph{Scott declined to join the Congressional Black Caucus.}
\end{itemize}

Scott declined to join the Congressional Black Caucus.

In March 2011, Scott co-sponsored a welfare reform bill that would deny
food stamps to families whose incomes were lowered to the point of
eligibility because a family member was participating in a labor strike.
He introduced legislation in July 2011 to strip the National Labor
Relations Board (NLRB) of its power to prohibit employers from
relocating to punish workers who join unions or strike. The rationale
for the legislation is that government agencies should not be able to
tell private employers where they can run a business. Scott described
the legislation as a commonsense proposal that would fix a flaw in
federal labor policy and benefit the national and local economies. The
NLRB had recently opposed the relocation of a Boeing production facility
from Washington state to South Carolina.

Scott successfully advocated for federal funds for a Charleston harbor
dredging project estimated at \$300 million, arguing that the project
was neither an earmark nor an example of wasteful government spending.
He said the project was merit-based and in the national interest because
larger cargo ships could use the port and jobs would be created.

During the summer 2011 debate over raising the U.S. debt ceiling, Scott
supported the inclusion of a balanced-budget Constitutional amendment in
the debt ceiling bill, and opposed legislation that did not include the
amendment. Before voting "no" on the final bill to raise the debt
ceiling, Scott and other first-term conservatives prayed for guidance in
a congressional chapel. Afterward, he said he had received divine
inspiration regarding his vote, and joined the rest of the South
Carolina congressional delegation in voting "no" on the measure.

Taxes and spending~-- Scott believes that federal spending and taxes
should be reduced, with a Balanced Budget Amendment and the FairTax
respectively implemented for spending and taxes.

Health care~-- Scott believes the Affordable Care Act should be
repealed. He has said that health care in the U.S. is among the greatest
in the world, that people all over the world come to study in American
medical schools, waiting lists are rare, and Americans are able to
choose their insurance, providers, and course of treatment. Scott
supports an alternative to the ACA that he says keeps its benefits while
controlling costs by reforming the medical tort system by having a limit
on non-economic damages and by reforming Medicare. In January 2014 he
signed an amicus brief in support of Senator Ron Johnson's legal
challenge against the U.S. Office of Personnel Management's Affordable
Care Act ruling.

Earmarks~-- Scott opposes earmarks, and yet he successfully advocated
for federal funds for a Charleston harbor dredging project estimated at
\$300 million.

Economic development~-- He supports infrastructure development and
public works for his district. He opposes restrictions on deepwater oil
drilling. He proposed the opportunity zone designation in the Tax Cuts
and Jobs Act of 2017.

Social issues~-- Scott describes himself as pro-life. He supports adult
and cord blood stem cell research but opposes taxpayer-funded embryonic
stem cell research and the creation of human embryos for
experimentation. He opposes assisted suicide and same-sex marriage.

Immigration~-- Scott supports federal legislation similar to the Arizona
law, Arizona SB 1070. He supports strengthening penalties for employers
who knowingly hire illegal immigrants. He also promotes cultural
assimilation by making English the official language in the government
and requiring new immigrants to learn English. He opposes a pathway to
citizenship for illegal aliens.

Labor~-- Scott introduced a bill that would deny food stamps to families
whose incomes were lowered to the point of eligibility because a family
member was participating in a labor strike.

Foreign policy~-- Scott advocates a continued military presence in
Afghanistan and believes early withdrawal would benefit Al-Qaeda. He
also views Iran as the world's most dangerous country and believes that
the US should aid pro-democracy groups there. Scott opposed the 2011
military intervention in Libya.

Police body cameras~-- After the shooting of Walter Scott (no relation),
Scott urged the Senate to hold hearings on police body cameras.

Committee assignments

The House Republican Steering Committee appointed Scott to the Committee
on Transportation and the Committee on Small Business. He was later
appointed to the Committee on Rules and relinquished his other two
committee assignments.

Committee on Rules\\
Subcommittee on Rules and the Organization of the House

\section{U.S. Senate}\label{u.s.-senate}

\section{2012 appointment}\label{appointment}

\begin{itemize}
\item
  \emph{Scott is the first African American to be a U.S.}
\item
  \emph{Senator from South Carolina.}
\item
  \emph{Of her decision to pick Scott, Haley said: "It is important to
  me, as a minority female, that Congressman Scott earned this seat, he
  earned this seat for the person that he is.}
\item
  \emph{News media reported that Scott, along with Rep. Trey Gowdy,
  former South Carolina Attorney General Henry McMaster, former First
  Lady of South Carolina Jenny Sanford, and Catherine Templeton,
  Director of the South Carolina Department of Health and Environmental
  Control, were on Haley's short list to replace DeMint.}
\end{itemize}

On December 17, 2012, South Carolina governor Nikki Haley announced she
would appoint Scott to replace retiring Senator Jim DeMint, who had
previously announced that he would retire from the Senate to become the
President of The Heritage Foundation. Scott is the first African
American to be a U.S. Senator from South Carolina. He was one of three
black U.S. Senators in the 113th Congress, alongside Mo Cowan and later
Cory Booker (and the first since Roland Burris retired in 2010 after
succeeding Barack Obama). He is the first African American to be a U.S.
Senator from the Southern United States since Reconstruction. From 1890
to 1908 Democratic-controlled state legislatures passed new
constitutions and laws that disfranchised most blacks and many poor
whites across the South, securing power for white politicians from in
the Democratic Party.

During two periods, first from January 2, 2013 until February 1, 2013,
and again from July 16, 2013 until October 31, 2013, Scott was the only
African-American Senator. He and Cowan were the first black senators to
serve alongside each other.

News media reported that Scott, along with Rep. Trey Gowdy, former South
Carolina Attorney General Henry McMaster, former First Lady of South
Carolina Jenny Sanford, and Catherine Templeton, Director of the South
Carolina Department of Health and Environmental Control, were on Haley's
short list to replace DeMint. Of her decision to pick Scott, Haley said:
"It is important to me, as a minority female, that Congressman Scott
earned this seat, he earned this seat for the person that he is. He
earned this seat with the results he has shown."

\section{2014 election}\label{election}

\begin{itemize}
\item
  \emph{Scott ran in November 2014 to serve the final two years of
  DeMint's term and won.}
\end{itemize}

Scott ran in November 2014 to serve the final two years of DeMint's term
and won.

\section{2016 election}\label{election-1}

\begin{itemize}
\item
  \emph{Scott won reelection to a first full term in office in November
  2016.}
\item
  \emph{In July 2018, Scott introduced a bipartisan bill, along with
  Democratic Senators Cory Booker and Kamala Harris, to make lynching a
  federal hate crime.}
\item
  \emph{In February 2019, Scott was one of sixteen senators to vote
  against legislation preventing a partial government shutdown and
  containing 1.375 billion for barriers along the U.S.-Mexico border
  that included 55 miles of fencing.}
\end{itemize}

Scott won reelection to a first full term in office in November 2016. He
was endorsed by the Club for Growth.

In July 2018, Scott introduced a bipartisan bill, along with Democratic
Senators Cory Booker and Kamala Harris, to make lynching a federal hate
crime.

In February 2019, Scott was one of sixteen senators to vote against
legislation preventing a partial government shutdown and containing
1.375 billion for barriers along the U.S.-Mexico border that included 55
miles of fencing.

\section{Positions}\label{positions}

\section{Environment}\label{environment}

\begin{itemize}
\item
  \emph{In 2017, Scott was one of 22 senators to sign a letter to
  President Donald Trump urging him to have the United States withdraw
  from the Paris Agreement.}
\item
  \emph{According to the Center for Responsive Politics, Scott has
  received over \$540,000 from oil, gas and coal interests since 2012.}
\end{itemize}

In 2017, Scott was one of 22 senators to sign a letter to President
Donald Trump urging him to have the United States withdraw from the
Paris Agreement. According to the Center for Responsive Politics, Scott
has received over \$540,000 from oil, gas and coal interests since 2012.

\section{Judicial nominations}\label{judicial-nominations}

\begin{itemize}
\item
  \emph{In a letter to the Wall Street Journal Scott said the
  publication was trying to ``deflect concerns'' about Farr's
  nomination.}
\item
  \emph{Further explaining his vote, Scott said the Republican Party was
  "not doing a very good job of avoiding the obvious potholes on race in
  America."}
\item
  \emph{Scott did not support Trump's nominee, Oregon's Ryan Bounds, to
  the 9th U.S.}
\end{itemize}

Scott did not support Trump's nominee, Oregon's Ryan Bounds, to the 9th
U.S. Circuit Court of Appeals, effectively "derailing" the nomination.
His decision was based on what he called Bounds's "bigoted statements he
made as a Stanford student in the 1990s." Marco Rubio joined him in
opposing the nomination shortly thereafter, prompting Mitch McConnell to
withdraw the nomination altogether.

In November 2018, Scott bucked his party in opposing Trump's nomination
of Thomas A. Farr for a federal judgeship. Farr had been accused of
voter suppression toward African-American voters. Scott cited Farr's
involvement in the 1984 and 1990 Senate campaigns of Jesse Helms, which
sought to suppress black voters, and a 1991 memo from the Department of
Justice under the George H. W. Bush administration that stated that
"Farr was the primary coordinator of the 1984 `ballot security' program
conducted by the NCGOP and 1984 Helms for Senate Committee. He
coordinated several `ballot security' activities in 1984, including a
postcard mailing to voters in predominantly black precincts which was
designed to serve as a basis to challenge voters on election day."
Further explaining his vote, Scott said the Republican Party was "not
doing a very good job of avoiding the obvious potholes on race in
America." In an editorial, the Wall Street Journal criticized Scott,
arguing that Democrats would see Farr's defeat as a "vindication of
their most underhanded and inflammatory racial tactics." In a letter to
the Wall Street Journal Scott said the publication was trying to
``deflect concerns'' about Farr's nomination.

\section{Healthcare}\label{healthcare}

\begin{itemize}
\item
  \emph{In January 2019, Scott was one of six senators to cosponsor the
  Health Insurance Tax Relief Act, delaying the Health Insurance Tax for
  two years.}
\end{itemize}

In January 2019, Scott was one of six senators to cosponsor the Health
Insurance Tax Relief Act, delaying the Health Insurance Tax for two
years.

\section{Trade}\label{trade}

\begin{itemize}
\item
  \emph{In January 2018, Scott was one of 36 Republican senators to sign
  a letter to Trump requesting he preserve the North American Free Trade
  Agreement by modernizing it for the economy of the 21st century.}
\end{itemize}

In January 2018, Scott was one of 36 Republican senators to sign a
letter to Trump requesting he preserve the North American Free Trade
Agreement by modernizing it for the economy of the 21st century.

Committee assignments

Committee on Armed Services\\
Subcommittee on Emerging Threats and Capabilities\\
Subcommittee on SeaPower

Committee on Banking, Housing, and Urban Affairs\\
Subcommittee on Financial Institutions and Consumer Protection\\
Subcommittee on Housing, Transportation, and Community Development
(Chairman)\\
Subcommittee on Securities, Insurance, and Investment

Committee on Finance\\
Subcommittee on Energy, Natural Resources, and Infrastructure\\
Subcommittee on Fiscal Responsibility and Economic Growth (Chairman)\\
Subcommittee on Taxation and IRS Oversight

Committee on Health, Education, Labor, and Pensions\\
Subcommittee on Employment and Workplace Safety\\
Subcommittee on Primary Health and Retirement Security

Committee on Small Business and Entrepreneurship

Special Committee on Aging

\section{Personal life}\label{personal-life}

\begin{itemize}
\item
  \emph{Republican leadership has praised Scott's background as an
  example of achieving the American dream according to a conservative
  model.}
\item
  \emph{Scott is unmarried.}
\item
  \emph{Scott is an evangelical Protestant.}
\end{itemize}

Scott is unmarried. He owns an insurance agency and is a partner in
Pathway Real Estate Group, LLC. Scott is an evangelical Protestant. He
is a member of Seacoast Church, a large evangelical church in
Charleston, and a former member of that church's board. Republican
leadership has praised Scott's background as an example of achieving the
American dream according to a conservative model.

\section{Electoral history}\label{electoral-history}

\section{See also}\label{see-also}

\begin{itemize}
\item
  \emph{List of African-American United States Senators}
\item
  \emph{List of African-American United States Representatives}
\item
  \emph{Black conservatism in the United States}
\end{itemize}

Black conservatism in the United States

List of African-American Republicans

List of African-American United States Representatives

List of African-American United States Senators

\section{References}\label{references}

\section{External links}\label{external-links}

\begin{itemize}
\item
  \emph{Tim Scott for Senate}
\item
  \emph{Tim Scott at Curlie}
\item
  \emph{Senator Tim Scott official U.S. Senate site}
\end{itemize}

Senator Tim Scott official U.S. Senate site

Tim Scott for Senate

Appearances on C-SPAN

Tim Scott at Curlie

Biography at the Biographical Directory of the United States Congress

Profile at Vote Smart

Financial information (federal office) at the Federal Election
Commission

Legislation sponsored at the Library of Congress
