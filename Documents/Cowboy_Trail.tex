\textbf{From Wikipedia, the free encyclopedia}

https://en.wikipedia.org/wiki/Cowboy\_Trail\\
Licensed under CC BY-SA 3.0:\\
https://en.wikipedia.org/wiki/Wikipedia:Text\_of\_Creative\_Commons\_Attribution-ShareAlike\_3.0\_Unported\_License

\section{Cowboy Trail}\label{cowboy-trail}

\begin{itemize}
\item
  \emph{It is Nebraska's first state recreational trail.}
\item
  \emph{The Cowboy Trail is a rail trail in northern Nebraska.}
\item
  \emph{The trail runs across the Outback area of Nebraska.}
\end{itemize}

The Cowboy Trail is a rail trail in northern Nebraska. It is a multi-use
recreational trail suitable for bicycling, walking and horseback riding.
It occupies an abandoned Chicago and North Western Railway corridor.
When complete, the trail will run from Chadron to Norfolk, a length of
321 miles (517~km), making it the longest rails-to-trails conversion in
the United States. It is Nebraska's first state recreational trail. The
trail runs across the Outback area of Nebraska.

\section{History}\label{history}

\begin{itemize}
\item
  \emph{The Nebraska Game and Parks Commission is responsible for the
  development and maintenance of the trail.}
\item
  \emph{The Cowboy Trail in that section was to be built on an easement
  parallel to the railroad.}
\item
  \emph{In view of the abandonment of the final section, details of
  where the last section of the Cowboy Trail will be built are still
  being worked out.}
\end{itemize}

Built by the Fremont, Elkhorn and Missouri Valley Railroad (a
predecessor company of the Chicago \& North Western Railway) in the late
1870s and early 1880s, the "Cowboy Line" was abandoned by the C\&NW west
of Norfolk in 1992; the section east of Norfolk was abandoned in 1982.
Only a small section from the 1982 abandonment was saved from Fremont to
Hooper. The following year (1993), the Rails-to-Trails Conservancy
purchased the railroad's right-of-way for \$6.2 million and donated it
to the state of Nebraska. The Nebraska Game and Parks Commission is
responsible for the development and maintenance of the trail.

Development of the trail has occurred at a rate of about 10 to 20 miles
(15 to 30~km) each year. In the summer of 2009, the final segment
between Valentine and Norfolk was completed, producing a continuous
segment of 195 miles (314~km).

A short-line railroad (the Nebkota Railway) did operate on the
westernmost 74 miles (119~km) of the Cowboy Trail (from Chadron to
Merriman) until 2007. The Cowboy Trail in that section was to be built
on an easement parallel to the railroad. In view of the abandonment of
the final section, details of where the last section of the Cowboy Trail
will be built are still being worked out.

\section{Trail guide}\label{trail-guide}

\begin{itemize}
\item
  \emph{A variety of landscapes are found along the trail: the Pine
  Ridge, the Sandhills, and the valleys of the Niobrara River, Long Pine
  Creek and the Elkhorn River.}
\item
  \emph{Completed sections of the trail are crushed limestone.}
\item
  \emph{Major cities on the trail include (from west to east):}
\item
  \emph{There are 29 communities along the length of the Cowboy Trail.}
\end{itemize}

There are 29 communities along the length of the Cowboy Trail. Major
cities on the trail include (from west to east):

Chadron

Gordon

Valentine

Ainsworth

O'Neill

Neligh

Norfolk

Trailheads are located in Valentine and Norfolk. Completed sections of
the trail are crushed limestone. There are 221 bridges on the trail; all
bridges have been converted for recreational use. The bridge across the
Niobrara River east of Valentine is a quarter-mile long (400~m) and 148
feet (45~m) high; the bridge across Long Pine Creek at Long Pine is 595
feet (181~m) long and 145 feet (44~m) high.

The trail parallels US 20 and US 275 for almost its entire length. A
variety of landscapes are found along the trail: the Pine Ridge, the
Sandhills, and the valleys of the Niobrara River, Long Pine Creek and
the Elkhorn River.

\section{References}\label{references}

\section{External links}\label{external-links}

\begin{itemize}
\item
  \emph{Nebraska's Cowboy Trail: A User's Guide}
\item
  \emph{American Trails -- An overview of Nebraska's Cowboy Trail}
\item
  \emph{Nebraska Game and Parks Commission - Cowboy Trail}
\item
  \emph{Nebraska's Cowboy Trail: maps, business listings, and trip
  planning resources}
\end{itemize}

Nebraska's Cowboy Trail: maps, business listings, and trip planning
resources

Nebraska Game and Parks Commission - Cowboy Trail

American Trails -- An overview of Nebraska's Cowboy Trail

Nebraska's Cowboy Trail: A User's Guide

Coordinates: 42°00′13.3″N 97°25′33.9″W / 42.003694°N 97.426083°W /
42.003694; -97.426083
