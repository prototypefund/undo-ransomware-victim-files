\textbf{From Wikipedia, the free encyclopedia}

https://en.wikipedia.org/wiki/King\%20assassination\%20riots\\
Licensed under CC BY-SA 3.0:\\
https://en.wikipedia.org/wiki/Wikipedia:Text\_of\_Creative\_Commons\_Attribution-ShareAlike\_3.0\_Unported\_License

\section{King assassination riots}\label{king-assassination-riots}

\begin{itemize}
\item
  \emph{Some of the biggest riots took place in Washington, D.C.,
  Baltimore, Chicago, and Kansas City.}
\item
  \emph{The King assassination riots, also known as the Holy Week
  Uprising, was a wave of civil disturbance which swept the United
  States following the assassination of Martin Luther King Jr. on April
  4, 1968.}
\end{itemize}

The King assassination riots, also known as the Holy Week Uprising, was
a wave of civil disturbance which swept the United States following the
assassination of Martin Luther King Jr. on April 4, 1968. It was the
greatest wave of social unrest the United States had experienced since
the Civil War. Some of the biggest riots took place in Washington, D.C.,
Baltimore, Chicago, and Kansas City.

\section{Inciting incident and riot}\label{inciting-incident-and-riot}

\begin{itemize}
\item
  \emph{The immediate cause of the rioting was the assassination of
  Martin Luther King.}
\item
  \emph{King was not only a leader in the civil rights movement, but
  also an advocate for nonviolence.}
\end{itemize}

The immediate cause of the rioting was the assassination of Martin
Luther King. King was not only a leader in the civil rights movement,
but also an advocate for nonviolence. He pursued direct engagement with
the political system (as opposed to the separatist ideas of black
nationalism). His death led to anger and disillusionment, and feelings
that now only violent resistance to white racism could be effective.

The rioters were mostly black; not all were poor. Middle-class blacks
also demonstrated against systemic inequality. Although the media called
these events "race riots", there were few confirmed acts of violence
between blacks and whites. White businesses tended to be targeted;
however, white public and community buildings such as schools and
churches were largely spared.

Compared to the previous summer of rioting, the number of fatalities was
lower, largely attributed to new procedures instituted by the federal
government, and orders not to fire on looters.

\section{City by city}\label{city-by-city}

\begin{itemize}
\item
  \emph{He is credited for averting major riots in New York with this
  direct response although minor disturbances still erupted in the
  city.}
\item
  \emph{In Indianapolis, Robert F. Kennedy's speech on the assassination
  of Martin Luther King Jr. is credited with preventing a riot there.}
\item
  \emph{In New York City, mayor John Lindsay traveled directly into
  Harlem, telling black residents that he regretted King's death and was
  working against poverty.}
\end{itemize}

In New York City, mayor John Lindsay traveled directly into Harlem,
telling black residents that he regretted King's death and was working
against poverty. He is credited for averting major riots in New York
with this direct response although minor disturbances still erupted in
the city.\\
In Indianapolis, Robert F. Kennedy's speech on the assassination of
Martin Luther King Jr. is credited with preventing a riot there.\\
In Boston, rioting may have been averted by a James Brown concert taking
place on the night of April 5, with Brown, Mayor Kevin White, and City
Councilor Tom Atkins speaking to the Garden crowd about peace and unity
before the show.

In Los Angeles, the Los Angeles Police Department and community
activists averted a repeat of the 1965 riots that devastated portions of
the city. Several memorials were held in tribute to King throughout the
Los Angeles area on the days leading into his funeral
service.{[}citation needed{]}

\includegraphics[width=5.50000in,height=3.66667in]{media/image1.jpg}\\
\emph{Damage to a Washington store following the riots}

\section{Washington, D.C.}\label{washington-d.c.}

\begin{itemize}
\item
  \emph{The riots utterly devastated Washington's inner city economy.}
\item
  \emph{The Washington, D.C., riots of April 4--8, 1968, resulted in
  Washington, along with Chicago and Baltimore, receiving the heaviest
  impact of the 110 cities to see unrest following the King
  assassination.}
\item
  \emph{The occupation of Washington was the largest of any American
  city since the Civil War.}
\end{itemize}

The Washington, D.C., riots of April 4--8, 1968, resulted in Washington,
along with Chicago and Baltimore, receiving the heaviest impact of the
110 cities to see unrest following the King assassination.

The ready availability of jobs in the growing federal government
attracted many to Washington in the 1960s, and middle class
African-American neighborhoods prospered. Despite the end of legally
mandated racial segregation, the historic neighborhoods of Shaw, the H
Street Northeast corridor, and Columbia Heights, centered at the
intersection of 14th and U Streets Northwest, remained the centers of
African-American commercial life in the city.

As word of King's murder by James Earl Ray in Memphis spread on the
evening of Thursday, April 4, crowds began to gather at 14th and U.
Stokely Carmichael led members of the Student Nonviolent Coordinating
Committee (SNCC) to stores in the neighborhood demanding that they close
out of respect. Although polite at first, the crowd fell out of control
and began breaking windows. By 11pm, widespread looting had begun.

Mayor-Commissioner Walter Washington ordered the damage cleaned up
immediately the next morning. However, anger was still evident on Friday
morning when Carmichael addressed a rally at Howard, warning of
violence. After the close of the rally, crowds walking down 7th Street
NW and in the H Street NE corridor came into violent confrontations with
police. By midday, numerous buildings were on fire, and firefighters
were prevented from responding by crowds attacking with bottles and
rocks.

Crowds of as many as 20,000 overwhelmed the District's 3,100-member
police force, and President Lyndon B. Johnson dispatched some 13,600
federal troops, including 1,750 federalized D.C. National Guard troops,
to assist them. Marines mounted machine guns on the steps of the Capitol
and Army troops from the 3rd Infantry guarded the White House. At one
point, on April 5, rioting reached within two blocks of the White House
before rioters retreated. The occupation of Washington was the largest
of any American city since the Civil War. Mayor Washington imposed a
curfew and banned the sale of alcohol and guns in the city. By the time
the city was considered pacified on Sunday, April 8, some 1,200
buildings had been burned, including over 900 stores. Damages reached
\$27 million.

The riots utterly devastated Washington's inner city economy. With the
destruction or closing of businesses, thousands of jobs were lost, and
insurance rates soared. Made uneasy by the violence, city residents of
all races accelerated their departure for suburban areas, depressing
property values. Crime in the burned out neighborhoods rose sharply,
further discouraging investment.

On some blocks, only rubble remained for decades. Columbia Heights and
the U Street corridor did not begin to recover economically until the
opening of the U Street and Columbia Heights Metro stations in 1991 and
1999, respectively, while the H Street NE corridor remained depressed
for several years longer.

Mayor-Commissioner Washington, who was the last presidentially appointed
mayor of Washington, went on to become the city's first elected mayor.

\section{Chicago}\label{chicago}

\begin{itemize}
\item
  \emph{On April 5, 1968, in Chicago, violence sparked on the West side
  of the city, and gradually expanded to consume a 28-block stretch of
  West Madison Street, with additional damage occurring on Roosevelt
  Road.}
\item
  \emph{Approximately 10,500 police were sent in, and by April 6, more
  than 6,700 Illinois National Guard troops had arrived in Chicago with
  5,000 regular Army soldiers from the 1st Armored and 5th Infantry
  Divisions being ordered into the city by President Johnson.}
\end{itemize}

On April 5, 1968, in Chicago, violence sparked on the West side of the
city, and gradually expanded to consume a 28-block stretch of West
Madison Street, with additional damage occurring on Roosevelt Road. The
North Lawndale and East Garfield Park neighborhoods on the West Side and
the Woodlawn neighborhood on the South Side experienced the majority of
the destruction and chaos. The rioters broke windows, looted stores, and
set buildings (both abandoned and occupied) on fire. Firefighters
quickly flooded the neighborhood, and Chicago's off-duty firefighters
were told to report for duty. There were 36 major fires reported between
4:00 pm and 10:00 pm alone. The next day, Mayor Richard J. Daley imposed
a curfew on anyone under the age of 21, closed the streets to automobile
traffic, and halted the sale of guns or ammunition.

Approximately 10,500 police were sent in, and by April 6, more than
6,700 Illinois National Guard troops had arrived in Chicago with 5,000
regular Army soldiers from the 1st Armored and 5th Infantry Divisions
being ordered into the city by President Johnson. The General in charge
declared that no one was allowed to have gatherings in the riot areas,
and he authorized the use of tear gas. Mayor Richard J. Daley gave
police the authority "to shoot to kill any arsonist or anyone with a
Molotov cocktail in his hand ... and ... to shoot to maim or cripple
anyone looting any stores in our city."

By the time order was restored on April 7, 11 people had died, 500 had
been injured, and 2,150 had been arrested. Over 200 buildings were
damaged in the disturbance with damage costs running up to \$10 million.

The south side ghetto had escaped the major chaos mainly because the two
large street gangs, the Blackstone Rangers and the East Side Disciples,
cooperated to control their neighborhoods. Many gang members did not
participate in the rioting, due in part to King's direct involvement
with these groups in 1966.{[}citation needed{]}

\section{Baltimore}\label{baltimore}

\begin{itemize}
\item
  \emph{The Baltimore riot of 1968 began two days after the murder.}
\item
  \emph{The riot was precipitated by King's assassination, but was also
  evidence of larger frustrations among the city's African-American
  population.}
\end{itemize}

The Baltimore riot of 1968 began two days after the murder. On Saturday,
April 6, the Governor of Maryland, Spiro T. Agnew, called out thousands
of National Guard troops and 500 Maryland State Police to quell the
disturbance. When it was determined that the state forces could not
control the riot, Agnew requested Federal troops from President Lyndon
B. Johnson. The riot was precipitated by King's assassination, but was
also evidence of larger frustrations among the city's African-American
population.

By Sunday evening, 5,000 paratroopers, combat engineers, and
artillerymen from the XVIII Airborne Corps in Fort Bragg, North
Carolina, specially trained in tactics, including sniper school, were on
the streets of Baltimore with fixed bayonets, and equipped with chemical
(CS) disperser backpacks. Two days later, they were joined by a Light
Infantry Brigade from Fort Benning, Georgia. With all the police and
troops on the streets, the situation began to calm down. The Federal
Bureau of Investigation reported that H. Rap Brown was in Baltimore
driving a Ford Mustang with Broward County, Florida tags, and was
assembling large groups of angry protesters and agitating them to
escalate the rioting. In several instances, these disturbances were
rapidly quelled through the use of bayonets and chemical dispersers by
the XVIII Airborne units. That unit arrested more than 3,000 detainees,
who were turned over to the Baltimore Police. A general curfew was set
at 6 p.m. in the city limits and martial law was enforced. As rioting
continued, African American plainclothes police officers and community
leaders were sent to the worst areas to prevent further violence. By the
end of the unrest, 6 people had died, 700 were injured, and 5,800 had
been arrested; property damage was estimated at over \$12 million.

One of the major outcomes of the riot was the attention Governor Agnew
received when he criticized local black leaders for not doing enough to
help stop the disturbance. While this angered blacks and white liberals,
it caught the attention of Republican presidential candidate Richard
Nixon, who was looking for someone on his ticket who could counter
George Wallace's American Independent Party campaign. Agnew became
Nixon's vice presidential running mate in 1968.

\section{Kansas City}\label{kansas-city}

\begin{itemize}
\item
  \emph{The rioting in Kansas City did not erupt on April 4, like other
  cities of the United States affected directly by the assassination of
  King, but rather on April 9 after local events within the city.}
\item
  \emph{The riot was sparked when Kansas City Police Department deployed
  tear gas against student protesters when they staged their
  performances outside City Hall.}
\end{itemize}

The rioting in Kansas City did not erupt on April 4, like other cities
of the United States affected directly by the assassination of King, but
rather on April 9 after local events within the city. The riot was
sparked when Kansas City Police Department deployed tear gas against
student protesters when they staged their performances outside City
Hall.

The deployment of tear gas dispersed the protesters from the area, but
other citizens of the city began to riot as a result of the police
action on the student protesters. The resulting effects of the riot
resulted in the arrest of over 100 adults, and left five dead and at
least 20 admitted to hospitals.

\section{Detroit}\label{detroit}

\begin{itemize}
\item
  \emph{Michigan Governor George W. Romney ordered the National Guard
  into Detroit.}
\end{itemize}

Although not as large as other cities, violent disturbances did erupt in
Detroit. Michigan Governor George W. Romney ordered the National Guard
into Detroit. One person was killed, and gangs tossed objects at cars
and smashed storefront windows along 12th Street on the west side.

\section{New York City}\label{new-york-city}

\begin{itemize}
\item
  \emph{Rioting erupted in New York the night King was murdered with
  Harlem, the largest African-American neighborhood in Manhattan,
  erupting into sporadic violence and looting.}
\item
  \emph{Tensions simmered down after Mayor John Lindsay traveled into
  the heart of the area and stated that he regretted King's wrongful
  death which led to the calming of residents although numerous
  businesses were still looted and set afire in Harlem and Brooklyn.}
\end{itemize}

Rioting erupted in New York the night King was murdered with Harlem, the
largest African-American neighborhood in Manhattan, erupting into
sporadic violence and looting. Tensions simmered down after Mayor John
Lindsay traveled into the heart of the area and stated that he regretted
King's wrongful death which led to the calming of residents although
numerous businesses were still looted and set afire in Harlem and
Brooklyn.

\section{Pittsburgh}\label{pittsburgh}

\begin{itemize}
\item
  \emph{The riot peaked on April 7 in which one person was killed and
  3,600 National Guardsmen were deployed into the city.}
\item
  \emph{Disturbances erupted in Pittsburgh on April 5 and continued
  through April 11.}
\item
  \emph{The riot left many of the city's black commercial districts in
  shambles and the areas most impacted by the unrest were slow to
  recover in the following decades.}
\end{itemize}

Disturbances erupted in Pittsburgh on April 5 and continued through
April 11. The riot peaked on April 7 in which one person was killed and
3,600 National Guardsmen were deployed into the city. Over 100
businesses were either looted or burned in the Hill District, Homewood,
and North Side neighborhoods with various structures being set afire by
arsonists. The riot left many of the city's black commercial districts
in shambles and the areas most impacted by the unrest were slow to
recover in the following decades.

\section{Cincinnati}\label{cincinnati}

\begin{itemize}
\item
  \emph{The Cincinnati riots were in response to the assassination of
  Martin Luther King Jr. on April 4, 1968.}
\end{itemize}

The Cincinnati riots were in response to the assassination of Martin
Luther King Jr. on April 4, 1968. Tension in the Avondale neighborhood
had already been high due to a lack of job opportunities for
African-American men, and the assassination escalated that tension.\\
On April 8, around 1,500 blacks attended a memorial held at a local
recreation center. An officer of the Congress of Racial Equality blamed
white Americans for King's death and urged the crowd to retaliate. The
crowd was orderly when it left the memorial and spilled out into the
street. Nearby James Smith, a black man, attempted to protect a jewelry
store from a robbery with his own shotgun. During the struggle with the
robbers, also black, Smith accidentally shot and killed his wife.

Rioting started after a false rumor was spread in the crowd that Smith's
wife was actually killed by a white police officer. Rioters smashed
store windows and looted merchandise. More than 70 fires had been set,
several of them major. During the rioting eight young African Americans
dragged a white student, Noel Wright, and his wife from their car in
Mount Auburn. Wright was stabbed to death and his wife was beaten. The
next night, the city was put under curfew, and nearly 1,500 National
Guardsmen were brought in to subdue the violence. Several days after the
riot started, two people were dead, hundreds were arrested, and the city
had suffered \$3 million in property damage.

\section{Trenton, New Jersey}\label{trenton-new-jersey}

\begin{itemize}
\item
  \emph{The Trenton Riots of 1968 were a major civil disturbance that
  took place during the week following the assassination of civil rights
  leader Martin Luther King in Memphis on April 4.}
\item
  \emph{Race riots broke out nationwide following the murder of the
  civil rights activist.}
\end{itemize}

The Trenton Riots of 1968 were a major civil disturbance that took place
during the week following the assassination of civil rights leader
Martin Luther King in Memphis on April 4. Race riots broke out
nationwide following the murder of the civil rights activist. More than
200 Trenton businesses, mostly in Downtown, were ransacked and burned.
More than 300 people, most of them young black men, were arrested on
charges ranging from assault and arson to looting and violating the
mayor's emergency curfew. In addition to 16 injured policemen, 15
firefighters were treated at city hospitals for smoke inhalation, burns,
sprains and cuts suffered while fighting raging blazes or for injuries
inflicted by rioters. Denizens of Trenton's urban core often pulled
false alarms and would then throw bricks at firefighters responding to
the alarm boxes. This experience, along with similar experiences in
other major cities, effectively ended the use of open-cab fire
engines.{[}citation needed{]} As an interim measure, the Trenton Fire
Department fabricated temporary cab enclosures from steel deck plating
until new equipment could be obtained. The losses incurred by downtown
businesses were initially estimated by the city to be \$7 million, but
the total of insurance claims and settlements came to \$2.5 million.

Trenton's Battle Monument neighborhood was hardest hit. Since the 1950s,
North Trenton had witnessed a steady exodus of middle-class residents,
and the riots spelled the end for North Trenton. By the 1970s, the
region had become one of the most blighted and crime-ridden in the city,
although gentrification in the area is revitalizing certain
sections.{[}citation needed{]}

\section{Wilmington, Delaware}\label{wilmington-delaware}

\begin{itemize}
\item
  \emph{During the riot, which occurred on April 9--10, 1968, the mayor
  asked for a small number of National Guardsmen to help restore order.}
\item
  \emph{The two-day riot that occurred after King's assassination was
  small compared with riots in other cities, but its aftermath -- a
  ​9~1⁄2-month occupation by the National Guard -- highlighted the depth
  of Wilmington's racial problem.}
\end{itemize}

The two-day riot that occurred after King's assassination was small
compared with riots in other cities, but its aftermath -- a ​9~1⁄2-month
occupation by the National Guard -- highlighted the depth of
Wilmington's racial problem. During the riot, which occurred on April
9--10, 1968, the mayor asked for a small number of National Guardsmen to
help restore order. Democratic Governor Charles L. Terry (a
southern-style Democrat) sent in the entire state National Guard and
refused to remove them after the rioting was brought under control.
Republican Russell W. Peterson defeated Governor Terry, and upon his
inauguration in January 1969, Governor Peterson ended the National
Guard's occupation in Wilmington.

The Occupation of Wilmington caused scars on the city and its people
that have lasted to this day. Some suburbanites grew fearful of
traveling into Wilmington in broad daylight, even to attend church on
Sunday morning. Over the next few years businesses relocated, taking
their employees, customers and tax payments with them.

\section{Louisville}\label{louisville}

\begin{itemize}
\item
  \emph{Police made 472 arrests related to the riots.}
\item
  \emph{As in many other cities around the country, there were unrest
  and riots partially in response to the assassination.}
\item
  \emph{The 1968 Louisville riots refers to riots in Louisville,
  Kentucky, in May 1968.}
\end{itemize}

The 1968 Louisville riots refers to riots in Louisville, Kentucky, in
May 1968. As in many other cities around the country, there were unrest
and riots partially in response to the assassination. On May 27, 1968, a
group of 400 people, mostly blacks, gathered at Twenty-Eight and
Greenwood Streets, in the Parkland neighborhood. The intersection, and
Parkland in general, had recently become an important location for
Louisville's black community, as the local NAACP branch had moved its
office there.

The crowd was protesting the possible reinstatement of a white officer
who had been suspended for beating an African-American man some weeks
earlier. Several community leaders arrived and told the crowd that no
decision had been reached, and alluded to disturbances in the future if
the officer was reinstated. By 8:30, the crowd began to disperse.

However, rumors (which turned out to be untrue) were spread that Student
Nonviolent Coordinating Committee speaker Stokely Carmichael's plane to
Louisville was being intentionally delayed by whites. After bottles were
thrown by the crowd, the crowd became unruly and police were called.
However the small and unprepared police response simply upset the crowd
more, which continued to grow. The police, including a captain who was
hit in the face by a bottle, retreated, leaving behind a patrol car,
which was turned over and burned.

By midnight, rioters had looted stores as far east as Fourth Street,
overturned cars and started fires.

Within an hour, Mayor Kenneth A. Schmied requested 700 Kentucky National
Guard troops and established a citywide curfew. Violence and vandalism
continued to rage the next day, but had subdued somewhat by May 29.
Business owners began to return, although troops remained until June 4.
Police made 472 arrests related to the riots. Two African-American
teenagers had died, and \$200,000 in damage had been done.

The disturbances had a longer-lasting effect. Most white business owners
quickly pulled out or were forced out of Parkland and surrounding areas.
Most white residents also left the West End, which had been almost
entirely white north of Broadway, from subdivision until the 1960s. The
riot would have effects that shaped the image which whites would hold of
Louisville's West End, that it was predominantly black and crime-ridden.

\includegraphics[width=5.50000in,height=3.66667in]{media/image2.jpg}\\
\emph{President Lyndon B. Johnson and Joe Califano chart riot outbreaks
in Washington, DC.}

\section{Federal response}\label{federal-response}

\begin{itemize}
\item
  \emph{On April 5, at 11:00 AM, Johnson met with an array of leaders in
  the Cabinet Room.}
\item
  \emph{At the meeting, Mayor Washington asked President Johnson to
  deploy troops to the District of Columbia.}
\item
  \emph{According to press secretary George Christian, Johnson was not
  surprised by the riots that followed: "What did you expect?}
\item
  \emph{On April 4, President Lyndon B. Johnson denounced King's
  murder.}
\end{itemize}

On April 4, President Lyndon B. Johnson denounced King's murder. He also
began to communicate with a host of mayors and governors preparing for a
reaction from black America. He cautioned against unnecessary force, but
felt like local governments would ignore his advice, saying to aides,
"I'm not getting through. They're all holing up like generals in a
dugout getting ready to watch a war."

On April 5, at 11:00 AM, Johnson met with an array of leaders in the
Cabinet Room. These included Vice President Hubert Humphrey, U.S.
Supreme Court Chief Justice Earl Warren, Supreme Court Justice Thurgood
Marshall, and federal judge Leon Higginbotham; government officials such
as secretary Robert Weaver and D.C. Mayor Walter Washington; legislators
Mike Mansfield, Everett Dirksen, William McCulloch; and civil rights
leaders Whitney Young, Roy Wilkins, Clarence Mitchell, Dorothy Height,
and Walter Fauntroy. Notably absent were representatives of more radical
groups such as SNCC and CORE. At the meeting, Mayor Washington asked
President Johnson to deploy troops to the District of Columbia. Richard
Hatcher, the newly elected black mayor of Gary, Indiana, spoke to the
group about white racism and his fears of racially motivated violence in
the future. Many of these leaders told Johnson that socially progressive
legislation would be the best response to the crisis. The meeting
concluded with prayers at the Washington National Cathedral.

According to press secretary George Christian, Johnson was not surprised
by the riots that followed: "What did you expect? I don't know why we're
so surprised. When you put your foot on a man's neck and hold him down
for three hundred years, and then you let him up, what's he going to do?
He's going to knock your block off."

\section{Military deployment}\label{military-deployment}

\begin{itemize}
\item
  \emph{After the Watts riots in 1965 and the Detroit riot of 1967, the
  military began preparing heavily for black insurrection.}
\item
  \emph{The Pentagon's Army Operations Center thus quickly began its
  response to the assassination on the night of April 4, directing air
  force transport planes to prepare for an occupation of Washington,
  D.C.}
\item
  \emph{On April 5, Johnson ordered mobilization of the Army and
  National Guard, particularly for D.C.}
\end{itemize}

After the Watts riots in 1965 and the Detroit riot of 1967, the military
began preparing heavily for black insurrection. The Pentagon's Army
Operations Center thus quickly began its response to the assassination
on the night of April 4, directing air force transport planes to prepare
for an occupation of Washington, D.C. The Army also dispatched
undercover agents to gather information.

On April 5, Johnson ordered mobilization of the Army and National Guard,
particularly for D.C.

\section{Legislative response}\label{legislative-response}

\begin{itemize}
\item
  \emph{These events led to the rapid passage of the Civil Rights Act of
  1968, Title VIII of which is known as the "Fair Housing Act".}
\item
  \emph{Some responded to the riots with suggestions for improving the
  conditions that engendered them.}
\item
  \emph{He urged Congress to pass the bill, starting with an April 5
  letter addressed to the Speaker of the United States House of
  Representatives, John William McCormack.}
\end{itemize}

Some responded to the riots with suggestions for improving the
conditions that engendered them. Many White House aides took the
opportunity to push their preferred programs for urban improvement. At
the same time, some members of Congress criticized Johnson. Senator
Richard Russell felt Johnson was not going far enough to suppress the
violence. Senator Robert Byrd suggested that Washington, D.C. ought to
be occupied indefinitely by the army.

Johnson chose to focus his political capital on a fair housing bill
proposed by Senator Sam Ervin. He urged Congress to pass the bill,
starting with an April 5 letter addressed to the Speaker of the United
States House of Representatives, John William McCormack. These events
led to the rapid passage of the Civil Rights Act of 1968, Title VIII of
which is known as the "Fair Housing Act".

\section{Communication with city and state
governments}\label{communication-with-city-and-state-governments}

\begin{itemize}
\item
  \emph{Audio records reveal a tense and variable relationship between
  Johnson and local officials.}
\item
  \emph{In conversations with Chicago mayor Richard J. Daley, Johnson
  describes the complications with ordering federal troops before local
  governments have exhausted all options.}
\item
  \emph{In the same call, Johnson told Daley he wanted to use a strategy
  of pre-emption: "I'd rather move them and not need them than need them
  and not have them."}
\end{itemize}

Audio records reveal a tense and variable relationship between Johnson
and local officials. In conversations with Chicago mayor Richard J.
Daley, Johnson describes the complications with ordering federal troops
before local governments have exhausted all options. Later, Johnson
would describe the domestic unrest as another front in the global war,
criticizing Daley for not requesting troops sooner. From the transcript:

In the same call, Johnson told Daley he wanted to use a strategy of
pre-emption: "I'd rather move them and not need them than need them and
not have them."

\section{Connection to local issues}\label{connection-to-local-issues}

\begin{itemize}
\item
  \emph{It was to these striking workers that King delivered his final
  speech, and in Memphis that he was killed.}
\item
  \emph{Negotiations on April 16 brought an end to the strike and a
  promise of better wages.}
\end{itemize}

The assassinations triggered active unrest in communities that were
already discontented. For example, the Memphis Sanitation Strike, which
was already underway, took on a new level of urgency. It was to these
striking workers that King delivered his final speech, and in Memphis
that he was killed. Negotiations on April 16 brought an end to the
strike and a promise of better wages.

In Oakland, increasing friction between Black Panthers and the police
led to the death of Bobby Hutton.

\section{Impact}\label{impact}

\begin{itemize}
\item
  \emph{Dozens of people were killed, and thousands of people injured,
  in the rioting over April 4 to 5.}
\end{itemize}

Dozens of people were killed, and thousands of people injured, in the
rioting over April 4 to 5.

\section{Physical}\label{physical}

\begin{itemize}
\item
  \emph{Some economists blame the riots of this period for subsequent
  social decay in urban communities, although many social scientists
  blame "federal disinvestments of the 1980s."}
\item
  \emph{Some areas were heavily damaged by the riots, and recovered
  slowly.}
\end{itemize}

Some areas were heavily damaged by the riots, and recovered slowly. In
Washington DC, poor urban planning decisions on the part of the federal
and local government were an obstacle to recovery. Some economists blame
the riots of this period for subsequent social decay in urban
communities, although many social scientists blame "federal
disinvestments of the 1980s."

\section{Political}\label{political}

\begin{itemize}
\item
  \emph{For some liberals and civil rights advocates, the riots were a
  turning point.}
\item
  \emph{On April 5, Johnson wrote a letter to the United States House of
  Representatives urging passage of the Civil Rights Act of 1968, which
  included the Fair Housing Act.}
\item
  \emph{The assassination and riots radicalized many, helping to fuel
  the Black Power movement.}
\end{itemize}

Dr. King had campaigned for a federal fair housing law throughout 1966,
but had not achieved it. Senator Walter Mondale advocated for the bill
in Congress, but noted that over successive years, a fair housing bill
was the most filibustered legislation in US history. It was opposed by
most Northern and Southern senators, as well as the National Association
of Real Estate Boards. Mondale commented that:

The riots quickly revived the bill. On April 5, Johnson wrote a letter
to the United States House of Representatives urging passage of the
Civil Rights Act of 1968, which included the Fair Housing Act. The Rules
Committee, "jolted by the repeated civil disturbances virtually outside
its door," finally ended its hearings on April 8. With newly urgent
attention from White House legislative director Joseph Califano and
Speaker of the House John McCormack, the bill---which was previously
stalled that year---passed the House by a wide margin on April 10.

For some liberals and civil rights advocates, the riots were a turning
point. They increased an already-strong trend toward racial segregation
and white flight in America's cities, strengthening racial barriers that
looked as though they might weaken. The riots were political fodder for
the Republican party, which used fears of black urban crime to garner
support for "law and order", especially in the 1968 presidential
campaign. The assassination and riots radicalized many, helping to fuel
the Black Power movement.

\section{See also}\label{see-also}

\begin{itemize}
\item
  \emph{Post-civil rights era African-American history}
\item
  \emph{List of incidents of civil unrest in the United States}
\end{itemize}

Red Summer (1919)

Long Hot Summer (1967)

Post-civil rights era African-American history

List of incidents of civil unrest in the United States

\section{References}\label{references}

\section{External links}\label{external-links}

\begin{itemize}
\item
  \emph{Letter to Maj. Gen. Thomas G. Wells authorizing him to command
  national guard and military forces for riot control in Memphis.}
\end{itemize}

Letter to Maj. Gen. Thomas G. Wells authorizing him to command national
guard and military forces for riot control in Memphis.
