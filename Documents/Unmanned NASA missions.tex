\textbf{From Wikipedia, the free encyclopedia}

https://en.wikipedia.org/wiki/Unmanned\%20NASA\%20missions\\
Licensed under CC BY-SA 3.0:\\
https://en.wikipedia.org/wiki/Wikipedia:Text\_of\_Creative\_Commons\_Attribution-ShareAlike\_3.0\_Unported\_License

\includegraphics[width=3.63733in,height=5.50000in]{media/image1.jpg}\\
\emph{Jupiter in December 2016 as seen by the Juno (spacecraft)}

\includegraphics[width=5.50000in,height=5.50000in]{media/image2.jpeg}\\
\emph{The collision of comet 9P/Tempel and the Deep Impact probe}

\section{List of uncrewed NASA
missions}\label{list-of-uncrewed-nasa-missions}

\begin{itemize}
\item
  \emph{Since 1958, NASA has overseen more than 1,000 uncrewed missions
  into Earth orbit or beyond.}
\item
  \emph{It has both launched its own missions, and provided funding for
  private-sector missions.}
\item
  \emph{A number of NASA missions, including the Explorers Program,
  Voyager program, and New Frontiers program, are still ongoing.}
\end{itemize}

Since 1958, NASA has overseen more than 1,000 uncrewed missions into
Earth orbit or beyond. It has both launched its own missions, and
provided funding for private-sector missions. A number of NASA missions,
including the Explorers Program, Voyager program, and New Frontiers
program, are still ongoing.

\section{List of missions}\label{list-of-missions}

\includegraphics[width=5.50000in,height=3.52344in]{media/image3.jpg}\\
\emph{Explorer 1 satellite.}

\section{Explorers Program
(1958--present)}\label{explorers-program-1958present}

\begin{itemize}
\item
  \emph{The Explorer program was later transferred to NASA, which
  continued to use the name for an ongoing series of relatively small
  space missions, typically an artificial satellite with a science
  focus.}
\item
  \emph{It has matured into one of NASA's lower-cost mission programs.}
\item
  \emph{Over the years, NASA has launched a series of Explorer
  spacecraft carrying a wide variety of scientific investigations.}
\end{itemize}

The Explorer program has launched more than 90 missions since it began
more than five decades ago. It has matured into one of NASA's lower-cost
mission programs.

The program started as a U.S. Army proposal to place a scientific
satellite into orbit during the International Geophysical Year
(1957--58). However, that proposal was rejected in favor of the U.S.
Navy's Project Vanguard. The Explorer program was later reestablished to
catch up with the Soviet Union after the launch of Sputnik 1 in October
1957. Explorer 1 was launched January 31, 1958; at this time the project
still belonged to the Army Ballistic Missile Agency (ABMA) and the Jet
Propulsion Laboratory (JPL). Besides being the first U.S. satellite, it
is known for discovering the Van Allen radiation belt.

The Explorer program was later transferred to NASA, which continued to
use the name for an ongoing series of relatively small space missions,
typically an artificial satellite with a science focus. Over the years,
NASA has launched a series of Explorer spacecraft carrying a wide
variety of scientific investigations.

\section{Pioneer program (1958--1978)}\label{pioneer-program-19581978}

\begin{itemize}
\item
  \emph{The Pioneer program was a series of NASA uncrewed space missions
  designed for planetary exploration.}
\item
  \emph{Additionally, the Pioneer mission to Venus consisted of two
  components, launched separately.}
\end{itemize}

The Pioneer program was a series of NASA uncrewed space missions
designed for planetary exploration. There were a number of missions in
the program, most notably Pioneer 10 and Pioneer 11, which explored the
outer planets and left the solar system. Both carry a golden plaque,
depicting a man and a woman and information about the origin and the
creators of the probes, should any extraterrestrials find them someday.

Additionally, the Pioneer mission to Venus consisted of two components,
launched separately. Pioneer Venus 1 (Pioneer Venus Orbiter) was
launched in May 1978 and remained in orbit until 1992. Pioneer Venus 2
(Pioneer Venus Multiprobe), launched in August 1978, sent four small
probes into the Venusian atmosphere.

\includegraphics[width=5.50000in,height=4.47883in]{media/image4.jpg}\\
\emph{Echo 2 in a hangar, N. Carolina. People can be seen on the floor.}

\section{Project Echo (1960--1964)}\label{project-echo-19601964}

\begin{itemize}
\item
  \emph{NASA's Echo 1 satellite was built by Gilmore Schjeldahl Company
  in Northfield, Minnesota.}
\item
  \emph{Each spacecraft was a metalized balloon satellite to be inflated
  in space and acting as a passive reflector of microwave signals.}
\end{itemize}

Project Echo was the first passive communications satellite experiment.
Each spacecraft was a metalized balloon satellite to be inflated in
space and acting as a passive reflector of microwave signals.
Communication signals were bounced off of them from one point on Earth
to another.\\
NASA's Echo 1 satellite was built by Gilmore Schjeldahl Company in
Northfield, Minnesota. Following the failure of the Delta rocket
carrying Echo 1 on May 13, 1960, Echo 1A was put successfully into orbit
by another Thor-Delta, and the first microwave transmission was received
on August 12, 1960.

Echo 2 was a 41.1-meter (135~ft) diameter metalized PET film balloon,
which was the last balloon satellite launched by Project Echo. It used
an improved inflation system to improve the balloon's smoothness and
sphericity. It was launched January 25, 1964, on a Thor Agena rocket.

\section{Ranger program (1961--1965)}\label{ranger-program-19611965}

\begin{itemize}
\item
  \emph{The Ranger program was a series of uncrewed space missions by
  the United States in the 1960s whose objective was to obtain the first
  close-up images of the surface of the Moon.}
\item
  \emph{Congress launched an investigation into "problems of management"
  at NASA Headquarters and JPL.}
\item
  \emph{The JPL mission designers planned multiple launches in each
  block, to maximize the engineering experience and scientific value of
  the mission and to assure at least one successful flight.}
\end{itemize}

The Ranger program was a series of uncrewed space missions by the United
States in the 1960s whose objective was to obtain the first close-up
images of the surface of the Moon. The Ranger spacecraft were designed
to take images of the lunar surface, returning those images until they
were destroyed upon impact. A series of mishaps, however, led to the
failure of the first five flights. Congress launched an investigation
into "problems of management" at NASA Headquarters and JPL. After
reorganizing the organization twice, Ranger 7 successfully returned
images in July 1964, followed by two more successful missions.

Ranger was originally designed, beginning in 1959, in three distinct
phases, called "blocks." Each block had different mission objectives and
progressively more advanced system design. The JPL mission designers
planned multiple launches in each block, to maximize the engineering
experience and scientific value of the mission and to assure at least
one successful flight. Total research, development, launch, and support
costs for the Ranger series of spacecraft (Rangers 1 through 9) was
approximately \$170~million.

\includegraphics[width=4.78010in,height=5.50000in]{media/image5.jpg}\\
\emph{Telstar 1}

\section{Telstar (1962--1963)}\label{telstar-19621963}

\begin{itemize}
\item
  \emph{Telstar was not a NASA program but rather a commercial
  communication satellite project.}
\item
  \emph{NASA's contributions to it were limited to launch services, as
  well as tracking and telemetry duties.}
\item
  \emph{Telstar 1 was launched on top of a Thor-Delta rocket on July 10,
  1962.}
\item
  \emph{Telstar 2 was launched May 7, 1963.}
\item
  \emph{It successfully relayed through space the first television
  pictures, telephone calls, and fax images, as well as providing the
  first live transatlantic television feed.}
\end{itemize}

Telstar was not a NASA program but rather a commercial communication
satellite project. NASA's contributions to it were limited to launch
services, as well as tracking and telemetry duties. The first two
Telstar satellites were experimental and nearly identical. Telstar 1 was
launched on top of a Thor-Delta rocket on July 10, 1962. It successfully
relayed through space the first television pictures, telephone calls,
and fax images, as well as providing the first live transatlantic
television feed. Telstar 2 was launched May 7, 1963.

Bell Telephone Laboratories designed and built the Telstar satellites.
They were prototypes intended to prove various concepts behind the large
constellation of orbiting satellites. Bell Telephone Laboratories also
developed much of the technology required for satellite communication,
including transistors, solar cells, and traveling wave tube amplifiers.
AT\&T built ground stations to handle Telstar communications.

\section{Mariner program (1962--1973)}\label{mariner-program-19621973}

\begin{itemize}
\item
  \emph{The Mariner program conducted by NASA launched a series of
  robotic interplanetary probes designed to investigate Mars, Venus and
  Mercury.}
\item
  \emph{The planned Mariner 11 and Mariner 12 vehicles evolved into
  Voyager 1 and Voyager 2, while the Viking 1 and Viking 2 Mars orbiters
  were enlarged versions of the Mariner 9 spacecraft.}
\end{itemize}

The Mariner program conducted by NASA launched a series of robotic
interplanetary probes designed to investigate Mars, Venus and Mercury.
The program included a number of firsts, including the first planetary
flyby, the first pictures from another planet, the first planetary
orbiter, and the first interplanetary gravity assist maneuver.

Of the ten vehicles in the Mariner series, seven were successful, and
three were lost. The planned Mariner 11 and Mariner 12 vehicles evolved
into Voyager 1 and Voyager 2, while the Viking 1 and Viking 2 Mars
orbiters were enlarged versions of the Mariner 9 spacecraft. Other
Mariner-based spacecraft included the Magellan probe to Venus and the
Galileo probe to Jupiter. The second-generation Mariner spacecraft,
called the Mariner Mark II series, eventually evolved into the
Cassini--Huygens probe, which spent more than 13 years in orbit around
Saturn.

All Mariner spacecraft were based on a hexagonal or octagonal "bus,"
which housed all of the electronics, and to which all components were
attached, such as antennae, cameras, propulsion, and power sources. All
probes except Mariner 1, Mariner 2 and Mariner 5 had TV cameras. The
first five Mariners were launched on Atlas-Agena rockets, while the last
five used the Atlas-Centaur.

\includegraphics[width=5.50000in,height=3.05067in]{media/image6.jpg}\\
\emph{Lunar Orbiter spacecraft (NASA)}

\section{Lunar Orbiter program
(1966--1967)}\label{lunar-orbiter-program-19661967}

\begin{itemize}
\item
  \emph{During the Lunar Orbiter missions, the first pictures of Earth
  as a whole were taken, beginning with Earth-rise over the lunar
  surface by Lunar Orbiter 1 in August 1966.}
\item
  \emph{The program produced the first photographs ever taken from lunar
  orbit.}
\item
  \emph{The Lunar Orbiter program was a series of five uncrewed lunar
  orbiter missions launched by the United States, starting in 1966.}
\end{itemize}

The Lunar Orbiter program was a series of five uncrewed lunar orbiter
missions launched by the United States, starting in 1966. It was
intended to help select landing sites for the Apollo program by mapping
the Moon's surface. The program produced the first photographs ever
taken from lunar orbit.

All five missions were successful, and 99\% of the Moon was mapped from
photographs taken with a resolution of 60 meters (200~ft) or better. The
first three missions were dedicated to imaging 20 potential human lunar
landing sites, selected based on Earth-based observations. These were
flown at low inclination orbits. The fourth and fifth missions were
devoted to broader scientific objectives and were flown in high-altitude
polar orbits. All Lunar Orbiter craft were launched by an Atlas-Agena D
launch vehicle.

During the Lunar Orbiter missions, the first pictures of Earth as a
whole were taken, beginning with Earth-rise over the lunar surface by
Lunar Orbiter 1 in August 1966. The first full picture of the whole
Earth was taken by Lunar Orbiter 5 on August 8, 1967. A second photo of
the whole Earth was taken by Lunar Orbiter 5 on November 10, 1967.

\includegraphics[width=5.50000in,height=5.07380in]{media/image7.jpg}\\
\emph{Apollo 12 astronaut inspecting Surveyor 3. Lunar module is seen in
the background. 1969}

\section{Surveyor program (1966--1968)}\label{surveyor-program-19661968}

\begin{itemize}
\item
  \emph{The program was implemented by NASA's Jet Propulsion Laboratory
  (JPL) to prepare for the Apollo program.}
\item
  \emph{All seven spacecraft are still on the Moon; none of the missions
  included returning them to Earth.}
\item
  \emph{The Surveyor Program was a NASA program that, from 1966 through
  1968, sent seven robotic spacecraft to the surface of the Moon.}
\end{itemize}

The Surveyor Program was a NASA program that, from 1966 through 1968,
sent seven robotic spacecraft to the surface of the Moon. Its primary
goal was to demonstrate the feasibility of soft landings on the Moon.
The program was implemented by NASA's Jet Propulsion Laboratory (JPL) to
prepare for the Apollo program. The total cost of the Surveyor program
was officially \$469~million.

Five of the Surveyor craft successfully soft-landed on the Moon. Two
failed: Surveyor 2 crashed at high velocity after a failed mid-course
correction, and Surveyor 4 was lost for contact 2.5 minutes before its
scheduled touch-down.

All seven spacecraft are still on the Moon; none of the missions
included returning them to Earth. Some parts of Surveyor 3 were returned
to Earth by the crew of Apollo 12, which landed near it in 1969.

\includegraphics[width=4.30467in,height=5.50000in]{media/image8.jpg}\\
\emph{Helios probe spacecraft}

\section{Helios probes (1974--1976)}\label{helios-probes-19741976}

\begin{itemize}
\item
  \emph{The Helios space probes completed their primary missions by the
  early 1980s, but they continued to send data up to 1985.}
\item
  \emph{A joint venture of the Federal Republic of Germany (West
  Germany) and NASA, the probes were launched from Cape Canaveral Air
  Force Station, Florida, on December 10, 1974, and January 15, 1976,
  respectively.}
\end{itemize}

Helios I and Helios II, also known as Helios-A and Helios-B, were a pair
of space probes launched into heliocentric orbit for the purpose of
studying solar processes. A joint venture of the Federal Republic of
Germany (West Germany) and NASA, the probes were launched from Cape
Canaveral Air Force Station, Florida, on December 10, 1974, and January
15, 1976, respectively. Helios 2 set a maximum speed record among
spacecraft at about 247,000 kilometers per hour (153,000~mph) relative
to the Sun (68.6 kilometers per second (42.6~mi/s) or 0.000229c). The
Helios space probes completed their primary missions by the early 1980s,
but they continued to send data up to 1985. The probes are no longer
functional but still remain in their elliptical orbit around the Sun.

\includegraphics[width=5.01600in,height=5.50000in]{media/image9.jpg}\\
\emph{Viking at Mars releasing the descent capsule, artist concept}

\section{Viking program (1975)}\label{viking-program-1975}

\begin{itemize}
\item
  \emph{The Viking program consisted of a pair of American space probes
  sent to Mars---Viking 1 and Viking 2.}
\item
  \emph{Viking 1 was launched on August 20, 1975, and the second craft,
  Viking 2, was launched on September 9, 1975, both riding atop Titan
  III-E rockets with Centaur upper stages.}
\item
  \emph{By discovering many geological forms that are typically formed
  from large amounts of water, the Viking program caused a revolution in
  scientific ideas about water on Mars.}
\end{itemize}

The Viking program consisted of a pair of American space probes sent to
Mars---Viking 1 and Viking 2. Each vehicle was composed of two main
parts, an orbiter designed to photograph the surface of Mars from orbit,
and a lander designed to study the planet from the surface. The orbiters
also served as communication relays for the landers once they touched
down. Viking 1 was launched on August 20, 1975, and the second craft,
Viking 2, was launched on September 9, 1975, both riding atop Titan
III-E rockets with Centaur upper stages. By discovering many geological
forms that are typically formed from large amounts of water, the Viking
program caused a revolution in scientific ideas about water on Mars.

The primary objectives of the Viking orbiters were to transport the
landers to Mars, perform reconnaissance to locate and certify landing
sites, act as communications relays for the landers, and to perform
their own scientific investigations. The orbiter, based on the earlier
Mariner 9 spacecraft, was an octagon approximately 2.5~m (8.2~ft)
across. The total launch mass was 2,328 kilograms (5,132~lb), of which
1,445 kilograms (3,186~lb) were propellant and attitude control gas.

\includegraphics[width=5.50000in,height=4.30375in]{media/image10.jpg}\\
\emph{Voyager probe}

\section{Voyager program
(1977--present)}\label{voyager-program-1977present}

\begin{itemize}
\item
  \emph{The Voyager program consists of a pair of uncrewed scientific
  probes, Voyager 1 and Voyager 2.}
\item
  \emph{Voyager 1 entered interstellar space in 2012.}
\end{itemize}

The Voyager program consists of a pair of uncrewed scientific probes,
Voyager 1 and Voyager 2. They were launched in 1977 to take advantage of
a favorable planetary alignment of the late 1970s. Although they were
originally designated to study just Jupiter and Saturn, Voyager 2 was
able to continue to Uranus and Neptune. Both missions have gathered
large amounts of data about the gas giants of the solar system, of which
little was previously known. Both probes have achieved escape velocity
from the Solar System and will never return. Voyager 1 entered
interstellar space in 2012.

As of January~19, 2019{[}update{]}, Voyager 1 was at a distance of
145.148~AU (13.492~billion miles (21.713×10\^{}9~km)) from the Earth,
traveling away from the Sun at a speed of about 10.6~mi/s (17.1~km/s),
which corresponds to a greater specific orbital energy than any other
probe.

\section{High Energy Astronomy Observatory 1
(1977)}\label{high-energy-astronomy-observatory-1-1977}

\begin{itemize}
\item
  \emph{The first of NASA's three High Energy Astronomy Observatories,
  HEAO 1, launched August 12, 1977, aboard an Atlas rocket with a
  Centaur upper stage, operated until January 9, 1979.}
\item
  \emph{HEAO 1 re-entered the Earth's atmosphere on March 15, 1979.}
\end{itemize}

The first of NASA's three High Energy Astronomy Observatories, HEAO 1,
launched August 12, 1977, aboard an Atlas rocket with a Centaur upper
stage, operated until January 9, 1979. During that time, it scanned the
X-ray sky almost three times over 0.2 keV -- 10 MeV, provided nearly
constant monitoring of X-ray sources near the ecliptic poles, as well as
more detailed studies of a number of objects through pointed
observations.

HEAO included four large X-ray and gamma-ray astronomy instruments,
known as A1, A2, A3, and A4, respectively (before launch, HEAO 1 was
known as HEAO A). The orbital inclination was about 22.7 degrees. HEAO 1
re-entered the Earth's atmosphere on March 15, 1979.

\includegraphics[width=5.50000in,height=4.46086in]{media/image11.jpg}\\
\emph{SMM satellite}

\section{Solar Maximum Mission (1980)}\label{solar-maximum-mission-1980}

\begin{itemize}
\item
  \emph{Although not unique in this endeavor, the SMM was notable in
  that its useful life compared with similar spacecraft was
  significantly increased by the direct intervention of a human space
  mission.}
\item
  \emph{It was launched on February 14, 1980.}
\item
  \emph{The Solar Maximum Mission ended on December 2, 1989, when the
  spacecraft re-entered the atmosphere and burned up.}
\end{itemize}

The Solar Maximum Mission satellite (or SolarMax) was designed to
investigate solar phenomena, particularly solar flares. It was launched
on February 14, 1980.

Although not unique in this endeavor, the SMM was notable in that its
useful life compared with similar spacecraft was significantly increased
by the direct intervention of a human space mission. During STS-41-C in
1984, the Space Shuttle Challenger intercepted the SMM, maneuvering it
into the shuttle's payload bay for maintenance and repairs. SMM had been
fitted with a shuttle "grapple fixture" so that the shuttle's robot arm
could grab it for repair.

The Solar Maximum Mission ended on December 2, 1989, when the spacecraft
re-entered the atmosphere and burned up.

\includegraphics[width=5.50000in,height=3.78307in]{media/image12.jpg}\\
\emph{IRAS beside some of its all-sky images}

\section{Infrared Astronomical Satellite
(1983)}\label{infrared-astronomical-satellite-1983}

\begin{itemize}
\item
  \emph{The spacecraft continues to orbit close to the Earth.}
\item
  \emph{The telescope was a joint project of the United States (NASA),
  the Netherlands (NIVR), and the United Kingdom (SERC).}
\end{itemize}

The Infrared Astronomical Satellite (IRAS) was the first-ever
space-based observatory to perform a survey of the entire sky at
infrared wavelengths. It discovered about 350,000 sources, many of which
are still awaiting identification. New discoveries included a dust disk
around Vega and the first images of the Milky Way Galaxy's core.

IRAS's life, like those of most infrared satellites that followed it,
was limited by its cooling system. To effectively work in the infrared
domain, the telescope must be cooled to cryogenic temperatures.
Superfluid helium kept IRAS at a temperature of 2 kelvins (about
−271~°C) by evaporation. The supply of liquid helium was depleted on
November 21, 1983, preventing further observations. The spacecraft
continues to orbit close to the Earth.

The telescope was a joint project of the United States (NASA), the
Netherlands (NIVR), and the United Kingdom (SERC). Over 250,000 infrared
sources were observed at 12, 25, 60, and 100 micrometer wavelengths.

\includegraphics[width=3.85733in,height=5.50000in]{media/image13.jpg}\\
\emph{The Magellan Probe prepared for launch}

\section{Magellan probe (1989)}\label{magellan-probe-1989}

\begin{itemize}
\item
  \emph{It was also the first deep-space probe to be launched on the
  Space Shuttle.}
\item
  \emph{The Magellan spacecraft was a space probe sent to the planet
  Venus, the first uncrewed interplanetary spacecraft to be launched by
  NASA since its successful Pioneer Orbiter, also to Venus, in 1978.}
\end{itemize}

The Magellan spacecraft was a space probe sent to the planet Venus, the
first uncrewed interplanetary spacecraft to be launched by NASA since
its successful Pioneer Orbiter, also to Venus, in 1978. It was also the
first deep-space probe to be launched on the Space Shuttle. In 1993, it
employed aerobraking techniques to lower its orbit. This was the first
prolonged use of the technique, which had been tested by Hiten in 1991.

Magellan created the first (and currently the best) high-resolution
mapping of the planet's surface features. Prior Venus missions had
created low-resolution radar globes of general, continent-sized
formations. Magellan, performed detailed imaging and analysis of
craters, hills, ridges, and other geologic formations, to a degree
comparable to the visible-light photographic mapping of other planets.

\includegraphics[width=4.40000in,height=5.50000in]{media/image14.jpg}\\
\emph{The Galileo probe}

\section{Galileo (1989)}\label{galileo-1989}

\begin{itemize}
\item
  \emph{It was launched on October 18, 1989, by the Space Shuttle
  Atlantis on the STS-34 mission.}
\item
  \emph{Galileo was an uncrewed spacecraft sent by NASA to study the
  planet Jupiter and its moons.}
\item
  \emph{Despite antenna problems, Galileo conducted the first asteroid
  flyby, discovered the first asteroid moon, was the first spacecraft to
  orbit Jupiter, and launched the first probe into Jupiter's
  atmosphere.}
\end{itemize}

Galileo was an uncrewed spacecraft sent by NASA to study the planet
Jupiter and its moons. It was launched on October 18, 1989, by the Space
Shuttle Atlantis on the STS-34 mission. It arrived at Jupiter on
December 7, 1995, via gravitational assist flybys of Venus and Earth.

Despite antenna problems, Galileo conducted the first asteroid flyby,
discovered the first asteroid moon, was the first spacecraft to orbit
Jupiter, and launched the first probe into Jupiter's atmosphere.
Galileo's prime mission was a two-year study of the Jovian system. The
spacecraft traveled around Jupiter in elongated ellipses, each orbit
lasting about two months. The differing distances from Jupiter afforded
by these orbits allowed Galileo to sample different parts of the
planet's extensive magnetosphere. The orbits were designed for close up
flybys of Jupiter's largest moons. Once Galileo's prime mission was
concluded, an extended mission followed starting on December 7, 1997;
the spacecraft made a number of close flybys of Jupiter's moons Europa
and Io.

On September 21, 2003, Galileo's mission was terminated by sending the
orbiter into Jupiter's atmosphere at a speed of nearly 50 kilometers per
second. The spacecraft was low on propellant; another reason for its
destruction was to avoid contamination of local moons, such as Europa,
with bacteria from Earth.

\includegraphics[width=5.50000in,height=4.12500in]{media/image15.jpeg}\\
\emph{The Hubble Space Telescope}

\section{Hubble Space Telescope
(1990)}\label{hubble-space-telescope-1990}

\begin{itemize}
\item
  \emph{The Hubble Space Telescope (HST) is a space telescope that was
  carried into orbit by a Space Shuttle in April 1990.}
\item
  \emph{The HST is a collaboration between NASA and the European Space
  Agency, and is one of NASA's Great Observatories, along with the
  Compton Gamma Ray Observatory, the Chandra X-ray Observatory, and the
  Spitzer Space Telescope.}
\end{itemize}

The Hubble Space Telescope (HST) is a space telescope that was carried
into orbit by a Space Shuttle in April 1990. It is named after American
astronomer Edwin Hubble. Although not the first space telescope, Hubble
is one of the largest and most versatile, and is well known as both a
vital research tool and a public relations boon for astronomy. The HST
is a collaboration between NASA and the European Space Agency, and is
one of NASA's Great Observatories, along with the Compton Gamma Ray
Observatory, the Chandra X-ray Observatory, and the Spitzer Space
Telescope. The HST's success has paved the way for greater collaboration
between the agencies.

The HST was created with a budget of \$2~billion and has continued
operation since 1990, delighting both scientists and the public. Some of
its images, such as the Hubble Deep Field, have become famous.

\includegraphics[width=5.50000in,height=5.29405in]{media/image16.jpg}\\
\emph{Ulysses (artist rendering)}

\section{Ulysses (1990)}\label{ulysses-1990}

\begin{itemize}
\item
  \emph{Mission scientists kept the fuel liquid by conducting short
  thruster burns, allowing the mission to continue.}
\item
  \emph{Ulysses is a decommissioned robotic space probe that was
  designed to study the Sun as a joint venture of NASA and the European
  Space Agency (ESA).}
\item
  \emph{The spacecraft's mission was to study the Sun at all latitudes.}
\end{itemize}

Ulysses is a decommissioned robotic space probe that was designed to
study the Sun as a joint venture of NASA and the European Space Agency
(ESA). Ulysses was launched on October 6, 1990, aboard Discovery
(mission STS-41). The spacecraft's mission was to study the Sun at all
latitudes. This required a major orbital plane shift, which was
accomplished by using an encounter with Jupiter. The need for a Jupiter
encounter meant that Ulysses could not be powered by solar cells and was
powered by a radioisotope thermoelectric generator (RTG) instead.

By February 2008, the power output from the RTG, which is generated by
heat from radioactive decay, had decreased enough to leave insufficient
power to keep the spacecraft's attitude control hydrazine fuel from
freezing. Mission scientists kept the fuel liquid by conducting short
thruster burns, allowing the mission to continue. The cessation of
mission operations and deactivation of the spacecraft was determined by
the inability to prevent attitude control fuel from freezing. The last
day for mission operations on Ulysses was June 30, 2009.

\includegraphics[width=5.50000in,height=4.28349in]{media/image17.jpg}\\
\emph{Upper Atmosphere Research Satellite (UARS) deployed}

\section{Upper Atmosphere Research Satellite
(1991)}\label{upper-atmosphere-research-satellite-1991}

\begin{itemize}
\item
  \emph{At about 6 tonnes, it was the heaviest NASA satellite to undergo
  uncontrolled atmospheric entry since Skylab in the summer of 1979.}
\item
  \emph{Planned for a three-year mission, it proved much more durable,
  allowing extended observation from its instrument suite.}
\item
  \emph{It was launched aboard Space Shuttle Discovery and deployed into
  space from the payload bay with its robotic arm, under guidance from
  the crew.}
\end{itemize}

The Upper Atmosphere Research Satellite (UARS) was a science satellite
used from 1991 to 2005 to study Earth's atmosphere, including the ozone
layer. Planned for a three-year mission, it proved much more durable,
allowing extended observation from its instrument suite. It was launched
aboard Space Shuttle Discovery and deployed into space from the payload
bay with its robotic arm, under guidance from the crew. The satellite
underwent atmospheric re-entry at about 04:00 24 September 2011 UTC. At
about 6 tonnes, it was the heaviest NASA satellite to undergo
uncontrolled atmospheric entry since Skylab in the summer of 1979.

\includegraphics[width=5.50000in,height=5.04279in]{media/image18.jpg}\\
\emph{Mars Pathfinder on Mars}

\includegraphics[width=3.68133in,height=5.50000in]{media/image19.jpg}\\
\emph{Kepler space telescope}

\includegraphics[width=5.50000in,height=4.59866in]{media/image20.jpg}\\
\emph{Genesis spacecraft}

\includegraphics[width=5.50000in,height=4.25258in]{media/image21.jpg}\\
\emph{Deep Impact space probe after impactor separation (artist
concept)}

\section{Discovery Program
(1992--present)}\label{discovery-program-1992present}

\begin{itemize}
\item
  \emph{Missions funded by NASA through this program include Mars
  Pathfinder, Kepler, Stardust, Genesis and Deep Impact.}
\item
  \emph{The mission was directed by the Jet Propulsion Laboratory, a
  division of the California Institute of Technology, responsible for
  NASA's Mars Exploration Program.}
\item
  \emph{Deep Impact is a NASA space probe launched on January 12, 2005.}
\item
  \emph{It was the second project from NASA's Discovery Program.}
\end{itemize}

NASA's Discovery Program (as compared to New Frontiers or Flagship
Programs) is a series of lower-cost, highly focused scientific space
missions that are exploring the Solar System. It was founded in 1992 to
implement then-NASA Administrator Daniel S. Goldin's vision of "faster,
better, cheaper" planetary missions. Discovery missions differ from
traditional NASA missions where targets and objectives are
pre-specified. Instead, these cost-capped missions are proposed and led
by a scientist called the Principal investigator (PI). Proposing teams
may include people from industry, small businesses, government
laboratories, and universities. Proposals are selected through a
competitive peer review process. The Discovery missions are adding
significantly to the body of knowledge about the Solar System.

NASA also accepts proposals for competitively selected Discovery Program
Missions of Opportunity. This provides opportunities to participate in
non-NASA missions by providing funding for a science instrument or
hardware components of a science instrument or to re-purpose an existing
NASA spacecraft.

Missions funded by NASA through this program include Mars Pathfinder,
Kepler, Stardust, Genesis and Deep Impact.

The Mars Pathfinder (MESUR Pathfinder) was launched on December 4, 1996,
just a month after the Mars Global Surveyor was launched. On board the
lander, later renamed the Carl Sagan Memorial Station, was a small rover
called Sojourner that executed many experiments on the Martian surface.
It was the second project from NASA's Discovery Program. The mission was
directed by the Jet Propulsion Laboratory, a division of the California
Institute of Technology, responsible for NASA's Mars Exploration
Program.

Stardust was a 300-kilogram robotic space probe launched by NASA on
February 7, 1999, to study the asteroid 5535 Annefrank and collect
samples from the coma of comet Wild 2. The primary mission was completed
January 15, 2006, when the sample return capsule returned to Earth.
Stardust intercepted comet Tempel 1 on February 15, 2011, a small Solar
System body previously visited by Deep Impact on July 4, 2005. Stardust
was decommissioned on March 25, 2011. It is the first sample return
mission to collect cosmic dust.

The Genesis spacecraft was a NASA sample return probe which collected a
sample of solar wind and returned it to Earth for analysis. It was the
first NASA sample return mission to return material since the Apollo
Program, and the first to return material from beyond the orbit of the
Moon. Genesis was launched on August 8, 2001, and crash-landed in Utah
on September 8, 2004, after a design flaw prevented the deployment of
its drogue parachute. The crash contaminated and damaged many of the
sample collectors, but many of them were successfully recovered.

Deep Impact is a NASA space probe launched on January 12, 2005. It was
designed to study the composition of the interior of comet 9P/Tempel, by
releasing an impactor into the comet. On July 4, 2005, the impactor
successfully collided with the comet's nucleus, excavating debris from
the interior of the nucleus. Photographs of the debris and impact crater
showed that the comet was very porous and its outgassing was chemically
diverse.

Kepler is a space observatory launched by NASA to discover Earth-like
planets orbiting other stars. The spacecraft, named in honor of the
17th-century German astronomer Johannes Kepler, was launched in March
2009. Kepler's primary mission ended in May 2013 when it lost a second
reaction wheel. The telescope's second mission, K2, began in May 2014.
As of February 2018, Kepler has discovered more than 2000 exoplanets.

\includegraphics[width=4.38053in,height=5.50000in]{media/image22.jpg}\\
\emph{Clementine satellite}

\section{Clementine (1994)}\label{clementine-1994}

\begin{itemize}
\item
  \emph{Clementine (officially called the Deep Space Program Science
  Experiment (DSPSE)) was a joint space project between the Ballistic
  Missile Defense Organization (BMDO, previously the Strategic Defense
  Initiative Organization, or SDIO) and NASA.}
\item
  \emph{Launched on January 25, 1994, the objective of the mission was
  to test sensors and spacecraft components under extended exposure to
  the space environment and to make scientific observations of the Moon
  and the near-Earth asteroid 1620 Geographos.}
\end{itemize}

Clementine (officially called the Deep Space Program Science Experiment
(DSPSE)) was a joint space project between the Ballistic Missile Defense
Organization (BMDO, previously the Strategic Defense Initiative
Organization, or SDIO) and NASA. Launched on January 25, 1994, the
objective of the mission was to test sensors and spacecraft components
under extended exposure to the space environment and to make scientific
observations of the Moon and the near-Earth asteroid 1620 Geographos.
The Geographos observations were not made due to a malfunction in the
spacecraft.

\includegraphics[width=4.74467in,height=5.50000in]{media/image23.jpg}\\
\emph{Artist's conception of the Mars Global Surveyor}

\section{Mars Global Surveyor (1996)}\label{mars-global-surveyor-1996}

\begin{itemize}
\item
  \emph{In January 2007 NASA officially ended the mission.}
\item
  \emph{The Mars Global Surveyor (MGS) was developed by NASA's Jet
  Propulsion Laboratory and launched November 1996.}
\item
  \emph{It completed its primary mission in January 2001 and was in its
  third extended mission phase when, on November 2, 2006, the spacecraft
  failed to respond to commands.}
\end{itemize}

The Mars Global Surveyor (MGS) was developed by NASA's Jet Propulsion
Laboratory and launched November 1996. It began the United States'
return to Mars after a 10-year absence. It completed its primary mission
in January 2001 and was in its third extended mission phase when, on
November 2, 2006, the spacecraft failed to respond to commands. In
January 2007 NASA officially ended the mission.

The Surveyor spacecraft used a series of high-resolution cameras to
explore the surface of Mars, returning more than 240,000 images from
September 1997 to November 2006. The surveyor had three cameras; a
high-resolution camera took black-and-white images (usually 1.5 to 12 m
per pixel), and red and blue wide-angle cameras took images for context
(240 m per pixel) and daily global images (7.5 kilometers (4.7~mi) per
pixel).

\includegraphics[width=5.50000in,height=3.66667in]{media/image24.jpg}\\
\emph{Artist's concept of Cassini's Saturn orbit insertion}

\section{Cassini--Huygens (1997)}\label{cassinihuygens-1997}

\begin{itemize}
\item
  \emph{Cassini was the fourth space probe to visit Saturn and the first
  to enter orbit.}
\item
  \emph{The mission was managed by NASA's Jet Propulsion Laboratory in
  the United States, where the orbiter was assembled.}
\item
  \emph{Cassini--Huygens was a joint NASA/ESA/ASI spacecraft mission
  studying the planet Saturn and its many natural satellites.}
\end{itemize}

Cassini--Huygens was a joint NASA/ESA/ASI spacecraft mission studying
the planet Saturn and its many natural satellites. It included a Saturn
orbiter and an atmospheric probe/lander for the moon Titan, although it
also returned data on a wide variety of other things including the
Heliosphere, Jupiter, and relativity tests. The Titan probe, Huygens,
entered and landed on Titan in 2005. Cassini was the fourth space probe
to visit Saturn and the first to enter orbit.

It launched on October 15, 1997, on a Titan IVB/Centaur and entered into
orbit around Saturn on July 1, 2004, after an interplanetary voyage
which included flybys of Earth, Venus, and Jupiter. On December 25,
2004, Huygens separated from the orbiter at approximately 02:00 UTC. It
reached Saturn's moon Titan on January 14, 2005, when it entered Titan's
atmosphere and descended down to the surface. It successfully returned
data to Earth, using the orbiter as a relay. This was the first landing
ever accomplished in the outer Solar System.

Sixteen European countries and the United States made up the team
responsible for designing, building, flying and collecting data from the
Cassini orbiter and Huygens probe. The mission was managed by NASA's Jet
Propulsion Laboratory in the United States, where the orbiter was
assembled. Huygens was developed by the European Space Research and
Technology Centre.

After several mission extensions, Cassini was deliberately plunged into
Saturn's atmosphere on September 15, 2017, to prevent contamination of
habitable moons.

\includegraphics[width=5.50000in,height=4.12500in]{media/image25.jpg}\\
\emph{NASA Earth observatories}

\section{Earth Observing System
(1997--present)}\label{earth-observing-system-1997present}

\begin{itemize}
\item
  \emph{The program is the centerpiece of NASA's Earth Science
  Enterprise (ESE).}
\item
  \emph{The Earth Observing System (EOS) is a program of NASA comprising
  a series of artificial satellite missions and scientific instruments
  in Earth orbit designed for long-term global observations of the land
  surface, biosphere, atmosphere, and oceans of the Earth.}
\item
  \emph{The satellite component of the program was launched in 1997.}
\end{itemize}

The Earth Observing System (EOS) is a program of NASA comprising a
series of artificial satellite missions and scientific instruments in
Earth orbit designed for long-term global observations of the land
surface, biosphere, atmosphere, and oceans of the Earth. The satellite
component of the program was launched in 1997. The program is the
centerpiece of NASA's Earth Science Enterprise (ESE). Missions carried
out through this program include SeaWiFS (1997), Landsat 7 (1999),
QuikSCAT (1999), Jason 1 (2001), GRACE (2002), Aqua (2002), Aura (2004)
and Aquarius (2011).

\includegraphics[width=5.50000in,height=4.31485in]{media/image26.jpg}\\
\emph{Artist rendering of Deep Space I's flyby of comet 19P/Borrelly}

\section{New Millennium Program
(1998--2006)}\label{new-millennium-program-19982006}

\begin{itemize}
\item
  \emph{With a refocusing of the program in 2000, the Deep Space series
  was renamed "Space Technology."}
\item
  \emph{The spacecraft in the New Millennium Program were originally
  named "Deep Space" (for missions demonstrating technology for
  planetary missions) and "Earth Observing" (for missions demonstrating
  technology for Earth-orbiting missions).}
\end{itemize}

New Millennium Program (NMP) is a NASA project with a focus on
engineering validation of new technologies for space applications.
Funding for the program was eliminated from the FY2009 budget by the
110th United States Congress, effectively leading to its cancellation.
The spacecraft in the New Millennium Program were originally named "Deep
Space" (for missions demonstrating technology for planetary missions)
and "Earth Observing" (for missions demonstrating technology for
Earth-orbiting missions). With a refocusing of the program in 2000, the
Deep Space series was renamed "Space Technology."

Deep Space 1 (DS1) is a spacecraft dedicated to testing a payload of
advanced, high-risk technologies. Launched on October 24, 1998, the Deep
Space 1 mission carried out a flyby of asteroid 9969 Braille, the
mission's science target. Its mission was extended twice to include an
encounter with Comet Borrelly and further engineering testing. Problems
during its initial stages and with its star tracker led to repeated
changes in mission configuration. Deep Space 1 tested twelve
technologies. It was the first spacecraft to use ion thrusters, in
contrast to the traditional chemical powered rockets.

The Deep Space series was continued by the Deep Space 2 probes, which
were launched in January 1999 on Mars Polar Lander and were intended to
strike the surface of Mars.

\includegraphics[width=5.50000in,height=3.71287in]{media/image27.jpg}\\
\emph{Artist's concept of the twin GRACE satellites}

\section{Gravity Recovery and Climate Experiment
(2002)}\label{gravity-recovery-and-climate-experiment-2002}

\begin{itemize}
\item
  \emph{The Gravity Recovery and Climate Experiment (GRACE), a joint
  mission of NASA and the German Aerospace Center, made detailed
  measurements of Earth's gravity field from its launch in March 2002
  until October 2017.}
\item
  \emph{The Jet Propulsion Laboratory was responsible for the overall
  mission management under the NASA ESSP program.}
\end{itemize}

The Gravity Recovery and Climate Experiment (GRACE), a joint mission of
NASA and the German Aerospace Center, made detailed measurements of
Earth's gravity field from its launch in March 2002 until October 2017.
The satellites were launched from Plesetsk Cosmodrome, Russia on a
Rockot launch vehicle. By measuring gravity, GRACE showed how mass is
distributed around the planet and how it varies over time. Data from the
GRACE satellites is an important tool for studying Earth's ocean,
geology, and climate.

GRACE was a collaborative endeavor involving the Center for Space
Research at the University of Texas, Austin; NASA's Jet Propulsion
Laboratory, Pasadena, Calif.; the German Space Agency and Germany's
National Research Center for Geosciences, Potsdam. The Jet Propulsion
Laboratory was responsible for the overall mission management under the
NASA ESSP program.

\includegraphics[width=5.50000in,height=4.40235in]{media/image28.jpg}\\
\emph{Artist's conception of MER on Mars}

\section{Mars Exploration Rover
(2003-2019)}\label{mars-exploration-rover-2003-2019}

\begin{itemize}
\item
  \emph{NASA's Mars Exploration Rover Mission (MER), was a robotic space
  mission involving two rovers exploring the planet Mars.}
\item
  \emph{The mission is managed for NASA by the Jet Propulsion
  Laboratory, which designed, built and is operating the rovers.}
\item
  \emph{The mission is part of NASA's Mars Exploration Program which
  includes three previous successful landers: the two Viking program
  landers in 1976 and Mars Pathfinder probe in 1997.}
\end{itemize}

NASA's Mars Exploration Rover Mission (MER), was a robotic space mission
involving two rovers exploring the planet Mars. The mission is managed
for NASA by the Jet Propulsion Laboratory, which designed, built and is
operating the rovers.

The mission began in 2003 with the sending of the two rovers---MER-A
Spirit and MER-B Opportunity---to explore the Martian surface and
geology. The mission's scientific objective is to search for and study
rocks and soils that indicate past water activity. The mission is part
of NASA's Mars Exploration Program which includes three previous
successful landers: the two Viking program landers in 1976 and Mars
Pathfinder probe in 1997.

The total cost of building, launching, landing and operating the rovers
on the surface for the initial 90-Martian-day (sol) primary mission was
US\$820~million. However, both rovers were able to continue functioning
beyond the initial 90-day mission, and received multiple mission
extensions. The Spirit rover remained operational until 2009, while the
Opportunity rover remained operational until 2018.

\includegraphics[width=5.50000in,height=4.48858in]{media/image29.jpg}\\
\emph{MESSENGER (artist concept)}

\section{MESSENGER (2004--2015)}\label{messenger-20042015}

\begin{itemize}
\item
  \emph{MESSENGER entered orbit around Mercury on March 18, 2011, and it
  reactivated its science instruments on March 24, returning the first
  photo from Mercury orbit on March 29.}
\item
  \emph{The spacecraft flew by Earth once and Venus twice.}
\item
  \emph{MESSENGER (an acronym of MErcury Surface, Space ENvironment,
  GEochemistry, and Ranging) was a robotic spacecraft that orbited the
  planet Mercury, the first spacecraft ever to do so.}
\end{itemize}

MESSENGER (an acronym of MErcury Surface, Space ENvironment,
GEochemistry, and Ranging) was a robotic spacecraft that orbited the
planet Mercury, the first spacecraft ever to do so. The 485-kilogram
(1,069~lb) spacecraft was launched aboard a Delta II rocket in August
2004 to study Mercury's chemical composition, geology, and magnetic
field.

MESSENGER used its instruments on a complex series of flybys that
allowed it to decelerate relative to Mercury using minimal fuel. The
spacecraft flew by Earth once and Venus twice. Then it flew by Mercury
three times, in January 2008, October 2008, and September 2009, becoming
the second mission to reach Mercury, after Mariner 10. MESSENGER entered
orbit around Mercury on March 18, 2011, and it reactivated its science
instruments on March 24, returning the first photo from Mercury orbit on
March 29.

MESSENGER crashed into Mercury on April 30, 2015, after running out of
propellant.

\section{New Frontiers program
(2006--present)}\label{new-frontiers-program-2006present}

\begin{itemize}
\item
  \emph{NASA is encouraging both domestic and international scientists
  to submit mission proposals for the project.}
\item
  \emph{The New Frontiers program is a series of space exploration
  missions being conducted by NASA with the purpose of researching
  several of the Sun's planets including Jupiter, Venus, and the dwarf
  planet Pluto.}
\end{itemize}

The New Frontiers program is a series of space exploration missions
being conducted by NASA with the purpose of researching several of the
Sun's planets including Jupiter, Venus, and the dwarf planet Pluto. NASA
is encouraging both domestic and international scientists to submit
mission proposals for the project.

New Frontiers was built on the approach used by the Discovery and
Explorer Programs of principal investigator-led missions. It is designed
for medium-class missions that could not be accomplished within the cost
and time constraints of the Discovery Program, but are not as large as
Flagship-class missions. There are currently three New Frontiers
missions in progress. New Horizons was launched on January 19, 2006, and
flew by Pluto in July 2015. A flyby of 2014 MU69 will take place in
2019. Juno was launched on August 5, 2011, and entered orbit around
Jupiter on July 4, 2016. OSIRIS-REx, launched on September 8, 2016,
plans on returning a sample to Earth on September 24, 2023, and if
successful, would be the first U.S. spacecraft to do so.

\includegraphics[width=5.50000in,height=5.07380in]{media/image30.jpg}\\
\emph{Artist's impression of the Phoenix spacecraft as it lands on Mars}

\section{Mars Scout Program
(2007--2008)}\label{mars-scout-program-20072008}

\begin{itemize}
\item
  \emph{Phoenix was a lander adapted from the canceled Mars Surveyor
  mission.}
\item
  \emph{The Mars Scout Program was a NASA initiative to send a series of
  small, low-cost robotic missions to Mars, competitively selected from
  proposals by the scientific community.}
\item
  \emph{The 90-day primary mission was successful, and the overall
  mission was concluded on November 10, 2008, after engineers were
  unable to contact the craft.}
\end{itemize}

The Mars Scout Program was a NASA initiative to send a series of small,
low-cost robotic missions to Mars, competitively selected from proposals
by the scientific community. Each Scout project was to cost less than
US\$485~million. The Phoenix lander and MAVEN orbiter were selected and
developed before the program was retired in 2010.

Phoenix was a lander adapted from the canceled Mars Surveyor mission.
Phoenix was launched on August 4, 2007, and landed in the icy northern
polar region of the planet on May 25, 2008. Phoenix was designed to
search for environments suitable for microbial life on Mars and to
research the history of water there. The 90-day primary mission was
successful, and the overall mission was concluded on November 10, 2008,
after engineers were unable to contact the craft. The lander last made a
brief communication with Earth on November 2, 2008.

\includegraphics[width=5.50000in,height=4.20489in]{media/image31.jpg}\\
\emph{Dawn, artist concept}

\section{Dawn (2007--2018)}\label{dawn-20072018}

\begin{itemize}
\item
  \emph{Dawn is the first spacecraft to visit either Vesta or Ceres.}
\item
  \emph{The Dawn mission is managed by NASA's Jet Propulsion
  Laboratory.}
\item
  \emph{In November 2018, NASA reported that Dawn had run out of fuel,
  effectively ending its mission; it will remain in orbit around Ceres,
  but can no longer communicate with Earth.}
\item
  \emph{Dawn is a NASA spacecraft tasked with the exploration and study
  of the asteroid Vesta and the dwarf planet Ceres, the two largest
  members of the asteroid belt.}
\end{itemize}

Dawn is a NASA spacecraft tasked with the exploration and study of the
asteroid Vesta and the dwarf planet Ceres, the two largest members of
the asteroid belt. The spacecraft was constructed with some European
cooperation, with components contributed by partners in Germany, Italy,
and the Netherlands. The Dawn mission is managed by NASA's Jet
Propulsion Laboratory.

Dawn is the first spacecraft to visit either Vesta or Ceres. It is also
the first spacecraft to orbit two separate extraterrestrial bodies,
using ion thrusters to travel between its targets. Previous multi-target
missions using conventional drives, such as the Voyager program, were
restricted to flybys.

Launched on September 27, 2007, Dawn entered orbit around Vesta on July
16, 2011, and explored it until September 5, 2012. Thereafter, the
spacecraft headed to Ceres and started to orbit the dwarf planet on
March 6, 2015. In November 2018, NASA reported that Dawn had run out of
fuel, effectively ending its mission; it will remain in orbit around
Ceres, but can no longer communicate with Earth.

\includegraphics[width=5.50000in,height=3.84895in]{media/image32.jpg}\\
\emph{Lunar Reconnaissance Orbiter, artist concept}

\section{Lunar Reconnaissance Orbiter
(2009)}\label{lunar-reconnaissance-orbiter-2009}

\begin{itemize}
\item
  \emph{The LRO mission is a precursor to future human missions to the
  Moon by NASA.}
\item
  \emph{Launched on June 18, 2009, in conjunction with the Lunar Crater
  Observation and Sensing Satellite (LCROSS), as the vanguard of NASA's
  Lunar Precursor Robotic Program, this is the first United States
  mission to the Moon in over ten years.}
\item
  \emph{LRO and LCROSS are the first missions launched as part of the
  United States's Vision for Space Exploration program.}
\end{itemize}

The Lunar Reconnaissance Orbiter (LRO) is a NASA robotic spacecraft
currently orbiting the Moon on a low 50 km polar mapping orbit.\\
The LRO mission is a precursor to future human missions to the Moon by
NASA. To this end, a detailed mapping program identifies safe landing
sites, locates potential resources on the Moon, characterizes the
radiation environment, and demonstrates new technology.

The probe has made a 3-D map of the Moon's surface and has provided some
of the first images of Apollo equipment left on the Moon.\\
The first images from LRO were published on July 2, 2009, showing a
region in the lunar highlands south of Mare Nubium (Sea of Clouds).

Launched on June 18, 2009, in conjunction with the Lunar Crater
Observation and Sensing Satellite (LCROSS), as the vanguard of NASA's
Lunar Precursor Robotic Program, this is the first United States mission
to the Moon in over ten years.\\
LRO and LCROSS are the first missions launched as part of the United
States's Vision for Space Exploration program.

\section{Mars Science Laboratory
(2011)}\label{mars-science-laboratory-2011}

\begin{itemize}
\item
  \emph{Mars Science Laboratory (MSL) is a NASA mission to land and
  operate a rover named "Curiosity" on the surface of Mars.}
\item
  \emph{On Mars, it is helping to assess Mars' habitability.}
\end{itemize}

Mars Science Laboratory (MSL) is a NASA mission to land and operate a
rover named "Curiosity" on the surface of Mars. It was launched by an
Atlas V rocket on November 26, 2011, and landed successfully on August
6, 2012, on the plains of Aeolis Palus in Gale Crater near Aeolis Mons
(formerly "Mount Sharp"). On Mars, it is helping to assess Mars'
habitability. It can chemically analyze samples by scooping up soil and
drilling rocks using a laser and sensor system.

The "Curiosity" rover is about two times longer and fives times more
massive than the "Spirit" or "Opportunity" Mars Exploration Rovers and
carries more than ten times the mass of scientific instruments.

\section{See also}\label{see-also}

\begin{itemize}
\item
  \emph{Launch Services Program}
\item
  \emph{Science Mission Directorate}
\end{itemize}

Launch Services Program

Science Mission Directorate

\section{References}\label{references}

\section{External links}\label{external-links}

\begin{itemize}
\item
  \emph{Directory of past, present and future missions from the Science
  Mission Directorate}
\end{itemize}

Directory of past, present and future missions from the Science Mission
Directorate
