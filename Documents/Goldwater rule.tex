\textbf{From Wikipedia, the free encyclopedia}

https://en.wikipedia.org/wiki/Goldwater\%20rule\\
Licensed under CC BY-SA 3.0:\\
https://en.wikipedia.org/wiki/Wikipedia:Text\_of\_Creative\_Commons\_Attribution-ShareAlike\_3.0\_Unported\_License

\includegraphics[width=4.20878in,height=5.50000in]{media/image1.jpg}\\
\emph{The original piece in Fact magazine which prompted the
introduction of the Goldwater rule. Likely costing Barry Goldwater a
large number of potential votes, this practice was later deemed
unethical by the APA.}

\section{Goldwater rule}\label{goldwater-rule}

\begin{itemize}
\item
  \emph{The Goldwater rule is the informal name given to section~7 in
  the American Psychiatric Association's (APA) Principles of Medical
  Ethics, which states that it is unethical for psychiatrists to give a
  professional opinion about public figures whom they have not examined
  in person, and from whom they have not obtained consent to discuss
  their mental health in public statements.}
\end{itemize}

The Goldwater rule is the informal name given to section~7 in the
American Psychiatric Association's (APA) Principles of Medical Ethics,
which states that it is unethical for psychiatrists to give a
professional opinion about public figures whom they have not examined in
person, and from whom they have not obtained consent to discuss their
mental health in public statements. It is named after former US Senator
and 1964 presidential candidate Barry Goldwater.

The issue arose in 1964 when Fact published the article "The Unconscious
of a Conservative: A Special Issue on the Mind of Barry Goldwater". The
magazine polled psychiatrists about US Senator Barry Goldwater and
whether he was fit to be president. Goldwater sued magazine editor Ralph
Ginzburg and managing editor Warren Boroson, and in Goldwater v.
Ginzburg (July 1969) received damages totaling \$75,000 (\$512,000
today).

\section{Description}\label{description}

\begin{itemize}
\item
  \emph{Section~7, which appeared in the first edition of the American
  Psychiatric Association's (APA) Principles of Medical Ethics in 1973
  and is still in effect as of 2018{[}update{]}, says:}
\end{itemize}

Section~7, which appeared in the first edition of the American
Psychiatric Association's (APA) Principles of Medical Ethics in 1973 and
is still in effect as of 2018{[}update{]}, says:

\section{American Psychological
Association}\label{american-psychological-association}

\begin{itemize}
\item
  \emph{The APA Ethics Code of the American Psychological Association, a
  different organization than the American Psychiatric Association, also
  supports a similar rule.}
\end{itemize}

The APA Ethics Code of the American Psychological Association, a
different organization than the American Psychiatric Association, also
supports a similar rule. In 2016, in response to the New York Times
article "Should Therapists Analyze Presidential Candidates?", American
Psychological Association President Susan H. McDaniel published a letter
in The New York Times in which she stated:

\section{American Medical
Association}\label{american-medical-association}

\begin{itemize}
\item
  \emph{In the fall of 2017, the American Medical Association's Council
  on Ethical and Judicial Affairs wrote new guidelines into the AMA Code
  of Medical Ethics, stating that physicians should refrain "from making
  clinical diagnoses about individuals (e.g., public officials,
  celebrities, persons in the news) they have not had the opportunity to
  personally examine."}
\end{itemize}

In the fall of 2017, the American Medical Association's Council on
Ethical and Judicial Affairs wrote new guidelines into the AMA Code of
Medical Ethics, stating that physicians should refrain "from making
clinical diagnoses about individuals (e.g., public officials,
celebrities, persons in the news) they have not had the opportunity to
personally examine."

\section{Violations}\label{violations}

\section{Regarding Donald Trump}\label{regarding-donald-trump}

\begin{itemize}
\item
  \emph{The American Psychoanalytic Association (APsaA), a different
  organization from the APA, sent a letter on June 6, 2017, that
  highlighted differences between the APA and APsaA ethical guidelines,
  stating that "The American Psychiatric Association's ethical stance on
  the Goldwater Rule applies to its members only.}
\end{itemize}

In 2016 and 2017, a number of psychiatrists and clinical psychologists
faced criticism for violating the Goldwater rule, as they claimed that
Donald Trump displayed "an assortment of personality problems, including
grandiosity, a lack of empathy, and 'malignant narcissism'", and that he
has a "dangerous mental illness", despite having never examined him.

John Gartner, a practicing psychologist, and the leader of the group
Duty to Warn, stated in April 2017 that: "We have an ethical
responsibility to warn the public about Donald Trump's dangerous mental
illness."

The American Psychoanalytic Association (APsaA), a different
organization from the APA, sent a letter on June 6, 2017, that
highlighted differences between the APA and APsaA ethical guidelines,
stating that "The American Psychiatric Association's ethical stance on
the Goldwater Rule applies to its members only. APsaA does not consider
political commentary by its individual members an ethical matter." In
July 2017, the website Stat published an article by Sharon Begley,
labeled "exclusive" and titled "Psychiatry Group Tells Members They Can
Defy 'Goldwater Rule' and Comment on Trump's Mental Health". The
article, with a photograph of Barry Goldwater as the headline image,
stated that "A leading psychiatry group has told its members they should
not feel bound by a longstanding rule against commenting publicly on the
mental state of public figures", first sourcing the statement to the
July 6 American Psychoanalytic Association (APsaA) letter, but also
claiming that it "represents the first significant crack in the
profession's decades-old united front aimed at preventing experts from
discussing the psychiatric aspects of politicians' behavior"; the
article then repeatedly referred to the "Goldwater rule", quoted an
unnamed source as saying "leadership has been extremely reluctant to
make a statement and publicly challenge the American Psychiatric
Association", and claimed that an unnamed "official" had said that
"Although the American Psychological Association 'prefers' that its
members not offer opinions on the psychology of someone they have not
examined, it does not have a Goldwater rule and is not considering
implementing one". Yahoo News reporter Michael Walsh criticized the Stat
article, saying it was "misleading" by stating that the letter
"represents the first significant crack": The American Psychiatric
Association retains the Goldwater rule, and the APsaA never had the rule
and was not changing. Also, even though the APsaA has no Goldwater rule
for its members and allows its members to give individual opinions about
specific political figures, its Executive Councilors unanimously
endorsed a policy that "the APsaA as an organization will speak to
issues only, not about specific political figures".

\section{See also}\label{see-also}

\begin{itemize}
\item
  \emph{Duty to warn §~Clinical psychology and psychiatry}
\end{itemize}

Duty to warn §~Clinical psychology and psychiatry

Bandy X. Lee

\section{References}\label{references}
