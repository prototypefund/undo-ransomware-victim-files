\textbf{From Wikipedia, the free encyclopedia}

https://en.wikipedia.org/wiki/First\%20100\%20days\%20of\%20Donald\%20Trump\%27s\%20presidency\\
Licensed under CC BY-SA 3.0:\\
https://en.wikipedia.org/wiki/Wikipedia:Text\_of\_Creative\_Commons\_Attribution-ShareAlike\_3.0\_Unported\_License

\section{First 100 days of Donald Trump's
presidency}\label{first-100-days-of-donald-trumps-presidency}

\begin{itemize}
\item
  \emph{The 100th day of his presidency was April 29, 2017.}
\item
  \emph{Trump's approval among his base was high, with 96\% of those who
  voted for him saying in an April 2017 poll that they would vote for
  him again.}
\item
  \emph{The first 100 days of Donald Trump's presidency began on January
  20, 2017, the day Donald Trump was inaugurated as the 45th President
  of the United States.}
\end{itemize}

The first 100 days of Donald Trump's presidency began on January 20,
2017, the day Donald Trump was inaugurated as the 45th President of the
United States. The first 100 days of a presidential term took on
symbolic significance during Franklin D. Roosevelt's first term in
office, and the period is considered a benchmark to measure the early
success of a president. The 100th day of his presidency was April 29,
2017. Trump first announced his plan for the first hundred days of his
presidency in Gettysburg, Pennsylvania, on October 23, 2016, before the
election.

One of Trump's major accomplishments, made as part of a "100-day
pledge", was the confirmation of Neil Gorsuch as an Associate Justice of
the Supreme Court of the United States. Structurally, President Trump
had the advantage of a Republican Party majority in the U.S. House of
Representatives and the Senate, but was unable to fulfill his major
pledges in his first 100 days and had an approval rating of between 40
and 42 percent, "the lowest for any first-term president at this point
in his tenure". Although he tried to make progress on one of his key
economic policies---the dismantling of the Dodd--Frank Wall Street
Reform and Consumer Protection Act---his failure to repeal the Patient
Protection and Affordable Care Act (ACA) in the first 100 days was a
major setback. He reversed his position on a number of issues including
labeling China as a currency manipulator, NATO, launching the 2017
Shayrat missile strike without congressional approval, the North
American Free Trade Agreement (NAFTA), renomination of Janet Yellen as
Chair of the Federal Reserve, and the nomination of Export-Import Bank
directors. Supporters claimed that as the first person in history to
have been elected President who has never held any military, political,
or government office of any type, he therefore faced a steep learning
curve. Trump's approval among his base was high, with 96\% of those who
voted for him saying in an April 2017 poll that they would vote for him
again.

Near the end of the 100 days, the Trump administration introduced a
broad outline of a sweeping tax reform focusing on deep tax cuts. While
it is intended to encourage economic growth, there were concerns from
some members of the United States Congress about raising the national
deficit. In spite of the sharp decline in gross domestic product (GDP)
growth in the first quarter of 2017---representing the weakest quarterly
economic growth in three years---the S\&P 500 was near an all-time high,
representing a 12\% rise from the first quarter of 2016, as investor
confidence remained elevated. Although Trump had to concede to delay
funding for the U.S.--Mexico border wall he had promised, narrowly
avoiding a government shutdown a few days before the end of the first
100 days, his rhetoric may have contributed to a sharp drop in the
number of illegal crossings at the Mexico--United States border.

Trump signed 24 executive orders in his first 100 days, the most
executive orders of any President since World War II. He also signed 22
presidential memoranda, 20 presidential proclamations, and 28 bills.
About a dozen of those bills roll-back regulations finalized during the
last months of his immediate predecessor Barack Obama's presidency using
the Congressional Review Act. Most of the other bills are "small-scale
measures that appoint personnel, name federal facilities or modify
existing programs." None of Trump's bills are considered to be "major
bills"---based on a "longstanding political-science standard for 'major
bills'." Presidential historian Michael Beschloss said that "based on a
legislative standard"---which is what the first 100 days has been judged
on since the tenure of President Franklin D. Roosevelt, who enacted 76
laws in 100 days including nine that were "major"---"Trump is really
pretty low down on the list."

\section{Pledges}\label{pledges}

\begin{itemize}
\item
  \emph{Trump pledged to do the following in his first 100 days:}
\end{itemize}

Trump pledged to do the following in his first 100 days:

Appoint judges "who will uphold the Constitution" and "defend the Second
Amendment"

Construct a wall on the southern U.S. border and limit illegal
immigration "to give unemployed Americans an opportunity to fill
good-paying jobs"

Re-assess trade agreements with other nations and "crack down" on
companies that send jobs overseas

the Affordable Care Act or ObamacareRepeal and replace the Patient
Protection and Affordable Care Act (commonly called the Affordable Care
Act or Obamacare)

Remove federal restrictions on energy production

Push for an amendment to the United States Constitution imposing term
limits on Congress

Eliminate gun-free zones

Formulate a rule on regulations "that for every one new regulation, two
old regulations must be eliminated"

Instruct the chairman of the Joint Chiefs of Staff to "develop a
comprehensive plan to protect America's vital infrastructure from
cyberattacks, and all other form of attacks."

Label China a "currency manipulator"

Enforce rules and regulations for China's unfair subsidy behavior.
Instruct the U.S. trade representative to bring trade cases against
China, both in U.S. and at the WTO.

Use every lawful presidential power to remedy trade disputes, including
the application of 45\% tariffs consistent with Section 201 and 301 of
the Trade Act of 1974, and Section 232 of the Trade Expansion Act of
1962 to stop China's illegal activities, including its theft of American
trade secrets.

\section{Inauguration}\label{inauguration}

\begin{itemize}
\item
  \emph{The first 100 days began with the inauguration on January 20,
  2017, at 12:00 pm.}
\item
  \emph{As Trump took the oath of office, the official @POTUS Twitter
  account switched to President Trump with previous tweets archived
  under @POTUS44.}
\end{itemize}

The first 100 days began with the inauguration on January 20, 2017, at
12:00 pm. This was the third presidential online portal transition and
the first to transition social media accounts such as Twitter. As Trump
took the oath of office, the official @POTUS Twitter account switched to
President Trump with previous tweets archived under @POTUS44. All 13
million followers of the POTUS account during Obama's administration
slowly transitioned{[}clarification needed{]} as well.

\section{Cabinet}\label{cabinet}

\begin{itemize}
\item
  \emph{Steve Mnuchin, who was nominated by Trump in November 2016, was
  finally confirmed on February 13, 2017, as Secretary of the Treasury
  department after lengthy confirmation hearings.}
\item
  \emph{On February 8, when Trump formally announced his
  24-member-cabinet---the largest cabinet of any President so
  far---fewer cabinet nominees had been confirmed than any other
  president except George Washington by the same length of time into his
  presidency.}
\item
  \emph{He was approved by the Senate Foreign Relations Committee on
  January 23, 2017, and by the full Senate in a 56--43 vote.}
\end{itemize}

On February 8, when Trump formally announced his 24-member-cabinet---the
largest cabinet of any President so far---fewer cabinet nominees had
been confirmed than any other president except George Washington by the
same length of time into his presidency. Trump's reorganization of the
cabinet removed the Chair of the Council of Economic Advisers that
President Obama had added in 2009. The Director of National Intelligence
and Director of the CIA were elevated to cabinet-level. During the
transition period, Trump had named a full slate of Cabinet and
Cabinet-level nominees, all of which require Senate confirmation except
for White House Chief of Staff and the vice presidency.

By April 29, almost all of his nominated cabinet members had been
confirmed, including Secretaries of State Rex Tillerson, Treasury Steven
Mnuchin, Defense James Mattis, Justice Jeff Sessions, the Interior Ryan
Zinke, Secretary of Agriculture Sonny Perdue, Commerce Wilbur Ross,
Secretary of Labor Alex Acosta, Health and Human Services HHS Tom Price,
Housing and Urban Development HUD Ben Carson, Transportation Elaine
Chao, Energy Rick Perry, Education Betsy DeVos, Veterans Affairs David
Shulkin, Homeland Security John Kelly, Director of the Central
Intelligence Agency (CIA) Mike Pompeo, UN Ambassador Nikki R. Haley,
Environmental Protection Agency (EPA) Scott Pruitt, Small Business
Administration Linda McMahon, Management and Budget OMB Mick Mulvaney,
and Director of National Intelligence Dan Coats. Only two were awaiting
confirmation---Trade Representative Robert Lighthizer and Council of
Economic Advisers CEA Kevin Hassett.

James Mattis was confirmed on January 20 as Secretary of Defense by a
vote of 98--1. Mattis had previously received a waiver of the National
Security Act of 1947, which requires a seven-year waiting period before
retired military personnel can assume the role of Secretary of Defense.
John Kelly was confirmed as United States Secretary of Homeland Security
on the first day by a vote of 88--11. Former ExxonMobil CEO Rex
Tillerson was sworn in as Secretary of State by Vice-President Mike
Pence on February 1. Trump nominated Tillerson for the position as top
U.S. diplomat (the equivalent of a foreign minister) on December 13,
2016. He was approved by the Senate Foreign Relations Committee on
January 23, 2017, and by the full Senate in a 56--43 vote. Nikki Haley
was confirmed as UN Ambassador with a Senate vote of 96 to 4.

On January 26, 2017, when Tillerson visited the United States State
Department, Undersecretaries Joyce Anne Barr, Patrick F. Kennedy,
Michele Bond, and Gentry O. Smith all simultaneously resigned from the
department. Former State Department chief of staff David Wade called the
resignations "the single biggest simultaneous departure of institutional
memory that anyone can remember." The Trump administration told CNN the
officials had been fired and the Chicago Tribune reported that several
senior state department career diplomats left the State Department,
claiming they "had been willing to remain at their posts but had no
expectation of staying."

On February 10, Tom Price was confirmed as Secretary of Health and Human
Services (HHS), a "\$1 trillion government department". HHS includes
National Institutes of Health (NIH), the Centers for Disease Control and
Prevention, and the Food and Drug Administration (FDA). Price, who is a
vocal opponent of the Affordable Care Act, will oversee its repeal and
replacement. He has published articles in the "small, conservative
medical association", the Association of American Physicians and
Surgeons, to which he belongs, that opposes mandatory vaccination and
continue to argue that the vaccines causes autism, a "discredited
conspiracy theory that Trump has long espoused". In response to
questions from Senators at the hearing as to whether he believes autism
is caused by vaccines, he responded, "I think the science in that
instance is that it does not".

Steve Mnuchin, who was nominated by Trump in November 2016, was finally
confirmed on February 13, 2017, as Secretary of the Treasury department
after lengthy confirmation hearings.

On February 16, the Senate voted 54 to 46 to advance Scott Pruitt's
nomination as Administrator of the Environmental Protection Agency. On
February 16, a District Court Judge in Oklahoma, Aletia Timmons, ordered
Pruitt to "turn over thousands of emails related to his communication
with the oil, gas and coal industry" in a case brought to court by the
Center for Media and Democracy. Lawmakers had criticized Pruitt who sued
the EPA 14 times on behalf of the State of Oklahoma.

Trump nominated Alexander Acosta as Secretary of Labor on February 16,
when his first nominee Andrew Puzder stepped down under a wave of
criticism for having employed an illegal immigrant as a former
housekeeper, for his "remarks on women and employees at his restaurants"
and for his "rancorous 1980s divorce".

\section{Notable non-Cabinet
positions}\label{notable-non-cabinet-positions}

\begin{itemize}
\item
  \emph{Trump's 36-year-old son-in-law, Jared Kushner is Trump's Senior
  Advisor alongside Stephen Miller.}
\item
  \emph{On February 20, 2017, Trump named "warrior-scholar deemed an
  expert in counter insurgency", Lieutenant General H. R. McMaster, to
  replace Flynn as National Security Advisor.}
\item
  \emph{Michael T. Flynn served as Trump's National Security Advisor
  from January 20 until his resignation on February 13, 2017.}
\end{itemize}

According to a database compiled by the Washington Post in collaboration
with the Partnership for Public Service, as of April 27, 473 of the 554
key executive branch nominations that require Presidential nomination
and Senate confirmation, had not yet been appointed, including "Cabinet
secretaries, deputy and assistant secretaries, chief financial officers,
general counsel, heads of agencies, ambassadors and other critical
leadership positions." Only three of the 119 Department of State
executive branch positions have been filled and only one position in the
Department of Defense---the Secretary of Defense, James Mattis---has
been filled out of 53 key positions. Trump has not yet nominated anyone
for 49 of these positions. On February 28, in an exclusive interview
Tuesday with Fox \& Friends, said, "a lot of those jobs, I don't want to
appoint, because they're unnecessary to have....You know, we have so
many people in government, even me. I look at some of the jobs and it's
people over people over people. I say, 'What do all these people do?'
You don't need all those jobs...Many of those jobs I don't want to fill.
I say, isn't that a good thing? That's not a bad thing. That's a good
thing. We're running a very good, efficient government."

Prior to taking office, Trump named several important White House
advisers to positions that do not require Senate confirmation, including
Stephen K. Bannon as his "senior counselor and chief West Wing
strategist" and Reince Priebus as Chief of Staff, with a mission "as
equal partners to transform the federal government." Other important
advisers outside of the Cabinet include (Counselor to the President)
Kellyanne Conway, Senior Advisor (National Security Advisor) Michael
Flynn (replaced by H. R. McMaster) and (National Trade Council) Director
Peter Navarro. (Homeland Security Adviser) Thomas P. Bossert,
(Regulatory Czar) Carl Icahn, (White House Counsel) Donald F. "Don"
McGahn II, and (Press Secretary) Sean Spicer.

Michael T. Flynn served as Trump's National Security Advisor from
January 20 until his resignation on February 13, 2017. He set a record
for the shortest tenure as National Security Advisor in American
history. The Justice Department warned the Trump administration that
Flynn, who had a "well-established history with Russia", may have been
"vulnerable to blackmail by Moscow." Flynn had "mischaracterized his
communications" with Russian Ambassador Sergey Kislyak to Vice President
Mike Pence regarding U.S. sanctions on Russia. Flynn's phone calls had
been "recorded by a government wiretap" and several days after Flynn was
named as Trump's Advisor, Sally Yates, who was then acting attorney
general, warned the White House that "Flynn was susceptible to blackmail
by the Russians because he had misled Mr. Pence and other officials".
According to a February 14 article by The New York Times, it was unclear
why the White House did not react to Yates' warning in early January.
There were also questions about how much was known in early January by
Bannon, Pence, Spicer, and Trump. Yates was fired on January 30, in an
unrelated incident.

On February 20, 2017, Trump named "warrior-scholar deemed an expert in
counter insurgency", Lieutenant General H. R. McMaster, to replace Flynn
as National Security Advisor. Trump overruled McMaster's attempt to
replace 30-year-old NSC aide Ezra Cohen-Watnick, a Mike Flynn appointee,
with Linda Weissgold, when Bannon and Kushner intervened on
Cohen-Watnick's behalf on March 11--12. Cohen-Watnick gathered
classified files on intelligence information on U.S. persons.

On January 28, 2017, Trump signed a Memorandum, the Organization of the
National Security Council and the Homeland Security Council which
restructured the Principals Committee---the senior policy committee---of
the National Security Council, assigning a permanent invitation to Steve
Bannon, White House Chief Strategist, while at the same time withdrawing
the permanent invitations of the Chairman of the Joint Chiefs of Staff
and Director of National Intelligence. On April 5, the 75th day of
Trump's presidency, under guidance from Army Lieutenant General H. R.
McMaster, the National Security Advisor (NSC advisor) who replaced Mike
Flynn, Trump removed Bannon, who has no security experience, from the
National Security Council's principals committee.

Trump's 36-year-old son-in-law, Jared Kushner is Trump's Senior Advisor
alongside Stephen Miller. "In his January interview with the Times of
London, Trump said that Kushner would be in charge of brokering peace in
the Israeli--Palestinian conflict. He is also a "top adviser on
relations with Canada, China and Mexico." On April 3, Kushner
accompanied the head of the Joint Chiefs of Staff, General Joseph F.
Dunford Jr. and Homeland Security Advisor Thomas P. Bossert to meet with
Iraqi Prime Minister Haider al-Abadi "to discuss the fight against the
Islamic State and whether the United States would leave troops in Iraq
afterward." Trump named Kushner as head of the White House Office of
American Innovation, (OAI), established on March 29 and mandated to use
ideas from the private-sector to overhaul all federal agencies and
departments in order to "spur job creation". One of the OAI's first
priorities is to modernize the technology of departments such as
Veterans Affairs. In his new position, Kushner will work with Chris
Christie, who will chair the newly established "President's Commission
on Combating Drug Addiction and the Opioid Crisis" in response to
Trump's pledge to combat opioid abuse.

On January 28, in his eleventh Presidential Memoranda, "Organization of
the National Security Council and the Homeland Security Council", White
House Chief Strategist, Steve Bannon, was designated as a regular
attendee to the National Security Council (NSC)′s Principals Committee,
a Cabinet-level senior interagency forum for considering national
security issues, in a departure from the previous format in which this
role is usually held for generals. While at first there was some
confusion over meeting attendees, Priebus clarified on January 30, that
defense officials could attend the meetings. On April 5, the 75th day of
Trump's presidency, under guidance from Army Lieutenant General H. R.
McMaster, the National Security Advisor (NSC advisor) who replaced Mike
Flynn, Trump removed Bannon, who has no security experience, from the
National Security Council's principals committee.

On February 2, Time published an article about Bannon as potentially,
the second most powerful man in the world, illustrated with a cover
labeling him as the "Great Manipulator." After only a fortnight into
Trump's presidency, NPR described Bannon as "the power behind the
throne" and the "gray eminence behind much of what Trump was
prioritizing", rivalling Kushner's and Priebus' roles. Mike Pence
affirmed in a PBS NewsHour report that only Trump was "in charge".

Bannon and Steve Miller have been called the "architects" of the
inaugural address, executive orders, including the controversial travel
and refugees EO, and presidential memoranda.

In an often-cited October 8, 2015, lengthy profile entitled "This Man Is
the Most Dangerous Political Operative in America" by Joshua Green, a
senior national correspondent for Bloomberg News, Green described how
Breitbart News with Bannon at its helm, had "championed Trump's
presidential candidacy" and helped "coalesce a splinter faction of
conservatives" who were irate over the way in which Fox News had treated
Trump. Green quoted then-Senator Jeff Sessions as an admirer of
Breitbart, which was "extraordinarily influential", with many radio
hosts "reading Breitbart every day". Trump cited Breitbart News to
vindicate his claims.

Stephen Miller, Trump's Senior Advisor, was Jeff Sessions'
communications director when he served as Senator for Alabama.
Thirty-one-year old Miller, Bannon, and Andrew Bremberg sent over 200
executive orders to federal agencies for review before January 20.
Miller has been an architect behind the inaugural address and the most
"contentious executive orders" including Executive Order 13769.

In a February 12 interview with ABC News anchor George Stephanopolous,
when asked to provide evidence "for Trump's "unfounded allegations"
where former Senator Kelly Ayotte lost her bid for election, and Trump
narrowly lost to Clinton in 2016, Miller suggested Stephanopolous
interview Kansas Senator, Kris Kobach, who relied upon a 2012 Pew
Research Center study in his voter fraud claims. The day before the
interview a Federal Election Commission Commissioner called on Trump to
provide evidence of what would "constitute thousands of felony criminal
offenses under New Hampshire law."

Gary Cohn, the former Goldman Sachs investment banker and executive,
took office on January 20, as Trump's Director of the National Economic
Council, (NEC), a position which did not require Congressional
confirmation, By February 11, 2017, The Wall Street Journal described
Cohn as an "economic-policy powerhouse" in Trump's administration and
The New York Times called him Trump's "go-to figure on matters related
to jobs, business and growth." While the confirmation of Trump's
December 12, 2016, nominee for Secretary of Treasury, Steven Mnuchin,
was delayed until February 13 by Congressional hearings, Cohn filled in
the "personnel vacuum" and pushed "ahead on taxes, infrastructure,
financial regulation and replacing health-care law." In November, Trump
considered offering Cohn the position as Secretary of Treasury. If Cohn
had stayed at Goldman Sachs, some believed he would have become CEO when
Lloyd Blankfein vacated that office and his \$285 million severance
package "raised eyebrows" according to CNN. Bannon and Cohn disagree on
the border-adjustment tax, the centerpiece of Paul Ryan's controversial
tax reforms presented on February 17, which includes a 20\% import tax,
export subsidies and a 15\% reduction in corporate tax rates that would,
among other things, pay for the Mexican wall, which according to a The
Washington Post study, would cost \$25 billion and which Trump stated
would cost \$12 billion.

\section{Domestic policy}\label{domestic-policy}

\section{United States Domestic Policy
Council}\label{united-states-domestic-policy-council}

\begin{itemize}
\item
  \emph{The Domestic Policy Council (DPC) consists of Trump and Andrew
  Bremberg as Directors with Paul Winfree as Deputy Assistant.}
\end{itemize}

The Domestic Policy Council (DPC) consists of Trump and Andrew Bremberg
as Directors with Paul Winfree as Deputy Assistant. Council attendees
include Mike Pence, Jeff Sessions, Tom Price, John F. Kelly, David
Shulkin, Ryan Zinke, Betsy DeVos, Ben Carson, Elaine Chao, Wilbur Ross,
Rick Perry, Steven Mnuchin, and-when appointed---the Secretary of Labor
and the Secretary of Agriculture. Additional attendees include Scott
Pruitt, Mick Mulvaney (Director of the Office of Management and Budget),
Gary Cohn, and---when appointed---the Chair of the Council of Economic
Advisers and the Director of the Office of National Drug Control Policy.
The Congressional Research Service describes DPC's role as analyses of
domestic policies and social programs including "education, labor and
worker safety; health-care insurance and financing; health services and
research; aging policy studies; Social Security, pensions and disability
insurance; immigration, homeland security, domestic intelligence and
criminal justice; and welfare, nutrition and housing programs."

\section{Withdrawal of the Affordable Care
Act}\label{withdrawal-of-the-affordable-care-act}

\begin{itemize}
\item
  \emph{The order states what Mr. Trump made clear during his campaign:
  that it is his administration's policy to seek the "prompt repeal" of
  Obamacare.}
\item
  \emph{The American Health Care Act of 2017 (AHCA), a bill to repeal
  and replace the ACA, was withdrawn in Congress on March 24, 2017 due
  to lack of support from within the Republican caucus.}
\end{itemize}

Within the first hours of Trump's presidency, he signed his first
executive order, Minimizing the Economic Burden of the Patient
Protection and Affordable Care Act Pending Repeal (EO 13765) to fulfill
part of his pledge to repeal the Patient Protection and Affordable Care
Act (ACA), which was part of a series of steps taken prior to 2017 to
repeal and defund the ACA, including most recently, the FY2017 budget
resolution, S.Con.Res. 3, that contained language allowing the repeal of
ACA through the budget reconciliation process. A CBO report estimated 18
million people would lose their insurance and premiums would rise by
20\% to 25\% in the first year after repealing Obamacare. Uninsured
could reach 32 million by 2026, while premiums could double. The order
states what Mr. Trump made clear during his campaign: that it is his
administration's policy to seek the "prompt repeal" of Obamacare. During
his Fox News interview with Bill O'Reilly airing before the Super Bowl,
Trump announced that the timeline for replacing Obamacare had to be
extended and that a replacement would probably not be ready until 2018.
Republicans are limited as to how much of ACA they can undo as they do
not have a 60-vote majority in the Senate. They also "must balance the
interests of insurers and medical providers". According to the March 13,
2017 report by the nonpartisan Congressional Budget Office and staff of
the Joint Committee on Taxation (JCT) on the budgetary impact of the
Republican bill to repeal and replace ACA over the coming decade, there
would be a \$337 billion reduction in the federal deficit and an
estimated loss of coverage to 24 million more Americans. The Republican
health-care plan was unveiled on March 6 and faced opposition from both
moderate and conservative Republicans, such as the House Freedom Caucus.
The American Health Care Act of 2017 (AHCA), a bill to repeal and
replace the ACA, was withdrawn in Congress on March 24, 2017 due to lack
of support from within the Republican caucus.

\section{Immigration policy}\label{immigration-policy}

\begin{itemize}
\item
  \emph{By April 3, according to ICE, there had been 35,604 removals in
  January and February 2017 compared to 35,255 in the same period in
  2016.}
\item
  \emph{Some of these "dreamers" in interviews with The Associated Press
  on April 21, said they "were not comforted by Trump's pledge"
  particularly since the April 18 deportation of 23-year-old "dreamer",
  Juan Manuel Montes.}
\item
  \emph{In an AP April 20 interview, Trump said that, "The dreamers
  should rest easy".}
\end{itemize}

In his first 100 days, President Trump set the tone of harsh immigration
policies, by signing executive orders to set in motion travel bans and
restrictions on refugees and immigrants from Muslim-majority countries,
increased immigration enforcement including deportations, and expanded
efforts to prevent illegal entry into the United States by building a
wall along the Mexico--United States border. While the numbers of people
deported were very similar to those in 2016, the categories of people
targeted for deportations was broadened during this period, which meant
that many more people are at a heightened risk of deportation. Secretary
Kelly clarified that Immigration and Customs Enforcement (ICE) "will no
longer exempt classes or categories of removable aliens from potential
enforcement." By April 3, according to ICE, there had been 35,604
removals in January and February 2017 compared to 35,255 in the same
period in 2016. But the "tough rhetoric" and some "high-profile Ice
operations" widely cited in the media resulted in widespread fear and
panic within immigrant communities.

In an AP April 20 interview, Trump said that, "The dreamers should rest
easy". There are 800,000 young people protected by Obama's "Deferred
Action for Childhood Arrivals" (DREAMERS) who came to the U.S. as
children and are living there illegally. Some of these "dreamers" in
interviews with The Associated Press on April 21, said they "were not
comforted by Trump's pledge" particularly since the April 18 deportation
of 23-year-old "dreamer", Juan Manuel Montes. Trump supporters who are
"immigration hard-liners", such as NumbersUSA and Mark Krikorian of the
Center for Immigration Studies, feel deceived by Trump's softening
stance on DREAMERs arguing that "{[}h{]}is promise on DACA was pretty
clear and unequivocal".

\section{Travel ban and refugee
suspension}\label{travel-ban-and-refugee-suspension}

\begin{itemize}
\item
  \emph{On the evening of January 30, Trump replaced acting Attorney
  General Sally Yates with Dana Boente.}
\item
  \emph{The restraining order was upheld by the United States Court of
  Appeals for the Ninth Circuit on February 9, 2017.}
\item
  \emph{Trump also replaced DHS's ICE Chief Daniel Ragsdale with Thomas
  Homan as Acting Director in the evening of January 30.}
\end{itemize}

On January 27, at 4:42 p.m EST, Trump signed Executive Order 13769,
entitled "Protecting the Nation From Terrorist Attacks by Foreign
Nationals" which temporarily suspends the U. S. Refugee Admissions
Program (USRAP) for 120 days and denies entry to citizens of Iraq, Iran,
Libya, Somalia, Sudan, Syria and Yemen for 90 days. The suspension for
Syrian refugees is for an indefinite period of time. The Economist
described the order as "drafted in secret, enacted in haste and unlikely
to fulfill its declared aim of sparing America from terrorism" with
"Republican allies" lamenting that a "fine, popular policy was marred by
its execution." Notably Saudi Arabia was not on the list though most of
the 9/11 hijackers were from there. See Provisions of Order 13769.

On February 4, the U.S. Department of Homeland Security and the State
Department suspended all actions to implement the week-old EO in
response to the February 3 ruling by federal judge James Robart which
blocked the EO. According to CNN and the Los Angeles Times, the
architects behind the order, were Stephen Miller and Steve Bannon. White
House officials deny that it was written without input from the U.S.
Department of Justice's Office of Legal Counsel (OLC). It was argued
that these 7 countries ranked among the lowest 15 out of the 104
countries evaluated by the Henley \& Partners Visa Restrictions Index in
2016 based on the "number of countries that their citizens can travel to
visa-free." For example, Germany ranks the highest at 177 points,
Afghanistan the lowest of all 104 at 25.:3 The order also calls for an
expedited completion and implementation of the Biometric Entry-Exit
Tracking System for all travelers coming into the United States. The
first legal challenge against the EO was filed on January 28, and within
two days there were dozens of ongoing lawsuits in the United States
federal courts. By February 3, federal judge, James Robart temporarily
blocked the week-old EO which opened American airports to visa holders
from the seven targeted countries. At the international level legal
concerns have been raised by the UN, Zeid Ra'ad al Hussein, who claimed
that "discrimination on nationality alone is forbidden under human
rights law." On January 30, in a telephone call to Trump, German
Chancellor Angela Merkel explained that his EO "ran counter to the
duties of all signatory states" to the Geneva Refugee Convention "to
take in war refugees on humanitarian grounds".

Thousands protested at airports and other locations throughout the
United States. Critics of the ban include most Democrats and several top
Republican Congressmen, former President Obama, the Council on
American--Islamic Relations, over a dozen state attorneys general,
thousands of academics, Nobel laureates, technology companies Iran,
France, Germany, and 800,000 petitioners in Britain. Supporters of the
ban include 82\% of GOP voters, Paul Ryan, Bob Goodlatte, Czech
President Miloš Zeman, and members of the European far right. According
to an IPSOS online poll conducted on January 31, in response to the
question, "Do you agree or disagree with the Executive Order that
President Trump signed blocking refugees and banning people from seven
Muslim majority countries from entering the U.S.?", 48\% of the 1,201
Americans polled agreed with the statement (23\% of the 453 Democrats,
82\% of the 478 Republicans, and 44\% of the Independents polled).

On the evening of January 30, Trump replaced acting Attorney General
Sally Yates with Dana Boente. Spicer's statement described Yates as an
"Obama administration appointee" who had "betrayed the Department of
Justice" by "refusing to enforce a legal order." In the Senate, Chuck
Schumer, called her firing a Monday Night Massacre in reference to
Nixon's firing of his attorney general, referred to as the Saturday
Night Massacre during Watergate. Trump also replaced DHS's ICE Chief
Daniel Ragsdale with Thomas Homan as Acting Director in the evening of
January 30.

In a live interview with Chris Wallace on January 29, Fox News Sunday,
Kellyanne Conway, justified the list of 7 countries by claiming that the
countries were originally identified as a threat in the Terrorist
Prevention Act passed by Congress in 2015. HUD's Visa Waiver Program
Improvement and Terrorist Travel Prevention Act of 2015, was extended
amid some controversy in February 2016, when it revoked the privilege of
traveling to the States without a visa to people who "had recently
traveled to Iraq, Syria, Iran or Sudan," as they were considered to be
high-risk. A spokesman for former president Obama issued a statement
stating, "The president {[}Obama{]} fundamentally disagrees with the
notion of discriminating against individuals because of their faith or
religion," In his final press statement as president, Obama stated,
"There's a difference between {[}the{]} normal functioning of politics
and certain issues or certain moments where I think our core values may
be at stake," and stated his intention to speak out if a situation is
serious enough. Obama encouraged Americans to protest the issue.

In response to a temporary restraining order (TRO) issued in the case of
State of Washington v. Trump, the Department of Homeland Security said
on February 4 that it had stopped enforcing the portions of the
executive order affected by the judgment, while the State Department
activated visas that had been previously suspended. The restraining
order was upheld by the United States Court of Appeals for the Ninth
Circuit on February 9, 2017.

On March 15, a United States Federal Judge, Derrick Watson of the
District Court of Hawaii, issued a 43-page ruling which blocked Trump's
revised March 6 executive order 13780 on the grounds that it violated
the First Amendment's Establishment Clause by disfavoring a particular
religion. The temporary restraining order was converted to a preliminary
injunction by Judge Watson on March 29. On an April 18 airing of the
Mark Levin Show Jeff Sessions commented, "We are confident that the
President will prevail on appeal and particularly in the Supreme Court,
if not the Ninth Circuit. So this is a huge matter. I really am amazed
that a judge sitting on an island in the Pacific can issue an order that
stops the President of the United States from what appears to be clearly
his statutory and Constitutional power."

\section{High-profile ICE operations}\label{high-profile-ice-operations}

\begin{itemize}
\item
  \emph{On February 8, 2017, Immigration and Customs Enforcement (ICE)
  agents arrested 35-year-old Guadalupe García de Rayos, when she
  attended her required annual review at the ICE office in Phoenix, and
  deported her to Mexico on the next day based on a removal order issued
  in 2013 by the Executive Office for Immigration Review.}
\item
  \emph{ICE officials in Los Angeles released a report on February 10,
  2017, that about 160 foreign nationals were arrested in a five-day
  operation.}
\end{itemize}

On February 8, 2017, Immigration and Customs Enforcement (ICE) agents
arrested 35-year-old Guadalupe García de Rayos, when she attended her
required annual review at the ICE office in Phoenix, and deported her to
Mexico on the next day based on a removal order issued in 2013 by the
Executive Office for Immigration Review. Immigrant advocates believe
that she is the first to be deported after the EO was signed and that
her deportation "reflects the severity" of the "crackdown" on illegal
immigration. ICE officials said that her case went through multiple
reviews in the immigration court system and that the "judges held she
did not have a legal basis to remain in the US". In 2008, she was
working at an amusement park in Mesa, Arizona when then-Sheriff Joe
Arpaio ordered a raid that resulted in her arrest and felony identity
theft conviction for possessing a false Social Security number. Arpaio
was a subject of several controversies during his tenure as sheriff. In
2015, the U.S. Department of Justice partially settled a lawsuit filed
against Arpaio for unlawful discriminatory police conduct, alleging that
Arpaio had overseen the worst pattern of racial profiling in U.S.
history. ICE officials in Los Angeles released a report on February 10,
2017, that about 160 foreign nationals were arrested in a five-day
operation. Of those, 150 had criminal histories, and of the remaining
arrests, five had final orders of removal or were previously deported.
Ninety-five percent were male. Under Trump's EO, the definition of
criminal is much more "sweeping" than Obama's, which "prioritized
expulsion of undocumented immigrants who threatened public safety or
national security, had ties to criminal gang activity, committed serious
felony offenses or were habitual misdemeanor criminal offenders" and a
single immigration officer decides. On the morning of February 14, ICE
officials entered the Des Moines, Washington family home of 23-year-old
Daniel Ramirez Medina on an arrest warrant for Ramirez' father, who was
taken into custody. Ramirez, who has no criminal record, entered the
United States illegally as a child, and was later able to get a legal
work permit through the 2012 Deferred Action for Childhood Arrivals
(DACA) policy, was placed in detention in the Northwest Detention
Center, Tacoma, Washington. According to ICE, Ramirez was detained based
on "his admitted gang affiliation and risk to public safety". According
to Ramirez's lawyer, Ramirez "unequivocally denies" these allegations
and claimed ICE agents "repeatedly pressured" Ramirez to "falsely admit"
gang "affiliation." "The case raises questions about what it could mean
for Dreamers, undocumented immigrants who were brought to the United
States as children."

\includegraphics[width=5.50000in,height=3.92857in]{media/image1.jpg}\\
\emph{President Trump signs an executive order at a ceremony at DHS
headquarters}

\section{U.S.--Mexico border wall
proposal}\label{u.s.mexico-border-wall-proposal}

\begin{itemize}
\item
  \emph{While visiting the Department of Homeland Security (DHS) on
  January 25, President Trump signed his third executive order Border
  Security and Immigration Enforcement Improvements (EO 13767) under the
  (INA), the Secure Fence Act, and the (IIRIRA) for the construction of
  a Mexican border wall to deter illegal migration and smuggling of
  illegal products.}
\item
  \emph{The Washington Post reported on April 25, that Trump had agreed
  to delaying funding for the construction of the wall until September
  to avoid a government shutdown.}
\item
  \emph{Trump plans on eventually negotiating a reimbursement from the
  Mexican government.}
\end{itemize}

While visiting the Department of Homeland Security (DHS) on January 25,
President Trump signed his third executive order Border Security and
Immigration Enforcement Improvements (EO 13767) under the (INA), the
Secure Fence Act, and the (IIRIRA) for the construction of a Mexican
border wall to deter illegal migration and smuggling of illegal
products. The existing Mexico--United States barrier is not one
continuous structure, but a series of physical walls and physical and
"virtual" fences monitored by the United States Border Patrol. The
proposed wall which would be "a contiguous, physical wall or other
similarly secure, contiguous, and impassable physical barrier" along the
entire length of the border, which Trump estimated in 2016 would cost
\$10 billion to \$12 billion, and by January 27 was estimated to be \$20
billion, to be initially paid by Congress. Trump plans on eventually
negotiating a reimbursement from the Mexican government. While the
Executive Order entitled "Border Security and Immigration Enforcement
Improvements", contains no information of payment, it requests federal
agency reports by late March 2017 which "identify and quantify all
sources of direct and indirect Federal aid or assistance to the
Government of Mexico on an annual basis over the past five years,
including all bilateral and multilateral development aid, economic
assistance, humanitarian aid, and military aid."

On January 27, Forbes cautioned that the 20\% Mexican Import Tariff on
all imported goods announced by Spicer to pay for the 1,933-mile
(3,111~km) frontier wall would be "paid by Americans". GOP donors,
Brothers Charles and David Koch, and their advocacy group, Americans For
Prosperity, oppose Paul Ryan's 'Buy American' Tax Plan, which they claim
would add a "whopping tax hike of more than \$1 trillion on American
families and small businesses over 10 years." The import tariff would
raise prices at Wal-Mart, for example, directly impacting lower income
families.

The Washington Post reported on April 25, that Trump had agreed to
delaying funding for the construction of the wall until September to
avoid a government shutdown.

\section{Sanctuary cities}\label{sanctuary-cities}

\begin{itemize}
\item
  \emph{On January 25, Trump signed an executive order, "Enhancing
  Public Safety in the Interior of the United States", to the Secretary
  of Homeland Security and the Attorney General and their departments
  and agencies to increase the enforcement of immigration laws which
  included the hiring of 10,000 "additional immigration officers."}
\item
  \emph{Jeff Sessions is considered to be an "inspiration" for Trump's
  anti-immigration policies.}
\end{itemize}

On January 25, Trump signed an executive order, "Enhancing Public Safety
in the Interior of the United States", to the Secretary of Homeland
Security and the Attorney General and their departments and agencies to
increase the enforcement of immigration laws which included the hiring
of 10,000 "additional immigration officers." His order requires the
cooperation of state and local authorities. The order states "sanctuary
jurisdictions" including "sanctuary cities" who refuse to comply will
not be "eligible to receive Federal grants, except as deemed necessary
for law enforcement purposes by the Attorney General or the Secretary".
Some officials claim that the "U.S. Constitution bars the federal
government from commandeering state officials or using federal funds to
"coerce" states into doing the bidding of Washington." Mayors of New
York, Boston, Denver, Los Angeles, San Francisco, and Seattle have
expressed concerns about the Order and do not want to "change the way
their cities treat immigrants." Jeff Sessions is considered to be an
"inspiration" for Trump's anti-immigration policies. On August 31, 2016,
Trump laid out a 10-step plan as part of his immigration policy where he
reiterated that all illegal immigrants are "subject to deportation" with
priority given to illegal immigrants who have committed significant
crimes and those who have overstayed visas. He noted that all those
seeking legalization would have to go home and re-enter the country
legally. In a meeting with concerned mayors, Sessions explained that the
Executive Order merely directs cities to enforce the preexisting
thirty-year-old law, 8 U.S.C. 1373 which means that "there is no
sanctuary city debate." On April 25, U.S. District Judge William Orrick
III sided with San Francisco and Santa Clara in their lawsuit against
the Trump administration, issuing a temporary injunction effectively
blocking the order targeting so-called sanctuary cities. Justice Orrick
said that the president "has no authority to attach new conditions to
federal spending". Judge Orrick issued a nationwide permanent injunction
on November 20, 2017, declaring that section 9(a) of Executive Order
13768 was "unconstitutional on its face" and violates "the separation of
powers doctrine and deprives {[}the plaintiffs{]} of their Tenth and
Fifth Amendment rights."

\section{Social policy}\label{social-policy}

\begin{itemize}
\item
  \emph{The proposed American Health Care Act, announced by
  Congressional Republicans in March 2017, would have made Planned
  Parenthood "ineligible for Medicaid reimbursements or federal family
  planning grants".}
\item
  \emph{On April 13, Trump quietly signed H.J.}
\item
  \emph{On January 23, Trump signed a Presidential Memorandum on the
  Mexico City Policy regarding federal funding to foreign NGOs.}
\end{itemize}

Trump's appointment of a conservative justice, Neil Gorsuch, his
reinstatement of the Mexico City Policy, and his signing H.J. Res.
43---HHS Title X Funding for Planned Parenthood Rule are in keeping with
his pro-life policy. On January 23, Trump signed a Presidential
Memorandum on the Mexico City Policy regarding federal funding to
foreign NGOs. This is a key point in the abortion debate as foreign NGOs
that receive US federal funding will no longer be able to offer, promote
or perform abortion services as part of family planning in their own
countries using non-U.S. government funds. Forbes claimed this could
"potentially affect \$9.5 billion" in programs that reach "225 million
women globally".

On April 13, Trump quietly signed H.J. Res. 43---HHS Title X Funding for
Planned Parenthood Rule--- reversing Obama's December 2016 regulation
which had mandated that Title X recipients---like states local and state
governments---distribute federal funds for services related to
contraception, sexually transmitted infections, fertility, pregnancy
care, and breast and cervical cancer screening to qualified health
providers, regardless of whether they also perform abortions". Bloomberg
noted that although this was "one of the few opportunities" Trump has
had in his first 100 days to enact legislation, he signed this bill in
private. The Obama rule never came into effect as it was blocked by a
federal judge. Republicans want to cut off federal funding from
health-care organizations such as Planned Parenthood that perform
abortions. Proponents of the bill claim it supports states' rights over
federalist rights. The bill was passed under the procedures of the
Congressional Review Act. In the Senate Vice-President Pence cast a
tie-breaking vote. This will be an issue at the end of the first 100
days as Congress tries to avoid a government shutdown. In Fiscal Year
2014, Planned Parenthood clinics received \$20.5 million of the \$252.6
million distributed under the Title X Family Planning grant program.

The proposed American Health Care Act, announced by Congressional
Republicans in March 2017, would have made Planned Parenthood
"ineligible for Medicaid reimbursements or federal family planning
grants".

\section{Suspended reduction of Federal Housing Mortgage Insurance
Premium
rates}\label{suspended-reduction-of-federal-housing-mortgage-insurance-premium-rates}

\begin{itemize}
\item
  \emph{Within the first hours of Trump's presidency, he "suspended
  indefinitely" the reduced "Mortgage Insurance Premiums for loans with
  Closing/Disbursement date on or after January 27, 2017", known as the
  Federal Housing Administration's (FHA) Annual Mortgage Insurance
  Premium (MIP) Rates managed under the United States Department of
  Housing and Urban Development (HUD).}
\end{itemize}

Within the first hours of Trump's presidency, he "suspended
indefinitely" the reduced "Mortgage Insurance Premiums for loans with
Closing/Disbursement date on or after January 27, 2017", known as the
Federal Housing Administration's (FHA) Annual Mortgage Insurance Premium
(MIP) Rates managed under the United States Department of Housing and
Urban Development (HUD). It is "effective immediately". Obama's rate cut
would have lowered borrowing costs for first time and low income house
buyers.

\section{Gun control}\label{gun-control}

\begin{itemize}
\item
  \emph{In February 2017, the Trump administration signed into law a
  bill that rolled back a regulation implemented by the Obama
  administration, which would have prohibited approximately 75,000
  individuals who were receiving Social Security disability and had
  representative payees, from owning guns.}
\end{itemize}

In February 2017, the Trump administration signed into law a bill that
rolled back a regulation implemented by the Obama administration, which
would have prohibited approximately 75,000 individuals who were
receiving Social Security disability and had representative payees, from
owning guns. The bill was supported by the ACLU, the National
Association for Mental Health, The American Association of People with
Disabilities, and the National Council on Disability, the Consortium for
Citizens with Disabilities, as well as other disability rights
advocates. The initial regulation was supported by the Brady Campaign to
Stop Gun Violence, Moms Demand Action Against Gun Violence, Democratic
gun control advocates, and some mental health experts.

\section{High-priority
infrastructure}\label{high-priority-infrastructure}

\begin{itemize}
\item
  \emph{On January 24, Trump signed his second Executive Order entitled
  Expediting Environmental Reviews and Approvals for High Priority
  Infrastructure Projects (EO 13766) which is part of a series of five
  executive orders to date.}
\item
  \emph{On April 19, Trump signed a bill that extended the VA's Choice
  beyond August.}
\end{itemize}

On January 24, Trump signed his second Executive Order entitled
Expediting Environmental Reviews and Approvals for High Priority
Infrastructure Projects (EO 13766) which is part of a series of five
executive orders to date. This Order was part of a series "designed to
speed environmental permitting and reviews" as " major infrastructure
projects trigger an array of overlapping environmental and natural
resource laws and requirements".

On April 19, Trump signed a bill that extended the VA's Choice beyond
August. The 2014 Veterans' Access to Care through Choice,
Accountability, and Transparency Act was enacted in by the Obama
administration in response to the Veterans Health Administration scandal
of 2014.

\section{Foreign policy}\label{foreign-policy}

\begin{itemize}
\item
  \emph{On April 10, The Wall Street Journal described Trump's foreign
  policy as moving away from the "America First", "isolationist"
  policies towards more "mainstream" and "conventional" tendencies under
  the more stabilizing influence of Tillerson, Mattis, McMaster, Ross,
  and Kushner.}
\item
  \emph{On the first day of Trump's presidency, the White House website
  had posted a 220-word description of its foreign policy.}
\end{itemize}

The main group advising the President on foreign affairs and national
security is the National Security Council (NSC) which coordinates
national agencies such as the secretaries of defense and state; the
secretaries of the army, navy, and air force.

On April 10, The Wall Street Journal described Trump's foreign policy as
moving away from the "America First", "isolationist" policies towards
more "mainstream" and "conventional" tendencies under the more
stabilizing influence of Tillerson, Mattis, McMaster, Ross, and Kushner.

On the first day of Trump's presidency, the White House website had
posted a 220-word description of its foreign policy. It was
protectionist with a focus on "America First", as was his inaugural
address. His three top priorities were to defeat ISIS, to rebuild the
military, and to embrace diplomacy.

\section{Defense}\label{defense}

\begin{itemize}
\item
  \emph{Trump requested \$30 billion for FY 2017 which ends in
  September, and an increase of \$54 billion to Defense Department for
  FY 2018.}
\item
  \emph{On April 6, 2017, Trump ordered a missile strike on Shayrat Air
  Base near Homs, in Syria.}
\item
  \emph{If that was a lie, then during Trump's term, South Korea will
  not trust whatever Trump says."}
\end{itemize}

At the time Trump took office, U.S. military spending had reached its
highest peak ever. Trump requested \$30 billion for FY 2017 which ends
in September, and an increase of \$54 billion to Defense Department for
FY 2018. The \$639 billion in FY2018 would result in deep cuts to many
other departments including the State Department, the diplomatic arm of
the administration.

After Trump's April 12 first face-to-face meeting with NATO Secretary
General Jens Stoltenberg, Trump announced that he had changed views
about NATO. Trump had previously complained that NATO was "obsolete" as
it did not fight terrorism. On March 18, Trump called on NATO's member
nations to contribute more to NATO. After the White House meeting, Trump
realized that NATO has been engaged in combating groups like ISIS. Trump
will maintain the "US commitment to NATO while reiterating its member
nations must step up their military financing".

On January 29, Trump authorized the first military operation of his
Presidency---a raid by US commandos on Al-Qaeda in Yakla, Baida in
Yemen. At least 14 jihadists were killed in the raid, as well as 10
civilians, including children. The raid also resulted in the death of
Chief Petty Officer William Owens a 36-year-old Virginia-based Navy
SEAL, the first U.S. combat casualty in Trump's presidency.

According to the New York Times, Owen's death "came after a chain of
mishaps and misjudgments that plunged the elite commandos into a
ferocious 50-minute firefight that also left three others wounded and a
\$75 million aircraft deliberately destroyed."

On April 6, 2017, Trump ordered a missile strike on Shayrat Air Base
near Homs, in Syria. 59 Tomahawk missiles were launched from the
USS~Ross~(DDG-71) and USS~Porter~(DDG-78) from the Mediterranean Sea.

On April 8, four days after North Korea had test-fired a ballistic
missile, an announcement by the United States Pacific Command (PACOM)
commander was posted via U.S. Third Fleet Public Affairs stating that
PACOM had ordered the USS Carl Vinson supercarrier to "sail north and
report on station in the Western Pacific Ocean".\\
It was a premature announcement that led to a "glitch-ridden sequence of
events"---a result of confusion created by a "miscommunication" between
the "Pentagon and the White House." On April 8 and April 9, media
outlets such as Fox News, RT, CNN, USA Today, BBC and others had
published the erroneous announcement that warships were heading to the
Korean Peninsula within the context of escalating US-North Korean
tensions. In an interview with FOX Business Network's Maria Bartiromo
that aired on April 12, President Trump warned, "We are sending an
armada. Very powerful. We have submarines. Very powerful. Far more
powerful than the aircraft carrier. That I can tell you." By April 17,
North Korea's deputy United Nations ambassador accused the United States
of "turning the Korean peninsula into "the world's biggest hotspot" and
the North Korean government stated "its readiness to declare war on the
United States if North Korean forces were to be attacked." On April 17,
the Defense News broke the story that the Carl Vinson and its escorts
were 3,500 miles from Korea, engaged in scheduled joint Royal Australian
Navy exercises in the Indian Ocean. According to Dana White, the
Pentagon's chief spokeswoman, the Carl Vinson was heading north on April
18. The Wall Street Journal reported on April 19, that the incident
sparked both "criticism and ridicule" as some felt "duped by Trump." In
the article, Hong Joon-pyo, a candidate in the 2017 South Korean
presidential election, was quoted as saying, "What President Donald
Trump said was important for the national security of South Korea. If
that was a lie, then during Trump's term, South Korea will not trust
whatever Trump says."

On April 13, the United States dropped a 'mother of all bombs' (MOAB) in
the Nangarhar Province Afghanistan---the first use of the bomb on the
battlefield. On April 8, Staff Sgt. Mark De Alencar was killed during an
operation against ISIS in Nangarhar Province.

The most consequential shift in Trump's defense policy was the April 6
cruise-missile launch at a Syrian airbase.

\includegraphics[width=5.50000in,height=3.09375in]{media/image2.jpg}\\
\emph{Peter Navarro, Director of the White House National Trade Council,
addresses President Donald Trump's promises to American people, workers,
and domestic manufacturers (Declaring American Economic Independence on
6/28/2016) in the Oval Office with Vice President Mike Pence and
Secretary of Commerce Wilbur Ross before President Trump signs 2
Executive Orders regarding trade in March 2017}

\section{Trade policies}\label{trade-policies}

\begin{itemize}
\item
  \emph{On March 31, President Donald Trump signed two executive orders
  on trade.}
\item
  \emph{Trump had been one of ExIm's harshest critics.}
\item
  \emph{On April 18, 2017, President Trump signed an executive order
  that directed federal agencies to implement a "Buy American, Hire
  American" strategy.}
\item
  \emph{Through the executive order, Trump states his broad policy of
  economic nationalism without having to go through Congress.}
\end{itemize}

On January 23, Trump fulfilled a campaign pledge by signing an executive
order withdrawing the United States from the Trans-Pacific Partnership
(TPP) or Trans Pacific Partnership Agreement (TPPA). According to the
BBC, Trump had pledged to withdraw from the Trans Pacific Partnership
(TPP) and he signed an executive order on the TPP his first few days.
However, the EO was largely symbolic since the deal has not been
ratified by a divided US Congress." The Trans-Pacific Partnership (TPP)
or Trans Pacific Partnership Agreement (TPPA), was a trade agreement
between the United States and eleven Pacific Rim nations---Australia,
Brunei, Canada, Chile, Japan, Malaysia, Mexico, New Zealand, Peru,
Singapore, and Vietnam that would have created a "free-trade zone for
about 40 percent of the world's economy."

On April 18, 2017, President Trump signed an executive order that
directed federal agencies to implement a "Buy American, Hire American"
strategy. The executive order directs federal agencies to implement a
new system that favors higher-skilled, higher-paid applicants. The order
is the first initiative in response to a key pledge made by Trump during
his presidential campaign to promote a 'Buy American, Hire American.'
The EO is intended to order federal agencies to review and propose
reforms to the H-1B visa system. Through the executive order, Trump
states his broad policy of economic nationalism without having to go
through Congress. Cabinet secretaries from Departments of Labor,
Justice, Homeland Security, and State will "fill in the details with
reports and recommendations about what the administration can legally
do." Trump argued that the EO would "end the 'theft of American
prosperity', which he said had been brought on by low-wage immigrant
labor."

On March 31, President Donald Trump signed two executive orders on
trade. One examines forms of "trade abuse," taking a country-by-country
as well as product and industry look over 90 days at cheating, law
enforcement, and currency misalignment by foreign countries that causes
U.S. trade deficits. President Trump said the order ensures "that we
fully collect all duties imposed on foreign importers that cheat, the
cheaters." Another to strengthen anti-dumping rules and countervailing
duties. The order directs Homeland Security, Commerce, and Treasury
departments to ensure enforcement and "those who break the rules will
face severe consequences".

Trump---who had been dismissive of the Export-Import Bank (ExIm)---made
an about-face on April 15 by nominating Scott Garrett as head of the
ExIm breaking a deadlock that had prevented the Bank from operating
since 2014. Although Trump had privately made known that he would not
side with "conservative Republicans, including those in his own
administration", who wanted to "cripple" the ExIm in February, he did
not announce it publicly until April 13, when he told The Wall Street
Journal that he would fill two seats of ExIm's five-seat board which
would allow the Bank to make loans greater than \$10 million. Trump had
been one of ExIm's harshest critics. Conservatives call it the 'Bank of
Boeing' and an 'epicenter of crony capitalism'. Its supporters such as
Boeing and General Electric Co, claim that it facilitates trade worth
billions of dollars in exports helping hundreds of businesses. Prior to
making the announcement, Trump held two significant meetings related to
ExIm---an April 3 formal visit with Egyptian president Abdel Fattah
el-Sisi who is negotiating for billions of dollars in ExIm financing and
an April 11 meeting with Boeing Chief Executive Jim McNerney. Sisi also
met with Lockheed Martin and General Electric CEOs during his visit to
the U.S. in April.

\section{International relations}\label{international-relations}

\section{Australia}\label{australia}

\begin{itemize}
\item
  \emph{The full transcript of that phone conversation was leaked in
  August 2017, and published by the Washington Post.}
\item
  \emph{A February 2 report by The Washington Post claimed that US
  President Donald Trump berated the Australian, Prime Minister Turnbull
  during one of Trump's first phone calls made to foreign officials.}
\end{itemize}

A February 2 report by The Washington Post claimed that US President
Donald Trump berated the Australian, Prime Minister Turnbull during one
of Trump's first phone calls made to foreign officials. Trump stated
that the 2016 asylum deal was an attempt to export the next Boston
bombers to the United States. The contentious deal involves a 2016
agreement between the Obama administration and Australia whereby the
U.S. would resettle 1,250 refugees held in controversial offshore
immigration detention facilities---Manus and Nauru islands. In return,
Australia would 'resettle refugees from El Salvador, Guatemala and
Honduras." The full transcript of that phone conversation was leaked in
August 2017, and published by the Washington Post. Later that day, Trump
explained that while he respected Australia, they, along with many other
countries, were "terribly taking advantage" of the United States. The
following day, Australian Ambassador to the United States Joe Hockey was
sent to the White House and held meetings with White House Chief
Strategist Steve Bannon and Chief of Staff Reince Priebus. Spicer
described the phone call as "very cordial". The 25-minute phone call on
January 28, was described as "acrimonious" by Reuters and Trump's "worst
call by far" with a foreign leader by the Washington Post. During a
joint news conference with Prime Minister Turnbull, Vice-President
Pence---who was on a "10-day, four-country trip" in April to the Pacific
Rim, announced that even if the United States did not "admire the
agreement", Trump had made it clear that the United States would honour
the 2016 agreement to resettle refugees. Turnbull responded, "whatever
the reservations of the president are", the decision "speaks volumes for
the commitment, the integrity of President Trump, and your
administration, sir, to honour that commitment." "The US is Australia's
most important security partner, while China is its most important
trading partner."

\includegraphics[width=4.81067in,height=5.50000in]{media/image3.jpg}\\
\emph{Prime Minister Justin Trudeau (left) and President Donald Trump
(right) meet in Washington in February 2017}

\section{Canada}\label{canada}

\begin{itemize}
\item
  \emph{Prime Minister Justin Trudeau met Trump in Washington, D.C. in
  February 2017.}
\end{itemize}

Prime Minister Justin Trudeau met Trump in Washington, D.C. in February
2017. Trudeau said that "The last thing Canadians expect is for me to
come down and lecture another country on how they choose to govern",
referring to Trump's "refugee ban"---Executive Order 13769. The two
leaders emphasized the importance of the two countries' ongoing
relationship, with Trudeau adding that "there are times when we have
differed in our approaches. And that's always been done firmly and
respectfully,". Commerce Secretary Wilbur Ross said that, "It has been a
bad week for U.S.-Canada trade relations", as he announced stiff tariffs
up stiff tariffs of up to 24\% on Canadian lumber on April 24, as dairy
product trade fell through. The Canada--United States softwood lumber
dispute has been since ongoing since the 1980s making it one of the
longest trade disputes between the two countries, as well as one of the
largest. Trump is under pressure to begin renegotiating NAFTA, the trade
deal between Canada, Mexico and the US. On April 25, Canada's
International Trade Minister Francois-Philippe Champagne and soft lumber
industry representatives promoted trade with China in Beijing in
response to what is perceived as U.S. protectionist policies.

\section{China}\label{china}

\begin{itemize}
\item
  \emph{Xi was concerned by the December 2, 2016, phone call from Taiwan
  President Tsai Ing-wen to Trump and Trump's questioning of the One
  China policy.}
\item
  \emph{On February 10, Trump and Xi Jinpin spoke on the phone for the
  first time since Donald Trump took office, during which Donald Trump
  committed to honoring the One China policy at Xi's request.}
\end{itemize}

The Mar-a-Lago summit meeting on April 6 and 7 between Trump and
President Xi Jinping of China, during the first 100 days of the new US
administration was heralded by The Telegraph as the "most significant
bilateral summit in decades." The South China Morning Post reported
that---in spite of differences regarding Taiwan, the South China Sea and
the most urgent issue---North Korea's nuclear programme---"the summit
between the US and Chinese presidents had both symbolic and tangible
successes." During the April 7--8 visit with Chinese President Xi
Jinping, Trump acknowledged that international relations are much more
complicated than he had imagined. In regards to North Korea, he had
hoped to negotiate better trade deals with China in exchange for China
dealing with the nuclear threat from North Korea. In an interview with
Wall Street Journal's Gerald F. Seib Trump said, "After listening for 10
minutes, I realized it's not so easy. I felt pretty strongly that they
had a tremendous power {[}over{]} North Korea. ... But it's not what you
would think." Trump also affirmed that North Korea was the United
States' "biggest international threat".

The BBC reported on April 19 that China "was 'seriously concerned" about
nuclear threats" as tensions between North Korea and the United States
escalated with a "war of words" between North Korea's leader Kim Jong-un
and the Trump administration. Recent threats included Vice-President
Mike Pence statement that the period of "strategic patience" was over
and his April 19 statement that the US "would meet any attack with an
'overwhelming response.' North Korea recently warned of "full-out
nuclear war if Washington takes military action against it." Trump has
called for China to rein in North Korea, but China Daily reported that
"Washington must be aware of the limitations to Beijing's abilities, and
refrain from assuming that the matter can be consigned entirely to
Beijing alone." China Daily considered the U.N. Security Council
statement adopted on April 20 condemning North Korea's recent attempted
missile launch, as an indication that the Trump administration is
considering a "diplomatic solution."

In an April 12 interview with Wall Street Journal, Trump said he had
changed his mind and he would not label China a currency manipulator,
which had been one of his 100-day pledges. By April he believed that
China had not been manipulating its currency for months. He did not want
to "jeopardize" talks with the Chinese "on confronting the threat of
North Koreas." Early in Trump's presidency, the world's largest
financial newspaper, Nikkei Asian Review, had reported on February 1,
that Trump had labelled China and Japan as currency manipulators.

The Trump administration confirmed its commitment to defend Japan
against China's claims to the Senkaku Islands in the East China Sea
through the Treaty of Mutual Cooperation and Security between the United
States and Japan during a U.S. Defense Secretary James Mattis's visit to
Japan on February 4. By February 9, US-Chinese relations---the most
important bilateral relationship---had remained strained, President Xi
Jinping and Trump had not spoken and this had "drawn increasing
scrutiny." Xi was concerned by the December 2, 2016, phone call from
Taiwan President Tsai Ing-wen to Trump and Trump's questioning of the
One China policy. On February 10, Trump and Xi Jinpin spoke on the phone
for the first time since Donald Trump took office, during which Donald
Trump committed to honoring the One China policy at Xi's request.

During the World Economic Forum annual meeting in Davos on January
17--20, China's President Xi Jinping, as keynote speaker, "vigorously"
defended globalization in a speech that the Financial Times described as
"one would have expected to come from a US president". Mr. Xi observed
that "blaming economic globalisation for the world's problems is
inconsistent with reality... globalisation has powered global growth and
facilitated movement of goods and capital, advances in science,
technology and civilisation, and interactions among people In 2015,
China became the United States' largest trade partner, placing Canada
second. The Times 2017 article, citing an analysis by Peterson Institute
for International Economics, noted that "China and Mexico together
account for a quarter of US trade". Concerns have been raised about
Trump's proposed imposition of a 45 percent tariff on imports from
China. On January 23, The U.S. Commerce Department announced new
countervailing duties (CVDs) ranging from 38.61 to 65.46 percent on
Chinese vehicles in the antidumping case. In 2015, over 8.9 million
Chinese truck and bus tires worth \$1.07 billion were imported to the
United States.

At his Senate confirmation hearing as Secretary of State, in
mid-January, Rex Tillerson's statements about the South China Sea, "set
the stage for a possible crisis between the world's two biggest
economies should his comments become official American policy" and "put
further strains on one of the world's most important bilateral
relationships." According to an article on January 28, in the South
China Morning Post, an official from China's Central Military
Commission's Defence Mobilisation Department, ranking Chinese military
official considers war between China and the United States a real
possibility during Trump's term as president. An article in The Guardian
claims, "The bad news is that if in the coming months or years Trump
faces an ignominious end to his presidency through scandal or
mismanagement, a national crisis~-- involving China, or ISIS or another
foreign actor~-- could allow him to cling to power."

\section{Egypt}\label{egypt}

\begin{itemize}
\item
  \emph{On April 3, Trump hosted a formal visit with Egyptian president
  Abdel Fattah el-Sisi, in an effort to "reset" relations between the
  two countries, offering the U.S. government's "strong backing."}
\item
  \emph{Trump nominated a new head of ExIm which facilitates its
  operation---the ExIm had been hamstrung since 2014 because of
  opposition by Republicans.}
\item
  \emph{Trump publicly stated that Sisi's autocratic leadership was
  'fantastic.'}
\end{itemize}

On April 3, Trump hosted a formal visit with Egyptian president Abdel
Fattah el-Sisi, in an effort to "reset" relations between the two
countries, offering the U.S. government's "strong backing." Ties between
the two countries were strained since Sisi deposed Egyptian president
Mohamed Morsi during the July 2013 military coup. Trump publicly stated
that Sisi's autocratic leadership was 'fantastic.' Sisi, who is seeking
"billions of dollars in financing" from the Export-Import Bank for large
investments in infrastructure investments, also met with the
representatives from the IMF, the World Bank, Lockheed Martin and
General Electric. Trump nominated a new head of ExIm which facilitates
its operation---the ExIm had been hamstrung since 2014 because of
opposition by Republicans. During his talks with Sisi in April, Senate
Foreign Relations Committee Chairman Bob Corker (R-Tenn.) had advocated
for the release of six humanitarian workers, including a U.S.
citizen---30-year-old Aya Hijazi and her husband, who had been
imprisoned in Egypt since May 1, 2014. A court in Egypt dropped all
charges against them on April 16.

\section{European Union}\label{european-union}

\begin{itemize}
\item
  \emph{In a 60-minute interview at Trump Tower in mid-January, with
  Michael Gove of the Times of London and Kai Diekmann of Bild, Trump
  praised Brexit, criticized NATO as "obsolete", and the European Union
  as "basically a vehicle for Germany."}
\item
  \emph{These "worrying declarations", among others, compelled the
  President of the European Council, Donald Tusk, to raise concerns in a
  letter to 27 European leaders, that the Trump administration seemed to
  "question the last 70 years of American foreign policy," placing the
  European Union in a "difficult situation."}
\end{itemize}

In a 60-minute interview at Trump Tower in mid-January, with Michael
Gove of the Times of London and Kai Diekmann of Bild, Trump praised
Brexit, criticized NATO as "obsolete", and the European Union as
"basically a vehicle for Germany." He said it was a "very catastrophic
mistake" on Angela Merkel's part to admit a million refugees~-- whom he
refers to as "illegals". These "worrying declarations", among others,
compelled the President of the European Council, Donald Tusk, to raise
concerns in a letter to 27 European leaders, that the Trump
administration seemed to "question the last 70 years of American foreign
policy," placing the European Union in a "difficult situation."

\section{Iran}\label{iran}

\begin{itemize}
\item
  \emph{Late on April 18, 2017, the Trump Administration certified that
  Iran had continued to comply with the 2015 nuclear framework
  agreement.}
\end{itemize}

There are no formal diplomatic relations between Iran and the United
States. Iranian citizens were temporary banned from entering the United
States by the executive order "Protecting the Nation From Foreign
Terrorist Entry Into the United States." Late on April 18, 2017, the
Trump Administration certified that Iran had continued to comply with
the 2015 nuclear framework agreement. During his campaign, Trump had
denounced the agreement as 'the worst deal ever' but was frustrated in
his plans to renegotiate the nuclear deal as "canceling the deal would
likely cause significant problems."

\section{Israel}\label{israel}

\begin{itemize}
\item
  \emph{Israel's prime minister Benjamin Netanyahu and Trump held their
  first official visit at the White House on February 15.}
\item
  \emph{At the press conference, Trump urged Netanyahu to "'hold back'
  on building Jewish settlements on territories occupied by Israel in
  1967 'for a little bit'."}
\item
  \emph{Trump's priority of destroying the Muslim radicals of Islamic
  State (IS)" differs from Netanyahu's.}
\end{itemize}

Israel's prime minister Benjamin Netanyahu and Trump held their first
official visit at the White House on February 15. At the press
conference, Trump urged Netanyahu to "'hold back' on building Jewish
settlements on territories occupied by Israel in 1967 'for a little
bit'." According to The Economist, Trump appeared to step back from the
"long-standing, bipartisan American insistence that peace can be reached
only through the establishment of a sovereign Palestinian state
alongside the Jewish one", the two-state solution. Trump's priority of
destroying the Muslim radicals of Islamic State (IS)" differs from
Netanyahu's. Israel is more concerned about "containing Iran, the
largest power in the Shia Muslim world. Given that Iran is itself
fighting IS in Syria and Iraq, the two goals could even be in conflict."
In a marked change from his visit to the White House under the Obama
administration, Netanyahu blurred the distinction by "denouncing both IS
and Iran in the same attack on 'militant Islam' and hailing Mr Trump's
'great courage' in tackling 'radical Islamic terror'".

\section{Mexico}\label{mexico}

\begin{itemize}
\item
  \emph{Mexican President Enrique Peña Nieto opposes Trump's approach to
  the renegotiation of NAFTA and the implications of Trump's Executive
  Order 13767.}
\item
  \emph{Since early in Trump's presidency, Mexico and United States
  faced a diplomatic crisis.}
\end{itemize}

Since early in Trump's presidency, Mexico and United States faced a
diplomatic crisis. Mexican President Enrique Peña Nieto opposes Trump's
approach to the renegotiation of NAFTA and the implications of Trump's
Executive Order 13767. After decades of cooperation between the two
nations relations between the US and Mexico are seriously weakened.

\section{North Korea}\label{north-korea}

\begin{itemize}
\item
  \emph{Trump received the news of the launch during the first official
  visit of Japan's prime minister, Shinzo Abe.}
\item
  \emph{They were dining at Mar-a-Lago, Trump's Florida resort.}
\item
  \emph{According to The Economist, on February 13, while Trump promised
  "to deal with the 'big, big' problem of North Korea 'very strongly'",
  he has few options.}
\end{itemize}

On February 12, North Korea tested a ballistic solid-fuel missile, the
Pukkuksong-2, which is part of a series of missile tests that have
largely defined the hostile North Korea--United States relations over
recent years. According to The Economist, on February 13, while Trump
promised "to deal with the 'big, big' problem of North Korea 'very
strongly'", he has few options. Trump received the news of the launch
during the first official visit of Japan's prime minister, Shinzo Abe.
They were dining at Mar-a-Lago, Trump's Florida resort.

\section{Russia}\label{russia}

\begin{itemize}
\item
  \emph{On February 24, Trump "risked triggering a new Cold War-style
  arms race between Washington and Moscow.}
\item
  \emph{On February 16, 2017, President Trump's Secretary of Defense,
  James Mattis, declared that the United States was not currently
  prepared to collaborate with Russia on military matters---including
  future anti-ISIL US operations.}
\end{itemize}

According to a Reuters report on February 9, 2017, in his first
60-minute telephone call with Putin, Putin inquired about extending New
START (Strategic Arms Reduction Treaty) a nuclear arms reduction treaty
between the United States and the Russia signed in 2010, which was
expected to last until 2021. and, after ratification, Trump denounced
the treaty claiming that it favored Russia and was "one of several bad
deals negotiated by the Obama administration". The New York Times
reported that on February 14, Russia deployed a new type of fully
operational ground-launched intermediate-range cruise missile that
"violates a landmark arms control treaty". The Americans have changed
its name from SSC-X-8 to SSC-8, reflecting its status as "operational"
not "X" referring to "in development".

On February 16, 2017, President Trump's Secretary of Defense, James
Mattis, declared that the United States was not currently prepared to
collaborate with Russia on military matters---including future anti-ISIL
US operations.

On February 24, Trump "risked triggering a new Cold War-style arms race
between Washington and Moscow. In an interview with Reuters, Trump said
that the "treaty limiting Russian and U.S. nuclear arsenals was a bad
deal for Washington" and he "would put the U.S. nuclear arsenal "at the
top of the pack." In response, Russia's Konstantin Kosachev wrote on his
Facebook page, "arguably Trump's most alarming statement on the subject
of relations with Russia".

Trump had "promised one of the 'greatest military buildups in American
history' in a feisty, campaign-style speech extolling robust
nationalism" at the annual Conservative Political Action Conference on
February 24 at National Harbor.

\section{Syria}\label{syria}

\begin{itemize}
\item
  \emph{On April 5, 2017, Trump responded to the April 4 chemical attack
  allegedly by Syrian Armed Forces on rebel-held Khan Shaykhun in Idlib
  Province, which enveloped men, women, and children in a suffocating
  fog of sarin gas, leaving more than eighty people dead and over three
  hundred more injured, saying that "...my attitude towards Syria and
  Assad has changed very much."}
\end{itemize}

On April 5, 2017, Trump responded to the April 4 chemical attack
allegedly by Syrian Armed Forces on rebel-held Khan Shaykhun in Idlib
Province, which enveloped men, women, and children in a suffocating fog
of sarin gas, leaving more than eighty people dead and over three
hundred more injured, saying that "...my attitude towards Syria and
Assad has changed very much." Both Tillerson and Nikki Haley had
previously stated that the Trump administration had no intention of
interfering in President Bashar Assad's leadership in the Syrian Civil
War, as the US focused on eliminating ISIS.

\section{United Kingdom}\label{united-kingdom}

\begin{itemize}
\item
  \emph{In January 2017, the Prime Minister Theresa May invited Trump to
  a state visit to the UK when she met Trump in Washington DC.}
\item
  \emph{The Speaker of the House of Commons, John Bercow, stated on
  February 6, 2017, that Trump would not be welcome to address
  parliament during any future state visit, drawing applause and
  cheering from some Members of Parliament.}
\end{itemize}

In January 2017, the Prime Minister Theresa May invited Trump to a state
visit to the UK when she met Trump in Washington DC. The visit was
planned to occur in June, although it may be delayed to July to coincide
with the upcoming G20 summit in Hamburg. Some sources have suggested
that the UK government may delay the visit until after the House of
Commons is in recess for the summer to avoid criticism from MPs. The
Speaker of the House of Commons, John Bercow, stated on February 6,
2017, that Trump would not be welcome to address parliament during any
future state visit, drawing applause and cheering from some Members of
Parliament.

More than 1,860,000 people signed a petition to prevent Trump from
making an official state visit, which states that such a visit "would
cause embarrassment to Her Majesty the Queen". The FCO responded to this
petition by stating that HM Government recognises the strong views
expressed by the many signatories of this petition, but does not support
this petition." Lord Ricketts, former Permanent Under-Secretary of State
for Foreign Affairs, said that the unprecedented speed of May's
invitation has put the Queen in a "very difficult situation". He
questioned whether Trump was "specially deserving of this exceptional
honour", given that US presidents are usually only invited to such
visits after at least a year in office. Writing to May, opposition
leader Jeremy Corbyn stated that the "invite should be withdrawn until
the executive orders are gone".

It was suggested that Trump's visit would have to take place outside
London, after Sir Bernard Hogan-Howe, the chief of the Metropolitan
Police, said that he had concerns about the visit given the number of
protests expected. One suggestion considered was for Trump to address a
rally at the National Exhibition Centre in Birmingham, a city where
50.4\% of voters voted to leave the EU, rather than London, which saw
59.9\% voting to remain. Local politicians and activists in Birmingham
promised to stage protests if the visit is moved, with Shabana Mahmood,
Labour MP for the Birmingham Ladywood constituency, saying that
"President Trump with his hateful and divisive rhetoric, policies and
Muslim ban is not welcome here."

During a March 14 Fox \& Friends interview, Andrew Napolitano said,
"Three intelligence sources have informed Fox News that President Obama
went outside the chain of command," using the British GCHQ to implement
surveillance on Donald Trump to avoid leaving "American fingerprints".
On March 16, Press Secretary Sean Spicer repeated Napolitano's claim at
a White House press briefing. The following day, a GCHQ spokesperson
called Napolitano's claim "utterly ridiculous". The White House denied
reports that it had apologized to the British government for the
accusation.

\section{Government and Finance (G\&F)}\label{government-and-finance-gf}

\begin{itemize}
\item
  \emph{The G\&F Division focuses on issues related to Congress, the
  executive and the judicial branches, the budget and appropriations,
  legislative process, homeland security, elections and certain
  financial issues such as public debt, inflation, savings, GDP,
  taxation and interest rates, banking, financial institutions,
  insurance and securities, public finance, fiscal and monetary policy,
  public debt, interest rates, gross domestic product, inflation and
  savings.}
\end{itemize}

The G\&F Division focuses on issues related to Congress, the executive
and the judicial branches, the budget and appropriations, legislative
process, homeland security, elections and certain financial issues such
as public debt, inflation, savings, GDP, taxation and interest rates,
banking, financial institutions, insurance and securities, public
finance, fiscal and monetary policy, public debt, interest rates, gross
domestic product, inflation and savings.

\includegraphics[width=5.50000in,height=3.70287in]{media/image4.jpg}\\
\emph{Judge Neil Gorsuch, his wife Louise, and President Donald Trump
during the announcement in the East Room of the White House}

\section{Supreme Court nomination}\label{supreme-court-nomination}

\begin{itemize}
\item
  \emph{According to CNN and Washington Post, on February 8, Gorsuch
  expressed concern that Trump's remarks on the judiciary were
  'demoralizing' and 'disheartening' to the independence of the
  judiciary.}
\item
  \emph{Following the February 3 ruling by federal judge James Robart,
  which temporarily blocked Trump's travel ban on people from seven
  Muslim countries, Trump has been openly critical of the Federal
  judiciary.}
\end{itemize}

On the evening of January 30, Trump announced his nomination of U.S.
Appeals Court judge Neil Gorsuch for the Supreme Court fulfilling his
campaign pledge that he would choose someone 'in the mold' of the late
Justice Antonin Scalia.

Following the February 3 ruling by federal judge James Robart, which
temporarily blocked Trump's travel ban on people from seven Muslim
countries, Trump has been openly critical of the Federal judiciary.
According to CNN and Washington Post, on February 8, Gorsuch expressed
concern that Trump's remarks on the judiciary were 'demoralizing' and
'disheartening' to the independence of the judiciary.

Gorsuch was approved by the Senate Judiciary committee on April 3.
Senate Republicans invoked the "nuclear option" after the April 6
filibuster that prevented cloture. After a year-long Republican block on
nominations, the Senate confirmed Gorsuch's nomination with a 54--45
vote, mainly along party lines. Gorsuch took office in a private
ceremony on April 10.

Hours after Gorsuch and four other Supreme Court conservatives justices
voted on April 20 to deny a stay of execution request from eight inmates
on Arkansas death-row, Ledell Lee was put to death with a lethal
injection, the first in Arkansas since 2005. Two inmates---Jack Jones
and Marcel Williams---received lethal injections on April 24.

\section{Monetary policy}\label{monetary-policy}

\begin{itemize}
\item
  \emph{This table shows some highs and lows of the Trade Weighted U.S.
  Dollar Index: Broad {[}TWEXB{]} from 2002 to April 2017.}
\item
  \emph{In the same interview, Trump said he would not label China a
  currency manipulator, which had been one of his 100-day pledges.}
\item
  \emph{Trump expressed concerns in that interview that, "I think our
  dollar is getting too strong, and partially that's my fault because
  people have confidence in me.}
\end{itemize}

On April 19, in an interview with The Wall Street Journal---in a
reversal of previous statements---Trump said he was considering keeping
Janet Yellen as chair of the Federal Reserve System, which oversees the
U.S. monetary policy. He explained that, "I do like a low-interest rate
policy, I must be honest with you." In the same interview, Trump said he
would not label China a currency manipulator, which had been one of his
100-day pledges. Trump expressed concerns in that interview that, "I
think our dollar is getting too strong, and partially that's my fault
because people have confidence in me. But that's hurting---that will
hurt ultimately." He believes a low dollar favors the U.S. in
international trade. From November 8, 2016---when Trump was elected---to
December 30, 2016, the trade-weighted average of the foreign exchange
value of the U.S. dollar (TWEXB) increased 4.4 percent. Towards the end
of the first 100 days, the TWEXB had dropped two percent. This table
shows some highs and lows of the Trade Weighted U.S. Dollar Index: Broad
{[}TWEXB{]} from 2002 to April 2017.

\section{Small government}\label{small-government}

\begin{itemize}
\item
  \emph{On January 23, President Trump signed an executive order that
  froze all federal hiring except for the military.}
\end{itemize}

On January 23, President Trump signed an executive order that froze all
federal hiring except for the military. The order specified that no new
positions can be created and no currently vacant positions may be filled
unless an agency head believes that the position is "necessary to meet
national security or public safety responsibilities". The order is due
to expire once the head of the Office of Management and Budget, Mick
Mulvaney, creates a "long-term plan to reduce the size of the Federal
Government's workforce through attrition."

On January 24, the Associated Press reported on emails from the
Administration to some government agencies sent shortly after the
inauguration, which "detailed specific prohibitions" banning certain
government agencies, such as the Agricultural Research Service
Agriculture Department from issuing "press releases, blog updates or
posts to the agency's social media accounts." In what the Associated
Press described as a "broader communications clampdown within the
executive branch," the Administration "instituted a media blackout." In
his January 25 press briefing, White House press secretary Sean Spicer
claimed that the emails did not come from the Administration: "They
haven't been directed by us to do anything...That directive did not come
from here."

On January 23, in a Presidential Memorandum, the president ordered a
temporary government-wide hiring freeze of the civilian work force in
the executive branch, which is managed by the Office of Personnel
Management. This will prevent federal agencies, except for the offices
of the new presidential appointees, national security, the military and
public safety, from filling vacant positions. The Brookings Institution
questioned whether this freeze would include financial regulators who
exercise independence from the executive branch---such as the Federal
Reserve Board of Governors (Fed), Office of the Comptroller of the
Currency (OCC), the Securities and Exchange Commission (SEC) among
others. In a Fox News report, based on statistics from the Office of
Personnel Management, the number of executive branch employees "hasn't
been this low since 1965" and has been "more or less steady" since 2001.

\section{Economic policy of Donald
Trump}\label{economic-policy-of-donald-trump}

\begin{itemize}
\item
  \emph{According to the April 28, 2017 Commerce Department report, in
  the first quarter of 2017, there was a "sharp decline from the 2.1\%
  in Q1 2016 to 0.7\% in Q1 2017---representing the weakest quarterly
  economic growth in three years.}
\item
  \emph{The report presents a statistical analysis of the American
  economy in the 2017 Q1---the gross domestic product (GDP).}
\end{itemize}

Trump's key economic policies included the dismantling of the
Dodd--Frank Wall Street Reform and Consumer Protection Act, and the
repeal of the Patient Protection and Affordable Care Act (ACA).

According to the April 28, 2017 Commerce Department report, in the first
quarter of 2017, there was a "sharp decline from the 2.1\% in Q1 2016 to
0.7\% in Q1 2017---representing the weakest quarterly economic growth in
three years. The report presents a statistical analysis of the American
economy in the 2017 Q1---the gross domestic product (GDP). In spite of
the soft GDP, by the end of Q1 2017, the S\&P 500 was near an all-time
high, representing a 12\% rise from the first quarter of 2016, as
investor confidence remained elevated based on Trump's promise to cut
taxes, deregulate and spend heavily on infrastructure such as roads and
bridges.

In March 2017, the unemployment rate fell to 4.5 percent and the
Consumer Sentiment Index reached 125.6, a level of consumer confidence
in the United States last seen in December 2000. It fell to 120.3 in
April. Consumer confidence or soft data contrasted with real consumer
spending or hard data, with a "big drop-off" in the amount Americans
actually spent during Trump's first 100 days.

\section{Changes to Dodd--Frank Wall Street Reform and Consumer
Protection
Act}\label{changes-to-doddfrank-wall-street-reform-and-consumer-protection-act}

\begin{itemize}
\item
  \emph{On February 3, after a meeting with his strategic and policy
  forum, which included Jamie Dimon, Chairman and CEO JPMorgan Chase,
  Trump issued an Executive Order, Core Principles for Regulating the
  United States Financial System, which directed the "Treasury secretary
  to submit a report on recommended changes to bank regulations in 120
  days."}
\end{itemize}

On February 3, after a meeting with his strategic and policy forum,
which included Jamie Dimon, Chairman and CEO JPMorgan Chase, Trump
issued an Executive Order, Core Principles for Regulating the United
States Financial System, which directed the "Treasury secretary to
submit a report on recommended changes to bank regulations in 120 days."
Trump wants to get "banks to lend money more aggressively" and wants to
make changes to the Dodd--Frank Wall Street Reform and Consumer
Protection Act (2010) which was enacted in response to the Great
Recession, bringing significant changes to U. S. financial regulation.

In an interview on February 3, with The Wall Street Journal, Trump's
National Economic Council Director, Gary Cohn, announced the planned
rollback of the fiduciary rule, which stated that brokers and advisers
who work with tax-advantaged retirement savings "must work in the best
interest of their clients" even at the expense of their own profits.

\section{Deregulation}\label{deregulation}

\begin{itemize}
\item
  \emph{One of the first acts by the Trump administration was an order
  signed by Chief of Staff Reince Priebus on January 20, under the
  subject "Regulatory Freeze Pending Review" to all Heads of Executive
  Departments and Agencies ordering agencies to immediately suspend all
  pending regulations and to "send no regulation" to the Office of
  Information and Regulatory Affairs (OFR) until the Trump
  administration can review them except for "emergency situations" or
  "urgent circumstances" allowed by the Director or Acting Director,
  Mark Sandy, of the Office of Management and Budget (OMB).}
\item
  \emph{On January 30, Trump signed his seventh Executive Order
  "Reducing Regulation and Controlling Regulatory Costs."}
\end{itemize}

One of the first acts by the Trump administration was an order signed by
Chief of Staff Reince Priebus on January 20, under the subject
"Regulatory Freeze Pending Review" to all Heads of Executive Departments
and Agencies ordering agencies to immediately suspend all pending
regulations and to "send no regulation" to the Office of Information and
Regulatory Affairs (OFR) until the Trump administration can review them
except for "emergency situations" or "urgent circumstances" allowed by
the Director or Acting Director, Mark Sandy, of the Office of Management
and Budget (OMB). This was comparable to prior moves by the Obama and
Bush administrations shortly after their inaugurations to revert
executive orders by outgoing presidents, signed in their final days in
office.

On January 30, Trump signed his seventh Executive Order "Reducing
Regulation and Controlling Regulatory Costs."

\section{Deregulation and
corporations}\label{deregulation-and-corporations}

\begin{itemize}
\item
  \emph{At a January 23 meeting with leaders of the United States'
  largest corporations, including Ford's Mark Fields, Dell Technologies'
  Michael Dell, Lockheed Martin's Marillyn Hewson, Under Armour's Kevin
  Plank, Arconic's Klaus Kleinfeld, Whirlpool's Jeff Fettig, Johnson \&
  Johnson's Alex Gorsky, Dow Chemical's Andrew Liveris, U.S. Steel's
  Mario Longhi, SpaceX's Elon Musk, International Paper's Mark Sutton,
  and Corning's Wendell Weeks promised to reward the companies who stay
  in the United States with aggressive cuts on U.S. federal regulations
  governing their companies by "75 percent or more."}
\end{itemize}

At a January 23 meeting with leaders of the United States' largest
corporations, including Ford's Mark Fields, Dell Technologies' Michael
Dell, Lockheed Martin's Marillyn Hewson, Under Armour's Kevin Plank,
Arconic's Klaus Kleinfeld, Whirlpool's Jeff Fettig, Johnson \& Johnson's
Alex Gorsky, Dow Chemical's Andrew Liveris, U.S. Steel's Mario Longhi,
SpaceX's Elon Musk, International Paper's Mark Sutton, and Corning's
Wendell Weeks promised to reward the companies who stay in the United
States with aggressive cuts on U.S. federal regulations governing their
companies by "75 percent or more."

\section{Trump meets with CEOs of pharmaceutical
companies}\label{trump-meets-with-ceos-of-pharmaceutical-companies}

\begin{itemize}
\item
  \emph{Following Trump's press conference on January 11, Fortune
  claimed that the largest pharmaceutical companies had lost over \$20
  billion in 20 minutes.}
\item
  \emph{Trump called for lower prices, "We have no choice.}
\item
  \emph{Following the morning meeting with CEOs on January 31, Trump
  abandoned his pledge to allow "Medicare negotiate bulk discounts in
  the price it pays for prescription drugs."}
\end{itemize}

On January 31, Trump met with CEOs of pharmaceutical firms, including
Novartis's Joseph Jimenez who also represented the Pharmaceutical
Research and Manufacturers of America---the pharmaceutical industry's
powerful lobbying group, Merck \& Co.'s Kenneth Frazier, Johnson \&
Johnson, Celgene's Robert Hugin, Eli Lilly, Amgen's Robert Bradway.
Trump called for lower prices, "We have no choice. For Medicare, for
Medicaid. We have to get the prices way down." In return, he promised to
boost the pharmaceutical companies competitiveness by curbing
regulations "from 9,000 pages" to "100 pages," and by lowering
pharmaceutical companies' tax rates. Trump noted that Food and Drug
Administration (FDA) approvals "force pharmaceutical companies to spend
years and billions of dollars developing drugs." He promised his
nomination for FDA Commissioner would oversee an FDA overhaul. In the
listening session with pharmaceutical industry leaders, Trump noted
that, "it costs sometimes \$2.5 billion on average, actually, to come up
with a new product. ...15 years, \$2.5 billion to come up with a product
where there's not even a safety problem. So it's crazy. I'm surprised
you can't get them to move faster than that."

Trump had promised in March 2016, to reform the pharmaceutical industry,
including the removal of existing free market barriers to allow
imported, dependable, safe, reliable, and cheaper drugs from overseas,
bringing more options to American consumers. Following Trump's press
conference on January 11, Fortune claimed that the largest
pharmaceutical companies had lost over \$20 billion in 20 minutes. The
Medicare Prescription Drug, Improvement, and Modernization Act (2003)
expressly prohibited Medicare from negotiating bulk prescription drug
prices and Trump had pledged to revert this. Following the morning
meeting with CEOs on January 31, Trump abandoned his pledge to allow
"Medicare negotiate bulk discounts in the price it pays for prescription
drugs."

\section{Limitations on executive agency members
lobbying}\label{limitations-on-executive-agency-members-lobbying}

\begin{itemize}
\item
  \emph{On January 28, Trump signed an Executive Order to fulfilling his
  campaign pledge to limit lobbying of executive agency members.}
\end{itemize}

On January 28, Trump signed an Executive Order to fulfilling his
campaign pledge to limit lobbying of executive agency members.

\section{Department of Justice}\label{department-of-justice}

\begin{itemize}
\item
  \emph{Trump revoked EO 13775 on March 31 with "Presidential Executive
  Order on Providing an Order of Succession Within the Department of
  Justice.}
\item
  \emph{Boente had replaced Acting Attorney General Sally Yates who was
  fired by Trump for ordering the Justice Department to not defend
  Trump's Executive Order 13769 which restricted entry to the United
  States.}
\end{itemize}

On February 8, Alabama Senator Jeff Sessions, who was nominated by Trump
in January, was confirmed as United States Attorney General (A.G.), the
head of the Justice Department per 28 U.S.C.~§~503. He is the United
States government's chief law enforcement officer and lawyer with
113,000 employees working under his leadership. According to The
Washington Post, Sessions' "conservative, populist views have shaped
many" of Trump's "early policies, including on immigration". The
nomination battle was described by The New York Times, as "a bitter and
racially charged". The confirmation process for Trump's nominee Senator
Jeff Sessions was described as " strikingly contentious" by The New York
Times; with Fox News calling it a "wild night", and CNN calling the
"rare rebuke" a "stunning moment" as Senator Mitch McConnell invoked
Rule XIX to silence Senator Elizabeth Warren for the rest of the
hearing. McConnell interrupted Warren as she read several pages by
Coretta Scott King and Senator Ted Kennedy regarding Session's alleged
racial bias from the 500-plus page transcript submitted in 1986, that
contributed to the decision by the then-Republican-led Judiciary
Committee to reject his nomination to a federal judgeship. Warren
immediately live-streamed her reading of the letter, critical of
Sessions, that the widow of Martin Luther King Jr. had written to
Senator Strom Thurmond in 1986. and numerous media outlets made the
full-text available.

Trump appointed Dana J. Boente to serve as acting Attorney General until
Session's Senate Confirmation. After firing Yates, Trumped signed his
eleventh Executive Order 13775 on February 9, specifically reversing the
DOJ's line of succession in Obama's EO 13762 in order to appoint the
United States Attorney for the Eastern District of Virginia---Dana J.
Boente---as Acting Attorney General. Trump revoked EO 13775 on March 31
with "Presidential Executive Order on Providing an Order of Succession
Within the Department of Justice.

Boente had replaced Acting Attorney General Sally Yates who was fired by
Trump for ordering the Justice Department to not defend Trump's
Executive Order 13769 which restricted entry to the United States. Yates
claimed that, "At present, I am not convinced that the defense of the
executive order is consistent with these responsibilities {[}of the
Department of Justice{]}, nor am I convinced that the executive order is
lawful".

\section{Voter fraud claims}\label{voter-fraud-claims}

\begin{itemize}
\item
  \emph{In January, US News reported that members of Trump's cabinet and
  family were registered to vote in multiple states.}
\item
  \emph{On January 25, Spicer confirmed in a press briefing that Trump
  continued to believe that "millions voted illegally in the election"
  based on "studies and evidence that people have presented him."}
\item
  \emph{Trump had announced on January 25 that he was conducting an
  investigation into voter fraud.}
\end{itemize}

Since November 2016, and during his presidency, Trump has repeated voter
fraud allegations that between 3 and 5 million people voted illegally
and cost him the popular vote to Hillary Clinton, and also that
thousands of voters were illegally bused from Massachusetts into New
Hampshire where former Senator Kelly Ayotte was defeated, and where
Trump narrowly lost to Clinton in 2016.

Trump had announced on January 25 that he was conducting an
investigation into voter fraud. He repeated unsubstantiated claims about
the number of fraudulent voters and referred to VoteStand founder Gregg
Phillips, who could not produce any evidence of voter fraud. In January,
US News reported that members of Trump's cabinet and family were
registered to vote in multiple states. On February 10, Federal Election
Commission (FEC) Commissioner, Ellen L. Weintraub, issued a statement
calling on Trump, to provide the evidence of what would "constitute
thousands of felony criminal offenses under New Hampshire law." By
February 12, Steve Miller was still unable to provide concrete evidence
to support claims of voter fraud in an interview with Stephanopolous,
but he seemed to direct Stephanopolous to the often-cited 2012 Pew
Research Center study. In fact, the 2012 PEW report entitled
"Inaccurate, Costly, and Inefficient Evidence That America's Voter
Registration System Needs an Upgrade," that was based on 2008 data, was
about "outdated voter rolls, not fraudulent votes" and "makes no mention
of noncitizens voting or registering to vote". The report showed that
because of inefficiencies in the voter system, 24 percent of eligible
citizens were not able to be registered, representing "51 million
citizens.":8 Problems related to voter registration often affected
"military personnel---especially those deployed overseas and their
families---who were almost twice as likely to report registration
problems as was the general public in 2008.":7 In November, "the former
director of Pew's election program" explained that, "We found millions
of out of date registration records due to people moving or dying, but
found no evidence that voter fraud resulted." On January 25, Spicer
confirmed in a press briefing that Trump continued to believe that
"millions voted illegally in the election" based on "studies and
evidence that people have presented him." This included an often-cited
and contested 2014 Old Dominion University study entitled, "Do
non-citizens vote in U.S. elections?". Using Cooperative Congressional
Election Study data from 2008 and 2010, the researchers had argued that
more than 14\% of non-citizens "indicated that they were registered to
vote".

\section{2018 United States federal
budget}\label{united-states-federal-budget}

\begin{itemize}
\item
  \emph{In late April 2017, Republicans have control of both Congress
  and the White House.}
\item
  \emph{To avert a possible government shutdown, the Trump
  administration face an April 28 deadline---the expiration of the
  December 10, 2016 continuing resolution (H.R.}
\item
  \emph{Trump submitted his first budget request which recommends
  funding levels for the next fiscal year 2018---covering the period
  from October 1, 2017 to September 30, 2018---to the 115th Congress.}
\end{itemize}

Trump submitted his first budget request which recommends funding levels
for the next fiscal year 2018---covering the period from October 1, 2017
to September 30, 2018---to the 115th Congress. Trump's request including
a \$639 billion defense budget and corresponding major cuts to other
federal departments.

To avert a possible government shutdown, the Trump administration face
an April 28 deadline---the expiration of the December 10, 2016
continuing resolution (H.R. 2028) (Public Law 114-254). Discussion time
on controversial issues such as funding for a border wall defunding
Planned Parenthood, was limited by the two-week Easter recess that began
on April 7. The government was shutdown during the Clinton and Obama
administrations as a result of clashes between Republicans in Congress
and Democrats in the White House. In late April 2017, Republicans have
control of both Congress and the White House. A shutdown would result in
"government agencies {[}locking{]} their doors, national parks
{[}refusing{]} visitors and federal workers {[}being{]} told not to
report to work". The appropriations process cannot be accomplished
without consulting the Democrats---unlike rolling back federal
regulations with Congressional Review Acts and attempts to repeal
Obamacare.

\section{Tax reform}\label{tax-reform}

\begin{itemize}
\item
  \emph{The White House memo entitled "2017 Tax Reform for Economic
  Growth and American Jobs" was presented on April 26 in what The Wall
  Street Journal described as his "finest moment" in the first 100 days
  and a policy and political success.}
\end{itemize}

The White House memo entitled "2017 Tax Reform for Economic Growth and
American Jobs" was presented on April 26 in what The Wall Street Journal
described as his "finest moment" in the first 100 days and a policy and
political success. Individual reform includes "reducing the 7 tax
brackets to 3 tax brackets for 10\%, 25\% and 35\%, doubling the
standard deduction, providing tax relief for families with child and
dependent care expenses." The taxation system will be simplified to
"eliminate targeted tax breaks that mainly benefit the wealthiest
taxpayers, protect the home ownership and charitable gift tax
deductions, repeal the Alternative Minimum Tax, repeal the death tax and
repeal the 3.8\% Obamacare tax that hits small businesses and investment
income." Business reform includes "15\% business tax rate, territorial
tax system to level the playing field for American companies, one-time
tax on trillions of dollars held overseas and elimination of tax breaks
for special interests." The memo did not provide legislative content but
rather broad outlines that will be developed in Congress but may face
some opposition from both sides.

\section{Energy, environmental, and science
policy}\label{energy-environmental-and-science-policy}

\section{Climate change}\label{climate-change}

\begin{itemize}
\item
  \emph{Trump proposed defunding the Clean Power Plan in his FY2018
  budget, and his March 28 executive order directed Environmental
  Protection Agency administrator Scott Pruitt to review the Clean Power
  Plan.}
\item
  \emph{Trump rescinded many Obama-era regulations aimed at cutting the
  volume of greenhouse gas emissions, which faced strong opposition and
  legal challenges.}
\end{itemize}

Trump rescinded many Obama-era regulations aimed at cutting the volume
of greenhouse gas emissions, which faced strong opposition and legal
challenges. The key focus of his deregulatory efforts was the Clean
Power Plan created under the Obama administration, which restricted GHG
emissions at coal-fired plants. Trump proposed defunding the Clean Power
Plan in his FY2018 budget, and his March 28 executive order directed
Environmental Protection Agency administrator Scott Pruitt to review the
Clean Power Plan. He also lifted a 14-month-old halt on new coal leases
on federal lands.

\section{Dakota Access and Keystone XL
pipelines}\label{dakota-access-and-keystone-xl-pipelines}

\begin{itemize}
\item
  \emph{In a meeting with small business leaders on January 30, Trump
  clarified that one of the reasons for approving the pipelines was to
  insist that pipeline makers implement a made-in-America approach.}
\item
  \emph{On January 24, Trump signed three Presidential Memoranda
  regarding construction of pipelines; "Regarding Construction of
  American Pipelines" was his fifth memoranda, "Regarding Construction
  of the Dakota Access Pipeline" was his sixth and the seventh was
  "Regarding Construction of the Keystone XL Pipeline."}
\end{itemize}

On January 24, Trump signed three Presidential Memoranda regarding
construction of pipelines; "Regarding Construction of American
Pipelines" was his fifth memoranda, "Regarding Construction of the
Dakota Access Pipeline" was his sixth and the seventh was "Regarding
Construction of the Keystone XL Pipeline." These were intended to "clear
the way to government approval" of the Dakota Access and the Keystone XL
pipelines. In a meeting with small business leaders on January 30, Trump
clarified that one of the reasons for approving the pipelines was to
insist that pipeline makers implement a made-in-America approach. He
revealed how the federal government could exercise eminent domain
strategically in the appropriation of private land, to pressure pipeline
makers to use American raw steel, for example.

\section{Deregulation on environmental policies and
programs}\label{deregulation-on-environmental-policies-and-programs}

\begin{itemize}
\item
  \emph{The Repeal of the Disclosure of Payments by Resource Extraction
  Issuers Rule (115-4) was signed into law by Trump on February 14,
  2017.}
\item
  \emph{Under the Congressional Review Act Congress passed the
  resolution to repeal on February 1 and the Senate also approved it on
  February 2.}
\item
  \emph{The Repeal of Stream Protection Rule (115-5) was signed into law
  by Trump on February 16.}
\end{itemize}

Then White House Chief of Staff Reince Priebus signed an order on
January 24, temporarily delaying the Environmental Protection Agency's
(EPA) 30 final regulations that were pending in the Federal Register
until March 21, 2017." Employees in the EPA's Office of Acquisition
Management, received an email "within hours of President Trump's
swearing in", from the new EPA administration, asking "that all contract
and grant awards be temporarily suspended, effective immediately" which
included "task orders and work assignments" until "further
clarification."

On February 1, the Trump administration published a Statement of
Administration Policy to allow coal companies to dump mining waste in
streams by nullifying the Department of the Interior regulation known as
the "Stream Protection Rule", established in the Obama administration.
Under the Congressional Review Act Congress passed the resolution to
repeal on February 1 and the Senate also approved it on February 2. The
Statement nullified the Waste Prevention, Production Subject to
Royalties, and Resource Conservation which limited venting, flaring, and
leaks during oil and natural gas production. The Repeal of Stream
Protection Rule (115-5) was signed into law by Trump on February 16.

Additionally, the February 1 policy statement nullified the rule on
Disclosure of Payments by Resource Extraction Issuers, a Securities and
Exchange Commission regulation which required resource extraction
issuers to report payments "to governments for the commercial
development of oil, natural gas or minerals." The Repeal of the
Disclosure of Payments by Resource Extraction Issuers Rule (115-4) was
signed into law by Trump on February 14, 2017.

On March 29, 2017, EPA Administrator Scott Pruitt overturned the 2015
EPA revocation and denied the 2007 administrative petition by the
Natural Resources Defense Council and the Pesticide Action Network North
America (PANNA) to ban the widely used Dow Chemical Company's
chlorpyrifos. The eight-year delay by the EPA to respond to PANNA, had
resulted in a court case, PANNA v. EPA, in which EPA was ordered to
respond by October 2015. EPA revoked "all tolerances for the insecticide
chlorpyrifos" and Pruitt overturned the 2015 decision.

On March 29, 2017, EPA Administrator Scott Pruitt overturned the 2015
EPA revocation and denied the administrative petition by the Natural
Resources Defense Council and the Pesticide Action Network North America
to ban chlorpyrifos.

Accompanied by coal executives and coal miners, Trump signed a "sweeping
executive order" on March 28, at the EPA. In his remarks he praised coal
miners along with pipelines and U.S. manufacturing and addressed the
coal miners directly, "Come on, fellas. Basically, you know what this
is? You know what it says, right? You're going back to work." Trump
instructed EPA "regulators to rewrite key rules curbing U.S. carbon
emissions and other environmental regulations."

\section{Acts of the 115th United States
Congress}\label{acts-of-the-115th-united-states-congress}

\begin{itemize}
\item
  \emph{The GAO Access and Oversight Act of 2017 (Pub.L.}
\item
  \emph{72) was the second law Trump signed as President.}
\item
  \emph{By April 10, Trump had signed 21 Acts of Congress into law under
  the 115th United States Congress---laws 115-2 through 115-22.}
\end{itemize}

By April 10, Trump had signed 21 Acts of Congress into law under the
115th United States Congress---laws 115-2 through 115-22. The GAO Access
and Oversight Act of 2017 (Pub.L.~115--3,H.R. 72) was the second law
Trump signed as President. The bill ensures that the Government
Accountability Office (GAO) has full access to the database, National
Directory of New Hires, to ensure that recipients of federal
means-tested programs like Unemployment Insurance, Supplemental
Nutrition Assistance Program (SNAP), Earned income tax credit (EITC),
and Temporary Assistance for Needy Families (TANF) are eligible, thereby
reducing government waste and increasing accountability.

\section{Congressional Review Act}\label{congressional-review-act}

\begin{itemize}
\item
  \emph{By April 6, Trump had signed into law 11 resolutions of
  disapproval under the CRA, after they were passed by the Republican
  majority in the House and Senate.}
\item
  \emph{On April 13, Trump signed the law which overturned the HHS Title
  X Funding for Planned Parenthood Rule.}
\item
  \emph{On February 16, Trump signed the Repeal of Stream Protection
  Rule (H.J.Res.}
\end{itemize}

Beginning in January, the Trump administration used the 1996
Congressional Review Act (CRA) to overturn regulations---some of them
major---finalized during the final months of Obama's tenure. By April 6,
Trump had signed into law 11 resolutions of disapproval under the CRA,
after they were passed by the Republican majority in the House and
Senate. Under the Congressional Review Act, Congress can circumvent the
Senate's filibuster to overturn legislation issued in the last 60 days
of the previous administration.

On February 14, the Repeal of the Disclosure of Payments by Resource
Extraction Issuers Rule (Pub.L.~115--4, H.J.Res. 41) was signed,
nullifying the Securities and Exchange Commission regulation known as
the "Disclosure of Payments by Resource Extraction Issuers" rule. The
SEC regulation was mandated by the Dodd-Frank Wall Street Reform and
Consumer Protection Act, which was similar to transparency initiatives
adopted by the European Union and Canada. Advocates argued that
"Disclosure of Payments" rule prevented companies from bribing foreign
governments and engaging in other forms of corruption. Those who argued
for the its repeal claimed that rule had placed an excessive burden on
American companies and created a competitive disadvantage.

On February 16, Trump signed the Repeal of Stream Protection Rule
(H.J.Res. 38 Pub.L.~115--5), which nullified the DOI regulation known as
the Stream Protection Rule.

On February 28, the Repeal of the Implementation of the NICS Improvement
Amendments Act of 2007 (H.J.Res. 40 Pub.L.~115--8) was signed into law,
which overturned the Social Security Administration related to the
implementation of the NICS Improvement Amendments Act of 2007, which had
amended the National Instant Criminal Background Check System to
prohibit those with severe mental illness from possessing firearms.

On March 27, Trump overturned the Bureau of Land Management's (BLM),
which nullified the "Waste Prevention, Production Subject to Royalties,
and Resource Conservation", also known as "Methane and Waste Prevention"
or "methane venting and flaring rule" which "limited venting, flaring,
and leaks during oil and natural gas production". with Bill (H.J.Res. 44
Pub.L.~115--11) disapproved the DOI rule relating to Bureau of Land
Management "regulations that established the procedures used to prepare,
revise, or amend land use plans pursuant to the Federal Land Policy and
Management Act of 1976". On the same day, he signed the "H.J.Res.37~--
Disapproving the rule submitted by the Department of Defense, the
General Services Administration, and the National Aeronautics and Space
Administration relating to the Federal Acquisition Regulation" (H.J.Res.
37 Pub.L.~115--11), which overturned the Federal Acquisition Regulation
(FAR) "Fair Pay and Safe Workplaces"---known by its opponents as the
"Blacklisting" Rule. On March 27, he also signed the ED State and Local
Education Accountability Rules (H.J.Res. 57 Pub.L.~115--13), which
overruled the Department of Education rule relating to accountability
and State plans under the Elementary and Secondary Education Act of 1965
and the ED Teacher Preparation Rule (H.J.Res. 58 Pub.L.~115--14),
overturning the Department of Education relating to teacher preparation
issues.

On March 31, Trump signed the DOL Unemployment Insurance Drug Testing
Rule (H.J.Res. 42 Pub.L.~115--17) "disapproving the DOL rule relating to
drug testing of unemployment compensation applicants."

Trump also signed the DOL Employee Retirement Income Security Act ERISA
Exemption for State-Run Retirement Plans Rule and the DOL ERISA
Exemption for Municipality-Run Retirement Plans Rules.

On April 3, Trump signed the Occupational Safety and Health
Administration (OSHA) "Volks" Rule measure (115-21 Pub.L.~115--21) which
overturned the DOL "Clarification of Employer's Continuing Obligation to
Make and Maintain an Accurate Record of Each Recordable Injury and
Illness" enacted in December 2016. On the same day he signed Public Law
115-22 which overturned the December 2, 2016 FCC Privacy Rule relating
to "Protecting the Privacy of Customers of Broadband and Other
Telecommunications Services"\\
and the Fish and Wildlife Service (FWS) Wildlife Management Rule
(H.J.Res. 69 Pub.L.~115--20) overturning DOI rule relating to
"Non-Subsistence Take of Wildlife, and Public Participation and Closure
Procedures, on National Wildlife Refuges in Alaska." Privacy advocates
expressed concern that Internet service providers (ISPs)---including the
largest ISPs, Comcast, Verizon, AT\&T, Time Warner, Cox Communications,
and CenturyLink Charter Communications and others---will create and
monetize detailed customer data such as Internet search history and
without consent. Supporters included Republicans who regarded the rule
as executive overreach and trade groups that represent Internet service
providers.

On April 13, Trump signed the law which overturned the HHS Title X
Funding for Planned Parenthood Rule.

\section{Speech to joint session of
Congress}\label{speech-to-joint-session-of-congress}

\begin{itemize}
\item
  \emph{Trump announced the creation of the Office of Victims of
  Immigration Crime Engagement (VOICE).}
\item
  \emph{The 45th President of the United States, Donald Trump, gave his
  first public address before a joint session of the United States
  Congress on February 28, 2017.}
\end{itemize}

The 45th President of the United States, Donald Trump, gave his first
public address before a joint session of the United States Congress on
February 28, 2017. Trump announced the creation of the Office of Victims
of Immigration Crime Engagement (VOICE).

\section{Protests}\label{protests}

\begin{itemize}
\item
  \emph{Protests against Donald Trump have occurred both in the United
  States and worldwide, following Donald Trump's 2016 presidential
  campaign, his electoral win, and through his inauguration.}
\item
  \emph{Day Without Immigrants 2017 and Not My Presidents Day were held
  on February 16 and 20, respectively.}
\item
  \emph{Later protests included the Tax Day March (April 15), March for
  Science (April 22), and People's Climate Mobilization (April 29).}
\end{itemize}

Protests against Donald Trump have occurred both in the United States
and worldwide, following Donald Trump's 2016 presidential campaign, his
electoral win, and through his inauguration.

On January 21, there were large demonstrations protesting Trump
worldwide in 673 cities, with estimates for the global total at
\textasciitilde{}5 million people. About half a million demonstrated in
the Women's March on Washington (in Washington, D.C.).

Day Without Immigrants 2017 and Not My Presidents Day were held on
February 16 and 20, respectively. Later protests included the Tax Day
March (April 15), March for Science (April 22), and People's Climate
Mobilization (April 29).

\section{Rallies}\label{rallies}

\begin{itemize}
\item
  \emph{March 4 Trump rallies, organized by Trump supporters, were held
  throughout the United States on March 4.}
\end{itemize}

March 4 Trump rallies, organized by Trump supporters, were held
throughout the United States on March 4.

\section{Media coverage}\label{media-coverage}

\begin{itemize}
\item
  \emph{Trump dismissed polls that gave lower numbers.}
\item
  \emph{When asked by an Associated Press journalist about Trump's
  performance at the press conference, Trump's supporters said he came
  across as the "champion of Middle America...taking on the
  establishment and making good on his campaign promises to put the
  country first."}
\end{itemize}

On February 16, Trump held an hour-and-a-quarter-long press conference
to "update the American people on the incredible progress that has been
made in the last four weeks since my inauguration." CNN described it as
an "animated and unorthodox" intervention in which Trump appeared to be
"deeply frustrated" by the way he was being portrayed by the media. The
media has often described the administration as chaotic, while Trump
claimed it was "running like a fine-tuned machine". Trump said that "the
stock market has hit record numbers .... there has been a tremendous
surge of optimism in the business world, and ...a new Rasmussen Reports'
poll which put his "approval rating at 55 percent and going up." Trump
dismissed polls that gave lower numbers. When asked by an Associated
Press journalist about Trump's performance at the press conference,
Trump's supporters said he came across as the "champion of Middle
America...taking on the establishment and making good on his campaign
promises to put the country first."

NBC News, The Huffington Post/YouGov, Gallup, SurveyMonkey, Rasmussen
Reports, the Associated Press/NORC, Pew Research Center, Quinnipiac
University, The Economist/YouGov,The Wall Street Journal, Reuters/Ipsos,
and ABC News/The Washington Post are among the organizations undertaking
opinion polls on Trump's approval ratings.

An April meeting of thirty White House staff members---including
Communications Director, Mike Dubke, Jessica Ditto, and Kellyanne
Conway---brainstormed on how to "repackage" the symbolic First 100
Days---which ends April 29---and to "rebrand Trump" by focusing on three
main areas---prosperity, accountability and safety. The first includes
"new manufacturing jobs, reduced regulations and pulling out of the
Trans-Pacific Partnership trade deal", the second "swamp-draining
campaign promises such as lobbying restrictions" and the third "the
dramatic reduction in border crossings and the strike in Syria".
Politico summarized this period as "marred by legislative stumbles,
legal setbacks, senior staff kneecapping one another, the resignation of
his national security adviser and near-daily headlines and headaches
about links to Russia." CNN called it "largely win-less", The Atlantic
described its as a "disaster" marked by "chaos, confusion, and
infighting" comparing it to Bill Clinton's in 1993. The Washington Times
claimed the numerous mainstream media descriptions of Trump's "worst 100
days" failed to mention the accomplishments: the TPP withdrawal, the
Keystone XL and Dakota Access Pipelines approvals, the proposed
"streamlined budget" with a "Reagan-era increase to national defense",
immigration laws enforcement "which decreased illegal border crossing by
40 percent in his first month", and Gorsuch's "incredibly smooth"
nomination to the Supreme Court, the Dow Jones 20,000-point threshold,
and rebounding manufacturing and mining jobs".

\section{Sean Spicer}\label{sean-spicer}

\begin{itemize}
\item
  \emph{On February 7, CNN reported that "President Donald Trump was
  disappointed with Spicer and with Priebus, who had recommended him.}
\item
  \emph{At his first official press conference, on January 21, Spicer
  criticized the media for underestimating the size of the crowds at the
  inauguration under Trump's direct orders.}
\end{itemize}

Sean Spicer was named as Trump's White House Press Secretary on December
22, 2016, and his Communications Director on December 24. after the
resignation of Jason Miller. At his first official press conference, on
January 21, Spicer criticized the media for underestimating the size of
the crowds at the inauguration under Trump's direct orders.

On February 1, Spicer held his sixth press briefing, which for the first
time included a number of Skype Seats as Chuck Todd had suggested on
January 23. Spicer fielded questions from Kim Kalunian (WPRI) in Rhode
Island, Natalie Herbick (Fox 8) in Cleveland, Ohio, Lars Larson of the
Lars Larson Show and Jeff Jobe of Jeff Jobe Publishing, South Central
Kentucky. CBS NEWS reported that some journalists labelled their
questions as "softball", others welcomed them. Spicer had also delivered
a tense five-minute post-inauguration news conference on January 21. The
Skype solution helped resolve a concern about moving to a larger press
room. By February 13, Jim Hoft, from Gateway Pundit and the "freshly
minted White House correspondent", 28-year-old artist Lucian Wintrich,
were granted White House press credentials and attended the press
conference with Trump and the Canadian prime minister, Justin Trudeau.

On February 4, Melissa McCarthy lampooned Spicer on Saturday Night Live.
On February 7, CNN reported that "President Donald Trump was
disappointed with Spicer and with Priebus, who had recommended him.

On February 24, journalists from The New York Times, The Los Angeles
Times, CNN and Politico, The Los Angeles Times, and BuzzFeed were barred
from Sean Spicer's small, off-camera press briefing or "gaggle", held in
his office. Conservative-leaning Breitbart News, One America News
Network, and The Washington Times were invited along with Fox News,
Reuters, Bloomberg News, CBS and Hearst Communications. Reporters from
the Associated Press and Time walked out of the briefing in protest.
Media outlets allowed into the gaggle shared full details of the
briefing, including their audio, with the entire press corps. Fox News
"joined a complaint by the chair of the five-network television pool",
although their journalist was not banned. The White House Communications
Agency (WHCA) lodged a complaint. Spicer explained that the White House
is fighting against "unfair coverage."

On April 11, while defending President Trump's decision to bomb Syria,
Spicer compared President Bashar al-Assad to Adolf Hitler and stated
that even Hitler had not used chemical weapons on his own people during
World War II, ignoring the Germany's use of gas chambers during the
Holocaust. Spicer apologized on the next day, saying, "I got into a
topic that I shouldn't have, and I screwed up."

\section{Kellyanne Conway}\label{kellyanne-conway}

\begin{itemize}
\item
  \emph{By February 3, televised interviews by Kellyanne Conway,
  Counselor to the President, were dominating the news cycle in the
  First 100 Days, according to the Washington Post claiming it was
  partly because of "misconstrued facts" and "falsehoods".}
\item
  \emph{Conway promoted Ivanka Trump's business On February 9, on Fox \&
  Friends in response to Nordstrom's decision to drop her products.}
\end{itemize}

By February 3, televised interviews by Kellyanne Conway, Counselor to
the President, were dominating the news cycle in the First 100 Days,
according to the Washington Post claiming it was partly because of
"misconstrued facts" and "falsehoods". Examples include the February 2
interview on Hardball with Chris Matthews, where she cited a fictitious
incident involving two Iraqi refugees in Kentucky in 2011, who she
claimed were the "masterminds behind Bowling Green massacre which she
claimed was "brand new information" that had "very little {[}media{]}
coverage."

Conway promoted Ivanka Trump's business On February 9, on Fox \& Friends
in response to Nordstrom's decision to drop her products. Organizations
filed formal ethics complaints against Conway for violating federal law
prohibiting use of a federal position "for the endorsement of any
product, service or enterprise". Public Citizen asked the Office of
Governmental Ethics (OGE) to investigate and Citizens for Responsibility
and Ethics in Washington filed a similar complaint.

\section{Investigations into Russian interference in the
election}\label{investigations-into-russian-interference-in-the-election}

\begin{itemize}
\item
  \emph{On April 6, 2017, Nunes temporarily recused himself from the
  Russia investigation, as the House Ethics Committee began
  investigating claims that he improperly disclosed classified
  information.}
\item
  \emph{Comey stated that the FBI has no evidence that corroborates
  Trump's March 4 wiretapping claim.}
\item
  \emph{On March 22, Devin Nunes, Republican chairman of the committee,
  held a press conference to reveal that, based on classified reports he
  had seen, U.S. intelligence agencies had incidentally collected
  communications of Trump's transition team, and that Trump associates'
  names were unmasked in the reports.}
\end{itemize}

Three separate investigations on Russian interference in the 2016 United
States elections include those undertaken by the FBI, the Senate
Intelligence Committee and the House Intelligence Committee.

On March 20, in a House Intelligence Committee public hearing FBI
Director James Comey confirmed that the FBI has been conducting a broad
counter-intelligence investigation of Russian interference in the
elections starting in July 2016, which includes investigations into
possible links between Trump associates and Russia. Comey stated that
the FBI has no evidence that corroborates Trump's March 4 wiretapping
claim. On March 22, Devin Nunes, Republican chairman of the committee,
held a press conference to reveal that, based on classified reports he
had seen, U.S. intelligence agencies had incidentally collected
communications of Trump's transition team, and that Trump associates'
names were unmasked in the reports.

The next House Intelligence Committee hearings will be closed and will
include NSA Director Mike Rogers and Comey. Nunes canceled the public
hearing with "former Acting Attorney General Sally Yates, former CIA
Director John Brennan, and former Director of National Intelligence
James Clapper". On April 6, 2017, Nunes temporarily recused himself from
the Russia investigation, as the House Ethics Committee began
investigating claims that he improperly disclosed classified
information. He called the allegations "entirely false". Mike Conaway
(R-TX) will replace Nunes to lead the investigation.

\section{Re-election campaign}\label{re-election-campaign}

\begin{itemize}
\item
  \emph{On February 19, Trump explained on Twitter that his statement
  was based on a February 17 televised Fox News show about immigration
  in Sweden.}
\item
  \emph{The first rally paid for by the campaign took place at the
  Orlando Melbourne International Airport near Orlando, Florida, on
  February 18, 2017.}
\item
  \emph{During the event, Trump defended his actions as President and
  criticized the media.}
\end{itemize}

Trump filed a form with the FEC declaring his eligibility to run for
re-election in 2020 within hours of his taking office. The first rally
paid for by the campaign took place at the Orlando Melbourne
International Airport near Orlando, Florida, on February 18, 2017. The
campaign rally was the earliest such event by any incumbent U.S.
President in history. During the event, Trump defended his actions as
President and criticized the media. He also claimed that the migrant
crisis had made countries like France, Germany, and Belgium unsafe. He
added Sweden to the list and referred to an event that had happened
"last night in Sweden. Sweden, who would believe this? Sweden. They took
in large numbers. They're having problems like they never thought
possible." On February 19, Trump explained on Twitter that his statement
was based on a February 17 televised Fox News show about immigration in
Sweden.

According to The New York Times, while Trump "did not directly state,
that a terrorist attack had taken place in Sweden", "the context of his
remarks... suggested that he thought it might have." The Swedish Embassy
in Washington, DC, requested clarification from the U.S. State
Department, and the Cabinet of Sweden requested an explanation from the
White House. The Swedish Embassy in the United States tweeted to offer
to inform the US administration in the future about Swedish immigration
and integration policies. On February 23, the Swedish Ministry for
Foreign Affairs published an article attempting to evaluate the
"simplistic and occasionally completely inaccurate information" that had
been disseminated about Sweden and the Swedish migration policy.

\section{See also}\label{see-also}

\begin{itemize}
\item
  \emph{Presidential transition of Donald Trump}
\item
  \emph{Political positions of Donald Trump}
\item
  \emph{Opinion polling on the Donald Trump administration}
\end{itemize}

A Better Way, Speaker of the House Paul Ryan's plan

First 100 days of Barack Obama's presidency

First 100 days of Franklin D. Roosevelt's presidency

Presidential transition of Donald Trump

Political positions of Donald Trump

Opinion polling on the Donald Trump administration

\section{Notes}\label{notes}

\section{References}\label{references}

\section{External links}\label{external-links}

\begin{itemize}
\item
  \emph{February 1, 2017.}
\item
  \emph{Retrieved April 27, 2017.}
\item
  \emph{Retrieved February 2, 2017.}
\end{itemize}

"Press Office of the White House". Washington, DC: The White House.
February 1, 2017. Retrieved February 2, 2017.

"Speeches and Remarks". Washington, DC: The White House. February 1,
2017. Retrieved February 2, 2017.

"Statements and Releases". Washington, DC: The White House. February 1,
2017. Retrieved February 2, 2017.

"Presidential Actions". Washington, DC: The White House. February 1,
2017. Retrieved February 2, 2017.

"Reducing Regulation and Controlling Regulatory Costs". Washington, DC:
The White House. February 1, 2017. Retrieved February 2, 2017.

"Mr. President, we have some questions. Welcome to \#100Days100Qs". The
World (radio program)---Public Radio International (PRI). Retrieved
April 27, 2017.

We asked Trump a question every day for his first 100 days. Here's what
we learned.---report by Anna Pratt for Public Radio International (April
29, 2017)
