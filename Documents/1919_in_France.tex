\textbf{From Wikipedia, the free encyclopedia}

https://en.wikipedia.org/wiki/1919\_in\_France\\
Licensed under CC BY-SA 3.0:\\
https://en.wikipedia.org/wiki/Wikipedia:Text\_of\_Creative\_Commons\_Attribution-ShareAlike\_3.0\_Unported\_License

\section{1919 in France}\label{in-france}

\begin{itemize}
\item
  \emph{Events from the year 1919 in France.}
\end{itemize}

Events from the year 1919 in France.

\section{Incumbents}\label{incumbents}

\begin{itemize}
\item
  \emph{President: Raymond Poincaré}
\item
  \emph{President of the Council of Ministers: Georges Clemenceau}
\end{itemize}

President: Raymond Poincaré

President of the Council of Ministers: Georges Clemenceau

\section{Events}\label{events}

\begin{itemize}
\item
  \emph{7 July - First customer takes delivery of a Citroën automobile.}
\item
  \emph{27 November - Treaty of Neuilly-sur-Seine is signed.}
\item
  \emph{April - Long-Berenger Oil Agreement is concluded between France
  and the United Kingdom over oil rights.}
\item
  \emph{16 November - Legislative Election held.}
\item
  \emph{30 November - Legislative Election held.}
\end{itemize}

18 January - The Paris Peace Conference, opens at the Quai d'Orsay, with
delegates from 27 nations attending for meetings at the Palace of
Versailles (anniversary of the 1871 proclamation of William I as German
Emperor at Versailles); for its duration Paris is effectively the center
of a world government. On 25 January the Conference agrees to establish
the League of Nations.

April - Long-Berenger Oil Agreement is concluded between France and the
United Kingdom over oil rights.

1 May (Premier Mai) - A large left-wing demonstration leads to a violent
confrontation with the police.

28 June - Treaty of Versailles is signed, officially ending World War I
and concluding the main sessions of the Paris Peace Conference.

7 July - First customer takes delivery of a Citroën automobile.

31 July - Perfumier L'Oréal is registered by Eugène Schueller.

10 September - Treaty of Saint-Germain-en-Laye is signed, ending World
War I with Austria.

16 November - Legislative Election held.

17 November - American expatriate Sylvia Beach opens the Shakespeare and
Company bookstore in Paris.

27 November - Treaty of Neuilly-sur-Seine is signed.

30 November - Legislative Election held.

\section{Sport}\label{sport}

\begin{itemize}
\item
  \emph{27 July - Tour de France ends, won by Fermin Lambot of Belgium.}
\item
  \emph{29 June - Tour de France begins.}
\end{itemize}

29 June - Tour de France begins.

27 July - Tour de France ends, won by Fermin Lambot of Belgium.

\section{Births}\label{births}

\section{January to June}\label{january-to-june}

\begin{itemize}
\item
  \emph{8 April - André Héléna, writer (died 1972)}
\item
  \emph{13 April - René Gallice, soccer player (died 1999)}
\item
  \emph{6 January - Jacques Laurent, writer and journalist (died 2000)}
\item
  \emph{21 June - Jean Joyet, painter (died 1994)}
\item
  \emph{7 April - Jackie Sardou, actress (died 1998)}
\item
  \emph{29 April - Gérard Oury, actor, writer and producer (died 2006)}
\end{itemize}

6 January - Jacques Laurent, writer and journalist (died 2000)

19 January - Simone Melchior, wife and business partner of Jacques-Yves
Cousteau (died 1990)

18 March - Michèle Arnaud, singer, producer and director (died 1998)

1 April - Jeannie Rousseau, Allied intelligence agent (died 2017)

7 April - Jackie Sardou, actress (died 1998)

8 April - André Héléna, writer (died 1972)

13 April - René Gallice, soccer player (died 1999)

21 April - André Bettencourt, Resistance fighter, politician and
Minister (died 2007)

29 April - Gérard Oury, actor, writer and producer (died 2006)

7 June - Roger Borniche, detective

21 June - Jean Joyet, painter (died 1994)

21 June - Guy Lux, game show host and producer (died 2003)

\section{July to September}\label{july-to-september}

\begin{itemize}
\item
  \emph{19 July - Solange Troisier, physician (died 2008)}
\item
  \emph{2 July - Henri Genès, actor and singer (died 2005)}
\item
  \emph{17 July - Jean Leymarie, art historian (died 2006)}
\item
  \emph{18 July - Daniel du Janerand, painter (died 1990)}
\item
  \emph{12 September - Jean Prouff, soccer player and manager (died
  2008)}
\item
  \emph{31 July - Maurice Boitel, painter (died 2007)}
\item
  \emph{9 September - Jacques Marin, actor (died 2001)}
\end{itemize}

2 July - Albert Batteux, international soccer player and manager (died
2003)

2 July - Henri Genès, actor and singer (died 2005)

17 July - Jean Leymarie, art historian (died 2006)

18 July - Daniel du Janerand, painter (died 1990)

19 July - Solange Troisier, physician (died 2008)

31 July - Maurice Boitel, painter (died 2007)

10 August - Sacha Vierny, cinematographer (died 2001)

11 August - Ginette Neveu, violinist (died 1949)

9 September - Jacques Marin, actor (died 2001)

12 September - Jean Prouff, soccer player and manager (died 2008)

\section{October to December}\label{october-to-december}

\begin{itemize}
\item
  \emph{7 October - Georges Duby, historian (died 1996)}
\item
  \emph{5 November - Félix Gaillard, Radical politician and Prime
  Minister of France (died 1970)}
\item
  \emph{3 October - Jean Lefebvre, actor (died 2004)}
\item
  \emph{8 October - André Valmy, actor (died 2015)}
\item
  \emph{20 October - André Pousse, actor (died 2005)}
\item
  \emph{10 November - François Périer, actor (died 2002)}
\end{itemize}

3 October - Jean Lefebvre, actor (died 2004)

7 October - Georges Duby, historian (died 1996)

8 October - André Valmy, actor (died 2015)

20 October - André Pousse, actor (died 2005)

5 November - Félix Gaillard, Radical politician and Prime Minister of
France (died 1970)

10 November - François Périer, actor (died 2002)

16 November - Georges-Hilaire Dupont, Roman Catholic bishop

18 November - Andrée Borrel, World War II heroine (executed) (died 1944)

11 December - Lucien Teisseire, road bicycle racer (died 2007)

30 December - François Bordes, scientist, geologist, and archaeologist
(died 1981)

\section{Full date unknown}\label{full-date-unknown}

\begin{itemize}
\item
  \emph{Pierre Garat, civil servant in Vichy France (died 1976)}
\item
  \emph{Laure Leprieur, radio personality (died 1999)}
\end{itemize}

Pierre Garat, civil servant in Vichy France (died 1976)

Laure Leprieur, radio personality (died 1999)

\section{Deaths}\label{deaths}

\begin{itemize}
\item
  \emph{30 October - Jules Develle, politician (born 1845)}
\item
  \emph{7 November - Jean Marie Antoine de Lanessan, statesman and
  naturalist (born 1843)}
\end{itemize}

2 September - Jean-Pierre Brisset, writer (born 1837)

30 October - Jules Develle, politician (born 1845)

7 November - Jean Marie Antoine de Lanessan, statesman and naturalist
(born 1843)

3 December - Pierre-Auguste Renoir, painter (born 1841)

21 December - Louis Diémer, pianist and composer (born 1843)

\section{See also}\label{see-also}

\begin{itemize}
\item
  \emph{List of French films of 1919}
\end{itemize}

List of French films of 1919

\section{References}\label{references}
