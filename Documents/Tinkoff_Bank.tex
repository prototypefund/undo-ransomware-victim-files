\textbf{From Wikipedia, the free encyclopedia}

https://en.wikipedia.org/wiki/Tinkoff\_Bank\\
Licensed under CC BY-SA 3.0:\\
https://en.wikipedia.org/wiki/Wikipedia:Text\_of\_Creative\_Commons\_Attribution-ShareAlike\_3.0\_Unported\_License

\section{Tinkoff Bank}\label{tinkoff-bank}

\begin{itemize}
\item
  \emph{Tinkoff Bank (Russian: Тинькофф банк), formerly Tinkoff Credit
  Systems (Russian: Тинькофф Кредитные Системы) is a Russian commercial
  bank based in Moscow and founded by Oleg Tinkov in 2006.}
\item
  \emph{As of 2016{[}update{]}, Tinkoff Bank has a credit rating of B+
  on the Fitch Ratings and B2 on the Moody's Rating, and is the second
  largest provider of credit cards in Russia.}
\end{itemize}

Tinkoff Bank (Russian: Тинькофф банк), formerly Tinkoff Credit Systems
(Russian: Тинькофф Кредитные Системы) is a Russian commercial bank based
in Moscow and founded by Oleg Tinkov in 2006. The bank does not have
branches and is considered a neobank. As of 2016{[}update{]}, Tinkoff
Bank has a credit rating of B+ on the Fitch Ratings and B2 on the
Moody's Rating, and is the second largest provider of credit cards in
Russia.

\section{History}\label{history}

\begin{itemize}
\item
  \emph{Tinkov invested around \$70 million in the bank, and based the
  bank on the American Capital One bank; Tinkov took over the
  Khimmashbank corporate bank in Moscow.}
\item
  \emph{In 2015, the bank was officially renamed Tinkoff Bank, and was
  also named the Best Internet Retail Bank in Russia by the Global
  Finance magazine.}
\end{itemize}

Entrepreneur Oleg Tinkov founded Tinkoff Credit Systems in 2006, after
working with consultants from Boston Consulting Group to see if a bank
without branches could work in Russia. Tinkov invested around \$70
million in the bank, and based the bank on the American Capital One
bank; Tinkov took over the Khimmashbank corporate bank in Moscow. In
2013, Tinkoff was listed on the London Stock Exchange, raising \$1.1
billion, and in the same year, the bank was named the Bank of the Year
by the Financial Times' Banker magazine.

In 2013, a Russian named Dmitry Agarkov attempted to sue the bank for 24
million rubles (\$724,000); Agarkov had edited a 2008 credit card
agreement with the bank, and his edits had been accepted by the bank.
The legal action was later withdrawn by both the parties after an
undisclosed settlement was reached.

In 2015, the bank was officially renamed Tinkoff Bank, and was also
named the Best Internet Retail Bank in Russia by the Global Finance
magazine.

\section{Professional cycling}\label{professional-cycling}

\begin{itemize}
\item
  \emph{From 2006--2008, Tinkoff were the sponsors of the Tinkoff Credit
  Systems UCI Professional Continental cycling team.}
\item
  \emph{For the 2016 cycling season, Tinkoff Bank became the sole
  sponsors of the cycling team.}
\item
  \emph{In June 2012, Tinkoff became the co-sponsors of the Team Saxo
  Bank, with the team being renamed Saxo Bank--Tinkoff Bank (later
  Saxo-Tinkoff, Tinkoff-Saxo and Tinkoff).}
\end{itemize}

From 2006--2008, Tinkoff were the sponsors of the Tinkoff Credit Systems
UCI Professional Continental cycling team. In June 2012, Tinkoff became
the co-sponsors of the Team Saxo Bank, with the team being renamed Saxo
Bank--Tinkoff Bank (later Saxo-Tinkoff, Tinkoff-Saxo and Tinkoff). For
the 2016 cycling season, Tinkoff Bank became the sole sponsors of the
cycling team.

\section{References}\label{references}
