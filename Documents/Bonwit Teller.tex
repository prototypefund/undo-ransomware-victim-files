\textbf{From Wikipedia, the free encyclopedia}

https://en.wikipedia.org/wiki/Bonwit\%20Teller\\
Licensed under CC BY-SA 3.0:\\
https://en.wikipedia.org/wiki/Wikipedia:Text\_of\_Creative\_Commons\_Attribution-ShareAlike\_3.0\_Unported\_License

\section{Bonwit Teller}\label{bonwit-teller}

\begin{itemize}
\item
  \emph{Bonwit Teller's parent company filed for bankruptcy in 1989,
  resulting in the closure of the bulk of the company's stores.}
\item
  \emph{Bonwit Teller \& Co. was a luxury department store in New York
  City founded by Paul Bonwit in 1895 at Sixth Avenue and 18th Street,
  and later a chain of department stores.}
\item
  \emph{Despite efforts over the years to restore it, the Bonwit Teller
  brand is now defunct.}
\end{itemize}

Bonwit Teller \& Co. was a luxury department store in New York City
founded by Paul Bonwit in 1895 at Sixth Avenue and 18th Street, and
later a chain of department stores. In 1897 Edmund D. Teller was
admitted to the partnership and the store moved to 23rd Street, east of
Sixth Avenue. Bonwit specialized in high-end women's apparel at a time
when many of its competitors were diversifying their product lines, and
Bonwit Teller became noted within the trade for the quality of its
merchandise as well as the above-average salaries paid to both buyers
and executives. The partnership was incorporated in 1907 and the store
made another move, this time to the corner of Fifth Avenue and 38th
Street.

Throughout much of the twentieth century, Bonwit Teller was one of a
group of upscale department stores on Fifth Avenue that catered to the
"carriage trade". Among its most notable peers were Peck \& Peck, Saks
Fifth Avenue and B. Altman and Company.

Bonwit changed ownership frequently, particularly after 1979. Bonwit
Teller's parent company filed for bankruptcy in 1989, resulting in the
closure of the bulk of the company's stores. Despite efforts over the
years to restore it, the Bonwit Teller brand is now defunct.

\section{Distinctive features}\label{distinctive-features}

\begin{itemize}
\item
  \emph{But after Bonwit Teller took over the store in April 1930, the
  architect Ely Jacques Kahn stripped the interior of its decorations.}
\item
  \emph{The Bonwit Teller's flagship uptown building at Fifth Avenue and
  56th Street, originally known as Stewart \& Company, was a women's
  clothing store in the "new luxury retailing district", designed by
  Whitney Warren and Charles Wetmore, and opened on October 16, 1929
  with Eleanor Roosevelt in attendance.}
\end{itemize}

The Bonwit Teller's flagship uptown building at Fifth Avenue and 56th
Street, originally known as Stewart \& Company, was a women's clothing
store in the "new luxury retailing district", designed by Whitney Warren
and Charles Wetmore, and opened on October 16, 1929 with Eleanor
Roosevelt in attendance. It was described by The New York Times as a
12-story emporium of "severe, almost unornamented limestone climbing to
a ziggurat of setbacks" --- as an "antithesis" of the nearby
"conventional 1928 Bergdorf Goodman.

The "stupendously luxurious" entrance sharply contrasted the severity of
the building itself. The entrance was "like a spilled casket of gems:
platinum, bronze, hammered aluminum, orange and yellow faience, and
tinted glass backlighted at night". The American Architect magazine
described it in 1929 as "a sparkling jewel in keeping with the character
of the store."

Originally, the "interior of Stewart \& Company was just as opulent as
the entrance: murals, decorative painting, and a forest of woods:
satinwood, butternut, walnut, cherry, rosewood, bubinga, maple, ebony,
red mahogany and Persian oak." But after Bonwit Teller took over the
store in April 1930, the architect Ely Jacques Kahn stripped the
interior of its decorations.

Two more floors were added to the main building in 1938 and a
twelve-story addition was made to the 56th Street frontage in
1939.{[}citation needed{]}

Over time, the 15-foot tall limestone relief panels, depicting nearly
nude women dancing, at the top of the Fifth Avenue facade, became a
"Bonwit Teller signature". Donald Trump, who purchased the building
thanks to Genesco's CEO John L. Hanigan, wanted to begin demolition in
1980. Trump "promised the limestone reliefs" to the Metropolitan Museum
of Art. When they were "jackhammered" "to bits" the act was condemned.
Through a spokesman named "John Baron"---who turned out to be Trump
himself---Trump said that his company had obtained three independent
appraisals of the sculptures, which he claimed had found them to be
"without artistic merit." An official at the Metropolitan Museum of Art
disputed the statement, stating: "Can you imagine the museum accepting
them if they were not of artistic merit? Architectural sculpture of this
quality is rare and would have made definite sense in our collection."
In addition to the relief panels, the huge Art Deco nickel grillwork
over the entrance to the store, which had also been promised to the
museum, disappeared. Again masquerading as his own spokesman "John
Baron," Trump said, "We don't know what happened to it."

\section{History}\label{history}

\section{Founding and early history
(1880s--1946)}\label{founding-and-early-history-1880s1946}

\begin{itemize}
\item
  \emph{When Bonwit's original business failed, Bonwit bought out his
  partner and opened a new store with Edmund D. Teller in 1898 on 23d
  Street between Fifth and Sixth Avenues.}
\item
  \emph{The firm was incorporated in 1907 as Bonwit Teller \& Company
  and in 1911 relocated yet again, this time to the corner of Fifth
  Avenue and Thirty-eighth Street.}
\end{itemize}

In the late 1880s, Paul Bonwit opened a small millinery shop at Sixth
Avenue and 18th Street in Manhattan's Ladies' Mile shopping district. In
1895, which the company often referred to as the year it was founded,
Bonwit opened another store on Sixth Avenue just four blocks uptown.
When Bonwit's original business failed, Bonwit bought out his partner
and opened a new store with Edmund D. Teller in 1898 on 23d Street
between Fifth and Sixth Avenues. The firm was incorporated in 1907 as
Bonwit Teller \& Company and in 1911 relocated yet again, this time to
the corner of Fifth Avenue and Thirty-eighth Street. The firm
specialized in high-end women's apparel at a time when many of its
competitors were diversifying their product lines, and Bonwit Teller
became noted within the trade for the quality of its merchandise as well
as the above-average salaries paid to both buyers and executives.

They announced that this new location would provide consumers with:

In 1930, with the retail trade in New York City moving uptown, the store
moved again, this time to a new address on Fifth Avenue. Bonwit took up
residence in the former Stewart \& Co. building at Fifty-sixth Street,
which would remain the company's flagship store for nearly fifty years.
The building had been designed by the architectural firm Warren and
Wetmore in 1929 and redesigned the next year by Ely Jacques Kahn for
Bonwit.

The company, in need of capital, partnered with noted financier Floyd
Odlum. Odlum, who had cashed in his stock holdings just prior to the
stock market crash of 1929, was investing in firms in financial distress
and in 1934 Odlum's Atlas Corporation acquired Bonwit Teller. Odlum's
wife, Hortense, who had already been serving as a consultant, was named
president of Bonwit Teller in 1938, making her the first female
president of a major department store in the United States. The Odlums
also retained a connection to the firm's founding family, naming Paul
Bonwit's son Walter Bonwit as vice president and general manager.

For a brief time in 1939-1940, the store owned radio station WHAT in
Philadelphia.

\section{Changing ownership
(1946--1979)}\label{changing-ownership-19461979}

\begin{itemize}
\item
  \emph{At the same time, Albert M. Greenfield's Philadelphia-based
  investment company Bankers Securities Corporation acquired Bonwit
  Teller's Philadelphia stores.}
\item
  \emph{Floyd and Hortense Odlum would sell their investment in Bonwit
  Teller to Walter Hoving's Hoving Corporation.}
\item
  \emph{With Bonwit Teller, Hoving would establish a strong retail
  presence on Fifth Avenue that would also include Tiffany \& Co.}
\end{itemize}

Floyd and Hortense Odlum would sell their investment in Bonwit Teller to
Walter Hoving's Hoving Corporation. At the same time, Albert M.
Greenfield's Philadelphia-based investment company Bankers Securities
Corporation acquired Bonwit Teller's Philadelphia stores. With Bonwit
Teller, Hoving would establish a strong retail presence on Fifth Avenue
that would also include Tiffany \& Co. Although Hoving was responsible
for the significant growth of Bonwit Teller, it was ultimately this
over-expansion, along with constantly changing ownership, that led to
the firm's collapse.

The company would undergo another ownership change just ten years later
with the acquisition of Bonwit by Genesco in 1956. At the time, Genesco
was a large conglomerate operating 64 apparel and retail companies.
While Genesco's portfolio included other upscale brands, including Henri
Bendel, the company was largely known as a shoe retailer. Bonwit Teller,
which had developed a cutting edge reputation promoting a young
Christian Dior and other prominent American designers, began to lose
both its fashion and sales momentum in the mid-1950s following the
acquisition by Genesco.

\section{Branch locations}\label{branch-locations}

\begin{itemize}
\item
  \emph{During this period, Bonwit grew at a much slower pace and with a
  lower degree of coordination than its peer, Saks Fifth Avenue, which
  was roughly the same size as Bonwit in the 1950s.}
\item
  \emph{In 1961, the company added a store in Short Hills and, in 1965,
  merged with the three-store Bonwit Teller Philadelphia chain
  (Philadelphia, Wynnewood, and Jenkintown).}
\end{itemize}

Bonwit Teller had started to expand as early as 1935 when it opened a
"season branch" in Palm Beach, then in 1941 it opened a full-time branch
in White Plains. This was followed by the opening of a Boston store in
1947 in the Back Bay neighborhood. By 1958, the store had locations in
New York, Manhasset, White Plains (which it moved to
Scarsdale/Eastchester next to a large Lord \& Taylor store), Cleveland,
Chicago, and Boston (234 Berkeley Street), as well as resort shops in
Miami Beach and Palm Beach. In 1961, the company added a store in Short
Hills and, in 1965, merged with the three-store Bonwit Teller
Philadelphia chain (Philadelphia, Wynnewood, and Jenkintown). Later
branches were located in Oak Brook, Troy (MI), Palm Desert, Beverly
Hills, Bal Harbour (replacing the Lincoln Road resort shop in Miami
Beach), Kansas City, Buffalo, Syracuse, and Columbia, South Carolina.

During this period, Bonwit grew at a much slower pace and with a lower
degree of coordination than its peer, Saks Fifth Avenue, which was
roughly the same size as Bonwit in the 1950s. During this period, Bonwit
did retain a role on the development of fashion and design, most notably
helping to launch the career of Calvin Klein.

\section{Decline and bankruptcy
(1979--1990)}\label{decline-and-bankruptcy-19791990}

\begin{itemize}
\item
  \emph{Bonwit would be replaced by another short-lived department store
  venture, Galeries Lafayette.}
\item
  \emph{Bonwit was once again put on the auction block but under the
  bankruptcy plan, Hooker liquidated most of the Bonwit stores.}
\item
  \emph{Ultimately, Bonwit only lasted a short time in its new location,
  before being closed in 1990.}
\end{itemize}

Allied Stores Corporation acquired the company, with the exception of
its flagship Fifth Avenue store, in 1979. Shortly thereafter, the
company's flagship store was sold separately to Donald Trump. Trump
demolished the flagship Manhattan location in 1980 to build the first
Trump Tower and Bonwit opened a new location, around the corner from its
original store, at Fifth Avenue and 56th Street. The new location would
be attached to Trump Tower's indoor mall and was constructed by joining
several adjoining buildings. The new store, with 84,000 square feet
(7,800~m2) of space, was significantly smaller than the original Bonwit
Teller with over 225,000 square feet (20,900~m2). Ultimately, Bonwit
only lasted a short time in its new location, before being closed in
1990. Bonwit would be replaced by another short-lived department store
venture, Galeries Lafayette.

In 1986, Bonwit's parent company was sold to Canadian entrepreneur
Robert Campeau. Just a year later, in 1987, the company was sold for
\$101 million to Hooker Corporation, an Australian developer that also
controlled B. Altman \& Company. Hooker would attempt an aggressive
expansion of the company's store base from 13 to 28 but losses mounted
and the company, with 17 stores in the U.S., filed Chapter 11 bankruptcy
in August 1989. Bonwit was once again put on the auction block but under
the bankruptcy plan, Hooker liquidated most of the Bonwit stores. As a
result, there was a sharp cutback in the number of stores, from 16 to 4,
effectively putting the other 12 out of business.

\section{Post-bankruptcy (1990--2000)}\label{post-bankruptcy-19902000}

\begin{itemize}
\item
  \emph{Pyramid reportedly lost \$60 million between 1990 and 1999
  operating Bonwit Teller.}
\item
  \emph{Pyramid included a Bonwit store as one of four major anchors in
  the company's then soon-to-open Carousel Center mall in Syracuse, New
  York, which opened later that year.}
\item
  \emph{The Pyramid Company purchased the Bonwit Teller name and its
  remaining stores from bankruptcy court for \$8 million in 1990.}
\end{itemize}

The Pyramid Company purchased the Bonwit Teller name and its remaining
stores from bankruptcy court for \$8 million in 1990. Pyramid included a
Bonwit store as one of four major anchors in the company's then
soon-to-open Carousel Center mall in Syracuse, New York, which opened
later that year. The company had plans to expand the store name
throughout the company's other two dozen malls and to create a new
flagship store in Manhattan. However, these plans never materialized.
The Syracuse store, the last remaining, closed in March 2000.

Pyramid reportedly lost \$60 million between 1990 and 1999 operating
Bonwit Teller. The amount was the subject of a lawsuit alleging company
chairman Robert Congel illegally transferred \$20 million of the debt to
partners in the company's Crossgates Mall in Albany, which never housed
a Bonwit Teller store.

\section{Since 2000}\label{since-2000}

\begin{itemize}
\item
  \emph{In June 2008, it was announced that Bonwit Teller "boutiques"
  would be opening in as many as twenty locations, beginning with New
  York and Los Angeles.}
\item
  \emph{In 2005, River West Brands, a Chicago-based brand revitalization
  company, announced it had formed Avenue Brands LLC to bring back
  Bonwit Teller as a luxury brand.}
\end{itemize}

In 2005, River West Brands, a Chicago-based brand revitalization
company, announced it had formed Avenue Brands LLC to bring back Bonwit
Teller as a luxury brand. The company was seeking to use the brand to
draw attention to a line of upscale apparel and accessories.

In June 2008, it was announced that Bonwit Teller "boutiques" would be
opening in as many as twenty locations, beginning with New York and Los
Angeles. Perhaps due to the subsequent recession, this venture never
materialized.

\section{Appearances in film}\label{appearances-in-film}

\begin{itemize}
\item
  \emph{Peter Campbell, advertising account executive, returns a Bonwit
  Teller dress to its Fifth Avenue store, where he discovers that Joan
  Holloway, a former co-worker, is now employed there as a sales clerk.}
\item
  \emph{Bonwit had been out of business for five years by that time.}
\item
  \emph{Wait till you see what I snagged at Bonwit's."}
\item
  \emph{Later in the movie, it transpires that Marcie Bonwit is an
  heiress to the Bonwit Teller fortune.}
\end{itemize}

In the 1957 film Desk Set, Bunny Watson, played by Katharine Hepburn,
greets her co-workers saying "Morning, kids. Wait till you see what I
snagged at Bonwit's."

In the opening scene of the 1961 film Breakfast at Tiffany's, when
Audrey Hepburn is driving up Fifth Avenue, the Bonwit Teller store next
to Tiffany's is clearly visible including with a flag in front of it.

In the opening sequence of the 1995 film Die Hard with a Vengeance,
Bonwit's Fifth Avenue store is bombed by villain Simon Gruber. Bonwit
had been out of business for five years by that time.

In the 1978 film Oliver's Story, starring Ryan O'Neal and Candice
Bergen, Candice plays the role of Marcie Bonwit. Later in the movie, it
transpires that Marcie Bonwit is an heiress to the Bonwit Teller
fortune.

In the 1979 film Rocky II, Rocky Balboa shops at Bonwit's store in
Philadelphia as part of a spending spree sequence. Rocky purchases an
expensive leather jacket (with a tiger design on the back), a fur coat
for his wife Adrian and expensive wristwatches for his brother-in-law
Paulie.

In 2009, Bonwit Teller was written into a scene in Mad Men, a television
series that explores the world of advertising. Peter Campbell,
advertising account executive, returns a Bonwit Teller dress to its
Fifth Avenue store, where he discovers that Joan Holloway, a former
co-worker, is now employed there as a sales clerk.

In the 2013 Hallmark Channel movie Window Wonderland, a window dresser
(Chyler Leigh) explains how Salvador Dalí dressed windows at Bonwit's in
his surrealist style.

In the television show The Knick (2014: Season 1, Episode 3), Effie
Barrow, wife of hospital administrator Herman Barrow, references
shopping for herself at Bonwit Teller.

In the television show The Deuce (2017: Season 1, Episode 4), the
reporter, Sandra Washington, was noted for not being a prostitute
because she had on Bonwit Teller shoes. Per the officer, "prostitutes
don't wear shoes that nice."

\section{References}\label{references}

\section{External links}\label{external-links}

\begin{itemize}
\item
  \emph{Media related to Bonwit Teller \& Co. at Wikimedia Commons}
\end{itemize}

Media related to Bonwit Teller \& Co. at Wikimedia Commons
