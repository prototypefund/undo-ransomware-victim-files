\textbf{From Wikipedia, the free encyclopedia}

https://en.wikipedia.org/wiki/Daimler-Benz\_DB\_600\_series\\
Licensed under CC BY-SA 3.0:\\
https://en.wikipedia.org/wiki/Wikipedia:Text\_of\_Creative\_Commons\_Attribution-ShareAlike\_3.0\_Unported\_License

\section{Daimler-Benz DB 600 series}\label{daimler-benz-db-600-series}

\begin{itemize}
\item
  \emph{Later DB series engines grew in bore, stroke, and horsepower,
  including the DB 603 and DB 605, but were generally similar to the
  pattern created with the DB 600.}
\item
  \emph{The decision by the RLM to concentrate on manufacturing aircraft
  engines using fuel injection systems rather than carburetors meant
  that the DB 600 was quickly superseded by the otherwise similar DB 601
  that included direct fuel injection.}
\item
  \emph{Most newer DB engine designs used in WW2 were based around this
  engine.}
\end{itemize}

The Daimler-Benz DB 600 series were a number of German aircraft engines
designed and built before and during World War II as part of a new
generation of German engine technology. The general layout was that of a
liquid-cooled, inverted V12 engine. The design originated to a private
venture project of Daimler-Benz, the F4 engine. Most newer DB engine
designs used in WW2 were based around this engine. The decision by the
RLM to concentrate on manufacturing aircraft engines using fuel
injection systems rather than carburetors meant that the DB 600 was
quickly superseded by the otherwise similar DB 601 that included direct
fuel injection. Later DB series engines grew in bore, stroke, and
horsepower, including the DB 603 and DB 605, but were generally similar
to the pattern created with the DB 600.

\section{Development}\label{development}

\section{Origins: the F4}\label{origins-the-f4}

\begin{itemize}
\item
  \emph{These were followed by the improved F4 B, which became the
  prototype for the DB 600.}
\end{itemize}

Based on the guidelines laid down by the Reichswehrministerium (Imperial
Ministry of Defence), in 1929 Daimler-Benz begun development of a new
aero engine of the 30-litre class: a liquid-cooled inverted-Vee
12-cylinder piston engine. This became the F4, and by 1931 two
prototypes were running on the test bench. These were followed by the
improved F4 B, which became the prototype for the DB 600.

\section{Daimler-Benz DB 600}\label{daimler-benz-db-600}

\begin{itemize}
\item
  \emph{Power at sea level\\
  Kurzleistung (short-term output): 1000 PS PS for 5 minutes\\
  Kampfleistung (combat output): 900 PS PS for 30 minutes\\
  Dauerleistung (continuous output): 900 PS PS, continuous}
\item
  \emph{DB 600 A¹ and B²}
\item
  \emph{Power at Sea level\\
  Startleistung (take-off output): 1050 PS for 1 minutes\\
  Kurzleistung (short-term output): 920 PS PS for 5 minutes\\
  Dauerleistung (continuous output): 775 PS PS, continuous\\
  Power at 13,120 feet (4000 m) rated altitude\\
  Kurzleistung (short-term output): 1050 PS PS for 5 minutes\\
  Dauerleistung (continuous output): 800 PS PS, continuous{[}1{]}}
\end{itemize}

In 1933, Daimler-Benz finally received a contract to develop its new
engine and to build six examples of the DB 600. For the year after, the
DB 600 was the only German aero engine in the 30-litre class.\\
In total, 2,281 DB 600s were built. Unlike the later engines of the DB
600 series, fuel mixture was developed by carburetor.

Variants:

DB 600 A¹ and B²

Developed in 1934, the A and B variants were identical except for their
reduction gear ratios: the former was intended for the Bf 109
single-engine interceptor, the latter variant for the Bf 110 heavy
fighter.

Power at sea level\\
Kurzleistung (short-term output): 1000 PS PS for 5 minutes\\
Kampfleistung (combat output): 900 PS PS for 30 minutes\\
Dauerleistung (continuous output): 900 PS PS, continuous

DB600C¹ and D²

As the A/B, but with supercharger and 13,120 feet (4000 m) rated
altitude.

Power at Sea level\\
Kurzleistung (short-term output): 950 PS PS for 5 minutes\\
Power at 13,120 feet (4000 m) rated altitude\\
Kurzleistung (short-term output): 910 PS PS for 5 minutes\\
Kampfleistung (combat output): 850 PS PS for 30 minutes\\
Dauerleistung (continuous output): 800 PS PS, continuous{[}1{]}

DB600G¹ and H²

Developed in 1936, the DB 600G and H offered increased power output,
otherwise were similar to the C/D variants.

Power at Sea level\\
Startleistung (take-off output): 1050 PS for 1 minutes\\
Kurzleistung (short-term output): 920 PS PS for 5 minutes\\
Dauerleistung (continuous output): 775 PS PS, continuous\\
Power at 13,120 feet (4000 m) rated altitude\\
Kurzleistung (short-term output): 1050 PS PS for 5 minutes\\
Dauerleistung (continuous output): 800 PS PS, continuous{[}1{]}

¹ Reduction Gearing = 1.55

² Reduction Gearing = 1.88

\section{Daimler-Benz DB 601}\label{daimler-benz-db-601}

\begin{itemize}
\item
  \emph{As the previous DB 600s, all DB 601-series had 33.9 litre
  displacement.}
\item
  \emph{The DB 601 was a development of the DB 600G with direct fuel
  injection, which permitted better fuel economy, and eliminated the
  carburetor cutting out the engine during negative G-loads.}
\item
  \emph{The assembly of the first prototypes began in September 1934,
  designated as DB 601 V. The first powered flight of a DB 601/0 was in
  a Ju 52 testbed on 11 June 1936.}
\end{itemize}

The DB 601 was a development of the DB 600G with direct fuel injection,
which permitted better fuel economy, and eliminated the carburetor
cutting out the engine during negative G-loads. The supercharger in the
new engine was driven through a stage-less, automatically controlled
hydraulic clutch. The supercharger speed was adjusted via the slip in
the clutch, controlled by a barometric device. This solution minimized
the power loss typical to multispeed superchargers with fixed
supercharger gear ratios.

The assembly of the first prototypes began in September 1934, designated
as DB 601 V. The first powered flight of a DB 601/0 was in a Ju 52
testbed on 11 June 1936. The first prototype with the direct fuel
injection, designated as F4E, was test run in 1935, and an order for 150
engines was placed in February 1937. Serial production begun in November
1937, and ended in 1943, after 19,000 examples of all types were
produced.

As the previous DB 600s, all DB 601-series had 33.9 litre displacement.
It was installed in the Heinkel He 112 (in 1937), the Heinkel He 111
(1938), the Fieseler Fi 167 (1938), the Messerschmitt Bf 109E (1938) and
various experimental types.

\section{Daimler-Benz DB 605}\label{daimler-benz-db-605}

\section{Daimler-Benz DB 603}\label{daimler-benz-db-603}

\section{See also}\label{see-also}

\begin{itemize}
\item
  \emph{Daimler-Benz DB 605}
\item
  \emph{Daimler-Benz DB 601}
\item
  \emph{Daimler-Benz DB 600}
\item
  \emph{Daimler-Benz DB 603}
\end{itemize}

Related development

Daimler-Benz DB 600

Daimler-Benz DB 601

Daimler-Benz DB 603

Daimler-Benz DB 605

Comparable engines

Allison V-1710

Hispano-Suiza 12Y

Junkers Jumo 210

Junkers Jumo 211

Klimov M-105

Mikulin AM-35

Rolls-Royce Merlin

Related lists

List of aircraft engines

List of aircraft engines of Germany during World War II

\section{References}\label{references}

\section{External links}\label{external-links}

\begin{itemize}
\item
  \emph{Aviation History.com, DB 600 series page}
\end{itemize}

Aviation History.com, DB 600 series page
