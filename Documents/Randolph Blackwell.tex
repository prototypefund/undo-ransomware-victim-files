\textbf{From Wikipedia, the free encyclopedia}

https://en.wikipedia.org/wiki/Randolph\%20Blackwell\\
Licensed under CC BY-SA 3.0:\\
https://en.wikipedia.org/wiki/Wikipedia:Text\_of\_Creative\_Commons\_Attribution-ShareAlike\_3.0\_Unported\_License

\section{Randolph Blackwell}\label{randolph-blackwell}

\begin{itemize}
\item
  \emph{In 1976 he was given the Martin Luther King, Jr.}
\item
  \emph{Randolph T. Blackwell (born March 10, 1927 in Greensboro, North
  Carolina, died May 21, 1981) was a veteran of the Civil Rights
  Movement, serving in Martin Luther King's Southern Christian
  Leadership Conference, amongst other organizations.}
\item
  \emph{Blackwell and Moses escaped injury but Travis was shot and
  hospitalized; the shooting brought national media attention to the
  struggle in the south, energized the civil rights movement, and forced
  the Kennedy administration to investigate.}
\end{itemize}

Randolph T. Blackwell (born March 10, 1927 in Greensboro, North
Carolina, died May 21, 1981) was a veteran of the Civil Rights Movement,
serving in Martin Luther King's Southern Christian Leadership
Conference, amongst other organizations. Coretta Scott King described
him as an "unsung giant" of nonviolent social change.

In the late 1920s and early 1930s, Blackwell's father was active in
Marcus Garvey's United Negro Improvement Association; Randolph attended
association meetings with his father, and visited the prison where
Garvey was held. In 1943, inspired by hearing Ella Baker speak, he
founded a youth chapter of the NAACP in Greensboro. As a student in
sociology at North Carolina A \& T University (from which he graduated
in 1949) he made an unsuccessful run for the state assembly. He earned a
law degree from Howard University in 1953, took an assistant
professorship at Winston-Salem Teacher's College and then became an
associate professor in 1954 at Alabama A \& M College, where he taught
government.

While at Alabama A \& M, Blackwell became a leader of the 1962 student
sit-ins in nearby Huntsville, Alabama. He left academia in 1963 and
became a field director in the Voter Education Project, an organization
that promoted voter registration among blacks in the South. In March
1963, while attempting to register black voters in Greenwood,
Mississippi with Bob Moses and Jimmy Travis of the Student Nonviolent
Coordinating Committee, the car they were driving was fired on.
Blackwell and Moses escaped injury but Travis was shot and hospitalized;
the shooting brought national media attention to the struggle in the
south, energized the civil rights movement, and forced the Kennedy
administration to investigate. Blackwell became the program director of
the Southern Christian Leadership Conference in 1964, but after a
disagreement with Hosea Williams, he left the organization in 1966 and
became the director of Southern Rural Action, an anti-poverty
organization in the Deep South.

From 1977 to 1979, in the presidency of Jimmy Carter, Blackwell was
director of the Office of Minority Business Enterprise in the U.S.
Department of Commerce, but was beset there by charges of mismanagement.

In 1976 he was given the Martin Luther King, Jr. Nonviolent Peace Prize,
and in 1978 the National Bar Association gave him their Equal Justice
Award.

\section{References}\label{references}
