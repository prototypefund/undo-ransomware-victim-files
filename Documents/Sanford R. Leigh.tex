\textbf{From Wikipedia, the free encyclopedia}

https://en.wikipedia.org/wiki/Sanford\%20R.\%20Leigh\\
Licensed under CC BY-SA 3.0:\\
https://en.wikipedia.org/wiki/Wikipedia:Text\_of\_Creative\_Commons\_Attribution-ShareAlike\_3.0\_Unported\_License

\section{Sanford R. Leigh}\label{sanford-r.-leigh}

\begin{itemize}
\item
  \emph{Sanford Rose Leigh (born 1934, Bridgeport, Connecticut), also
  known as Sandy Leigh (and after his amnesia Guy Wilson) was an
  activist during the Civil Rights Movement and the director of the
  largest project in Mississippi Freedom Summer, the Hattiesburg
  Project.}
\end{itemize}

Sanford Rose Leigh (born 1934, Bridgeport, Connecticut), also known as
Sandy Leigh (and after his amnesia Guy Wilson) was an activist during
the Civil Rights Movement and the director of the largest project in
Mississippi Freedom Summer, the Hattiesburg Project.

\section{Early life}\label{early-life}

\begin{itemize}
\item
  \emph{Leigh could type 120 words a minute and his efficiency and
  competence made him invaluable to the organization.}
\item
  \emph{After the March, Leigh joined the Student Nonviolent
  Coordinating Committee in Atlanta.}
\item
  \emph{Leigh became the assistant to Bayard Rustin, when Rustin was
  organizing the 1963 March On Washington.}
\item
  \emph{Leigh was born in 1934, in Bridgeport, Connecticut to West
  Indian parents who died in an automobile accident when he was in his
  teens.}
\end{itemize}

Leigh was born in 1934, in Bridgeport, Connecticut to West Indian
parents who died in an automobile accident when he was in his teens. His
older sister and her husband assumed his care. After college, and
Reserve Officers' Training Corps, Leigh, who was fluent in five
languages, attended Army Language School at Yale, served as a
lieutenant, mostly at Fort Leonard Wood, and rose to Captain. He then
worked as a technical writer in Connecticut.

Leigh became the assistant to Bayard Rustin, when Rustin was organizing
the 1963 March On Washington. After the March, Leigh joined the Student
Nonviolent Coordinating Committee in Atlanta. In SNCC he worked at times
with Communications Director, Julian Bond, and manned the WATS-line.
WATS was SNCC's main means of communicating with the activists in the
hamlets of the South. WATS saved money and had the advantage of avoiding
putting calls through the local telephone operators, who could listen to
the calls and were often very friendly with the constabulary and the Ku
Klux Klan. Leigh could type 120 words a minute and his efficiency and
competence made him invaluable to the organization.

\section{The Hattiesburg Project}\label{the-hattiesburg-project}

\begin{itemize}
\item
  \emph{In January 1964, Leigh went to Hattiesburg, Mississippi to work
  on Freedom Day, a massive Voting Rights action in the town.}
\item
  \emph{Under Leigh, the Hattiesburg Project grew to be the largest and
  most diverse in Mississippi Freedom Summer.}
\item
  \emph{Leigh managed the program in Southeastern Mississippi.}
\item
  \emph{Mississippi Freedom Democratic Party registered Negro voters,
  who were barred from voting in Mississippi, and ran candidates
  opposing the Democratic Party nominees.}
\end{itemize}

In January 1964, Leigh went to Hattiesburg, Mississippi to work on
Freedom Day, a massive Voting Rights action in the town. Shortly
thereafter, when a SNCC Field Secretary had to leave the Hattiesburg
project, it was felt that Leigh's maturity, diplomacy and firmness made
him the best candidate for the job. He became almost a son to Mrs Lenon
E. Woods, who sponsored the project by housing the office downstairs
from her Woods Guest House, in which she lived. Hers was the only
"Negro" hotel --- the only lodging for African-American travelers --- in
all Southern Mississippi. Mrs Woods owned most of the land under the
Negro business district of Hattiesburg. She was also a silent partner as
a landowner in parts of the White downtown area, which she, as a person
of color, could not own publicly. On the eve of Freedom Day, Mrs Woods
chased off a crowd of lawmen, firemen and city officials who had come to
arrest Leigh just before the massive Voter Registration drive.

Under Leigh, the Hattiesburg Project grew to be the largest and most
diverse in Mississippi Freedom Summer. It had seven Freedom Schools, two
community centers and three libraries (persons of color could not use
the town library and had no borrowing privileges). The Freedom Summer
project provided legal services donated by lawyers from three
organizations, medical services provided by specialists who rotated
through, usually during their summer vacations, and teams of ministers
who came to work on voter registration under the direction of Rev. Bob
Beech of the National Council of Churches Ministry, which also sponsored
a local Ministers' Union.

Leigh also helped manage the U.S. Senate campaign of Mississippi Freedom
Democratic Party candidate Victoria Gray Adams who sought to oppose the
segregationist, John Stennis. Mississippi Freedom Democratic Party
registered Negro voters, who were barred from voting in Mississippi, and
ran candidates opposing the Democratic Party nominees. The campaign was
to challenge the Mississippi Democratic Party at the 1964 convention in
Atlantic City. The segregationist Democratic Party ran the state, and
MFDP sought to unseat them and show the national party that people of
color would be a voting bloc equal to the segregationists, if allowed to
register to vote.

When the Department of Economic Opportunity launched Head Start in 1965,
newspapers, segregationist congressmen, and local governments denounced
it as a Communist conspiracy. Leigh managed the program in Southeastern
Mississippi. Head Start was a natural successor to the Freedom Schools.
Funding was controlled through local governments, which tried to
sabotage the program. They refused the grants and funding. In Congress
and locally, governments struggled to wrest control from the local
people who had staffed the new program.

\section{Later life}\label{later-life}

\begin{itemize}
\item
  \emph{Leigh later worked as aide de camp for Stokeley Carmichael until
  Carmichaels' marriage to Miriam Makeba.}
\item
  \emph{Leigh relocated to New York, was employed as an Administrative
  Assistant by Bechtel, and as an organist at the Abyssinian Baptist
  Church.}
\item
  \emph{In 1972 police found Leigh in a subway in Harlem, brutally
  beaten.}
\end{itemize}

Leigh later worked as aide de camp for Stokeley Carmichael until
Carmichaels' marriage to Miriam Makeba. He then became an assistant to
Walter Washington, the first Black Mayor of Washington, DC. Leigh
relocated to New York, was employed as an Administrative Assistant by
Bechtel, and as an organist at the Abyssinian Baptist Church.

In 1972 police found Leigh in a subway in Harlem, brutally beaten. He
suffered amnesia, and his friends searched in vain for six months, until
he told Harlem Hospital social workers the name someone called him in a
dream. When he began to regain his memory he was found beaten near his
room in the YMCA in 1974. He suffered brain damage, never recovered his
memory, and was placed in adult home care.

\section{References}\label{references}
