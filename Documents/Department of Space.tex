\textbf{From Wikipedia, the free encyclopedia}

https://en.wikipedia.org/wiki/Department\%20of\%20Space\\
Licensed under CC BY-SA 3.0:\\
https://en.wikipedia.org/wiki/Wikipedia:Text\_of\_Creative\_Commons\_Attribution-ShareAlike\_3.0\_Unported\_License

\section{Department of Space}\label{department-of-space}

\begin{itemize}
\item
  \emph{The Indian space program under the DoS aims to promote the
  development and application of space science and technology for the
  socio-economic benefit of the country.}
\item
  \emph{The Department of Space (DoS) (IAST: Aṃtarikṣa Vibhāga) is an
  Indian government department responsible for administration of the
  Indian space program.}
\end{itemize}

The Department of Space (DoS) (IAST: Aṃtarikṣa Vibhāga) is an Indian
government department responsible for administration of the Indian space
program. It manages several agencies and institutes related to space
exploration and space technologies.\\
The Indian space program under the DoS aims to promote the development
and application of space science and technology for the socio-economic
benefit of the country. It includes two major satellite systems, INSAT
for communication, television broadcasting and meteorological services,
and Indian Remote Sensing Satellites (IRS) system for resources
monitoring and management. It has also developed two satellite launch
vehicles, Polar Satellite Launch Vehicle (PSLV) and Geosynchronous
Satellite Launch Vehicle (GSLV), to place IRS and INSAT class satellites
in orbit.

\includegraphics[width=5.50000in,height=4.53297in]{media/image1.jpg}\\
\emph{Organization chart showing structure of the Department of Space.}

\section{Agencies and institutes}\label{agencies-and-institutes}

\begin{itemize}
\item
  \emph{The Department of Space manages the following agencies and
  institutes:}
\item
  \emph{Indian Space Research Organisation (ISRO) -- The primary
  research and development arm of the DoS.}
\item
  \emph{Space Applications Centre (SAC), Ahmedabad.}
\item
  \emph{Indian Institute of Space Science and Technology (IIST),
  Thiruvananthapuram -- India's space university.}
\end{itemize}

The Department of Space manages the following agencies and institutes:

Indian Space Research Organisation (ISRO) -- The primary research and
development arm of the DoS.\\
Vikram Sarabhai Space Centre (VSSC), Thiruvananthapuram.\\
Liquid Propulsion Systems Centre (LPSC), Thiruvananthapuram.\\
Satish Dhawan Space Centre (SDSC-SHAR), Sriharikota.\\
ISRO Satellite Centre (ISAC), Bangalore.\\
Space Applications Centre (SAC), Ahmedabad.\\
National Remote Sensing Centre (NRSC), Hyderabad.\\
ISRO Inertial Systems Unit (IISU), Thiruvananthapuram.\\
Development and Educational Communication Unit (DECU), Ahmedabad.\\
Master Control Facility (MCF), Hassan.\\
ISRO Telemetry, Tracking and Command Network (ISTRAC), Bangalore.\\
Laboratory for Electro-Optics Systems (LEOS), Bangalore.\\
Indian Institute of Remote Sensing (IIRS), Dehradun.

Antrix Corporation -- The marketing arm of ISRO.

Physical Research Laboratory (PRL), Ahmedabad.

National Atmospheric Research Laboratory (NARL), Gadanki.

North-Eastern Space Applications Centre (NE-SAC), Umiam.

Semi-Conductor Laboratory (SCL), Mohali.

Indian Institute of Space Science and Technology (IIST),
Thiruvananthapuram -- India's space university.

\section{History}\label{history}

\begin{itemize}
\item
  \emph{In 1962, the Department of Atomic Energy (DAE) set up Indian
  National Committee for Space Research (INCOSPAR), with Dr. Vikram
  Sarabhai as chairman, to organise a national space programme.}
\item
  \emph{Dr. Kailasavadivoo Sivan is the current chairman, Space
  Commission, Secretary, Department of Space.}
\item
  \emph{The Government of India constituted the Space Commission and
  established the Department of Space (DoS) in 1972 and brought ISRO
  under DoS management on 1 June 1972.}
\end{itemize}

In 1961, the Government of India entrusted the responsibility for space
research and for the peaceful use of outer space to the Department of
Atomic Energy (DAE), then under the leadership of Dr. Homi J. Bhabha. In
1962, the Department of Atomic Energy (DAE) set up Indian National
Committee for Space Research (INCOSPAR), with Dr. Vikram Sarabhai as
chairman, to organise a national space programme.

In 1969, (INCOSPAR) was reconstituted as an advisory body under the
India National Science Academy (INSA) and the Indian Space Research
Organisation was established. The Government of India constituted the
Space Commission and established the Department of Space (DoS) in 1972
and brought ISRO under DoS management on 1 June 1972.

Dr. Kailasavadivoo Sivan is the current chairman, Space Commission,
Secretary, Department of Space. Vanditha Sharma is the Additional
Secretary of the department.

\section{See also}\label{see-also}

\begin{itemize}
\item
  \emph{Indian Space Research Organisation}
\item
  \emph{Indian Institute of Space Science and Technology}
\end{itemize}

Indian Space Research Organisation

Swami Vivekananda Planetarium, Mangalore

Indian Institute of Space Science and Technology

\section{References}\label{references}

\section{External links}\label{external-links}

\begin{itemize}
\item
  \emph{Official website of the Indian Space Research Organisation.}
\item
  \emph{Official website of the Department of Space.}
\end{itemize}

Official website of the Department of Space.

Official website of the Indian Space Research Organisation.
