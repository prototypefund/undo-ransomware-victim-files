\textbf{From Wikipedia, the free encyclopedia}

https://en.wikipedia.org/wiki/Salvador\%20Allende\\
Licensed under CC BY-SA 3.0:\\
https://en.wikipedia.org/wiki/Wikipedia:Text\_of\_Creative\_Commons\_Attribution-ShareAlike\_3.0\_Unported\_License

\section{Salvador Allende}\label{salvador-allende}

\begin{itemize}
\item
  \emph{Salvador Guillermo Allende Gossens (/ɑːˈjɛndeɪ/; American
  Spanish:~{[}salβaˈðoɾ ɡiˈʎeɾmo aˈʎende ˈɣosens{]}; 26 June 1908 -- 11
  September 1973) was a Chilean democratic socialist politician and
  physician, President of Chile from 1970 until 1973, and head of the
  Popular Unity political coalition government; he was the first ever
  Marxist to be elected president in a country with liberal democracy.}
\item
  \emph{On 11 September 1973, the military moved to oust Allende in a
  coup d'état supported by the United States Central Intelligence Agency
  (CIA).}
\end{itemize}

Salvador Guillermo Allende Gossens (/ɑːˈjɛndeɪ/; American
Spanish:~{[}salβaˈðoɾ ɡiˈʎeɾmo aˈʎende ˈɣosens{]}; 26 June 1908 -- 11
September 1973) was a Chilean democratic socialist politician and
physician, President of Chile from 1970 until 1973, and head of the
Popular Unity political coalition government; he was the first ever
Marxist to be elected president in a country with liberal democracy.

Allende's involvement in Chilean political life spanned a period of
nearly forty years, having covered the posts of senator, deputy and
cabinet minister. As a life-long committed member of the Socialist Party
of Chile, whose foundation he had actively contributed to, he
unsuccessfully ran for the national presidency in the 1952, 1958, and
1964 elections. In 1970, he won the presidency in a close three-way
race. He was elected in a run-off by Congress as no candidate had gained
a majority.

On 11 September 1973, the military moved to oust Allende in a coup
d'état supported by the United States Central Intelligence Agency (CIA).
As troops surrounded La Moneda Palace, he gave his last speech vowing
not to resign. Later that day, Allende committed suicide with an AK47
rifle gifted to him by Fidel Castro, according to an investigation
conducted by a Chilean court with the assistance of international
experts in 2011.

Following Allende's death, General Augusto Pinochet refused to return
authority to a civilian government, and Chile was later ruled by a
military junta that was in power up until 1990, ending more than four
decades of uninterrupted democratic rule. The military junta that took
over dissolved the Congress of Chile, suspended the Constitution, and
began a persecution of alleged dissidents, in which thousands of
civilians were kidnapped, tortured, and murdered.

\section{Early life}\label{early-life}

\begin{itemize}
\item
  \emph{He was the son of Salvador Allende Castro and Laura Gossens
  Uribe.}
\item
  \emph{Allende was born on 26 June 1908 in Valparaíso.}
\item
  \emph{Allende then graduated with a medical degree in 1933 from the
  University of Chile.}
\item
  \emph{Salvador Allende was of Basque and Belgian (Walloons) descent.}
\item
  \emph{Allende attended high school at the Liceo Eduardo de la Barra in
  Valparaíso.}
\end{itemize}

Allende was born on 26 June 1908 in Valparaíso. He was the son of
Salvador Allende Castro and Laura Gossens Uribe. Allende's family
belonged to the Chilean upper middle class and had a long tradition of
political involvement in progressive and liberal causes. His grandfather
was a prominent physician and a social reformist who founded one of the
first secular schools in Chile. Salvador Allende was of Basque and
Belgian (Walloons) descent.

Allende attended high school at the Liceo Eduardo de la Barra in
Valparaíso. As a teenager, his main intellectual and political influence
came from the shoe-maker Juan De Marchi, an Italian-born anarchist.
Allende was a talented athlete in his youth, being a member of the
Everton de Viña del Mar sports club (named after the more famous English
football club of the same name), where he is said to have excelled at
the long jump. Allende then graduated with a medical degree in 1933 from
the University of Chile. During his time at medical school Allende was
influenced by Professor Max Westenhofer, a German pathologist who
emphasized the social determinants of disease and social medicine.

\includegraphics[width=4.18251in,height=5.50000in]{media/image1.jpg}\\
\emph{Salvador Allende in 1964.}

\section{Political involvement up to
1970}\label{political-involvement-up-to-1970}

\begin{itemize}
\item
  \emph{In 1958, again as the FRAP candidate, Allende obtained 28.5\% of
  the vote.}
\item
  \emph{Upon entering the government, Allende relinquished his
  congressional seat for Valparaíso, which he had won in 1937.}
\item
  \emph{Allende had a close relationship with the Chilean Communist
  Party from the beginning of his political career.}
\item
  \emph{Allende co-founded a section of the Socialist Party of Chile
  (founded in 1933 with Marmaduque Grove and others) in Valparaíso and
  became its chairman.}
\end{itemize}

Allende co-founded a section of the Socialist Party of Chile (founded in
1933 with Marmaduque Grove and others) in Valparaíso and became its
chairman. He married Hortensia Bussi with whom he had three daughters.
He was a Freemason, a member of the Lodge Progreso No. 4 in Valparaíso.
In 1933, he published his doctoral thesis Higiene Mental y Delincuencia
(Crime and Mental Hygiene) in which he criticized Cesare Lombroso's
proposals.

In 1938, Allende was in charge of the electoral campaign of the Popular
Front headed by Pedro Aguirre Cerda. The Popular Front's slogan was
"Bread, a Roof and Work!" After its electoral victory, he became
Minister of Health in the Reformist Popular Front government which was
dominated by the Radicals. While serving in this position, Allende was
responsible for the passage of a wide range of progressive social
reforms, including safety laws protecting workers in the factories,
higher pensions for widows, maternity care, and free lunch programmes
for schoolchildren.

Upon entering the government, Allende relinquished his congressional
seat for Valparaíso, which he had won in 1937. Around that time, he
wrote La Realidad Médico Social de Chile (The social and medical reality
of Chile). After the Kristallnacht in Nazi Germany, Allende was one of
76 members of the Congress who sent a telegram to Adolf Hitler
denouncing the persecution of Jews. Following President Aguirre Cerda's
death in 1941, he was again elected deputy while the Popular Front was
renamed Democratic Alliance.

In 1945, Allende became senator for the Valdivia, Llanquihue, Chiloé,
Aisén and Magallanes provinces; then for Tarapacá and Antofagasta in
1953; for Aconcagua and Valparaíso in 1961; and once more for Chiloé,
Aisén and Magallanes in 1969. He became president of the Chilean Senate
in 1966. During the Fifties, Allende introduced legislation that
established the Chilean national health service, the first program in
the Americas to guarantee universal health care.

His three unsuccessful bids for the presidency (in the 1952, 1958 and
1964 elections) prompted Allende to joke that his epitaph would be "Here
lies the next President of Chile." In 1952, as candidate for the Frente
de Acción Popular (Popular Action Front, FRAP), he obtained only 5.4\%
of the votes, partly due to a division within socialist ranks over
support for Carlos Ibáñez. In 1958, again as the FRAP candidate, Allende
obtained 28.5\% of the vote. This time, his defeat was attributed to
votes lost to the populist Antonio Zamorano.

Declassified documents show that from 1962 through 1964, the CIA spent a
total of \$2.6 million to finance the campaign of Eduardo Frei and spent
\$3 million in anti-Allende propaganda "to scare voters away from
Allende's FRAP coalition". The CIA considered its role in the victory of
Frei a great success. They argued that "the financial and organizational
assistance given to Frei, the effort to keep Durán in the race, the
propaganda campaign to denigrate Allende---were 'indispensable
ingredients of Frei's success'", and they thought that his chances of
winning and the good progress of his campaign would have been doubtful
without the covert support of the Government of the United States. Thus,
in 1964 Allende lost once more as the FRAP candidate, polling 38.6\% of
the votes against 55.6\% for Christian Democrat Eduardo Frei. As it
became clear that the election would be a race between Allende and Frei,
the political right~-- which initially had backed Radical Julio Durán--
settled for Frei as "the lesser evil".

Allende had a close relationship with the Chilean Communist Party from
the beginning of his political career. On his fourth (and successful)
bid for the presidency, the Communist Party supported him as the
alternate for its own candidate, the world-renowned poet Pablo Neruda.

During his presidential term, Allende shared positions held by the
Communists, in opposition to the views of the socialists. Some argue,
however, that this was reversed at the end of his period in office.

\includegraphics[width=5.50000in,height=3.64399in]{media/image2.jpg}\\
\emph{Chilean workers marching in support of Allende in 1964.}

\section{1970 election}\label{election}

\begin{itemize}
\item
  \emph{Allende won the 1970 Chilean presidential election as leader of
  the Unidad Popular ("Popular Unity") coalition.}
\item
  \emph{Indeed, former president Jorge Alessandri had been elected in
  1958 with only 31.6\% of the popular vote, defeating Allende.}
\item
  \emph{Some critics have interpreted Allende's responses as an
  admission that signing the Statute was only a tactical move.}
\end{itemize}

Allende won the 1970 Chilean presidential election as leader of the
Unidad Popular ("Popular Unity") coalition. On 4 September 1970, he
obtained a narrow plurality of 36.2\% to 34.9\% over Jorge Alessandri, a
former president, with 27.8\% going to a third candidate (Radomiro
Tomic) of the Christian Democratic Party (PDC), whose electoral platform
was similar to Allende's. According to the Chilean Constitution of the
time, if no presidential candidate obtained a majority of the popular
vote, Congress would choose one of the two candidates with the highest
number of votes as the winner. Tradition was for Congress to vote for
the candidate with the highest popular vote, regardless of margin.
Indeed, former president Jorge Alessandri had been elected in 1958 with
only 31.6\% of the popular vote, defeating Allende.

One month after the election, on 20 October, while the Senate had still
to reach a decision and negotiations were actively in place between the
Christian Democrats and the Popular Unity, General René Schneider,
Commander in Chief of the Chilean Army, was shot resisting a kidnap
attempt by a group led by General Roberto Viaux. Hospitalized, he died
of his wounds three days later, on 23 October. Schneider was a defender
of the "constitutionalist" doctrine that the army's role is exclusively
professional, its mission being to protect the country's sovereignty and
not to interfere in politics.

General Schneider's death was widely disapproved of and, for the time,
ended military opposition to Allende, whom the congress finally chose on
24 October. On 26 October, President Eduardo Frei named General Carlos
Prats as commander in chief of the army to replace René Schneider.

Allende assumed the Presidency on 3 November 1970 after signing a
Statute of Constitutional Guarantees proposed by the Christian Democrats
in return for their support in Congress. In an extensive interview with
Régis Debray in 1972, Allende explained his reasons for agreeing to the
guarantees. Some critics have interpreted Allende's responses as an
admission that signing the Statute was only a tactical move.

\includegraphics[width=5.00000in,height=5.50000in]{media/image3.jpg}\\
\emph{President Salvador Allende in 1970}

\includegraphics[width=5.50000in,height=3.89151in]{media/image4.png}\\
\emph{Chile real wages between 1967 and 1977. Orange lines mark the
beginning and end of Allende's presidency.}

\section{Presidency}\label{presidency}

\begin{itemize}
\item
  \emph{Allende also froze all prices while raising salaries.}
\item
  \emph{The Allende Government also increased enrollment in secondary
  education from 38\% in 1970 to 51\% in 1974.}
\item
  \emph{Allende's actions were eventually declared unlawful by the
  Chilean appeals court and the government was ordered to return trucks
  to their owners.}
\end{itemize}

Upon assuming power, Allende began to carry out his platform of
implementing a socialist programme called La vía chilena al socialismo
("the Chilean Path to Socialism"). This included nationalization of
large-scale industries (notably copper mining and banking), and
government administration of the health care system, educational system
(with the help of a United States educator, Jane A. Hobson-Gonzalez from
Kokomo, Indiana), a programme of free milk for children in the schools
and shanty towns of Chile, and an expansion of the land seizure and
redistribution already begun under his predecessor Eduardo Frei
Montalva, who had nationalized between one-fifth and one-quarter of all
the properties listed for takeover. Allende also intended to improve the
socio-economic welfare of Chile's poorest citizens; a key element was to
provide employment, either in the new nationalized enterprises or on
public work projects.

In November 1970, 3,000 scholarships were allocated to Mapuche children
in an effort to integrate the indigenous minority into the educational
system, payment of pensions and grants was resumed, an emergency plan
providing for the construction of 120,000 residential buildings was
launched, all part-time workers were granted rights to social security,
a proposed electricity price increase was withdrawn, diplomatic
relations were restored with Cuba, and political prisoners were granted
an amnesty. In December that same year, bread prices were fixed, 55,000
volunteers were sent to the south of the country to teach writing and
reading skills and provide medical attention to a sector of the
population that had previously been ignored, a central commission was
established to oversee a tri-partite payment plan in which equal place
was given to government, employees and employers, and a protocol
agreement was signed with the United Centre of Workers which granted
workers representational rights on the funding board of the Social
Planning Ministry. An obligatory minimum wage for workers of all ages
(including apprentices) was established, free milk was introduced for
expectant and nursing mothers and for children between the ages of 7 and
14, free school meals were established, rent reductions were carried
out, and the construction of the Santiago subway was rescheduled so as
to serve working-class neighbourhoods first. Workers benefited from
increases in social security payments, an expanded public works program,
and a modification of the wage and salary adjustment mechanism (which
had originally been introduced in the Forties to cope with the country's
permanent inflation), while middle-class Chileans benefited from the
elimination of taxes on modest incomes and property. In addition,
state-sponsored programs distributed free food to the country's neediest
citizens, and in the countryside, peasant councils were established to
mobilise agrarian workers and small proprietors. In the government's
first budget (presented to the Chilean congress in November 1970), the
minimum taxable income level was raised, removing from the tax pool 35\%
of those who had paid taxes on earnings in the previous year. In
addition, the exemption from general taxation was raised to a level
equivalent to twice the minimum wage. Exemptions from capital taxes were
also extended, which benefitted 330,000 small proprietors. The extra
increases that Frei promised to the armed forces were also fully paid.
According to one estimate, purchasing power went up by 28\% between
October 1970 and July 1971.

The rate of inflation fell from 36.1\% in 1970 to 22.1\% in 1971, while
average real wages rose by 22.3\% during 1971. Minimum real wages for
blue-collar workers were increased by 56\% during the first quarter of
1971, while in the same period real minimum wages for white-collar
workers were increased by 23\%, a development that decreased the
differential ratio between blue- and white-collar workers' minimum wage
from 49\% (1970) to 35\% (1971). Central government expenditures went up
by 36\% in real terms, raising the share of fiscal spending in GDP from
21\% (1970) to 27\% (1971), and as part of this expansion, the public
sector engaged in a huge housing program, starting to build 76,000
houses in 1971, compared to 24,000 for 1970. During a 1971 emergency
program, over 89,000 houses were built, and during Allende's three years
as president an average of 52,000 houses were constructed annually.
Although the acceleration of inflation in 1972 and 1973 eroded part of
the initial increase in wages, they still rose (on average) in real
terms during the 1971--73 period.

Allende's first step in early 1971 was to raise minimum wages (in real
terms) for blue-collar workers by 37\%--41\% and 8\%--10\% for
white-collar workers. Educational, food, and housing assistance was
significantly expanded, with public-housing starts going up twelvefold
and eligibility for free milk extended from age 6 to age 15. A year
later, blue-collar wages were raised by 27\% in real terms and
white-collar wages became fully indexed. Price controls were also set
up, while the Allende Government introduced a system of distribution
networks through various agencies (including local committees on supply
and prices) to ensure that the new rules were adhered to by shopkeepers.

The new Minister of Agriculture, Jacques Chonchol, promised to
expropriate all estates which were larger than eighty "basic" hectares.
This promise was kept, with no farm in Chile exceeding this limit by the
end of 1972. Within eighteen months, the Latifundia (extensive
agricultural estates) had been abolished. The agrarian reform had
involved the expropriation of 3,479 properties which, added to the 1,408
properties incorporated under the Frei Government, made up some 40\% of
the total agricultural land area in the country.

Particularly in rural areas, the Allende Government launched a campaign
against illiteracy, while adult education programs expanded, together
with educational opportunities for workers. From 1971 through to 1973,
enrollments in kindergarten, primary, secondary, and postsecondary
schools all increased. The Allende Government encouraged more doctors to
begin their practices in rural and low-income urban areas, and built
additional hospitals, maternity clinics, and especially neighborhood
health centers that remained open longer hours to serve the poor.
Improved sanitation and housing facilities for low-income neighborhoods
also equalized health care benefits, while hospital councils and local
health councils were established in neighborhood health centers as a
means of democratizing the administration of health policies. These
councils gave central government civil servants, local government
officials, health service employees, and community workers the right to
review budgetary decisions.

The Allende government also sought to bring the arts (both serious and
popular) to the mass of the Chilean population by funding a number of
cultural endeavours. With eighteen-year-olds and illiterates now granted
the right to vote, mass participation in decision-making was encouraged
by the Allende government, with traditional hierarchical structures now
challenged by socialist egalitarianism. The Allende Government was able
to draw upon the idealism of its supporters, with teams of "Allendistas"
travelling into the countryside and shanty towns to perform volunteer
work. The Allende Government also worked to transform Chilean popular
culture through formal changes to school curriculum and through broader
cultural education initiatives, such as state-sponsored music festivals
and tours of Chilean folklorists and nueva canción musicians. In 1971,
the purchase of a private publishing house by the state gave rise to
"Editorial Quimantu", which became the center of the Allende
Government's cultural activities. In the space of 2 years, 12~million
copies of books, magazines, and documents (8~million of which were
books) specializing in social analysis, were published. Cheap editions
of great literary works were produced on a weekly basis, and in most
cases were sold out within a day. Culture came into the reach of the
masses for the first time, who responded enthusiastically. "Editorial
Quimantu" encouraged the establishment of libraries in community
organizations and trade unions. Through the supply of cheap textbooks,
it enabled the Left to progress through the ideological content of the
literature made available to workers.

To improve social and economic conditions for women, the Women's
Secretariat was established in 1971, which took on issues such as public
laundry facilities, public food programs, day-care centers, and women's
health care (especially prenatal care). The duration of maternity leave
was extended from 6 to 12 weeks, while the Allende Government veered the
educational system towards poorer Chileans by expanding enrollments
through government subsidies. A "democratisation" of university
education was carried out, making the system tuition-free. This led to
an 89\% rise in university enrollments between 1970 and 1973. The
Allende Government also increased enrollment in secondary education from
38\% in 1970 to 51\% in 1974. Enrollment in education reached record
levels, including 3.6 million young people, and 8~million school
textbooks were distributed among 2.6 million pupils in primary
education. An unprecedented 130,000 students were enrolled by the
universities, which became accessible to peasants and workers. The
illiteracy rate was reduced from 12\% in 1970 to 10.8\% in 1972, while
the growth in primary school enrollment increased from an annual average
of 3.4\% in the period 1966--70 to 6.5\% in 1971--1972. Secondary
education grew at a rate of 18.2\% in 1971--1972, and the average school
enrollment of children between the ages of 6 and 14 rose from 91\%
(1966--70) to 99\%.

Social spending was dramatically increased, particularly for housing,
education, and health, while a major effort was made to redistribute
wealth to poorer Chileans. As a result of new initiatives in nutrition
and health, together with higher wages, many poorer Chileans were able
to feed themselves and clothe themselves better than they had been able
to before. Public access to the social security system was increased,
while state benefits such as family allowances were raised
significantly. The redistribution of income enabled wage and salary
earners to increase their share of national income from 51.6\% (the
annual average between 1965 and 1970) to 65\% while family consumption
increased by 12.9\% in the first year of the Allende Government. In
addition, while the average annual increase in personal spending had
been 4.8\% in the period 1965--70, it reached 11.9\% in 1971. During the
first two years of Allende's presidency, state expenditure on health
rose from around 2\% to nearly 3.5\% of GDP. According to Jennifer E.
Pribble, this new spending "was reflected not only in public health
campaigns, but also in the construction of health infrastructure". Small
programs targeted at women were also experimented with, such as
cooperative laundries and communal food preparation, together with an
expansion of child-care facilities.

The National Supplementary Food Program was extended to all primary
school and to all pregnant women, regardless of their employment or
income condition. Complementary nutritional schemes were applied to
malnourished children, while antenatal care was emphasized. Under
Allende, the proportion of children under the age of 6 with some form of
malnutrition fell by 17\%. Apart from the existing Supply and Prices
councils (community-based bodies which controlled the distribution of
essential groups in working-class districts, and were a popular, not
government, initiative), community-based distribution centers and shops
were developed, which sold directly in working-class neighborhoods. The
Allende Government felt obliged to increase its intervention in
marketing activities, and state involvement in grocery distribution
reached 33\%. The CUT (central labor confederation) was accorded legal
recognition, and its membership grew from 700,000 to almost 1~million.
In enterprises in the Area of Social Ownership, an assembly of the
workers elected half of the members of the management council for each
company. These bodies replaced the former board of directors.

Minimum pensions were increased by amounts equal to two or three times
the inflation rate, and between 1970 and 1972, such pensions increased
by a total of 550\%. The incomes of 300,000 retirement pensioners were
increased by the government from one-third of the minimum salary to the
full amount. Labor insurance cover was extended to 200,000 market
traders, 130,000 small shop proprietors, 30,000 small industrialists,
small owners, transport workers, clergy, professional sportsmen, and
artesans. The public health service was improved, with the establishment
of a system of clinics in working-class neighborhoods on the peripheries
of the major cities, providing a health center for every 40,000
inhabitants. Statistics for construction in general, and house-building
in particular, reached some of the highest levels in the history of
Chile. Four million square metres were completed in 1971--72, compared
to an annual average of two-and-a-half million between 1965 and 1970.
Workers were able to acquire goods which had previously been beyond
their reach, such as heaters, refrigerators, and television sets. As
further noted by Ricardo Israel Zipper,

"By now meat was no longer a luxury, and the children of working people
were adequately supplied with shoes and clothing. The popular living
standards were improved in terms of the employment situation, social
services, consumption levels, and income distribution."

Chilean presidents were allowed a maximum term of six years, which may
explain Allende's haste to restructure the economy. Not only was a major
restructuring program organized (the Vuskovic plan), he had to make it a
success if a socialist or communist successor to Allende was going to be
elected. In the first year of Allende's term, the short-term economic
results of Minister of the Economy Pedro Vuskovic's expansive monetary
policy were highly favorable: 12\% industrial growth and an 8.6\%
increase in GDP, accompanied by major declines in inflation (down from
34.9\% to 22.1\%) and unemployment (down to 3.8\%). However, by 1972,
the Chilean escudo had an inflation rate of 140\%. The average Real GDP
contracted between 1971 and 1973 at an annual rate of 5.6\% ("negative
growth"); and the government's fiscal deficit soared while foreign
reserves declined. The combination of inflation and government-mandated
price-fixing, together with the "disappearance" of basic commodities
from supermarket shelves, led to the rise of black markets in rice,
beans, sugar, and flour. The Chilean economic situation was also
somewhat exacerbated due to a US-backed campaign to fund worker strikes
in certain sectors of the economy. The Allende government announced it
would default on debts owed to international creditors and foreign
governments. Allende also froze all prices while raising salaries. His
implementation of these policies was strongly opposed by landowners,
employers, businessmen and transporters associations, and some civil
servants and professional unions. The rightist opposition was led by the
National Party, the Roman Catholic Church (which in 1973 was displeased
with the direction of educational policy), and eventually the Christian
Democrats. There were growing tensions with foreign multinational
corporations and the government of the United States.

Allende also undertook the pioneeristic Project Cybersyn, a distributed
decision support system for decentralized economic planning, developed
by British cybernetics expert Stafford Beer. Based on the experimental
viable system model and the neural network approach to organizational
design, the Project consisted of four modules: a network of telex
machines (Cybernet) in all state-run enterprises that would transmit and
receive information with the government in Santiago. Information from
the field would be fed into statistical modeling software (Cyberstride)
that would monitor production indicators, such as raw material supplies
or high rates of worker absenteeism, in "almost" real time, alerting the
workers in the first case and, in abnormal situations, if those
parameters fell outside acceptable ranges by a very large degree, also
the central government. The information would also be input into an
economic simulation software (CHECO, for CHilean ECOnomic simulator)
which featured a Bayesian filtering and control setting that the
government could use to forecast the possible outcome of economic
decisions. Finally, a sophisticated operations room (Opsroom) would
provide a space where managers could see relevant economic data,
formulate feaseble responses to emergencies, and transmit advice and
directives to enterprises and factories in alarm situations by using the
telex network. In conjunction with the system, it was also planned by
the Cybersyn development team the so-called Cyberfolk device system, a
closed television circuit connected to an interactive apparatus that
would enable the citizenry to actively participate in economic and
political decision-making.

In 1971, Chile re-established diplomatic relations with Cuba, joining
Mexico and Canada in rejecting a previously established Organization of
American States convention prohibiting governments in the Western
Hemisphere from establishing diplomatic relations with Cuba. Shortly
afterward, Cuban president Fidel Castro made a month-long visit to
Chile. Originally the visit was supposed to be one week; however, Castro
enjoyed Chile and one week led to another.

In October 1972, the first of what were to be a wave of strikes was led
first by truckers, and later by small businessmen, some (mostly
professional) unions and some student groups. Other than the inevitable
damage to the economy, the chief effect of the 24-day strike was to
induce Allende to bring the head of the army, general Carlos Prats, into
the government as Interior Minister. Allende also instructed the
government to begin requisitioning trucks in order to keep the nation
from coming to a halt. Government supporters also helped to mobilize
trucks and buses but violence served as a deterrent to full
mobilization, even with police protection for the strike-breakers.
Allende's actions were eventually declared unlawful by the Chilean
appeals court and the government was ordered to return trucks to their
owners.

Throughout this presidency racial tensions between the poor descendants
of indigenous people, who supported Allende's reforms, and the white
elite increased.

Allende raised wages on a number of occasions throughout 1970 and 1971,
but these wage hikes were negated by the in-tandem inflation of Chile's
fiat currency. Although price rises had also been high under Frei (27\%
a year between 1967 and 1970), a basic basket of consumer goods rose by
120\% from 190 to 421 escudos in one month alone, August 1972. In the
period 1970--72, while Allende was in government, exports fell 24\% and
imports rose 26\%, with imports of food rising an estimated 149\%.

Export income fell due to a hard-hit copper industry: the price of
copper on international markets fell by almost a third, and
post-nationalization copper production fell as well. Copper is Chile's
single most important export (more than half of Chile's export receipts
were from this sole commodity). The price of copper fell from a peak of
\$66 per ton in 1970 to only \$48--9 in 1971 and 1972. Chile was already
dependent on food imports, and this decline in export earnings coincided
with declines in domestic food production following Allende's agrarian
reforms.

Throughout his presidency, Allende remained at odds with the Chilean
Congress, which was dominated by the Christian Democratic Party. The
Christian Democrats (who had campaigned on a socialist platform in the
1970 elections, but drifted away from those positions during Allende's
presidency, eventually forming a coalition with the National Party)
{[}citation needed{]}, continued to accuse Allende of leading Chile
toward a Cuban-style dictatorship, and sought to overturn many of his
more radical policies. Allende and his opponents in Congress repeatedly
accused each other of undermining the Chilean Constitution and acting
undemocratically.

Allende's increasingly bold socialist policies (partly in response to
pressure from some of the more radical members within his coalition),
combined with his close contacts with Cuba, heightened fears in
Washington. The Nixon administration continued exerting economic
pressure on Chile via multilateral organizations, and continued to back
Allende's opponents in the Chilean Congress. Almost immediately after
his election, Nixon directed CIA and U.S. State Department officials to
"put pressure" on the Allende government. His economic policies were
used by economists Rudi Dornbusch and Sebastian Edwards to coin the term
macroeconomic populism.

\section{Foreign relations during Allende's
presidency}\label{foreign-relations-during-allendes-presidency}

\begin{itemize}
\item
  \emph{{[}citation needed{]} Allende's government was disappointed that
  it received far less economic assistance from the USSR than it hoped
  for.}
\item
  \emph{When Allende visited the USSR in late 1972 in search of more aid
  and additional lines of credit, after 3 years, he was turned down.}
\item
  \emph{Allende's Popular Unity government tried to maintain normal
  relations with the United States.}
\end{itemize}

Allende's Popular Unity government tried to maintain normal relations
with the United States. But when Chile nationalized its copper industry,
Washington cut off United States credits and increased its support to
opposition. Forced to seek alternative sources of trade and finance,
Chile gained commitments from the Soviet Union to invest some
\$400~million in Chile in the next six years.{[}citation needed{]}
Allende's government was disappointed that it received far less economic
assistance from the USSR than it hoped for. Trade between the two
countries did not significantly increase and the credits were mainly
linked to the purchase of Soviet equipment. Moreover, credits from the
Soviet Union were much less than those provided to the People's Republic
of China and countries of Eastern Europe. When Allende visited the USSR
in late 1972 in search of more aid and additional lines of credit, after
3 years, he was turned down.

\section{US involvement}\label{us-involvement}

\begin{itemize}
\item
  \emph{In fact, open US military aid to Chile continued during the
  Allende administration, and the national government was very much
  aware of this, although there is no record that Allende himself
  believed that such assistance was anything but beneficial to Chile.}
\item
  \emph{The United States opposition to Allende started several years
  before he was elected President of Chile.}
\item
  \emph{In September 1970, President Nixon informed the CIA that an
  Allende government in Chile would not be acceptable and authorized
  \$10~million to stop Allende from coming to power or unseat him.}
\end{itemize}

The United States opposition to Allende started several years before he
was elected President of Chile. Declassified documents show that from
1962 through 1964, the CIA spent \$3 million in anti-Allende propaganda
"to scare voters away from Allende's FRAP coalition", and spent a total
of \$2.6 million to finance the presidential campaign of Eduardo Frei.

The possibility of Allende winning Chile's 1970 election was deemed a
disaster by a US administration that wanted to protect US geopolitical
interests by preventing the spread of Communism during the Cold War. In
September 1970, President Nixon informed the CIA that an Allende
government in Chile would not be acceptable and authorized \$10~million
to stop Allende from coming to power or unseat him. A CIA document
declared, "It is firm and continuing policy that Allende be overthrown
by a coup." Henry Kissinger's 40 Committee and the CIA planned to impede
Allende's investiture as President of Chile with covert efforts known as
"Track I" and "Track II"; Track I sought to prevent Allende from
assuming power via so-called "parliamentary trickery", while under the
Track II initiative, the CIA tried to convince key Chilean military
officers to carry out a coup.

Additionally, some point to the involvement of the Defense Intelligence
Agency agents that allegedly secured the missiles used to bombard La
Moneda Palace. In fact, open US military aid to Chile continued during
the Allende administration, and the national government was very much
aware of this, although there is no record that Allende himself believed
that such assistance was anything but beneficial to Chile.

During Nixon's presidency, United States officials attempted to prevent
Allende's election by financing political parties aligned with
opposition candidate Jorge Alessandri and supporting strikes in the
mining and transportation sectors. After the 1970 election, the Track I
operation attempted to incite Chile's outgoing president, Eduardo Frei
Montalva, to persuade his party (PDC) to vote in Congress for
Alessandri. Under the plan, Alessandri would resign his office
immediately after assuming it and call new elections. Eduardo Frei would
then be constitutionally able to run again (since the Chilean
Constitution did not allow a president to hold two consecutive terms,
but allowed multiple non-consecutive ones), and presumably easily defeat
Allende. The Chilean Congress instead chose Allende as President, on the
condition that he would sign a "Statute of Constitutional Guarantees"
affirming that he would respect and obey the Chilean Constitution and
that his reforms would not undermine any of its elements.

Track II was aborted, as parallel initiatives already underway within
the Chilean military rendered it moot.

During the second term of office of Democratic President Bill Clinton,
the CIA acknowledged having played a role in Chilean politics before the
coup, but its degree of involvement is debated. The CIA was notified by
its Chilean contacts of the impending coup two days in advance but
contends it "played no direct role in" the coup.

Much of the internal opposition to Allende's policies came from the
business sector, and recently released United States government
documents confirm that the United States indirectly funded the truck
drivers' strike, which exacerbated the already chaotic economic
situation before the coup.

The most prominent United States corporations in Chile before Allende's
presidency were the Anaconda and Kennecott copper companies and ITT
Corporation, International Telephone and Telegraph. Both copper
corporations aimed to expand privatized copper production in the city of
Sewell in the Chilean Andes, where the world's largest underground
copper mine "El Teniente", was located. At the end of 1968, according to
US Department of Commerce data, United States corporate holdings in
Chile amounted to \$964~million. Anaconda and Kennecott accounted for
28\% of United States holdings, but ITT had by far the largest holding
of any single corporation, with an investment of \$200~million in Chile.
In 1970, before Allende was elected, ITT owned 70\% of Chitelco, the
Chilean Telephone Company and funded El Mercurio, a Chilean right-wing
newspaper. Documents released in 2000 by the CIA confirmed that before
the elections of 1970, ITT gave \$700,000 to Allende's conservative
opponent, Jorge Alessandri, with help from the CIA on how to channel the
money safely. ITT president Harold Geneen also offered \$1~million to
the CIA to help defeat Allende in the elections.

After General Pinochet assumed power, United States Secretary of State
Henry Kissinger told President Richard Nixon that the United States
"didn't do it", but "we helped them...created the conditions as great as
possible". (referring to the coup itself). Recent documents declassified
under the Clinton administration's Chile Declassification Project show
that the United States government and the CIA sought to overthrow
Allende in 1970 immediately before he took office ("Project FUBELT").
Many documents regarding the United States intervention in Chile remain
classified.

\section{Relationships with the Soviet
Union}\label{relationships-with-the-soviet-union}

\begin{itemize}
\item
  \emph{Andrew alleges that the KGB said that Allende "was made to
  understand the necessity of reorganizing Chile's army and intelligence
  services, and of setting up a relationship between Chile's and the
  USSR's intelligence services".}
\item
  \emph{For instance, Allende received the Lenin Peace Prize from the
  Soviet Union in 1972.}
\end{itemize}

Political and moral support came mostly through the Communist Party and
unions of the Soviet Union. For instance, Allende received the Lenin
Peace Prize from the Soviet Union in 1972. However, there were some
fundamental differences between Allende and Soviet political analysts,
who believed that some violence -- or measures that those analysts
"theoretically considered to be just" -- should have been used.
Declarations from KGB General Nikolai Leonov, former Deputy Chief of the
First Chief Directorate of the KGB, confirmed that the Soviet Union
supported Allende's government economically, politically and militarily.
Leonov stated in an interview at the Chilean Center of Public Studies
(CEP) that the Soviet economic support included over \$100~million in
credit, three fishing ships (that distributed 17,000 tons of frozen fish
to the population), factories (as help after the 1971 earthquake), 3,100
tractors, 74,000 tons of wheat and more than a million tins of condensed
milk.

In mid-1973 the USSR had approved the delivery of weapons (artillery,
tanks) to the Chilean Army. However, when news of an attempt from the
Army to depose Allende through a coup d'état reached Soviet officials,
the shipment was redirected to another country.

Allende is mentioned in a book written by the official historian of the
British Intelligence MI5 Christopher Andrew. According to SIS and
Andrew, the book is based on the handwritten notes of KGB archivist
defector Vasili Mitrokhin. The book also named several Italians of the
left as informants or KGB agents, and the right wing Prime Minister of
the time, Silvio Berlusconi, opened an investigation to target his
opponents. As Mitrokhin's information was very old, and most of the
people in his files were dead or retired, he failed to find any evidence
that any of the accused by the book were KGB agents or informants. Some
Italian Ministers dismissed the archive as "not a dossier from the KGB
but one about the KGB constructed by British counter-espionage agents
based on the confession of an ex-agent, if there is one, and 'Mitrokhin'
is just a codename for an MI5 operation". The Indigenous Congress party
referred to the book as "pure sensationalism not even remotely based on
facts or records" and pointed out that the book is not based on any
official documents from the Soviet Union. Additionally, many scholars or
experts in the field are skeptical about the reliability of Vasili
Mitrokhin's claims, and believe that the origin of the source is
doubtful or mysterious.

Andrew alleges that the KGB said that Allende "was made to understand
the necessity of reorganizing Chile's army and intelligence services,
and of setting up a relationship between Chile's and the USSR's
intelligence services". The Soviet Union observed closely whether this
alternative form of socialism could work, and they did not interfere
with the Chileans' decisions. Nikolai Leonov affirms that whenever he
tried to give advice to Latin American leaders he was usually turned
down by them, and he was told that they had their own understanding on
how to conduct political business in their countries. Leonov adds that
the relationships of KGB agents with Latin American leaders did not
involve intelligence, because their intelligence target was the United
States. Since many North Americans were living in the region, they were
focusing in recruiting agents from the United States. Latin America was
also a good region for KGB agents to get in touch with their informants
from the CIA or other contacts from the United States than inside that
country.

\section{Crisis}\label{crisis}

\begin{itemize}
\item
  \emph{On 9 August, President Allende appointed General Carlos Prats as
  Minister of Defence.}
\item
  \emph{For months, Allende had feared calling upon the Carabineros
  ("Carabineers", the national police force), suspecting them of
  disloyalty to his government.}
\item
  \emph{In August 1973, a constitutional crisis occurred, and the
  Supreme Court of Chile publicly complained about the inability of the
  Allende government to enforce the law of the land.}
\end{itemize}

On 29 June 1973, Colonel Roberto Souper surrounded the presidential
palace, La Moneda, with his tank regiment but failed to depose the
government. That failed coup d'état -- known as the Tanquetazo ("tank
putsch") -- organised by the nationalist Patria y Libertad paramilitary
group, was followed by a general strike at the end of July that included
the copper miners of El Teniente.

In August 1973, a constitutional crisis occurred, and the Supreme Court
of Chile publicly complained about the inability of the Allende
government to enforce the law of the land. On 22 August, the Chamber of
Deputies (with the Christian Democrats uniting with the National Party)
accused the government of unconstitutional acts through Allende's
refusal to promulgate constitutional amendments, already approved by the
Chamber, which would have prevented his government from continuing his
massive nationalization plan and called upon the military to enforce
constitutional order.

For months, Allende had feared calling upon the Carabineros
("Carabineers", the national police force), suspecting them of
disloyalty to his government. On 9 August, President Allende appointed
General Carlos Prats as Minister of Defence. On 24 August 1973, General
Prats was forced to resign both as defense minister and as the
commander-in-chief of the army, embarrassed by both the Alejandrina Cox
incident and a public protest in front of his house by the wives of his
generals. General Augusto Pinochet replaced him as Army
commander-in-chief the same day.

According to Chilean political scientist Arturo Valenzuela (later
becoming a United States citizen and Assistant Secretary of State for
Hemispheric Affairs in the Obama administration), a greater share of the
blame for the breakdown in Chilean democracy lay with the leftist
Allende government. While each side increasingly distrusted the other,
the extreme leftists accelerated the process and left less room for
political moderation than the extreme rightists. He writes "By its
actions, the revolutionary Left, which had always ridiculed the
possibility of a socialist transformation through peaceful means, was
engaged in a self-fulfilling prophecy."

\section{Supreme Court's resolution}\label{supreme-courts-resolution}

\begin{itemize}
\item
  \emph{On 26 May 1973, the Supreme Court of Chile unanimously denounced
  the Allende government's disruption of the legality of the nation in
  its failure to uphold judicial decisions, because of its continual
  refusal to permit police execution of judicial decisions contrary to
  the government's own measures.}
\end{itemize}

On 26 May 1973, the Supreme Court of Chile unanimously denounced the
Allende government's disruption of the legality of the nation in its
failure to uphold judicial decisions, because of its continual refusal
to permit police execution of judicial decisions contrary to the
government's own measures.

\section{Chamber of Deputies'
resolution}\label{chamber-of-deputies-resolution}

\begin{itemize}
\item
  \emph{Specifically, the Socialist government of President Allende was
  accused of:}
\end{itemize}

On 22 August 1973, the Christian Democrats and the National Party
members of the Chamber of Deputies joined together to vote 81 to 47 in
favor of a resolution that asked the authorities to "put an immediate
end" to "breach{[}es of{]} the Constitution\ldots{}with the goal of
redirecting government activity toward the path of law and ensuring the
Constitutional order of our Nation, and the essential underpinnings of
democratic co-existence among Chileans."

The resolution declared that Allende's government sought "to conquer
absolute power with the obvious purpose of subjecting all citizens to
the strictest political and economic control by the State... {[}with{]}
the goal of establishing... a totalitarian system" and claimed that the
government had made "violations of the Constitution... a permanent
system of conduct." Essentially, most of the accusations were about
disregard by the Socialist government of the separation of powers, and
arrogating legislative and judicial prerogatives to the executive branch
of government.

Specifically, the Socialist government of President Allende was accused
of:

Ruling by decree, thwarting the normal legislative system

Refusing to enforce judicial decisions against its partisans; not
carrying out sentences and judicial resolutions that contravened its
objectives

Ignoring the decrees of the independent General Comptroller's Office

Sundry media offenses; usurping control of the National Television
Network and applying economic pressure against those media organizations
that are not unconditional supporters of the government

Allowing its Socialist supporters to assemble with arms, and preventing
the same by its right-wing opponents

Supporting more than 1,500 illegal takeovers of farms

Illegal repression of the El Teniente miners' strike

Illegally limiting emigration

Finally, the resolution condemned the creation and development of
government-protected {[}socialist{]} armed groups, which were said to be
"headed towards a confrontation with the armed forces". President
Allende's efforts to re-organize the military and the police forces were
characterized as "notorious attempts to use the armed and police forces
for partisan ends, destroy their institutional hierarchy, and
politically infiltrate their ranks".

\section{President Allende's
response}\label{president-allendes-response}

\begin{itemize}
\item
  \emph{President Allende wrote: "Chilean democracy is a conquest by all
  of the people.}
\end{itemize}

Two days later, on 24 August 1973, President Allende responded,
characterising the Congress's declaration as "destined to damage the
country's prestige abroad and create internal confusion", predicting "It
will facilitate the seditious intention of certain sectors." He noted
that the declaration (passed 81--47 in the Chamber of Deputies) had not
obtained the two-thirds Senate majority "constitutionally required" to
convict the president of abuse of power: essentially, the Congress were
"invoking the intervention of the armed forces and of Order against a
democratically-elected government" and "subordinat{[}ing{]} political
representation of national sovereignty to the armed institutions, which
neither can nor ought to assume either political functions or the
representation of the popular will."

Allende argued he had obeyed constitutional means for including military
men to the cabinet at the service of civic peace and national security,
defending republican institutions against insurrection and terrorism. In
contrast, he said that Congress was promoting a coup d'état or a civil
war with a declaration full of affirmations that had already been
refuted beforehand and which, in substance and process (directly handing
it to the ministers rather than directly handing it to the President)
violated a dozen articles of the (then-current) Constitution. He further
argued that the legislature was usurping the government's executive
function.

President Allende wrote: "Chilean democracy is a conquest by all of the
people. It is neither the work nor the gift of the exploiting classes,
and it will be defended by those who, with sacrifices accumulated over
generations, have imposed it...With a tranquil conscience...I sustain
that never before has Chile had a more democratic government than that
over which I have the honor to preside...I solemnly reiterate my
decision to develop democracy and a state of law to their ultimate
consequences...Congress has made itself a bastion against the
transformations...and has done everything it can to perturb the
functioning of the finances and of the institutions, sterilizing all
creative initiatives."

Adding that economic and political means would be needed to relieve the
country's current crisis, and that the Congress were obstructing said
means; having already paralyzed the State, they sought to destroy it. He
concluded by calling upon the workers, all democrats and patriots to
join him in defending the Chilean Constitution and the revolutionary
process.

\section{Coup}\label{coup}

\begin{itemize}
\item
  \emph{On 11 September 1973, the Chilean military, aided by the United
  States and its CIA, staged a coup against Allende.}
\item
  \emph{In early September 1973, Allende floated the idea of resolving
  the constitutional crisis with a plebiscite.}
\end{itemize}

In early September 1973, Allende floated the idea of resolving the
constitutional crisis with a plebiscite.{[}citation needed{]} His speech
outlining such a solution was scheduled for 11 September, but he was
never able to deliver it. On 11 September 1973, the Chilean military,
aided by the United States and its CIA, staged a coup against Allende.

\section{Death}\label{death}

\begin{itemize}
\item
  \emph{According to Isabel Allende Bussi---the daughter of Salvador
  Allende and currently a member of the Chilean Senate---the Allende
  family has long accepted that the former President shot himself,
  telling the BBC that: "The report conclusions are consistent with what
  we already believed.}
\item
  \emph{Alternative views regarding the death of Salvador Allende:}
\end{itemize}

Just before the capture of La Moneda (the Presidential Palace), with
gunfire and explosions clearly audible in the background, Allende gave
his farewell speech to Chileans on live radio, speaking of himself in
the past tense, of his love for Chile and of his deep faith in its
future. He stated that his commitment to Chile did not allow him to take
an easy way out, and he would not be used as a propaganda tool by those
he called "traitors" (he refused an offer of safe passage), clearly
implying he intended to fight to the end.

Shortly afterwards, the coup plotters announced that Allende had
committed suicide. An official announcement declared that the weapon he
had used was an automatic rifle. Before his death he had been
photographed several times holding an AK-47, a gift from Fidel Castro.
He was found dead with this gun, according to contemporaneous statements
made by officials in the Pinochet regime.

Lingering doubts regarding the manner of Allende's death persisted
throughout the period of the Pinochet regime. Many Chileans and
independent observers refused to accept on faith the government's
version of events amid speculation that Allende had been murdered by
government agents. When in 2011 a Chilean court opened a criminal
investigation into the circumstances of Allende's death, Pinochet had
long since left power and Chile had meanwhile become one of the most
stable democracies in the Americas according to The Economist magazine's
democracy index.

The ongoing criminal investigation led to a May 2011 court order that
Allende's remains be exhumed and autopsied by an international team of
experts. Results of the autopsy were officially released in mid-July
2011. The team of experts concluded that the former president had shot
himself with an AK-47 assault rifle. In December 2011 the judge in
charge of the investigation affirmed the experts' findings and ruled
Allende's death a suicide. On 11 September 2012, the 39th anniversary of
Allende's death, a Chilean appeals court unanimously upheld the trial
court's ruling, officially closing the case.

The Guardian, a leading UK newspaper, reported that a scientific autopsy
of the remains had confirmed that "Salvador Allende committed suicide
during the 1973 coup that toppled his socialist government." The
Guardian went on to say that:

According to Isabel Allende Bussi---the daughter of Salvador Allende and
currently a member of the Chilean Senate---the Allende family has long
accepted that the former President shot himself, telling the BBC that:
"The report conclusions are consistent with what we already believed.
When faced with extreme circumstances, he made the decision of taking
his own life, instead of being humiliated."

The definitive and unanimous results produced by the 2011 Chilean
judicial investigation appear to have laid to rest decades of nagging
suspicions that Allende might have been assassinated by the Chilean
Armed Forces. But public acceptance of the suicide theory had already
been growing for much of the previous decade. In a post-junta Chile
where restrictions on free speech were steadily eroding, independent and
seemingly reliable witnesses at last began to tell their stories to the
news media and to human rights researchers. The cumulative weight of the
facts reported by these witnesses provided factual support for many
previously unconfirmed details relating to Allende's death.

The widespread acceptance of suicide as the cause Salvador Allende's
death was, however, preceded by decades of speculation and
controversializing about the circumstances surrounding his death.
Several examples of pre-2011 speculation are shown below or on the
Wikipedia page regarding the Death of Salvador Allende.

Alternative views regarding the death of Salvador Allende:

On 31 May 2011 TVN, the state television station, reported the discovery
of a secret 300-page military account of Allende's death. The document
had been kept in the home of a former military justice official, and was
discovered when his house was destroyed in the 2010 earthquake. After
reviewing the report, two forensic experts told TVN "that they are
inclined to conclude that Allende was assassinated".

Forensic expert Luis Ravanal has been studying Allende's autopsy since
2007. Ravanal says he found details in the 1973 autopsy report are not
consistent with suicide and that the wounds on the body were likely
caused by more than one gun. The cranium, he says, shows evidence of a
first shot with a small gun, like a pistol, and then, a second shot from
a larger weapon---like an AK-47---which could mean that Allende was shot
and killed, then shot a second time with his own gun, to make it look
like suicide.

In his 2004 documentary Salvador Allende, Patricio Guzmán incorporates a
graphic image of Allende's corpse in the position it was found after his
death. According to Guzmán's documentary, Allende shot himself with a
pistol and not a rifle.

\includegraphics[width=4.56867in,height=5.50000in]{media/image5.jpg}\\
\emph{An East German stamp commemorating Allende}

\section{Family}\label{family}

\begin{itemize}
\item
  \emph{Well-known relatives of Salvador Allende include his daughter
  Isabel Allende (a politician) and his first cousin once removed Isabel
  Allende (a writer).}
\end{itemize}

Well-known relatives of Salvador Allende include his daughter Isabel
Allende (a politician) and his first cousin once removed Isabel Allende
(a writer).

\section{Memorials}\label{memorials}

\begin{itemize}
\item
  \emph{Memorials to Allende include a statue in front of the Palacio de
  la Moneda.}
\item
  \emph{Allende Avenue in Harlow, Essex, is named after him.}
\item
  \emph{There is a square named after him in Viladecans, near Barcelona,
  called Plaza de Salvador Allende.}
\item
  \emph{The Salvador Allende Port is located near downtown Managua.}
\item
  \emph{Allende is buried in the general cemetery of Santiago.}
\end{itemize}

Memorials to Allende include a statue in front of the Palacio de la
Moneda. The placement of the statue was controversial; it was placed
facing the eastern edge of the Plaza de la Ciudadanía, a plaza which
contains memorials to a number of Chilean statesmen. However, the statue
is not located in the plaza, but rather on a surrounding sidewalk facing
an entrance to the plaza.

Allende is buried in the general cemetery of Santiago. His tomb is a
major tourist attraction.

In Nicaragua, the tourist port of Managua is named after him. The
Salvador Allende Port is located near downtown Managua.

The broken glasses of Allende were given to the Chilean National History
Museum in 1996 by a woman who had found them in La Moneda in 1973.

A residential street in Toronto has also been named after him.

Allende Avenue in Harlow, Essex, is named after him. There is also a
square in the 7th arrondissement of Paris named after him, near the
Chilean embassy.

There is a square named after him in Viladecans, near Barcelona, called
Plaza de Salvador Allende.

\section{Notes}\label{notes}

\section{References}\label{references}

\section{Sources}\label{sources}

\begin{itemize}
\item
  \emph{Operation Guide for the Conspiration in Chile, Washington:
  United States National Security Council.}
\item
  \emph{Chile, el golpe y los gringos.}
\item
  \emph{CIA Activities in Chile CIA response to the Hinchey amendment on
  www.cia.gov}
\item
  \emph{Chile and the United States: Declassified Documents relating to
  the Military Coup, 1970--1976, (From the United States' National
  Security Archive).}
\end{itemize}

Chile and the United States: Declassified Documents relating to the
Military Coup, 1970--1976, (From the United States' National Security
Archive).

Thomas Karamessines (1970). Operation Guide for the Conspiration in
Chile, Washington: United States National Security Council.

La Batalla de Chile -- Cuba/Chile/França/Venezuela, 1975, 1977 e 1979.
Director Patricio Guzmán. Duration: 272 minutes. ‹See Tfd›(in Spanish)

Márquez, Gabriel García. Chile, el golpe y los gringos. Crónica de una
tragedia organizada, Manágua, Nicaragua: Radio La Primeirissima, 11 de
setembro de 2006 ‹See Tfd›(in Spanish)

Kornbluh, Peter. El Mercurio file, The., Columbia Journalism Review,
Sep/Oct 2003

CIA Activities in Chile CIA response to the Hinchey amendment on
www.cia.gov

\section{Further reading}\label{further-reading}

\begin{itemize}
\item
  \emph{Rencontre symbolique entre deux processus historiques {[}i.e.,
  de Cuba et de Chile{]}.}
\item
  \emph{Comite central del Partido comunista de Cuba: Comisión de
  orientación revolucionaria.}
\end{itemize}

Comite central del Partido comunista de Cuba: Comisión de orientación
revolucionaria. Rencontre symbolique entre deux processus historiques
{[}i.e., de Cuba et de Chile{]}. La Habana, Cuba: Éditions polituques,
1972.

\section{External links}\label{external-links}

\begin{itemize}
\item
  \emph{Salvador Allende's "Last Words".}
\item
  \emph{The transcript of the last radio broadcast of Chilean President
  Salvador Allende, made on 11 September 1973, at 9:10~am.}
\item
  \emph{Photos of the public places named in homage to the President
  Allende all around the world}
\item
  \emph{An Interview with Salvadore Allende: President of Chile,
  interviewed by Saul Landau, Dove Films, 1971, 32~min.}
\end{itemize}

Photos of the public places named in homage to the President Allende all
around the world

Salvador Allende's "Last Words". Spanish text with English translation.
The transcript of the last radio broadcast of Chilean President Salvador
Allende, made on 11 September 1973, at 9:10~am. MP3 audio available
here.

Caso Pinochet. While nominally a page about the Pinochet case, this
large collection of links includes Allende's dissertation and numerous
documents (mostly PDFs) related to the dissertation and to the
controversy about it, ranging from the Cesare Lombroso material
discussed in Allende's dissertation to a collective telegram of protest
over Kristallnacht signed by Allende ‹See Tfd›(in Spanish).

An Interview with Salvadore Allende: President of Chile, interviewed by
Saul Landau, Dove Films, 1971, 32~min. (previously unreleased):\\
Video (Spanish with English subtitles) in El Clarin de Chile.
(Alternative location at Google Video)

11 September 1973, When US-Backed Pinochet Forces Took Power in Chile --
video report by Democracy Now!

Why Allende had to die, 2013 reprint of a story from March 1974, by
Gabriel Garcia Marquez
