\textbf{From Wikipedia, the free encyclopedia}

https://en.wikipedia.org/wiki/Reformed\%20Church\%20in\%20America\\
Licensed under CC BY-SA 3.0:\\
https://en.wikipedia.org/wiki/Wikipedia:Text\_of\_Creative\_Commons\_Attribution-ShareAlike\_3.0\_Unported\_License

\section{Reformed Church in America}\label{reformed-church-in-america}

\begin{itemize}
\item
  \emph{The denomination is in full communion with the Evangelical
  Lutheran Church in America, Presbyterian Church (USA), and United
  Church of Christ and is a denominational partner of the Christian
  Reformed Church in North America.}
\item
  \emph{The Reformed Church in America (RCA) is a mainline Reformed
  Protestant denomination in Canada and the United States.}
\end{itemize}

The Reformed Church in America (RCA) is a mainline Reformed Protestant
denomination in Canada and the United States. It has about 223,675
members, with the total declining in recent decades. From its beginning
in 1628 until 1819, it was the North American branch of the Dutch
Reformed Church.

The RCA is a member of the National Council of Churches (founding
member), the World Council of Churches (WCC), Christian Churches
Together, and the World Communion of Reformed Churches (WCRC). Some
parts of the denomination belong to the National Association of
Evangelicals, the Canadian Council of Churches, and the Evangelical
Fellowship of Canada. The denomination is in full communion with the
Evangelical Lutheran Church in America, Presbyterian Church (USA), and
United Church of Christ and is a denominational partner of the Christian
Reformed Church in North America.

\section{Names}\label{names}

\begin{itemize}
\item
  \emph{Colloquially, it is sometimes referred loosely to as the Dutch
  Reformed Church in America, or simply as the Dutch Reformed Church
  when an American context has already been provided.}
\item
  \emph{In 1819, it incorporated as the Reformed Protestant Dutch
  Church.}
\end{itemize}

Colloquially, it is sometimes referred loosely to as the Dutch Reformed
Church in America, or simply as the Dutch Reformed Church when an
American context has already been provided. In 1819, it incorporated as
the Reformed Protestant Dutch Church. The current name was chosen in
1867.

\section{History}\label{history}

\begin{itemize}
\item
  \emph{During Dutch rule, the RCA was the established church of the
  colony and was under the authority of the classis of Amsterdam.}
\item
  \emph{In 1628 Jonas Michaelius organized the first Dutch Reformed
  congregation in New Amsterdam, now New York City, called the Reformed
  Protestant Dutch Church, now the Marble Collegiate Church.}
\item
  \emph{Following the American Civil War, in 1867 it formally adopted
  the name "Reformed Church in America".}
\end{itemize}

The early settlers in the Dutch colony of New Netherland first held
informal meetings for worship. In 1628 Jonas Michaelius organized the
first Dutch Reformed congregation in New Amsterdam, now New York City,
called the Reformed Protestant Dutch Church, now the Marble Collegiate
Church. During Dutch rule, the RCA was the established church of the
colony and was under the authority of the classis of Amsterdam.

Even after the English captured the colony in 1664, all Dutch Reformed
ministers were still trained in the Netherlands. Services in the RCA
remained in Dutch until 1764. (Dutch language use faded thereafter until
the new wave of Dutch immigration in the mid-19th century. This revived
use of the language among Dutch descendants and in some churches.)

In 1747 the church in the Netherlands had given permission to form an
assembly in North America; in 1754, the assembly declared itself
independent of the classis of Amsterdam. This American classis secured a
charter in 1766 for Queens College (now Rutgers University) in New
Jersey. In 1784 John Henry Livingston was appointed as professor of
theology, marking the beginning of the New Brunswick Theological
Seminary.

The Dutch-speaking community, including farmers and traders, prospered
in the former New Netherland, dominating New York City, the Hudson
Valley, and parts of New Jersey while maintaining a significant presence
in southeastern Pennsylvania, southwestern Connecticut, and Long Island.

In the early 18th century nearly 3,000 Palatine German refugees came to
New York. Most worked first in English camps along the Hudson River to
pay off their passage (paid by Queen Anne's government) before they were
allowed land in the Schoharie and Mohawk Valleys. There they created
numerous German-speaking Lutheran and Reformed churches, such as those
at Fort Herkimer and German Flatts. Thousands more immigrated to
Pennsylvania in the 18th century. They used German as the language in
their churches and schools for nearly 100 years, and recruited some of
their ministers from Germany. By the early 20th century, most of their
churches had joined the RCA.

During the American Revolution, a bitter internal struggle broke out in
the Dutch Reformed church, with lines of division following
ecclesiastical battles that had gone on for twenty years between the
"coetus" and "conferentie" factions. One source indicates that
defections may have occurred as early as 1737.

A spirit of amnesty made possible the church's survival after the war.
The divisiveness was also healed when the church sent members on an
extensive foreign missions program in the early 19th century.

In 1792 the classis adopted a formal constitution; and in 1794 the
denomination held its first general synod. Following the American Civil
War, in 1867 it formally adopted the name "Reformed Church in America".
In the nineteenth century in New York and New Jersey, ethnic Dutch
descendants struggled to preserve their European standards and
traditions while developing a taste for revivalism and an American
identity.

\includegraphics[width=3.61533in,height=5.50000in]{media/image1.jpg}\\
\emph{Phebe Johnson Ditmis (January 4, 1824 -- December 27, 1866) was
the wife of Reformed Church of Queens pastor George Onderdonk Ditmis
(July 22, 1818 -- February 1, 1896).}

\section{19th century}\label{th-century}

\begin{itemize}
\item
  \emph{Some members owned slaves, the most famous of the slaves being
  Sojourner Truth, and the church did not support abolitionism.}
\item
  \emph{Although some ministers favored revivals, generally the church
  did not support either the First or the Second Great Awakenings, which
  created much evangelical fervor.}
\end{itemize}

Some members owned slaves, the most famous of the slaves being Sojourner
Truth, and the church did not support abolitionism. In rural areas,
ministers preached in Dutch until about 1830--1850, then switched to
English, at the same time finally dropping the use of many traditional
Dutch clothing and customs. Although some ministers favored revivals,
generally the church did not support either the First or the Second
Great Awakenings, which created much evangelical fervor.

\section{Midwest}\label{midwest}

\begin{itemize}
\item
  \emph{They organized the Christian Reformed Church in North America
  (CRC), and other churches followed.}
\item
  \emph{In 1837 Pastor Abram D. Wilson and his wife Julia Evertson
  Wilson from New Jersey established the first Dutch Reformed church
  west of the Allegany Mountains in Fairview, Illinois.}
\end{itemize}

Immigration from the Netherlands in the mid-19th century led to the
expansion of the RCA into the Midwest. In 1837 Pastor Abram D. Wilson
and his wife Julia Evertson Wilson from New Jersey established the first
Dutch Reformed church west of the Allegany Mountains in Fairview,
Illinois. Hope College and Western Theological Seminary were founded in
Holland, Michigan, Central College in Pella, Iowa, and Northwestern
College in Orange City, Iowa. In the 1857 Secession, a group of more
conservative members in Michigan led by Gijsbert Haan separated from the
RCA. They organized the Christian Reformed Church in North America
(CRC), and other churches followed. In 1882 another group of
congregations left for the CRC, mirroring developments in the church in
the Netherlands.

\section{Post-World War II}\label{post-world-war-ii}

\begin{itemize}
\item
  \emph{After 1945 the RCA expanded into Canada, which was the
  destination of a large group of Dutch emigrants.}
\item
  \emph{It was a charter member of the Presbyterian Alliance, the
  Federal Council of Churches, and the World Council of Churches.}
\item
  \emph{Between 1949 and 1958, the RCA opened 120 churches among
  non-Dutch suburban communities, appealing to mainline Protestants.}
\end{itemize}

After 1945 the RCA expanded into Canada, which was the destination of a
large group of Dutch emigrants. Between 1949 and 1958, the RCA opened
120 churches among non-Dutch suburban communities, appealing to mainline
Protestants. It was a charter member of the Presbyterian Alliance, the
Federal Council of Churches, and the World Council of Churches.

\section{Recent decline}\label{recent-decline}

\begin{itemize}
\item
  \emph{In the last thirty years, the church has lost more than one
  third of its membership.}
\item
  \emph{Due to theological changes reflecting new interpretations of
  Scripture, the adoption of the Belhar Confession, the removal of the
  conscience clauses related to women's ordination, and growing
  acceptance of homosexual behavior, a number of congregations have left
  the RCA to join the Presbyterian Church in America (PCA) which is more
  conservative on these issues.}
\end{itemize}

Like most other mainline denominations, the RCA has had a declining
membership during the last thirty years. In 2016 the total membership
was less than 220,000, down from about 300,000 in 2000 and 360,000 in
1980. In the last thirty years, the church has lost more than one third
of its membership.

Due to theological changes reflecting new interpretations of Scripture,
the adoption of the Belhar Confession, the removal of the conscience
clauses related to women's ordination, and growing acceptance of
homosexual behavior, a number of congregations have left the RCA to join
the Presbyterian Church in America (PCA) which is more conservative on
these issues.

\section{Beliefs}\label{beliefs}

\begin{itemize}
\item
  \emph{These include the historic Apostles' Creed, Nicene Creed, and
  Athanasian Creed; the traditional Reformed Belgic Confession, the
  Heidelberg Catechism (with its compendium), the Canons of Dort, and
  the Belhar Confession.}
\item
  \emph{The RCA confesses several statements of doctrine and faith.}
\end{itemize}

The RCA confesses several statements of doctrine and faith. These
include the historic Apostles' Creed, Nicene Creed, and Athanasian
Creed; the traditional Reformed Belgic Confession, the Heidelberg
Catechism (with its compendium), the Canons of Dort, and the Belhar
Confession.

\section{Life issues}\label{life-issues}

\begin{itemize}
\item
  \emph{The Reformed Church also condemns the death penalty.}
\item
  \emph{The General Synod in 2000 expressed seven reasons why the Church
  opposes it:}
\item
  \emph{The church personnel should promote "Christian alternatives to
  abortion", and church members are asked to "support efforts for
  constitutional changes" to protect the "unborn".}
\end{itemize}

The RCA opposes euthanasia. The report of the Commission on Christian
Action stated in 1994: "What Christians say about issues of morality
ought to be and usually is a reflection of their fundamental faith
convictions. There are at least three of these convictions that appear
especially relevant to the question of whether it is acceptable for
Christians to seek a physician's assistance in committing suicide in the
midst of extreme suffering./ A fundamental conviction Christians have is
that they do not belong to themselves. Life, despite its circumstances,
is a gift from God, and each individual is its steward... Contemporary
arguments for the 'right' to assistance to commit suicide are based on
ideas of each individual's autonomy over his or her life. Christians
cannot claim such autonomy; Christians acknowledge that they belong to
God... Christians yield their personal autonomy and accept a special
obligation, as the first answer of the Heidelberg Catechism invites
people to confess: 'I am not my own, but belong---body and soul, in life
and in death---to my faithful Savior, Jesus Christ' (Heidelberg
Catechism, Q\&A 1)... A decision to take one's own life thus appears to
be a denial that one belongs to God./ A second conviction is that God
does not abandon people in times of suffering... Christians express
their faith in God's love by trusting in God's care for them. A decision
to end one's life would appear to be a cessation of that trust...
Suffering calls upon people to trust God even in the valley of the
shadow of death. It calls on people to let God, and not suffering,
determine the agenda of their life and their death./ A third conviction
is that in the community of God's people, caring for those who are dying
is a burden Christians are willing to share. Both living and dying
should occur within a caring community, and in the context of death,
Christian discipleship takes the form of caring for those who are
dying./ This is an era when many people find legislating morality a
questionable practice. Should Christians promote legislation which
embodies their conclusions about the morality of physician-assisted
suicide?... If Christians are to be involved in debating laws regulating
assisted suicide, it will be out of a concern for the health and
well-being of society... As a society, there is no common understanding
that gives any universal meaning to 'detrimental'. In humility,
Christians can simply acknowledge this, and proceed\ldots{}to share our
own unique perspectives, inviting others to consider them and the faith
that gives them meaning."

The Reformed Church also condemns the death penalty. The General Synod
in 2000 expressed seven reasons why the Church opposes it:

Capital punishment is incompatible with the Spirit of Christ and the
ethic of love. The law of love does not negate justice, but it does
nullify the motives of vengeance and retribution by forcing us to think
in terms of redemption, rehabilitation, and reclamation. The Christ who
refused to endorse the stoning of the woman taken in adultery would have
us speak to the world of compassion, not vengeance.

Capital punishment is of doubtful value as a deterrent. The capital
punishment as a deterrent argument assumes a criminal will engage in a
kind of rational, cost-benefit analysis before he or she commits murder.
Most murders, however, are crimes of passion or are committed under the
influence of drugs or alcohol. This does not excuse the perpetrator of
responsibility for the crime, but it does show that in most cases
capital punishment as a deterrent won't work.

Capital punishment results in inequities of application. Numerous
studies since 1965 have shown that racial factors play a significant
role in determining whether or not a person receives a sentence of
death.

Capital punishment is a method open to irremediable mistakes. The
increasing number of innocent defendants being found on death row is a
clear sign that the process for sentencing people to death is fraught
with fundamental errors---errors which cannot be remedied once an
execution occurs.

Capital punishment ignores corporate and community guilt. Such factors
may diminish but certainly do not destroy the responsibility of the
individual. Yet society also bears some responsibility for directing
efforts and resources toward correcting those conditions that may foster
such behavior.

Capital punishment perpetuates the concepts of vengeance and
retaliation. As an agency of society, the state should not become an
avenger for individuals; it should not presume the authority to satisfy
divine justice by vengeful methods.

Capital punishment ignores the entire concept of rehabilitation. The
Christian faith should be concerned not with retribution, but with
redemption. Any method which closes the door to all forgiveness, and to
any hope of redemption, cannot stand the test of our faith.

The General Synod resolution expressed its will "to urge members of the
Reformed Church in America to contact their elected officials, urging
them to advocate for the abolition of capital punishment and to call for
an immediate moratorium on executions."

The RCA is generally opposed to abortion. The position of the General
Synod, stated in 1973 and later affirmed, has been that "in principle"
abortion "should not be practiced at all", but in a "complex society" of
competing evils there "could be exceptions". However, abortion should
never be chosen as a matter of "individual convenience". The church
personnel should promote "Christian alternatives to abortion", and
church members are asked to "support efforts for constitutional changes"
to protect the "unborn".

\section{Homosexuality}\label{homosexuality}

\begin{itemize}
\item
  \emph{In 2014, the General Synod recommended that the Commission on
  Church Order begin the process of defining marriage as heterosexual.}
\item
  \emph{On May 5, 2017, the United Church of Christ and Reformed Church
  in America congregations that support LGBT inclusion announced the
  formation of an association for dually-affiliated congregations.}
\end{itemize}

Since 1978 the General Synod has made a number of statements on
homosexuality. Homosexual acts are considered sinful and "contrary to
the will of God". But homosexuals are not to be blamed for their
condition. The church must affirm civil rights for homosexuals while it
condemns homosexual behavior (1978). The church must seek to lift the
homosexual's "burden of guilt", recognizing that homosexuality is a
"complex phenomenon" (1979). The church should encourage "love and
sensitivity towards such persons as fellow human beings" (1990). In 1994
the Synod condemned the humiliation and degradation of homosexuals and
confessed that many members had not listened to the "heartfelt cries" of
homosexual persons struggling for "self-acceptance and dignity." While
calling for compassion, patience, and loving support toward those who
struggle with same-sex desires, the 2012 General Synod determined that
it is a "disciplinable offense" to advocate for homosexual behavior or
provide leadership for a service of a same-sex marriage. The following
year, however, the General Synod essentially rescinded this statement
and rebuked the 2012 delegates for demonstrating "a lack of decorum and
civility," and usurping constitutional authority. In 2014, the General
Synod recommended that the Commission on Church Order begin the process
of defining marriage as heterosexual. However, in 2015, the General
Synod approved a process for studying a way "to address the questions of
human sexuality". Also in 2015, Hope College in Michigan, affiliated
with the RCA, officially decided to provide benefits to employees'
same-sex spouses though the school continues to maintain a statement on
sexuality that espouses a traditional definition of marriage.

Additionally, a number of congregations and the Classis of New Brunswick
have voted to publicly affirm LGBT members. Some of those congregations,
including congregations dually affiliated with the RCA and United Church
of Christ, have already begun performing same-sex marriages. "Some RCA
churches have gay pastors, but their ordination is from other
denominations". On May 5, 2017, the United Church of Christ and Reformed
Church in America congregations that support LGBT inclusion announced
the formation of an association for dually-affiliated congregations.

In April 2016, a working committee of the RCA developed a report on
human sexuality. The report offers different options, for review by the
General Synod, and includes the option to define marriage as between a
man and woman or to define marriage as between two persons thus allowing
same-gender marriages. Of these options, General Synod 2016 voted to
define marriage as "man/woman". However, that vote needed to receive the
support of 2/3 of the classes and be ratified again in 2017.

In March 2017, the proposal to define marriage as "man/woman" did not
receive the necessary votes from 2/3 of the classes, and, as a result,
it did not pass. On June 12, 2017, the General Synod voted for a
"recommendation {[}which{]} says, 'faithful adherence to the RCA's
Standards, therefore, entails the affirmation that marriage is between
one man and one woman.'" Also, in 2017, the RCA ordained the first
openly gay and married pastor who was 'out' when he began the ordination
process.

\section{Women's ordination}\label{womens-ordination}

\begin{itemize}
\item
  \emph{By 1980 the General Synod of the RCA amended the Book of Church
  Order (BCO) to clarify their position on women's ordination, including
  amending the language of Part I, Article 1, Section 3 of the BCO from
  "persons" to "men and women".}
\item
  \emph{In 2012 by a vote of 143 to 69, the General Synod of the RCA
  voted to remove the conscience clauses.}
\end{itemize}

The RCA first admitted women to the offices of deacon and elder in 1972
and first ordained women in 1979. By 1980 the General Synod of the RCA
amended the Book of Church Order (BCO) to clarify their position on
women's ordination, including amending the language of Part I, Article
1, Section 3 of the BCO from "persons" to "men and women".

In 1980 the RCA added a conscience clause to the BCO stating, "If
individual members of the classis find that their consciences, as
illuminated by Scripture, would not permit them to participate in the
licensure, ordination or installation of women as ministers of the Word,
they shall not be required to participate in decisions or actions
contrary to their consciences, but may not obstruct the classis in
fulfilling its responsibility to arrange for the care, ordination, and
installation of women candidates and ministers by means mutually agreed
on by such women and the classis" (Part II, Article 2, Section 7).

In 2012 by a vote of 143 to 69, the General Synod of the RCA voted to
remove the conscience clauses. However, the vote by the General Synod
had to be approved by a majority of the classes (a classis serving the
same function as a presbytery). Eventually, 31 classes voted in favor of
removal, with 14 voting to retain them, and the vote was ratified at the
RCA's 2013 General Synod.

\section{Polity}\label{polity}

\begin{itemize}
\item
  \emph{Measures passed at General Synod are executed and overseen by
  the General Synod Council.}
\item
  \emph{Council members are appointed by the General Synod.}
\item
  \emph{The Government, along with the Formularies and the By-laws of
  the General Synod, are published annually in a volume known as The
  Book of Church Order.}
\item
  \emph{Min., was installed as the current General Secretary at the 2018
  General Synod.}
\end{itemize}

The RCA has a presbyterian polity where authority is divided among
representative bodies: consistories, classes, regional synods, and the
General Synod. The General Synod meets annually and is the
representative body of the entire denomination, establishing its
policies, programs, and agenda. Measures passed at General Synod are
executed and overseen by the General Synod Council. Council members are
appointed by the General Synod. A General Secretary oversees day-to-day
operations. The Rev. Eddy Alemán, D. Min., was installed as the current
General Secretary at the 2018 General Synod.

The Constitution of the RCA consists of three parts: the Liturgy, the
Government, and the Doctrinal Standards. The Government, along with the
Formularies and the By-laws of the General Synod, are published annually
in a volume known as The Book of Church Order.

\section{Colleges and seminaries}\label{colleges-and-seminaries}

\begin{itemize}
\item
  \emph{Students who do not attend or receive their Master of Divinity
  degree from one of the two seminaries operated by the RCA are
  certified and credentialed for ministry in the RCA through the
  Ministerial Formation Certification Agency in Paramount, California.}
\item
  \emph{New Brunswick Theological Seminary, New Brunswick, New Jersey}
\end{itemize}

Colleges

Central College, Pella, Iowa

Hope College, Holland, Michigan

Northwestern College, Orange City, Iowa

Seminaries

New Brunswick Theological Seminary, New Brunswick, New Jersey

Western Theological Seminary, Holland, Michigan

Certification agencies

Students who do not attend or receive their Master of Divinity degree
from one of the two seminaries operated by the RCA are certified and
credentialed for ministry in the RCA through the Ministerial Formation
Certification Agency in Paramount, California.

\section{Ecumenical relations}\label{ecumenical-relations}

\begin{itemize}
\item
  \emph{Along with their Formula of Agreement partners, the RCA retains
  close fellowship with the Christian Reformed Church in North America
  (CRC).}
\item
  \emph{Through a document known as A Formula of Agreement, the RCA has
  full communion with the Presbyterian Church (U.S.A.), the United
  Church of Christ, and the Evangelical Lutheran Church in America.}
\end{itemize}

Through a document known as A Formula of Agreement, the RCA has full
communion with the Presbyterian Church (U.S.A.), the United Church of
Christ, and the Evangelical Lutheran Church in America. The relationship
between the United Church of Christ and the RCA has been the subject of
controversy within the RCA, particularly a resolution by the UCC General
Synod in 2005 regarding homosexuality. The ELCA's affirmation of the
ordination of homosexuals as clergy in 2009 prompted some RCA
conservatives to call for a withdrawal from the Formula of Agreement. In
2012 RCA discussed its own position regarding homosexuality. The two
denominations undertook a dialogue and in 1999 produced a document
discussing their differences (PDF).

Along with their Formula of Agreement partners, the RCA retains close
fellowship with the Christian Reformed Church in North America (CRC). In
2005 the RCA and CRC voted to allow for the exchange of ministers. Faith
Alive Christian Resources, the CRC's publishing arm, is also used by the
RCA and in 2013 published a joint hymnal for use in both denominations.
The two denominations have also collaborated on various other ministry
ventures, voted to merge pension plans in 2013 in conformity with the
Affordable Care Act, and plan to hold back-to-back General Synods at
Central College in Pella, Iowa, in 2014.

\includegraphics[width=3.91600in,height=5.50000in]{media/image2.jpg}\\
\emph{John Scudder, Sr., a Dutch Reformed minister, started a family of
missionaries in India in 1819}

\section{Notable members}\label{notable-members}

\begin{itemize}
\item
  \emph{Francis D. "Hap" Moran, professional football player New York
  Giants, deacon and elder in the Reformed Church in America}
\item
  \emph{The Schuller Family - Robert Schuller, Robert A. Schuller, Bobby
  Schuller, All Reformed Church in America pastors}
\item
  \emph{Edward Becenti, Navajo interpreter and son of Chief Judge
  Becenti (Navajo), translated Bible verses and songs into the Navajo
  language for the Christian Reformed Church in New Mexico}
\end{itemize}

Edward Wilmot Blyden, educator, writer, diplomat and politician

Vern Den Herder, professional football player in the NFL (1972
undefeated Miami Dolphins)

Everett Dirksen, senator

B.D. Dykstra, writer and educator

Geronimo

Jack Hanna, American zoologist

Peter Hoekstra, congressman

Evel Knievel, motorcycle stuntman and daredevil

Kyle Korver, professional basketball player in the NBA

Francis D. "Hap" Moran, professional football player New York Giants,
deacon and elder in the Reformed Church in America

A. J. Muste, writer, professor, pacifist

Jim Nantz, TV sportscaster

Louis P. Pojman, philosopher

Norman Vincent Peale, preacher

Theodore Roosevelt, American President

Marge Roukema, Congresswoman, a convert from Roman Catholicism

Albert Janse Ryckman, Mayor of Albany, New York (1702--1703), captain of
the Albany Militia, prominent Albany brewmaster of the late 17th
century; deacon in the Dutch Reformed Church

The Schuller Family - Robert Schuller, Robert A. Schuller, Bobby
Schuller, All Reformed Church in America pastors

John Scudder, Sr., missionary for the Arcot Mission

Philip Schuyler, a leader of the American Revolution

Martin Van Buren, American President

Fez Whatley, radio personality

Andrew Yang, entrepreneur and 2020 presidential candidate

The Reverend Clark V. Poling, one of the Four Chaplains

Edward Becenti, Navajo interpreter and son of Chief Judge Becenti
(Navajo), translated Bible verses and songs into the Navajo language for
the Christian Reformed Church in New Mexico

\section{See also}\label{see-also}

\begin{itemize}
\item
  \emph{Christian Reformed Church in North America}
\item
  \emph{List of Reformed denominations}
\item
  \emph{American Reformed Mission}
\end{itemize}

American Reformed Mission

Christian Reformed Church in North America

List of Reformed denominations

\section{References}\label{references}

\section{Citations}\label{citations}

\begin{itemize}
\item
  \emph{24
  https://www.rca.org/ministerial-formation-certification-agency-0}
\end{itemize}

24 https://www.rca.org/ministerial-formation-certification-agency-0

\section{Sources}\label{sources}

\begin{itemize}
\item
  \emph{Family Quarrels in the Dutch Reformed Churches in the 19th
  Century: The Pillar Church Sesquicentennial Lectures (Historical
  Series of the Reformed Church in America) (2000) excerpt and text
  search}
\item
  \emph{Historical Series of the Reformed Church in America.}
\item
  \emph{''Women in the History of the Reformed Church in America (1999)
  182 pp.}
\item
  \emph{The Reformed Church in the Netherlands, 1340--1840 (1884)}
\end{itemize}

Birch, J. J. The Pioneering Church in the Mohawk Valley (1955)

DeJong, Gerald F. The Dutch Reformed Church in the American Colonies
(1978) 279 pp.

Fabend, H. H. Zion on the Hudson: Dutch New York and New Jersey in the
Age of Revivals (2000)

House, Renee S., and John W. Coakley, eds. ''Women in the History of the
Reformed Church in America (1999) 182 pp. Historical Series of the
Reformed Church in America. no. 5.

Hansen, M.G. The Reformed Church in the Netherlands, 1340--1840 (1884)

Swierenga, Robert, and Elton J. Bruins. Family Quarrels in the Dutch
Reformed Churches in the 19th Century: The Pillar Church
Sesquicentennial Lectures (Historical Series of the Reformed Church in
America) (2000) excerpt and text search

Swierenga, Robert. The Dutch in America: Immigration, Settlement, and
Cultural Change (1985)

Swierenga, Robert. Faith and Family: Dutch Immigration and Settlement in
the United States, 1820--1920 (2000)

\section{External links}\label{external-links}

\begin{itemize}
\item
  \emph{~"Reformed Church in the United States"~.}
\item
  \emph{"Reformed Church in America, The"~.}
\end{itemize}

Official website

~Edward P. Johnson (1920). "Reformed Church in America, The"~.
Encyclopedia Americana.

~"Reformed Church in the United States"~. Encyclopædia Britannica. 23
(11th ed.). 1911. pp.~24--25.
