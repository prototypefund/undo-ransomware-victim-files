\textbf{From Wikipedia, the free encyclopedia}

https://en.wikipedia.org/wiki/Civil\%20rights\%20movement\%20in\%20popular\%20culture\\
Licensed under CC BY-SA 3.0:\\
https://en.wikipedia.org/wiki/Wikipedia:Text\_of\_Creative\_Commons\_Attribution-ShareAlike\_3.0\_Unported\_License

\section{Civil rights movement in popular
culture}\label{civil-rights-movement-in-popular-culture}

\begin{itemize}
\item
  \emph{These depictions in the arts keep alive the ideals and deeds of
  the people who organized, supported, and participated in this
  nonviolent movement.}
\item
  \emph{The 1954 to 1968 civil rights movement contributed strong
  cultural threads to American and international theater, song, film,
  television, and folk art.}
\end{itemize}

The 1954 to 1968 civil rights movement contributed strong cultural
threads to American and international theater, song, film, television,
and folk art. These depictions in the arts keep alive the ideals and
deeds of the people who organized, supported, and participated in this
nonviolent movement.

\section{Film}\label{film}

\section{Dramatizations}\label{dramatizations}

\begin{itemize}
\item
  \emph{The Butler (2013), depicts a civil rights movement training
  session conducted during the Nashville Student Movement by James
  Lawson and other civil rights movement events.}
\item
  \emph{All the Way (2016), focusing on Lyndon B. Johnson's successful
  attempt to pass the Civil Rights Act of 1964.}
\end{itemize}

Mississippi Burning (1988), about the 1964 murders of Chaney, Goodman,
and Schwerner in Mississippi.

Hairspray (1988, 2007 remake), features a major subplot about
demonstrations against racial segregation in Baltimore, Maryland.

The Long Walk Home (1990), portrays a woman who is boycotting city buses
during the 1955-1956 Montgomery Bus Boycott.

Malcolm X (1992), a biopic focused on the life and assassination of
Malcolm X.

Ghosts of Mississippi (1996), an account of the murder of Mississippi
activist Medgar Evers and the subsequent investigation.

The Chamber (1996)

Selma, Lord, Selma (1999), follows the life of 11-year-old Sheyann Webb
during the events leading up to the 1965 Selma to Montgomery march and
its "Bloody Sunday".

Our Friend, Martin (1999 animated)

Boycott (2001), depicts some of the events of the 1955-56 Montgomery Bus
Boycott.

The Rosa Parks Story (2002), the life of the key figure in the
Montgomery Bus Boycott.

The Butler (2013), depicts a civil rights movement training session
conducted during the Nashville Student Movement by James Lawson and
other civil rights movement events.

Selma (2014), focusing on the events leading up to, during, and after
the 1965 Selma to Montgomery marches, including the 1965 Voting Rights
Act.

All the Way (2016), focusing on Lyndon B. Johnson's successful attempt
to pass the Civil Rights Act of 1964.

\section{Documentaries}\label{documentaries}

\begin{itemize}
\item
  \emph{A Time for Justice (1994), a short history of the civil rights
  movement narrated by Julian Bond.}
\item
  \emph{Soundtrack for a Revolution (2009), focuses on some of the songs
  sung during the civil rights movement.}
\item
  \emph{Freedom Summer (2014), documents the events of the 1964
  Mississippi Freedom Summer movement.}
\item
  \emph{Crossing in St. Augustine (2010), produced by Andrew Young, who
  participated in the civil rights movement in St. Augustine in 1964.}
\end{itemize}

Crisis: Behind a Presidential Commitment (1963), first-hand
as-it-happened account of the University of Alabama "Stand in the
Schoolhouse Door" integration crisis of June 1963.

Nine from Little Rock (1964), about the Little Rock Nine who enrolled in
an all-white Arkansas high school in 1957.

The March (1964), about the 1963 March on Washington, was made for the
United States Information Agency.

Louisiana Diary (1964) follows the Congress of Racial Equality (CORE)
from July to August 1963, as they undertake an African American voter
registration drive in Plaquemine, Louisiana.

Cicero March (1966), details a civil rights march held by the Congress
of Racial Equality on September 4, 1966 in Cicero, Illinois, soon after
the 1966 Chicago open housing movement ended.

King: A Filmed Record... Montgomery to Memphis (1970)

Malcolm X (1972), based on The Autobiography of Malcolm X.

Freedom on My Mind (1994), documents efforts to register
African-American voters in Mississippi, Freedom Summer, and the
formation of the Mississippi Freedom Democratic Party.

A Time for Justice (1994), a short history of the civil rights movement
narrated by Julian Bond.

4 Little Girls (1997), focusing on the 1963 events surrounding the
bombing of the 16th Street Baptist Church just after the Birmingham
campaign.

Mighty Times: The Legacy of Rosa Parks (2002), created with archival
footage

February One: The Story of the Greensboro Four (2003), documents the
1960 Greensboro lunch counter sit-ins and the four college students
involved.

The Murder of Emmett Till (2003) about the murder and the impact of
Emmett Till's open-casket funeral.

Brother Outsider (2003), about the life of civil rights organizer Bayard
Rustin.

Home of the Brave (2004), documents the life and murder of Viola Liuzzo.

Mighty Times: The Children's March (2004) about the 1963 Birmingham
campaign and its marches by schoolchildren.

Dare Not Walk Alone (2006) focuses on the 1964 St. Augustine movement.

Mississippi Cold Case (2007), chronicles the Ku Klux Klan murders of two
young black men in Mississippi in 1964 during Freedom Summer, and the
21st-century quest for justice by the brother of one of those murdered.

The Witness: From the Balcony of Room 306 (2008), details the events
surrounding the assassination of Martin Luther King Jr. at the Lorraine
Motel in Memphis, Tennessee.

Neshoba (2008), chronicles the events and thinking in Neshoba County,
Mississippi, 40 years after the 1964 murders of Chaney, Goodman, and
Schwerner.

Soundtrack for a Revolution (2009), focuses on some of the songs sung
during the civil rights movement.

Crossing in St. Augustine (2010), produced by Andrew Young, who
participated in the civil rights movement in St. Augustine in 1964.

The Barber of Birmingham (2011), about James Armstrong, a voting rights
activist and an original flag bearer for the 1965 Selma to Montgomery
marches.

Alpha Man: The Brotherhood of MLK (2011), about King's fraternity.

Julian Bond: Reflections from the Frontlines of the Civil Rights
Movement (2012), on the life and thoughts of activist Julian Bond.

The March (2013), documents the 1963 March on Washington and the "I Have
a Dream" speech by King.

Freedom Summer (2014), documents the events of the 1964 Mississippi
Freedom Summer movement.

\section{Television}\label{television}

\begin{itemize}
\item
  \emph{Eyes on the Prize (1987-1990), a 14-hour documentary series
  chronicling the civil rights movement.}
\item
  \emph{Freedom Song (2000), a film based on true stories of the civil
  rights movement in Mississippi, involving voting rights, Freedom
  Summer, and the Student Nonviolent Coordinating Committee (SNCC).}
\item
  \emph{King (1978 miniseries) about Southern Christian Leadership
  Conference chairman and movement spokesman, Martin Luther King Jr.}
\end{itemize}

Attack on Terror: The FBI vs. the Ku Klux Klan (1975) two-part
television movie dramatizing the events following the 1964 disappearance
and murder of three civil rights workers in Mississippi.

King (1978 miniseries) about Southern Christian Leadership Conference
chairman and movement spokesman, Martin Luther King Jr.

Crisis at Central High (1981), made-for-television movie about the
Little Rock Integration Crisis of 1957.

For Us the Living: The Medgar Evers Story (1983), PBS biopic about
assassinated Mississippi civil rights activist Medgar Evers, his work,
and his family.

Eyes on the Prize (1987-1990), a 14-hour documentary series chronicling
the civil rights movement.

My Past Is My Own (1989), a portrayal of students organizing an early
1960s civil rights movement sit-in.

Murder in Mississippi (1990) movie following the last weeks of three
civil rights workers, Michael "Mickey" Schwerner, Andrew Goodman and
James Chaney, and the events leading up to their disappearance and
subsequent murder during Freedom Summer.

Separate But Equal (1991), depicts the landmark Supreme Court
desegregation case Brown v. Board of Education, based on the phrase
"Separate but equal".

The Ernest Green Story (1993), film chronicling the true story of Ernest
Green (Morris Chestnut) and eight other high-school students (dubbed the
"Little Rock Nine") and the 1957 integration of Little Rock Central High
School in Little Rock, Arkansas.

George Wallace (1997), a film about George Wallace, the Alabama
governor, and his involvement in many of the events of the era including
the "Stand in the Schoolhouse Door".

Ruby Bridges (1998), the true story of six-year-old Ruby Bridges who, in
1960, became the first black student to integrate an elementary school
in the South.

Any Day Now (1998-2002), series with a major subplot involving the
Birmingham campaign.

Freedom Song (2000), a film based on true stories of the civil rights
movement in Mississippi, involving voting rights, Freedom Summer, and
the Student Nonviolent Coordinating Committee (SNCC).

Sins of the Father (2002) chronicles the 1963 16th Street Baptist Church
bombing in Birmingham, Alabama in which four young African American
girls were killed while attending Sunday school.

Freedom Riders (2011), a PBS film marking the 50th anniversary of the
first Freedom Ride in May, 1961.

Hairspray Live! (2016), a presentation of the John Waters musical about
a fictional Baltimore desegregation of a television dance program.

Rosa (Doctor Who) (2018), an episode of the popular science-fiction
series depicts Rosa Parks and her 1955 sit-in which began the Montgomery
Bus Boycott.

\section{Music}\label{music}

\begin{itemize}
\item
  \emph{"We Shall Overcome", gospel-based song that became an anthem for
  the civil rights movement.}
\item
  \emph{"MLK" (1984) by U2, a lullaby to honor Martin Luther King, Jr.}
\item
  \emph{"If I Can Dream" (1968), recorded by Elvis Presley in honor of
  King soon after King's death.}
\item
  \emph{"We Shall Not Be Moved", spiritual-based song often sung during
  the civil rights movement.}
\end{itemize}

"We Shall Overcome", gospel-based song that became an anthem for the
civil rights movement.

"We Shall Not Be Moved", spiritual-based song often sung during the
civil rights movement.

"Keep Your Eyes on the Prize", sung during the Movement actions, based
on the traditional folk song "Gospel Plow".

"This Little Light of Mine", originally a hymn, the lyrics were modified
as it became a movement anthem.

"Fables of Faubus" (1957), Charles Mingus's jazz composition written and
performed in response to the Little Rock Nine incident

"The Death of Emmett Till" (1962), one of several songs Bob Dylan paid
tribute to civil rights; this one a reference to the Murder of Emmett
Till

"Oxford Town" (1962), written and sung by Bob Dylan, pertains to James
Meredith's enrollment at the University of Mississippi.

"Alabama" (1963), John Coltrane's jazz composition response to a 1963
church bombing that killed four young girls.

"Birmingham Sunday" (1964), Richard Fariña's response to the Birmingham
church bombing recorded by Joan Baez, Fariña's sister-in-law, on her
1964 album Joan Baez/5.

"Mississippi Goddamn" (1964), Nina Simone's response to the murder of
Medgar Evers.

"Only a Pawn in Their Game" (1964), Bob Dylan's response to the murder
of Medgar Evers.

"Keep on Pushing" (1964), rhythm and blues hit single by The
Impressions.

"Here's to the State of Mississippi", (1965) a protest song by Phil Ochs
that criticizes the state of Mississippi for its mistreatment of African
Americans.

"Eve of Destruction" (1965) references the Selma to Montgomery marches.

"Abraham, Martin and John" (1968), a tribute to Abraham Lincoln, Martin
Luther King Jr, John F. Kennedy, and Robert Kennedy written by Dick
Holler and first recorded by Dion.

"If I Can Dream" (1968), recorded by Elvis Presley in honor of King soon
after King's death.

Scenes from the Life of a Martyr (1981), a 16-part oratorio composed by
Undine Smith Moore in memory of King.

"MLK" (1984) by U2, a lullaby to honor Martin Luther King, Jr.

"Pride (In the Name of Love)" (1984) a song about King by U2

Joseph Schwantner: New Morning for the World; Nicolas Flagello: The
Passion of Martin Luther King (1995), an album of classical music by the
Oregon Symphony in honor of King.

"Up to the Mountain (MLK Song)" (2006), Patty Griffin's song about the
emotions surrounding King's 1968 I've Been to the Mountaintop speech.

"A Dream" (2006), by Common for the film Freedom Writers, uses King's "I
Have a Dream" speech

"Glory" (2014), from the film Selma, won both the Golden Globe and
Academy Award for Best Original Song.

\section{Theater}\label{theater}

\begin{itemize}
\item
  \emph{James Baldwin: A Soul on Fire (1999), set in Baldwin's apartment
  on the morning of May~24, 1963, immediately before Baldwin and other
  Black leaders are scheduled to meet with Attorney General Robert F.
  Kennedy concerning events in the civil rights movement.}
\item
  \emph{All the Way (2012), a play about President Lyndon Johnson and
  his work to pass the 1964 Civil Rights Act.}
\item
  \emph{I Dream (2010), a musical about the life of Martin Luther King
  Jr.}
\end{itemize}

The Meeting (1987), a play about an imaginary 1965 meeting between
Martin Luther King Jr. and Malcolm X in a hotel in Harlem.

James Baldwin: A Soul on Fire (1999), set in Baldwin's apartment on the
morning of May~24, 1963, immediately before Baldwin and other Black
leaders are scheduled to meet with Attorney General Robert F. Kennedy
concerning events in the civil rights movement.

Hairspray (2002), a musical based on the 1988 film described above.

The State of Mississippi and the Face of Emmett Till (2003) is a play
centered on the murder and subsequent open-casket funeral of Emmett
Till.

The Mountaintop (2009), a play set in Room 306 of the Lorraine Motel the
night before King's assassination.

I Dream (2010), a musical about the life of Martin Luther King Jr.

All the Way (2012), a play about President Lyndon Johnson and his work
to pass the 1964 Civil Rights Act.

\section{Graphic non-fiction}\label{graphic-non-fiction}

\begin{itemize}
\item
  \emph{Martin Luther King and the Montgomery Story (1957), graphic
  portrayal of the 1955-56 Montgomery Bus Boycott.}
\item
  \emph{March (2013), the life and events of the Selma to Montgomery
  march as remembered by activist John Lewis.}
\end{itemize}

Martin Luther King and the Montgomery Story (1957), graphic portrayal of
the 1955-56 Montgomery Bus Boycott.

March (2013), the life and events of the Selma to Montgomery march as
remembered by activist John Lewis.

Darkroom: A Memoir in Black and White (2012) by Lila Quintero Weaver,
graphic memoir recounting Weaver's childhood during the 1960s in Marion,
Alabama. Published in Spanish as Cuarto oscuro: Recuerdos en blanco y
negro.

\section{Art}\label{art}

\begin{itemize}
\item
  \emph{Civil Rights Memorial (1989), a memorial fountain in Montgomery,
  Alabama designed by Maya Lin dedicated to 41 people who died in the
  civil rights movement.}
\item
  \emph{Martin Luther King, Jr. Memorial (2011), showcases the statue of
  Martin Luther King Jr. by Lei Yixin and several surrounding art pieces
  and quotations on the National Mall in Washington, D.C.}
\end{itemize}

The Problem We All Live With (1964), a painting by Norman Rockwell
depicting Ruby Bridges, the six-year-old African-American girl who, in
1960, was the first to desegregate William Frantz Elementary School in
the South during the New Orleans school desegregation crisis.

Bust of Martin Luther King Jr. (1970), by Charles Alston, has been
featured in the Oval Office of the White House by the Obama and Trump
presidential administrations.

U.S. Capitol Rotunda sculpture (1986), a bust of Martin Luther King Jr.,
dedicated by John Wilson.

Civil Rights Memorial (1989), a memorial fountain in Montgomery, Alabama
designed by Maya Lin dedicated to 41 people who died in the civil rights
movement.

Rosa Parks (2009), a statue in Eugene, Oregon portrays activist Rose
Parks waiting for a bus.

Martin Luther King, Jr. Memorial (2011), showcases the statue of Martin
Luther King Jr. by Lei Yixin and several surrounding art pieces and
quotations on the National Mall in Washington, D.C.

Rosa Parks (2013), statue in National Statuary Hall, Capitol Building,
Washington, D.C.

\section{See also}\label{see-also}

\begin{itemize}
\item
  \emph{Birmingham Civil Rights National Monument}
\item
  \emph{List of photographers of the civil rights movement}
\item
  \emph{Memorials to Martin Luther King Jr.}
\end{itemize}

List of photographers of the civil rights movement

Freedom Songs

Birmingham Civil Rights National Monument

Freedom Riders National Monument

A Force More Powerful, 1999 documentary and 2000 television series

Memorials to Martin Luther King Jr.

\section{References}\label{references}

\section{External links}\label{external-links}

\begin{itemize}
\item
  \emph{"People Get Ready": Music and the Civil Rights Movement of the
  1950s and 1960s}
\end{itemize}

"People Get Ready": Music and the Civil Rights Movement of the 1950s and
1960s
