\textbf{From Wikipedia, the free encyclopedia}

https://en.wikipedia.org/wiki/Denmark\\
Licensed under CC BY-SA 3.0:\\
https://en.wikipedia.org/wiki/Wikipedia:Text\_of\_Creative\_Commons\_Attribution-ShareAlike\_3.0\_Unported\_License

\section{Denmark}\label{denmark}

\begin{itemize}
\item
  \emph{Denmark is considered to be one of the most economically and
  socially developed countries in the world.}
\item
  \emph{Denmark (Danish: Danmark, pronounced~{[}ˈtænmak{]} (listen)),
  officially the Kingdom of Denmark, is a Nordic country and the
  southernmost of the Scandinavian nations.}
\item
  \emph{Denmark remained neutral during World War I.}
\end{itemize}

Denmark (Danish: Danmark, pronounced~{[}ˈtænmak{]} (listen)), officially
the Kingdom of Denmark, is a Nordic country and the southernmost of the
Scandinavian nations. Denmark lies southwest of Sweden and south of
Norway, and is bordered to the south by Germany. The Kingdom of Denmark
also comprises two autonomous constituent countries in the North
Atlantic Ocean: the Faroe Islands and Greenland. Denmark proper consists
of a peninsula, Jutland, and an archipelago of 443 named islands, with
the largest being Zealand, Funen and the North Jutlandic Island. The
islands are characterised by flat, arable land and sandy coasts, low
elevation and a temperate climate. Denmark has a total area of
42,924~km2 (16,573~sq~mi), land area of 42,394~km2 (16,368~sq~mi), and
the total area including Greenland and the Faroe Islands is
2,210,579~km2 (853,509~sq~mi), and a population of 5.8 million (as of
2018{[}update{]}).

The unified kingdom of Denmark emerged in the 10th century as a
proficient seafaring nation in the struggle for control of the Baltic
Sea. Denmark, Sweden, and Norway were ruled together under one sovereign
ruler in the Kalmar Union, established in 1397 and ending with Swedish
secession in 1523. The areas of Denmark and Norway remained under the
same monarch until 1814, Denmark--Norway. Beginning in the 17th century,
there were several devastating wars with the Swedish Empire, ending with
large cessions of territory to Sweden. After the Napoleonic Wars, Norway
was ceded to Sweden, while Denmark kept the Faroe Islands, Greenland,
and Iceland. In the 19th century there was a surge of nationalist
movements, which were defeated in the 1864 Second Schleswig War. Denmark
remained neutral during World War I. In April 1940, a German invasion
saw brief military skirmishes while the Danish resistance movement was
active from 1943 until the German surrender in May 1945. An
industrialised exporter of agricultural produce in the second half of
the 19th century, Denmark introduced social and labour-market reforms in
the early 20th century that created the basis for the present welfare
state model with a highly developed mixed economy.

The Constitution of Denmark was signed on 5 June 1849, ending the
absolute monarchy, which had begun in 1660. It establishes a
constitutional monarchy organised as a parliamentary democracy. The
government and national parliament are seated in Copenhagen, the
nation's capital, largest city, and main commercial centre. Denmark
exercises hegemonic influence in the Danish Realm, devolving powers to
handle internal affairs. Home rule was established in the Faroe
Islands\\
in 1948; in Greenland home rule was established in 1979 and further
autonomy in 2009. Denmark became a member of the European Economic
Community (now the EU) in 1973, but negotiated certain opt-outs; it
retains its own currency, the krone. It is among the founding members of
NATO, the Nordic Council, the OECD, OSCE, and the United Nations; it is
also part of the Schengen Area.

Denmark is considered to be one of the most economically and socially
developed countries in the world. Danes enjoy a high standard of living
and the country ranks highly in some metrics of national performance,
including education, health care, protection of civil liberties,
democratic governance, prosperity, and human development. The country
ranks as having the world's highest social mobility, a high level of
income equality, has the lowest perceived level of corruption in the
world, the eleventh-most developed in the world, has one of the world's
highest per capita incomes, and one of the world's highest personal
income tax rates.

\section{Etymology}\label{etymology}

\begin{itemize}
\item
  \emph{The etymology of the word Denmark, and especially the
  relationship between Danes and Denmark and the unifying of Denmark as
  one kingdom, is a subject which attracts debate.}
\item
  \emph{The inhabitants of Denmark are there called tani ({[}danɪ{]}),
  or "Danes", in the accusative.}
\end{itemize}

The etymology of the word Denmark, and especially the relationship
between Danes and Denmark and the unifying of Denmark as one kingdom, is
a subject which attracts debate. This is centered primarily on the
prefix "Dan" and whether it refers to the Dani or a historical person
Dan and the exact meaning of the -"mark" ending.

Most handbooks derive the first part of the word, and the name of the
people, from a word meaning "flat land", related to German Tenne
"threshing floor", English den "cave". The -mark is believed to mean
woodland or borderland (see marches), with probable references to the
border forests in south Schleswig.

The first recorded use of the word Danmark within Denmark itself is
found on the two Jelling stones, which are runestones believed to have
been erected by Gorm the Old (c. 955) and Harald Bluetooth (c. 965). The
larger stone of the two is popularly cited as Denmark's "baptismal
certificate" (dåbsattest), though both use the word "Denmark", in the
form of accusative .mw-parser-output
.script-runic\{font-family:"BabelStone Runic Beagnoth","BabelStone Runic
Beorhtnoth","BabelStone Runic Beorhtric","BabelStone Runic
Beowulf","BabelStone Runic Berhtwald","BabelStone Runic
Byrhtferth",Junicode,Kelvinch,"Free Monospaced",Code2000,Hnias,"Noto
Sans Runic","Segoe UI Historic","Segoe UI Symbol"\}ᛏᛅᚾᛘᛅᚢᚱᚴ tanmaurk
({[}danmɒrk{]}) on the large stone, and genitive ᛏᛅᚾᛘᛅᚱᚴᛅᚱ "tanmarkar"
(pronounced {[}danmarkaɽ{]}) on the small stone. The inhabitants of
Denmark are there called tani ({[}danɪ{]}), or "Danes", in the
accusative.

\section{History}\label{history}

\section{Prehistory}\label{prehistory}

\begin{itemize}
\item
  \emph{The Roman provinces maintained trade routes and relations with
  native tribes in Denmark, and Roman coins have been found in Denmark.}
\item
  \emph{The earliest archaeological findings in Denmark date back to the
  Eem interglacial period from 130,000--110,000 BC.}
\item
  \emph{Denmark has been inhabited since around 12,500 BC and
  agriculture has been evident since 3900 BC.}
\end{itemize}

The earliest archaeological findings in Denmark date back to the Eem
interglacial period from 130,000--110,000 BC. Denmark has been inhabited
since around 12,500 BC and agriculture has been evident since 3900 BC.
The Nordic Bronze Age (1800--600 BC) in Denmark was marked by burial
mounds, which left an abundance of findings including lurs and the Sun
Chariot.

During the Pre-Roman Iron Age (500 BC~-- AD 1), native groups began
migrating south, and the first tribal Danes came to the country between
the Pre-Roman and the Germanic Iron Age, in the Roman Iron Age (AD
1--400). The Roman provinces maintained trade routes and relations with
native tribes in Denmark, and Roman coins have been found in Denmark.
Evidence of strong Celtic cultural influence dates from this period in
Denmark and much of North-West Europe and is among other things
reflected in the finding of the Gundestrup cauldron.

The tribal Danes came from the east Danish islands (Zealand) and Scania
and spoke an early form of North Germanic. Historians believe that
before their arrival, most of Jutland and the nearest islands were
settled by tribal Jutes. The Jutes migrated to Great Britain eventually,
some as mercenaries of Brythonic King Vortigern, and were granted the
south-eastern territories of Kent, the Isle of Wight and other areas,
where they settled. They were later absorbed or ethnically cleansed by
the invading Angles and Saxons, who formed the Anglo-Saxons. The
remaining Jutish population in Jutland assimilated in with the settling
Danes.

A short note about the Dani in "Getica" by the historian Jordanes is
believed to be an early mention of the Danes, one of the ethnic groups
from whom modern Danes are descended. The Danevirke defence structures
were built in phases from the 3rd century forward and the sheer size of
the construction efforts in AD 737 are attributed to the emergence of a
Danish king. A new runic alphabet was first used around the same time
and Ribe, the oldest town of Denmark, was founded about AD 700.

\section{Viking and Middle Ages}\label{viking-and-middle-ages}

\begin{itemize}
\item
  \emph{More Anglo-Saxon pence of this period have been found in Denmark
  than in England.}
\item
  \emph{Under the reign of Gudfred in 804 the Danish kingdom may have
  included all the lands of Jutland, Scania and the Danish islands,
  excluding Bornholm.}
\item
  \emph{However, even from the start, Margaret may not have been so
  idealistic---treating Denmark as the clear "senior" partner of the
  union.}
\item
  \emph{Later that year, Denmark entered into a union with Norway.}
\end{itemize}

From the 8th to the 10th century the wider Scandinavian region was the
source of Vikings. They colonised, raided, and traded in all parts of
Europe. The Danish Vikings were most active in the eastern and southern
British Isles and Western Europe. They conquered and settled parts of
England (known as the Danelaw) under King Sweyn Forkbeard in 1013, and
France where Danes and Norwegians founded Normandy with Rollo as head of
state. More Anglo-Saxon pence of this period have been found in Denmark
than in England.

Denmark was largely consolidated by the late 8th century and its rulers
are consistently referred to in Frankish sources as kings (reges). Under
the reign of Gudfred in 804 the Danish kingdom may have included all the
lands of Jutland, Scania and the Danish islands, excluding Bornholm.\\
The extant Danish monarchy traces its roots back to Gorm the Old, who
established his reign in the early 10th century. As attested by the
Jelling stones, the Danes were Christianised around 965 by Harald
Bluetooth, the son of Gorm. It is believed that Denmark became Christian
for political reasons so as not to get invaded by the rising Christian
power in Europe, the Holy Roman Empire, which was an important trading
area for the Danes. In that case, Harald built six fortresses around
Denmark called Trelleborg and built a further Danevirke. In the early
11th century, Canute the Great won and united Denmark, England, and
Norway for almost 30 years with a Scandinavian army.

Throughout the High and Late Middle Ages, Denmark also included
Skåneland (the areas of Scania, Halland, and Blekinge in present-day
south Sweden) and Danish kings ruled Danish Estonia, as well as the
duchies of Schleswig and Holstein. Most of the latter two now form the
state of Schleswig-Holstein in northern Germany.

In 1397, Denmark entered into a personal union with Norway and Sweden,
united under Queen Margaret I. The three countries were to be treated as
equals in the union. However, even from the start, Margaret may not have
been so idealistic---treating Denmark as the clear "senior" partner of
the union. Thus, much of the next 125 years of Scandinavian history
revolves around this union, with Sweden breaking off and being
re-conquered repeatedly. The issue was for practical purposes resolved
on 17 June 1523, as Swedish King Gustav Vasa conquered the city of
Stockholm. The Protestant Reformation spread to Scandinavia in the
1530s, and following the Count's Feud civil war, Denmark converted to
Lutheranism in 1536. Later that year, Denmark entered into a union with
Norway.

\section{Early modern history
(1536--1849)}\label{early-modern-history-15361849}

\begin{itemize}
\item
  \emph{Denmark kept the possessions of Iceland (which retained the
  Danish monarchy until 1944), the Faroe Islands and Greenland, all of
  which had been governed by Norway for centuries.}
\item
  \emph{Apart from the Nordic colonies, Denmark continued to rule over
  Danish India from 1620 to 1869, the Danish Gold Coast (Ghana) from
  1658 to 1850, and the Danish West Indies from 1671 to 1917.}
\end{itemize}

After Sweden permanently broke away from the personal union, Denmark
tried on several occasions to reassert control over its neighbour. King
Christian IV attacked Sweden in the 1611--1613 Kalmar War but failed to
accomplish his main objective of forcing it to return to the union. The
war led to no territorial changes, but Sweden was forced to pay a war
indemnity of 1 million silver riksdaler to Denmark, an amount known as
the Älvsborg ransom. King Christian used this money to found several
towns and fortresses, most notably Glückstadt (founded as a rival to
Hamburg) and Christiania. Inspired by the Dutch East India Company, he
founded a similar Danish company and planned to claim Ceylon as a
colony, but the company only managed to acquire Tranquebar on India's
Coromandel Coast. Denmark's large colonial aspirations included a few
key trading posts in Africa and India. While Denmark's trading posts in
India were of little note, it played an important role in the highly
lucrative transatlantic slave trade, through its trading outposts in
Fort Cristiansborg in Osu, Ghana though which 1.5 million slaves were
traded. While the Danish colonial empire was sustained by trade with
other major powers, and plantations -- ultimately a lack of resources
led to its stagnation.

In the Thirty Years' War, Christian tried to become the leader of the
Lutheran states in Germany but suffered a crushing defeat at the Battle
of Lutter. The result was that the Catholic army under Albrecht von
Wallenstein was able to invade, occupy, and pillage Jutland, forcing
Denmark to withdraw from the war. Denmark managed to avoid territorial
concessions, but King Gustavus Adolphus' intervention in Germany was
seen as a sign that the military power of Sweden was on the rise while
Denmark's influence in the region was declining. Swedish armies invaded
Jutland in 1643 and claimed Scania in 1644.

In the 1645 Treaty of Brømsebro, Denmark surrendered Halland, Gotland,
the last parts of Danish Estonia, and several provinces in Norway. In
1657, King Frederick III declared war on Sweden and marched on
Bremen-Verden. This led to a massive Danish defeat and the armies of
King Charles X Gustav of Sweden conquered Jutland, Funen, and much of
Zealand before signing the Peace of Roskilde in February 1658, which
gave Sweden control of Scania, Blekinge, Trøndelag, and the island of
Bornholm. Charles X Gustav quickly regretted not having ruined Denmark
and in August 1658, he began a two-year-long siege of Copenhagen but he
failed to take the capital. In the ensuing peace settlement, Denmark
managed to maintain its independence and regain control of Trøndelag and
Bornholm.

Denmark tried but failed to regain control of Scania in the Scanian War
(1675--1679). After the Great Northern War (1700--21), Denmark managed
to regain control of the parts of Schleswig and Holstein ruled by the
house of Holstein-Gottorp in the 1720 Treaty of Frederiksborg and the
1773 Treaty of Tsarskoye Selo, respectively. Denmark prospered greatly
in the last decades of the 18th century due to its neutral status
allowing it to trade with both sides in the many contemporary wars. In
the Napoleonic Wars, Denmark traded with both France and the United
Kingdom and joined the League of Armed Neutrality with Russia, Sweden,
and Prussia. The British considered this a hostile act and attacked
Copenhagen in 1801 and 1807, in one case carrying off the Danish fleet,
in the other, burning large parts of the Danish capital. This led to the
so-called Danish-British Gunboat War. British control of the waterways
between Denmark and Norway proved disastrous to the union's economy and
in 1813 Denmark--Norway went bankrupt.

The union was dissolved by the Treaty of Kiel in 1814; the Danish
monarchy "irrevocably and forever" renounced claims to the Kingdom of
Norway in favour of the Swedish king. Denmark kept the possessions of
Iceland (which retained the Danish monarchy until 1944), the Faroe
Islands and Greenland, all of which had been governed by Norway for
centuries. Apart from the Nordic colonies, Denmark continued to rule
over Danish India from 1620 to 1869, the Danish Gold Coast (Ghana) from
1658 to 1850, and the Danish West Indies from 1671 to 1917.

\includegraphics[width=5.50000in,height=3.68457in]{media/image1.jpg}\\
\emph{The National Constitutional Assembly was convened by King
Frederick VII in 1848 to adopt the Constitution of Denmark.}

\section{Constitutional monarchy
(1849--present)}\label{constitutional-monarchy-1849present}

\begin{itemize}
\item
  \emph{Denmark maintained its neutral stance during World War I.}
\item
  \emph{In 1939 Denmark signed a 10-year non-aggression pact with Nazi
  Germany but Germany invaded Denmark on 9 April 1940 and the Danish
  government quickly surrendered.}
\item
  \emph{After these events, Denmark pursued a policy of neutrality in
  Europe.}
\item
  \emph{Denmark was defeated and obliged to cede Schleswig and Holstein
  to Prussia.}
\end{itemize}

A nascent Danish liberal and national movement gained momentum in the
1830s; after the European Revolutions of 1848, Denmark peacefully became
a constitutional monarchy on 5 June 1849. A new constitution established
a two-chamber parliament. Denmark faced war against both Prussia and
Habsburg Austria in what became known as the Second Schleswig War,
lasting from February to October 1864. Denmark was defeated and obliged
to cede Schleswig and Holstein to Prussia. This loss came as the latest
in the long series of defeats and territorial losses that had begun in
the 17th century. After these events, Denmark pursued a policy of
neutrality in Europe.

Industrialisation came to Denmark in the second half of the 19th
century. The nation's first railways were constructed in the 1850s, and
improved communications and overseas trade allowed industry to develop
in spite of Denmark's lack of natural resources. Trade unions developed,
starting in the 1870s. There was a considerable migration of people from
the countryside to the cities, and Danish agriculture became centred on
the export of dairy and meat products.

Denmark maintained its neutral stance during World War I. After the
defeat of Germany, the Versailles powers offered to return the region of
Schleswig-Holstein to Denmark. Fearing German irredentism, Denmark
refused to consider the return of the area without a plebiscite; the two
Schleswig Plebiscites took place on 10 February and 14 March 1920,
respectively. On 10 July 1920, Northern Schleswig was recovered by
Denmark, thereby adding some 163,600 inhabitants and 3,984 square
kilometres (1,538~sq~mi). The country's first social democratic
government took office in 1924.

In 1939 Denmark signed a 10-year non-aggression pact with Nazi Germany
but Germany invaded Denmark on 9 April 1940 and the Danish government
quickly surrendered. World War II in Denmark was characterised by
economic co-operation with Germany until 1943, when the Danish
government refused further co-operation and its navy scuttled most of
its ships and sent many of its officers to Sweden, which was neutral.
The Danish resistance performed a rescue operation that managed to
evacuate several thousand Jews and their families to safety in Sweden
before the Germans could send them to death camps. Some Danes supported
Nazism by joining the Danish Nazi Party or volunteering to fight with
Germany as part of the Frikorps Danmark. Iceland severed ties with
Denmark and became an independent republic in 1944; Germany surrendered
in May 1945; in 1948, the Faroe Islands gained home rule; in 1949,
Denmark became a founding member of NATO.

Denmark was a founding member of European Free Trade Association (EFTA).
During the 1960s, the EFTA countries were often referred to as the Outer
Seven, as opposed to the Inner Six of what was then the European
Economic Community (EEC). In 1973, along with Britain and Ireland,
Denmark joined the European Economic Community (now the European Union)
after a public referendum. The Maastricht Treaty, which involved further
European integration, was rejected by the Danish people in 1992; it was
only accepted after a second referendum in 1993, which provided for four
opt-outs from policies. The Danes rejected the euro as the national
currency in a referendum in 2000. Greenland gained home rule in 1979 and
was awarded self-determination in 2009. Neither the Faroe Islands nor
Greenland are members of the European Union, the Faroese having declined
membership of the EEC in 1973 and Greenland in 1986, in both cases
because of fisheries policies.

Constitutional change in 1953 led to a single-chamber parliament elected
by proportional representation, female accession to the Danish throne,
and Greenland becoming an integral part of Denmark. The centre-left
Social Democrats led a string of coalition governments for most of the
second half of the 20th century, introducing the Nordic welfare model.
The Liberal Party and the Conservative People's Party have also led
centre-right governments.

\includegraphics[width=5.50000in,height=4.10040in]{media/image2.jpg}\\
\emph{A satellite image of Jutland and the Danish islands}

\includegraphics[width=5.11267in,height=5.50000in]{media/image3.png}\\
\emph{A map showing major urban areas, islands and connecting bridges}

\section{Geography}\label{geography}

\begin{itemize}
\item
  \emph{No location in Denmark is farther from the coast than 52~km
  (32~mi).}
\item
  \emph{Although once extensively forested, today Denmark largely
  consists of arable land.}
\item
  \emph{The Kingdom of Denmark includes two overseas territories, both
  well to the west of Denmark: Greenland, the world's largest island,
  and the Faroe Islands in the North Atlantic Ocean.}
\end{itemize}

Located in Northern Europe, Denmark consists of the peninsula of Jutland
and 443 named islands (1,419 islands above 100 square metres
(1,100~sq~ft) in total). Of these, 74 are inhabited (January 2015), with
the largest being Zealand, the North Jutlandic Island, and Funen. The
island of Bornholm is located east of the rest of the country, in the
Baltic Sea. Many of the larger islands are connected by bridges; the
Øresund Bridge connects Zealand with Sweden; the Great Belt Bridge
connects Funen with Zealand; and the Little Belt Bridge connects Jutland
with Funen. Ferries or small aircraft connect to the smaller islands.
The four cities with populations over 100,000 are the capital Copenhagen
on Zealand; Aarhus and Aalborg in Jutland; and Odense on Funen.

The country occupies a total area of 42,924 square kilometres
(16,573~sq~mi) The area of inland water is 700~km2 (270~sq~mi),
variously stated as from 500 -- 700~km2 (193--270 sq mi). Lake Arresø
northwest of Copenhagen is the largest lake. The size of the land area
cannot be stated exactly since the ocean constantly erodes and adds
material to the coastline, and because of human land reclamation
projects (to counter erosion). Post-glacial rebound raises the land by a
bit less than 1~cm (0.4~in) per year in the north and east, extending
the coast. A circle enclosing the same area as Denmark would be 234
kilometres (145 miles) in diameter with a circumference of 736~km
(457~mi) (land area only:232.33 and 730 respectively). It shares a
border of 68 kilometres (42~mi) with Germany to the south and is
otherwise surrounded by 8,750~km (5,437~mi) of tidal shoreline
(including small bays and inlets). No location in Denmark is farther
from the coast than 52~km (32~mi). On the south-west coast of Jutland,
the tide is between 1 and 2~m (3.28 and 6.56~ft), and the tideline moves
outward and inward on a 10~km (6.2~mi) stretch. Denmark's territorial
waters total 105,000 square kilometres (40,541 square miles).

Denmark's northernmost point is Skagen point (the north beach of the
Skaw) at 57° 45' 7" northern latitude; the southernmost is Gedser point
(the southern tip of Falster) at 54° 33' 35" northern latitude; the
westernmost point is Blåvandshuk at 8° 4' 22" eastern longitude; and the
easternmost point is Østerskær at 15° 11' 55" eastern longitude. This is
in the Ertholmene archipelago 18 kilometres (11~mi) north-east of
Bornholm. The distance from east to west is 452 kilometres (281~mi),
from north to south 368 kilometres (229~mi).

The country is flat with little elevation, having an average height
above sea level of 31 metres (102~ft). The highest natural point is
Møllehøj, at 170.86 metres (560.56~ft). A sizeable portion of Denmark's
terrain consists of rolling plains whilst the coastline is sandy, with
large dunes in northern Jutland. Although once extensively forested,
today Denmark largely consists of arable land. It is drained by a dozen
or so rivers, and the most significant include the Gudenå, Odense,
Skjern, Suså and Vidå---a river that flows along its southern border
with Germany.

The Kingdom of Denmark includes two overseas territories, both well to
the west of Denmark: Greenland, the world's largest island, and the
Faroe Islands in the North Atlantic Ocean. These territories are
self-governing and form part of the Danish Realm.

\section{Climate}\label{climate}

\begin{itemize}
\item
  \emph{Because of Denmark's northern location, there are large seasonal
  variations in daylight.}
\item
  \emph{The most extreme temperatures recorded in Denmark, since 1874
  when recordings began, was 36.4~°C (97.5~°F) in 1975 and −31.2~°C
  (−24.2~°F) in 1982.}
\item
  \emph{Denmark has a temperate climate, characterised by mild winters,
  with mean temperatures in January of 1.5~°C (34.7~°F), and cool
  summers, with a mean temperature in August of 17.2~°C (63.0~°F).}
\end{itemize}

Denmark has a temperate climate, characterised by mild winters, with
mean temperatures in January of 1.5~°C (34.7~°F), and cool summers, with
a mean temperature in August of 17.2~°C (63.0~°F). The most extreme
temperatures recorded in Denmark, since 1874 when recordings began, was
36.4~°C (97.5~°F) in 1975 and −31.2~°C (−24.2~°F) in 1982. Denmark has
an average of 179 days per year with precipitation, on average receiving
a total of 765 millimetres (30~in) per year; autumn is the wettest
season and spring the driest. The position between a continent and an
ocean means that weather often changes.

Because of Denmark's northern location, there are large seasonal
variations in daylight. There are short days during the winter with
sunrise coming around 8:45~am and sunset 3:45~pm (standard time), as
well as long summer days with sunrise at 4:30~am and sunset at 10~pm
(daylight saving time).

\section{Ecology}\label{ecology}

\begin{itemize}
\item
  \emph{Denmark is also home to smaller mammals, such as polecats, hares
  and hedgehogs.}
\item
  \emph{Denmark belongs to the Boreal Kingdom and can be subdivided into
  two ecoregions: the Atlantic mixed forests and Baltic mixed forests.}
\item
  \emph{Approximately 400 bird species inhabit Denmark and about 160 of
  those breed in the country.}
\end{itemize}

Denmark belongs to the Boreal Kingdom and can be subdivided into two
ecoregions: the Atlantic mixed forests and Baltic mixed forests. Almost
all of Denmark's primeval temperate forests have been destroyed or
fragmented, chiefly for agricultural purposes during the last millennia.
The deforestation has created large swaths of heathland and devastating
sand drifts. In spite of this, there are several larger second growth
woodlands in the country and, in total, 12.9\% of the land is now
forested. Norway spruce is the most widespread tree (2017), being
important in the production of Christmas trees.

Roe deer occupy the countryside in growing numbers, and large-antlered
red deer can be found in the sparse woodlands of Jutland. Denmark is
also home to smaller mammals, such as polecats, hares and hedgehogs.
Approximately 400 bird species inhabit Denmark and about 160 of those
breed in the country. Large marine mammals include healthy populations
of Harbour porpoise, growing numbers of pinnipeds and occasional visits
of large whales, including blue whales and orcas. Cod, herring and
plaice are abundant fish in Danish waters and form the basis for a large
fishing industry.

\section{Environment}\label{environment}

\begin{itemize}
\item
  \emph{Denmark performs worst (i.e.}
\item
  \emph{Denmark's territories, Greenland and the Faroe Islands, catch
  approximately 650 whales per year.}
\item
  \emph{Denmark has an outstanding performance in the global
  Environmental Performance Index (EPI) with an overall ranking of 4 out
  of 180 countries in 2016.}
\item
  \emph{The environmental areas where Denmark performs best (i.e.}
\end{itemize}

Land and water pollution are two of Denmark's most significant
environmental issues, although much of the country's household and
industrial waste is now increasingly filtered and sometimes recycled.
The country has historically taken a progressive stance on environmental
preservation; in 1971 Denmark established a Ministry of Environment and
was the first country in the world to implement an environmental law in
1973. To mitigate environmental degradation and global warming the
Danish Government has signed the Climate Change-Kyoto Protocol. However,
the national ecological footprint is 8.26 global hectares per person,
which is very high compared to a world average of 1.7 in 2010.
Contributing factors to this value are an exceptional high value for
cropland but also a relatively high value for grazing land, which may be
explained by the substantially high meat production in Denmark (115.8
kilograms (255~lb) meat annually per capita) and the large economic role
of the meat and dairy industries. In December 2014, the Climate Change
Performance Index for 2015 placed Denmark at the top of the table,
explaining that although emissions are still quite high, the country was
able to implement effective climate protection policies.

Denmark has an outstanding performance in the global Environmental
Performance Index (EPI) with an overall ranking of 4 out of 180
countries in 2016. This recent and significant increase in ranking and
performance is mostly due to remarkable achievements in energy
efficiency and reductions in CO2 emission levels. A future
implementation of air quality improvements are expected. The EPI was
established in 2001 by the World Economic Forum as a global gauge to
measure how well individual countries perform in implementing the United
Nations' Sustainable Development Goals. The environmental areas where
Denmark performs best (i.e. lowest ranking) are sanitation (12), water
resource management (13) and health impacts of environmental issues
(14), followed closely by the area of biodiversity and habitat. The
latter are due to the many protection laws and protected areas of
significance within the country even though the EPI is not considering
how well these laws and regulations are affecting the current
biodiversity and habitats in reality; one of many weaknesses in the EPI.
Denmark performs worst (i.e. highest ranking) in the areas of
environmental effects of fisheries (128) and forest management (96). The
very poor ranking in the fisheries area are due to alarmingly low and
continually rapidly declining fish stocks, placing Denmark among the
worst performing countries of the world. Denmark's territories,
Greenland and the Faroe Islands, catch approximately 650 whales per
year. Greenland's quotas for the catch of whales are determined
according to the advice of the International Whaling Commission (IWC),
having quota decision-making powers.

\section{Administrative divisions}\label{administrative-divisions}

\begin{itemize}
\item
  \emph{Denmark, with a total area of 43,094 square kilometres
  (16,639~sq~mi), is divided into five administrative regions (Danish:
  regioner).}
\item
  \emph{The easternmost land in Denmark, the Ertholmene archipelago,
  with an area of 39 hectares (0.16 sq mi), is neither part of a
  municipality nor a region but belongs to the Ministry of Defence.}
\end{itemize}

Denmark, with a total area of 43,094 square kilometres (16,639~sq~mi),
is divided into five administrative regions (Danish: regioner). The
regions are further subdivided into 98 municipalities (kommuner). The
easternmost land in Denmark, the Ertholmene archipelago, with an area of
39 hectares (0.16 sq mi), is neither part of a municipality nor a region
but belongs to the Ministry of Defence.

The regions were created on 1 January 2007 to replace the 16 former
counties. At the same time, smaller municipalities were merged into
larger units, reducing the number from 270. Most municipalities have a
population of at least 20,000 to give them financial and professional
sustainability, although a few exceptions were made to this rule. The
administrative divisions are led by directly elected councils, elected
proportionally every four years; the most recent Danish local elections
were held on 21 November 2017. Other regional structures use the
municipal boundaries as a layout, including the police districts, the
court districts and the electoral wards.

\section{Regions}\label{regions}

\begin{itemize}
\item
  \emph{The area and populations of the regions vary widely; for
  example, the Capital Region, which encompasses the Copenhagen
  metropolitan area with the exception of the subtracted province East
  Zealand but includes the Baltic Sea island of Bornholm, has a
  population three times larger than that of North Denmark Region, which
  covers the more sparsely populated area of northern Jutland.}
\end{itemize}

The governing bodies of the regions are the regional councils, each with
forty-one councillors elected for four-year terms. The councils are
headed by regional district chairmen (regionsrådsformanden), who are
elected by the council.\\
The areas of responsibility for the regional councils are the national
health service, social services and regional development. Unlike the
counties they replaced, the regions are not allowed to levy taxes and
the health service is partly financed by a national health care
contribution until 2018 (sundhedsbidrag), partly by funds from both
government and municipalities. From 1 January 2019 this contribution
will be abolished, as it is being replaced by higher income tax instead.

The area and populations of the regions vary widely; for example, the
Capital Region, which encompasses the Copenhagen metropolitan area with
the exception of the subtracted province East Zealand but includes the
Baltic Sea island of Bornholm, has a population three times larger than
that of North Denmark Region, which covers the more sparsely populated
area of northern Jutland. Under the county system certain densely
populated municipalities, such as Copenhagen Municipality and
Frederiksberg, had been given a status equivalent to that of counties,
making them first-level administrative divisions. These sui generis
municipalities were incorporated into the new regions under the 2007
reforms.

\section{Greenland and the Faroe
Islands}\label{greenland-and-the-faroe-islands}

\begin{itemize}
\item
  \emph{The Kingdom of Denmark is a unitary state that comprises, in
  addition to Denmark proper, two autonomous constituent countries in
  the North Atlantic Ocean: Greenland and the Faroe Islands.}
\end{itemize}

The Kingdom of Denmark is a unitary state that comprises, in addition to
Denmark proper, two autonomous constituent countries in the North
Atlantic Ocean: Greenland and the Faroe Islands. They have been
integrated parts of the Danish Realm since the 18th century; however,
due to their separate historical and cultural identities, these parts of
the Realm have extensive political powers and have assumed legislative
and administrative responsibility in a substantial number of fields.
Home rule was granted to the Faroe Islands in 1948 and to Greenland in
1979, each having previously had the status of counties.

Greenland and the Faroe Islands have their own home governments and
parliaments and are effectively self-governing in regards to domestic
affairs. High Commissioners (Rigsombudsmand) act as representatives of
the Danish government in the Faroese Løgting and in the Greenlandic
Parliament, but they cannot vote. The Faroese home government is defined
to be an equal partner with the Danish national government, while the
Greenlandic people are defined as a separate people with the right to
self-determination.

\section{Politics}\label{politics}

\begin{itemize}
\item
  \emph{Politics in Denmark operate under a framework laid out in the
  Constitution of Denmark.}
\end{itemize}

Politics in Denmark operate under a framework laid out in the
Constitution of Denmark. First written in 1849, it establishes a
sovereign state in the form of a constitutional monarchy, with a
representative parliamentary system. The monarch officially retains
executive power and presides over the Council of State (privy council).
In practice, the duties of the Monarch are strictly representative and
ceremonial, such as the formal appointment and dismissal of the Prime
Minister and other Government ministers. The Monarch is not answerable
for his or her actions, and their person is sacrosanct. Hereditary
monarch Queen Margrethe II has been head of state since 14 January 1972.

\section{Government}\label{government}

\begin{itemize}
\item
  \emph{It is the legislature of the Kingdom of Denmark, passing acts
  that apply in Denmark and, variably, Greenland and the Faroe Islands.}
\item
  \emph{The Danish Parliament is unicameral and called the Folketing
  (Danish: Folketinget).}
\item
  \emph{Denmark is a representative democracy with universal suffrage.}
\end{itemize}

The Danish Parliament is unicameral and called the Folketing (Danish:
Folketinget). It is the legislature of the Kingdom of Denmark, passing
acts that apply in Denmark and, variably, Greenland and the Faroe
Islands. The Folketing is also responsible for adopting the state's
budgets, approving the state's accounts, appointing and exercising
control of the Government, and taking part in international
co-operation. Bills may be initiated by the Government or by members of
parliament. All bills passed must be presented before the Council of
State to receive Royal Assent within thirty days in order to become law.

Denmark is a representative democracy with universal suffrage.
Membership of the Folketing is based on proportional representation of
political parties, with a 2\% electoral threshold. Danes elect 175
members to the Folketing, with Greenland and the Faroe Islands electing
an additional two members each---179 members in total. Parliamentary
elections are held at least every four years, but it is within the
powers of the Prime Minister to ask the Monarch to call for an election
before the term has elapsed. On a vote of no confidence, the Folketing
may force a single minister or an entire government to resign.

The Government of Denmark operates as a cabinet government, where
executive authority is exercised---formally, on behalf of the Monarch,
by the Prime Minister and other cabinet ministers, who head ministries.
As the executive branch, the Cabinet is responsible for proposing bills
and a budget, executing the laws, and guiding the foreign and internal
policies of Denmark. The position of prime minister belongs to the
person most likely to command the confidence of a majority in the
Folketing; this is usually the current leader of the largest political
party or, more effectively, through a coalition of parties. A single
party generally does not have sufficient political power in terms of the
number of seats to form a cabinet on its own; Denmark has often been
ruled by coalition governments, themselves sometimes minority
governments dependent on non-government parties.

Following a general election defeat, in June 2015 Helle
Thorning-Schmidt, leader of the Social Democrats (Socialdemokraterne),
resigned as Prime Minister. She was succeeded by Lars Løkke Rasmussen,
the leader of the Liberal Party (Venstre). Rasmussen became the leader
of a cabinet that, unusually, consisted entirely of ministers from his
own party. In the next cabinet, created in November 2016, there are
several political parties represented.

\includegraphics[width=5.20300in,height=5.50000in]{media/image4.jpg}\\
\emph{King Christian V presiding over the Supreme Court in 1697 }

\section{Law and judicial system}\label{law-and-judicial-system}

\begin{itemize}
\item
  \emph{The Kingdom of Denmark does not have a single unified judicial
  system -- Denmark has one system, Greenland another, and the Faroe
  Islands a third.}
\item
  \emph{However, decisions by the highest courts in Greenland and the
  Faroe Islands may be appealed to the Danish High Courts.}
\item
  \emph{Denmark has a civil law system with some references to Germanic
  law.}
\end{itemize}

Denmark has a civil law system with some references to Germanic law.
Denmark resembles Norway and Sweden in never having developed a case-law
like that of England and the United States nor comprehensive codes like
those of France and Germany. Much of its law is customary.

The judicial system of Denmark is divided between courts with regular
civil and criminal jurisdiction and administrative courts with
jurisdiction over litigation between individuals and the public
administration. Articles sixty-two and sixty-four of the Constitution
ensure judicial independence from government and Parliament by providing
that judges shall only be guided by the law, including acts, statutes
and practice. The Kingdom of Denmark does not have a single unified
judicial system -- Denmark has one system, Greenland another, and the
Faroe Islands a third. However, decisions by the highest courts in
Greenland and the Faroe Islands may be appealed to the Danish High
Courts. The Danish Supreme Court is the highest civil and criminal court
responsible for the administration of justice in the Kingdom.

\section{Foreign relations}\label{foreign-relations}

\begin{itemize}
\item
  \emph{Denmark wields considerable influence in Northern Europe and is
  a middle power in international affairs.}
\item
  \emph{Following World War II, Denmark ended its two-hundred-year-long
  policy of neutrality.}
\item
  \emph{The foreign policy of Denmark is substantially influenced by its
  membership of the European Union (EU); Denmark including Greenland
  joined the European Economic Community (EEC), the EU's predecessor, in
  1973.}
\end{itemize}

Denmark wields considerable influence in Northern Europe and is a middle
power in international affairs. In recent years, Greenland and the Faroe
Islands have been guaranteed a say in foreign policy issues such as
fishing, whaling, and geopolitical concerns. The foreign policy of
Denmark is substantially influenced by its membership of the European
Union (EU); Denmark including Greenland joined the European Economic
Community (EEC), the EU's predecessor, in 1973. Denmark held the
Presidency of the Council of the European Union on seven occasions, most
recently from January to June 2012. Following World War II, Denmark
ended its two-hundred-year-long policy of neutrality. It has been a
founding member of the North Atlantic Treaty Organization (NATO) since
1949, and membership remains highly popular.

As a member of Development Assistance Committee (DAC), Denmark has for a
long time been among the countries of the world contributing the largest
percentage of gross national income to development aid. In 2015, Denmark
contributed 0.85\% of its gross national income (GNI) to foreign aid and
was one of only six countries meeting the longstanding UN target of
0.7\% of GNI. The country participates in both bilateral and
multilateral aid, with the aid usually administered by the Ministry of
Foreign Affairs. The organisational name of Danish International
Development Agency (DANIDA) is often used, in particular when operating
bilateral aid.

\includegraphics[width=5.50000in,height=3.52263in]{media/image5.JPG}\\
\emph{Danish MP-soldiers conducting advanced law enforcement training}

\section{Military}\label{military}

\begin{itemize}
\item
  \emph{Denmark is a long-time supporter of international peacekeeping,
  but since the NATO bombing of Yugoslavia in 1999 and the War in
  Afghanistan in 2001, Denmark has also found a new role as a warring
  nation, participating actively in several wars and invasions.}
\item
  \emph{Denmark's armed forces are known as the Danish Defence (Danish:
  Forsvaret).}
\item
  \emph{These initiatives are often described by the authorities as part
  of a new "active foreign policy" of Denmark.}
\end{itemize}

Denmark's armed forces are known as the Danish Defence (Danish:
Forsvaret). The Minister of Defence is commander-in-chief of the Danish
Defence, and serves as chief diplomatic official abroad. During
peacetime, the Ministry of Defence employs around 33,000 in total. The
main military branches employ almost 27,000: 15,460 in the Royal Danish
Army, 5,300 in the Royal Danish Navy and 6,050 in the Royal Danish Air
Force (all including conscripts).{[}citation needed{]} The Danish
Emergency Management Agency employs 2,000 (including conscripts), and
about 4,000 are in non-branch-specific services like the Danish Defence
Command and the Danish Defence Intelligence Service. Furthermore, around
55,000 serve as volunteers in the Danish Home Guard.

Denmark is a long-time supporter of international peacekeeping, but
since the NATO bombing of Yugoslavia in 1999 and the War in Afghanistan
in 2001, Denmark has also found a new role as a warring nation,
participating actively in several wars and invasions. This relatively
new situation has stirred some internal critique, but the Danish
population has generally been very supportive, in particular of the War
in Afghanistan. The Danish Defence has around 1,400 staff in
international missions, not including standing contributions to NATO
SNMCMG1. Danish forces were heavily engaged in the former Yugoslavia in
the UN Protection Force (UNPROFOR), with IFOR, and now SFOR. Between
2003 and 2007, there were approximately 450 Danish soldiers in Iraq.
Denmark also strongly supported American operations in Afghanistan and
has contributed both monetarily and materially to the ISAF. These
initiatives are often described by the authorities as part of a new
"active foreign policy" of Denmark.

\includegraphics[width=5.50000in,height=3.41000in]{media/image6.jpg}\\
\emph{Denmark is a leading producer of pork, and the largest exporter of
pork products in the EU.}

\section{Economy}\label{economy}

\begin{itemize}
\item
  \emph{Denmark has a developed mixed economy that is classed as a
  high-income economy by the World Bank.}
\item
  \emph{According to the International Monetary Fund, Denmark has the
  world's highest minimum wage.}
\item
  \emph{Ranked by turnover in Denmark, the largest Danish companies are:
  A.P.}
\item
  \emph{Denmark has the fourth highest ratio of tertiary degree holders
  in the world.}
\end{itemize}

Denmark has a developed mixed economy that is classed as a high-income
economy by the World Bank. In 2017 it ranked 16th in the world in terms
of gross national income (PPP) per capita and 10th in nominal GNI per
capita. Denmark's economy stands out as one of the most free in the
Index of Economic Freedom and the Economic Freedom of the World. It is
the 10th most competitive economy in the world, and 6th in Europe,
according to the World Economic Forum in its Global Competitiveness
Report 2018.

Denmark has the fourth highest ratio of tertiary degree holders in the
world. The country ranks highest in the world for workers' rights. GDP
per hour worked was the 13th highest in 2009. The country has a market
income inequality close to the OECD average, but after taxes and public
cash transfers the income inequality is considerably lower. According to
Eurostat, Denmark's Gini coefficient for disposable income was the
7th-lowest among EU countries in 2017.\\
According to the International Monetary Fund, Denmark has the world's
highest minimum wage. As Denmark has no minimum wage legislation, the
high wage floor has been attributed to the power of trade unions. For
example, as the result of a collective bargaining agreement between the
3F trade union and the employers group Horesta, workers at McDonald's
and other fast food chains make the equivalent of US\$20 an hour, which
is more than double what their counterparts earn in the United States,
and have access to five weeks' paid vacation, parental leave and a
pension plan. Union density in 2015 was 68\%.

Once a predominantly agricultural country on account of its arable
landscape, since 1945 Denmark has greatly expanded its industrial base
and service sector. By 2017 services contributed circa 75\% of GDP,
manufacturing about 15\% and agriculture less than 2\%. Major industries
include wind turbines, pharmaceuticals, medical equipment, machinery and
transportation equipment, food processing, and construction. Circa 60\%
of the total export value is due to export of goods, and the remaining
40\% is from service exports, mainly sea transport. The country's main
export goods are: wind turbines, pharmaceuticals, machinery and
instruments, meat and meat products, dairy products, fish, furniture and
design. Denmark is a net exporter of food and energy and has for a
number of years had a balance of payments surplus which has transformed
the country from a net debitor to a net creditor country. By 1 July
2018, the net international investment position (or net foreign assets)
of Denmark was equal to 64.6\% of GDP.

A liberalisation of import tariffs in 1797 marked the end of
mercantilism and further liberalisation in the 19th and the beginning of
the 20th century established the Danish liberal tradition in
international trade that was only to be broken by the 1930s. Even when
other countries, such as Germany and France, raised protection for their
agricultural sector because of increased American competition resulting
in much lower agricultural prices after 1870, Denmark retained its free
trade policies, as the country profited from the cheap imports of
cereals (used as feedstuffs for their cattle and pigs) and could
increase their exports of butter and meat of which the prices were more
stable. Today, Denmark is part of the European Union's internal market,
which represents more than 508 million consumers. Several domestic
commercial policies are determined by agreements among European Union
(EU) members and by EU legislation. Support for free trade is high among
the Danish public; in a 2016 poll 57\% responded saw globalisation as an
opportunity whereas 18\% viewed it as a threat. 70\% of trade flows are
inside the European Union. As of 2017{[}update{]}, Denmark's largest
export partners are Germany, Sweden, the United Kingdom and the United
States.

Denmark's currency, the krone (DKK), is pegged at approximately 7.46
kroner per euro through the ERM II. Although a September 2000 referendum
rejected adopting the euro, the country follows the policies set forth
in the Economic and Monetary Union of the European Union and meets the
economic convergence criteria needed to adopt the euro. The majority of
the political parties in the Folketing support joining the Economic and
Monetary Union of the European Union\textbar{}EMU, but since 2010
opinion polls have consistently shown a clear majority against adopting
the euro. In May 2018, 29\% of respondents from Denmark in a
Eurobarometer opinion poll stated that they were in favour of the EMU
and the euro, whereas 65\% were against it.

Ranked by turnover in Denmark, the largest Danish companies are: A.P.
Møller-Mærsk (international shipping), Novo Nordisk (pharmaceuticals),
ISS A/S (facility services), Vestas (wind turbines), Arla Foods (dairy),
DSV (transport), Carlsberg Group (beer), Salling Group (retail), Ørsted
A/S (power), Danske Bank.

\section{Public policy}\label{public-policy}

\begin{itemize}
\item
  \emph{Denmark has a corporate tax rate of 22\% and a special
  time-limited tax regime for expatriates.}
\item
  \emph{Denmark has the 2nd lowest relative poverty rate in the OECD,
  below the 11.3\% OECD average.}
\item
  \emph{The share of the population reporting that they feel that they
  cannot afford to buy sufficient food in Denmark is less than half of
  the OECD average.}
\end{itemize}

Danes enjoy a high standard of living and the Danish economy is
characterised by extensive government welfare provisions. Denmark has a
corporate tax rate of 22\% and a special time-limited tax regime for
expatriates. The Danish taxation system is broad based, with a 25\%
value-added tax, in addition to excise taxes, income taxes and other
fees. The overall level of taxation (sum of all taxes, as a percentage
of GDP) was 46\% in 2017. The tax structure of Denmark (the relative
weight of different taxes) differs from the OECD average, as the Danish
tax system in 2015 was characterized by substantially higher revenues
from taxes on personal income and a lower proportion of revenues from
taxes on corporate income and gains and property taxes than in OECD
generally, whereas no revenues at all derive from social security
contributions. The proportion deriving from payroll taxes, VAT, and
other taxes on goods and services correspond to the OECD average

As of 2014{[}update{]}, 6\% of the population was reported to live below
the poverty line, when adjusted for taxes and transfers. Denmark has the
2nd lowest relative poverty rate in the OECD, below the 11.3\% OECD
average. The share of the population reporting that they feel that they
cannot afford to buy sufficient food in Denmark is less than half of the
OECD average.

\section{Labour market}\label{labour-market}

\begin{itemize}
\item
  \emph{With an employment rate in 2017 of 74.2\% for people aged
  15--64-years, Denmark ranks 9th highest among the OECD countries, and
  above the OECD average of 67.8\%.}
\item
  \emph{Like other Nordic countries, Denmark has adopted the Nordic
  Model, which combines free market capitalism with a comprehensive
  welfare state and strong worker protection.}
\item
  \emph{As a result of its acclaimed "flexicurity" model, Denmark has
  the freest labour market in Europe, according to the World Bank.}
\end{itemize}

Like other Nordic countries, Denmark has adopted the Nordic Model, which
combines free market capitalism with a comprehensive welfare state and
strong worker protection. As a result of its acclaimed "flexicurity"
model, Denmark has the freest labour market in Europe, according to the
World Bank. Employers can hire and fire whenever they want
(flexibility), and between jobs, unemployment compensation is relatively
high (security). According to OECD, initial as well as long-term net
replacement rates for unemployed persons were 65\% of previous net
income in 2016, against an OECD average of 53\%. Establishing a business
can be done in a matter of hours and at very low costs. No restrictions
apply regarding overtime work, which allows companies to operate 24
hours a day, 365 days a year. With an employment rate in 2017 of 74.2\%
for people aged 15--64-years, Denmark ranks 9th highest among the OECD
countries, and above the OECD average of 67.8\%. The unemployment rate
was 5.7\% in 2017, which is considered close to or below its structural
level.

The level of unemployment benefits is dependent on former employment and
normally on membership of an unemployment fund, which is usually closely
connected to a trade union, and previous payment of contributions. Circa
65\% of the financing comes from earmarked member contributions, whereas
the remaining third originates from the central government and hence
ultimately from general taxation.

\section{Science and technology}\label{science-and-technology}

\begin{itemize}
\item
  \emph{Denmark has a long tradition of scientific and technological
  invention and engagement, and has been involved internationally from
  the very start of the scientific revolution.}
\item
  \emph{In the software and electronic field, Denmark contributed to
  design and manufacturing of Nordic Mobile Telephones, and the
  now-defunct Danish company DanCall was among the first to develop GSM
  mobile phones.}
\end{itemize}

Denmark has a long tradition of scientific and technological invention
and engagement, and has been involved internationally from the very
start of the scientific revolution. In current times, Denmark is
participating in many high-profile international science and technology
projects, including CERN, ITER, ESA, ISS and E-ELT.

In the 20th century, Danes have also been innovative in several fields
of the technology sector. Danish companies have been influential in the
shipping industry with the design of the largest and most energy
efficient container ships in the world, the Maersk Triple E class, and
Danish engineers have contributed to the design of MAN Diesel engines.
In the software and electronic field, Denmark contributed to design and
manufacturing of Nordic Mobile Telephones, and the now-defunct Danish
company DanCall was among the first to develop GSM mobile phones.

Life science is a key sector with extensive research and development
activities. Danish engineers are world-leading in providing diabetes
care equipment and medication products from Novo Nordisk and, since
2000, the Danish biotech company Novozymes, the world market leader in
enzymes for first generation starch based bioethanol, has pioneered
development of enzymes for converting waste to cellulosic ethanol.
Medicon Valley, spanning the Øresund Region between Zealand and Sweden,
is one of Europe's largest life science clusters, containing a large
number of life science companies and research institutions located
within a very small geographical area.

Danish-born computer scientists and software engineers have taken
leading roles in some of the world's programming languages: Anders
Hejlsberg (Turbo Pascal, Delphi, C\#); Rasmus Lerdorf (PHP); Bjarne
Stroustrup (C++); David Heinemeier Hansson (Ruby on Rails); Lars Bak, a
pioneer in virtual machines (V8, Java VM, Dart). Physicist Lene
Vestergaard Hau is the first person to stop light, leading to advances
in quantum computing, nanoscale engineering and linear optics.

\section{Energy}\label{energy}

\begin{itemize}
\item
  \emph{Denmark's electricity sector has integrated energy sources such
  as wind power into the national grid.}
\item
  \emph{On 6 September 2012, Denmark launched the biggest wind turbine
  in the world, and will add four more over the next four years.}
\item
  \emph{Denmark is connected by electric transmission lines to other
  European countries.}
\end{itemize}

Denmark has considerably large deposits of oil and natural gas in the
North Sea and ranks as number 32 in the world among net exporters of
crude oil and was producing 259,980 barrels of crude oil a day in 2009.
Denmark is a long-time leader in wind power: In 2015 wind turbines
provided 42.1\% of the total electricity consumption. In May
2011{[}update{]} Denmark derived 3.1\% of its gross domestic product
from renewable (clean) energy technology and energy efficiency, or
around \euro{}6.5~billion (\$9.4~billion). Denmark is connected by
electric transmission lines to other European countries. On 6 September
2012, Denmark launched the biggest wind turbine in the world, and will
add four more over the next four years.{[}needs update{]}

Denmark's electricity sector has integrated energy sources such as wind
power into the national grid. Denmark now aims to focus on intelligent
battery systems (V2G) and plug-in vehicles in the transport sector. The
country is a member nation of the International Renewable Energy Agency
(IRENA).

\section{Transport}\label{transport}

\begin{itemize}
\item
  \emph{Construction of the Fehmarn Belt Fixed Link, connecting Denmark
  and Germany with a second link, will start in 2015.}
\item
  \emph{With Norway and Sweden, Denmark is part of the Scandinavian
  Airlines flag carrier.}
\item
  \emph{Cycling in Denmark is a very common form of transport,
  particularly for the young and for city dwellers.}
\end{itemize}

Significant investment has been made in building road and rail links
between regions in Denmark, most notably the Great Belt Fixed Link,
which connects Zealand and Funen. It is now possible to drive from
Frederikshavn in northern Jutland to Copenhagen on eastern Zealand
without leaving the motorway. The main railway operator is DSB for
passenger services and DB Schenker Rail for freight trains. The railway
tracks are maintained by Banedanmark. The North Sea and the Baltic Sea
are intertwined by various, international ferry links. Construction of
the Fehmarn Belt Fixed Link, connecting Denmark and Germany with a
second link, will start in 2015. Copenhagen has a rapid transit system,
the Copenhagen Metro, and an extensive electrified suburban railway
network, the S-train. In the four largest cities -- Copenhagen, Aarhus,
Odense, Aalborg -- light rail systems are planned to be in operation
around 2020.

Cycling in Denmark is a very common form of transport, particularly for
the young and for city dwellers. With a network of bicycle routes
extending more than 12,000~km and an estimated 7,000~km of segregated
dedicated bicycle paths and lanes, Denmark has a solid bicycle
infrastructure.

Private vehicles are increasingly used as a means of transport. Because
of the high registration tax (150\%), VAT (25\%), and one of the world's
highest income tax rates, new cars are very expensive. The purpose of
the tax is to discourage car ownership.\\
In 2007, an attempt was made by the government to favour environmentally
friendly cars by slightly reducing taxes on high mileage vehicles.
However, this has had little effect, and in 2008 Denmark experienced an
increase in the import of fuel inefficient old cars, as the cost for
older cars---including taxes---keeps them within the budget of many
Danes.\\
As of 2011{[}update{]}, the average car age is 9.2 years.

With Norway and Sweden, Denmark is part of the Scandinavian Airlines
flag carrier. Copenhagen Airport is Scandinavia's busiest passenger
airport, handling over 25 million passengers in 2014. Other notable
airports are Billund Airport, Aalborg Airport, and Aarhus Airport.

\section{Demographics}\label{demographics}

\begin{itemize}
\item
  \emph{The population of Denmark, as registered by Statistics Denmark,
  was 5.781 million in January 2018.}
\item
  \emph{Denmark is a historically homogeneous nation.}
\item
  \emph{There are no official statistics on ethnic groups, but according
  to 2018 figures from Statistics Denmark, 86.7\% of the population was
  of Danish descent, defined as having at least one parent who was born
  in Denmark and has Danish citizenship.}
\end{itemize}

The population of Denmark, as registered by Statistics Denmark, was
5.781 million in January 2018. Denmark has one of the oldest populations
in the world, with the average age of 41.9 years, with 0.97 males per
female. Despite a low birth rate, the population is growing at an
average annual rate of 0.59\% because of net immigration and increasing
longevity. The World Happiness Report frequently ranks Denmark's
population as the happiest in the world. This has been attributed to the
country's highly regarded education and health care systems, and its low
level of income inequality.

Denmark is a historically homogeneous nation. However, as with its
Scandinavian neighbours, Denmark has recently transformed from a nation
of net emigration, up until World War II, to a nation of net
immigration. Today, residence permits are issued mostly to immigrants
from other EU countries (54\% of all non-Scandinavian immigrants in
2017). Another 31\% of residence permits were study- or work-related,
4\% were issued to asylum seekers and 10\% to persons who arrive as
family dependants. Overall, the net migration rate in 2017 was 2.1
migrant(s)/1,000 population, somewhat lower than the United Kingdom and
the other Nordic countries.

There are no official statistics on ethnic groups, but according to 2018
figures from Statistics Denmark, 86.7\% of the population was of Danish
descent, defined as having at least one parent who was born in Denmark
and has Danish citizenship. The remaining 13.3\% were of foreign
background, defined as immigrants or descendants of recent immigrants.
With the same definition, the most common countries of origin were
Turkey, Poland, Syria, Germany, Iraq, Romania, Lebanon, Pakistan, Bosnia
and Hercegovina, and Somalia.

\section{Languages}\label{languages}

\begin{itemize}
\item
  \emph{Denmark had 25,900 native speakers of German in 2007 (mostly in
  the South Jutland area).}
\item
  \emph{Danish is the de facto national language of Denmark.}
\end{itemize}

Danish is the de facto national language of Denmark. Faroese and
Greenlandic are the official languages of the Faroe Islands and
Greenland respectively. German is a recognised minority language in the
area of the former South Jutland County (now part of the Region of
Southern Denmark), which was part of the German Empire prior to the
Treaty of Versailles. Danish and Faroese belong to the North Germanic
(Nordic) branch of the Indo-European languages, along with Icelandic,
Norwegian, and Swedish. There is a limited degree of mutual
intelligibility between Danish, Norwegian, and Swedish. Danish is more
distantly related to German, which is a West Germanic language.
Greenlandic or "Kalaallisut" belongs to the Eskimo--Aleut languages; it
is closely related to the Inuit languages in Canada, such as Inuktitut,
and entirely unrelated to Danish.

A large majority (86\%) of Danes speak English as a second language,
generally with a high level of proficiency. German is the second-most
spoken foreign language, with 47\% reporting a conversational level of
proficiency. Denmark had 25,900 native speakers of German in 2007
(mostly in the South Jutland area).

\section{Religion}\label{religion}

\begin{itemize}
\item
  \emph{In January 2018, 75.3\% of the population of Denmark were
  members of the Church of Denmark (Den Danske Folkekirke), the
  officially established church, which is Protestant in classification
  and Lutheran in orientation.}
\item
  \emph{Christianity is the dominant religion in Denmark.}
\end{itemize}

Christianity is the dominant religion in Denmark. In January 2018,
75.3\% of the population of Denmark were members of the Church of
Denmark (Den Danske Folkekirke), the officially established church,
which is Protestant in classification and Lutheran in orientation. This
is down 0.6\% compared to the year earlier and 1.6\% down compared to
two years earlier. Despite the high membership figures, only 3\% of the
population regularly attend Sunday services and only 19\% of Danes
consider religion to be an important part of their life.

The Constitution states that a member of the Royal Family must be a
member of the Church of Denmark, though the rest of the population is
free to adhere to other faiths. In 1682 the state granted limited
recognition to three religious groups dissenting from the Established
Church: Roman Catholicism, the Reformed Church and Judaism, although
conversion to these groups from the Church of Denmark remained illegal
initially. Until the 1970s, the state formally recognised "religious
societies" by royal decree. Today, religious groups do not need official
government recognition, they can be granted the right to perform
weddings and other ceremonies without this recognition. Denmark's
Muslims make up approximately 5.3\% of the population and form the
country's second largest religious community and largest minority
religion. The Danish Foreign Ministry estimates that other religious
groups comprise less than 1\% of the population individually and
approximately 2\% when taken all together.

According to a 2010 Eurobarometer Poll, 28\% of Danish citizens polled
responded that they "believe there is a God", 47\% responded that they
"believe there is some sort of spirit or life force" and 24\% responded
that they "do not believe there is any sort of spirit, God or life
force". Another poll, carried out in 2009, found that 25\% of Danes
believe Jesus is the son of God, and 18\% believe he is the saviour of
the world.

\includegraphics[width=4.09883in,height=5.50000in]{media/image7.jpg}\\
\emph{The oldest surviving Danish lecture plan dated 1537 from the
University of Copenhagen}

\section{Education}\label{education}

\begin{itemize}
\item
  \emph{Danish universities offer international students a range of
  opportunities for obtaining an internationally recognised
  qualification in Denmark.}
\item
  \emph{All educational programmes in Denmark are regulated by the
  Ministry of Education and administered by local municipalities.}
\item
  \emph{All university and college (tertiary) education in Denmark is
  free of charges; there are no tuition fees to enrol in courses.}
\end{itemize}

All educational programmes in Denmark are regulated by the Ministry of
Education and administered by local municipalities. Folkeskole covers
the entire period of compulsory education, encompassing primary and
lower secondary education. Most children attend folkeskole for 10 years,
from the ages of 6 to 16. There are no final examinations, but pupils
can choose to sit an exam when finishing ninth grade (14--15 years old).
The test is obligatory if further education is to be attended.
Alternatively pupils can attend an independent school (friskole), or a
private school (privatskole), such as Christian schools or Waldorf
schools.

Following graduation from compulsory education, there are several
continuing educational opportunities; the Gymnasium (STX) attaches
importance in teaching a mix of humanities and science, Higher Technical
Examination Programme (HTX) focuses on scientific subjects and the
Higher Commercial Examination Programme emphasises on subjects in
economics. Higher Preparatory Examination (HF) is similar to Gymnasium
(STX), but is one year shorter. For specific professions, there is
vocational education, training young people for work in specific trades
by a combination of teaching and apprenticeship.

The government records upper secondary school completion rates of 95\%
and tertiary enrollment and completion rates of 60\%. All university and
college (tertiary) education in Denmark is free of charges; there are no
tuition fees to enrol in courses. Students aged 18 or above may apply
for state educational support grants, known as Statens Uddannelsesstøtte
(SU), which provides fixed financial support, disbursed monthly. Danish
universities offer international students a range of opportunities for
obtaining an internationally recognised qualification in Denmark. Many
programmes may be taught in the English language, the academic lingua
franca, in bachelor's degrees, master's degrees, doctorates and student
exchange programmes.

\section{Health}\label{health}

\begin{itemize}
\item
  \emph{As of 2012{[}update{]}, Denmark spends 11.2\% of its GDP on
  health care; this is up from 9.8\% in 2007 (US\$3,512 per capita).}
\item
  \emph{This places Denmark above the OECD average and above the other
  Nordic countries.}
\item
  \emph{As of 2015{[}update{]}, Denmark has a life expectancy of 80.6
  years at birth (78.6 for men, 82.5 for women), up from 76.9 years in
  2000.}
\end{itemize}

As of 2015{[}update{]}, Denmark has a life expectancy of 80.6 years at
birth (78.6 for men, 82.5 for women), up from 76.9 years in 2000. This
ranks it 27th among 193 nations, behind the other Nordic countries. The
National Institute of Public Health of the University of Southern
Denmark has calculated 19 major risk factors among Danes that contribute
to a lowering of the life expectancy; this includes smoking, alcohol,
drug abuse and physical inactivity. Although the obesity rate is lower
than in North America and most other European countries, the large
number of Danes becoming overweight is an increasing problem and results
in an annual additional consumption in the health care system of DKK
1,625 million. In a 2012 study, Denmark had the highest cancer rate of
all countries listed by the World Cancer Research Fund International;
researchers suggest the reasons are better reporting, but also lifestyle
factors like heavy alcohol consumption, smoking and physical inactivity.

Denmark has a universal health care system, characterised by being
publicly financed through taxes and, for most of the services, run
directly by the regional authorities. One of the sources of income is a
national health care contribution (sundhedsbidrag) (2007--11:8\%;
'12:7\%; '13:6\%; '14:5\%; '15:4\%; '16:3\%; '17:2\%; '18:1\%; '19:0\%)
but it is being phased out and will be gone from January 2019, with the
income taxes in the lower brackets being raised gradually each year
instead. Another source comes from the municipalities that had their
income taxes raised by 3 percentage points from 1 January 2007, a
contribution confiscated from the former county tax to be used from 1
January 2007 for health purposes by the municipalities instead. This
means that most health care provision is free at the point of delivery
for all residents. Additionally, roughly two in five have complementary
private insurance to cover services not fully covered by the state, such
as physiotherapy. As of 2012{[}update{]}, Denmark spends 11.2\% of its
GDP on health care; this is up from 9.8\% in 2007 (US\$3,512 per
capita). This places Denmark above the OECD average and above the other
Nordic countries.

\section{Ghettos}\label{ghettos}

\begin{itemize}
\item
  \emph{Denmark is the only country to officially use the word 'ghetto'
  in the 21st century to denote certain residential areas.}
\item
  \emph{Since 2010, the Danish Ministry of Transport, Building and
  Housing publishes the ghettolisten (List of ghettos) which in 2018
  consists of 25 areas.}
\item
  \emph{In 2017, 8.7\% of Denmark's population consisted of non-Western
  immigrants or their descendants.}
\end{itemize}

Denmark is the only country to officially use the word 'ghetto' in the
21st century to denote certain residential areas. Since 2010, the Danish
Ministry of Transport, Building and Housing publishes the ghettolisten
(List of ghettos) which in 2018 consists of 25 areas. As a result, the
term is widely used in the media and common parlance. The legal
designation is applied to areas based on the residents' income levels,
employment status, education levels, criminal convictions and
'non-Western' ethnic background. In 2017, 8.7\% of Denmark's population
consisted of non-Western immigrants or their descendants. The population
proportion of 'ghetto residents' with non-Western background was 66.5\%.
In 2018, the government has proposed measures to solve the issue of
integration and to rid the country of 'parallel societies and ghettos by
2030'. The measures focus on physical redevelopment, control over who is
allowed to live in these areas, crime abatement and education. These
policies have been criticized for undercutting 'equality before law' and
for portraying immigrants, especially Muslim immigrants, in a bad light.
While some proposals like restricting 'ghetto children' to their homes
after 8 p.m. have been rejected for being too radical, most of the 22
proposals have been agreed upon by a parliamentary majority.

\section{Culture}\label{culture}

\begin{itemize}
\item
  \emph{A major feature of Danish culture is Jul (Danish Christmas).}
\item
  \emph{In 1969, Denmark was the first country to legalise pornography,
  and in 2012, Denmark replaced its "registered partnership" laws, which
  it had been the first country to introduce in 1989, with
  gender-neutral marriage, and allowed same-sex marriages to be
  performed in the Church of Denmark.}
\end{itemize}

Denmark shares strong cultural and historic ties with its Scandinavian
neighbours Sweden and Norway. It has historically been one of the most
socially progressive cultures in the world. In 1969, Denmark was the
first country to legalise pornography, and in 2012, Denmark replaced its
"registered partnership" laws, which it had been the first country to
introduce in 1989, with gender-neutral marriage, and allowed same-sex
marriages to be performed in the Church of Denmark. Modesty and social
equality are important parts of Danish culture.

The astronomical discoveries of Tycho Brahe (1546--1601), Ludwig A.
Colding's (1815--88) neglected articulation of the principle of
conservation of energy, and the contributions to atomic physics of Niels
Bohr (1885--1962) indicate the range of Danish scientific achievement.
The fairy tales of Hans Christian Andersen (1805--1875), the
philosophical essays of Søren Kierkegaard (1813--55), the short stories
of Karen Blixen (penname Isak Dinesen), (1885--1962), the plays of
Ludvig Holberg (1684--1754), and the dense, aphoristic poetry of Piet
Hein (1905--96), have earned international recognition, as have the
symphonies of Carl Nielsen (1865--1931). From the mid-1990s, Danish
films have attracted international attention, especially those
associated with Dogme 95 like those of Lars von Trier.

A major feature of Danish culture is Jul (Danish Christmas). The holiday
is celebrated throughout December, starting either at the beginning of
Advent or on 1 December with a variety of traditions, culminating with
the Christmas Eve meal.

There are five Danish heritage sites inscribed on the UNESCO World
Heritage list in Northern Europe: Christiansfeld, a Moravian Church
Settlement, the Jelling Mounds (Runic Stones and Church), Kronborg
Castle, Roskilde Cathedral, and The par force hunting landscape in North
Zealand.

\section{Media}\label{media}

\begin{itemize}
\item
  \emph{In 1834, the first liberal, factual newspaper appeared, and the
  1849 Constitution established lasting freedom of the press in
  Denmark.}
\item
  \emph{Danish cinema dates back to 1897 and since the 1980s has
  maintained a steady stream of productions due largely to funding by
  the state-supported Danish Film Institute.}
\item
  \emph{The Danish filmmaker Carl Th.}
\end{itemize}

Danish mass media date back to the 1540s, when handwritten fly sheets
reported on the news. In 1666, Anders Bording, the father of Danish
journalism, began a state paper. In 1834, the first liberal, factual
newspaper appeared, and the 1849 Constitution established lasting
freedom of the press in Denmark. Newspapers flourished in the second
half of the 19th century, usually tied to one or another political party
or trade union. Modernisation, bringing in new features and mechanical
techniques, appeared after 1900. The total circulation was 500,000 daily
in 1901, more than doubling to 1.2 million in 1925. The German
occupation during World War II brought informal censorship; some
offending newspaper buildings were simply blown up by the Nazis. During
the war, the underground produced 550 newspapers---small,
surreptitiously printed sheets that encouraged sabotage and resistance.

Danish cinema dates back to 1897 and since the 1980s has maintained a
steady stream of productions due largely to funding by the
state-supported Danish Film Institute. There have been three big
internationally important waves of Danish cinema: erotic melodrama of
the silent era; the increasingly explicit sex films of the 1960s and
1970s; and lastly, the Dogme 95 movement of the late 1990s, where
directors often used hand-held cameras to dynamic effect in a conscious
reaction against big-budget studios. Danish films have been noted for
their realism, religious and moral themes, sexual frankness and
technical innovation. The Danish filmmaker Carl Th. Dreyer (1889--1968)
is considered one of the greatest directors of early cinema.

Other Danish filmmakers of note include Erik Balling, the creator of the
popular Olsen-banden films; Gabriel Axel, an Oscar-winner for Babette's
Feast in 1987; and Bille August, the Oscar-, Palme d'Or- and Golden
Globe-winner for Pelle the Conqueror in 1988. In the modern era, notable
filmmakers in Denmark include Lars von Trier, who co-created the Dogme
movement, and multiple award-winners Susanne Bier and Nicolas Winding
Refn. Mads Mikkelsen is a world-renowned Danish actor, having starred in
films such as King Arthur, Casino Royale, the Danish film The Hunt, and
the American TV series Hannibal. Another renowned Danish actor Nikolaj
Coster-Waldau is internationally known for playing the role of Jaime
Lannister in the HBO series Game of Thrones.

Danish mass media and news programming are dominated by a few large
corporations. In printed media JP/Politikens Hus and Berlingske Media,
between them, control the largest newspapers Politiken, Berlingske
Tidende and Jyllands-Posten and major tabloids B.T. and Ekstra Bladet.
In television, publicly owned stations DR and TV 2 have large shares of
the viewers. DR in particular is famous for its high quality TV-series
often sold to foreign broadcasters and often with leading female
characters like internationally known actresses Sidse Babett Knudsen and
Sofie Gråbøl. In radio, DR has a near monopoly, currently broadcasting
on all four nationally available FM channels, competing only with local
stations.

\section{Music}\label{music}

\begin{itemize}
\item
  \emph{Denmark has been a part of the Eurovision Song Contest since
  1957.}
\item
  \emph{Denmark has won the contest three times, in 1963, 2000 and
  2013.}
\item
  \emph{Denmark's most famous classical composer is Carl Nielsen,
  especially remembered for his six symphonies and his Wind Quintet,
  while the Royal Danish Ballet specialises in the work of the Danish
  choreographer August Bournonville.}
\end{itemize}

Copenhagen and its multiple outlying islands have a wide range of folk
traditions. The Royal Danish Orchestra is among the world's oldest
orchestras. Denmark's most famous classical composer is Carl Nielsen,
especially remembered for his six symphonies and his Wind Quintet, while
the Royal Danish Ballet specialises in the work of the Danish
choreographer August Bournonville. Danes have distinguished themselves
as jazz musicians, and the Copenhagen Jazz Festival has acquired an
international reputation. The modern pop and rock scene has produced a
few names of note internationally, including Aqua, Alphabeat, D-A-D,
King Diamond, Kashmir, Lukas Graham, Mew, Michael Learns to Rock, MØ, Oh
Land, The Raveonettes and Volbeat, among others. Lars Ulrich, the
drummer of the band Metallica, has become the first Danish musician to
be inducted into the Rock and Roll Hall of Fame.

Roskilde Festival near Copenhagen is the largest music festival in
Northern Europe since 1971 and Denmark has many recurring music
festivals of all genres throughout, including Aarhus International Jazz
Festival, Skanderborg Festival, The Blue Festival in Aalborg, Esbjerg
International Chamber Music Festival and Skagen Festival among many
others.

Denmark has been a part of the Eurovision Song Contest since 1957.
Denmark has won the contest three times, in 1963, 2000 and 2013.

\section{Architecture and design}\label{architecture-and-design}

\begin{itemize}
\item
  \emph{Denmark's architecture became firmly established in the Middle
  Ages when first Romanesque, then Gothic churches and cathedrals sprang
  up throughout the country.}
\item
  \emph{Danish design is a term often used to describe a style of
  functionalistic design and architecture that was developed in the
  mid-20th century, originating in Denmark.}
\end{itemize}

Denmark's architecture became firmly established in the Middle Ages when
first Romanesque, then Gothic churches and cathedrals sprang up
throughout the country. From the 16th century, Dutch and Flemish
designers were brought to Denmark, initially to improve the country's
fortifications, but increasingly to build magnificent royal castles and
palaces in the Renaissance style.\\
During the 17th century, many impressive buildings were built in the
Baroque style, both in the capital and the provinces. Neoclassicism from
France was slowly adopted by native Danish architects who increasingly
participated in defining architectural style. A productive period of
Historicism ultimately merged into the 19th-century National Romantic
style.

The 20th century brought along new architectural styles; including
expressionism, best exemplified by the designs of architect Peder
Vilhelm Jensen-Klint, which relied heavily on Scandinavian brick Gothic
traditions; and Nordic Classicism, which enjoyed brief popularity in the
early decades of the century. It was in the 1960s that Danish architects
such as Arne Jacobsen entered the world scene with their highly
successful Functionalist architecture. This, in turn, has evolved into
more recent world-class masterpieces including Jørn Utzon's Sydney Opera
House and Johan Otto von Spreckelsen's Grande Arche de la Défense in
Paris, paving the way for a number of contemporary Danish designers such
as Bjarke Ingels to be rewarded for excellence both at home and abroad.

Danish design is a term often used to describe a style of
functionalistic design and architecture that was developed in the
mid-20th century, originating in Denmark. Danish design is typically
applied to industrial design, furniture and household objects, which
have won many international awards. The Royal Porcelain Factory is
famous for the quality of its ceramics and export products worldwide.
Danish design is also a well-known brand, often associated with
world-famous, 20th-century designers and architects such as Børge
Mogensen, Finn Juhl, Hans Wegner, Arne Jacobsen, Poul Henningsen and
Verner Panton. Other designers of note include Kristian Solmer Vedel
(1923--2003) in the area of industrial design, Jens Quistgaard
(1919--2008) for kitchen furniture and implements and Ole Wanscher
(1903--1985) who had a classical approach to furniture design.

\section{Literature and philosophy}\label{literature-and-philosophy}

\begin{itemize}
\item
  \emph{Another Danish philosopher of note is Grundtvig, whose
  philosophy gave rise to a new form of non-aggressive nationalism in
  Denmark, and who is also influential for his theological and
  historical works.}
\item
  \emph{Saxo Grammaticus, normally considered the first Danish writer,
  worked for bishop Absalon on a chronicle of Danish history (Gesta
  Danorum).}
\end{itemize}

The first known Danish literature is myths and folklore from the 10th
and 11th century. Saxo Grammaticus, normally considered the first Danish
writer, worked for bishop Absalon on a chronicle of Danish history
(Gesta Danorum). Very little is known of other Danish literature from
the Middle Ages. With the Age of Enlightenment came Ludvig Holberg whose
comedy plays are still being performed.

In the late 19th century, literature was seen as a way to influence
society. Known as the Modern Breakthrough, this movement was championed
by Georg Brandes, Henrik Pontoppidan (awarded the Nobel Prize in
Literature) and J.~P. Jacobsen. Romanticism influenced the renowned
writer and poet Hans Christian Andersen, known for his stories and fairy
tales, e.g. The Ugly Duckling, The Little Mermaid and The Snow Queen. In
recent history Johannes Vilhelm Jensen was also awarded the Nobel Prize
for Literature. Karen Blixen is famous for her novels and short stories.
Other Danish writers of importance are Herman Bang, Gustav Wied, William
Heinesen, Martin Andersen Nexø, Piet Hein, Hans Scherfig, Klaus
Rifbjerg, Dan Turèll, Tove Ditlevsen, Inger Christensen and Peter Høeg.

Danish philosophy has a long tradition as part of Western philosophy.
Perhaps the most influential Danish philosopher was Søren Kierkegaard,
the creator of Christian existentialism. Kierkegaard had a few Danish
followers, including Harald Høffding, who later in his life moved on to
join the movement of positivism. Among Kierkegaard's other followers
include Jean-Paul Sartre who was impressed with Kierkegaard's views on
the individual, and Rollo May, who helped create humanistic psychology.
Another Danish philosopher of note is Grundtvig, whose philosophy gave
rise to a new form of non-aggressive nationalism in Denmark, and who is
also influential for his theological and historical works.

\includegraphics[width=4.25333in,height=5.50000in]{media/image8.jpg}\\
\emph{Woman in front of a Mirror, (1841), by Christoffer Wilhelm
Eckersberg}

\section{Painting and photography}\label{painting-and-photography}

\begin{itemize}
\item
  \emph{Danish photography has developed from strong participation and
  interest in the very beginnings of the art of photography in 1839 to
  the success of a considerable number of Danes in the world of
  photography today.}
\item
  \emph{While Danish art was influenced over the centuries by trends in
  Germany and the Netherlands, the 15th- and 16th-century church
  frescos, which can be seen in many of the country's older churches,
  are of particular interest as they were painted in a style typical of
  native Danish painters.}
\end{itemize}

While Danish art was influenced over the centuries by trends in Germany
and the Netherlands, the 15th- and 16th-century church frescos, which
can be seen in many of the country's older churches, are of particular
interest as they were painted in a style typical of native Danish
painters.

The Danish Golden Age, which began in the first half of the 19th
century, was inspired by a new feeling of nationalism and romanticism,
typified in the later previous century by history painter Nicolai
Abildgaard. Christoffer Wilhelm Eckersberg was not only a productive
artist in his own right but taught at the Royal Danish Academy of Fine
Arts where his students included notable painters such as Wilhelm Bendz,
Christen Købke, Martinus Rørbye, Constantin Hansen, and Wilhelm
Marstrand.

In 1871, Holger Drachmann and Karl Madsen visited Skagen in the far
north of Jutland where they quickly built up one of Scandinavia's most
successful artists' colonies specialising in Naturalism and Realism
rather than in the traditional approach favoured by the Academy. Hosted
by Michael and his wife Anna, they were soon joined by P.S. Krøyer, Carl
Locher and Laurits Tuxen. All participated in painting the natural
surroundings and local people. Similar trends developed on Funen with
the Fynboerne who included Johannes Larsen, Fritz Syberg and Peter
Hansen, and on the island of Bornholm with the Bornholm school of
painters including Niels Lergaard, Kræsten Iversen and Oluf Høst.

Painting has continued to be a prominent form of artistic expression in
Danish culture, inspired by and also influencing major international
trends in this area. These include impressionism and the modernist
styles of expressionism, abstract painting and surrealism. While
international co-operation and activity has almost always been essential
to the Danish artistic community, influential art collectives with a
firm Danish base includes De Tretten (1909--1912), Linien (1930s and
1940s), COBRA (1948--51), Fluxus (1960s and 1970s), De Unge Vilde
(1980s) and more recently Superflex (founded in 1993). Most Danish
painters of modern times have also been very active with other forms of
artistic expressions, such as sculpting, ceramics, art installations,
activism, film and experimental architecture. Notable Danish painters
from modern times representing various art movements include Theodor
Philipsen (1840--1920, impressionism and naturalism), Anna Klindt
Sørensen (1899--1985, expressionism), Franciska Clausen (1899--1986,
Neue Sachlichkeit, cubism, surrealism and others), Henry Heerup
(1907--1993, naivism), Robert Jacobsen (1912--1993, abstract painting),
Carl Henning Pedersen (1913--2007, abstract painting), Asger Jorn
(1914--1973, Situationist, abstract painting), Bjørn Wiinblad
(1918--2006, art deco, orientalism), Per Kirkeby (b. 1938,
neo-expressionism, abstract painting), Per Arnoldi (b. 1941, pop art),
Michael Kvium (b. 1955, neo-surrealism) and Simone Aaberg Kærn (b. 1969,
superrealism).

Danish photography has developed from strong participation and interest
in the very beginnings of the art of photography in 1839 to the success
of a considerable number of Danes in the world of photography today.
Pioneers such as Mads Alstrup and Georg Emil Hansen paved the way for a
rapidly growing profession during the last half of the 19th century.
Today Danish photographers such as Astrid Kruse Jensen and Jacob Aue
Sobol are active both at home and abroad, participating in key
exhibitions around the world.

\section{Cuisine}\label{cuisine}

\begin{itemize}
\item
  \emph{Denmark is known for its Carlsberg and Tuborg beers and for its
  akvavit and bitters.}
\item
  \emph{The traditional cuisine of Denmark, like that of the other
  Nordic countries and of Northern Germany, consists mainly of meat,
  fish and potatoes.}
\item
  \emph{As a result of these developments, Denmark now have a
  considerable number of internationally acclaimed restaurants of which
  several have been awarded Michelin stars.}
\end{itemize}

The traditional cuisine of Denmark, like that of the other Nordic
countries and of Northern Germany, consists mainly of meat, fish and
potatoes. Danish dishes are highly seasonal, stemming from the country's
agricultural past, its geography, and its climate of long, cold winters.

The open sandwiches on rye bread, known as smørrebrød, which in their
basic form are the usual fare for lunch, can be considered a national
speciality when prepared and decorated with a variety of fine
ingredients. Hot meals traditionally consist of ground meats, such as
frikadeller (meat balls of veal and pork) and hakkebøf (minced beef
patties), or of more substantial meat and fish dishes such as flæskesteg
(roast pork with crackling) and kogt torsk (poached cod) with mustard
sauce and trimmings. Denmark is known for its Carlsberg and Tuborg beers
and for its akvavit and bitters.

Since around 1970, chefs and restaurants across Denmark have introduced
gourmet cooking, largely influenced by French cuisine. Also inspired by
continental practices, Danish chefs have recently developed a new
innovative cuisine and a series of gourmet dishes based on high-quality
local produce known as New Danish cuisine. As a result of these
developments, Denmark now have a considerable number of internationally
acclaimed restaurants of which several have been awarded Michelin stars.
This includes Geranium and Noma in Copenhagen.

\section{Sports}\label{sports}

\begin{itemize}
\item
  \emph{Denmark's numerous beaches and resorts are popular locations for
  fishing, canoeing, kayaking, and many other water-themed sports.}
\item
  \emph{In 2019, the Danish men's national handball team won their first
  World Championship title in the tournament that was co-hosted between
  Germany and Denmark.}
\item
  \emph{Sports are popular in Denmark, and its citizens participate in
  and watch a wide variety.}
\end{itemize}

Sports are popular in Denmark, and its citizens participate in and watch
a wide variety. The national sport is football, with over 320,000
players in more than 1600 clubs. Denmark qualified six times
consecutively for the European Championships between 1984 and 2004, and
were crowned European champions in 1992; other significant achievements
include winning the Confederations Cup in 1995 and reaching the
quarter-final of the 1998 World Cup. Notable Danish footballers include
Allan Simonsen, named the best player in Europe in 1977, Peter
Schmeichel, named the "World's Best Goalkeeper" in 1992 and 1993, and
Michael Laudrup, named the best Danish player of all time by the Danish
Football Association.

There is much focus on handball, too. The women's national team
celebrated great successes during the 1990s. On the men's side, Denmark
has won eight medals---two gold (in 2008 and 2012), three silver (in
2011, 2013 and 2014) and three bronze (in 2002, 2004 and 2006)---the
most that have been won by any team in European Handball Championship
history. In 2019, the Danish men's national handball team won their
first World Championship title in the tournament that was co-hosted
between Germany and Denmark.{[}citation needed{]}

In recent years, Denmark has made a mark as a strong cycling nation,
with Michael Rasmussen reaching King of the Mountains status in the Tour
de France in 2005 and 2006. Other popular sports include golf---which is
mostly popular among those in the older demographic; tennis---in which
Denmark is successful on a professional level; basketball---Denmark
joined the international governing body FIBA in 1951; rugby---the Danish
Rugby Union dates back to 1950; hockey--- often competing in the top
division in the Men's World Championships; rowing---Denmark specialise
in lightweight rowing and are particularly known for their lightweight
coxless four, having won six gold and two silver World Championship
medals and three gold and two bronze Olympic medals; and several indoor
sports---especially badminton, table tennis and gymnastics, in each of
which Denmark holds World Championships and Olympic medals. Denmark's
numerous beaches and resorts are popular locations for fishing,
canoeing, kayaking, and many other water-themed sports.

\section{See also}\label{see-also}

\begin{itemize}
\item
  \emph{Outline of Denmark}
\end{itemize}

Index of Denmark-related articles

Outline of Denmark

\section{Notes}\label{notes}

\section{References}\label{references}

\section{External links}\label{external-links}

\begin{itemize}
\item
  \emph{"Regions of Denmark".}
\item
  \emph{Wikimedia Atlas of Denmark}
\item
  \emph{Denmark at Curlie}
\item
  \emph{"Denmark".}
\item
  \emph{"Old Denmark in Cyberspace -- Information about Denmark -- the
  Danes".}
\item
  \emph{Denmark entry at Encyclopædia Britannica.}
\item
  \emph{Google news Denmark}
\end{itemize}

Denmark.dk

"Denmark". The World Factbook. Central Intelligence Agency.

Denmark entry at Encyclopædia Britannica.

A guide to Danish Culture at Denmark.net.

Denmark at UCB Libraries GovPubs.

Denmark at Curlie

Denmark profile from the BBC News.

Tourism portal at VisitDenmark.

Key Development Forecasts for Denmark from International Futures.

Government

Stm.dk -- official Danish government web site

um.dk -- official Ministry of Foreign Affairs of Denmark web site

Statistics Denmark (DST) -- Key figures from the Danish bureau of
statistics.

"Regions of Denmark". Statoids.

Maps

Wikimedia Atlas of Denmark

Geographic data related to Denmark at OpenStreetMap

Satellite image of Denmark at the NASA Earth Observatory.

Trade

World Bank Summary Trade Statistics Denmark

News and media

Google news Denmark

Google news Kingdom of Denmark - term used to include Greenland and the
Faroe Islands

History of Denmark: Primary Documents

‹See Tfd›(in Danish) Krak printable mapsearch

‹See Tfd›(in Swedish) ‹See Tfd›(in English) Ministry of the Environment
National Survey and Cadastre{[}permanent dead link{]}

"Old Denmark in Cyberspace -- Information about Denmark -- the Danes".
Archived from the original on 8 February 2006. Retrieved 24 July 2009.

Other

Vifanord.de -- library of scientific information on the Nordic and
Baltic countries.

Coordinates: 56°N 10°E / 56°N 10°E / 56; 10
