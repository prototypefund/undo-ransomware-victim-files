\textbf{From Wikipedia, the free encyclopedia}

https://en.wikipedia.org/wiki/Space\%20Shuttle\%20Columbia\%20disaster\\
Licensed under CC BY-SA 3.0:\\
https://en.wikipedia.org/wiki/Wikipedia:Text\_of\_Creative\_Commons\_Attribution-ShareAlike\_3.0\_Unported\_License

\section{Space Shuttle ''Columbia''
disaster}\label{space-shuttle-columbia-disaster}

\begin{itemize}
\item
  \emph{During the launch of STS-107, Columbia's 28th mission, a piece
  of foam insulation broke off from the Space Shuttle external tank and
  struck the left wing of the orbiter.}
\item
  \emph{On February 1, 2003, the Space Shuttle Columbia disintegrated
  upon atmospheric entry, killing all seven crew members.}
\item
  \emph{After the disaster, Space Shuttle flight operations were
  suspended for more than two years, as they had been after the
  Challenger disaster.}
\end{itemize}

On February 1, 2003, the Space Shuttle Columbia disintegrated upon
atmospheric entry, killing all seven crew members. The disaster was the
second fatal accident in the Space Shuttle program, after Challenger,
which broke apart and killed the seven-member crew 73 seconds after
liftoff in 1986.

During the launch of STS-107, Columbia's 28th mission, a piece of foam
insulation broke off from the Space Shuttle external tank and struck the
left wing of the orbiter. A few previous shuttle launches had seen
damage ranging from minor to nearly catastrophic from foam shedding, but
some engineers suspected that the damage to Columbia was more serious.
NASA managers limited the investigation, reasoning that the crew could
not have fixed the problem if it had been confirmed. When Columbia
re-entered the atmosphere of Earth, the damage allowed hot atmospheric
gases to penetrate the heat shield and destroy the internal wing
structure, which caused the spacecraft to become unstable and break
apart.

After the disaster, Space Shuttle flight operations were suspended for
more than two years, as they had been after the Challenger disaster.
Construction of the International Space Station (ISS) was put on hold;
the station relied entirely on the Russian Roscosmos State Space
Corporation for resupply for 29 months until Shuttle flights resumed
with STS-114 and 41 months for crew rotation until STS-121.

Several technical and organizational changes were made, including adding
a thorough on-orbit inspection to determine how well the shuttle's
thermal protection system had endured the ascent, and keeping a
designated rescue mission ready in case irreparable damage was found.
Except for one final mission to repair the Hubble Space Telescope,
subsequent shuttle missions were flown only to the ISS so that the crew
could use it as a haven in case damage to the orbiter prevented safe
reentry.

\includegraphics[width=5.50000in,height=4.40235in]{media/image1.jpg}\\
\emph{The crew of STS-107 in October 2001. From left to right: Brown,
Husband, Clark, Chawla, Anderson, McCool, Ramon}

\section{Crew}\label{crew}

\begin{itemize}
\item
  \emph{Commander: Rick D. Husband, a U.S. Air Force colonel and
  mechanical engineer, who piloted a previous shuttle during the first
  docking with the International Space Station (STS-96)}
\item
  \emph{Mission specialist: Kalpana Chawla, aerospace engineer who was
  on her second space mission}
\end{itemize}

Commander: Rick D. Husband, a U.S. Air Force colonel and mechanical
engineer, who piloted a previous shuttle during the first docking with
the International Space Station (STS-96)

Pilot: William C. McCool, a U.S. Navy commander

Payload commander: Michael P. Anderson, a U.S. Air Force lieutenant
colonel, physicist, and mission specialist who was in charge of the
science mission

Payload specialist: Ilan Ramon, a colonel in the Israeli Air Force and
the first Israeli astronaut

Mission specialist: Kalpana Chawla, aerospace engineer who was on her
second space mission

Mission specialist: David M. Brown, a U.S. Navy captain trained as an
aviator and flight surgeon. Brown worked on scientific experiments.

Mission specialist: Laurel Blair Salton Clark, a U.S. Navy captain and
flight surgeon. Clark worked on biological experiments.

\includegraphics[width=3.71067in,height=5.50000in]{media/image2.jpg}\\
\emph{Columbia lifting off on its final mission. The light-colored
triangle visible at the base of the strut near the nose of the orbiter
is the left bipod foam ramp.}

\includegraphics[width=5.50000in,height=3.10772in]{media/image3.jpg}\\
\emph{Close-up of the left bipod foam ramp that broke off and damaged
the shuttle wing}

\section{Debris strike during launch}\label{debris-strike-during-launch}

\begin{itemize}
\item
  \emph{The Columbia Accident Investigation Board determined that this
  delay had nothing to do with the catastrophic failure.}
\item
  \emph{Mission STS-107 was the 113th Space Shuttle launch.}
\item
  \emph{As demonstrated by ground experiments conducted by the Columbia
  Accident Investigation Board, this likely created a
  six-to-ten-inch-diameter (15 to 25~cm) hole, allowing hot gases to
  enter the wing when Columbia later re-entered the atmosphere.}
\end{itemize}

The shuttle's main fuel tank was covered in thermal insulation foam
intended to prevent ice from forming when the tank is full of liquid
hydrogen and oxygen. Such ice could damage the shuttle if shed during
lift-off.

Mission STS-107 was the 113th Space Shuttle launch. Planned to begin on
January 11, 2001, the mission was delayed 18~times and eventually
launched on January 16, 2003, following STS-113. The Columbia Accident
Investigation Board determined that this delay had nothing to do with
the catastrophic failure.

At 81.7~seconds after launch from Kennedy Space Center's LC-39-A, a
suitcase-sized piece of foam broke off from the external tank (ET),
striking Columbia's left wing reinforced carbon-carbon (RCC) panels. As
demonstrated by ground experiments conducted by the Columbia Accident
Investigation Board, this likely created a six-to-ten-inch-diameter (15
to 25~cm) hole, allowing hot gases to enter the wing when Columbia later
re-entered the atmosphere. At the time of the foam strike, the orbiter
was at an altitude of about 65,600 feet (20.0~km; 12.42~mi), traveling
at Mach~2.46. {[}citation needed{]}

The left bipod foam ramp is an approximately three-foot-long (1~m)
aerodynamic component made entirely of foam. The foam, not normally
considered to be a structural material, is required to bear some
aerodynamic loads. Because of these special requirements, the
casting-in-place and curing of the ramps may be performed only by a
senior technician. The bipod ramp (having left and right sides) was
originally designed to reduce aerodynamic stresses around the bipod
attachment points at the external tank, but it was proven unnecessary in
the wake of the accident and was removed from the external tank design
for tanks flown after STS-107 (another foam ramp along the liquid oxygen
line was also later removed from the tank design to eliminate it as a
foam debris source, after analysis and tests proved this change safe).

Bipod ramp insulation had been observed falling off, in whole or in
part, on four previous flights: STS-7 (1983), STS-32 (1990), STS-50
(1992), and most recently STS-112 (just two launches before STS-107).
All affected shuttle missions completed successfully. NASA management
came to refer to this phenomenon as "foam shedding". As with the O-ring
erosion problems that ultimately doomed the Space Shuttle Challenger,
NASA management became accustomed to these phenomena when no serious
consequences resulted from these earlier episodes. This phenomenon was
termed "normalization of deviance" by sociologist Diane Vaughan in her
book on the Challenger launch decision process.

As it happened, STS-112 had been the first flight with the "ET cam", a
video feed mounted on the ET for the purpose of giving greater insight
to the foam shedding problem. During that launch a chunk of foam broke
away from the ET bipod ramp and hit the SRB-ET attach ring near the
bottom of the left solid rocket booster (SRB) causing a dent four inches
wide and three inches deep in it. After STS-112, NASA leaders analyzed
the situation and decided to press ahead under the justification that
"{[}t{]}he ET is safe to fly with no new concerns (and no added risk)"
of further foam strikes.

Video taken during lift-off of STS-107 was routinely reviewed two hours
later and revealed nothing unusual. The following day, higher-resolution
film that had been processed overnight revealed the foam debris striking
the left wing, potentially damaging the thermal protection on the Space
Shuttle. At the time, the exact location where the foam struck the wing
could not be determined due to the low resolution of the tracking camera
footage.

Meanwhile, NASA's judgment about the risks was revisited. Linda Ham,
chair of the Mission Management Team (MMT), said, "Rationale was lousy
then and still is." Ham and Shuttle Program manager Ron Dittemore had
both been present at the October 31, 2002, meeting where the decision to
continue with launches was made.

Post-disaster analysis revealed that two previous shuttle launches
(STS-52 and -62) also had bipod ramp foam loss that went undetected. In
addition, protuberance air load (PAL) ramp foam had also shed pieces,
and there were also spot losses from large-area foams.

\section{Flight risk management}\label{flight-risk-management}

\begin{itemize}
\item
  \emph{Details of the DOD's unfulfilled participation with Columbia
  remain secret; retired NASA official Wayne Hale stated in 2012 that
  "activity regarding other national assets and agencies remains
  classified and I cannot comment on that aspect of the Columbia
  tragedy".}
\item
  \emph{Before the flight NASA believed that the RCC was very durable.}
\end{itemize}

In a risk-management scenario similar to the Challenger disaster, NASA
management failed to recognize the relevance of engineering concerns for
safety and suggestions for imaging to inspect possible damage, and
failed to respond to engineers' requests about the status of astronaut
inspection of the left wing. Engineers made three separate requests for
Department of Defense (DOD) imaging of the shuttle in orbit to determine
damage more precisely. While the images were not guaranteed to show the
damage, the capability existed for imaging of sufficient resolution to
provide meaningful examination. NASA management did not honor the
requests and in some cases intervened to stop the DOD from assisting.
The CAIB recommended subsequent shuttle flights be imaged while in orbit
using ground-based or space-based DOD assets. Details of the DOD's
unfulfilled participation with Columbia remain secret; retired NASA
official Wayne Hale stated in 2012 that "activity regarding other
national assets and agencies remains classified and I cannot comment on
that aspect of the Columbia tragedy".

Throughout the risk assessment process, senior NASA managers were
influenced by their belief that nothing could be done even if damage
were detected. This affected their stance on investigation urgency,
thoroughness and possible contingency actions. They decided to conduct a
parametric "what-if" scenario study more suited to determine risk
probabilities of future events, instead of inspecting and assessing the
actual damage. The investigation report in particular singled out NASA
manager Linda Ham for exhibiting this attitude. In 2013, Hale recalled
that Director of Mission Operations Jon C. Harpold shared with him
before Columbia's destruction a mindset which Hale himself later agreed
was widespread at the time, even among the astronauts themselves:

Much of the risk assessment hinged on damage predictions to the thermal
protection system. These fall into two categories: damage to the silica
tile on the wing lower surface, and damage to the reinforced
carbon-carbon (RCC) leading-edge panels. The TPS includes a third
category of components, thermal insulating blankets, but damage
predictions are not typically performed on them. Damage assessments on
the thermal blankets can be performed after an anomaly has been
observed, and this was done at least once after the return to flight
following Columbia's loss.

Before the flight NASA believed that the RCC was very durable. Charles
F. Bolden, who worked on tile-damage scenarios and repair methods early
in his astronaut career, said in 2004 that:

Damage-prediction software was used to evaluate possible tile and RCC
damage. The tool for predicting tile damage was known as "Crater",
described by several NASA representatives in press briefings as not
actually a software program but rather a statistical spreadsheet of
observed past flight events and effects. The "Crater" tool predicted
severe penetration of multiple tiles by the impact if it struck the TPS
tile area, but NASA engineers downplayed this. It had been shown that
the model overstated damage from small projectiles, and engineers
believed that the model would also overstate damage from larger Spray-On
Foam Insulation (SOFI) impacts. The program used to predict RCC damage
was based on small ice impacts the size of cigarette butts, not larger
SOFI impacts, as the ice impacts were the only recognized threats to RCC
panels up to that point. Under one of 15 predicted SOFI impact paths,
the software predicted an ice impact would completely penetrate the RCC
panel. Engineers downplayed this, too, believing that impacts of the
less dense SOFI material would result in less damage than ice impacts.
In an e-mail exchange, NASA managers questioned whether the density of
the SOFI could be used as justification for reducing predicted damage.
Despite engineering concerns about the energy imparted by the SOFI
material, NASA managers ultimately accepted the rationale to reduce
predicted damage of the RCC panels from possible complete penetration to
slight damage to the panel's thin coating.

Ultimately the NASA Mission Management Team felt there was insufficient
evidence to indicate that the strike was an unsafe situation, so they
declared the debris strike a "turnaround" issue (not of highest
importance) and denied the requests for the Department of Defense
images.

On January 23, flight director Steve Stich sent an e-mail to Columbia,
informing commander Husband and pilot McCool of the foam strike while
unequivocally dismissing any concerns about entry safety.

Edward Tufte, an expert in information design and presentation, remarked
on poor modes of communication during the assessment made on the ground,
before Columbia's reentry. NASA and Boeing favored Microsoft PowerPoint
for conveying information. PowerPoint uses multi-level bullet points and
orients towards single-page-of-information groupings. This is not ideal
for complex scientific and engineering reports and may have caused
recipients to draw incorrect conclusions. In particular, the slide
format may have emphasized optimistic options and glossed over the more
accurate pessimistic viewpoints.

\includegraphics[width=5.50000in,height=4.12500in]{media/image4.jpg}\\
\emph{Columbia at about 08:57. Debris is visible coming from the left
wing (bottom). The image was taken at Starfire Optical Range at Kirtland
Air Force Base.}

\includegraphics[width=5.50000in,height=4.64976in]{media/image5.jpg}\\
\emph{Columbia debris (in red, orange, and yellow) detected by National
Weather Service radar over Texas and Louisiana}

\section{Re-entry timeline}\label{re-entry-timeline}

\begin{itemize}
\item
  \emph{08:57:24 (EI+795): Columbia passed just north of Albuquerque.}
\item
  \emph{Before the orbiter broke up at 09:00:18, the Columbia cabin
  pressure was nominal and the crew was capable of conscious actions.}
\item
  \emph{08:55:32 (EI+683): Columbia crossed from Nevada into Utah.}
\item
  \emph{Columbia was scheduled to land at 09:16 EST.}
\item
  \emph{08:55:52 (EI+703): Columbia crossed from Utah into Arizona.}
\end{itemize}

Columbia was scheduled to land at 09:16 EST.

02:30 EST, February 1, 2003: The Entry Flight Control Team began duty in
the Mission Control Center.

The Flight Control Team had not been working on any issues or problems
related to the planned de-orbit and re-entry of Columbia. In particular,
the team had indicated no concerns about the debris that hit the left
wing during ascent, and treated the re-entry like any other. The team
worked through the de-orbit preparation checklist and re-entry checklist
procedures. Weather forecasters, with the help of pilots in the Shuttle
Training Aircraft, evaluated landing-site weather conditions at the
Kennedy Space Center.

08:00: Mission Control Center Entry Flight Director LeRoy Cain polled
the Mission Control room for a GO/NO-GO decision for the de-orbit burn.

All weather observations and forecasts were within guidelines set by the
flight rules, and all systems were normal.

08:10: The Capsule Communicator (CAPCOM) (and astronaut) Charles O.
Hobaugh told the crew that they were GO for de-orbit burn.

08:15:30 (EI-1719): Husband and McCool executed the de-orbit burn using
Columbia's two Orbital Maneuvering System engines.

The Orbiter was upside down and tail-first over the Indian Ocean at an
altitude of 175 miles (282~km) and speed of 17,500 miles per hour
(28,200~km/h) when the burn was executed. A 2-minute, 38-second de-orbit
burn during the 255th orbit slowed the Orbiter to begin its re-entry
into the atmosphere. The burn proceeded normally, putting the crew under
about one-tenth gravity. Husband then turned Columbia right side up,
facing forward with the nose pitched up.

08:44:09 (EI+000): Entry Interface (EI), arbitrarily defined as the
point at which the Orbiter entered the discernible atmosphere at 400,000
feet (120~km; 76~mi), occurred over the Pacific Ocean.

As Columbia descended, the heat of reentry caused wing leading-edge
temperatures to rise steadily, reaching an estimated 2,500~°F (1,370~°C)
during the next six minutes. (As former Space Shuttle Program Manager
Wayne Hale said in a press briefing, about 90\% of this heating is the
result of compression of the atmospheric gas caused by the orbiter's
supersonic flight, rather than the result of friction.)

08:48:39 (EI+270): A sensor on the left wing leading edge spar showed
strains higher than those seen on previous Columbia re-entries.

This was recorded only on the Modular Auxiliary Data System, which is
similar in concept to a flight data recorder, and was not sent to ground
controllers or shown to the crew.

08:49:32 (EI+323): Columbia executed a planned roll to the right. Speed:
Mach 24.5.

Columbia began a banking turn to manage lift and therefore limit the
Orbiter's rate of descent and heating.

08:50:53 (EI+404): Columbia entered a 10-minute period of peak heating,
during which the thermal stresses were at their maximum. Speed: Mach
24.1; altitude: 243,000 feet (74~km; 46.0~mi).

08:52:00 (EI+471): Columbia was about 300 miles (480~km) west of the
California coastline.

The wing leading-edge temperatures usually reached 2,650~°F (1,450~°C)
at this point.

08:53:26 (EI+557): Columbia crossed the California coast west of
Sacramento. Speed: Mach 23; altitude: 231,600 feet (70.6~km; 43.86~mi).

The Orbiter's wing leading edge typically reached more than 2,800~°F
(1,540~°C) at this point.

08:53:46 (EI+577): Various people on the ground saw signs of debris
being shed. Speed: Mach 22.8; altitude: 230,200 feet (70.2~km;
43.60~mi).

The hot air surrounding the Orbiter suddenly brightened, causing a
streak in the Orbiter's luminescent trail that was quite noticeable in
the pre-dawn skies over the West Coast. Observers witnessed four similar
events during the following 23~seconds. Dialogue on some of the amateur
footage indicates the observers were aware of the abnormality of what
they were filming.

08:54:24 (EI+615): The MMACS officer, Jeff Kling, informed the Flight
Director that "four hydraulic fluid temperature sensors in the left wing
had stopped reporting." In Mission Control, re-entry had been proceeding
normally up to this point.

08:54:25 (EI+616): Columbia crossed from California into Nevada
airspace. Speed: Mach 22.5; altitude: 227,400 feet (69.3~km; 43.07~mi).

Witnesses observed a bright flash at this point and 18 similar events in
the next four minutes.

08:55:00 (EI+651): Nearly 11 minutes after Columbia re-entered the
atmosphere, wing leading-edge temperatures normally reached nearly
3,000~°F (1,650~°C).

08:55:32 (EI+683): Columbia crossed from Nevada into Utah. Speed: Mach
21.8; altitude: 223,400 feet (68.1~km; 42.31~mi).

08:55:52 (EI+703): Columbia crossed from Utah into Arizona.

08:56:30 (EI+741): Columbia began a roll reversal, turning from right to
left over Arizona.

08:56:45 (EI+756): Columbia crossed from Arizona to New Mexico. Speed:
Mach 20.9; altitude: 219,000 feet (67~km; 41.5~mi).

08:57:24 (EI+795): Columbia passed just north of Albuquerque.

08:58:00 (EI+831): At this point, wing leading-edge temperatures
typically decreased to 2,880~°F (1,580~°C).

08:58:20 (EI+851): Columbia crossed from New Mexico into Texas. Speed:
Mach 19.5; altitude: 209,800 feet (63.9~km; 39.73~mi).

At about this time, the Orbiter shed a Thermal Protection System tile,
the most westerly piece of debris that has been recovered. Searchers
found the tile in a field in Littlefield, Texas, just northwest of
Lubbock.

08:59:15 (EI+906): MMACS informed the Flight Director that "pressure
readings on both left main landing-gear tires were indicating "off-scale
low"."

"Off-scale low" is a reading that falls below the minimum capability of
the sensor, and it usually indicates that the sensor has stopped
functioning, due to internal or external factors, not that the quantity
it measures is actually below the sensor's minimum response value.

08:59:32 (EI+923): A broken response from the mission commander was
recorded: "Roger, uh, bu~-- {[}cut off in mid-word{]}~..." It was the
last communication from the crew and the last telemetry signal received
in Mission Control. The Flight Director then instructed the Capsule
Communicator (CAPCOM) to let the crew know that Mission Control saw the
messages and was evaluating the indications, and added that the Flight
Control Team did not understand the crew's last transmission.

08:59:37 (EI+928): Hydraulic pressure, which is required to move the
flight control surfaces, was lost at about 08:59:37. At that time, the
Master Alarm would have sounded for the loss of hydraulics, used to move
flight control surfaces. The shuttle would have started to roll and yaw
uncontrollably, and the crew would have become aware of the serious
problem.

09:00:18 (EI+969): Videos and eyewitness reports by observers on the
ground in and near Dallas indicated that the Orbiter had disintegrated
overhead, continued to break up into smaller pieces, and left multiple
ion trails, as it continued eastward. In Mission Control, while the loss
of signal was a cause for concern, there was no sign of any serious
problem. Before the orbiter broke up at 09:00:18, the Columbia cabin
pressure was nominal and the crew was capable of conscious actions.
Although the crew module remained mostly intact through the breakup, it
was damaged enough that it lost pressure at a rate fast enough to
incapacitate the crew within seconds, and was completely depressurized
no later than 09:00:53.

09:00:57 (EI+1008): The crew module, intact to this point, was seen
breaking into small subcomponents. It disappeared from view at 09:01:10.
The crew members, if not already dead, were killed no later than this
point.

09:05: Residents of north central Texas, particularly near Tyler,
reported a loud boom, a small concussion wave, smoke trails and debris
in the clear skies above the counties east of Dallas.

09:12:39 (EI+1710): After hearing of reports of the orbiter being seen
to break apart, Entry Flight Director LeRoy Cain declared a contingency
(events leading to loss of the vehicle) and alerted search-and-rescue
teams in the debris area. He called on the Ground Controller to "lock
the doors", meaning no one would be permitted to enter or leave until
everything needed for investigation of the accident had been secured.
Two minutes later, Mission Control put contingency procedures into
effect.

\section{Crew survivability aspects}\label{crew-survivability-aspects}

\begin{itemize}
\item
  \emph{The crew were exposed to five lethal events:88 in the following
  order:}
\item
  \emph{In 2008, NASA released a detailed report on survivability
  aspects of the Columbia reentry.}
\end{itemize}

In 2008, NASA released a detailed report on survivability aspects of the
Columbia reentry. In 2014, NASA released a further report detailing the
aeromedical aspects of the disaster. The crew were exposed to five
lethal events:88 in the following order:

\section{Depressurization}\label{depressurization}

\begin{itemize}
\item
  \emph{After the initial loss of control, Columbia's "cabin pressure
  was nominal and the crew was capable of conscious actions".}
\item
  \emph{:2-88 During this period the crew attempted to regain control of
  the shuttle.}
\end{itemize}

After the initial loss of control, Columbia's "cabin pressure was
nominal and the crew was capable of conscious actions".:2-88 During this
period the crew attempted to regain control of the shuttle.:3-70 As
Columbia spun out of control, aerodynamic forces caused the orbiter to
yaw to the right, exposing its underside to extreme aerodynamic forces
and causing the orbiter to break up. Depressurization began when the
shuttle forebody separated from the midbody 41 seconds after loss of
control. The crew module pressure vessel was penetrated when it collided
with the fuselage, and the "depressurization rate was high enough to
incapacitate the crewmembers within seconds so that they were unable to
perform actions such as lowering their visors." The crew lost
consciousness, suffering massive pulmonary barotrauma, ebullism and
cessation of respiration.:89,101-103

\section{Off-nominal dynamic G
environment}\label{off-nominal-dynamic-g-environment}

\begin{itemize}
\item
  \emph{The shuttle's separated nose section rotated unsteadily about
  all three axes.}
\item
  \emph{The crew (now unconscious or deceased) were unable to brace
  against this motion, and were also harmed by aspects of their
  protective equipment:}
\item
  \emph{Non-conformal helmets: unlike a racing helmet, the ACES suit
  helmets allowed the crew's heads to move inside the helmet, causing
  blunt force trauma during collisions.}
\end{itemize}

The shuttle's separated nose section rotated unsteadily about all three
axes. The crew (now unconscious or deceased) were unable to brace
against this motion, and were also harmed by aspects of their protective
equipment:

Lack of upper-body and arm/leg restraints: the crew's torsos were free
to move because "the strap velocity was lower than the locking threshold
velocity of the inertia reel system" and because the seat restraints did
not prevent lateral movement. Fractures consistent with flailing arms
and legs were also observed.:91,105-106

Non-conformal helmets: unlike a racing helmet, the ACES suit helmets
allowed the crew's heads to move inside the helmet, causing blunt force
trauma during collisions. The helmet neck ring acted as a fulcrum for
cervical vertebrae fractures as the skull whipped backwards, as well as
inflicting jaw injuries when wind blasted the helmet off.:105,104,111

\section{Separation of the crew members from the crew module and the
seats}\label{separation-of-the-crew-members-from-the-crew-module-and-the-seats}

\begin{itemize}
\item
  \emph{As the crew module disintegrated, the crew received lethal
  trauma from their seat restraints and were exposed to the hostile
  aerodynamic and thermal environment of re-entry, as well as molten
  Columbia debris.}
\end{itemize}

As the crew module disintegrated, the crew received lethal trauma from
their seat restraints and were exposed to the hostile aerodynamic and
thermal environment of re-entry, as well as molten Columbia
debris.:92,108-110

\section{Exposure to high-speed / high-altitude
environment}\label{exposure-to-high-speed-high-altitude-environment}

\begin{itemize}
\item
  \emph{":93 NASA stated that despite not being certified for those
  conditions, the ACES suit "may potentially be capable of protecting
  the crew" above 100,000 feet, :1-29 although in Columbia's case they
  had already been destroyed by the cabin's thermal environment during
  breakup.}
\end{itemize}

After separation from the crew module, the deceased crewmembers entered
an environment with "lack of oxygen, low atmospheric pressure, high
thermal loads as a result of deceleration from high Mach numbers, shock
wave interactions, aerodynamic accelerations, and exposure to cold
temperatures.":93 NASA stated that despite not being certified for those
conditions, the ACES suit "may potentially be capable of protecting the
crew" above 100,000 feet, :1-29 although in Columbia's case they had
already been destroyed by the cabin's thermal environment during
breakup.

\section{Ground impact}\label{ground-impact}

\begin{itemize}
\item
  \emph{Although some of the crew were not wearing gloves or helmets
  during reentry and some were not properly restrained in their seats,
  doing these things would have added nothing to their survival chances
  other than perhaps keeping them alive and conscious another 30 or so
  seconds.}
\item
  \emph{All evidence indicated that crew error was in no way responsible
  for the disintegration of the orbiter, and they had acted correctly
  and according to procedure at the first indication of trouble.}
\end{itemize}

The crewmembers had lethal-level injuries sustained from ground
impact.:94 The official NASA report omitted some of the more graphic
details on the recovery of the remains; witnesses reported finds such as
a human heart and parts of femur bones.

All evidence indicated that crew error was in no way responsible for the
disintegration of the orbiter, and they had acted correctly and
according to procedure at the first indication of trouble. Although some
of the crew were not wearing gloves or helmets during reentry and some
were not properly restrained in their seats, doing these things would
have added nothing to their survival chances other than perhaps keeping
them alive and conscious another 30 or so seconds.

\section{Presidential response}\label{presidential-response}

\begin{itemize}
\item
  \emph{The Columbia is lost; there are no survivors".}
\end{itemize}

At 14:04 EST (19:04 UTC), President George W. Bush said, "This day has
brought terrible news and great sadness to our country~... The Columbia
is lost; there are no survivors". Despite the disaster, Bush said, "The
cause in which they died will continue~... Our journey into space will
go on". Bush later declared East Texas a federal disaster area, allowing
federal agencies to help with the recovery effort.

\includegraphics[width=5.50000in,height=3.58696in]{media/image6.jpg}\\
\emph{A grid on the floor is used to organize recovered debris}

\includegraphics[width=5.50000in,height=3.66667in]{media/image7.jpg}\\
\emph{Recovered power-head of one of Columbia's main engines}

\includegraphics[width=5.50000in,height=3.74497in]{media/image8.jpg}\\
\emph{The glow of reentry as seen out of the front windows}

\section{Recovery of debris}\label{recovery-of-debris}

\begin{itemize}
\item
  \emph{All recovered non-human Columbia debris is stored in unused
  office space at the Vehicle Assembly Building, except for parts of the
  crew compartment, which are kept separate.}
\item
  \emph{Five of the seven crew of Columbia were found within the first
  three days after the shuttle's breakup; the last two were not found
  for another 10 days after that.}
\end{itemize}

More than 2,000 debris fields were found in sparsely populated areas
from Nacogdoches in East Texas, where a large amount of debris fell, to
western Louisiana and the southwestern counties of Arkansas. A large
amount of debris was recovered between Tyler, Texas and Palestine,
Texas. One debris field has been mapped along a path stretching from
south of Fort Worth to Hemphill, Texas, as well as into parts of
Louisiana. Places that had debris included Stephen F. Austin State
University in Nacogdoches and several casinos in Shreveport, Louisiana.
Along with pieces of the shuttle and bits of equipment, searchers also
found human body parts, including arms, feet, a torso, and a heart.
These recoveries occurred along a line south of Hemphill, Texas and west
of the Toledo Bend Reservoir. Much of the terrain being searched for the
crew was densely forested and often difficult to traverse. Five of the
seven crew of Columbia were found within the first three days after the
shuttle's breakup; the last two were not found for another 10 days after
that.

In the months after the disaster, the largest-ever organized ground
search took place.\\
Thousands of volunteers descended upon Texas to participate in the
effort to gather the Shuttle's remains. According to Mike Ciannilli,
Project Manager of the Columbia Research and Preservation Office,
"{[}these people{]} put their life on hold to help out the nation's
space program," showing "what space means to people."\\
NASA issued warnings to the public that any debris could contain
hazardous chemicals, that it should be left untouched, its location
reported to local emergency services or government authorities, and that
anyone in unauthorized possession of debris would be prosecuted. Because
of the widespread area, volunteer amateur radio operators accompanied
the search teams to provide communications support.

A group of small (one-millimeter or 0.039-inch) adult Caenorhabditis
elegans worms, living in petri dishes enclosed in aluminum canisters,
survived reentry and impact with the ground and were recovered weeks
after the disaster. The culture was found to be alive on April 28, 2003.
The worms were part of a biological research in canisters experiment
designed to study the effect of weightlessness on physiology; the
experiment was conducted by Cassie Conley, NASA's planetary protection
officer.

Debris Search Pilot Jules F. Mier Jr. and Debris Search Aviation
Specialist Charles Krenek died in a helicopter crash that injured three
others during the search.

Some Texas residents recovered some of the debris, ignoring the
warnings, and attempted to sell it on the online auction site eBay,
starting at \$10,000. The auction was quickly removed, but prices for
Columbia merchandise such as programs, photographs and patches, went up
dramatically following the disaster, creating a surge of
Columbia-related listings. A three-day amnesty offered for "looted"
shuttle debris brought in hundreds of illegally recovered pieces. About
40,000 recovered pieces of debris have never been identified. The
largest pieces recovered include the front landing gear and a window
frame.

On May 9, 2008, it was reported that data from a disk drive on board
Columbia had survived the shuttle accident, and while part of the 340~MB
drive was damaged, 99\% of the data was recovered. The drive was used to
store data from an experiment on the properties of shear thinning.

On July 29, 2011, Nacogdoches authorities told NASA that a
four-foot-diameter (1.2~m) piece of debris had been found in a lake.
NASA identified the piece as a power reactant storage and distribution
tank.

All recovered non-human Columbia debris is stored in unused office space
at the Vehicle Assembly Building, except for parts of the crew
compartment, which are kept separate.

\section{Crew cabin video}\label{crew-cabin-video}

\begin{itemize}
\item
  \emph{The 13-minute recording shows the flight crew astronauts
  conducting routine re-entry procedures and joking with each other.}
\item
  \emph{The recording, which on normal flights would have continued
  through landing, ends about four minutes before the shuttle began to
  disintegrate and 11 minutes before Mission Control lost the signal
  from the orbiter.}
\end{itemize}

Among the recovered items was a videotape recording made by the
astronauts during the start of re-entry. The 13-minute recording shows
the flight crew astronauts conducting routine re-entry procedures and
joking with each other. None gives any indication of a problem. In the
video, the flight-deck crew puts on their gloves and passes the video
camera around to record plasma and flames visible outside the windows of
the orbiter (a normal occurrence). The recording, which on normal
flights would have continued through landing, ends about four minutes
before the shuttle began to disintegrate and 11 minutes before Mission
Control lost the signal from the orbiter.

\section{Investigation}\label{investigation}

\includegraphics[width=5.50000in,height=3.67647in]{media/image9.jpg}\\
\emph{Mock-up of a Space Shuttle leading edge made with an RCC-panel
taken from Atlantis. Simulation of known and possible conditions of the
foam impact on Columbia's final launch showed brittle fracture of RCC.}

\section{Initial investigation}\label{initial-investigation}

\begin{itemize}
\item
  \emph{After the loss of Columbia, NASA concluded that mistakes during
  installation were the likely cause of foam loss, and retrained
  employees at Michoud Assembly Facility in Louisiana to apply foam
  without defects.}
\item
  \emph{NASA Space Shuttle Program Manager Ron Dittemore reported that
  "The first indication was loss of temperature sensors and hydraulic
  systems on the left wing.}
\end{itemize}

NASA Space Shuttle Program Manager Ron Dittemore reported that "The
first indication was loss of temperature sensors and hydraulic systems
on the left wing. They were followed seconds and minutes later by
several other problems, including loss of tire pressure indications on
the left main gear and then indications of excessive structural
heating". Analysis of 31~seconds of telemetry data which had initially
been filtered out because of data corruption within it showed the
shuttle fighting to maintain its orientation, eventually using maximum
thrust from its Reaction Control System jets.

The investigation focused on the foam strike from the very beginning.
Incidents of debris strikes from ice and foam causing damage during
take-off were already well known, and had damaged orbiters, most
noticeably during STS-45, STS-27, and STS-87. After the loss of
Columbia, NASA concluded that mistakes during installation were the
likely cause of foam loss, and retrained employees at Michoud Assembly
Facility in Louisiana to apply foam without defects. Tile damage had
also been traced to ablating insulating material from the cryogenic fuel
tank in the past.

\section{Columbia Accident Investigation
Board}\label{columbia-accident-investigation-board}

\begin{itemize}
\item
  \emph{Columbia's flight data recorder was found near Hemphill, Texas,
  on March 19, 2003.}
\item
  \emph{The tests demonstrated that a foam impact of the type Columbia
  sustained could seriously breach the thermal protection system on the
  wing leading edge.}
\item
  \emph{After the initial Shuttle test-flights were completed, the
  recorder was never removed from Columbia, and it was still functioning
  on the crashed flight.}
\end{itemize}

Following protocols established after the loss of Challenger, an
independent investigating board was created immediately after the
accident. The Columbia Accident Investigation Board, or CAIB, was
chaired by retired U.S. Navy Admiral Harold W. Gehman, Jr., and
consisted of expert military and civilian analysts who investigated the
accident in detail.

Columbia's flight data recorder was found near Hemphill, Texas, on March
19, 2003. Unlike commercial jet aircraft, the space shuttles did not
have flight data recorders intended for after-crash analysis. Instead,
the vehicle data were transmitted in real time to the ground via
telemetry. Since Columbia was the first shuttle, it had a special flight
data OEX (Orbiter EXperiments) recorder, designed to help engineers
better understand vehicle performance during the first test flights.
After the initial Shuttle test-flights were completed, the recorder was
never removed from Columbia, and it was still functioning on the crashed
flight. It recorded many hundreds of parameters, and contained very
extensive logs of structural and other data, which allowed the CAIB to
reconstruct many of the events during the process leading to breakup.
Investigators could often use the loss of signals from sensors on the
wing to track how the damage progressed. This was correlated with
forensic debris analysis conducted at Lehigh University and other tests
to obtain a final conclusion about the probable course of events.

Beginning on May 30, 2003, foam impact tests were performed by Southwest
Research Institute. They used a compressed air gun to fire a foam block
of similar size and mass to that which struck Columbia, at the same
estimated speed. To represent the leading edge of Columbia's left wing,
RCC panels from NASA stock, along with the actual leading-edge panels
from Enterprise , which were fiberglass, were mounted to a simulating
structural metal frame. At the beginning of testing, the likely impact
site was estimated to be between RCC panel 6 and 9, inclusive. Over many
days, dozens of the foam blocks were shot at the wing leading edge model
at various angles. These produced only cracks or surface damage to the
RCC panels.

During June, further analysis of information from Columbia's flight data
recorder narrowed the probable impact site to one single panel: RCC wing
panel 8. On July 7, in a final round of testing, a block fired at the
side of an RCC panel 8 created a hole 16 by 16.7 inches (41 by 42~cm) in
that protective RCC panel. The tests demonstrated that a foam impact of
the type Columbia sustained could seriously breach the thermal
protection system on the wing leading edge.

\section{Conclusions}\label{conclusions}

\begin{itemize}
\item
  \emph{NASA had commissioned this group, "to perform a comprehensive
  analysis of the accident, focusing on factors and events affecting
  crew survival, and to develop recommendations for improving crew
  survival for all future human space flight vehicles."}
\item
  \emph{On December 30, 2008, NASA released a further report, entitled
  Columbia Crew Survival Investigation Report, produced by a second
  commission, the Spacecraft Crew Survival Integrated Investigation Team
  (SCSIIT).}
\end{itemize}

On August 26, 2003, the CAIB issued its report on the accident. The
report confirmed the immediate cause of the accident was a breach in the
leading edge of the left wing, caused by insulating foam shed during
launch. The report also delved deeply into the underlying organizational
and cultural issues that led to the accident. The report was highly
critical of NASA's decision-making and risk-assessment processes. It
concluded the organizational structure and processes were sufficiently
flawed and that a compromise of safety was expected no matter who was in
the key decision-making positions. An example was the position of
Shuttle Program Manager, where one individual was responsible for
achieving safe, timely launches and acceptable costs, which are often
conflicting goals. The CAIB report found that NASA had accepted
deviations from design criteria as normal when they happened on several
flights and did not lead to mission-compromising consequences. One of
those was the conflict between a design specification stating that the
thermal protection system was not designed to withstand significant
impacts and the common occurrence of impact damage to it during flight.
The board made recommendations for significant changes in processes and
organizational culture.

On December 30, 2008, NASA released a further report, entitled Columbia
Crew Survival Investigation Report, produced by a second commission, the
Spacecraft Crew Survival Integrated Investigation Team (SCSIIT). NASA
had commissioned this group, "to perform a comprehensive analysis of the
accident, focusing on factors and events affecting crew survival, and to
develop recommendations for improving crew survival for all future human
space flight vehicles." The report concluded that: "The Columbia
depressurization event occurred so rapidly that the crew members were
incapacitated within seconds, before they could configure the suit for
full protection from loss of cabin pressure. Although circulatory
systems functioned for a brief time, the effects of the depressurization
were severe enough that the crew could not have regained consciousness.
This event was lethal to the crew."

The report also concluded:

The crew did not have time to prepare themselves. Some crew members were
not wearing their safety gloves, and one crew member was not wearing a
helmet. New policies gave the crew more time to prepare for descent.

The crew's safety harnesses malfunctioned during the violent descent.
The harnesses on the three remaining shuttles were upgraded after the
accident.

The key recommendations of the report included that future spacecraft
crew survival systems should not rely on manual activation to protect
the crew.

\section{Other contributing factors}\label{other-contributing-factors}

\begin{itemize}
\item
  \emph{Unintended consequences of decisions contributed to the failure:
  the original tank white paint was removed to save 600~lb (270~kg),
  exposing the rust-orange-colored foam; the tank foam chemical
  composition was altered to meet Environmental Protection Agency
  requirements, weakening it; upgrades to the leading edge proposed in
  the early 1990s were not funded because NASA was working on the
  later-cancelled VentureStar single-stage-to-orbit shuttle
  replacement.}
\end{itemize}

Unintended consequences of decisions contributed to the failure: the
original tank white paint was removed to save 600~lb (270~kg), exposing
the rust-orange-colored foam; the tank foam chemical composition was
altered to meet Environmental Protection Agency requirements, weakening
it; upgrades to the leading edge proposed in the early 1990s were not
funded because NASA was working on the later-cancelled VentureStar
single-stage-to-orbit shuttle replacement.

\section{Possible emergency
procedures}\label{possible-emergency-procedures}

\begin{itemize}
\item
  \emph{As mission control could deorbit an empty shuttle, but could not
  control the orbiter's reentry and landing, it would likely have sent
  Columbia into the Pacific Ocean; NASA later developed the Remote
  Control Orbiter system to permit mission control to land a shuttle.}
\item
  \emph{The CAIB determined that this would have allowed Columbia to
  stay in orbit until flight day 30 (February 15).}
\end{itemize}

One question of special importance was whether NASA could have saved the
astronauts had they known of the danger. This would have to involve
either rescue or repair~-- docking at the International Space Station
for use as a haven while awaiting rescue (or to use the Soyuz to
systematically ferry the crew to safety) would have been impossible due
to the different orbital inclination of the vehicles.

The CAIB determined that a rescue mission, though risky, might have been
possible provided NASA management had taken action soon enough.
Normally, a rescue mission is not possible, due to the time required to
prepare a shuttle for launch, and the limited consumables (power, water,
air) of an orbiting shuttle. Atlantis was well along in processing for a
planned March 1 launch on STS-114, and Columbia carried an unusually
large quantity of consumables due to an Extended Duration Orbiter
package. The CAIB determined that this would have allowed Columbia to
stay in orbit until flight day 30 (February 15). NASA investigators
determined that Atlantis processing could have been expedited with no
skipped safety checks for a February 10 launch. Hence, if nothing went
wrong, there was a five-day overlap for a possible rescue. As mission
control could deorbit an empty shuttle, but could not control the
orbiter's reentry and landing, it would likely have sent Columbia into
the Pacific Ocean; NASA later developed the Remote Control Orbiter
system to permit mission control to land a shuttle.

NASA investigators determined that on-orbit repair by the shuttle
astronauts was possible but overall considered "high risk", primarily
due to the uncertain resiliency of the repair using available materials
and the anticipated high risk of doing additional damage to the Orbiter.
Columbia did not carry the Canadarm, or Remote Manipulator System, which
would normally be used for camera inspection or transporting a
spacewalking astronaut to the wing. Therefore, an unusual emergency
extra-vehicular activity (EVA) would have been required. While there was
no astronaut EVA training for maneuvering to the wing, astronauts are
always prepared for a similarly difficult emergency EVA to close the
external tank umbilical doors located on the orbiter underside, which is
necessary for reentry. Similar methods could have reached the shuttle
left wing for inspection or repair.

For the repair, the CAIB determined that the astronauts would have to
use tools and small pieces of titanium, or other metal, scavenged from
the crew cabin. These metals would help protect the wing structure and
would be held in place during re-entry by a water-filled bag that had
turned into ice in the cold of space. The ice and metal would help
restore wing leading edge geometry, preventing a turbulent airflow over
the wing and therefore keeping heating and burn-through levels low
enough for the crew to survive re-entry and bail out before landing. The
CAIB could not determine whether a patched-up left wing would have
survived even a modified re-entry, and concluded that the rescue option
would have had a considerably higher chance of bringing Columbia's crew
back alive.

\includegraphics[width=5.50000in,height=3.64078in]{media/image10.jpg}\\
\emph{A makeshift memorial at the main entrance to the Lyndon B. Johnson
Space Center in Houston, Texas}

\section{Memorials}\label{memorials}

\begin{itemize}
\item
  \emph{NASA named a supercomputer "Columbia" in the crew's honor in
  2004.}
\item
  \emph{NASA named several places in honor of Columbia and the crew.}
\item
  \emph{The Columbia Memorial Space Center is a museum built in honor of
  the Columbia in Downey, California.}
\item
  \emph{A starship on Star Trek: Enterprise was named NX-02 Columbia in
  honor of the Columbia.}
\end{itemize}

On February 4, 2003, President George W. Bush and his wife Laura led a
memorial service for the astronauts' families at the Lyndon B. Johnson
Space Center. Two days later, Vice President Dick Cheney and his wife
Lynne led a similar service at Washington National Cathedral. Patti
LaBelle sang "Way Up There" as part of the service.

On February 2, 2003, and throughout March, April, and May 2003, large
memorial Catholic Brazilian masses and Roman Catholic memorial concerts
were held in Rio de Janeiro, Sao Paulo, and other cities in Brazil where
Brazilian Catholic priest Marcelo Rossi and his concert partner Belo
sang a Christian hymn "Noites Traicoeiras" (Treacherous Nights) as
tribute to the seven Columbia astronauts, as well as the other seven
crew members who lost their lives in the Space Shuttle Challenger
disaster in 1986. The concerts were televised to millions throughout
Brazil and the world.{[}citation needed{]}

On March 26, the United States House of Representatives' Science
Committee approved funds for the construction of a memorial at Arlington
National Cemetery for the STS-107 crew. A similar memorial was built at
the cemetery for the last crew of Challenger. On October 28, 2003, the
names of the astronauts were added to the Space Mirror Memorial at the
Kennedy Space Center Visitor Complex in Merritt Island, Florida,
alongside the names of several astronauts and cosmonauts who have died
in the line of duty.

On April 1, 2003, the Opening Day of baseball season, the Houston Astros
(named in honor of the U.S. space program) honored the Columbia crew by
having seven simultaneous first pitches thrown by family and friends of
the crew. For the National Anthem, 107 NASA personnel, including flight
controllers and others involved in Columbia's final mission, carried a
U.S. flag onto the field. In addition, the Astros wore the mission patch
on their sleeves and replaced all dugout advertising with the mission
patch logo for the entire season.

On February 1, 2004, the first anniversary of the Columbia disaster,
Super Bowl XXXVIII held in Houston's Reliant Stadium began with a
pregame tribute to the crew of the Columbia by singer Josh Groban
performing "You Raise Me Up", with the crew of STS-114, the first
post-Columbia Space Shuttle mission, in attendance.

In 2004, Bush conferred posthumous Congressional Space Medals of Honor
to all 14 crew members lost in the Challenger and Columbia accidents.

NASA named several places in honor of Columbia and the crew. Seven
asteroids discovered in July 2001 at the Mount Palomar observatory were
officially given the names of the seven astronauts: 51823 Rickhusband,
51824 Mikeanderson, 51825 Davidbrown, 51826 Kalpanachawla, 51827
Laurelclark, 51828 Ilanramon, 51829 Williemccool. On Mars, the landing
site of the rover Spirit was named Columbia Memorial Station, and
included a memorial plaque to the Columbia crew mounted on the back of
the high gain antenna. A complex of seven hills east of the Spirit
landing site was dubbed the Columbia Hills; each of the seven hills was
individually named for a member of the crew, and Husband Hill in
particular was ascended and explored by the rover. In 2006, the IAU
approved naming of a cluster of seven small craters in the Apollo basin
on the far side of the Moon after the astronauts. Back on Earth, NASA's
National Scientific Balloon Facility was renamed the Columbia Scientific
Balloon Facility.

Other tributes included the decision by Amarillo, Texas, to rename its
airport Rick Husband Amarillo International Airport after the Amarillo
native. Washington State Route 904 was renamed Lt. Michael P. Anderson
Memorial Highway, as it runs through Cheney, Washington, the town where
he graduated from high school. A newly constructed elementary school
located on Fairchild Air Force Base near Spokane, Washington, was named
Michael Anderson Elementary School. Anderson had attended fifth grade at
Blair Elementary, the base's previous elementary school, while his
father was stationed there. A mountain peak near Kit Carson Peak and
Challenger Point in the Sangre de Cristo Range was renamed Columbia
Point, and a dedication plaque was placed on the point in August 2003.
Seven dormitories were named in honor of Columbia crew members at the
Florida Institute of Technology, Creighton University, The University of
Texas at Arlington, and the Columbia Elementary School in the Brevard
County School District. The Huntsville City Schools in Huntsville,
Alabama, a city strongly associated with NASA, named their most recent
high school Columbia High School as a memorial to the crew. A Department
of Defense school in Guam was renamed Commander William C. McCool
Elementary School. The City of Palmdale, California, the birthplace of
the entire shuttle fleet, changed the name of the thoroughfare Avenue M
to Columbia Way. In Avondale, Arizona, the Avondale Elementary School
where Michael Anderson's sister worked had sent a T-shirt with him into
space. It was supposed to have an assembly when he returned from space.
The school was later renamed Michael Anderson Elementary.

The first dedicated meteorological satellite launched by the Indian
Space Research Organisation (ISRO) on September 2, 2002, named Metsat-1,
was later renamed Kalpana-1 by Indian Prime Minister Atal Bihari
Vajpayee in memory of India-born Kalpana Chawla.

In October 2004, both houses of Congress passed a resolution authored by
U.S. Representative Lucille Roybal-Allard and co-sponsored by the entire
contingent of California representatives to Congress changing the name
of Downey, California's Space Science Learning Center to the Columbia
Memorial Space Science Learning Center. The facility is located at the
former manufacturing site of the space shuttles, including Columbia and
Challenger.

The U.S. Air Force's Squadron Officer School at Maxwell Air Force Base,
Alabama, renamed their auditorium in Husband's honor. He was a graduate
of the program. The U.S. Test Pilot School at Edwards Air Force Base in
California named its pilot lounge for Husband.

NASA named a supercomputer "Columbia" in the crew's honor in 2004. It
was located at the NASA Advanced Supercomputing Division at Ames
Research Center on Moffett Federal Airfield near Mountain View,
California. The first part of the system, built in 2003, known as
"Kalpana" was dedicated to Chawla, who worked at Ames prior to joining
the Space Shuttle program.

A U.S. Navy compound at a major coalition military base in Afghanistan
is named Camp McCool. In addition, the athletic field at McCool's alma
mater, Coronado High School in Lubbock, Texas, was renamed the Willie
McCool Track and Field.

A proposed reservoir in Cherokee County in Eastern Texas is to be named
Lake Columbia.

Ilan Ramon High School was established in 2006 in Hod HaSharon, Israel,
in tribute to the first Israeli astronaut. The school's symbol shows the
planet Earth with an aircraft orbiting around it.

The National Naval Medical Center dedicated Laurel Clark Memorial
Auditorium on July 11, 2003. Gamma Phi Beta sorority, of which Clark was
a member, created the Laurel Clark Foundation in her honor. A fountain
in downtown Racine, Wisconsin, which Clark considered her hometown, was
named for her.

PS 58 in Staten Island, New York, was named Space Shuttle Columbia
School in honor of the failed mission.

The Challenger Columbia Stadium in League City, Texas is named in honor
of the victims of both the Columbia disaster as well as the Challenger
disaster in 1986.

A tree for each astronaut was planted in NASA's Astronaut Memorial Grove
at the Johnson Space Center in Houston, Texas, not far from the Saturn V
building, along with trees for each astronaut from the Apollo 1 and
Challenger disasters. Tours of the space center pause briefly near the
grove for a moment of silence, and the trees can be seen from nearby
NASA Road 1.

Columbia Colles, a range of hills on Pluto discovered by the New
Horizons spacecraft in July 2015, was named in honor of the victims of
the disaster.

A starship on Star Trek: Enterprise was named NX-02 Columbia in honor of
the Columbia.

A photo tribute commemorating the Columbia and its crew is displayed in
the "Wings of Fame" section of the queue for Soarin' Around the World at
Disney California Adventure park alongside many other famous air and
space craft.

The Columbia Memorial Space Center is a museum built in honor of the
Columbia in Downey, California.

\section{Effect on space programs}\label{effect-on-space-programs}

\begin{itemize}
\item
  \emph{Following the loss of Columbia, the space shuttle program was
  suspended.}
\item
  \emph{On July 26, 2005, at 10:39 EST, Space Shuttle Discovery cleared
  the tower on the "Return to Flight" mission STS-114, marking the
  shuttle's return to space.}
\item
  \emph{The Columbia Crew Survival Investigation Report released by NASA
  on December 30, 2008, made further recommendations to improve a crew's
  survival chances on future space vehicles, such as the then planned
  Orion spacecraft.}
\end{itemize}

Following the loss of Columbia, the space shuttle program was suspended.
The further construction of the International Space Station (ISS) was
also delayed, as the space shuttles were the only available delivery
vehicle for station modules. The station was supplied using Russian
unmanned Progress ships, and crews were exchanged using Russian-manned
Soyuz spacecraft, and forced to operate on a skeleton crew of two.

Less than a year after the accident, President Bush announced the Vision
for Space Exploration, calling for the space shuttle fleet to complete
the ISS, with retirement by 2010 following the completion of the ISS, to
be replaced by a newly developed Crew Exploration Vehicle for travel to
the Moon and Mars. NASA planned to return the space shuttle to service
around September 2004; that date was pushed back to July 2005.

On July 26, 2005, at 10:39 EST, Space Shuttle Discovery cleared the
tower on the "Return to Flight" mission STS-114, marking the shuttle's
return to space. Overall the STS-114 flight was highly successful, but a
similar piece of foam from a different portion of the tank was shed,
although the debris did not strike the Orbiter. Due to this, NASA once
again grounded the shuttles until the remaining problem was understood
and a solution implemented. After delaying their re-entry by two days
due to adverse weather conditions, Commander Eileen Collins and Pilot
James M. Kelly returned Discovery safely to Earth on August 9, 2005.
Later that same month, the external tank construction site at Michoud
was damaged by Hurricane Katrina. At the time, there was concern that
this would set back further shuttle flights by at least two months and
possibly more.

The actual cause of the foam loss on both Columbia and Discovery was not
determined until December 2005, when x-ray photographs of another tank
showed that thermal expansion and contraction during filling, not human
error, caused cracks that led to foam loss. NASA's Hale formally
apologized to the Michoud workers who had been blamed for the loss of
Columbia for almost three years.

The second "Return to Flight" mission, STS-121, launched on July 4,
2006, at 14:37:55 (EDT), after two previous launch attempts were
scrubbed because of lingering thunderstorms and high winds around the
launch pad. The launch took place despite objections from its chief
engineer and safety head. This mission increased the ISS crew to three.
A 5-inch (130~mm) crack in the foam insulation of the external tank gave
cause for concern, but the Mission Management Team gave the go for
launch. Space Shuttle Discovery touched down successfully on July 17,
2006, at 09:14:43 (EDT) on Runway 15 at the Kennedy Space Center.

On August 13, 2006, NASA announced that STS-121 had shed more foam than
they had expected. While this did not delay the launch for the next
mission---STS-115, originally set to lift off on August 27 ---the
weather and other technical glitches did, with a lightning strike,
Hurricane Ernesto and a faulty fuel tank sensor combining to delay the
launch until September 9. On September 19, landing was delayed an extra
day to examine Atlantis after objects were found floating near the
shuttle in the same orbit. When no damage was detected, Atlantis landed
successfully on September 21.

The Columbia Crew Survival Investigation Report released by NASA on
December 30, 2008, made further recommendations to improve a crew's
survival chances on future space vehicles, such as the then planned
Orion spacecraft. These included improvements in crew restraints,
finding ways to deal more effectively with catastrophic cabin
depressurization, more "graceful degradation" of vehicles during a
disaster so that crews will have a better chance at survival, and
automated parachute systems.

\section{Sociocultural aftermath}\label{sociocultural-aftermath}

\section{Fears of terrorism}\label{fears-of-terrorism}

\begin{itemize}
\item
  \emph{Security surrounding the launch and landing of the space shuttle
  had been increased because the crew included the first Israeli
  astronaut.}
\end{itemize}

After the shuttle's breakup, there were some initial fears that
terrorists might have been involved, but no evidence of that has ever
surfaced. Security surrounding the launch and landing of the space
shuttle had been increased because the crew included the first Israeli
astronaut. The Merritt Island launch facility, like all sensitive
government areas, had increased security after the September 11 attacks.

\section{Purple streak image}\label{purple-streak-image}

\begin{itemize}
\item
  \emph{The San Francisco Chronicle reported that an amateur astronomer
  had taken a five-second exposure that appeared to show "a purplish
  line near the shuttle", resembling lightning, during re-entry.}
\end{itemize}

The San Francisco Chronicle reported that an amateur astronomer had
taken a five-second exposure that appeared to show "a purplish line near
the shuttle", resembling lightning, during re-entry. The CAIB report
concluded that the image was the result of "camera vibrations during a
long-exposure".

\section{2003 Armageddon film hoax}\label{armageddon-film-hoax}

\begin{itemize}
\item
  \emph{In a hoax inspired by the destruction of Columbia, some images
  that were purported to be satellite photographs of the Shuttle's
  "explosion" turned out to be screen captures from the Space Shuttle
  destruction scene of Armageddon.}
\end{itemize}

In response to the disaster, FX canceled its scheduled airing two nights
later of the 1998 film Armageddon, in which the Space Shuttle Atlantis
is depicted as being destroyed by asteroid fragments. In a hoax inspired
by the destruction of Columbia, some images that were purported to be
satellite photographs of the Shuttle's "explosion" turned out to be
screen captures from the Space Shuttle destruction scene of Armageddon.

\section{Music}\label{music}

\begin{itemize}
\item
  \emph{The 2008 album Columbia: We Dare to Dream by Anne Cabrera was
  written as a tribute to Space Shuttle Columbia STS-107, the crew,
  support teams, recovery teams, and the crew's families.}
\item
  \emph{The Hungarian composer Peter Eötvös wrote a piece named Seven
  for solo violin and orchestra in 2006 in memory of the crew of
  Columbia.}
\end{itemize}

The 2003 album Bananas by Deep Purple includes "Contact Lost", an
instrumental piece written by guitarist Steve Morse in remembrance of
the loss. Morse is donating his songwriting royalties to the families of
the astronauts.

Catherine Faber and Callie Hills (the folk group known as Echo's
Children) included a memorial song titled "Columbia" on their 2004 album
From the Hazel Tree.

The 2005 album Ultimatum by The Long Winters contains the song "The
Commander Thinks Aloud", which was songwriter/singer John Roderick's
musing on the crew's perspective of the unexpected catastrophe. In
addition, the January 30, 2015 episode of Hrishikesh Hirway's Song
Exploder podcast presented an interview with John Roderick about the
songwriting and recording process for "The Commander Thinks Aloud".

The Hungarian composer Peter Eötvös wrote a piece named Seven for solo
violin and orchestra in 2006 in memory of the crew of Columbia. Seven
was premiered in 2007 by violinist Akiko Suwanai, conducted by Pierre
Boulez, and it was recorded in 2012 with violinist Patricia
Kopatchinskaja and the composer conducting.

The 2008 album Columbia: We Dare to Dream by Anne Cabrera was written as
a tribute to Space Shuttle Columbia STS-107, the crew, support teams,
recovery teams, and the crew's families. A copy of the album on compact
disc was flown aboard Space Shuttle Discovery mission STS-131 to the
International Space Station by astronaut Clayton Anderson in April 2010.

The Scottish Folk-Rock band Runrig included a song titled "Somewhere" on
their album The Story (2016); the song was dedicated to Laurel Clark
(who had become a fan of the band during her Navy service in Scotland),
and includes a piece of her wake up song, followed by some radio
chatter, at the end.

\section{See also}\label{see-also}

\begin{itemize}
\item
  \emph{Space Shuttle Challenger disaster}
\item
  \emph{Columbia Point}
\item
  \emph{Criticism of the Space Shuttle program}
\end{itemize}

Apollo 1

Space Shuttle Challenger disaster

Criticism of the Space Shuttle program

Engineering disasters

Expedition 6

Columbia Point

\section{Notes}\label{notes}

\section{References}\label{references}

\begin{itemize}
\item
  \emph{~This article incorporates~public domain material from websites
  or documents of the National Aeronautics and Space Administration.}
\end{itemize}

~This article incorporates~public domain material from websites or
documents of the National Aeronautics and Space Administration.

\section{External links}\label{external-links}

\begin{itemize}
\item
  \emph{Columbia Crew Survival Investigation Report PDF}
\item
  \emph{NASA's Space Shuttle Columbia and her crew}
\end{itemize}

Orbiter Wing Leading Edge Protection (upgrade proposed for 1999, but
cancelled)

NASA's Space Shuttle Columbia and her crew

NASA STS-107 Crew Memorial web page

Columbia Crew Survival Investigation Report PDF

Doppler radar animation of the debris after break up

President Bush's remarks at memorial service~-- February 4, 2003

The CBS News Space Reporter's Handbook STS-51L/107 Supplement

The 13-min. Crew cabin video (subtitled). Ends 4-min. before the shuttle
began to disintegrate.

photos of recovered debris stored on the 16th floor of the Vehicle
Assembly Building at KSC

Coordinates: 32°57′22″N 99°2′29″W / 32.95611°N 99.04139°W / 32.95611;
-99.04139
