\textbf{From Wikipedia, the free encyclopedia}

https://en.wikipedia.org/wiki/Drosera\_microphylla\\
Licensed under CC BY-SA 3.0:\\
https://en.wikipedia.org/wiki/Wikipedia:Text\_of\_Creative\_Commons\_Attribution-ShareAlike\_3.0\_Unported\_License

\section{Drosera microphylla}\label{drosera-microphylla}

\begin{itemize}
\item
  \emph{In 1848, Jules Émile Planchon described the new species
  D.~calycina, which was later reduced to synonymy with D.~microphylla.}
\item
  \emph{Drosera microphylla, the golden rainbow, is an erect perennial
  tuberous species in the carnivorous plant genus Drosera.}
\item
  \emph{This taxon was also reduced to a synonym of D.~microphylla.}
\end{itemize}

Drosera microphylla, the golden rainbow, is an erect perennial tuberous
species in the carnivorous plant genus Drosera. It is endemic to Western
Australia and grows on granite outcrops or in sandy or laterite soils.
D.~microphylla produces small, circular, peltate carnivorous leaves
along erect stems that can be 10--40~cm (4--16~in) high. It blooms from
June to September, displaying its large golden sepals and smaller,
variably-coloured petals. In populations near Perth, the petals are red,
whereas petal colour near Albany tends to be orange. Some plants east of
Esperance have white petals.

D.~microphylla was first described and named by Stephan Endlicher in
1837. In 1848, Jules Émile Planchon described the new species
D.~calycina, which was later reduced to synonymy with D.~microphylla.
George Bentham described the new variety D.~calycina~var.~minor in 1864.
This taxon was also reduced to a synonym of D.~microphylla. Lastly, in
his 1906 taxonomic monograph of the family Droseraceae, Ludwig Diels
also described a new variety, D.~microphylla~var.~macropetala, which was
also later reduced to a synonym.

\section{See also}\label{see-also}

\begin{itemize}
\item
  \emph{List of Drosera species}
\end{itemize}

List of Drosera species

\section{References}\label{references}
