\textbf{From Wikipedia, the free encyclopedia}

https://en.wikipedia.org/wiki/Canadian\_literature\\
Licensed under CC BY-SA 3.0:\\
https://en.wikipedia.org/wiki/Wikipedia:Text\_of\_Creative\_Commons\_Attribution-ShareAlike\_3.0\_Unported\_License

\section{Canadian literature}\label{canadian-literature}

\begin{itemize}
\item
  \emph{Canadian literature (widely abbreviated as CanLit) is literature
  originating from Canada.}
\item
  \emph{Influences on Canadian writers are broad, both geographically
  and historically.}
\item
  \emph{Canadian writers have produced a variety of genres.}
\end{itemize}

Canadian literature (widely abbreviated as CanLit) is literature
originating from Canada. Canadian writers have produced a variety of
genres. Influences on Canadian writers are broad, both geographically
and historically.

Since before European contact and the Confederation of Canada,
Indigenous people in North America have occupied the land and have
maintained a rich and diverse history of culture, identity, language,
art and literature. "Indigenous literature" is a problematic term, as
every cultural group has its own distinct oral tradition, language, and
cultural practices. Therefore, Indigenous literatures in Canada is a
more inclusive term for understanding the variety of languages and
traditions across communities.

After the colonization of Canada, the dominant European cultures were
originally English, French, and Gaelic. After Prime Minister Pierre
Trudeau's "Announcement of Implementation of Policy of Multiculturalism
within Bilingual Framework" in 1971, Canadian critics and academics
gradually began to recognize that there existed a more diverse
population of readers and writers.{[}citation needed{]} The country's
literature has been strongly influenced by international immigration,
particularly in recent decades. Since the 1980s Canada's ethnic and
cultural diversity have been openly reflected in its literature, with
many of its most prominent writers focusing on ethnic minority identity,
duality and cultural differences. However, Canadians have been less
willing to acknowledge the diverse languages of Canada, besides English
and French.

\section{Categories}\label{categories}

\begin{itemize}
\item
  \emph{By author: Canadian women; Acadians, Aboriginal peoples in
  Canada; Irish Canadians; Italian-Canadians: South-Asian-Canadian}
\item
  \emph{Literary period: "The Confederation Poets", "Canadian
  postmoderns" or "Canadian Poets Between the Wars."}
\end{itemize}

Regional--such as the prairie novel or Quebec theatre.

By author: Canadian women; Acadians, Aboriginal peoples in Canada; Irish
Canadians; Italian-Canadians: South-Asian-Canadian

Literary period: "The Confederation Poets", "Canadian postmoderns" or
"Canadian Poets Between the Wars."

\includegraphics[width=4.68653in,height=5.50000in]{media/image1.jpg}\\
\emph{Gabrielle Roy was a notable French Canadian author.}

\section{French-Canadian literature}\label{french-canadian-literature}

\begin{itemize}
\item
  \emph{All books it contained were moved to the Canadian parliament in
  Montreal when the two Canadas, lower and upper, were united.}
\item
  \emph{On April 25, 1849, a dramatic event occurred: the Canadian
  parliament was burned by furious people along with thousands of French
  Canadian books and a few hundred of English books.}
\end{itemize}

In 1802, the Lower Canada legislative library was founded, being one of
the first in Occident, the first in the Canadas. For comparison, the
library of the British House of Commons was founded sixteen years later.
The library had some rare titles about geography, natural science and
letters. All books it contained were moved to the Canadian parliament in
Montreal when the two Canadas, lower and upper, were united. On April
25, 1849, a dramatic event occurred: the Canadian parliament was burned
by furious people along with thousands of French Canadian books and a
few hundred of English books. This is why some people still affirm
today, falsely, that from the early settlements until the 1820s, Quebec
had virtually no literature. Though historians, journalists, and learned
priests published, overall the total output that remain from this period
and that had been kept out of the burned parliament is small.

It was the rise of Quebec patriotism and the 1837 Lower Canada
Rebellion, in addition to a modern system of primary school education,
which led to the rise of French-Canadian fiction. L'influence d'un livre
by Philippe-Ignace-Francois Aubert de Gaspé is widely regarded as the
first French-Canadian novel. The genres which first became popular were
the rural novel and the historical novel. French authors were
influential, especially authors like Balzac.

In 1866, Father Henri-Raymond Casgrain became one of Quebec's first
literary theorists. He argued that literature's goal should be to
project an image of proper Catholic morality. However, a few authors
like Louis-Honoré Fréchette and Arthur Buies broke the conventions to
write more interesting works.

This pattern continued until the 1930s with a new group of authors
educated at the Université Laval and the Université de Montréal. Novels
with psychological and sociological foundations became the norm.
Gabrielle Roy and Anne Hébert even began to earn international acclaim,
which had not happened to French-Canadian literature before. During this
period, Quebec theatre, which had previously been melodramas and
comedies, became far more involved.

French-Canadian literature began to greatly expand with the turmoil of
the Second World War, the beginnings of industrialization in the 1950s,
and most especially the Quiet Revolution in the 1960s. French-Canadian
literature also began to attract a great deal of attention globally,
with Acadian novelist Antonine Maillet winning the Prix Goncourt. An
experimental branch of Québécois literature also developed; for instance
the poet Nicole Brossard wrote in a formalist style.\\
In 1979, Roch Carrier wrote the story The Hockey Sweater, which
highlighted the cultural and social tensions between English and French
speaking Canada.

\section{Before Confederation}\label{before-confederation}

\begin{itemize}
\item
  \emph{Moreover, their books often dealt with survival and the rugged
  Canadian environment; these themes re-appear in other Canadian works,
  including Margaret Atwood's Survival.}
\item
  \emph{However, one of the earliest "Canadian" writers virtually always
  included in Canadian literary anthologies is Thomas Chandler
  Haliburton (1796--1865), who died just two years before Canada's
  official birth.}
\end{itemize}

Because Canada only officially became a country on July 1, 1867, it has
been argued that literature written before this time was colonial. For
example, Susanna Moodie and Catharine Parr Traill, English sisters who
adopted the country as their own, moved to Upper Canada in 1832. They
recorded their experiences as pioneers in Parr Traill's The Backwoods of
Canada (1836) and Canadian Crusoes (1852), and Moodie's Roughing It in
the Bush (1852) and Life in the Clearings (1853). However, both women
wrote until their deaths, placing them in the country for more than 50
years and certainly well past Confederation. Moreover, their books often
dealt with survival and the rugged Canadian environment; these themes
re-appear in other Canadian works, including Margaret Atwood's Survival.
Moodie and Parr Traill's sister, Agnes Strickland, remained in England
and wrote elegant royal biographies, creating a stark contrast between
Canadian and English literatures.

However, one of the earliest "Canadian" writers virtually always
included in Canadian literary anthologies is Thomas Chandler Haliburton
(1796--1865), who died just two years before Canada's official birth. He
is remembered for his comic character, Sam Slick, who appeared in The
Clockmaker and other humorous works throughout Haliburton's life.

\includegraphics[width=3.52000in,height=5.50000in]{media/image2.jpg}\\
\emph{Charles G. D. Roberts was a poet that belonged to an informal
group known as the Confederation Poets.}

\section{After 1867}\label{after-1867}

\begin{itemize}
\item
  \emph{Canadian author Farley Mowat is best known for his work Never
  Cry Wolf (1963) and his Governor General's Award-winning children's
  book, Lost in the Barrens (1956).}
\item
  \emph{The best-known Canadian children's writers include L. M.
  Montgomery and Monica Hughes.}
\item
  \emph{Following World War II, writers such as Mavis Gallant, Mordecai
  Richler, Norman Levine, Margaret Laurence and Irving Layton added to
  the Modernist influence to Canadian literature previously introduced
  by F.~R. Scott, A.~J.~M. Smith and others associated with the McGill
  Fortnightly.}
\end{itemize}

A group of poets now known as the "Confederation Poets", including
Charles G. D. Roberts, Archibald Lampman, Bliss Carman, Duncan Campbell
Scott, and William Wilfred Campbell, came to prominence in the 1880s and
1890s. Choosing the world of nature as their inspiration, their work was
drawn from their own experiences and, at its best, written in their own
tones. Isabella Valancy Crawford, Frederick George Scott, and Francis
Sherman are also sometimes associated with this group.

During this period, E. Pauline Johnson and William Henry Drummond were
writing popular poetry - Johnson's based on her part-Mohawk heritage,
and Drummond, the Poet of the Habitant, writing dialect verse.

Reacting against a tradition that emphasised the wilderness, poet
Leonard Cohen's novel Beautiful Losers (1966), was labelled by one
reviewer "the most revolting book ever written in Canada". However,
Cohen is perhaps best known as a folk singer and songwriter, with an
international following.

Canadian author Farley Mowat is best known for his work Never Cry Wolf
(1963) and his Governor General's Award-winning children's book, Lost in
the Barrens (1956).

Following World War II, writers such as Mavis Gallant, Mordecai Richler,
Norman Levine, Margaret Laurence and Irving Layton added to the
Modernist influence to Canadian literature previously introduced by
F.~R. Scott, A.~J.~M. Smith and others associated with the McGill
Fortnightly. This influence, at first, was not broadly appreciated.
Norman Levine's Canada Made Me, a travelogue that presented a sour
interpretation of the country in 1958, for example, was widely rejected.

After 1967, the country's centennial year, the national government
increased funding to publishers and numerous small presses began
operating throughout the country.\\
The best-known Canadian children's writers include L. M. Montgomery and
Monica Hughes.

\section{Contemporary Canadian literature: After
1967}\label{contemporary-canadian-literature-after-1967}

\begin{itemize}
\item
  \emph{Lawrence Hill's Book of Negroes won the 2008 Commonwealth
  Writers' Prize Overall Best Book Award, while Alice Munro became the
  first Canadian to win the Nobel Prize in Literature in 2013.}
\item
  \emph{This group, along with Nobel Laureate Alice Munro, who has been
  called the best living writer of short stories in English, were the
  first to elevate Canadian Literature to the world stage.}
\end{itemize}

Arguably, the best-known living Canadian writer internationally
(especially since the deaths of Robertson Davies and Mordecai Richler)
is Margaret Atwood, a prolific novelist, poet, and literary critic. Some
great 20th-century Canadian authors include Margaret Laurence, and
Gabrielle Roy.

This group, along with Nobel Laureate Alice Munro, who has been called
the best living writer of short stories in English, were the first to
elevate Canadian Literature to the world stage. During the post-war
decades only a handful of books of any literary merit were published
each year in Canada, and Canadian literature was viewed as an appendage
to British and American writing. When academic Clara Thomas decided in
the 1940s to concentrate on Canadian literature for her master's thesis,
the idea was so novel and so radical that word of her decision reached
The Globe and Mail books editor William Arthur Deacon, who then
personally reached out to Thomas to pledge his and the newspaper's
resources in support of her work.

Other major Canadian novelists include Carol Shields, Lawrence Hill, and
Alice Munro. Carol Shields novel The Stone Diaries won the 1995 Pulitzer
Prize for Fiction, and another novel, Larry's Party, won the Orange
Prize in 1998. Lawrence Hill's Book of Negroes won the 2008 Commonwealth
Writers' Prize Overall Best Book Award, while Alice Munro became the
first Canadian to win the Nobel Prize in Literature in 2013. Munro also
received the Man Booker International Prize in 2009.

In the 1960s, a renewed sense of nation helped foster new voices in
Canadian poetry, including: Margaret Atwood, Michael Ondaatje, Leonard
Cohen, Eli Mandel and Margaret Avison. Others such as Al Purdy, Milton
Acorn, and Earle Birney, already published, produced some of their best
work during this period.

The TISH Poetry movement in Vancouver brought about poetic innovation
from Jamie Reid, George Bowering, Fred Wah, Frank Davey, Daphne Marlatt,
David Cull, and Lionel Kearns.

More recently a younger generation of Canadian poets has been expanding
the boundaries of originality: Christian Bök, Ken Babstock, Karen Solie,
Lynn Crosbie, Patrick Lane, George Elliott Clarke and Barry Dempster
have all imprinted their unique consciousnesses onto the map of Canadian
imagery.\\
A notable anthology of Canadian poetry is The New Oxford book of
Canadian Verse, edited by Margaret Atwood (.mw-parser-output
cite.citation\{font-style:inherit\}.mw-parser-output .citation
q\{quotes:"\textbackslash{}"""\textbackslash{}"""'""'"\}.mw-parser-output
.citation .cs1-lock-free
a\{background:url("//upload.wikimedia.org/wikipedia/commons/thumb/6/65/Lock-green.svg/9px-Lock-green.svg.png")no-repeat;background-position:right
.1em center\}.mw-parser-output .citation .cs1-lock-limited
a,.mw-parser-output .citation .cs1-lock-registration
a\{background:url("//upload.wikimedia.org/wikipedia/commons/thumb/d/d6/Lock-gray-alt-2.svg/9px-Lock-gray-alt-2.svg.png")no-repeat;background-position:right
.1em center\}.mw-parser-output .citation .cs1-lock-subscription
a\{background:url("//upload.wikimedia.org/wikipedia/commons/thumb/a/aa/Lock-red-alt-2.svg/9px-Lock-red-alt-2.svg.png")no-repeat;background-position:right
.1em center\}.mw-parser-output .cs1-subscription,.mw-parser-output
.cs1-registration\{color:\#555\}.mw-parser-output .cs1-subscription
span,.mw-parser-output .cs1-registration span\{border-bottom:1px
dotted;cursor:help\}.mw-parser-output .cs1-ws-icon
a\{background:url("//upload.wikimedia.org/wikipedia/commons/thumb/4/4c/Wikisource-logo.svg/12px-Wikisource-logo.svg.png")no-repeat;background-position:right
.1em center\}.mw-parser-output
code.cs1-code\{color:inherit;background:inherit;border:inherit;padding:inherit\}.mw-parser-output
.cs1-hidden-error\{display:none;font-size:100\%\}.mw-parser-output
.cs1-visible-error\{font-size:100\%\}.mw-parser-output
.cs1-maint\{display:none;color:\#33aa33;margin-left:0.3em\}.mw-parser-output
.cs1-subscription,.mw-parser-output .cs1-registration,.mw-parser-output
.cs1-format\{font-size:95\%\}.mw-parser-output
.cs1-kern-left,.mw-parser-output
.cs1-kern-wl-left\{padding-left:0.2em\}.mw-parser-output
.cs1-kern-right,.mw-parser-output
.cs1-kern-wl-right\{padding-right:0.2em\}ISBN~0-19-540450-5).

Anne Carson is probably the best known Canadian poet living today.
Carson in 1996 won the Lannan Literary Award for poetry. The
foundation's awards in 2006 for poetry, fiction and nonfiction each came
with \$US 150,000.

\section{Canadian authors who have won international
awards}\label{canadian-authors-who-have-won-international-awards}

\begin{itemize}
\item
  \emph{Lawrence Hill's The Book of Negroes won the 2008 Commonwealth
  Writers' Prize Overall Best Book Award.}
\item
  \emph{In 1992, Michael Ondaatje became the first Canadian to win the
  Booker Prize for The English Patient.}
\item
  \emph{Alice Munro became the first Canadian to win the Nobel Prize in
  Literature in 2013.}
\end{itemize}

In 1992, Michael Ondaatje became the first Canadian to win the Booker
Prize for The English Patient.

Margaret Atwood won the Booker in 2000 for The Blind Assassin and Yann
Martel won it in 2002 for Life of Pi.

Alistair MacLeod won the 2001 International Dublin Literary Award for No
Great Mischief and Rawi Hage won it in 2008 for De Niro's Game.

Carol Shields's The Stone Diaries won the 1995 Pulitzer Prize for
Fiction, and in 1998 her novel Larry's Party won the Orange Prize.

Lawrence Hill's The Book of Negroes won the 2008 Commonwealth Writers'
Prize Overall Best Book Award.

Alice Munro became the first Canadian to win the Nobel Prize in
Literature in 2013. Munro also received the Man Booker International
Prize in 2009

Margaret Atwood received the Peace Prize of the German Book Trade in
2017

\section{Awards}\label{awards}

\begin{itemize}
\item
  \emph{Governor-General's Awards for the best Canadian children's
  literature, text-based or illustrated, in both English and French}
\item
  \emph{Floyd S. Chalmers Canadian Play Awards for best Canadian play
  staged by a Canadian theatre company}
\item
  \emph{There are a number of notable Canadian awards for literature:}
\item
  \emph{Giller Prize for the best Canadian novel or book of short
  stories in English}
\end{itemize}

There are a number of notable Canadian awards for literature:

The Atlantic Writers Competition highlights talent across the Atlantic
Provinces.

Books in Canada First Novel Award for the best first novel of the year

Canadian Authors Association Awards for Adult Literature, honouring
works by Canadian writers that achieve excellence without sacrificing
popular appeal since 1975

CBC Literary Awards

Canada Council Molson Prize for distinguished contributions to Canada's
cultural and intellectual heritage

Danuta Gleed Literary Award for a first collection of short fiction by a
Canadian author writing in English

Dayne Ogilvie Prize for an emerging writer in the lesbian, gay, bisexual
or transgender communities

Dorothy Livesay Poetry Prize for the best collection of poetry by a
resident of British Columbia

Doug Wright Awards for graphic literature and novels

Ethel Wilson Fiction Prize for the best novel or collection of short
stories by a resident of British Columbia

Floyd S. Chalmers Canadian Play Awards for best Canadian play staged by
a Canadian theatre company

Hilary Weston Writers' Trust Prize for Nonfiction for best work of
nonfiction

Gerald Lampert Award for the best new poet

Lane Anderson Award for best Canadian non-fiction science

Giller Prize for the best Canadian novel or book of short stories in
English

Governor General's Awards for the best Canadian fiction, poetry,
non-fiction, drama, and translation, in both English and French

Griffin Poetry Prize for the best book of poetry, one award each for a
Canadian poet and an international poet

Indigenous Voices Awards for works of literature by First Nations, Métis
and Inuit writers

Marian Engel Award for female writers in mid-career

Matt Cohen Award to honour a Canadian writer for a lifetime of
distinguished achievement

Milton Acorn Poetry Awards for an outstanding "people's poet"

National Business Book Award

Pat Lowther Award for poetry written by a woman

Prix Aurora Awards for Canadian science fiction and fantasy, in English
and French

Prix Athanase-David for a Quebec writer

Prix Gilles-Corbeil for a Quebec writer in honour of his or her lifetime
body of work (presented every three years)

Prix Trillium for the best work by a Franco-Ontarian writer

Prix littéraire France-Québec

Quebec Writers' Federation Awards for the best fiction, poetry,
non-fiction, children's and young adult literature, first book by
English Quebec writers, and the best translation (English and French
alternate years)

RBC Bronwen Wallace Award for Emerging Writers

Rogers Writers' Trust Fiction Prize for the best work of fiction

Shaughnessy Cohen Award for Political Writing

Stephen Leacock Award For Humour

Trillium Book Award for the best work by an Ontario writer

W.O. Mitchell Literary Prize for a writer who has made a distinguished
lifetime contribution both to Canadian literature and to mentoring new
writers

Room of One's Own Annual Award for poetry and literature

3-Day Novel Contest annual literary marathon, born in Canada

Writers' Trust Engel/Findley Award for a distinguished writer in
mid-career

Writers' Trust / McClelland \& Stewart Journey Prize

Awards For Children's and Young Adult Literature:

Young Adult Novel Prize of the Atlantic Writers Competition

R.Ross Annett Award for Children's Literature

Geoffrey Bilson Award for Historical Fiction

Ann Connor Brimer Award

Canadian Library Association Book of the Year Award for Children

CLA Young Adult Canadian Book Award

Sheila A. Egoff Children's Literature Prize

Elizabeth Mrazik-Cleaver Canadian Picture Book Award

Floyd S. Chalmers Award for Theatre for Young Adults

Amelia Frances Howard-Gibbon Illustrator's Award

Information Book of the Year

I0DE Book Award

Manitoba Young Reader's Choice Award

Max and Greta Ebel Memorial Award for Children's Writing

Norma Fleck Award for children's non-fiction

Governor-General's Awards for the best Canadian children's literature,
text-based or illustrated, in both English and French

QWF Prize for Children's and Young Adult Literature

Vicky Metcalf Award for Children's Literature

\section{Further reading}\label{further-reading}

\section{See also}\label{see-also}

\begin{itemize}
\item
  \emph{Canadian poetry}
\item
  \emph{List of Canadian writers}
\item
  \emph{Literature of Newfoundland and Labrador}
\item
  \emph{Canadian science fiction}
\item
  \emph{Canadian content}
\item
  \emph{List of Canadian short story writers}
\item
  \emph{The Canadian Centenary Series}
\end{itemize}

Canadian poetry

Canadian science fiction

List of Canadian writers

List of Canadian short story writers

The Canadian Centenary Series

Canada Reads

List of fiction set in Toronto

Canadian content

Literature of Newfoundland and Labrador

Theatre of Canada

\section{References}\label{references}

\section{External links}\label{external-links}

\begin{itemize}
\item
  \emph{Canadian Literature - CanLit}
\item
  \emph{Introduction - Canadian Writers - Library and Archives Canada}
\item
  \emph{Canadian Literature - Historica - The Canadian Encyclopedia
  Library}
\item
  \emph{Studies in Canadian Literature / Études en littérature
  canadienne - University of New Brunswick}
\item
  \emph{Canadian Writers - Resource for Canadian authors publishing in
  English or French - Athabasca University, Alberta}
\end{itemize}

Introduction - Canadian Writers - Library and Archives Canada

Canadian Literature - CanLit

Canadian Literature - Historica - The Canadian Encyclopedia Library

Canadian Writers - Resource for Canadian authors publishing in English
or French - Athabasca University, Alberta

Studies in Canadian Literature / Études en littérature canadienne -
University of New Brunswick
