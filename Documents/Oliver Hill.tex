\textbf{From Wikipedia, the free encyclopedia}

https://en.wikipedia.org/wiki/Oliver\%20Hill\\
Licensed under CC BY-SA 3.0:\\
https://en.wikipedia.org/wiki/Wikipedia:Text\_of\_Creative\_Commons\_Attribution-ShareAlike\_3.0\_Unported\_License

\section{Oliver Hill}\label{oliver-hill}

\begin{itemize}
\item
  \emph{Oliver White Hill, Sr. (May 1, 1907 -- August 5, 2007) was an
  American civil rights attorney from Richmond, Virginia.}
\end{itemize}

Oliver White Hill, Sr. (May 1, 1907 -- August 5, 2007) was an American
civil rights attorney from Richmond, Virginia. His work against racial
discrimination helped end the doctrine of "separate but equal." He also
helped win landmark legal decisions involving equality in pay for black
teachers, access to school buses, voting rights, jury selection, and
employment protection. He retired in 1998 after practicing law for
almost 60 years. Among his numerous awards was the Presidential Medal of
Freedom, which U.S. President Bill Clinton awarded him in 1999.

\section{Childhood, education and family
life}\label{childhood-education-and-family-life}

\begin{itemize}
\item
  \emph{Their son, Oliver White Hill Jr., was born on September 19, 1949
  in Richmond, after Hill returned from his World War II service.}
\item
  \emph{Young Oliver got along very well with Joseph Hill, and
  eventually changed his birth certificate to reflect Hill's surname.}
\item
  \emph{Marshall graduated first in his law school class in 1933, and
  Oliver White Hill second.}
\end{itemize}

Oliver White was born in Richmond, Virginia, on May 1, 1907. His father,
William Henry White Jr., abandoned his mother Olivia Lewis White Hill
(1888--1980) shortly after the boy's birth, although W.H. White Jr.
briefly returned six months later before leaving Richmond permanently.
Though uncommon and difficult to obtain at the time, his mother thus
obtained a divorce in 1911. When Oliver was 9 years old, after the
deaths of his maternal grandmother and paternal grandfather, W.H. White
Jr. returned briefly to Richmond and asked his son if he wanted to live
with him in New York City (Oliver declined the offer).

Because Olivia Hill worked at the Homestead Resort in Hot Springs,
Virginia, during the spring and fall seasons, and a related resort in
Bermuda during the winter, Oliver was raised by her grandmother and
grandaunt in a small house on St. James Street in a predominantly
African-American section of Richmond. When Oliver was six years old, his
mother Olivia Hill returned to Richmond for her mother's funeral, and
introduced Oliver to her new husband, Joseph Cartwright Hill, who worked
as a bellman at the Homestead resort. Oliver's maternal grandmother had
moved to Scranton, Pennsylvania, but returned to Richmond shortly before
her death. His paternal grandfather William Henry White Sr. had founded
Mount Carmel Baptist Church in Richmond, which the family attended and
where Oliver attended Sunday school, but Rev. White died on August 13,
1913, not long after grandmother Lewis. His paternal grandmother, Kate
Garnet White, was reputedly part Native American, but had little to do
with Oliver and his mother. Ancestors of both families had come from
Chesterfield County, and at least some were likely enslaved before the
American Civil War. Young Oliver got along very well with Joseph Hill,
and eventually changed his birth certificate to reflect Hill's surname.

Joseph Hill moved his wife and Oliver to Roanoke, where he operated a
pool hall until Prohibition made that uneconomic, so Joseph and soon
Olivia Hill resumed their hospitality industry careers. The Hill family
lived in the same house as Bradford Pentecost and his wife Lelia (d.
1943), who had no children, but often took in boarders who worked on the
Norfolk and Southern Railroad like Mr. Pentecost (a cook). Hot Springs
had no schools for black children, so Oliver remained in Roanoke, where
he attended segregated schools until the eighth grade (the last offered
to blacks in the city at the time). He also obtained his first jobs---at
a local ice cream parlor (until the local police cited it for violating
child labor laws), as well as delivering newspapers and ice, finding
more strenuous and well-paying work as he grew stronger. During this
time, the Pentecost family bought a larger house, 401 Gilmer Avenue.
Hill came to consider Roanoke his childhood home. He later specifically
remembered not minding serving food to strikebreakers during the
Railroad Strike of 1922, because the striking unions were all-white, and
sought to limit Negro employees to hard labor. Mrs. Pentecost tried to
keep Oliver from working on the railroad, because her brother dropped
out of college to work, and never returned, although many of her
boarders were taking a year off working to pay for college.

In 1916, the Hills moved to Washington, D.C., where Joseph Hill worked
at the Navy Yard during the First World War. Oliver was in the sixth
grade, but he did not like the D.C. elementary school he attended for a
semester, and so was allowed to return to Roanoke and his foster
parents, the Pentecosts. In 1923, further education being unavailable to
him in Roanoke, Hill moved to Washington D.C. to attend (and graduate
from) Dunbar High School, which at the time may have offered the best
education available to black children in the country. At first Oliver
was behind a semester academically, and also lacked scholarly
seriousness. He also played various sports-especially tennis in Roanoke,
but baseball, football and basketball at Dunbar (which did not have a
tennis team).

Joseph Hill's brother Samuel worked for the post office in Washington,
D.C., and in his off-hours worked as a lawyer handling mostly wills and
real estate transactions. Samuel Hill died of a cerebral hemorrhage when
Oliver was a college sophomore, and his widow gave Oliver his law books,
which piqued his interest in law school. Upon learning that the Supreme
Court had taken away many rights of African Americans, and that in the
1920s Congress could not even pass legislation outlawing lynching
Negroes, Oliver White Hill determined to go to law school and reverse
the Plessy v. Ferguson decision issued slightly before his birth.

Hill performed various part-time jobs in D.C. during his college years
at Howard University and later the Howard University School of Law. He
spent summers earning money for his education at various resorts in the
Mid-Atlantic region, including Eagles Mere, Pennsylvania, Pittsfield,
Massachusetts, and Oswegatchie, Connecticut, as well as for the Canadian
Pacific Railway.

After earning his undergraduate degree in 1930, Oliver attended Howard
law school. There, Hill was a classmate and close friend of future
Supreme Court Associate Justice Thurgood Marshall, although they were
leaders of the rival Omega Psi Phi and Alpha Phi Alpha fraternities.
Both studied under Charles Hamilton Houston, the chief architect in
challenging Jim Crow laws through legal means. Marshall graduated first
in his law school class in 1933, and Oliver White Hill second.

Hill courted and married Beresenia Ann Walker (April 8, 1911 --
September 27, 1993) of Richmond on September 5, 1934. She taught school
in Washington during his early years of practice in Roanoke, and he soon
moved back to Washington. She was the daughter of Andrew J. Walker and
Yetta Lee Brown, and niece of Maggie Lena Walker. Their son, Oliver
White Hill Jr., was born on September 19, 1949 in Richmond, after Hill
returned from his World War II service.

\section{Career}\label{career}

\section{Early years}\label{early-years}

\begin{itemize}
\item
  \emph{After some meetings at Howard, Hill returned to Virginia in
  1939, thinking to establish a law firm with J. Byron Hopkins (a class
  above Hill at Howard Law) and J. Thomas Hewin Jr. (whose father had an
  established practice in Richmond).}
\item
  \emph{Ransom, Hill won his first civil rights case.}
\item
  \emph{Hill also traveled with Jesse Tinsley on NAACP errands and
  speaking assignments, and served the Virginia Teachers Association
  (because the Virginia Education Association only represented white
  teachers).}
\end{itemize}

Hill began practicing law in Roanoke during the Great Depression,
sharing an office with J. Henry Clayer (a lawyer who once worked in the
district attorney's office in Chicago), and a dentist and a physician.
It was a general practice and also involved criminal work in surrounding
counties, where blacks encountered prejudice. Howard Law School had
received funds to challenge segregation of Negroes, but that endowment
was nearly wiped out during the 1929 stock market crash. In 1935, Hill
helped organize the Virginia State Conference of NAACP branches, with
the help of his friend Leon A. Ransom. Printer W.P. Milner of the
Norfolk Journal and Guide newspaper was the first President and Dr.
Jesse M. Tinsley (a Richmond dentist and president of the Richmond
branch) was the Vice-President. When Milner was fired from his job for
union activities, Tinsley became the state conference's President (and
remained so for 30 years).

However, the new general practice did not thrive in Roanoke, and he
missed his wife, so Hill returned to Washington D.C. in June 1936. He
and his college friend William T. Whitehead frequently took jobs as
waiters, and also tried organizing waiters and cooks for the Congress of
Industrial Organizations because the American Federation of Labor unions
were white or segregated.

After some meetings at Howard, Hill returned to Virginia in 1939,
thinking to establish a law firm with J. Byron Hopkins (a class above
Hill at Howard Law) and J. Thomas Hewin Jr. (whose father had an
established practice in Richmond). Hill also traveled with Jesse Tinsley
on NAACP errands and speaking assignments, and served the Virginia
Teachers Association (because the Virginia Education Association only
represented white teachers). He also met and worked some cases in
outlying communities with Martin A. Martin of Danville. In 1942, Martin
became the first African-American lawyer in the U.S. Department of
Justice's Trial Division, but he did not like his assignments and
resigned a year later to work in Richmond with Hill and Spottswood W.
Robinson III, forming Hill, Martin \& Robinson at 623 N. Third Street in
Richmond.

In 1940, working with fellow attorneys Thurgood Marshall, William H.
Hastie, and Leon A. Ransom, Hill won his first civil rights case. The
decision in Alston v. School Board of Norfolk, Va., 112 F.2d 992 (4th
Cir.), cert. denied 311 U.S. 693 (1940) gained pay equity for black
teachers. The firm also worked to equalize school facilities and obtain
bus transportation for black pupils.

On Labor Day, 1942, Hill accompanied five schoolgirls and two fathers to
two high schools in Sussex County, Virginia. The county had 23 one-,
two- or three- room elementary schools (all lacking indoor plumbing)
serving 1,902 children of African-American parents, but offered them
education beyond the seventh grade only at a Training School in Waverly,
Virginia (and no transportation). As of June 1942, Sussex County also
had four new consolidated elementary and high schools for its 867 white
students. The surprised principals of Jarrett High School and Stony
Creek High School denied the African-American girls admission to their
respective schools. Hill politely thanked them, left, and soon filed a
lawsuit in the federal district courthouse in Richmond, seeking to
declare the disparities between white and black high school students
unconstitutional. The suit was dismissed when the county obtained three
buses to provide trips to the Training School, and later admitted over
60 black high school students to Waverly High School. However, of those
girls, only Helen Owens ever obtained a high school diploma, and that
was from Peabody High School in Petersburg; three of her fellow
plaintiffs attended the Sussex Training Academy, and another attended
but did not graduate from Peabody.

\section{Wartime service}\label{wartime-service}

\begin{itemize}
\item
  \emph{Unlike Tucker, Hill was not allowed to enlist in Officer
  Candidate School, but instead served in a unit of black engineers, and
  performed mostly support duties as a Staff Sergeant.}
\item
  \emph{In 1943, although Hill was 36 years old, somehow he was drafted
  during World War II.}
\item
  \emph{Like his partner Samuel W. Tucker and other African-Americans,
  Hill experienced racial discrimination during his military service,
  particularly by white officers.}
\end{itemize}

In 1943, although Hill was 36 years old, somehow he was drafted during
World War II. He chose to join the United States Army, rather than the
United States Navy which he thought at the time only allowed black
sailors to perform mess-hall duties. Like his partner Samuel W. Tucker
and other African-Americans, Hill experienced racial discrimination
during his military service, particularly by white officers. Unlike
Tucker, Hill was not allowed to enlist in Officer Candidate School, but
instead served in a unit of black engineers, and performed mostly
support duties as a Staff Sergeant. He credited the unprofessional
racist comments of the unit's white chaplain (who tried to stop white
English people from fraternizing with the black soldiers) with saving
him and his unit from near-certain death during the D-Day Normandy
invasion. Hill served in the European Theatre of World War II until V-E
Day, when his unit was shipped to the Pacific, from where he was
ultimately discharged.

\section{Politics}\label{politics}

\begin{itemize}
\item
  \emph{Hill ran again in 1949 and became the first African American on
  the City Council of Richmond since Reconstruction.}
\item
  \emph{This allowed Hill to make his only attempts for election to
  public office.}
\item
  \emph{Returning to his law practice in Richmond, Hill also served as
  chief of the Virginia branch's legal staff.}
\item
  \emph{In 1947, Hill persuaded W. Lester Banks to act as the Virginia
  Chapter's Executive Director and handle the organization's day-to-day
  activities.}
\end{itemize}

Returning to his law practice in Richmond, Hill also served as chief of
the Virginia branch's legal staff. In 1947, Hill persuaded W. Lester
Banks to act as the Virginia Chapter's Executive Director and handle the
organization's day-to-day activities. This allowed Hill to make his only
attempts for election to public office.

In 1947, he first ran for the City Council of Richmond (which had
changed its system to nine members elected at-large rather than by
districts as before the war), but came in 10th in a race for 9 seats.
Hill ran again in 1949 and became the first African American on the City
Council of Richmond since Reconstruction. At the time, the city's
population was about 30\% black, and Hill said he hoped that his
election would not only help remove prejudice against blacks in the
city, but also give Richmonders a better experience of the
responsibilities of citizenship. However, Hill did not win re-election
in the next election (1951), failing to make the last available seat by
44 votes, because controversy over his legal work discussed below had
begun, and because Hill also supported an unpopular highway project.

\section{Civil rights pioneer}\label{civil-rights-pioneer}

\begin{itemize}
\item
  \emph{He continued civil rights litigation as a partner of Hill,
  Tucker and Marsh in Richmond until he retired in 1998.}
\item
  \emph{The Virginia NAACP's efforts continued, and Hill returned to his
  Virginia practice and leadership of the state's legal staff in 1965,
  after passage of the Civil Rights Act of 1964 and the Voting Rights
  Act of 1965.}
\item
  \emph{Nonetheless, Hill and his clients continued their legal battles
  to assert their civil rights.}
\end{itemize}

As head of the Virginia branch's legal staff, which also included his
law partner Spottswood W. Robinson III and a dozen others, Hill filed
dozens of lawsuits over the state. They won over \$50 million in
improvements for black students and teachers. Hill believed schools
could be the crux for desegregation, but he was also a realist,
acknowledging that Southern techniques for "getting along" among races
both caused whites to believe most blacks preferred segregation, when in
fact they bitterly opposed it.

An early case in the Virginia Supreme Court won equal transportation for
black school children. In 1951, the team took up the cause of the
African-American students at the segregated R.R. Moton High School in
Farmville who had walked out of their dilapidated school. The subsequent
lawsuit, Davis v. County School Board of Prince Edward County, later
became one of the five cases decided under Brown v. Board of Education
before the Supreme Court of the United States in 1954.

During the 1940s and 1950s, the safety of Hill's life and family were
often threatened as a result of his legal work. Crank calls (many with
threats) came all through the day and night until the family learned to
take their telephone off the hook at night-time (much to the telephone
company's displeasure, but then it also refused to trace the crank and
threatening calls which had provoked that self-help). Hill's young son
was not allowed to answer the telephone, and at one point in 1955 a
cross was burned on the Hills' lawn.

Nonetheless, Hill and his clients continued their legal battles to
assert their civil rights. After the Brown decisions of 1954 and 1955,
Virginia's dominant Byrd Organization adopted a policy known as massive
resistance to avoid desegregation. A special legislative session in 1956
passed a legislative package known as the Stanley plan. This included
two special legislative committees with enhanced powers, and which came
to harass the NAACP. It also permitted the governor (then Thomas B.
Stanley, followed by J. Lindsay Almond Jr.)to close schools which
desegregated, as well as provided tuition grant support of segregation
academies set up to avoid the extant public schools. In 1959, after
public schools had been closed in several localities, notably Prince
Edward Public Schools, Norfolk Public Schools and Warren County Public
Schools, the Virginia Supreme Court and a federal 3-judge panel on
January 19, 1959, finally ruled most of the Stanley plan and Virginia's
law prohibiting integrated public schools unconstitutional. Not long
after, Governor Almond abruptly dropped "Massive Resistance" as an
official state policy; schools in Norfolk and Arlington integrated
peacefully on February 5, 1959, and the schools in Front Royal and all
locations except in Prince Edward County reopened. Nonetheless, Prince
Edward County schools only reopened in 1964 after the U.S. Supreme
Court's decision in Griffin v. County School Board of Prince Edward
County.

After Massive Resistance collapsed, Hill accepted a job with the Federal
Housing Administration, putting his private legal practice on hold for
five years, but working to desegregate public housing nationwide.

The Virginia NAACP's efforts continued, and Hill returned to his
Virginia practice and leadership of the state's legal staff in 1965,
after passage of the Civil Rights Act of 1964 and the Voting Rights Act
of 1965. Those statutes, implementation of new U.S. Department of Health
Education and Welfare regulations (which provided additional funds for
districts complying with the racial desegregation mandate), as well as
Green v. School Board of New Kent County (1968) finally tipped the
balance toward integrated Virginia's public schools. Green, the crucial
ruling against freedom of choice plans was argued by Hill's law partner
Samuel W. Tucker, supported by a young lawyer Hill had recruited, Henry
L. Marsh, III.

He continued civil rights litigation as a partner of Hill, Tucker and
Marsh in Richmond until he retired in 1998. One of the last partners he
brought into the firm, Clarence Dunnaville had worked with him in his
youth on the school desegregation cases and continued his work through
the Oliver Hill Foundation, which seeks to reuse Hill's former home in
Roanoke to provide legal services to the poor through third year
students at the Washington and Lee School of Law. The firm closed in
late 2015 after Henry Marsh III decided to focus on his duties in the
General Assembly.

\section{Death and legacy}\label{death-and-legacy}

\begin{itemize}
\item
  \emph{Hill spent some of his final years working on his autobiography
  with Professor Jonathan K. Stubbs.The Big Bang: Brown v. Board of
  Education, The Autobiography of Oliver W. Hill, Sr.}
\item
  \emph{Oliver Hill outlived his beloved wife Bernie by more than a
  decade, and also outlived many of his contemporaries from civil rights
  struggles.}
\end{itemize}

Oliver Hill outlived his beloved wife Bernie by more than a decade, and
also outlived many of his contemporaries from civil rights struggles. He
mourned his partner and former Richmond judge, Harold M. Marsh Sr.,
gunned down in 1997 while stopped at a traffic light a half mile from
the courthouse that would soon bear his and his brothers' names by a
tenant behind in rent and facing eviction. Hill spent some of his final
years working on his autobiography with Professor Jonathan K. Stubbs.The
Big Bang: Brown v. Board of Education, The Autobiography of Oliver W.
Hill, Sr. It was published in 2000 and reprinted for his 100th birthday
in 2007. Hill also gave an oral history interview to Virginia
Commonwealth University scholars in 2002. In January 2004, he was a
featured panelist during Howard University's celebration of the lawyers
who contributed to the Brown decision, on its 50th anniversary.

On Sunday, August 5, 2007, Oliver Hill died peacefully during breakfast
at his home in Richmond of natural causes at the age of 100 years. Later
that day, Virginia Governor Tim Kaine ordered flags to be flown at half
mast to honor Hill and issued a statement:

More than 1200 people viewed his body as it rested in the Executive
Mansion before his funeral at the Greater Richmond Convention Center,
near where his law office had stood for decades. He is survived by his
son, Oliver Hill Jr., professor of psychology at Virginia State
University and executive director of its research foundation. He is
buried in Richmond's Forest Lawn cemetery. His papers are held by
Virginia State University, and awaiting processing.

\section{Lifetime honors}\label{lifetime-honors}

\begin{itemize}
\item
  \emph{In 1996, Richmond's Oliver Hill Courts Building, housing the
  Juvenile and Domestic Relations courtrooms, was named for him, and
  each September remembers Hill.}
\item
  \emph{Beginning in 2002, the Virginia State Bar has awarded the Oliver
  White Hill Law School Pro Bono Award annually to one law student who
  demonstrates an exceptional commitment to public or community
  service.}
\end{itemize}

In 1959 the National Bar Association named Hill its Lawyer of the
Year.\\
In 1980 the NAACP awarded Hill its William Robert Ming Advocacy Award.\\
In 1986 the NAACP Legal Defense and Educational Fund accorded Hill its
Simple Justice Award.\\
In 1989, the Richmond Bar Association established the Hill-Tucker Public
Service Award\\
In 1993 the American Bar Association gave Hill its Justice Thurgood
Marshall Award.\\
In 1996, Richmond's Oliver Hill Courts Building, housing the Juvenile
and Domestic Relations courtrooms, was named for him, and each September
remembers Hill.\\
In 1999 President Bill Clinton awarded Hill the Presidential Medal of
Freedom.\\
In 2000, Hill received the American Bar Association Medal and the
National Bar Association Hero of the Law award. That September (2000),
Hill and other NAACP Legal Defense Fund lawyers received the Harvard
Medal of Freedom for their role in the landmark Brown v. Board of
Education decision.\\
Beginning in 2002, the Virginia State Bar has awarded the Oliver White
Hill Law School Pro Bono Award annually to one law student who
demonstrates an exceptional commitment to public or community service.\\
In 2003, a bronze bust of Hill was unveiled outside the Greater Richmond
Convention Center. The Black History Museum and Cultural Center of
Virginia also has another bust.\\
In 2005 Hill received the Spingarn Medal, the NAACP's highest honor.
That October (2005), Virginia Governor Mark R. Warner dedicated the
newly renovated Virginia Finance building in Virginia's Capitol Square
in Hill's honor. The Oliver W. Hill Building became the first
state-owned building as well as the first in Virginia's Capitol Square
to be named for an African American.

\section{Posthumous honors}\label{posthumous-honors}

\begin{itemize}
\item
  \emph{is now home to the Oliver Hill Mentoring Program, a cooperative
  effort of the Big Brothers Big Sisters of Southwest Virginia and
  Roanoke City Public Schools."}
\item
  \emph{In July 2008 the Virginia Civil Rights Memorial was dedicated in
  Capitol Square, honoring the legacy of Hill, his fellow attorneys and
  clients.}
\end{itemize}

In July 2008 the Virginia Civil Rights Memorial was dedicated in Capitol
Square, honoring the legacy of Hill, his fellow attorneys and clients.

The following year, the state's Department of Historic Resources
approved four plaques that honor Hill and his legacy: that by the
Greater Richmond Convention Center marks the former location of his law
office. The memorial in Norfolk recalls his first important legal
victory, Alston v. School Board of Norfolk (1940), as well as Beckett v.
Norfolk School Board (1957). The Prince Edward County marker
commemorates his victory in Davis v. School Board of Prince Edward
County. The Roanoke marker commemorates Hill's early years in the city
and early law practice.

A street in Richmond's Shockoe Bottom (a former slave trading
neighborhood) named "Oliver Hill Way" is now one of the proposed
boundaries of a redevelopment project.

The City of Roanoke, Virginia renamed its courthouse in his honor on May
1st, 2019. The City of Roanoke had previously dedicated a marker in
front of his childhood home. Accordng to the Roanoke Times, "His
childhood house at 401 Gilmer Ave. N.W. is now home to the Oliver Hill
Mentoring Program, a cooperative effort of the Big Brothers Big Sisters
of Southwest Virginia and Roanoke City Public Schools."
\textless{}ref\textgreater{}\textless{}www.roanoke.com/news/local/roanoke/roanoke-unveils-historical-marker-for-oliver-hill-/ref\textgreater{}

\section{References}\label{references}

\section{External links}\label{external-links}

\begin{itemize}
\item
  \emph{Howard University School of Law, Brown at 50 bios, Oliver Hill
  webpage}
\item
  \emph{Bond, Julian, Interview with Oliver W. Hill, Virginia Quarterly
  Review, Winter 2004.}
\item
  \emph{Oliver Hill's oral history video excerpts at The National
  Visionary Leadership Project}
\item
  \emph{"Civil Rights Lawyer Oliver Hill Dies at 100", NPR, August 6,
  2007}
\end{itemize}

Howard University School of Law, Brown at 50 bios, Oliver Hill webpage

"Civil Rights Lawyer Oliver Hill Dies at 100", NPR, August 6, 2007

Virginia Historical Society re Massive Resistance

"They Closed Our Schools," the story of Massive Resistance and the
closing of the Prince Edward County, Virginia public schools

Bond, Julian, Interview with Oliver W. Hill, Virginia Quarterly Review,
Winter 2004.

Oliver Hill's oral history video excerpts at The National Visionary
Leadership Project

Oliver Hill 2002 oral history video from the Voices of Freedom
Collection of the VCU Libraries

Edds, Margaret We Face The Dawn: Oliver Hill, Spottswood Robinson and
the Legal Team that Dismantled Jim Crow University of Virginia Press,
2018
