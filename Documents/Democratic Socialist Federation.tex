\textbf{From Wikipedia, the free encyclopedia}

https://en.wikipedia.org/wiki/Democratic\%20Socialist\%20Federation\\
Licensed under CC BY-SA 3.0:\\
https://en.wikipedia.org/wiki/Wikipedia:Text\_of\_Creative\_Commons\_Attribution-ShareAlike\_3.0\_Unported\_License

\section{Democratic Socialist
Federation}\label{democratic-socialist-federation}

\begin{itemize}
\item
  \emph{In December of that year, the Socialist Party--Social Democratic
  Federation voted to change its name to Social Democrats, USA.}
\item
  \emph{The Democratic Socialist Federation was founded by members of
  the Social Democratic Federation who had opposed the latter's 1956
  reunification with the Socialist Party of America in 1956.}
\item
  \emph{The Federation merged with the Socialist Party in March 1972.}
\end{itemize}

The Democratic Socialist Federation was founded by members of the Social
Democratic Federation who had opposed the latter's 1956 reunification
with the Socialist Party of America in 1956.

The Federation merged with the Socialist Party in March 1972. In
December of that year, the Socialist Party--Social Democratic Federation
voted to change its name to Social Democrats, USA.

\section{Convention of December 1972}\label{convention-of-december-1972}

\begin{itemize}
\item
  \emph{Changing the name of the Socialist Party to "Social Democrats
  USA" was intended to be "realistic."}
\item
  \emph{The Party changed its name to "Social Democrats,~USA" by a vote
  of 73 to~34.}
\item
  \emph{Because the Socialist Party no longer sponsored candidates in
  Presidential elections, continued use of the name "Party" was
  "misleading" and hindered the recruiting of activists who participated
  in the Democratic Party, according to the majority report.}
\end{itemize}

In its 1972 Convention, the Socialist Party had two Co-Chairmen, Bayard
Rustin and Charles~S. Zimmerman (of the International Ladies
Garment~Workers' Union, ILGWU) and a First National Vice Chairman,
James~S. Glaser, who were re-elected by acclamation.

The Party changed its name to "Social Democrats,~USA" by a vote of 73
to~34. Changing the name of the Socialist Party to "Social Democrats
USA" was intended to be "realistic." The New York Times observed that
the Socialist Party had last sponsored a candidate for President in
1956, who received only 2,121 votes, which were cast in only six states.
Because the Socialist Party no longer sponsored candidates in
Presidential elections, continued use of the name "Party" was
"misleading" and hindered the recruiting of activists who participated
in the Democratic Party, according to the majority report. The name
"Socialist" was replaced by "Social Democrats" because many American
associated the word "socialism" with Soviet communism. Moreover, the
organization sought to distinguish itself from two small Marxist
parties.

\section{References}\label{references}

\section{Further reading}\label{further-reading}

\begin{itemize}
\item
  \emph{The 1972 reunification of the Democratic Socialist Federation
  with the Socialist Party of America is discussed in Maurice Isserman,
  The Other American: The Life of Michael Harrington (New York:
  PublicAffairs, 2000).}
\end{itemize}

The 1972 reunification of the Democratic Socialist Federation with the
Socialist Party of America is discussed in Maurice Isserman, The Other
American: The Life of Michael Harrington (New York: PublicAffairs,
2000). ISBN~1-891620-30-4.
