\textbf{From Wikipedia, the free encyclopedia}

https://en.wikipedia.org/wiki/Science\%20Mission\%20Directorate\\
Licensed under CC BY-SA 3.0:\\
https://en.wikipedia.org/wiki/Wikipedia:Text\_of\_Creative\_Commons\_Attribution-ShareAlike\_3.0\_Unported\_License

\section{Science Mission Directorate}\label{science-mission-directorate}

\begin{itemize}
\item
  \emph{The Science Mission Directorate (SMD) of the National
  Aeronautics and Space Administration (NASA) engages the United States'
  science community, sponsors scientific research, and develops and
  deploys satellites and probes in collaboration with NASA's partners
  around the world to answer fundamental questions requiring the view
  from and into space.}
\item
  \emph{The SMD portfolio is contributing to NASA's achievement of the
  Vision for Space Exploration by striving to:}
\item
  \emph{Explore the solar system for scientific purposes while
  supporting safe robotic and human exploration of space.}
\end{itemize}

The Science Mission Directorate (SMD) of the National Aeronautics and
Space Administration (NASA) engages the United States' science
community, sponsors scientific research, and develops and deploys
satellites and probes in collaboration with NASA's partners around the
world to answer fundamental questions requiring the view from and into
space.

The Science Mission Directorate also sponsors research that both
enables, and is enabled by, NASA's exploration activities. The SMD
portfolio is contributing to NASA's achievement of the Vision for Space
Exploration by striving to:

Understand the history of Mars and the formation of the solar system. By
understanding the formation of diverse terrestrial planets (with
atmospheres) in the solar system, researchers learn more about Earth's
future and the most promising opportunities for habitation beyond our
planet. For example, differences in the impacts of collisional processes
on Earth, the Moon, and Mars can provide clues about differences in
origin and evolution of each of these bodies.

Search for Earth-like planets and habitable environments around other
stars. SMD pursues multiple research strategies with the goal of
developing effective astronomically-detectable signatures of biological
processes. The study of the Earth-Sun system may help researchers
identify atmospheric biosignatures that distinguish Earth-like (and
potentially habitable) planets around nearby stars. An understanding of
the origin of life and the time evolution of the atmosphere on Earth may
reveal likely signatures of life on extrasolar planets.

Explore the solar system for scientific purposes while supporting safe
robotic and human exploration of space. For example, large-scale coronal
mass ejections from the Sun can cause potentially lethal consequences
for improperly shielded human flight systems, as well as some types of
robotic systems. SMD's pursuit of interdisciplinary scientific research
focus areas will help predict potentially harmful conditions in space
and protect NASA's robotic and human explorers.

\section{Leadership}\label{leadership}

\begin{itemize}
\item
  \emph{Earth Science Division Director: Michael Freilich}
\item
  \emph{Astrophysics Division Director: Paul Hertz}
\item
  \emph{Thomas Zurbuchen is the Associate Administrator for the Science
  Mission Directorate beginning October 3, 2016.}
\item
  \emph{Planetary Science Division Director: James L. Green}
\item
  \emph{Heliophysics Division Director: Steven W. Clarke}
\end{itemize}

Thomas Zurbuchen is the Associate Administrator for the Science Mission
Directorate beginning October 3, 2016. Recent Associate Administrators
for the SMD include Edward J. Weiler (1998--2004, 2008--2011), Mary L.
Cleave (2004--2005), Alan Stern (2007--2008) and John M. Grunsfeld
(2012-2016). Stern resigned 25 March 2008, to be effective 11 April,
over disagreements with Administrator Michael D. Griffin.

Associate Administrator: Thomas Zurbuchen

Deputy Associate Administrator: Geoffery L. Yoder

Heliophysics Division Director: Steven W. Clarke

Earth Science Division Director: Michael Freilich

Planetary Science Division Director: James L. Green

Astrophysics Division Director: Paul Hertz

Strategic Integration and Management Division Director: Dan
Woods{[}permanent dead link{]}

Resource Management Division Director: Craig Tupper

\section{References}\label{references}

\section{External links}\label{external-links}

\begin{itemize}
\item
  \emph{About the Science Mission Directorate}
\end{itemize}

About the Science Mission Directorate
