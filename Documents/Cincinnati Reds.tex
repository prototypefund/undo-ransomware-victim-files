\textbf{From Wikipedia, the free encyclopedia}

https://en.wikipedia.org/wiki/Cincinnati\%20Reds\\
Licensed under CC BY-SA 3.0:\\
https://en.wikipedia.org/wiki/Wikipedia:Text\_of\_Creative\_Commons\_Attribution-ShareAlike\_3.0\_Unported\_License

\section{Cincinnati Reds}\label{cincinnati-reds}

\begin{itemize}
\item
  \emph{The Reds compete in Major League Baseball (MLB) as a member club
  of the National League (NL) Central division.}
\item
  \emph{The Reds played in the NL West division from 1969 to 1993,
  before joining the Central division in 1994.}
\item
  \emph{The Cincinnati Reds are an American professional baseball team
  based in Cincinnati, Ohio.}
\item
  \emph{For 1882--2018, the Reds' overall win-loss record is
  10,524--10,306 (a 0.505 winning percentage).}
\end{itemize}

The Cincinnati Reds are an American professional baseball team based in
Cincinnati, Ohio. The Reds compete in Major League Baseball (MLB) as a
member club of the National League (NL) Central division. They were a
charter member of the American Association in 1882 and joined the NL in
1890.

The Reds played in the NL West division from 1969 to 1993, before
joining the Central division in 1994. They have won five World Series
titles, nine NL pennants, one AA pennant, and 10 division titles. The
team plays its home games at Great American Ball Park, which opened in
2003 replacing Riverfront Stadium. Bob Castellini has been chief
executive officer since 2006.

For 1882--2018, the Reds' overall win-loss record is 10,524--10,306 (a
0.505 winning percentage).

\section{Franchise history}\label{franchise-history}

\section{The birth of the Reds and the American Association
(1881--1889)}\label{the-birth-of-the-reds-and-the-american-association-18811889}

\begin{itemize}
\item
  \emph{The Reds' first game was a 12--3 victory over the St. Louis
  club.}
\item
  \emph{The origins of the modern Cincinnati Reds can be traced to the
  expulsion of an earlier team bearing that name.}
\item
  \emph{Cincinnati's expulsion from the National League incensed
  Cincinnati Enquirer sports editor O. P. Caylor, who made two attempts
  to form a new league on behalf of the receivers for the now bankrupt
  Reds franchise.}
\end{itemize}

The origins of the modern Cincinnati Reds can be traced to the expulsion
of an earlier team bearing that name. In 1876, Cincinnati became one of
the charter members of the new National League, but the club ran afoul
of league organizer and long-time president William Hulbert for selling
beer during games and renting out their ballpark on Sundays. Both were
important activities to entice the city's large German population. While
Hulbert made clear his distaste for both beer and Sunday baseball at the
founding of the league, neither practice was actually against league
rules in those early years. On October 6, 1880, however, seven of the
eight team owners pledged at a special league meeting to formally ban
both beer and Sunday baseball at the regular league meeting that
December. Only Cincinnati president W. H. Kennett refused to sign the
pledge, so the other owners formally expelled Cincinnati for violating a
rule that would not actually go into effect for two more months.

Cincinnati's expulsion from the National League incensed Cincinnati
Enquirer sports editor O. P. Caylor, who made two attempts to form a new
league on behalf of the receivers for the now bankrupt Reds franchise.
When these attempts failed, he formed a new independent ballclub known
as the Red Stockings in the Spring of 1881, and brought the team to St.
Louis for a weekend exhibition. The Reds' first game was a 12--3 victory
over the St. Louis club. After the 1881 series proved a success, Caylor
and a former president of the old Reds named Justus Thorner received an
invitation from Philadelphia businessman Horace Phillips to attend a
meeting of several clubs in Pittsburgh with the intent of establishing a
rival to the National League. Upon arriving in the city, however, Caylor
and Thorner discovered that no other owners had decided to accept the
invitation, with even Phillips not bothering to attend his own meeting.
By chance, the duo met a former pitcher named Al Pratt, who hooked them
up with former Pittsburgh Alleghenys president H. Denny McKnight.
Together, the three men hatched a scheme to form a new league by sending
a telegram to each of the other owners who were supposed to attend the
meeting stating that he was the only person who did not attend and that
everyone else was enthusiastic about the new venture and eager to attend
a second meeting in Cincinnati. The ploy worked, and the American
Association was officially formed at the Hotel Gibson in Cincinnati with
the new Reds a charter member with Thorner as president.

Led by the hitting of third baseman Hick Carpenter, the defense of
future Hall of Fame second baseman Bid McPhee, and the pitching of
40-game-winner Will White, the Reds won the inaugural AA pennant in
1882. With the establishment of the Union Association Justus Thorner
left the club to finance the Cincinnati Outlaw Reds and managed to
acquire the lease on the Reds Bank Street Grounds playing field, forcing
new president Aaron Stern to relocate three blocks away at the hastily
built League Park. The club never placed higher than second or lower
than fifth for the rest of its tenure in the American Association.

\includegraphics[width=5.50000in,height=3.69153in]{media/image1.jpg}\\
\emph{Cincinnati Reds baseball team in 1909}

\section{The National League returns to Cincinnati
(1890--1911)}\label{the-national-league-returns-to-cincinnati-18901911}

\begin{itemize}
\item
  \emph{Like the previous decade, the 1900s (decade) were not kind to
  the Reds, as much of the decade was spent in the league's second
  division.}
\item
  \emph{It was also at this time that the team first shortened their
  name from "Red Stockings" to "Reds".}
\item
  \emph{The Reds wandered through the 1890s signing local stars and
  aging veterans.}
\end{itemize}

The Cincinnati Red Stockings left the American Association on November
14, 1889 and joined the National League along with the Brooklyn
Bridegrooms after a dispute with St. Louis Browns owner Chris Von Der
Ahe over the selection of a new league president. The National League
was happy to accept the teams in part due to the emergence of the new
Player's League. This new league, an early failed attempt to break the
reserve clause in baseball, threatened both existing leagues. Because
the National League decided to expand while the American Association was
weakening, the team accepted an invitation to join the National League.
It was also at this time that the team first shortened their name from
"Red Stockings" to "Reds". The Reds wandered through the 1890s signing
local stars and aging veterans. During this time, the team never
finished above third place (1897) and never closer than 10​1⁄2 games
(1890).

At the start of the 20th century, the Reds had hitting stars Sam
Crawford and Cy Seymour. Seymour's .377 average in 1905 was the first
individual batting crown won by a Red. In 1911, Bob Bescher stole 81
bases, which is still a team record. Like the previous decade, the 1900s
(decade) were not kind to the Reds, as much of the decade was spent in
the league's second division.

\includegraphics[width=3.08070in,height=5.50000in]{media/image2.jpg}\\
\emph{Hall of famer Edd Roush led Cincinnati to the 1919 World Series.}

\section{Redland Field to the Great Depression
(1912--1932)}\label{redland-field-to-the-great-depression-19121932}

\begin{itemize}
\item
  \emph{The 1918 team finished fourth, and new manager Pat Moran led the
  Reds to an NL pennant in 1919, in what the club advertised as its
  "Golden Anniversary".}
\item
  \emph{After 1926, and well into the 1930s, the Reds were second
  division dwellers.}
\item
  \emph{By the late 1910s the Reds began to come out of the second
  division.}
\item
  \emph{By 1920, the "Black Sox" scandal had brought a taint to the
  Reds' first championship.}
\end{itemize}

In 1912, the club opened a new steel-and-concrete ballpark, Redland
Field (later to be known as Crosley Field). The Reds had been playing
baseball on that same site, the corner of Findlay and Western Avenues on
the city's west side, for 28 years, in wooden structures that had been
occasionally damaged by fires. By the late 1910s the Reds began to come
out of the second division. The 1918 team finished fourth, and new
manager Pat Moran led the Reds to an NL pennant in 1919, in what the
club advertised as its "Golden Anniversary". The 1919 team had hitting
stars Edd Roush and Heinie Groh while the pitching staff was led by Hod
Eller and left-hander Harry "Slim" Sallee. The Reds finished ahead of
John McGraw's New York Giants, and then won the world championship in
eight games over the Chicago White Sox.

By 1920, the "Black Sox" scandal had brought a taint to the Reds' first
championship. After 1926, and well into the 1930s, the Reds were second
division dwellers. Eppa Rixey, Dolf Luque and Pete Donohue were pitching
stars, but the offense never lived up to the pitching. By 1931, the team
was bankrupt, the Great Depression was in full swing and Redland Field
was in a state of disrepair.

\section{Championship baseball and revival
(1933--1940)}\label{championship-baseball-and-revival-19331940}

\begin{itemize}
\item
  \emph{By 1939, they were National League champions, but in the World
  Series, they were swept by the New York Yankees.}
\item
  \emph{The Reds, throughout the 1930s, became a team of "firsts".}
\item
  \emph{By 1938 the Reds, now led by manager Bill McKechnie, were out of
  the second division finishing fourth.}
\item
  \emph{MacPhail began to develop the Reds' minor league system and
  expanded the Reds' fan base.}
\end{itemize}

Powel Crosley, Jr., an electronics magnate who, with his brother Lewis
M. Crosley, produced radios, refrigerators, and other household items,
bought the Reds out of bankruptcy in 1933, and hired Larry MacPhail to
be the General Manager. Crosley had started WLW radio, the Reds flagship
radio broadcaster, and the Crosley Broadcasting Corporation in
Cincinnati, where he was also a prominent civic leader. MacPhail began
to develop the Reds' minor league system and expanded the Reds' fan
base. The Reds, throughout the 1930s, became a team of "firsts". The
now-renamed Crosley Field became the host of the first night game in
1935, which was also the first baseball fireworks night, the fireworks
at the game were shot by Joe Rozzi of Rozzi's Famous Fireworks. Johnny
Vander Meer became the only pitcher in major league history to throw
back-to-back no-hitters in 1938. Thanks to Vander Meer, Paul Derringer
and second baseman/third baseman-turned-pitcher Bucky Walters, the Reds
had a solid pitching staff. The offense came around in the late 1930s.
By 1938 the Reds, now led by manager Bill McKechnie, were out of the
second division finishing fourth. Ernie Lombardi was named the National
League's Most Valuable Player in 1938. By 1939, they were National
League champions, but in the World Series, they were swept by the New
York Yankees. In 1940, they repeated as NL Champions, and for the first
time in 21 years, the Reds captured a World championship, beating the
Detroit Tigers 4 games to 3. Frank McCormick was the 1940 NL MVP. Other
position players included Harry Craft, Lonny Frey, Ival Goodman, Lew
Riggs and Bill Werber.

\section{1941--1969}\label{section}

\begin{itemize}
\item
  \emph{The Reds' rules also included conservative uniforms.}
\item
  \emph{The Reds did not recover from this trade until the rise of the
  "Big Red Machine" of the 1970s.}
\item
  \emph{World War II and age finally caught up with the Reds.}
\item
  \emph{The Rosie Reds are still in existence, and are currently the
  oldest fan club in Major League Baseball.}
\end{itemize}

World War II and age finally caught up with the Reds. Throughout the
1940s and early 1950s, Cincinnati finished mostly in the second
division. In 1944, Joe Nuxhall (who was later to become part of the
radio broadcasting team), at age 15, pitched for the Reds on loan from
Wilson Junior High school in Hamilton, Ohio. He became the youngest
player ever to appear in a major league game---a record that still
stands today. Ewell "The Whip" Blackwell was the main pitching stalwart
before arm problems cut short his career. Ted Kluszewski was the NL home
run leader in 1954. The rest of the offense was a collection of
over-the-hill players and not-ready-for-prime-time youngsters.

In April 1953, the Reds announced a preference to be called the
"Redlegs", saying that the name of the club had been "Red Stockings" and
then "Redlegs". A newspaper speculated that it was due to the developing
political connotation of the word 'red' to mean Communism. From 1956 to
1960, the club's logo was altered to remove the term "REDS" from the
inside of the "wishbone C" symbol. The "REDS" reappeared on the 1961
uniforms, but the point of the C was removed, leaving a smooth,
non-wishbone curve. The traditional home-uniform logo was restored in
1967.

In 1956, led by National League Rookie of the Year Frank Robinson, the
Redlegs hit 221 HR to tie the NL record. By 1961, Robinson was joined by
Vada Pinson, Wally Post, Gordy Coleman, and Gene Freese. Pitchers Joey
Jay, Jim O'Toole, and Bob Purkey led the staff.

The Reds captured the 1961 National League pennant, holding off the Los
Angeles Dodgers and the San Francisco Giants, only to be defeated by the
perennially powerful New York Yankees in the World Series.

The Reds had winning teams during the rest of the 1960s, but did not
produce any championships. They won 98 games in 1962, paced by Purkey's
23, but finished third. In 1964, they lost the pennant by one game to
the Cardinals after having taken first place when the Phillies collapsed
in September. Their beloved manager Fred Hutchinson died of cancer just
weeks after the end of the 1964 season. The failure of the Reds to win
the 1964 pennant led to owner Bill DeWitt's selling off key components
of the team, in anticipation of relocating the franchise. In response to
DeWitt's threatened move, the women of Cincinnati banded together to
form the Rosie Reds to urge DeWitt to keep the franchise in Cincinnati.
The Rosie Reds are still in existence, and are currently the oldest fan
club in Major League Baseball. After the 1965 season he executed what
may be the most lopsided trade in baseball history, sending former Most
Valuable Player Frank Robinson to the Baltimore Orioles for pitchers
Milt Pappas and Jack Baldschun, and outfielder Dick Simpson. Robinson
went on to win the MVP and triple crown in the American league for 1966,
and lead Baltimore to its first ever World Series title in a sweep of
the Los Angeles Dodgers. The Reds did not recover from this trade until
the rise of the "Big Red Machine" of the 1970s.

Starting in the early 1960s, the Reds' farm system began producing a
series of stars, including Jim Maloney (the Reds' pitching ace of the
1960s), Pete Rose, Tony Pérez, Johnny Bench, Lee May, Tommy Helms,
Bernie Carbo, Hal McRae, Dave Concepción, and Gary Nolan. The tipping
point came in 1967 with the appointment of Bob Howsam as general
manager. That same year the Reds avoided a move to San Diego when the
city of Cincinnati and Hamilton County agreed to build a state of the
art, downtown stadium on the edge of the Ohio River. The Reds entered
into a 30-year lease in exchange for the stadium commitment keeping the
franchise in its original home city. In a series of strategic moves,
Howsam brought in key personnel to complement the homegrown talent. The
Reds' final game at Crosley Field, home to more than 4,500 baseball
games, was played on June 24, 1970, a 5--4 victory over the San
Francisco Giants.

Under Howsam's administration starting in the late 1960s, the Reds
instituted a strict rule barring the team's players from wearing facial
hair and long hair. The clean cut look was meant to present the team as
wholesome in an era of turmoil. All players coming to the Reds were
required to shave and cut their hair for the next three decades. Over
the years, the rule was controversial, but persisted well into the
ownership of Marge Schott. On at least one occasion, in the early 1980s,
enforcement of this rule lost them the services of star reliever and
Ohio native Rollie Fingers, who would not shave his trademark handlebar
mustache in order to join the team. The rule was not officially
rescinded until 1999 when the Reds traded for slugger Greg Vaughn, who
had a goatee. The New York Yankees continue to have a similar rule
today, though unlike the Reds during this period, Yankees players are
permitted to have mustaches. Much like when players leave the Yankees
today, players who left the Reds took advantage with their new teams;
Pete Rose, for instance, grew his hair out much longer than would be
allowed by the Reds once he signed with the Philadelphia Phillies in
1979.

The Reds' rules also included conservative uniforms. In Major League
Baseball, a club generally provides most of the equipment and clothing
needed for play. However, players are required to supply their gloves
and shoes themselves. Many players enter into sponsorship arrangements
with shoe manufacturers, but through the mid-1980s, the Reds had a
strict rule that players were to wear only plain black shoes with no
prominent logo. Reds players decried what they considered to be the
boring color choice as well as the denial of the opportunity to earn
more money through shoe contracts. A compromise was struck in 1985 in
which players could paint red marks on their black shoes, then the
following year, they were allowed to wear all-red shoes.

\section{The Big Red Machine
(1970--1976)}\label{the-big-red-machine-19701976}

\begin{itemize}
\item
  \emph{So far in MLB history, the 1975 and '76 Reds were the last NL
  team to repeat as champions.}
\item
  \emph{The Reds won the NL West by ten games.}
\item
  \emph{After splitting the first four games, the Reds took Game 5.}
\item
  \emph{The 1972 Reds won the NL West in baseball's first ever
  strike-shortened season and defeated the Pittsburgh Pirates in an
  exciting five-game playoff series.}
\end{itemize}

In 1970, little known George "Sparky" Anderson was hired as manager, and
the Reds embarked upon a decade of excellence, with a team that came to
be known as "The Big Red Machine". Playing at Crosley Field until June
30, 1970, when the Reds moved into brand-new Riverfront Stadium, a
52,000 seat multi-purpose venue on the shores of the Ohio River, the
Reds began the 1970s with a bang by winning 70 of their first 100 games.
Johnny Bench, Tony Pérez, Pete Rose, Lee May and Bobby Tolan were the
early Red Machine offensive leaders; Gary Nolan, Jim Merritt, Wayne
Simpson and Jim McGlothlin led a pitching staff which also contained
veterans Tony Cloninger and Clay Carroll and youngsters Pedro Borbón and
Don Gullett. The Reds breezed through the 1970 season, winning the NL
West and captured the NL pennant by sweeping the Pittsburgh Pirates in
three games. By the time the club got to the World Series, however, the
Reds pitching staff had run out of gas and the veteran Baltimore
Orioles, led by Hall of Fame third baseman and World Series MVP Brooks
Robinson, beat the Reds in five games.

After the disastrous 1971 season (the only season of the 1970s during
which the Reds finished with a losing record) the Reds reloaded by
trading veterans Jimmy Stewart, May, and Tommy Helms for Joe Morgan,
César Gerónimo, Jack Billingham, Ed Armbrister, and Denis Menke.
Meanwhile, Dave Concepción blossomed at shortstop. 1971 was also the
year a key component of the future world championships was acquired in
George Foster from the San Francisco Giants in a trade for shortstop
Frank Duffy.

The 1972 Reds won the NL West in baseball's first ever strike-shortened
season and defeated the Pittsburgh Pirates in an exciting five-game
playoff series. They then faced the Oakland Athletics in the World
Series. Six of the seven games were won by one run. With powerful
slugger Reggie Jackson sidelined by an injury incurred during Oakland's
playoff series, Ohio native Gene Tenace got a chance to play in the
series, delivering four home runs that tied the World Series record for
homers, propelling Oakland to a dramatic seven-game series win. This was
one of the few World Series in which no starting pitcher for either side
pitched a complete game.

The Reds won a third NL West crown in 1973 after a dramatic second half
comeback, that saw them make up ​10~1⁄2 games on the Los Angeles Dodgers
after the All-Star break. However they lost the NL pennant to the New
York Mets in five games in the NLCS. In game one, Tom Seaver faced Jack
Billingham in a classic pitching duel, with all three runs of the 2--1
margin being scored on home runs. John Milner provided New York's run
off Billingham, while Pete Rose tied the game in the seventh inning off
Seaver, setting the stage for a dramatic game ending home run by Johnny
Bench in the bottom of the ninth. The New York series provided plenty of
controversy with the riotous behavior of Shea Stadium fans towards Pete
Rose when he and Bud Harrelson scuffled after a hard slide by Rose into
Harrelson at second base during the fifth inning of Game 3. A full
bench-clearing fight resulted after Harrelson responded to Rose's
aggressive move to prevent him from completing a double play by calling
him a name. This also led to two more incidents in which play was
stopped. The Reds trailed 9--3 and New York's manager, Yogi Berra, and
legendary outfielder Willie Mays, at the request of National League
president Warren Giles, appealed to fans in left field to restrain
themselves. The next day the series was extended to a fifth game when
Rose homered in the 12th inning to tie the series at two games each.

The Reds won 98 games in 1974 but they finished second to the 102-win
Los Angeles Dodgers. The 1974 season started off with much excitement,
as the Atlanta Braves were in town to open the season with the Reds.
Hank Aaron entered opening day with 713 home runs, one shy of tying Babe
Ruth's record of 714. The first pitch Aaron swung at in the 1974 season
was the record tying home run off Jack Billingham. The next day the
Braves benched Aaron, hoping to save him for his record-breaking home
run on their season-opening homestand. The commissioner of baseball,
Bowie Kuhn, ordered Braves management to play Aaron the next day, where
he narrowly missed the historic home run in the fifth inning. Aaron went
on to set the record in Atlanta two nights later. The 1974 season was
also the debut of Hall of Fame radio announcer Marty Brennaman, who
replaced Al Michaels, after Michaels left the Reds to broadcast for the
San Francisco Giants.

With 1975, the Big Red Machine lineup solidified with the "Great Eight"
starting team of Johnny Bench (catcher), Tony Pérez (first base), Joe
Morgan (second base), Dave Concepción (shortstop), Pete Rose (third
base), Ken Griffey (right field), César Gerónimo (center field), and
George Foster (left field). The starting pitchers included Don Gullett,
Fred Norman, Gary Nolan, Jack Billingham, Pat Darcy, and Clay Kirby. The
bullpen featured Rawly Eastwick and Will McEnaney combining for 37
saves, and veterans Pedro Borbón and Clay Carroll. On Opening Day, Rose
still played in left field, Foster was not a starter, while John
Vukovich, an off-season acquisition, was the starting third baseman.
While Vuckovich was a superb fielder, he was a weak hitter. In May, with
the team off to a slow start and trailing the Dodgers, Sparky Anderson
made a bold move by moving Rose to third base, a position where he had
very little experience, and inserting Foster in left field. This was the
jolt that the Reds needed to propel them into first place, with Rose
proving to be reliable on defense, while adding Foster to the outfield
gave the offense some added punch. During the season, the Reds compiled
two notable streaks: (1) by winning 41 out of 50 games in one stretch,
and (2) by going a month without committing any errors on defense.

In the 1975 season, Cincinnati clinched the NL West with 108 victories,
then swept the Pittsburgh Pirates in three games to win the NL pennant.
In the World Series, the Boston Red Sox were the opponents. After
splitting the first four games, the Reds took Game 5. After a three-day
rain delay, the two teams met in Game 6, one of the most memorable
baseball games ever played and considered by many to be the best World
Series game ever. The Reds were ahead 6--3 with 5 outs left, when the
Red Sox tied the game on former Red Bernie Carbo's three-run home run.
It was Carbo's second pinch-hit three-run homer in the series. After a
few close-calls either way, Carlton Fisk hit a dramatic 12th inning home
run off the foul pole in left field to give the Red Sox a 7--6 win and
force a deciding Game 7. Cincinnati prevailed the next day when Morgan's
RBI single won Game 7 and gave the Reds their first championship in 35
years. The Reds have not lost a World Series game since Carlton Fisk's
home run, a span of 9 straight wins.

1976 saw a return of the same starting eight in the field. The starting
rotation was again led by Nolan, Gullett, Billingham, and Norman, while
the addition of rookies Pat Zachry and Santo Alcalá comprised an
underrated staff in which four of the six had ERAs below 3.10. Eastwick,
Borbon, and McEnaney shared closer duties, recording 26, 8, and 7 saves
respectively. The Reds won the NL West by ten games. They went
undefeated in the postseason, sweeping the Philadelphia Phillies
(winning Game 3 in their final at-bat) to return to the World Series.
They continued to dominate by sweeping the Yankees in the newly
renovated Yankee Stadium, the first World Series games played in Yankee
Stadium since 1964. This was only the second ever sweep of the Yankees
in the World Series. In winning the Series, the Reds became the first NL
team since the 1921--22 New York Giants to win consecutive World Series
championships, and the Big Red Machine of 1975--76 is considered one of
the best teams ever. So far in MLB history, the 1975 and '76 Reds were
the last NL team to repeat as champions.

Beginning with the 1970 National League pennant, the Reds beat either of
the two Pennsylvania-based clubs, the Philadelphia Phillies or the
Pittsburgh Pirates to win their pennants (Pirates in 1970, 1972, 1975,
and 1990, Phillies in 1976), making The Big Red Machine part of the
rivalry between the two Pennsylvania teams. In 1979, Pete Rose added
further fuel in The Big Red Machine being part of the rivalry when he
signed with the Phillies and helped them win their first World Series
championship in 1980.

\section{The Machine dismantled
(1977--1989)}\label{the-machine-dismantled-19771989}

\begin{itemize}
\item
  \emph{In August 1984, Pete Rose was reacquired and hired to be the
  Reds player-manager.}
\item
  \emph{The Reds fell to the bottom of the Western Division for the next
  few years.}
\item
  \emph{The Reds also had a bullpen star in John Franco, who was with
  the team from 1984 to 1989.}
\item
  \emph{By 1982, the Reds were a shell of the original Red Machine; they
  lost 101 games that year.}
\end{itemize}

The later years of the 1970s brought turmoil and change. Popular Tony
Pérez was sent to Montreal after the 1976 season, breaking up the Big
Red Machine's starting lineup. Manager Sparky Anderson and General
Manager Bob Howsam later considered this trade the biggest mistake of
their careers. Starting pitcher Don Gullett left via free agency and
signed with the New York Yankees. In an effort to fill that gap, a trade
with the Oakland A's for starting ace Vida Blue was arranged during the
1976--77 off-season. However, Bowie Kuhn, the Commissioner of Baseball,
vetoed the trade for the stated reason of maintaining competitive
balance in baseball. Some have suggested that the actual reason had more
to due with Kuhn's continued feud with Oakland A's owner Charlie Finley.
On June 15, 1977, the Reds acquired Mets' franchise pitcher Tom Seaver
for Pat Zachry, Doug Flynn, Steve Henderson, and Dan Norman. In other
deals that proved to be less successful, the Reds traded Gary Nolan to
the Angels for Craig Hendrickson, Rawly Eastwick to St. Louis for Doug
Capilla and Mike Caldwell to Milwaukee for Rick O'Keeffe and Garry Pyka,
and got Rick Auerbach from Texas. The end of the Big Red Machine era was
heralded by the replacement of General Manager Bob Howsam with Dick
Wagner.

In Rose's last season as a Red, he gave baseball a thrill as he
challenged Joe DiMaggio's 56-game hitting streak, tying for the
second-longest streak ever at 44 games. The streak came to an end in
Atlanta after striking out in his fifth at bat in the game against Gene
Garber. Rose also earned his 3,000th hit that season, on his way to
becoming baseball's all-time hits leader when he rejoined the Reds in
the mid-1980s. The year also witnessed the only no-hitter of Hall of
Fame pitcher Tom Seaver's career, coming against the St. Louis Cardinals
on June 16, 1978.

After the 1978 season and two straight second-place finishes, Wagner
fired manager Anderson---an unpopular move. Pete Rose, who since 1963
had played almost every position for the team except pitcher, shortstop,
and catcher, signed with Philadelphia as a free agent. By 1979, the
starters were Bench (c), Dan Driessen (1b), Morgan (2b), Concepción
(ss), Ray Knight (3b), with Griffey, Foster, and Geronimo again in the
outfield. The pitching staff had experienced a complete turnover since
1976 except for Fred Norman. In addition to ace starter Tom Seaver; the
remaining starters were Mike LaCoss, Bill Bonham, and Paul Moskau. In
the bullpen, only Borbon had remained. Dave Tomlin and Mario Soto worked
middle relief with Tom Hume and Doug Bair closing. The Reds won the 1979
NL West behind the pitching of Tom Seaver but were dispatched in the NL
playoffs by Pittsburgh. Game 2 featured a controversial play in which a
ball hit by Pittsburgh's Phil Garner was caught by Cincinnati outfielder
Dave Collins but was ruled a trap, setting the Pirates up to take a 2--1
lead. The Pirates swept the series 3 games to 0 and went on to win the
World Series against the Baltimore Orioles.

The 1981 team fielded a strong lineup, but with only Concepción, Foster,
and Griffey retaining their spots from the 1975--76 heyday. After Johnny
Bench was able to play only a few games at catcher each year after 1980
due to ongoing injuries, Joe Nolan took over as starting catcher.
Driessen and Bench shared 1st base, and Knight starred at third. Morgan
and Geronimo had been replaced at second base and center field by Ron
Oester and Dave Collins. Mario Soto posted a banner year starting on the
mound, only surpassed by the outstanding performance of Seaver's Cy
Young runner-up season. La Coss, Bruce Berenyi, and Frank Pastore
rounded out the starting rotation. Hume again led the bullpen as closer,
joined by Bair and Joe Price. In 1981, Cincinnati had the best overall
record in baseball, but they finished second in the division in both of
the half-seasons that were created after a mid-season players' strike,
and missed the playoffs. To commemorate this, a team photo was taken,
accompanied by a banner that read "Baseball's Best Record 1981".

By 1982, the Reds were a shell of the original Red Machine; they lost
101 games that year. Johnny Bench, after an unsuccessful transition to
3rd base, retired a year later.

After the heartbreak of 1981, General Manager Dick Wagner pursued the
strategy of ridding the team of veterans including third-baseman Knight
and the entire starting outfield of Griffey, Foster, and Collins. Bench,
after being able to catch only seven games in 1981, was moved from
platooning at first base to be the starting third baseman; Alex Treviño
became the regular starting catcher. The outfield was staffed with Paul
Householder, César Cedeño, and future Colorado Rockies \& Pittsburgh
Pirates manager Clint Hurdle on opening day. Hurdle was an immediate
bust, and rookie Eddie Milner took his place in the starting outfield
early in the year. The highly touted Householder struggled throughout
the year despite extensive playing time. Cedeno, while providing steady
veteran play, was a disappointment, and was unable to recapture his
glory days with the Houston Astros. The starting rotation featured the
emergence of a dominant Mario Soto, and featured strong years by Pastore
and Bruce Berenyi, but Seaver was injured all year, and their efforts
were wasted without a strong offensive lineup. Tom Hume still led the
bullpen, along with Joe Price. But the colorful Brad "The Animal" Lesley
was unable to consistently excel, and former all-star Jim Kern was a big
disappointment. Kern was also publicly upset over having to shave off
his prominent beard to join the Reds, and helped force the issue of
getting traded during mid-season by growing it back. The season also saw
the midseason firing of Manager John McNamara, who was replaced as
skipper by Russ Nixon.

The Reds fell to the bottom of the Western Division for the next few
years. After the 1982 season, Seaver was traded back to the Mets. The
year 1983 found Dann Bilardello behind the plate, Bench returning to
part-time duty at first base, rookies Nick Esasky taking over at third
base and Gary Redus taking over from Cedeno. Tom Hume's effectiveness as
a closer had diminished, and no other consistent relievers emerged. Dave
Concepción was the sole remaining starter from the Big Red Machine era.

Wagner's tenure ended in 1983, when Howsam, the architect of the Big Red
Machine, was brought back. The popular Howsam began his second term as
Reds' General Manager by signing Cincinnati native Dave Parker as a free
agent from Pittsburgh. In 1984 the Reds began to move up, depending on
trades and some minor leaguers. In that season Dave Parker, Dave
Concepción and Tony Pérez were in Cincinnati uniforms. In August 1984,
Pete Rose was reacquired and hired to be the Reds player-manager. After
raising the franchise from the grave, Howsam gave way to the
administration of Bill Bergesch, who attempted to build the team around
a core of highly regarded young players in addition to veterans like
Parker. However, he was unable to capitalize on an excess of young and
highly touted position players including Kurt Stillwell, Tracy Jones,
and Kal Daniels by trading them for pitching. Despite the emergence of
Tom Browning as rookie of the year in 1985 when he won 20 games, the
rotation was devastated by the early demise of Mario Soto's career to
arm injury.

Under Bergesch, from 1985--89 the Reds finished second four times. Among
the highlights, Rose became the all-time hits leader, Tom Browning threw
a perfect game, Eric Davis became the first player in baseball history
to hit at least 35 home runs and steal 50 bases, and Chris Sabo was the
1988 National League Rookie of the Year. The Reds also had a bullpen
star in John Franco, who was with the team from 1984 to 1989. Rose once
had Concepción pitch late in a game at Dodger Stadium. Following the
release of the Dowd Report which accused Rose for betting on baseball
games, in 1989 Rose was banned from baseball by Commissioner Bart
Giamatti, who declared Rose guilty of "conduct detrimental to baseball".
Controversy also swirled around Reds owner Marge Schott, who was accused
several times of ethnic and racial slurs.

\section{World Championship and the end of an era
(1990--2002)}\label{world-championship-and-the-end-of-an-era-19902002}

\begin{itemize}
\item
  \emph{The Reds swept the heavily favored Oakland Athletics in four
  straight, and extended a Reds winning streak in the World Series to
  nine consecutive games.}
\item
  \emph{The Reds did not have another winning season until 2010.}
\item
  \emph{In winning the World Series the Reds became the only National
  League team to go wire to wire.}
\end{itemize}

In 1987, General Manager Bergesch was replaced by Murray Cook, who
initiated a series of deals that would finally bring the Reds back to
the championship, starting with acquisitions of Danny Jackson and José
Rijo. An aging Dave Parker was let go after a revival of his career in
Cincinnati following the Pittsburgh drug trials. Barry Larkin emerged as
the starting shortstop over Kurt Stillwell, who along with reliever
Power, was traded for Jackson. In 1989, Cook was succeeded by Bob Quinn,
who put the final pieces of the championship puzzle together, with the
acquisitions of Hal Morris, Billy Hatcher and Randy Myers.

In 1990, the Reds under new manager Lou Piniella shocked baseball by
leading the NL West from wire-to-wire. Winning their first nine games,
they started off 33--12 and maintained their lead throughout the year.
Led by Chris Sabo, Barry Larkin, Eric Davis, Paul O'Neill and Billy
Hatcher in the field, and by José Rijo, Tom Browning and the "Nasty
Boys" of Rob Dibble, Norm Charlton and Randy Myers on the mound, the
Reds took out the Pirates in the NLCS. The Reds swept the heavily
favored Oakland Athletics in four straight, and extended a Reds winning
streak in the World Series to nine consecutive games. The World Series,
however, saw Eric Davis severely bruise a kidney diving for a fly ball
in Game 4, and his play was greatly limited the next year. In winning
the World Series the Reds became the only National League team to go
wire to wire.

In 1992, Quinn was replaced in the front office by Jim Bowden. On the
field, manager Lou Piniella wanted outfielder Paul O'Neill to be a
power-hitter to fill the void Eric Davis left when he was traded to the
Los Angeles Dodgers in exchange for Tim Belcher. However, O'Neill only
hit .246 and 14 homers. The Reds returned to winning after a losing
season in 1991, but 90 wins was only enough for second place behind the
division-winning Atlanta Braves. Before the season ended, Piniella got
into an altercation with reliever Rob Dibble. In the off season, Paul
O'Neill was traded to the New York Yankees for outfielder Roberto Kelly.
Kelly was a disappointment for the Reds over the next couple of years,
while O'Neill blossomed, leading a down-trodden Yankees franchise to a
return to glory. Also, the Reds would replace their "Big Red Machine"
era uniforms in favor of a pinstriped uniform with no sleeves.

For the 1993 season Piniella was replaced by fan favorite Tony Pérez,
but he lasted only 44 games at the helm, replaced by Davey Johnson. With
Johnson steering the team, the Reds made steady progress. In 1994, the
Reds were in the newly created National League Central Division with the
Chicago Cubs, St. Louis Cardinals, as well as fellow rivals Pittsburgh
Pirates and Houston Astros. By the time the strike hit, the Reds
finished a half-game ahead of the Astros for first-place in the NL
Central. By 1995, the Reds won the division thanks to Most Valuable
Player Barry Larkin. After defeating the NL West champion Dodgers in the
first NLDS since 1981, they lost to the Atlanta Braves.

Team owner Marge Schott announced mid-season that Johnson would be gone
by the end of the year, regardless of outcome, to be replaced by former
Reds third baseman Ray Knight. Johnson and Schott had never gotten along
and she did not approve of Johnson living with his fiancée before they
were married, In contrast, Knight, along with his wife, professional
golfer Nancy Lopez, were friends of Schott. The team took a dive under
Knight and he was unable to complete two full seasons as manager,
subject to complaints in the press about his strict managerial style.

In 1999 the Reds won 96 games, led by manager Jack McKeon, but lost to
the New York Mets in a one game playoff. Earlier that year, Schott sold
controlling interest in the Reds to Cincinnati businessman Carl Lindner.
Despite an 85--77 finish in 2000, and being named 1999 NL manager of the
year, McKeon was fired after the 2000 season. The Reds did not have
another winning season until 2010.

\section{Contemporary era
(2003--present)}\label{contemporary-era-2003present}

\begin{itemize}
\item
  \emph{In 2016, the Reds broke the record for home runs allowed during
  a single season.}
\item
  \emph{Following the season Dan O'Brien was hired as the Reds' 16th
  General Manager.}
\item
  \emph{The Reds ended the season at 79-83.}
\item
  \emph{The Reds lost in a 3-game sweep of the NLDS to Philadelphia.}
\item
  \emph{The Reds won the 2012 NL Central Division Title.}
\end{itemize}

Riverfront Stadium, by then known as Cinergy Field, was demolished in
2002. Great American Ball Park opened in 2003 with high expectations for
a team led by local favorites, including outfielder Ken Griffey, Jr.,
shortstop Barry Larkin, and first baseman Sean Casey. Although
attendance improved considerably with the new ballpark, the team
continued to lose. Schott had not invested much in the farm system since
the early 1990s, leaving the team relatively thin on talent. After years
of promises that the club was rebuilding toward the opening of the new
ballpark, General Manager Jim Bowden and manager Bob Boone were fired on
July 28. This broke up the father-son combo of manager Bob Boone and
third baseman Aaron Boone, and Aaron was soon traded to the New York
Yankees. Tragedy struck in November when Dernell Stenson, a promising
young outfielder for the Reds, was shot and killed during a carjack.
Following the season Dan O'Brien was hired as the Reds' 16th General
Manager.{[}citation needed{]}

The 2004 and 2005 seasons continued the trend of big hitting, poor
pitching, and poor records. Griffey, Jr. joined the 500 home run club in
2004, but was again hampered by injuries. Adam Dunn emerged as
consistent home run hitter, including a 535-foot (163~m) home run
against José Lima. He also broke the major league record for strikeouts
in 2004. Although a number of free agents were signed before 2005, the
Reds were quickly in last place and manager Dave Miley was forced out in
the 2005 mid season and replaced by Jerry Narron. Like many other small
market clubs, the Reds dispatched some of their veteran players and
began entrusting their future to a young nucleus that included Adam Dunn
and Austin Kearns.

Late summer 2004 saw the opening of the Cincinnati Reds Hall of Fame
(HOF). The Reds HOF had been in existence in name only since the 1950s,
with player plaques, photos and other memorabilia scattered throughout
their front offices. Ownership and management desired a stand-alone
facility, where the public could walk through inter-active displays, see
locker room recreations, watch videos of classic Reds moments and peruse
historical items. The first floor houses a movie theater which resembles
an older, ivy-covered brick wall ball yard. The hallways contain many
vintage photographs. The rear of the building features a three-story
wall containing a baseball for every hit Pete Rose had during his
career. The third floor contains interactive exhibits including a
pitcher's mound, radio booth, and children's area where the fundamentals
of baseball are taught through videos featuring former Reds
players.{[}citation needed{]}

Robert Castellini took over as controlling owner from Lindner in 2006.
Castellini promptly fired general manager Dan O'Brien and hired Wayne
Krivsky. The Reds made a run at the playoffs but ultimately fell short.
The 2007 season was again mired in mediocrity. Midway through the season
Jerry Narron was fired as manager and replaced by Pete Mackanin. The
Reds ended up posting a winning record under Mackanin, but finished the
season in 5th place in the Central Division. Mackanin was manager in an
interim capacity only, and the Reds, seeking a big name to fill the
spot, ultimately brought in Dusty Baker. Early in the 2008 season,
Krivsky was fired and replaced by Walt Jocketty. Though the Reds did not
win under Krivsky, he is credited with revamping the farm system and
signing young talent that could potentially lead the Reds to success in
the future.

The Reds failed to post winning records in both 2008 and 2009. In 2010,
with NL MVP Joey Votto and Gold Glovers Brandon Phillips and Scott Rolen
the Reds posted a 91-71 record and were NL Central champions. The
following week, the Reds became only the second team in MLB history to
be no-hit in a postseason game when Philadelphia's Roy Halladay shut
down the National League's number one offense in game one of the NLDS.
The Reds lost in a 3-game sweep of the NLDS to Philadelphia.

After coming off their surprising 2010 NL Central Division Title, the
Reds fell short of many expectations for the 2011 season. Multiple
injuries and inconsistent starting pitching played a big role in their
mid-season collapse, along with a less productive offense as compared to
the previous year. The Reds ended the season at 79-83. The Reds won the
2012 NL Central Division Title. On September 28, Homer Bailey threw a
1-0 no-hitter against the Pittsburgh Pirates at PNC Park, this was the
first Reds no-hitter since Tom Browning's perfect game in September of
the 1988 season. Finishing with a 97--65 record, they earned the second
seed in the Division Series and a match-up with the eventual World
Series champion San Francisco Giants. After taking a 2--0 lead with road
victories at AT\&T Park, they headed home looking to win the series.
However, they lost three straight at their home ballpark to become the
first National League team since the Cubs in 1984 to lose a division
series after leading 2--0.

In the off-season, the team traded outfielder Drew Stubbs, as part of a
three team deal with the Arizona Diamondbacks and Cleveland Indians, to
the Indians, and in turn received right fielder Shin-Soo Choo. On July
2, 2013, Homer Bailey pitched a no-hitter against the San Francisco
Giants for a 4-0 Reds victory, making Bailey the third pitcher in Reds
history with two complete game no-hitters in their career.

Following six consecutive losses to close out the 2013 season, including
a loss to the Pittsburgh Pirates, at PNC Park, in the National League
wild-card playoff game, the Reds decided to fire Dusty Baker. During his
six years as manager, Baker led the Reds to the playoff three times;
however, they never advanced beyond the first round.

On October 22, 2013, the Reds hired pitching coach Bryan Price to
replace Baker as manager.

Under Bryan Price, the Reds were led by pitchers Johnny Cueto and the
hard-throwing Cuban Aroldis Chapman. While the offense was led by
all-star third baseman Todd Frazier, Joey Votto, and Brandon Phillips.
Although with plenty of star power, the Reds never got off to a good
start and ending the season in lowly fourth place in the division to go
along with a 76-86 record. During the offseason, the Reds traded
pitchers Alfredo Simón to the Tigers and Mat Latos to the Marlins. In
return, they acquired young talents such as Eugenio Suárez and Anthony
DeSclafani. They also acquired veteran slugger Marlon Byrd from the
Phillies to play left field.

The Reds' 2015 season wasn't much better, as they finished with the
second worst record in the league with a record of 64-98, their worst
finish since 1982. The Reds were forced to trade star pitchers Johnny
Cueto (to the Kansas City Royals) and Mike Leake (to the San Francisco
Giants), receiving minor league pitching prospects for both. Shortly
after the season's end, the Reds traded home run derby champion Todd
Frazier to the Chicago White Sox, and closing pitcher Aroldis Chapman to
the New York Yankees.

In 2016, the Reds broke the record for home runs allowed during a single
season. The previous record holder was the 1996 Detroit Tigers with 241
longballs yielded to opposing teams. The Reds went 68-94, and again were
one of the worst teams in the MLB. The Reds traded outfielder Jay Bruce
to the Mets just before the July 31st non-waiver trade deadline in
exchange for two prospects, infielder Dilson Herrera and pitcher Max
Wotell. During the offseason, the Reds traded Brandon Phillips to the
Atlanta Braves in exchange for two minor league pitchers.

\section{Ballpark}\label{ballpark}

\begin{itemize}
\item
  \emph{Crosley served as the home field for the Reds for two World
  Series titles and five National League pennants.}
\item
  \emph{League Park II was the third home field for the Reds from 1894
  to 1901, and then moved to the Palace of the Fans which served as the
  home of the Reds in the 1910s.}
\item
  \emph{Along with serving as the home field for the Reds, the stadium
  also holds the Cincinnati Reds Hall of Fame.}
\end{itemize}

The Cincinnati Reds play their home games at Great American Ball Park,
located at 100 Joe Nuxhall Way, in downtown Cincinnati. Great American
Ball Park opened in 2003 at the cost of \$290 million and has a capacity
of 42,271. Along with serving as the home field for the Reds, the
stadium also holds the Cincinnati Reds Hall of Fame. The Hall of Fame
was added as a part of Reds tradition allowing fans to walk through the
history of the franchise as well as participating in many interactive
baseball features.

Great American Ball Park is the seventh home of the Cincinnati Reds,
built immediately to the north of the site on which Riverfront Stadium,
later named Cinergy Field, once stood. The first ballpark the Reds
occupied was Bank Street Grounds from 1882 to 1883 until they moved to
League Park I in 1884, where they would remain until 1893. Through the
late 1890s and early 1900s (decade), the Reds moved to two different
parks where they stayed for less than ten years. League Park II was the
third home field for the Reds from 1894 to 1901, and then moved to the
Palace of the Fans which served as the home of the Reds in the 1910s. It
was in 1912 that the Reds moved to Crosley Field which they called home
for fifty-eight years. Crosley served as the home field for the Reds for
two World Series titles and five National League pennants. Beginning
June 30, 1970, and during the dynasty of the Big Red Machine, the Reds
played in Riverfront Stadium, appropriately named due to its location
right by the Ohio River. Riverfront saw three World Series titles and
five National League pennants. It was in the late 1990s that the city
agreed to build two separate stadiums on the riverfront for the Reds and
the Cincinnati Bengals. Thus, in 2003, the Reds began a new era with the
opening of the current stadium.

The Reds hold their spring training in Goodyear, Arizona at Goodyear
Ballpark. The Reds moved into this stadium and the Cactus League in 2010
after staying in the Grapefruit League for most of their history. The
Reds share Goodyear Park with their rivals in Ohio, the Cleveland
Indians.

\section{Logos and uniforms}\label{logos-and-uniforms}

\section{Logo}\label{logo}

\begin{itemize}
\item
  \emph{For most of the history of the Reds, especially during the early
  history, the Reds logo has been simply the wishbone "C" with the word
  "REDS" inside, the only colors used being red and white.}
\item
  \emph{Throughout the history of the Cincinnati Reds, many different
  variations of the classic wishbone "C" logo have been introduced.}
\end{itemize}

Throughout the history of the Cincinnati Reds, many different variations
of the classic wishbone "C" logo have been introduced. For most of the
history of the Reds, especially during the early history, the Reds logo
has been simply the wishbone "C" with the word "REDS" inside, the only
colors used being red and white. However, during the 1950s, during the
renaming and re-branding of the team as the Cincinnati Redlegs because
of the connections to communism of the word 'Reds', the color blue was
introduced as part of the Reds color combination. During the 1960s and
1970s the Reds saw a move towards the more traditional colors,
abandoning the navy blue. A new logo also appeared with the new era of
baseball in 1972, when the team went away from the script "REDS" inside
of the "C", instead, putting their mascot Mr. Redlegs in its place as
well as putting the name of the team inside of the wishbone "C". In the
1990s the more traditional, early logos of Reds came back with the
current logo reflecting more of what the team's logo was when they were
first founded.

\section{Uniform}\label{uniform}

\begin{itemize}
\item
  \emph{During this era, the Reds wore all-red caps both at home and on
  the road.}
\item
  \emph{At home, the Reds wore white caps with the red bill with the
  oval C in red, white sleeveless jerseys with red pinstripes, with the
  oval C-REDS logo in black with red lettering on the left breast and
  the number in red on the right.}
\end{itemize}

Along with the logo, the Reds' uniforms have been changed many different
times throughout their history. Following their departure from being
called the "Redlegs" in 1956 the Reds made a groundbreaking change to
their uniforms with the use of sleeveless jerseys, seen only once before
in the Major Leagues by the Chicago Cubs. At home and away, the cap was
all-red with a white wishbone C insignia. The long-sleeved undershirts
were red. The uniform was plain white with a red wishbone C logo on the
left and the uniform number on the right. On the road the wishbone C was
replaced by the mustachioed "Mr. Red" logo, the pillbox-hat-wearing man
with a baseball for a head. The home stockings were red with six white
stripes. The away stockings had only three white stripes.

The Reds changed uniforms again in 1961, when they replaced the
traditional wishbone C insignia with an oval C logo, but continued to
use the sleeveless jerseys. At home, the Reds wore white caps with the
red bill with the oval C in red, white sleeveless jerseys with red
pinstripes, with the oval C-REDS logo in black with red lettering on the
left breast and the number in red on the right. The gray away uniform
included a gray cap with the red oval C and a red bill. Their gray away
uniforms, which also included a sleeveless jersey, bore CINCINNATI in an
arched block style across with the number below on the left. In 1964,
players' last names were placed on the back of each set of uniforms,
below the numbers. Those uniforms were scrapped after the 1966 season.

However, the Cincinnati uniform design most familiar to baseball
enthusiasts is the one whose basic form, with minor variations, held
sway for the 26 seasons from 1967 to 1992. Most significantly, the point
was restored to the C insignia, making it a wishbone again. During this
era, the Reds wore all-red caps both at home and on the road. The caps
bore the simple wishbone C insignia in white. The uniforms were standard
short-sleeved jerseys and standard trousers---white at home and grey on
the road. The home uniform featured the Wishbone C-REDS logo in red with
white type on the left breast and the uniform number in red on the
right. The away uniform bore CINCINNATI in an arched block style across
the front with the uniform number below on the left. Red, long-sleeved
undershirts and plain red stirrups over white sanitary stockings
completed the basic design.

The 1993 uniforms (which did away with the pullovers and brought back
button-down jerseys) kept white and gray as the base colors for the home
and away uniforms, but added red pinstripes. The home jerseys were
sleeveless, showing more of the red undershirts. The color scheme of the
C-REDS logo on the home uniform was reversed, now red lettering on a
white background. A new home cap was created that had a red bill and a
white crown with red pinstripes and a red wishbone C insignia. The away
uniform kept the all-red cap, but moved the uniform number to the left,
to more closely match the home uniform. The only additional change to
these uniforms was the introduction of black as a primary color of the
Reds in 1999, especially on their road uniforms.

The Reds latest uniform change came in December 2006 which differed
significantly from the uniforms worn during the previous eight seasons.
The home caps returned to an all-red design with a white wishbone C,
lightly outlined in black. Caps with red crowns and a black bill became
the new road caps. Additionally, the sleeveless jersey was abandoned for
a more traditional design. The numbers and lettering for the names on
the backs of the jerseys were changed to an early-1900s style typeface,
and a handlebar mustached "Mr. Redlegs" -- reminiscent of the logo used
by the Reds in the 1950s and 1960s -- was placed on the left sleeve.

\section{Awards and accolades}\label{awards-and-accolades}

\section{Team captains}\label{team-captains}

\begin{itemize}
\item
  \emph{14 Pete Rose 1970--1978}
\end{itemize}

Jake Daubert 1919--1924

14 Pete Rose 1970--1978

13 Dave Concepción 1983--1988

11 Barry Larkin 1997--2004

\section{Retired numbers}\label{retired-numbers}

\begin{itemize}
\item
  \emph{The Cincinnati Reds have retired ten numbers in franchise
  history, as well as honoring Jackie Robinson, whose number is retired
  league-wide around Major League Baseball.}
\item
  \emph{On April 15, 1997, \#42 was retired throughout Major League
  Baseball in honor of Jackie Robinson.}
\end{itemize}

The Cincinnati Reds have retired ten numbers in franchise history, as
well as honoring Jackie Robinson, whose number is retired league-wide
around Major League Baseball.

All of the retired numbers are located at Great American Ball Park
behind home-plate on the outside of the press box. Along with the
retired player and manager number, the following broadcasters are
honored with microphones by the broadcast booth: Marty Brennaman, Waite
Hoyt, and Joe Nuxhall.

On April 15, 1997, \#42 was retired throughout Major League Baseball in
honor of Jackie Robinson.

\section{Baseball Hall of Famers}\label{baseball-hall-of-famers}

\section{Ford C. Frick Award
recipients}\label{ford-c.-frick-award-recipients}

\section{MLB All-Star Games}\label{mlb-all-star-games}

\begin{itemize}
\item
  \emph{The Reds have hosted the Major League Baseball All-Star Game a
  record five times: twice at Crosley Field (1938, 1953), twice at
  Riverfront Stadium (1970, 1988), and once at Great American Ball Park
  (2015).}
\end{itemize}

The Reds have hosted the Major League Baseball All-Star Game a record
five times: twice at Crosley Field (1938, 1953), twice at Riverfront
Stadium (1970, 1988), and once at Great American Ball Park (2015). (The
record of five is shared with the Cleveland Indians and the Pittsburgh
Pirates, however, the Indians are scheduled to host the game again in
2019.)

\section{Ohio Cup}\label{ohio-cup}

\begin{itemize}
\item
  \emph{In its first series it was a single-game cup, played each year
  at minor-league Cooper Stadium in Columbus, was staged just days
  before the start of each new Major League Baseball season.}
\item
  \emph{The Ohio Cup was an annual pre-season baseball game, which
  pitted the Ohio rivals Cleveland Indians and Cincinnati Reds.}
\end{itemize}

The Ohio Cup was an annual pre-season baseball game, which pitted the
Ohio rivals Cleveland Indians and Cincinnati Reds. In its first series
it was a single-game cup, played each year at minor-league Cooper
Stadium in Columbus, was staged just days before the start of each new
Major League Baseball season.

A total of eight Ohio Cup games were played, in 1989 to 1996, with the
Indians winning six of them. The winner of the game each year was
awarded the Ohio Cup in postgame ceremonies. The Ohio Cup was a favorite
among baseball fans in Columbus, with attendances regularly topping
15,000.

The Ohio Cup games ended with the introduction of regular-season
interleague play in 1997. Thereafter, the two teams competed annually in
the regular-season Battle of Ohio or Buckeye Series. The Ohio Cup was
revived in 2008 as a reward for the team with the better overall record
in the Reds-Indians series each year.

\section{Media}\label{media}

\section{Radio}\label{radio}

\begin{itemize}
\item
  \emph{Prior to that, the Reds were heard over: WKRC, WCPO, WSAI and
  WCKY.}
\item
  \emph{The Reds' flagship radio station has been WLW, 700AM since
  1969.}
\item
  \emph{Marty Brennaman has been the Reds' play-by-play voice since 1974
  and has won the Ford C. Frick Award for his work, which includes his
  famous call of "... and this one belongs to the Reds!"}
\end{itemize}

The Reds' flagship radio station has been WLW, 700AM since 1969. Prior
to that, the Reds were heard over: WKRC, WCPO, WSAI and WCKY. WLW, a
50,000-watt station, is "clear channel" in more than one way, as
iHeartMedia owns the "blowtorch" outlet which is also known as "The
Nation's Station".

Marty Brennaman has been the Reds' play-by-play voice since 1974 and has
won the Ford C. Frick Award for his work, which includes his famous call
of "... and this one belongs to the Reds!" after a win. Joining him for
years on color was former Reds pitcher Joe Nuxhall, who worked in the
radio booth from 1967 (the year after his retirement as an active
player) until 2004, plus three more seasons doing select home games
until his death, in 2007.

In 2007, Thom Brennaman, a veteran announcer seen nationwide on Fox
Sports, joined his father Marty in the radio booth. Retired relief
pitcher Jeff Brantley, formerly of ESPN, also joined the network in
2007. As of 2010{[}update{]}, Brantley and Thom Brennaman's increased TV
schedule (see below) has led to more appearances for Jim Kelch, who had
filled in on the network since 2008.

\section{Television}\label{television}

\begin{itemize}
\item
  \emph{NBC affiliate WLWT carried Reds games from 1948 to 1995.}
\item
  \emph{The Reds also added former Cincinnati First Baseman Sean Casey
  -- known as "The Mayor" by Reds fans -- to do color commentary for
  approximately 15 games in 2011.}
\item
  \emph{The last regularly-scheduled, over-the-air broadcasts of Reds
  games were on WSTR-TV from 1996 to 1998.}
\end{itemize}

Televised games are seen exclusively on Fox Sports Ohio and Fox Sports
Indiana. In addition, Fox Sports South televises Fox Sports Ohio
broadcasts of Reds games to Tennessee and western North Carolina. George
Grande, who hosted the first SportsCenter on ESPN in 1979, was the
play-by-play announcer, usually alongside Chris Welsh, from 1993 until
his retirement during the final game of the 2009 season. Since 2009,
Grande has worked part-time for the Reds as play-by-play announcer in
September when Thom Brennaman is covering the NFL for Fox Sports. He has
also made guest appearances throughout the season. Brennaman has been
the head play-by-play commentator since 2010, with Welsh and Brantley
sharing time as the color commentators. Paul Keels, who left in 2011 to
become the play-by-play announcer for the Ohio State Buckeyes Radio
Network, was the Reds' backup play-by-play television announcer during
the 2010 season. Jim Kelch served as Keels' replacement. The Reds also
added former Cincinnati First Baseman Sean Casey -- known as "The Mayor"
by Reds fans -- to do color commentary for approximately 15 games in
2011.

NBC affiliate WLWT carried Reds games from 1948 to 1995. Among those
that have called games for WLWT include Waite Hoyt, Ray Lane, Steve
Physioc, Johnny Bench, Joe Morgan, and Ken Wilson. Al Michaels, who
established a long career with ABC and NBC, spent three years in
Cincinnati early in his career. The last regularly-scheduled,
over-the-air broadcasts of Reds games were on WSTR-TV from 1996 to 1998.
Since 2010, WKRC-TV has simulcast Opening Day games with Fox Sports
Ohio.

\section{Current roster}\label{current-roster}

\section{Minor league affiliations}\label{minor-league-affiliations}

\section{References}\label{references}

\section{External links}\label{external-links}

\begin{itemize}
\item
  \emph{First person interview conducted on March 28, 2012 with Johnny
  Bench, Hall of Fame Catcher for the Cincinnati Reds.}
\item
  \emph{Reds Minor Leagues News}
\item
  \emph{Cincinnati Reds official website}
\end{itemize}

Cincinnati Reds official website

Reds Minor Leagues News

SCSR / 19th Century Cincinnati Base Ball

Voices of Oklahoma interview with Johnny Bench. First person interview
conducted on March 28, 2012 with Johnny Bench, Hall of Fame Catcher for
the Cincinnati Reds.
