\textbf{From Wikipedia, the free encyclopedia}

https://en.wikipedia.org/wiki/Violence\%20Against\%20Women\%20Act\\
Licensed under CC BY-SA 3.0:\\
https://en.wikipedia.org/wiki/Wikipedia:Text\_of\_Creative\_Commons\_Attribution-ShareAlike\_3.0\_Unported\_License

\section{Violence Against Women Act}\label{violence-against-women-act}

\begin{itemize}
\item
  \emph{As a result of the United States federal government shutdown of
  2018--2019, the Violence Against Women Act expired on December 21,
  2018.}
\item
  \emph{The Act also established the Office on Violence Against Women
  within the Department of Justice.}
\item
  \emph{\\
  The Violence Against Women Act of 1994 (VAWA) is a United States
  federal law (Title IV, sec.}
\end{itemize}

The Violence Against Women Act of 1994 (VAWA) is a United States federal
law (Title IV, sec. 40001-40703 of the Violent Crime Control and Law
Enforcement Act, H.R. 3355) signed as Pub.L.~103--322 by President Bill
Clinton on September 13, 1994 (codified in part at 42 U.S.C. sections
13701 through 14040). The Act provided \$1.6 billion toward
investigation and prosecution of violent crimes against women, imposed
automatic and mandatory restitution on those convicted, and allowed
civil redress in cases prosecutors chose to leave un-prosecuted. The Act
also established the Office on Violence Against Women within the
Department of Justice.

VAWA was drafted by the office of Senator Joe Biden (D-DE) and
co-written by Democrat Louise Slaughter, the Representative from New
York, with support from a broad coalition of advocacy groups. The Act
passed through Congress with bipartisan support in 1994, clearing the
United States House of Representatives by a vote of 235--195 and the
Senate by a vote of 61--38, although the following year House
Republicans attempted to cut the Act's funding. In the 2000 Supreme
Court case United States v. Morrison, a sharply divided Court struck
down the VAWA provision allowing women the right to sue their attackers
in federal court. By a 5--4 majority, the Court overturned the provision
as exceeding the federal government's powers under the Commerce Clause.

VAWA was reauthorized by bipartisan majorities in Congress in 2000 and
again in December 2005. The Act's 2012 renewal was opposed by
conservative Republicans, who objected to extending the Act's
protections to same-sex couples and to provisions allowing battered
undocumented immigrants to claim temporary visas, but it was
reauthorized in 2013, after a long legislative battle. As a result of
the United States federal government shutdown of 2018--2019, the
Violence Against Women Act expired on December 21, 2018. It was
temporarily reinstated via a short-term spending bill on January 25,
2019, but expired again on February 15, 2019. The House of
Representatives passed a bill reauthorizing VAWA in April 2019; the
bill, which includes new provisions protecting transgender victims and
banning individuals convicted of domestic abuse from purchasing
firearms, has yet to be considered by the Senate as of April 11.

\section{Background}\label{background}

\begin{itemize}
\item
  \emph{Additionally, VAWA provides specific support for work with
  tribes and tribal organizations to end domestic violence, dating
  violence, sexual assault, and stalking against Native American women.}
\item
  \emph{In the United States, according to the National Intimate Partner
  Sexual Violence Survey of 2010 1 in 6 women suffered some kind of
  sexual violence induced by their intimate partner during the course of
  their lives.}
\item
  \emph{Research on Violence Against Native American Women}
\item
  \emph{Violence on College Campuses Grants}
\end{itemize}

The World Conference on Human Rights, held in Vienna, Austria, in 1993,
and the Declaration on the Elimination of Violence Against Women in the
same year, concluded that civil society and governments have
acknowledged that domestic violence is a public health policy and human
rights concern. In the United States, according to the National Intimate
Partner Sexual Violence Survey of 2010 1 in 6 women suffered some kind
of sexual violence induced by their intimate partner during the course
of their lives.

The Violence Against Women Act was developed and passed as a result of
extensive grassroots efforts in the late 1980s and early 1990s, with
advocates and professionals from the battered women's movement, sexual
assault advocates, victim services field, law enforcement agencies,
prosecutors' offices, the courts, and the private bar urging Congress to
adopt significant legislation to address domestic and sexual violence
{[}citation needed{]}. One of the greatest successes of VAWA is its
emphasis on a coordinated community response to domestic violence, sex
dating violence, sexual assault, and stalking; courts, law enforcement,
prosecutors, victim services, and the private bar currently work
together in a coordinated effort that did not exist before at the state
and local levels {[}citation needed{]}. VAWA also supports the work of
community-based organizations that are engaged in work to end domestic
violence, dating violence, sexual assault, and stalking; particularly
those groups that provide culturally and linguistically specific
services. Additionally, VAWA provides specific support for work with
tribes and tribal organizations to end domestic violence, dating
violence, sexual assault, and stalking against Native American women.

Many grant programs authorized in VAWA have been funded by the U.S.
Congress. The following grant programs, which are administered primarily
through the Office on Violence Against Women in the U.S. Department of
Justice have received appropriations from Congress:

STOP Grants (State Formula Grants)

Transitional Housing Grants

Grants to Encourage Arrest and Enforce Protection Orders

Court Training and Improvement Grants

Research on Violence Against Native American Women

National Tribal Sex Offender Registry

Stalker Reduction Database

Federal Victim Assistants

Sexual Assault Services Program

Services for Rural Victims

Civil Legal Assistance for Victims

Elder Abuse Grant Program

Protections and Services for Disabled Victims

Combating Abuse in Public Housing

National Resource Center on Workplace Responses

Violence on College Campuses Grants

Safe Havens Project

Engaging Men and Youth in Prevention

\section{Debate and legal standing}\label{debate-and-legal-standing}

\begin{itemize}
\item
  \emph{The ACLU, in its July 27, 2005 'Letter to the Senate Judiciary
  Committee Regarding the Violence Against Women Act of 2005, S. 1197'
  stated that "VAWA is one of the most effective pieces of legislation
  enacted to end domestic violence, dating violence, sexual assault, and
  stalking.}
\item
  \emph{It has dramatically improved the law enforcement response to
  violence against women and has provided critical services necessary to
  support women in their struggle to overcome abusive situations".}
\end{itemize}

The American Civil Liberties Union (ACLU) had originally expressed
concerns about the Act, saying that the increased penalties were rash,
that the increased pretrial detention was "repugnant" to the U.S.
Constitution, that the mandatory HIV testing of those only charged but
not convicted was an infringement of a citizen's right to privacy, and
that the edict for automatic payment of full restitution was
non-judicious (see their paper: "Analysis of Major Civil Liberties
Abuses in the Crime Bill Conference Report as Passed by the House and
the Senate", dated September 29, 1994). In 2005, the ACLU had, however,
enthusiastically supported reauthorization of VAWA on the condition that
the "unconstitutional DNA provision" be removed. That provision would
have allowed law enforcement to take DNA samples from arrestees or even
from those who had simply been stopped by police without the permission
of a court.

The ACLU, in its July 27, 2005 'Letter to the Senate Judiciary Committee
Regarding the Violence Against Women Act of 2005, S. 1197' stated that
"VAWA is one of the most effective pieces of legislation enacted to end
domestic violence, dating violence, sexual assault, and stalking. It has
dramatically improved the law enforcement response to violence against
women and has provided critical services necessary to support women in
their struggle to overcome abusive situations".

Some activists oppose the bill. Janice Shaw Crouse, a senior fellow at
the conservative, evangelistic Christian Concerned Women for America's
Beverly LaHaye Institute, called the Act a "boondoggle" which "ends up
creating a climate of suspicion where all men are feared or viewed as
violent and all women are viewed as victims". She described the Act in
2012 as creating a "climate of false accusations, rush to judgment and
hidden agendas" and criticized it for failing to address the factors
identified by the Centers for Disease Control and Prevention as leading
to violent, abusive behavior. Conservative activist Phyllis Schlafly
denounced VAWA as a tool to "fill feminist coffers" and argued that the
Act promoted "divorce, breakup of marriage and hatred of men".

In 2000, the Supreme Court of the United States held part of VAWA
unconstitutional on federalism grounds in United States v. Morrison.
That decision invalidated only the civil remedy provision of VAWA. The
provisions providing program funding were unaffected.

In 2005, the reauthorization of VAWA (as HR3402) defined what population
benefited under the term of "Underserved Populations" described as "
Populations underserved because of geographic location, underserved
racial and ethnic populations, populations underserved because of
special needs (such as language barriers, disabilities, alienage status,
or age) and any other population determined to be underserved by the
Attorney General or by the Secretary of Health and Human Services as
appropriate". The reauthorization also "Amends the Omnibus Crime Control
and Safe Streets Act of 1968" to "prohibit officials from requiring sex
offense victims to submit to a polygraph examination as a condition for
proceeding with an investigation or prosecution of a sex offense."

In 2011, the law expired. In 2012 the law was up for reauthorization in
Congress. Different versions of the legislation have been passed along
party lines in the Senate and House, with the Republican-sponsored House
version favoring the reduction of services to undocumented immigrants
and LGBT individuals. Another area of contention is the provision of the
law giving Native American tribal authorities jurisdiction over sex
crimes involving non-Native Americans on tribal lands. This provision is
considered to have constitutional implications,{[}citation needed{]} as
non-tribes people are under the jurisdiction of the United States
federal government and are granted the protections of the U.S.
Constitution, protections that tribal courts do not often have. The two
bills were pending reconciliation, and a final bill did not reach the
President's desk before the end of the year, temporarily ending the
coverage of the Act after 18 years, as the 112th Congress adjourned.

\section{2012--13 legislative battle and
reauthorization}\label{legislative-battle-and-reauthorization}

\begin{itemize}
\item
  \emph{In April 2012, the Senate voted to reauthorize the Violence
  Against Women Act, and the House subsequently passed its own measure
  (omitting provisions of the Senate bill that would protect gays,
  Native Americans living in reservations, and immigrants who are
  victims of domestic violence).}
\item
  \emph{On February 12, 2013, the Senate passed an extension of the
  Violence Against Women Act by a vote of 78--22.}
\item
  \emph{On March 7, 2013, President Barack Obama signed the Violence
  Against Women Reauthorization Act of 2013.}
\end{itemize}

When a bill reauthorizing the act was introduced in 2012, it was opposed
by conservative Republicans, who objected to extending the Act's
protections to same-sex couples and to provisions allowing battered
foreigners residing in the country illegally to claim temporary visas,
also known as U visas. The U visa is restricted to 10,000 applicants
annually whereas the number of applicants far exceeds these 10,000 for
each fiscal year. In order to be considered for the U visa, one of the
requirements for immigrant women is that they need to cooperate in the
detention of the abuser. Studies show that 30 to 50\% of immigrant women
are suffering from physical violence and 62\% experience physical or
psychological abuse in contrast to only 21\% of citizens in the United
States.

In April 2012, the Senate voted to reauthorize the Violence Against
Women Act, and the House subsequently passed its own measure (omitting
provisions of the Senate bill that would protect gays, Native Americans
living in reservations, and immigrants who are victims of domestic
violence). Reconciliation of the two bills was stymied by procedural
measures, leaving the re-authorization in question. The Senate's 2012
re-authorization of VAWA was not brought up for a vote in the House.

In 2013, the question of jurisdiction over offenses in Indian country
continued to be at issue over the question of whether defendants who are
not tribal members would be treated fairly by tribal courts or afforded
constitutional guarantees.

On February 12, 2013, the Senate passed an extension of the Violence
Against Women Act by a vote of 78--22. The measure went to the House of
Representatives where jurisdiction of tribal courts and inclusion of
same-sex couples were expected to be at issue. Possible solutions
advanced were permitting either removal or appeal to federal courts by
non-tribal defendants. The Senate had tacked on the Trafficking Victims
Protection Act which is another bone of contention due to a clause which
requires provision of reproductive health services to victims of sex
trafficking.

On February 28, 2013, in a 286--138 vote, the House passed the Senate's
all-inclusive version of the bill. House Republicans had previously
hoped to pass their own version of the measure---one that substantially
weakened the bill's protections for certain categories. The stripped
down version, which allowed only limited protection for LGBT and Native
Americans, was rejected 257 to 166. The renewed act expanded federal
protections to gay, lesbian, and transgender individuals, Native
Americans and immigrants.

On March 7, 2013, President Barack Obama signed the Violence Against
Women Reauthorization Act of 2013.

\section{After passage}\label{after-passage}

\begin{itemize}
\item
  \emph{138 House Republicans voted against the version of the act that
  became law.}
\end{itemize}

138 House Republicans voted against the version of the act that became
law. However, several, including Steve King (R-Iowa), Bill Johnson
(R-Ohio), Tim Walberg (R-Michigan), Vicky Hartzler (R-Missouri), Keith
Rothfus (R-Pennsylvania), and Tim Murphy (R-Pennsylvania), claimed to
have voted in favor of the act. Some have called this claim disingenuous
because the group only voted in favor of a GOP proposed alternative
version of the bill that did not contain provisions intended to protect
gays, lesbians and transgender individuals, Native Americans and
undocumented immigrants.

\section{Reauthorizations}\label{reauthorizations}

\begin{itemize}
\item
  \emph{The Act's 2012 renewal was opposed by conservative Republicans,
  who objected to extending the Act's protections to same-sex couples
  and to provisions allowing battered undocumented immigrants to claim
  temporary visas.}
\item
  \emph{On April 4, 2019, the reauthorization act passed in the House by
  a vote of 263-158.}
\item
  \emph{As a result of the United States federal government shutdown of
  2018--2019, the Violence Against Women Act expired on December 21,
  2018.}
\end{itemize}

VAWA was reauthorized by bipartisan majorities in Congress in 2000 (H.R.
1248, Roll Call 415-3), and again in December 2005, and signed by
President George W. Bush. The Act's 2012 renewal was opposed by
conservative Republicans, who objected to extending the Act's
protections to same-sex couples and to provisions allowing battered
undocumented immigrants to claim temporary visas. Ultimately, VAWA was
again reauthorized in 2013, after a long legislative battle throughout
2012--2013.

On September 12, 2013, at an event marking the 19th anniversary of the
bill, Vice President Joe Biden criticized the Republicans who slowed the
passage of the reauthorization of the act as being "this sort of
Neanderthal crowd".

As a result of the United States federal government shutdown of
2018--2019, the Violence Against Women Act expired on December 21, 2018.
It was temporarily reauthorized by a short-term spending bill on January
25, 2019, but expired again on February 15, 2019.

On April 4, 2019, the reauthorization act passed in the House by a vote
of 263-158. All Democrats voting joined by 33 Republicans voted for
passage. New York Representative Elise Stefanik said Democrats, "...have
refused to work with Republicans in a meaningful way," adding, the House
bill will do nothing but "collect dust" in the GOP-controlled Senate.

\section{Programs and services}\label{programs-and-services}

\begin{itemize}
\item
  \emph{Protections for victims who are evicted from their homes because
  of events related to domestic violence or stalking}
\item
  \emph{The Violence Against Women laws provide programs and services,
  including:}
\item
  \emph{Programs to meet the needs of immigrant women and women of
  different races or ethnicities}
\item
  \emph{Legal aid for survivors of domestic violence}
\item
  \emph{Community violence prevention programs}
\end{itemize}

The Violence Against Women laws provide programs and services,
including:

Federal rape shield law.

Community violence prevention programs

Protections for victims who are evicted from their homes because of
events related to domestic violence or stalking

Funding for victim assistance services, like rape crisis centers and
hotlines

Programs to meet the needs of immigrant women and women of different
races or ethnicities

Programs and services for victims with disabilities

Legal aid for survivors of domestic violence

\includegraphics[width=4.01867in,height=5.50000in]{media/image1.jpg}\\
\emph{Restraining order granted to a Wisconsin woman against her abuser,
noting the nationwide applicability of the order under Full Faith and
Credit}

\section{Restraining orders}\label{restraining-orders}

\begin{itemize}
\item
  \emph{When a woman---the Wisconsin Coalition Against Domestic Violence
  generally refers to petitioners as female as most are women---is the
  beneficiary of an order of protection, per VAWA it is generally
  enforceable nationwide under the terms of full faith and credit.}
\end{itemize}

When a woman---the Wisconsin Coalition Against Domestic Violence
generally refers to petitioners as female as most are women---is the
beneficiary of an order of protection, per VAWA it is generally
enforceable nationwide under the terms of full faith and credit.
Although the order may be granted only in a specific state, full faith
and credit requires that it be enforced in other states as though the
order was granted in their states.18 U.S.C.~§~2265

\section{Persons who are covered under VAWA immigration
provisions}\label{persons-who-are-covered-under-vawa-immigration-provisions}

\begin{itemize}
\item
  \emph{The following persons are eligible to benefit from the
  immigration provisions of VAWA:}
\item
  \emph{VAWA allows for the possibility that certain individuals who
  might not otherwise be eligible for immigration benefits may petition
  for US permanent residency on the grounds of a close relationship with
  a US citizen or permanent resident who has been abusing them.}
\end{itemize}

VAWA allows for the possibility that certain individuals who might not
otherwise be eligible for immigration benefits may petition for US
permanent residency on the grounds of a close relationship with a US
citizen or permanent resident who has been abusing them. The following
persons are eligible to benefit from the immigration provisions of VAWA:

A wife or husband who has been abused by a U.S. citizen or permanent
resident (Green Card holder) spouse. The petition will also cover the
petitioner's children under age 21.

A child abused by a U.S. citizen or permanent resident parent. The
petition can be filed by an abused child or by her parent on the child's
behalf.

A parent who has been abused by a U.S. citizen child who is at least 21
years old.

\section{Coverage of male victims}\label{coverage-of-male-victims}

\begin{itemize}
\item
  \emph{Jan Brown, the Founder and Executive Director of the Domestic
  Abuse Helpline for Men and Women contends that the Act may not be
  sufficient to ensure equal access to services.}
\item
  \emph{Although the title of the Act and the titles of its sections
  refer to victims of domestic violence as women, the operative text is
  gender-neutral, providing coverage for male victims as well.}
\end{itemize}

Although the title of the Act and the titles of its sections refer to
victims of domestic violence as women, the operative text is
gender-neutral, providing coverage for male victims as well. Individual
organizations have not been successful in using VAWA to provide equal
coverage for men. The law has twice been amended in attempts to address
this situation. The 2005 reauthorization added a non-exclusivity
provision clarifying that the title should not be construed to prohibit
male victims from receiving services under the Act. The 2013
reauthorization added a non-discrimination provision that prohibits
organizations receiving funding under the Act from discriminating on the
basis of sex, although the law allows an exception for "sex segregation
or sex-specific programming" when it is deemed to be "necessary to the
essential operations of a program." Jan Brown, the Founder and Executive
Director of the Domestic Abuse Helpline for Men and Women contends that
the Act may not be sufficient to ensure equal access to services.

\section{Related developments}\label{related-developments}

\begin{itemize}
\item
  \emph{The ultimate aims of both groups are to help improve and/or
  protect the well-being and safety of women and girls in the United
  States.}
\item
  \emph{Official federal government groups that have developed, being
  established by President Barack Obama, in relation to the Violence
  Against Women Act include the White House Council on Women and Girls
  and the White House Task Force to Protect Students from Sexual
  Assault.}
\end{itemize}

Official federal government groups that have developed, being
established by President Barack Obama, in relation to the Violence
Against Women Act include the White House Council on Women and Girls and
the White House Task Force to Protect Students from Sexual Assault. The
ultimate aims of both groups are to help improve and/or protect the
well-being and safety of women and girls in the United States.

\section{See also}\label{see-also}

\begin{itemize}
\item
  \emph{International Violence Against Women Act}
\item
  \emph{Women's shelter}
\item
  \emph{Violence against men}
\item
  \emph{Outline of domestic violence}
\end{itemize}

International Violence Against Women Act

Outline of domestic violence

Violence against men

Women's shelter

\section{References}\label{references}

\section{External links}\label{external-links}

\begin{itemize}
\item
  \emph{World Health Organization Multi-country Study on Women's Health
  and Domestic Violence against Women 2005}
\item
  \emph{Office on Violence Against Women}
\item
  \emph{Violence Against Women and Department of Justice Reauthorization
  Act of 2005}
\item
  \emph{Privacy Provisions of the Violence Against Women Act}
\item
  \emph{Violence Against Women Act (VAWA) Provides Protections for
  Immigrant Women and Victims of Crime}
\end{itemize}

Violence Against Women Act (VAWA) Provides Protections for Immigrant
Women and Victims of Crime

Violence Against Women and Department of Justice Reauthorization Act of
2005

Office on Violence Against Women

Privacy Provisions of the Violence Against Women Act

World Health Organization Multi-country Study on Women's Health and
Domestic Violence against Women 2005

VAWA 2005 Fact Sheet
