\textbf{From Wikipedia, the free encyclopedia}

https://en.wikipedia.org/wiki/Johnnie\%20Carr\\
Licensed under CC BY-SA 3.0:\\
https://en.wikipedia.org/wiki/Wikipedia:Text\_of\_Creative\_Commons\_Attribution-ShareAlike\_3.0\_Unported\_License

\section{Johnnie Carr}\label{johnnie-carr}

\begin{itemize}
\item
  \emph{Carr was a childhood friend of Rosa Parks and is considered,
  along with Parks, King, E. D. Nixon and others to be an important face
  in the Civil Rights Movement in Montgomery, Alabama.}
\item
  \emph{According to Morris Dees, one of three founders of Montgomery's
  Southern Poverty Law Center, "Johnnie Carr is one of the three major
  icons of the Civil Rights Movement: Dr. King, Rosa Parks and Johnnie
  Carr.}
\end{itemize}

Johnnie Rebecca Daniels Carr (January 26, 1911 -- February 22, 2008) was
a leader in the Civil Rights Movement in the United States from 1955
until her death.

In 1967, Carr became President of the Montgomery Improvement
Association, succeeding Martin Luther King, Jr. Carr held this office
until she died.

Carr was a childhood friend of Rosa Parks and is considered, along with
Parks, King, E. D. Nixon and others to be an important face in the Civil
Rights Movement in Montgomery, Alabama. According to Morris Dees, one of
three founders of Montgomery's Southern Poverty Law Center, "Johnnie
Carr is one of the three major icons of the Civil Rights Movement: Dr.
King, Rosa Parks and Johnnie Carr. I think ultimately, when the final
history books are written, she'll be one of the few people remembered
for that terrific movement."

Civil rights pioneer and U.S. Representative John Lewis, D-Ga., said,
"Mrs. Carr must be looked on as one of the founders of a new America
because she was there with Rosa Parks, E. D. Nixon, Martin Luther King
Jr. and so many others."

In 1944, Carr, along with her husband Arlam Carr, Rosa Parks and Raymond
Parks, E.D. Nixon, E. G. Jackson, and Irene West organized to defend a
black woman near Montgomery who was gang raped by six white men. This
core of activists, who canvassed neighborhoods, raised money, and sent
petitions and postcards to the governor and attorney general of Alabama,
later became part of the movement that supported Martin Luther King, Jr.

Carr died of a stroke at the age of 97.

\section{References}\label{references}

\section{Further readings}\label{further-readings}

\begin{itemize}
\item
  \emph{Women and the civil rights movement, 1954-1965.}
\item
  \emph{Freedom facts and firsts~: 400 years of the African American
  civil rights experience.}
\end{itemize}

Houck, Davis W.; Dixon, David E. (2010). Women and the civil rights
movement, 1954-1965. Jackson: University of Mississippi press.
ISBN~9781604731071.

Smith, Jessie Carney (2009). Freedom facts and firsts~: 400 years of the
African American civil rights experience. Canton: Visible Ink Press.
ISBN~9781578591923.

\section{External links}\label{external-links}

\begin{itemize}
\item
  \emph{Alabama Civil Rights collection - The Jack Rabin Collection on
  Alabama Civil Rights and Southern Activists, at Penn State University,
  includes materials and oral history interviews of the Montgomery
  Improvement Association.}
\end{itemize}

Obituary in The Times, 9 March 2008

Alabama Civil Rights collection - The Jack Rabin Collection on Alabama
Civil Rights and Southern Activists, at Penn State University, includes
materials and oral history interviews of the Montgomery Improvement
Association.
