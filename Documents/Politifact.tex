\textbf{From Wikipedia, the free encyclopedia}

https://en.wikipedia.org/wiki/Politifact\\
Licensed under CC BY-SA 3.0:\\
https://en.wikipedia.org/wiki/Wikipedia:Text\_of\_Creative\_Commons\_Attribution-ShareAlike\_3.0\_Unported\_License

\section{PolitiFact}\label{politifact}

\begin{itemize}
\item
  \emph{PolitiFact has won several awards, and has been both praised and
  criticized by independent observers, conservatives and liberals
  alike.}
\item
  \emph{Both PolitiFact and PunditFact were funded primarily by the
  Tampa Bay Times and ad revenues generated on the website until 2018,
  and the Times continues to sell ads for the site now that it is part
  of Poynter, a non-profit organization that also owns the newspaper.}
\end{itemize}

PolitiFact.com is a nonprofit project operated by the Poynter Institute
in St. Petersburg, Florida, with offices there and in Washington, D.C..
It began in 2007 as a project of the Tampa Bay Times (then the St.
Petersburg Times), with reporters and editors from the newspaper and its
affiliated news media partners reporting on the accuracy of statements
made by elected officials, candidates, their staffs, lobbyists, interest
groups and others involved in U.S. politics. Its journalists evaluate
original statements and publish their findings on the PolitiFact.com
website, where each statement receives a "Truth-O-Meter" rating. The
ratings range from "True" for completely accurate statements to "Pants
on Fire" (from the taunt "Liar, liar, pants on fire") for false and
ridiculous claims.

PunditFact, a related site that was also created by the Times' editors,
is devoted to fact-checking claims made by political pundits. Both
PolitiFact and PunditFact were funded primarily by the Tampa Bay Times
and ad revenues generated on the website until 2018, and the Times
continues to sell ads for the site now that it is part of Poynter, a
non-profit organization that also owns the newspaper. PolitiFact
increasingly relies on grants from several nonpartisan organizations,
and in 2017 launched a membership campaign and began accepting donations
from readers.

In addition to political claims, the site monitors the progress elected
officials make on their campaign promises, including a "Trump-O-Meter"
for President Donald Trump and an "Obameter" for President Barack Obama.
PolitiFact.com's local affiliates review promises by elected officials
of regional relevance, as evidenced by PolitiFact Tennessee's
"Haslam-O-Meter" tracking Tennessee Governor Bill Haslam's efforts and
Wisconsin's "Walk-O-Meter" tracking Wisconsin Governor Scott Walker's
efforts.

PolitiFact has won several awards, and has been both praised and
criticized by independent observers, conservatives and liberals alike.
Both liberal and conservative bias have been alleged at different
points, and criticisms have been made that PolitiFact attempts to
fact-check statements that cannot be truly "fact-checked".

\section{History}\label{history}

\begin{itemize}
\item
  \emph{In March 2010, the Times and its partner newspaper, The Miami
  Herald, launched PolitiFact Florida, which focuses on Florida issues.}
\item
  \emph{In 2014, The Plain Dealer ended its partnership with
  PolitiFact.com after they reduced their news staff and were unwilling
  to meet "the required several PolitiFact investigations per week".}
\item
  \emph{The Knoxville News Sentinel ended its relationship with
  PolitiFact.com after 2012.}
\end{itemize}

PolitiFact.com was started in August 2007 by Times Washington Bureau
Chief Bill Adair, in conjunction with the Congressional Quarterly.

In January 2010, PolitiFact.com expanded to its second newspaper, the
Cox Enterprises --owned Austin American-Statesman in Austin, Texas; the
feature, called PolitiFact Texas, covers issues that are relevant to
Texas and the Austin area.

In March 2010, the Times and its partner newspaper, The Miami Herald,
launched PolitiFact Florida, which focuses on Florida issues. The Times
and the Herald share resources on some stories that relate to Florida.

Since then, PolitiFact.com expanded to other papers, such as The Atlanta
Journal-Constitution, The Providence Journal, Milwaukee Journal
Sentinel, The Plain Dealer, Richmond Times-Dispatch, the Knoxville News
Sentinel and The Oregonian.\\
The Knoxville News Sentinel ended its relationship with PolitiFact.com
after 2012.

In 2013, Adair was named Knight Professor of the Practice of Journalism
and Public Policy at Duke University, and stepped down as Bureau Chief
at the Times and as editor at PolitiFact.com. The Tampa Bay Times'
senior reporter, Alex Leary, succeeded Bill Adair as Bureau Chief on
July 1, 2013, and Angie Drobnic Holan was appointed editor of PolitiFact
in October 2013. Adair remains a PolitiFact.com contributing editor.

In 2014, The Plain Dealer ended its partnership with PolitiFact.com
after they reduced their news staff and were unwilling to meet "the
required several PolitiFact investigations per week".

The organization was acquired in February 2018 by the Poynter Institute,
a non-profit journalism education and news media research center that
also owns the Tampa Bay Times.

\section{"Lie of the Year"}\label{lie-of-the-year}

\begin{itemize}
\item
  \emph{PolitiFact's 2017 Lie of the Year was Donald Trump's claim that
  Russian election interference is a "made-up story."}
\item
  \emph{PolitiFact's 2016 Lie of the Year was "fake news" referring to
  fabricated news stories including the Pizzagate conspiracy theory.}
\item
  \emph{PolitiFact's 2015 Lie of the Year was the "various statements"
  made by 2016 Republican presidential candidate Donald Trump.}
\end{itemize}

Since 2009, PolitiFact.com has declared one political statement from
each year to be the "Lie of the Year".

2009

In December 2009, they declared the Lie of the Year to be Sarah Palin's
assertion that the Patient Protection and Affordable Care Act of 2009
would lead to government "death panels" that dictated which types of
patients would receive treatment.

2010

In December 2010, PolitiFact.com dubbed the Lie of the Year to be the
contention among some opponents of the Patient Protection and Affordable
Care Act that it represented a "government takeover of healthcare".
PolitiFact.com argued that this was not the case, since all health care
and insurance would remain in the hands of private companies.

2011

PolitiFact's Lie of the Year for 2011 was a statement by the Democratic
Congressional Campaign Committee (DCCC) that a 2011 budget proposal by
Congressman Paul Ryan, entitled The Path to Prosperity and voted for
overwhelmingly by Republicans in the House and Senate, meant that
"Republicans voted to end Medicare". PolitiFact determined that, though
the Republican plan would make significant changes to Medicare, it would
not end it. PolitiFact had originally labeled nine similar statements as
"false" or "pants on fire" since April 2011.

2012

For 2012, PolitiFact chose the claim made by Republican presidential
candidate Mitt Romney that President Obama "sold Chrysler to Italians
who are going to build Jeeps in China" at the cost of American jobs.
(The "Italians" in the quote was a reference to Fiat, who had purchased
a majority share of Chrysler in 2011 after a U.S. government bailout of
Chrysler.) PolitiFact had rated the claim "Pants on Fire" in October.
PolitiFact's assessment quoted a Chrysler spokesman who had said, "Jeep
has no intention of shifting production of its Jeep models out of North
America to China." As of 2016, 96.7 percent of Jeeps sold in the U.S
were assembled in the U.S., with roughly 70 percent North American parts
content. (The vehicle with the most North American parts content came in
at 75\%).

2013

The 2013 Lie of the Year was President Barack Obama's promise that "If
you like your health care plan, you can keep it". As evidence,
PolitiFact cited 4 million cancellation letters sent to American health
insurance consumers. PolitiFact also noted that in an online poll,
readers overwhelmingly agreed with the selection. This stands in stark
contrast to its October 9, 2008 statement that Obama's ``description of
his plan is accurate, and we rate his statement True.''

2014

PolitiFact's 2014 Lie of the Year was "Exaggerations about Ebola",
referring to 16 separate statements made by various commentators and
politicians about the Ebola virus being "easy to catch, that illegal
immigrants may be carrying the virus across the southern border, that it
was all part of a government or corporate conspiracy". These claims were
made in the midst of the Ebola virus epidemic in West Africa when four
cases were diagnosed in the United States in travelers from West Africa
and nurses who treated them. PolitiFact wrote, "The claims -- all wrong
-- distorted the debate about a serious public health issue."

2015

PolitiFact's 2015 Lie of the Year was the "various statements" made by
2016 Republican presidential candidate Donald Trump. Politifact found
that 76\% of Trump's statements that they reviewed were rated "Mostly
False," "False" or "Pants on Fire". Statements that were rated "Pants on
Fire" included his assertion that the Mexican government sends "the bad
ones over" the border into the United States, and his claim that he saw
"thousands and thousands" of people cheering the collapse of the World
Trade Center on 9/11.

2016

PolitiFact's 2016 Lie of the Year was "fake news" referring to
fabricated news stories including the Pizzagate conspiracy theory.

2017

PolitiFact's 2017 Lie of the Year was Donald Trump's claim that Russian
election interference is a "made-up story." The annual poll found
56.36\% of the 5080 respondents agreed that Trump's "Pants on Fire"
statement deserved the distinction. Raul Labrador's statement that "
Nobody dies because they don't have access to health care," and Sean
Spicer's statement that "{[}Trump's audience{]} was the largest audience
to witness an inauguration, period," came in second and third place
getting 14.47\% and 14.25\% of the vote respectively. In its article,
PolitiFact points to multiple occasions where Donald Trump stated that
Russia had not interfered with the election despite multiple government
agencies claiming otherwise.

2018

Survivors of the Stoneman Douglas High School shooting were the subject
of lies that became Politifact's 2018 Lie of the Year. Students
including Emma González and David Hogg, who became prominent gun control
activists in the wake of the shooting and helped organize the March for
our Lives, were the subjects of conspiracy theories that they were
"crisis actors". The lies were spread on blogs and social media by
sources including InfoWars.

\section{Reception}\label{reception}

\begin{itemize}
\item
  \emph{In February 2011, University of Minnesota political science
  professor Eric Ostermeier analyzed 511 PolitiFact stories issued from
  January 2010 through January 2011.}
\item
  \emph{TV critic James Poniewozik at Time characterized PolitiFact as
  having the "hard-earned and important position as referee in the
  mudslinging contest---a 'truth vigilante,' as it were", and
  "PolitiFact is trying to do the right thing here.}
\end{itemize}

PolitiFact.com was awarded the Pulitzer Prize for National Reporting in
2009 for "its fact-checking initiative during the 2008 presidential
campaign that used probing reporters and the power of the World Wide Web
to examine more than 750 political claims, separating rhetoric from
truth to enlighten voters".

A Wall Street Journal opinion editorial by Joseph Rago in December 2010
called PolitiFact "part of a larger journalistic trend that seeks to
recast all political debates as matters of lies, misinformation and
'facts,' rather than differences of world view or principles".

TV critic James Poniewozik at Time characterized PolitiFact as having
the "hard-earned and important position as referee in the mudslinging
contest---a 'truth vigilante,' as it were", and "PolitiFact is trying to
do the right thing here. And despite the efforts of partisans to work
the refs by complaining about various calls they've made in the past,
they're generally doing a hard, important thing well. They often do it
better than the rest of the political media, and the political press
owes them for doing it." Poniewozik also suggested, "they need to
improve their rating system, to address the irresponsible, the
unprovable, the dubious. Otherwise, they're doing exactly what they were
founded to stop: using language to spread false impressions."

Mark Hemingway of The Weekly Standard criticized all fact-checking
projects by news organizations, including PolitiFact, the Associated
Press and the Washington Post, writing that they "aren't about checking
facts so much as they are about a rearguard action to keep inconvenient
truths out of the conversation".

In February 2011, University of Minnesota political science professor
Eric Ostermeier analyzed 511 PolitiFact stories issued from January 2010
through January 2011. He found that the number of statements analyzed
from Republicans and from Democrats was comparable, but Republicans have
been assigned substantially harsher grades, receiving 'false' or 'pants
on fire' more than three times as often as Democrats. Ostermeier
suggested that this may indicate a bias in the selection of statements
to analyze, concluding: "The question is not whether PolitiFact will
ultimately convert skeptics on the right that they do not have ulterior
motives in the selection of what statements are rated, but whether the
organization can give a convincing argument that either a) Republicans
in fact do lie much more than Democrats, or b) if they do not, that it
is immaterial that PolitiFact covers political discourse with a frame
that suggests this is the case." PolitiFact editor Bill Adair responded
in MinnPost: "Eric Ostermeier's study is particularly timely because
we've heard a lot of charges this week that we are biased---from
liberals {[}...{]} So we're accustomed to hearing strong reactions from
people on both ends of the political spectrum. We are a news
organization and we choose which facts to check based on news judgment."

In December 2011, Northeastern University journalism professor Dan
Kennedy wrote in the Huffington Post that the problem with fact-checking
projects was "there are only a finite number of statements that can be
subjected to thumbs-up/thumbs-down fact-checking".

Matt Welch, in the February 2013 issue of Reason magazine, criticized
PolitiFact and other media fact-checkers for focusing much more on
statements by politicians about their opponents, rather than statements
by politicians and government officials about their own policies, thus
serving as "a check on the exercise of rhetoric" but not "a check on the
exercise of power".

\section{See also}\label{see-also}

\begin{itemize}
\item
  \emph{The Fact Checker}
\item
  \emph{Fairness and Accuracy in Reporting}
\item
  \emph{FactCheck.org}
\end{itemize}

The Fact Checker

Fairness and Accuracy in Reporting

FactCheck.org

\section{References}\label{references}

\section{External links}\label{external-links}

\begin{itemize}
\item
  \emph{Official website}
\end{itemize}

Official website
