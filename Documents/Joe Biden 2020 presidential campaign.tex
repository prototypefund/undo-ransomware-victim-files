\textbf{From Wikipedia, the free encyclopedia}

https://en.wikipedia.org/wiki/Joe\%20Biden\%202020\%20presidential\%20campaign\\
Licensed under CC BY-SA 3.0:\\
https://en.wikipedia.org/wiki/Wikipedia:Text\_of\_Creative\_Commons\_Attribution-ShareAlike\_3.0\_Unported\_License

\section{Joe Biden 2020 presidential
campaign}\label{joe-biden-2020-presidential-campaign}

\begin{itemize}
\item
  \emph{The 2020 presidential campaign of Joe Biden was announced early
  in the morning of April 25, 2019, with a video announcement.}
\end{itemize}

The 2020 presidential campaign of Joe Biden was announced early in the
morning of April 25, 2019, with a video announcement. Joe Biden, the
former Vice President of the United States and a former U.S. senator
from Delaware, had been the subject of widespread speculation as a
potential 2020 candidate after declining to be a candidate in the 2016
election.

\section{Background}\label{background}

\section{Previous presidential
campaigns}\label{previous-presidential-campaigns}

\begin{itemize}
\item
  \emph{Biden's 2020 presidential campaign is his third attempt to seek
  election for president of the United States.}
\end{itemize}

Biden's 2020 presidential campaign is his third attempt to seek election
for president of the United States. His first campaign was made in the
1988 Democratic Party primaries where he was initially considered one of
the potentially strongest candidates. However, newspapers revealed
plagiarism by Biden in law school records and in speeches, a scandal
which led to his withdrawal from the race in September 1987.

He made the second attempt during the 2008 Democratic Party primaries,
where he focused on his plan to achieve political success in the Iraq
War through a system of federalization. Like his first presidential bid,
Biden failed to garner endorsements and support that he withdrew from
the race after his poor performance in the Iowa caucus in January 3,
2008. He was eventually chosen by Barack Obama as his running mate and
won the general election as vice president of the United States, being
sworn in on January 20, 2009.

\section{Speculation}\label{speculation}

\begin{itemize}
\item
  \emph{Time for Biden, a political action committee, was formed in
  January 2018, seeking Biden's entry into the 2020 Democratic Party
  presidential primaries.}
\item
  \emph{In March 2018, Politico reported that Biden's team was
  considering a number of options to distinguish their campaign, such as
  announcing at the outset a younger vice presidential candidate from
  outside of politics, and also reported that Biden had rejected a
  proposition to commit to only serving one term as president.}
\item
  \emph{Vice President Joe Biden was seen as a potential candidate to
  succeed Barack Obama in the 2016 presidential election.}
\end{itemize}

Vice President Joe Biden was seen as a potential candidate to succeed
Barack Obama in the 2016 presidential election. On October 21, 2015,
following the death of his son Beau, Biden announced that he would not
seek the Democratic presidential nomination in 2016.

During a tour of the United States Senate with reporters on December 5,
2016, Biden refused to rule out a potential bid for the presidency in
the 2020 presidential election. He reasserted his ambivalence about
running on an appearance of The Late Show with Stephen Colbert on
December 7, in which he stated "never say never" about running for
president in 2020, while also admitting he did not see a scenario in
which he would run for office again. He seemingly announced on January
13, 2017, exactly one week prior to the expiration of his vice
presidential term, that he would not run. However, four days later, he
seemed to backtrack, stating "I'll run if I can walk." In September
2017, Biden's daughter Ashley indicated her belief that he was thinking
about running in 2020.

Time for Biden, a political action committee, was formed in January
2018, seeking Biden's entry into the 2020 Democratic Party presidential
primaries. In February 2018, Biden informed a group of longtime foreign
policy aides that he was "keeping his 2020 options open".

In March 2018, Politico reported that Biden's team was considering a
number of options to distinguish their campaign, such as announcing at
the outset a younger vice presidential candidate from outside of
politics, and also reported that Biden had rejected a proposition to
commit to only serving one term as president. On July 17, 2018, he told
a forum held in Bogota, Colombia, that he would decide if he would
formally declare as a candidate by January 2019. On February 4, with no
decision having been forthcoming from Biden, Edward-Isaac Dovere of The
Atlantic wrote that Biden was "very close to saying yes" but that some
close to him are worried he will have a last-minute change of heart, as
he did in 2016. Dovere reported that Biden was concerned about the
effect another presidential run could have on his family and reputation,
as well as fundraising struggles and perceptions about his age and
relative centrism compared to other declared and potential candidates.
Conversely, his "sense of duty", offense at the Trump presidency, the
lack of foreign policy experience among other Democratic hopefuls and
his desire to foster "bridge-building progressivism" in the party were
said to be factors prompting him to run.

\section{Announcement}\label{announcement}

\begin{itemize}
\item
  \emph{On April 19, 2019, The Atlantic reported that Biden planned to
  officially announce his campaign on April 24, 2019 in a video
  announcement, followed by a subsequent launch rally in Philadelphia,
  Pennsylvania or Charlottesville, Virginia.}
\item
  \emph{Biden released a video formally announcing his campaign early on
  April 25.}
\end{itemize}

On March 12, 2019, he told a gathering of supporters that he may need
their energy "in a few weeks". Five days later, Biden accidentally said
that he would be a candidate in the slip of his tongue at a dinner in
Dover, Delaware.

On April 19, 2019, The Atlantic reported that Biden planned to
officially announce his campaign on April 24, 2019 in a video
announcement, followed by a subsequent launch rally in Philadelphia,
Pennsylvania or Charlottesville, Virginia. In the days before his
expected launch, several major Democratic donors received requests to
donate to his campaign committee, to be named "Biden for President".
However, subsequent reports on April 22 indicated that Biden's plans
remained uncertain, with no known launch date, locations for campaign
rallies unknown, and no permits secured for an event in Philadelphia;
though associates continued to plan a fundraiser on April 25 in
Philadelphia hosted by Comcast executive vice president David L. Cohen,
it is unclear whether the fundraiser will be held as planned, though his
associates have continued to solicit donations in the days leading up to
his expected announcement. Subsequent reports indicated that Biden would
formally enter the race on April 25, 2019, so as to avoid overshadowing
the She the People forum on the day before, and reserved the Teamsters
Local 249 union hall in Pittsburgh for April 29.

Biden released a video formally announcing his campaign early on April
25.

\section{Fundraising}\label{fundraising}

\begin{itemize}
\item
  \emph{On April 26, 2019, Biden's campaign announced that they had
  raised \$6.3 million in the first 24 hours, surpassing all other
  candidates' first 24 hour fundraising totals for the Democratic
  presidential nomination at that time.}
\end{itemize}

On April 26, 2019, Biden's campaign announced that they had raised \$6.3
million in the first 24 hours, surpassing all other candidates' first 24
hour fundraising totals for the Democratic presidential nomination at
that time.

\section{Strategy and tactics}\label{strategy-and-tactics}

\begin{itemize}
\item
  \emph{According to Natasha Korecki and Marc Caputo from Politico, the
  Biden campaign is running a campaign on the premise that the
  Democratic base isn't nearly as liberal or youthful as perceived.}
\item
  \emph{It's older than you think it is.'' From April 25, 2019, to May
  25, 2019, Biden's campaign has spent 83\% of his total \$1.2 million
  Facebook ad money on targeting voters 45 years and older.}
\end{itemize}

According to Natasha Korecki and Marc Caputo from Politico, the Biden
campaign is running a campaign on the premise that the Democratic base
isn't nearly as liberal or youthful as perceived. Privately, several
Biden advisers acknowledge that their theory is based on polling data
and voting trends, contending that the media is pushing the idea of a
hyper-progressive Democratic electorate being propagated by a Twitter
bubble and being out of touch with the average rank-and-file Democrat.
In April 2019, Biden told reports that ``The fact of the matter is the
vast majority of the members of the Democratic Party are still basically
liberal to moderate Democrats in the traditional sense,''. Biden also
told those reports that he describes himself as an "Obama-Biden
Democrat". Two Biden advisers who declined to speak on record said
``There's a big disconnect between the media narrative and what the
primary electorate looks like and thinks, versus the media narrative and
the Twitter narrative,'' and ``The Democratic primary universe is far
less liberal. It's older than you think it is.'' From April 25, 2019, to
May 25, 2019, Biden's campaign has spent 83\% of his total \$1.2 million
Facebook ad money on targeting voters 45 years and older. No other top
2020 Democratic candidate has pursued a similar strategy in the primary.

\section{Campaign staff}\label{campaign-staff}

\begin{itemize}
\item
  \emph{Additionally, on May 31st, the Biden campaign announced that
  Congressman Cedric Richmond would join the campaign as the national
  co-chairman.}
\item
  \emph{On May 22, 2019, the magazine Ebony reported that Biden had
  begun assembling his 2019 presidential campaign team, to be
  headquartered in Philadelphia.}
\end{itemize}

On May 22, 2019, the magazine Ebony reported that Biden had begun
assembling his 2019 presidential campaign team, to be headquartered in
Philadelphia. His team includes campaign manager Greg Schultz and
director of strategic communications Kamau Mandela Marshall, who both
previously worked in the Obama administration as well as other senior
advisors from the Obama administration. Additionally, on May 31st, the
Biden campaign announced that Congressman Cedric Richmond would join the
campaign as the national co-chairman.

\section{Endorsements}\label{endorsements}

\section{Political positions}\label{political-positions}

\begin{itemize}
\item
  \emph{On June 5, 2019, the Biden campaign confirmed to NBC News that
  Biden still supports the Hyde Amendment, something no other Democratic
  presidential candidate came out in support of.}
\item
  \emph{On May 21, 2019, a Biden campaign aide clarified to The
  Associated Press that Biden would support immediately federal
  legislation codifying the Roe v. Wade precedent into statute.}
\end{itemize}

Although generally viewed as a moderate, Biden declared himself as a
candidate with the most progressive record.

On April 29, 2019, Biden came out in favor of a public option for health
insurance and outlawing non-compete clauses for low-wage workers.

On May 21, 2019, a Biden campaign aide clarified to The Associated Press
that Biden would support immediately federal legislation codifying the
Roe v. Wade precedent into statute.

On June 1, 2019, Biden gave a keynote address to hundreds of activists
and donors at the Human Rights Campaign's annual Ohio gala. He declared
his top legislative priority was passing the Equality Act. He attacked
Donald Trump for banning transgender troops in the U.S. military, allow
individuals in the medical field to deny to treat LGBTQ individuals, and
allow homeless shelters to deny transgender occupants.

On June 4, 2019, the Biden campaign released a \$1.7 trillion climate
plan that embraced the framework of the Green New Deal.

On June 5, 2019, the Biden campaign confirmed to NBC News that Biden
still supports the Hyde Amendment, something no other Democratic
presidential candidate came out in support of. Biden's campaign also
told NBC News that Biden would be open to repealing the Hyde Amendment
if abortion access protections currently under Roe v. Wade were
threatened. On June 6, 2019, Biden, at the Democratic National
Committee's African American Leadership Council Summit in Atlanta,
Georgia, stated he now supports repealing the Hyde Amendment, crediting
his change in position, in part, to recent efforts by Republicans
passing anti-abortion state laws, which he called ``extreme laws.'' Also
at the summit, he focused on economic inequality for African Americans,
education access, criminal justice reform, healthcare, and voter
suppression in the south.

\section{See also}\label{see-also}

\begin{itemize}
\item
  \emph{2020 Democratic Party presidential primaries}
\end{itemize}

2020 Democratic Party presidential primaries

\section{References}\label{references}
