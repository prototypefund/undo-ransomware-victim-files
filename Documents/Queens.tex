\textbf{From Wikipedia, the free encyclopedia}

https://en.wikipedia.org/wiki/Queens\\
Licensed under CC BY-SA 3.0:\\
https://en.wikipedia.org/wiki/Wikipedia:Text\_of\_Creative\_Commons\_Attribution-ShareAlike\_3.0\_Unported\_License

\section{Queens}\label{queens}

\begin{itemize}
\item
  \emph{Queens is the easternmost of the five boroughs of New York
  City.}
\item
  \emph{Queens became a borough during the consolidation of New York
  City in 1898, and from 1683 until 1899, the County of Queens included
  what is now Nassau County.}
\item
  \emph{Queens has the most diversified economy of the five boroughs of
  New York City.}
\end{itemize}

Queens is the easternmost of the five boroughs of New York City. It is
the largest borough geographically and is adjacent to the borough of
Brooklyn at the southwestern end of Long Island. To its east is Nassau
County. Queens also shares water borders with the boroughs of Manhattan
and the Bronx. Coterminous with Queens County since 1899, the borough of
Queens is the second largest in population (after Brooklyn), with an
estimated 2,358,582 residents in 2017, approximately 48 percent of them
foreign-born. Queens County also is the second most populous county in
the U.S. state of New York, behind Brooklyn, which is coterminous with
Kings County. Queens is the fourth most densely populated county among
New York City's boroughs, as well as in the United States. If each of
New York City's boroughs were an independent city, Queens would be the
nation's fourth most populous, after Los Angeles, Chicago, and Brooklyn.
Queens is the most ethnically diverse urban area in the world. It is
also the most ethnically diverse county in the United States.

Queens was established in 1683 as one of the original 12 counties of New
York. The settlement was presumably named for the English queen
Catherine of Braganza (1638--1705). Queens became a borough during the
consolidation of New York City in 1898, and from 1683 until 1899, the
County of Queens included what is now Nassau County.

Queens has the most diversified economy of the five boroughs of New York
City. It is home to John F. Kennedy International Airport and LaGuardia
Airport, both among the world's busiest, which in turn makes the
airspace above Queens among the busiest in the United States. Landmarks
in Queens include Flushing Meadows--Corona Park; Citi Field, home to the
New York Mets baseball team; the USTA Billie Jean King National Tennis
Center, site of the US Open tennis tournament; Kaufman Astoria Studios;
Silvercup Studios; and Aqueduct Racetrack. The borough has diverse
housing, ranging from high-rise apartment buildings in the urban areas
of western and central Queens, such as Ozone Park, Jackson Heights,
Flushing, Astoria, and Long Island City, to somewhat more suburban
neighborhoods in the eastern part of the borough, including
Douglaston--Little Neck and Bayside. The Queens Night Market in Flushing
Meadows-Corona Park attracts over 10,000 people nightly to sample food
from over 85 countries.

\section{History}\label{history}

\includegraphics[width=4.48800in,height=5.50000in]{media/image1.jpg}\\
\emph{Catherine of Braganza, former Queen of England}

\section{Colonial and post-colonial
history}\label{colonial-and-post-colonial-history}

\begin{itemize}
\item
  \emph{On May 8, 1884, Rikers Island was transferred to New York
  County.}
\item
  \emph{On June 8, 1881, North Brother Island was transferred to New
  York County.}
\item
  \emph{Queens, like the rest of what became New York City and Long
  Island, remained under British occupation after the Battle of Long
  Island in 1776 and was occupied throughout most of the rest of the
  Revolutionary War.}
\item
  \emph{It was an original county of New York State, one of twelve
  created on November 1, 1683.}
\end{itemize}

European colonization brought Dutch and English settlers, as a part of
the New Netherland colony. First settlements occurred in 1635 followed
by early colonizations at Maspeth in 1642, and Vlissingen (now Flushing)
in 1643. Other early settlements included Newtown (now Elmhurst) and
Jamaica. However, these towns were mostly inhabited by English settlers
from New England via eastern Long Island (Suffolk County) subject to
Dutch law. After the capture of the colony by the English and its
renaming as New York in 1664, the area (and all of Long Island) became
known as Yorkshire.

The Flushing Remonstrance signed by colonists in 1657 is considered a
precursor to the United States Constitution's provision on freedom of
religion in the Bill of Rights. The signers protested the Dutch colonial
authorities' persecution of Quakers in what is today the borough of
Queens.

Originally, Queens County included the adjacent area now comprising
Nassau County. It was an original county of New York State, one of
twelve created on November 1, 1683. The county is assumed to have been
named after Catherine of Braganza, since she was queen of England at the
time (she was Portugal's royal princess Catarina daughter of King John
IV of Portugal). The county was founded alongside Kings County
(Brooklyn, which was named after her husband, King Charles II), and
Richmond County (Staten Island, named after his illegitimate son, the
1st Duke of Richmond). However, the namesake is in dispute; while
Catherine's title seems the most likely namesake, no historical evidence
of official declaration has been found. On October 7, 1691, all counties
in the Colony of New York were redefined. Queens gained North and South
Brother Islands as well as Huletts Island (today known as Rikers
Island). On December 3, 1768, Queens gained other islands in Long Island
Sound that were not already assigned to a county but that did not abut
on Westchester County (today's Bronx County).

Queens played a minor role in the American Revolution, as compared to
Brooklyn, where the Battle of Long Island was largely fought. Queens,
like the rest of what became New York City and Long Island, remained
under British occupation after the Battle of Long Island in 1776 and was
occupied throughout most of the rest of the Revolutionary War. Under the
Quartering Act, British soldiers used, as barracks, the public inns and
uninhabited buildings belonging to Queens residents. Even though many
local people were against unannounced quartering, sentiment throughout
the county remained in favor of the British crown. The quartering of
soldiers in private homes, except in times of war, was banned by the
Third Amendment to the United States Constitution. Nathan Hale was
captured by the British on the shore of Flushing Bay in Queens before
being executed by hanging in Manhattan for gathering intelligence.

From 1683 until 1784, Queens County consisted of five towns: Flushing,
Hempstead, Jamaica, Newtown, and Oyster Bay. On April 6, 1784, a sixth
town, the Town of North Hempstead, was formed through secession by the
northern portions of the Town of Hempstead. The seat of the county
government was located first in Jamaica, but the courthouse was torn
down by the British during the American Revolution to use the materials
to build barracks. After the war, various buildings in Jamaica
temporarily served as courthouse and jail until a new building was
erected about 1787 (and later completed) in an area near Mineola (now in
Nassau County) known then as Clowesville.

The 1850 United States Census was the first in which the population of
the three western towns exceeded that of the three eastern towns that
are now part of Nassau County. Concerns were raised about the condition
and distance of the old courthouse, and several sites were in contention
for the construction of a new one.

In 1870, Long Island City split from the Town of Newtown, incorporating
itself as a city, consisting of what had been the Village of Astoria and
some unincorporated areas within the Town of Newtown. Around 1874, the
seat of county government was moved to Long Island City from Mineola.

On March 1, 1860, the eastern border between Queens County (later Nassau
County) and Suffolk County was redefined with no discernible change. On
June 8, 1881, North Brother Island was transferred to New York County.
On May 8, 1884, Rikers Island was transferred to New York County.

In 1885, Lloyd Neck, which was part of the Town of Oyster Bay and was
earlier known as Queens Village, seceded from Queens and became part of
the Town of Huntington in Suffolk County. On April 16, 1964, South
Brother Island was transferred to Bronx County.

\includegraphics[width=5.50000in,height=2.98119in]{media/image2.jpg}\\
\emph{Queens Boulevard, looking east from Van Dam Street, in 1920. The
newly built IRT Flushing Line is in the boulevard's median.}

\includegraphics[width=5.50000in,height=3.97016in]{media/image3.jpg}\\
\emph{Laurel Hill Chemical Works, 1883. Parts of Queens were becoming
industrial suburbs}

\section{Incorporation as borough}\label{incorporation-as-borough}

\begin{itemize}
\item
  \emph{In later years, Queens was the site of the 1939 New York World's
  Fair and the 1964 New York World's Fair.}
\item
  \emph{From 1915 onward, much of Queens was connected to the New York
  City Subway system.}
\item
  \emph{The New York City Borough of Queens was authorized on May 4,
  1897, by a vote of the New York State Legislature after an 1894
  referendum on consolidation.}
\end{itemize}

The New York City Borough of Queens was authorized on May 4, 1897, by a
vote of the New York State Legislature after an 1894 referendum on
consolidation. The eastern 280 square miles (730~km2) of Queens that
became Nassau County was partitioned on January 1, 1899.

Queens Borough was established on January 1, 1898.

"The city of Long Island City, the towns of Newtown, Flushing and
Jamaica, and that part of the town of Hempstead, in the county of
Queens, which is westerly of a straight line drawn through the middle of
the channel between Rockaway Beach and Shelter Island, in the county of
Queens, to the Atlantic Ocean" was annexed to New York City, dissolving
all former municipal governments (Long Island City, the county
government, all towns, and all villages) within the new borough. The
areas of Queens County that were not part of the consolidation plan,
consisting of the towns of North Hempstead and Oyster Bay, and the major
remaining portion of the Town of Hempstead, remained part of Queens
County until they seceded to form the new Nassau County on January 1,
1899. At this point, the boundaries of Queens County and the Borough of
Queens became coterminous. With consolidation, Jamaica once again became
the county seat, though county offices now extend to nearby Kew Gardens
also.

The borough's administrative and court buildings are presently located
in Kew Gardens and downtown Jamaica respectively, two neighborhoods that
were villages of the former Town of Jamaica.

From 1905 to 1908 the Long Island Rail Road in Queens became
electrified. Transportation to and from Manhattan, previously by ferry
or via bridges in Brooklyn, opened up with the Queensboro Bridge
finished in 1909, and with railway tunnels under the East River in 1910.
From 1915 onward, much of Queens was connected to the New York City
Subway system. With the 1915 construction of the Steinway Tunnel
carrying the IRT Flushing Line between Queens and Manhattan, and the
robust expansion of the use of the automobile, the population of Queens
more than doubled in the 1920s, from 469,042 in 1920 to 1,079,129 in
1930.

In later years, Queens was the site of the 1939 New York World's Fair
and the 1964 New York World's Fair. LaGuardia Airport, in northern
Queens, opened in 1939. Idlewild Airport, in southern Queens and now
called JFK Airport, opened in 1948. American Airlines Flight 587 took
off from the latter airport on November 12, 2001, but ended up crashing
in Queens' Belle Harbor area, killing 265 people. In late October 2012,
much of Queens' Breezy Point area was destroyed by a massive six-alarm
fire caused by Hurricane Sandy.

\section{Geography}\label{geography}

\begin{itemize}
\item
  \emph{Brooklyn, the only other New York City borough on geographic
  Long Island, lies just south and west of Queens, with Newtown Creek,
  an estuary that flows into the East River, forming part of the
  border.}
\item
  \emph{North of Queens are Flushing Bay and the Flushing River,
  connecting to the East River.}
\item
  \emph{Nassau County is east of Queens on Long Island.}
\end{itemize}

Queens is located on the far western portion of geographic Long Island
and includes a few smaller islands, most of which are in Jamaica Bay,
forming part of the Gateway National Recreation Area, which in turn is
one of the National Parks of New York Harbor. According to the United
States Census Bureau, Queens County has a total area of 178 square miles
(460~km2), of which 109 square miles (280~km2) is land and 70 square
miles (180~km2) (39\%) is water.

Brooklyn, the only other New York City borough on geographic Long
Island, lies just south and west of Queens, with Newtown Creek, an
estuary that flows into the East River, forming part of the border. To
the west and north is the East River, across which is Manhattan to the
west and The Bronx to the north. Nassau County is east of Queens on Long
Island. Staten Island is southwest of Brooklyn, and shares only a
3-mile-long water border (in the Outer Bay) with Queens. North of Queens
are Flushing Bay and the Flushing River, connecting to the East River.
The East River opens into Long Island Sound. The midsection of Queens is
crossed by the Long Island straddling terminal moraine created by the
Wisconsin Glacier. The Rockaway Peninsula, the southernmost part of all
of Long Island, sits between Jamaica Bay and the Atlantic Ocean,
featuring 7 miles (11~km) of beaches.

\includegraphics[width=5.50000in,height=2.75000in]{media/image4.jpg}\\
\emph{NASA Landsat satellite image of Long Island and surrounding areas}

\section{Climate}\label{climate}

\begin{itemize}
\item
  \emph{On January 24, 2016, 30.5 inches (77~cm) of snow fell, which is
  the record in Queens.}
\item
  \emph{Under the Köppen climate classification, using the 32~°F (0~°C)
  coldest month (January) isotherm, Queens and the rest of New York City
  have a humid subtropical climate (Cfa) with partial shielding from the
  Appalachian Mountains and moderating influences from the Atlantic
  Ocean.}
\end{itemize}

Under the Köppen climate classification, using the 32~°F (0~°C) coldest
month (January) isotherm, Queens and the rest of New York City have a
humid subtropical climate (Cfa) with partial shielding from the
Appalachian Mountains and moderating influences from the Atlantic Ocean.
Queens receives precipitation throughout the year, with an average of
44.8 inches (114~cm) per year. In an average year, there will be 44 days
with either moderate or heavy rain.

An average winter will have 22 days with some snowfall, of which 9 days
have at least 1 inch (2.5~cm) of snowfall. Summer is typically hot,
humid, and wet. An average year will have 17 days with a high
temperature of 90~°F (32~°C) or warmer. In an average year, there are 14
days where the temperature does not go above 32~°F (0~°C) all day.
Spring and autumn can vary from chilly to very warm.

The highest temperature ever recorded at LaGuardia Airport was 107~°F
(42~°C) on July 3, 1966. The highest temperature ever recorded at John
F. Kennedy International Airport was 104~°F (40~°C), also on July 3,
1966. LaGuardia Airport's record-low temperature was −7~°F (−22~°C) on
February 15, 1943, the effect of which was exacerbated by a shortage of
heating oil and coal. John F. Kennedy International Airport's record-low
temperature was −2~°F (−19~°C), on February 8, 1963, and January 21,
1985. On January 24, 2016, 30.5 inches (77~cm) of snow fell, which is
the record in Queens.

Tornadoes are generally rare; the most recent tornado, an EF0, touched
down in College Point on August 3, 2018, causing minor damage. Prior to
that, there was a tornado in Breezy Point on September 8, 2012, which
damaged the roofs of some homes, and an EF1 tornado in Flushing on
September 26, 2010.

\section{Neighborhoods}\label{neighborhoods}

\begin{itemize}
\item
  \emph{Residents of Queens often closely identify with their
  neighborhood rather than with the borough or city.}
\item
  \emph{Four United States Postal Service postal zones serve Queens,
  based roughly on those serving the towns in existence at the
  consolidation of the five boroughs into New York City: Long Island
  City (ZIP codes starting with 111), Jamaica (114), Flushing (113), and
  Far Rockaway (116).}
\end{itemize}

Four United States Postal Service postal zones serve Queens, based
roughly on those serving the towns in existence at the consolidation of
the five boroughs into New York City: Long Island City (ZIP codes
starting with 111), Jamaica (114), Flushing (113), and Far Rockaway
(116). In addition, the Floral Park post office (110), based in Nassau
County, serves a small part of northeastern Queens. Each of these main
post offices have neighborhood stations with individual ZIP codes, and
unlike the other boroughs, these station names are often used in
addressing letters. These ZIP codes do not always reflect traditional
neighborhood names and boundaries; "East Elmhurst", for example, was
largely coined by the USPS and is not an official community. Most
neighborhoods have no solid boundaries. The Forest Hills and Rego Park
neighborhoods, for instance, overlap.

Residents of Queens often closely identify with their neighborhood
rather than with the borough or city. The borough is a patchwork of
dozens of unique neighborhoods, each with its own distinct identity:

Flushing, one of the largest neighborhoods in Queens, has a large and
growing Asian community. The community consists of Chinese, Koreans, and
South Asians. Asians have now expanded eastward along the Northern
Boulevard axis through Murray Hill, Whitestone, Bayside,
Douglaston--Little Neck, and eventually into adjacent Nassau County.
These neighborhoods historically contained Italian Americans and Greeks,
as well as Latino Americans. The busy intersection of Main Street,
Kissena Boulevard, and 41st Avenue defines the center of Downtown
Flushing and the Flushing Chinatown (法拉盛華埠), known as the "Chinese
Times Square" or the "Chinese Manhattan". The segment of Main Street
between Kissena Boulevard and Roosevelt Avenue, punctuated by the Long
Island Rail Road trestle overpass, represents the cultural heart of the
Flushing Chinatown. Housing over 30,000 individuals born in China alone,
the largest by this metric outside Asia, Flushing has become home to the
largest and one of the fastest-growing Chinatowns in the world as the
heart of over 250,000 ethnic Chinese in Queens, representing the largest
Chinese population of any U.S. municipality other than New York City in
total. Conversely, the Flushing Chinatown has also become the epicenter
of organized prostitution in the United States, importing women from
China, Korea, Thailand, and Eastern Europe to sustain the underground
North American sex trade.

Howard Beach, Whitestone, and Middle Village are home to large Italian
American populations.

Ozone Park and South Ozone Park have large Italian, Hispanic, and
Guyanese populations.

Rockaway Beach has a large Irish American population.

Astoria, in the northwest, is traditionally home to one of the largest
Greek populations outside Greece, it also has large Spanish American,
Albanian American, Bosnian American, Bulgarian American, Croatian
American, Romanian American and Italian American communities, and is
also home to a growing population of Arabs, South Asians, and young
professionals from Manhattan. Nearby Long Island City is a major
commercial center and the home to Queensbridge, the largest housing
project in North America.

Maspeth and Ridgewood are home to many Eastern European immigrants such
as Romanian, Polish, Serbian, Albanian, and other Slavic populations.
Ridgewood also has a large Hispanic population.

Jackson Heights, Elmhurst, and East Elmhurst make up an conglomeration
of Hispanic, Asian, Tibetan, and South Asian communities.

Woodside is home to a large Filipino American community and has a
"Little Manila" as well a large Irish American population. There is also
a large presence of Filipino Americans in Queens Village and in Hollis.

Richmond Hill, in the south, is often thought of as "Little Guyana" for
its large Guyanese community.

Rego Park, Forest Hills, Kew Gardens, and Kew Gardens Hills have
traditionally large Jewish populations (historically from Germany and
eastern Europe; though more recent immigrants are from Israel, Iran, and
the former Soviet Union). These neighborhoods are also known for large
and growing Asian communities, mainly immigrants from China.

Jamaica Estates, Jamaica Hills, Hillcrest, Fresh Meadows, and Hollis
Hills are also populated with many people of Jewish background. Many
Asian families reside in parts of Fresh Meadows as well.

Jamaica is home to large African American and Caribbean populations.
There are also middle-class African American and Caribbean neighborhoods
such as Saint Albans, Queens Village, Cambria Heights, Springfield
Gardens, Rosedale, Laurelton, and Briarwood along east and southeast
Queens.

Bellerose and Floral Park, originally home to many Irish Americans, is
home to a growing South Asian population, predominantly Indian
Americans.

Corona and Corona Heights, once considered the "Little Italy" of Queens,
was a predominantly Italian community with a strong African American
community in the northern portion of Corona and adjacent East Elmhurst.
From the 1920s through the 1960s, Corona remained a close-knit
neighborhood. Corona today has the highest concentration of Latinos of
any Queens neighborhood, with an increasing Chinese American population,
located between Elmhurst and Flushing.

\section{Demographics}\label{demographics}

\section{2010 U.S. Census}\label{u.s.-census}

\begin{itemize}
\item
  \emph{In the 2010 United States Census, Queens recorded a population
  of 2,230,722.}
\item
  \emph{The New York City Department of City Planning was alarmed by the
  negligible reported increase in population between 2000 and 2010.}
\end{itemize}

In the 2010 United States Census, Queens recorded a population of
2,230,722. There were 780,117 households enumerated, with an average of
2.82 persons per household. The population density was 20,465.3
inhabitants per square mile (7,966.9/km2). There were 835,127 housing
units at an average density of 7,661.7 per square mile (2,982.6/km2).
The racial makeup of the county was 39.7\% White, 19.1\% Black or
African American, 0.7\% Native American, 22.9\% Asian, 0.1\% Pacific
Islander, 12.9\% from other races, and 4.5\% from two or more races.
27.5\% of the population were Hispanic or Latino of any race. The
non-Hispanic/Latino white population was 27.6\%.

The New York City Department of City Planning was alarmed by the
negligible reported increase in population between 2000 and 2010. Areas
with high proportions of immigrants and undocumented aliens are
traditionally undercounted for a variety of reasons, often based on a
mistrust of government officials or an unwillingness to be identified.
In many cases, counts of vacant apartment units did not match data from
local surveys and reports from property owners.

\section{Population estimates since
2010}\label{population-estimates-since-2010}

\begin{itemize}
\item
  \emph{27.9\% of Queens's population was of Hispanic or Latino origin
  (of any race).}
\item
  \emph{Queens' estimated population represented 27.4\% of New York
  City's population of 8,622,698; 30.0\% of Long Island's population of
  7,869,820; and 11.9\% of New York State's population of 19,849,399.}
\end{itemize}

As of 2017, the population of Queens was estimated by the United States
Census Bureau to have increased to 2,358,582, a rise of 5.7\%. Queens'
estimated population represented 27.4\% of New York City's population of
8,622,698; 30.0\% of Long Island's population of 7,869,820; and 11.9\%
of New York State's population of 19,849,399.

According to 2012 census estimates, 27.2\% of the population was
Non-Hispanic White, 20.9\% Black or African American, 24.8\% Asian,
12.9\% from some other race, and 2.7\% of two or more races. 27.9\% of
Queens's population was of Hispanic or Latino origin (of any race).

\section{Ethnic groups}\label{ethnic-groups}

\begin{itemize}
\item
  \emph{In 2010, Queens held a disproportionate share of several Asian
  communities within New York City, relative to its overall population,
  as follows:}
\item
  \emph{Queens is home to 49.6\% of the city's Asian population.}
\item
  \emph{The Jewish Community Study of New York 2011, sponsored by the
  UJA-Federation of New York, found that about 9\% of Queens residents
  were Jews.}
\end{itemize}

According to a 2001 Claritas study, Queens was the most diverse county
in the United States among counties of 100,000+ population. A 2014
analysis by The Atlantic found Queens County to be the 3rd most racially
diverse county-equivalent in the United States---behind Aleutians West
Census Area and Aleutians East Borough in Alaska---as well as the most
diverse county in New York. In Queens, approximately 48.5\% of the
population was foreign-born as of 2010. Of that, 49.5\% were born in
Latin America, 33.5\% in Asia, 14.8\% in Europe, 1.8\% in Africa, and
0.4\% in North America. Roughly 2.1\% of the population was born in
Puerto Rico, a U.S. territory, or abroad to American parents. In
addition, 51.2\% of the population was born in the United States.
Approximately 44.2\% of the population over 5 years of age speak English
at home; 23.8\% speak Spanish at home. Also, 16.8\% of the populace
speak other Indo-European languages at home. Another 13.5\% speak a
non-Indo-European Asian language or language of the Pacific Islands at
home.

Among the Asian population, people of Chinese ethnicity make up the
largest ethnic group at 10.2\% of Queens' population, with about 237,484
people; the other East and Southeast Asian groups are: Koreans (2.9\%),
Filipinos (1.7\%), Japanese (0.3\%), Thais (0.2\%), Vietnamese (0.2\%),
and Indonesians and Burmese both make up 0.1\% of the population. People
of South Asian descent make up 7.8\% of Queens' population: Indians
(5.3\%), Bangladeshi (1.5\%), Pakistanis (0.7\%), and Nepali (0.2\%).

Among the Hispanic population, Puerto Ricans make up the largest ethnic
group at 4.6\%, next to Mexicans, who make up 4.2\% of the population,
and Dominicans at 3.9\%. Central Americans make up 2.4\% and are mostly
Salvadorans. South Americans constitute 9.6\% of Queens's population,
mainly of Ecuadorian (4.4\%) and Colombian descent (3.2\%).

Some main European ancestries in Queens as of 2000 include:

Italian: 8.4\%

Irish: 5.5\%

German: 3.5\%

Polish: 2.7\%

Russian: 2.3\%

Greek: 2.0\%

The Hispanic or Latino population increased by 61\% to 597,773 between
1990 and 2006 and now accounts for 26.5\% of the borough's population.
Queens is now home to hundreds of thousands of Latinos and Hispanics:

Queens has the largest Colombian population in the city, accounting for
76.6\% of the city's total Colombian population, for a total of 80,116.

Queens has the largest Ecuadorian population in the city, accounting for
62.2\% of the city's total Ecuadorian population, for a total of
101,339.

Queens has the largest Peruvian population in the city, accounting for
69.9\% of the city's total Peruvian population, for a total of 30,825.

Queens has the largest Salvadoran population in the city, accounting for
50.7\% of the city's for a total population of 25,235.

The Mexican population in Queens has increased 45.7\% to 71,283, the
second highest in the city, after Brooklyn.

Queens is home to 49.6\% of the city's Asian population. Among the five
boroughs, Queens has the largest population of Chinese, Indian, Korean,
Filipino, Bangladeshi and Pakistani Americans. Queens has the largest
Asian American population by county outside the Western United States;
according to the 2006 American Community Survey, Queens ranks fifth
among US counties with 477,772 (21.18\%) Asian Americans, behind Los
Angeles County, California, Honolulu County, Hawaii, Santa Clara County,
California, and Orange County, California.

The borough is also home to one of the highest concentrations of Indian
Americans in the nation, with an estimated population of 144,896 in 2014
(6.24\% of the 2014 borough population), as well as Pakistani Americans,
who number at 15,604. Queens has the second largest Sikh population in
the nation after California.

In 2010, Queens held a disproportionate share of several Asian
communities within New York City, relative to its overall population, as
follows:

Chinese: 200,205; 39.8\% of the city's total Chinese population.

Indian: 117,550; 64\% Asian Indian population.

Korean: 64,107; 66.4\% of the city's total Korean population.

Filipino: 38,163; 61.3\% of the city's total Filipino population.

Bangladeshi: 18,310; 66\% of the city's total Bangladeshi population.

Pakistani: 10,884; 39.5\% of the city's total Pakistani population.

Queens has the third largest Bosnian population in the United States
behind only St. Louis and Chicago, numbering more than 15,000.

The Jewish Community Study of New York 2011, sponsored by the
UJA-Federation of New York, found that about 9\% of Queens residents
were Jews. In 2011, there were about 198,000 Jews in Queens, making it
home to about 13\% of all people in Jewish households in the
eight-county area consisting of the Five Boroughs and Westchester,
Nassau, and Suffolk counties. Russian-speaking Jews make up 28\% of the
Jewish population in Queens, the largest in any of the eight counties.

There were 782,664 households out of which 31.5\% had children under the
age of 18 living with them, 46.9\% were married couples living together,
16.0\% had a female householder with no husband present, and 31.3\% were
non-families. 25.6\% of all households were made up of individuals and
9.7\% had someone living alone who was 65 years of age or older. The
average household size was 2.81 and the average family size was 3.39.

In the county, the population was spread out with 22.8\% under the age
of 18, 9.6\% from 18 to 24, 33.1\% from 25 to 44, 21.7\% from 45 to 64,
and 12.7\% who were 65 years of age or older. The median age was 35
years. For every 100 females, there were 92.9 males. For every 100
females age 18 and over, there were 89.6 males.

The median income for a household in the county was \$37,439, and the
median income for a family was \$42,608. Males had a median income of
\$30,576 versus \$26,628 for females. The per capita income for the
county was \$19,222. About 16.9\% of families and 24.7\% of the
population were below the poverty line, including 18.8\% of those under
age 18 and 13.0\% of those age 65 or over. In Queens, the black
population earns more than whites on average.\\
Many of these African Americans live in quiet, middle class suburban
neighborhoods near the Nassau County border, such as Laurelton and
Cambria Heights which have large black populations whose family income
is higher than average. The migration of European Americans from parts
of Queens has been long ongoing with departures from Ozone Park,
Woodhaven, Bellerose, Floral Park, and Flushing (most of the outgoing
population has been replaced with Asian Americans). Neighborhoods such
as Whitestone, College Point, North Flushing, Auburndale, Bayside,
Middle Village, and Douglaston--Little Neck have not had a substantial
exodus of white residents, but have seen an increase of Asian
population, mostly Chinese and Korean. Queens has experienced a real
estate boom making most of its neighborhoods desirable for people who
want to reside near Manhattan but in a less urban setting.

\section{Culture}\label{culture}

\begin{itemize}
\item
  \emph{In the 1940s, Queens was an important center of jazz; such jazz
  luminaries as Louis Armstrong, Charlie Parker, and Ella Fitzgerald
  took up residence in Queens, seeking refuge from the segregation they
  found elsewhere in New York.}
\item
  \emph{Stating that Queens is "quickly becoming its hippest" but that
  "most travelers haven't clued in\ldots{} yet," the Lonely Planet
  stated that "nowhere is the image of New York as the global melting
  pot truer than Queens."}
\end{itemize}

Queens has been the center of a major artistic movement in the form of
punk rock with The Ramones originating out of Forest Hills, it has also
been the home of such notable artists as Tony Bennett, Francis Ford
Coppola, Paul Simon, and Robert Mapplethorpe. The current poet laureate
of Queens is Paolo Javier.

Queens has notably fostered African-American culture, with
establishments such as The Afrikan Poetry Theatre and the Black Spectrum
Theater Company catering specifically to African Americans in Queens. In
the 1940s, Queens was an important center of jazz; such jazz luminaries
as Louis Armstrong, Charlie Parker, and Ella Fitzgerald took up
residence in Queens, seeking refuge from the segregation they found
elsewhere in New York. Additionally, many notable hip-hop acts hail from
Queens, including Nas, Run-D.M.C., Kool G Rap, A Tribe Called Quest, LL
Cool J, Mobb Deep, 50 Cent, Nicki Minaj, and Heems of Das Racist.

Queens hosts various museums and cultural institutions that serve its
diverse communities. They range from the historical (such as the John
Bowne House) to the scientific (such as the New York Hall of Science),
from conventional art galleries (such as the Noguchi Museum) to unique
graffiti exhibits (such as 5 Pointz). Queens's cultural institutions
include, but are not limited to:

The travel magazine Lonely Planet also named Queens the top destination
in the country for 2015 for its cultural and culinary diversity. Stating
that Queens is "quickly becoming its hippest" but that "most travelers
haven't clued in\ldots{} yet," the Lonely Planet stated that "nowhere is
the image of New York as the global melting pot truer than Queens."

\section{Languages}\label{languages}

\begin{itemize}
\item
  \emph{In total, 56.16\% (1,160,483) of Queens's population aged five
  and older spoke a language at home other than English.}
\end{itemize}

There are 138 languages spoken in the borough. As of 2010, 43.84\%
(905,890) of Queens residents aged five and older spoke only English at
home, while 23.88\% (493,462) spoke Spanish, 8.06\% (166,570) Chinese,
3.44\% (71,054) various Indic languages, 2.74\% (56,701) Korean, 1.67\%
(34,596) Russian, 1.56\% (32,268) Italian, 1.54\% (31,922) Tagalog,
1.53\% (31,651) Greek, 1.32\% (27,345) French Creole, 1.17\% (24,118)
Polish, 0.96\% (19,868) Hindi, 0.93\% (19,262) Urdu, 0.92\% (18,931)
other Asian languages, 0.80\% (16,435) other Indo-European languages,
0.71\% (14,685) French, 0.61\% (12,505) Arabic, 0.48\% (10,008)
Serbo-Croatian, and Hebrew was spoken as a main language by 0.46\%
(9,410) of the population over the age of five. In total, 56.16\%
(1,160,483) of Queens's population aged five and older spoke a language
at home other than English.

\section{Food}\label{food}

\begin{itemize}
\item
  \emph{The cuisine available in Queens reflects its vast cultural
  diversity.}
\end{itemize}

The cuisine available in Queens reflects its vast cultural diversity.
The cuisine of a particular neighborhood often represents its
demographics; for example, Astoria hosts many Greek restaurants, in
keeping with its traditionally Greek population. Jackson Heights is
known for its prominent Indian cuisine and also many Latin American
eateries.

\section{Government}\label{government}

\begin{itemize}
\item
  \emph{Since New York City's consolidation in 1898, Queens has been
  governed by the New York City Charter that provides for a strong
  mayor-council system.}
\item
  \emph{The Queens Library is governed by a 19-member Board of Trustees,
  who are appointed by the Mayor of New York City and the Borough
  President of Queens.}
\item
  \emph{Although Queens is heavily Democratic, it is considered a swing
  county in New York politics.}
\end{itemize}

Since New York City's consolidation in 1898, Queens has been governed by
the New York City Charter that provides for a strong mayor-council
system. The centralized New York City government is responsible for
public education, correctional institutions, public safety, recreational
facilities, sanitation, water supply, and welfare services in Queens.
The Queens Library is governed by a 19-member Board of Trustees, who are
appointed by the Mayor of New York City and the Borough President of
Queens.

Since 1990 the Borough President has acted as an advocate for the
borough at the mayoral agencies, the City Council, the New York state
government, and corporations. Queens' Borough President is Melinda Katz,
elected in November 2013 as a Democrat with 80.3\% of the vote. Queens
Borough Hall is the seat of government and is located in Kew Gardens.

The Democratic Party holds most public offices. Sixty-three percent of
registered Queens voters are Democrats. Local party platforms center on
affordable housing, education and economic development. Controversial
political issues in Queens include development, noise, and the cost of
housing.

Each of the city's five counties has its own criminal court system and
District Attorney, the chief public prosecutor who is directly elected
by popular vote. Richard A. Brown, who ran on both the Republican and
Democratic Party tickets, was the District Attorney of Queens County
from 1991-2018. The new DA as of June 2019 is Tiffany Caban.\\
Queens has 12 seats on the New York City Council, the second largest
number among the five boroughs. It is divided into 14 community
districts, each served by a local Community Board. Community Boards are
representative bodies that field complaints and serve as advocates for
local residents.

Although Queens is heavily Democratic, it is considered a swing county
in New York politics. Republican political candidates who do well in
Queens usually win citywide or statewide elections. Republicans such as
former Mayors Rudolph Giuliani and Michael Bloomberg won majorities in
Queens. Republican State Senator Serphin Maltese represented a district
in central and southern Queens for twenty years until his defeat in 2008
by Democratic City Councilman Joseph Addabbo. In 2002, Queens voted
against incumbent Republican Governor of New York George Pataki in favor
of his Democratic opponent, Carl McCall by a slim margin.

On the national level, Queens has not voted for a Republican candidate
in a presidential election since 1972, when Queens voters chose Richard
Nixon over George McGovern. Since the 1996 presidential election,
Democratic presidential candidates have received over 70\% of the
popular vote in Queens.

\section{Representatives in the U.S.
Congress}\label{representatives-in-the-u.s.-congress}

\begin{itemize}
\item
  \emph{Hakeem Jeffries (first elected in 2012) represents New York's
  8th congressional district, which includes the southwest Queens
  neighborhoods of Ozone Park and Howard Beach.}
\item
  \emph{Carolyn Maloney (first elected in 1992) represents New York's
  12th congressional district, which includes the western Queens
  neighborhoods of Astoria, Long Island City, Sunnyside, and Maspeth.}
\end{itemize}

In 2018, seven Democrats represented Queens in the United States House
of Representatives.

Thomas Suozzi (first elected in 2016) represents New York's 3rd
congressional district, which covers the northeast Queens neighborhoods
of Little Neck, Whitestone, Glen Oaks, and Floral Park. The district
also covers the North Shore of Nassau County.

Gregory Meeks (first elected in 1998) represents New York's 5th
congressional district, which covers the entire Rockaway Peninsula as
well as the southeast Queens neighborhoods of Broad Channel, Cambria
Heights, Hollis, Jamaica, Laurelton, Queens Village, Rosedale, Saint
Albans, Springfield Gardens, and South Ozone Park. The district also
includes John F. Kennedy International Airport.

Grace Meng (first elected in 2012) represents New York's 6th
congressional district, which includes the central and eastern Queens
neighborhoods of Auburndale, Bayside, Elmhurst, Flushing, Forest Hills,
Glendale, Kew Gardens, Maspeth, Middle Village, Murray Hill, and Rego
Park.

Nydia Velázquez (first elected in 1992) represents New York's 7th
congressional district, which includes the southwest Queens
neighborhoods of Maspeth, Ridgewood, and Woodhaven. The district also
covers central and western Brooklyn and the Lower East Side of
Manhattan.

Hakeem Jeffries (first elected in 2012) represents New York's 8th
congressional district, which includes the southwest Queens
neighborhoods of Ozone Park and Howard Beach. The district also covers
central and southern Brooklyn.

Carolyn Maloney (first elected in 1992) represents New York's 12th
congressional district, which includes the western Queens neighborhoods
of Astoria, Long Island City, Sunnyside, and Maspeth. The district also
covers the East Side of Manhattan.

Alexandria Ocasio-Cortez (first elected in 2018) represents New York's
14th congressional district, which includes the northwest Queens
neighborhoods of Astoria, College Point, Corona, East Elmhurst, Jackson
Heights, Woodside, and Elmhurst. The district also covers the East
Bronx.

\section{Economy}\label{economy}

\begin{itemize}
\item
  \emph{Queens has the second-largest economy of New York City's five
  boroughs, following Manhattan.}
\item
  \emph{In 2004, Queens had 15.2\% (440,310) of all private sector jobs
  in New York City and 8.8\% of private sector wages.}
\item
  \emph{Queens is home to two of the three major New York City area
  airports, JFK International Airport and LaGuardia Airport.}
\end{itemize}

Queens has the second-largest economy of New York City's five boroughs,
following Manhattan. In 2004, Queens had 15.2\% (440,310) of all private
sector jobs in New York City and 8.8\% of private sector wages. Queens
has the most diversified economy of the five boroughs, with occupations
spread relatively evenly across the health care, retail trade,
manufacturing, construction, transportation, and film and television
production sectors, such that no single sector is overwhelmingly
dominant.

The diversification in Queens' economy is reflected in the large amount
of employment in the export-oriented portions of its economy---such as
transportation, manufacturing, and business services---that serve
customers outside the region. This accounts for more than 27\% of all
Queens jobs and offers an average salary of \$43,727, 14\% greater than
that of jobs in the locally oriented sector.

The borough's largest employment sector---trade, transportation, and
utilities---accounted for nearly 30\% of all jobs in 2004. Queens is
home to two of the three major New York City area airports, JFK
International Airport and LaGuardia Airport. These airports are among
the busiest in the world, leading the airspace above Queens to be the
most congested in the country. This airline industry is particularly
important to the economy of Queens, providing almost one quarter of the
sector's employment and more than 30\% of the sector's wages.

Education and health services is the next largest sector in Queens and
comprised almost 24\% of the borough's jobs in 2004. The manufacturing
and construction industries in Queens are the largest of the City and
account for nearly 17\% of the borough's private sector jobs. Comprising
almost 17\% of the jobs in Queens is the information, financial
activities, and business and professional services sectors.

As of 2003{[}update{]}, Queens had almost 40,000 business
establishments. Small businesses act as an important part of the
borough's economic vitality with two thirds of all business employing
between one and four people.

Several large companies have their headquarters in Queens, including
watchmaker Bulova, based in East Elmhurst; internationally renowned
piano manufacturer Steinway \& Sons in Astoria; Glacéau, the makers of
Vitamin Water, headquartered in Whitestone; and JetBlue Airways, an
airline based in Long Island City.

Long Island City is a major manufacturing and back office center.
Flushing is a major commercial hub for Chinese American and Korean
American businesses, while Jamaica is the major civic and transportation
hub for the borough.

\section{Sports}\label{sports}

\begin{itemize}
\item
  \emph{Belmont Park Racetrack is mostly in Nassau County, New York
  however a section of the property including the Belmont Park station
  on the Long Island Rail Road is in Queens.}
\item
  \emph{Shea Stadium, the former home of the Mets and the New York Jets
  of the National Football League, as well as the temporary home of the
  New York Yankees and the New York Giants Football Team stood where
  Citi Field's parking lot is now located, operating from 1964 to 2008.}
\end{itemize}

Citi Field is a 41,922-seat stadium opened in April 2009 in Flushing
Meadows--Corona Park that is the home ballpark of the New York Mets of
Major League Baseball. Shea Stadium, the former home of the Mets and the
New York Jets of the National Football League, as well as the temporary
home of the New York Yankees and the New York Giants Football Team stood
where Citi Field's parking lot is now located, operating from 1964 to
2008.

The US Open tennis tournament has been played since 1978 at the USTA
Billie Jean King National Tennis Center, located just south of Citi
Field. With a capacity of 23,771, Arthur Ashe Stadium is the biggest
tennis stadium in the world. The US Open was formerly played at the West
Side Tennis Club in Forest Hills. South Ozone Park is the home of
Aqueduct Racetrack, which is operated by the New York Racing Association
and offers Thoroughbred horse-racing from late October/early November
through April. Belmont Park Racetrack is mostly in Nassau County, New
York however a section of the property including the Belmont Park
station on the Long Island Rail Road is in Queens.

\section{Transportation}\label{transportation}

\begin{itemize}
\item
  \emph{According to the 2010 Census, 36\% of all Queens households did
  not own a car; the citywide rate is 53\%.}
\end{itemize}

According to the 2010 Census, 36\% of all Queens households did not own
a car; the citywide rate is 53\%. Therefore, mass transit is also used.

\section{Airports}\label{airports}

\begin{itemize}
\item
  \emph{Owned by the City of New York and managed since 1947 by the Port
  Authority of New York and New Jersey, the airport's runways and six
  terminals cover an area of 4,930 acres (2,000~ha) on Jamaica Bay in
  southeastern Queens.}
\item
  \emph{Queens has crucial importance in international and interstate
  air traffic, with two of the New York metropolitan area's three major
  airports located there.}
\end{itemize}

Queens has crucial importance in international and interstate air
traffic, with two of the New York metropolitan area's three major
airports located there.

John F. Kennedy International Airport, with 27.4 million international
passengers in 2014 (of 53.2 million passengers, overall), is the busiest
airport in the United States by international passenger traffic. Owned
by the City of New York and managed since 1947 by the Port Authority of
New York and New Jersey, the airport's runways and six terminals cover
an area of 4,930 acres (2,000~ha) on Jamaica Bay in southeastern Queens.
The airport's original official name was New York International Airport,
although it was commonly known as Idlewild, with the name changed to
Kennedy in December 1963 to honor the recently assassinated president.

LaGuardia Airport is located in Flushing, in northern Queens, on
Flushing Bay. Originally opened in 1939, the airport's two runways and
four terminals cover 680 acres (280~ha), serving 28.4 million passengers
in 2015. In 2014, citing outdated conditions in the airport's terminals,
Vice President Joe Biden compared LaGuardia Airport to a "third world
country". In 2015, the Port Authority of New York and New Jersey began a
\$4 billion project to completely reconstruct LaGuardia Airport's
terminals and entryways, with an estimated completion in 2021.

\section{Public transportation}\label{public-transportation}

\begin{itemize}
\item
  \emph{A streetcar line connecting Queens with Brooklyn was proposed by
  the city in February 2016.}
\item
  \emph{About 100 local bus routes operate within Queens, and another 20
  express routes shuttle commuters between Queens and Manhattan, under
  the MTA New York City Bus and MTA Bus brands.}
\item
  \emph{Twelve New York City Subway routes traverse Queens, serving 81
  stations on seven main lines.}
\end{itemize}

Twelve New York City Subway routes traverse Queens, serving 81 stations
on seven main lines. The A, G, J/Z, and M routes connect Queens to
Brooklyn without going through Manhattan first. The F, M, N, and R
trains connect Queens and Brooklyn via Manhattan, while the E, W, and
7/\textless{}7\textgreater{} trains connect Queens to Manhattan only.
Trains on the M service go through Queens twice in the same trip; both
of its full-length termini, in Middle Village and Forest Hills, are in
Queens.

A commuter train system, the Long Island Rail Road, operates 22 stations
in Queens with service to Manhattan, Brooklyn, and Long Island. Jamaica
station is a hub station where all the lines in the system but one (the
Port Washington Branch) converge. It is the busiest commuter rail hub in
the United States. There are also several stations where LIRR passengers
can transfer to the subway. Sunnyside Yard is used to store Amtrak
intercity and NJ Transit commuter trains from Penn Station in Manhattan.
The US\$11.1 billion East Side Access project, which will bring LIRR
trains to Grand Central Terminal in Manhattan, is under construction and
is scheduled to open in 2022; this project will create a new train
tunnel beneath the East River, connecting Long Island City in Queens
with the East Side of Manhattan.

The elevated AirTrain people mover system connects JFK International
Airport to the New York City Subway and the Long Island Rail Road along
the Van Wyck Expressway; a separate AirTrain system is planned alongside
the Grand Central Parkway to connect LaGuardia Airport to these transit
systems. Plans were announced in July 2015 to entirely rebuild LaGuardia
Airport itself in a multibillion-dollar project to replace its aging
facilities, and this project would accommodate the new AirTrain
connection.

About 100 local bus routes operate within Queens, and another 20 express
routes shuttle commuters between Queens and Manhattan, under the MTA New
York City Bus and MTA Bus brands.

A streetcar line connecting Queens with Brooklyn was proposed by the
city in February 2016. The planned timeline calls for service to begin
around 2024.

\includegraphics[width=5.50000in,height=2.53458in]{media/image5.JPG}\\
\emph{Newtown Creek with the Midtown Manhattan skyline in the
background.}

\section{Water transit}\label{water-transit}

\begin{itemize}
\item
  \emph{One year-round scheduled ferry service connects Queens and
  Manhattan.}
\item
  \emph{The ferry opened in May 2017, with the Queens neighborhoods of
  Rockaway and Astoria served by their eponymous routes.}
\item
  \emph{New York Water Taxi operates service across the East River from
  Hunters Point in Long Island City to Manhattan at 34th Street and
  south to Pier 11 at Wall Street.}
\end{itemize}

One year-round scheduled ferry service connects Queens and Manhattan.
New York Water Taxi operates service across the East River from Hunters
Point in Long Island City to Manhattan at 34th Street and south to Pier
11 at Wall Street. In 2007, limited weekday service was begun between
Breezy Point, the westernmost point in the Rockaways, to Pier 11 via the
Brooklyn Army Terminal. Summertime weekend service provides service from
Lower Manhattan and southwest Brooklyn to the peninsula's Gateway
beaches.

In the aftermath of Hurricane Sandy on October 29, 2012, massive
infrastructure damage to the IND Rockaway Line (A train) south of the
Howard Beach -- JFK Airport station severed all direct subway
connections between the Rockaway Peninsula and Broad Channel, Queens and
the Queens mainland for many months. Ferry operator SeaStreak began
running a city-subsidized ferry service between a makeshift ferry slip
at Beach 108th Street and Beach Channel Drive in Rockaway Park, Queens,
and Pier 11/Wall Street, then continuing on to the East 34th Street
Ferry Landing.

In August 2013, a stop was added at Brooklyn Army Terminal. Originally
intended as just a stopgap alternative transportation measure until
subway service was restored to the Rockaways, the ferry proved to be
popular with both commuters and tourists and was extended several times,
as city officials evaluated the ridership numbers to determine whether
to establish the service on a permanent basis. Between its inception and
December 2013, the service had carried close to 200,000 riders.

When the city government announced its budget in late June 2014 for the
upcoming fiscal year beginning July 1, the ferry only received a \$2
million further appropriation, enough to temporarily extend it again
through October, but did not receive the approximately \$8 million
appropriation needed to keep the service running for the full fiscal
year. Despite last-minute efforts by local transportation advocates,
civic leaders and elected officials, ferry service ended on October 31,
2014. They promised to continue efforts to have the service restored.

In February 2015, Mayor Bill de Blasio announced that the city
government would begin a citywide ferry service called NYC Ferry to
extend ferry transportation to communities in the city that have been
traditionally undeserved by public transit. The ferry opened in May
2017, with the Queens neighborhoods of Rockaway and Astoria served by
their eponymous routes. A third route, the East River Ferry, serves
Hunter's Point South.

\section{Roads}\label{roads}

\includegraphics[width=5.50000in,height=4.12500in]{media/image6.jpg}\\
\emph{Air Train JFK path above the Van Wyck Expressway}

\section{Highways}\label{highways}

\begin{itemize}
\item
  \emph{The Belt Parkway begins at the Gowanus Expressway in Brooklyn,
  and extends east into Queens, past Aqueduct Racetrack and JFK
  Airport.}
\item
  \emph{Queens is traversed by three trunk east-west highways.}
\item
  \emph{The Long Island Expressway (Interstate 495) runs from the Queens
  Midtown Tunnel on the west through the borough to Nassau County on the
  east.}
\end{itemize}

Queens is traversed by three trunk east-west highways. The Long Island
Expressway (Interstate 495) runs from the Queens Midtown Tunnel on the
west through the borough to Nassau County on the east. The Grand Central
Parkway, whose western terminus is the Triborough Bridge, extends east
to the Queens/Nassau border, where its name changes to the Northern
State Parkway. The Belt Parkway begins at the Gowanus Expressway in
Brooklyn, and extends east into Queens, past Aqueduct Racetrack and JFK
Airport. On its eastern end at the Queens/Nassau border, it splits into
the Southern State Parkway which continues east, and the Cross Island
Parkway which turns north.

There are also several major north-south highways in Queens, including
the Brooklyn-Queens Expressway (Interstate 278), the Van Wyck Expressway
(Interstate 678), the Clearview Expressway (Interstate 295), and the
Cross Island Parkway.

\section{Streets}\label{streets}

\begin{itemize}
\item
  \emph{Broad Channel's streets were a continuation of the mainland
  Queens grid in the 1950s; formerly the highest numbered avenue in
  Queens was 208th Avenue rather today's 165th Avenue in Howard Beach \&
  Hamilton Beach.}
\item
  \emph{The streets of Queens are laid out in a semi-grid system, with a
  numerical system of street names (similar to Manhattan and the
  Bronx).}
\item
  \emph{The Topographical Bureau, Borough of Queens, worked out the
  details.}
\end{itemize}

The streets of Queens are laid out in a semi-grid system, with a
numerical system of street names (similar to Manhattan and the Bronx).
Nearly all roadways oriented north-south are "Streets", while east-west
roadways are "Avenues", beginning with the number 1 in the west for
Streets and in the north for Avenues. In some parts of the borough,
several consecutive streets may share numbers (for instance, 72nd Street
followed by 72nd Place and 72nd Lane, or 52nd Avenue followed by 52nd
Road, 52nd Drive, and 52nd Court), often causing confusion for
non-residents. In addition, incongruous alignments of street grids,
unusual street paths due to geography, or other circumstances often lead
to the skipping of numbers (for instance, on Ditmars Boulevard, 70th
Street is followed by Hazen Street which is followed by 49th Street).
Numbered roads tend to be residential, although numbered commercial
streets are not rare. A fair number of streets that were country roads
in the 18th and 19th centuries (especially major thoroughfares such as
Northern Boulevard, Queens Boulevard, Hillside Avenue, and Jamaica
Avenue) carry names rather than numbers, typically though not uniformly
called "Boulevards" or "Parkways".

Queens house numbering was designed to provide convenience in locating
the address itself; the first half of a number in a Queens address
refers to the nearest cross street, the second half refers to the house
or lot number from where the street begins from that cross street,
followed by the name of the street itself. For example, to find an
address in Queens, 14-01 120th Street, one could ascertain from the
address structure itself that the listed address is at the intersection
of 14th Avenue and 120th Street, and that the address must be closest to
14th Avenue rather than 15th Avenue, as it is the first lot on the
block. This pattern doesn't stop when a street is named, assuming that
there is an existing numbered cross-street. For example, Queens College
is situated at 65--30 Kissena Boulevard, and is so named because the
cross-street closest to the entrance is 65th Avenue.

Many of the village street grids of Queens had only worded names, some
were numbered according to local numbering schemes, and some had a mix
of words and numbers. In the early 1920s a "Philadelphia Plan" was
instituted to overlay one numbered system upon the whole borough. The
Topographical Bureau, Borough of Queens, worked out the details. Subway
stations were only partly renamed, and some, including those along the
IRT Flushing Line (7 and \textless{}7\textgreater{}​ trains), now share
dual names after the original street names. In 2012, some numbered
streets in the Douglaston Hill Historic District were renamed to their
original names, with 43rd Avenue becoming Pine Street.

The Rockaway Peninsula does not follow the same system as the rest of
the borough and has its own numbering system. Streets are numbered in
ascending order heading west from near the Nassau County border, and are
prefixed with the word "Beach." Streets at the easternmost end, however,
are nearly all named. Bayswater, which is on Jamaica Bay, has its
numbered streets prefixed with the word "Bay" rather than "Beach".
Another deviation from the norm is Broad Channel; it maintains the
north-south numbering progression but uses only the suffix "Road," as
well as the prefixes "West" and "East," depending on location relative
to Cross Bay Boulevard, the neighborhood's major through street. Broad
Channel's streets were a continuation of the mainland Queens grid in the
1950s; formerly the highest numbered avenue in Queens was 208th Avenue
rather today's 165th Avenue in Howard Beach \& Hamilton Beach. The other
exception is the neighborhood of Ridgewood, which for the most part
shares a grid and house numbering system with the Brooklyn neighborhood
of Bushwick. The grid runs east-west from the LIRR Bay Ridge Branch
right-of-way to Flushing Avenue; and north-south from Forest Avenue in
Ridgewood to Bushwick Avenue in Brooklyn before adjusting to meet up
with the Bedford-Stuyvesant grid at Broadway. All streets on the grid
have names.

\section{Bridges and tunnels}\label{bridges-and-tunnels}

\begin{itemize}
\item
  \emph{A lesser bridge connects Grand Avenue in Queens to Grand Street
  in Brooklyn.}
\item
  \emph{Queens is connected to Manhattan Island by the Triborough
  Bridge, the Queensboro Bridge, and the Queens--Midtown Tunnel, as well
  as to Roosevelt Island by the Roosevelt Island Bridge.}
\item
  \emph{Greenpoint Avenue Bridge) connects the sections of Greenpoint
  Avenue in Greenpoint and Long Island City.}
\end{itemize}

Queens is connected to the Bronx by the Bronx--Whitestone Bridge, the
Throgs Neck Bridge, the Triborough Bridge (also known as the Robert F.
Kennedy Bridge), and the Hell Gate Bridge. Queens is connected to
Manhattan Island by the Triborough Bridge, the Queensboro Bridge, and
the Queens--Midtown Tunnel, as well as to Roosevelt Island by the
Roosevelt Island Bridge.

While most of the Queens/Brooklyn border is on land, the Kosciuszko
Bridge crosses the Newtown Creek connecting Maspeth to Greenpoint,
Brooklyn. The Pulaski Bridge connects McGuinness Boulevard in Greenpoint
to 11th Street, Jackson Avenue, and Hunters Point Avenue in Long Island
City. The J. J. Byrne Memorial Bridge (a.k.a. Greenpoint Avenue Bridge)
connects the sections of Greenpoint Avenue in Greenpoint and Long Island
City. A lesser bridge connects Grand Avenue in Queens to Grand Street in
Brooklyn.

The Cross Bay Veterans Memorial Bridge, built in 1939, traverses Jamaica
Bay to connect the Rockaway Peninsula to Broad Channel and the rest of
Queens. Constructed in 1937, the Marine Parkway--Gil Hodges Memorial
Bridge links Flatbush Avenue, Brooklyn's longest thoroughfare, with
Jacob Riis Park and the western end of the Peninsula. Both crossings
were built and continue to be operated by what is now known as MTA
Bridges and Tunnels. The IND Rockaway Line parallels the Cross Bay, has
a mid-bay station at Broad Channel which is just a short walk from the
Jamaica Bay Wildlife Refuge, now part of Gateway National Recreation
Area and a major stop on the Atlantic Flyway.

\section{Education}\label{education}

\section{Elementary and secondary
education}\label{elementary-and-secondary-education}

\begin{itemize}
\item
  \emph{Located in the York College, City University of New York Campus
  in Jamaica, the Queens High School for the Sciences at York College,
  which places emphasis on both science and mathematics, ranks as one of
  the best high schools in both the state and the country.}
\item
  \emph{One of the nine Specialized High Schools in New York City is
  located in Queens.}
\end{itemize}

Elementary and secondary school education in Queens is provided by a
vast number of public and private institutions. Public schools in the
borough are managed by the New York City Department of Education, the
largest public school system in the United States. Most private schools
are affiliated to or identify themselves with the Roman Catholic or
Jewish religious communities. Townsend Harris High School is a Queens
public magnet high school for the humanities consistently ranked as
among the top 100 high schools in the United States. One of the nine
Specialized High Schools in New York City is located in Queens. Located
in the York College, City University of New York Campus in Jamaica, the
Queens High School for the Sciences at York College, which places
emphasis on both science and mathematics, ranks as one of the best high
schools in both the state and the country. It is one of the smallest
Specialized High Schools that requires an entrance exam, the Specialized
High Schools Admissions Test. The school has a student body of around
400 students.

\section{Postsecondary institutions}\label{postsecondary-institutions}

\begin{itemize}
\item
  \emph{Queensborough Community College, originally part of the State
  University of New York, is in Bayside and is now part of CUNY.}
\item
  \emph{Queens College is also the host of CUNY's law school.}
\item
  \emph{The Queens College Campus is also the home of Townsend Harris
  High School and the Queens College School for Math, Science, and
  Technology (PS/IS 499).}
\item
  \emph{Its presence underscores the importance of aviation to the
  Queens economy.}
\end{itemize}

LaGuardia Community College, part of the City University of New York
(CUNY), is known as "The World's Community College" for its diverse
international student body representing more than 150 countries and
speaking over 100 languages. The college has been named a National
Institution of Excellence by the Policy Center on the First Year of
College and one of the top three largest community colleges in the
United States. The college hosts the LaGuardia and Wagner Archives.

Queens College is one of the elite colleges in the CUNY system.
Established in 1937 to offer a strong liberal arts education to the
residents of the borough, Queens College has over 16,000 students
including more than 12,000 undergraduates and over 4,000 graduate
students. Students from 120 different countries speaking 66 different
languages are enrolled at the school, which is located in Flushing.
Queens College is also the host of CUNY's law school. The Queens College
Campus is also the home of Townsend Harris High School and the Queens
College School for Math, Science, and Technology (PS/IS 499).

Queensborough Community College, originally part of the State University
of New York, is in Bayside and is now part of CUNY. It prepares students
to attend senior colleges mainly in the CUNY system.

St. John's University is a private, coeducational Roman Catholic
university founded in 1870 by the Vincentian Fathers. With over 19,000
students, St. John's is known for its pharmacy, business and law
programs as well as its men's basketball and soccer teams.

Vaughn College of Aeronautics and Technology is a private, cutting edge,
degree granting institution located across the Grand Central Parkway
from LaGuardia Airport. Its presence underscores the importance of
aviation to the Queens economy.

York College is one of CUNY's leading general-purpose liberal arts
colleges, granting bachelor's degrees in more than 40 fields, as well as
a combined BS/MS degree in Occupational Therapy. Noted for its Health
Sciences Programs York College is also home to the Northeast Regional
Office of the Food and Drug Administration.

\includegraphics[width=5.50000in,height=4.60894in]{media/image7.JPG}\\
\emph{A branch of the Queens Library in Flushing.}

\section{Queens Library}\label{queens-library}

\begin{itemize}
\item
  \emph{Dating back to the foundation of the first Queens library in
  Flushing in 1858, the Queens Borough Public Library is one of the
  largest public library systems in the United States.}
\item
  \emph{Separate from the New York Public Library, it is composed of 63
  branches throughout the borough.}
\item
  \emph{The Queens Borough Public Library is the public library system
  for the borough and one of three library systems serving New York
  City.}
\end{itemize}

The Queens Borough Public Library is the public library system for the
borough and one of three library systems serving New York City. Dating
back to the foundation of the first Queens library in Flushing in 1858,
the Queens Borough Public Library is one of the largest public library
systems in the United States. Separate from the New York Public Library,
it is composed of 63 branches throughout the borough. In fiscal year
2001, the Library achieved a circulation of 16.8 million. First in
circulation in New York State since 1985, the Library has maintained the
highest circulation of any city library in the country since 1985 and
the highest circulation of any library in the nation since 1987. The
Library maintains collections in many languages, including Spanish,
Chinese, Korean, Russian, Haitian Creole, Polish, and six Indic
languages, as well as smaller collections in 19 other languages.

\section{Notable people}\label{notable-people}

\begin{itemize}
\item
  \emph{Actress Mae West also lived in Queens.}
\item
  \emph{Queens has also been home to athletes such as professional
  basketball player Rafer Alston Basketball players Kareem Abdul-Jabbar
  and Metta World Peace were both born in Queens, as was Olympic athlete
  Bob Beamon.}
\item
  \emph{Various public figures have grown up or lived in Queens.}
\item
  \emph{Mafia boss John Gotti lived in Queens for many years.}
\end{itemize}

Various public figures have grown up or lived in Queens. Musicians who
have lived in the borough include rappers LL Cool J, A Tribe Called
Quest, Nas, Mobb Deep, Action Bronson, Onyx, Ja Rule, 50 Cent,
Run--D.M.C., Nicki Minaj, Rich The Kid; Jason Griffiths Music Executive
(Capitol Records) singers Nadia Ali, and Tony Bennett;, jazz greats
Louis Armstrong and Norman Mapp both resided in Corona, rock duo Simon
\& Garfunkel; and guitarists Scott Ian and Johnny Ramone. K-pop rapper
Mark Lee from the boy group NCT grew up in Queens before moving to
Canada. Actors such as Adrien Brody, and Lucy Liu and Idina Menzel have
been born and/or raised in Queens. Actress Mae West also lived in
Queens. Writers from Queens include John Guare (The House of Blue
Leaves) and Laura Z. Hobson (Gentleman's Agreement). Physician Joshua
Prager was born in Whitestone. Mafia boss John Gotti lived in Queens for
many years.

Donald Trump, a businessman who became the 45th President of the United
States, was born in Jamaica Hospital Medical Center and raised at 81-15
Wareham Place in Jamaica Estates, later moving to Midland Parkway. He
was preceded in the White House by former First Ladies Nancy Reagan, who
lived in Flushing as a child and Barbara Bush, who was born at Booth
Memorial Hospital in Flushing. Theodore Roosevelt, the 26th President,
lived at Sagamore Hill in Oyster Bay from the mid-1880s until he died;
the area was considered part of Queens until the formation of
neighboring Nassau County in 1899.\\
Queens has also been home to athletes such as professional basketball
player Rafer Alston Basketball players Kareem Abdul-Jabbar and Metta
World Peace were both born in Queens, as was Olympic athlete Bob Beamon.
Tennis star John McEnroe was born in Douglaston. Hall of Fame baseball
pitcher Whitey Ford grew up in Astoria.

Queens has also served as a setting for fictional characters, one of the
more famous being Peter Parker/Spider-Man from Marvel Comics. The
character grew up in Forest Hills with his Aunt May and Uncle Ben.

\section{See also}\label{see-also}

\begin{itemize}
\item
  \emph{List of counties in New York}
\item
  \emph{National Register of Historic Places listings in Queens County,
  New York}
\end{itemize}

List of counties in New York

National Register of Historic Places listings in Queens County, New York

\section{Notes}\label{notes}

\section{References}\label{references}

\section{Further reading}\label{further-reading}

\begin{itemize}
\item
  \emph{City Limits: A Social History of Queens (Kendall/Hunt Publishing
  Company, 1983)}
\item
  \emph{History of Queens County, New York (WW Munsell, 1882)}
\item
  \emph{An Annotated Bibliography of the History of Queens County, New
  York (Queens College, 1977) 218 pages}
\end{itemize}

Copquin, Claudia Gryvatz. The Neighborhoods of Queens (Yale University
Press, 2007); Guide to 99 neighborhoods

Glascock, Mary A. An Annotated Bibliography of the History of Queens
County, New York (Queens College, 1977) 218 pages

Lieberman, Janet E. and Richard K. Lieberman. City Limits: A Social
History of Queens (Kendall/Hunt Publishing Company, 1983)

McGovern, Brendan, and John W. Frazier. "Evolving Ethnic Settlements in
Queens: Historical and Current Forces Reshaping Human Geography." Focus
on Geography (2015) 58\#1 pp: 11-26.

Miyares, Ines M. "From Exclusionary Covenant to Ethnic Hyperdiversity in
Jackson Heights, Queens*." Geographical Review (2004) 94\#4 pp: 462-483.

History of Queens County, New York (WW Munsell, 1882)

\section{External links}\label{external-links}

\begin{itemize}
\item
  \emph{Queens Buzz}
\item
  \emph{Official History Page of the Queens Borough President's Office}
\item
  \emph{Long list compiled by the Queens Tribune.}
\item
  \emph{They Came from Queens.}
\end{itemize}

Official History Page of the Queens Borough President's Office

La Guardia and Wagner Archives/Queens Local History Collection

They Came from Queens. Long list compiled by the Queens Tribune.

Queens Buzz
