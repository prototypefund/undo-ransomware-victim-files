\textbf{From Wikipedia, the free encyclopedia}

https://en.wikipedia.org/wiki/Intraoperative\_electron\_radiation\_therapy\\
Licensed under CC BY-SA 3.0:\\
https://en.wikipedia.org/wiki/Wikipedia:Text\_of\_Creative\_Commons\_Attribution-ShareAlike\_3.0\_Unported\_License

\section{Intraoperative electron radiation
therapy}\label{intraoperative-electron-radiation-therapy}

\begin{itemize}
\item
  \emph{Intraoperative electron radiation therapy is the application of
  electron radiation directly to the residual tumor or tumor bed during
  cancer surgery.}
\item
  \emph{Electron beams are useful for intraoperative radiation treatment
  because, depending on the electron energy, the dose falls off rapidly
  behind the target site, therefore sparing underlying healthy tissue.}
\end{itemize}

Intraoperative electron radiation therapy is the application of electron
radiation directly to the residual tumor or tumor bed during cancer
surgery. Electron beams are useful for intraoperative radiation
treatment because, depending on the electron energy, the dose falls off
rapidly behind the target site, therefore sparing underlying healthy
tissue. IOERT has been called "precision radiotherapy," because the
physician has direct visualization of the tumor and can exclude normal
tissue from the field while protecting critical structures within the
field and underlying the target volume. One advantage of IOERT is that
it can be given at the time of surgery when microscopic residual tumor
cells are most vulnerable to destruction. Also, IOERT is often used in
combination with external beam radiotherapy (EBR) because it results in
less integral doses and shorter treatment times.

\section{Medical uses}\label{medical-uses}

\begin{itemize}
\item
  \emph{IOERT has a long history of clinical applications, with
  promising results, in the management of solid tumors (e.g., pancreatic
  cancer, locally advanced and recurrent rectal cancer, breast tumors,
  sarcomas, and selected gynaecologic and genitourinary malignancies,
  neuroblastomas and brain tumors.}
\item
  \emph{The following is a list of disease sites currently treated by
  IOERT:}
\end{itemize}

IOERT has a long history of clinical applications, with promising
results, in the management of solid tumors (e.g., pancreatic cancer,
locally advanced and recurrent rectal cancer, breast tumors, sarcomas,
and selected gynaecologic and genitourinary malignancies, neuroblastomas
and brain tumors. In virtually every tumor site, electron IORT improves
local control, reducing the need for additional surgeries or
interventions. The following is a list of disease sites currently
treated by IOERT:

\section{Breast cancer}\label{breast-cancer}

\begin{itemize}
\item
  \emph{Other studies show that IOERT provides acceptable results when
  treating breast cancer in low-risk patients.}
\item
  \emph{More research is needed for defining the optimal dose of IOERT,
  alone or in combination with EBRT, and for determining when it may be
  appropriate to use it as part of the treatment for higher risk
  patients.}
\end{itemize}

Since 1975, breast cancer rates have declined in the U.S., largely due
to mammograms and the use of adjuvant treatments such as radiotherapy.
Local recurrence rates are greatly reduced by postoperative
radiotherapy, which translates into improved survival: Preventing four
local recurrences can prevent one breast cancer death. In one of the
largest published studies so far called (ELIOT), researchers found that
after treating 574 patients with full-dose IOERT with 21~Gy, at a median
follow-up of 20~months, there was an in-breast tumor recurrence rate of
only 1.05\%. Other studies show that IOERT provides acceptable results
when treating breast cancer in low-risk patients. More research is
needed for defining the optimal dose of IOERT, alone or in combination
with EBRT, and for determining when it may be appropriate to use it as
part of the treatment for higher risk patients.

\section{Colorectal cancer}\label{colorectal-cancer}

\begin{itemize}
\item
  \emph{Over the past 30 years, treatment of locally advanced colorectal
  cancer has evolved, particularly in the area of local control --
  stopping the spread of cancer from the tumor site.}
\item
  \emph{IOERT shows promising results.}
\end{itemize}

Over the past 30 years, treatment of locally advanced colorectal cancer
has evolved, particularly in the area of local control -- stopping the
spread of cancer from the tumor site. IOERT shows promising results.
When combined with preoperative external beam irradiation plus
chemotherapy and maximal surgical resection, it may be a successful
component in the treatment of high-risk patients with locally advanced
primary or locally recurrent cancers.

\section{Gynecological cancer}\label{gynecological-cancer}

\begin{itemize}
\item
  \emph{Further research into radiation doses and how to best combine
  IOERT with other interventions will help to define the sequencing of
  treatment and the patients who would most benefit from receiving
  electron IORT, as part of the multimodality treatment of this
  disease.}
\item
  \emph{Studies suggest that electron IORT may play an important and
  useful role in the treatment of patients with locally advanced and
  recurrent gynecologic cancers, especially for patients with locally
  recurrent cancer after treatment for their primary lesion.}
\end{itemize}

Studies suggest that electron IORT may play an important and useful role
in the treatment of patients with locally advanced and recurrent
gynecologic cancers, especially for patients with locally recurrent
cancer after treatment for their primary lesion. Further research into
radiation doses and how to best combine IOERT with other interventions
will help to define the sequencing of treatment and the patients who
would most benefit from receiving electron IORT, as part of the
multimodality treatment of this disease.

\section{Head and neck cancer}\label{head-and-neck-cancer}

\begin{itemize}
\item
  \emph{Furthermore, research shows that a boost given by IOERT reduces
  the ability for surviving tumor cells to replicate, creating extra
  time for healing of the surgical wound before EBRT is administered.}
\item
  \emph{IOERT is an effective means of treating locally advanced or
  recurrent head and neck cancers.}
\end{itemize}

Head and neck cancers are often difficult to treat and have a high rate
of recurrence or metastasis. IOERT is an effective means of treating
locally advanced or recurrent head and neck cancers. Furthermore,
research shows that a boost given by IOERT reduces the ability for
surviving tumor cells to replicate, creating extra time for healing of
the surgical wound before EBRT is administered.

\section{Pancreatic cancer}\label{pancreatic-cancer}

\begin{itemize}
\item
  \emph{In the U.S., pancreatic cancer is the fourth leading cause of
  cancer death, even though there has been a slight improvement in
  mortality rates in recent years.}
\item
  \emph{Although the optimal treatment plan remains debated, a
  combination of radiotherapy and chemotherapy is favored in the U.S. As
  part of a multimodality treatment, IOERT appears to reduce local
  recurrence when combined with EBRT, chemoradiation, and surgical
  resection.}
\end{itemize}

In the U.S., pancreatic cancer is the fourth leading cause of cancer
death, even though there has been a slight improvement in mortality
rates in recent years. Although the optimal treatment plan remains
debated, a combination of radiotherapy and chemotherapy is favored in
the U.S. As part of a multimodality treatment, IOERT appears to reduce
local recurrence when combined with EBRT, chemoradiation, and surgical
resection.

\section{Soft tissue sarcomas}\label{soft-tissue-sarcomas}

\begin{itemize}
\item
  \emph{In studies regarding the delivery of therapeutic radiation in
  the limb-sparing approach to extremity soft tissue sarcomas, electron
  IORT has been called `precision radiotherapy' by some, because the
  treating physician has direct visualization of the tumor or surgical
  cavity and can manually exclude normal tissue from the field.}
\end{itemize}

Soft tissue sarcomas can be effectively treated by electron IORT, which
appears to be gaining acceptance as the current practice for sarcomas in
combination with EBRT (preferably preoperative) and maximal resection.
Used together, IOERT and EBRT appear to be improving local control, and
this method is being refined so that it can effectively be used in
combination with other interventions if indicated. In studies regarding
the delivery of therapeutic radiation in the limb-sparing approach to
extremity soft tissue sarcomas, electron IORT has been called `precision
radiotherapy' by some, because the treating physician has direct
visualization of the tumor or surgical cavity and can manually exclude
normal tissue from the field.

\section{History}\label{history}

\begin{itemize}
\item
  \emph{With the Japanese IOERT technique, relatively large single doses
  of radiation were administered during surgery, and most patients
  received no follow-up external radiation treatment.}
\item
  \emph{These advantages made electrons the preferred radiation for
  IOERT.}
\item
  \emph{They built an OR adjacent to the radiation therapy department.}
\end{itemize}

Spanish and German doctors, in 1905 and 1915 respectively, used
intraoperative radiation therapy (IORT) in an attempt to eradicate
residual tumors left behind after surgical resection. However, radiation
equipment in the early twentieth-century could only deliver low energy
X-rays, which had relatively poor penetration; high doses of radiation
could not be applied externally without doing unacceptable damage to
normal tissues. IORT treatments with low energy or "orthovoltage" X-rays
gained advocates throughout the 1930s and 1940s, but the results were
inconsistent. The X-rays penetrated beyond the tumor bed to the normal
tissues beneath, had poor dose distributions, and took a relatively long
time to administer. The technique was largely abandoned in the late
1950s with the advent of megavoltage radiation equipment, which enabled
the delivery of more penetrating external radiation.

In 1965, the modern era of IOERT began in Japan at Kyoto University
where patients were treated with electrons generated by a betatron
Compared with other forms of IORT such as orthovoltage X-ray beams,
electron beams improved IOERT dose distributions, limited penetration
beyond the tumor, and delivered the required dose much more rapidly.
Normal tissue beneath the tumor bed could be protected and shielded, if
required, and the treatment took only a few minutes to deliver. These
advantages made electrons the preferred radiation for IOERT. The
technique gained favor in Japan. Other Japanese hospitals initiated
IOERT using electron beams, principally generated from linear particle
accelerators. At most institutions, patients were operated on in the
operating room (OR) and were transported to the radiation facility for
treatment.

With the Japanese IOERT technique, relatively large single doses of
radiation were administered during surgery, and most patients received
no follow-up external radiation treatment. Even though this reduced the
overall dose that could potentially be delivered to the tumor site, the
early Japanese results were impressive, particularly for gastric cancer.

The Japanese experience was encouraging enough for several U.S. centers
to institute IOERT programs. The first one began at Howard University in
1976 and followed the Japanese protocol of a large, single dose. Howard
built a standard radiation therapy facility with one room that could be
used as an OR as well as for conventional treatment. Because the
radiation equipment was also used for conventional therapy, the
competition for the machine limited the number of patients that could be
scheduled for IOERT.

In 1978, Massachusetts General Hospital (MGH) started an IORT program.
The MGH doctors scheduled one of their conventional therapy rooms for
IOERT one afternoon a week, performed surgery in the OR, and transported
the patient to the radiation therapy room during surgery. This used the
radiation equipment more efficiently and required no additional capital
outlay. However, about 30-50\% of the patients planned for IOERT were
found to be unsuitable candidates for IORT at the time of surgery,
mainly because the disease had spread to adjacent organs. This factor,
combined with the risks and complexities of moving a patient during
surgery, severely limited the number of patients who could be treated
using the MGH method of IOERT. Consequently, conventional fractionated
external beam irradiation was added to the IOERT dose, either prior to
or subsequent to the surgery, in the MGH IOERT program.

The National Cancer Institute (NCI) started an IOERT program in 1979.
Their approach combined maximal surgical resection and IOERT and, in
most cases, did not include conventional external beam therapy as part
of the treatment. Because the NCI protocol relied on IOERT radiation
alone, the IOERT fields were often very large, sometimes requiring two
or three adjacent and overlapping fields to cover the tumor site. While
the NCI results for these very large tumors were not encouraging, they
showed that even the combination of aggressive surgery and large IOERT
fields had acceptable toxicity. Furthermore, they introduced several
technical innovations to IOERT, including the use of television for
simultaneous periscopic viewing of the tumor by the surgical team.

In 1981 the Mayo Clinic tried yet another arrangement. They built an OR
adjacent to the radiation therapy department. Potential IOERT patients
underwent surgery in the regular OR suite. If they were found to be
candidates for IOERT, a second surgical procedure was scheduled in the
OR adjacent to the radiation facility. By scheduling only those patients
known to be suitable for IOERT, they made more efficient use of their
radiation therapy machine, but at the cost of subjecting patients to a
second surgery. Subsequently, the Mayo Clinic remodeled an OR and
installed a conventional radiation therapy machine with its required
massive shield walls, and the clinic now routinely treats over 100 IORT
patients per year. After 1985, Siemens Medical Systems offered a
specialized LINAC for IOERT. It was designed to be used in the OR, but
it weighed more than eight tons and required about 100 tons of
shielding. This proved to be too expensive an approach for the medical
community, and only seven of these specialized units were ever sold.

Dedicating an OR to IOERT increases the number of patients that can be
treated and eliminates the risks of double surgeries and moving a
patient during surgery. It also eliminates the complex logistics
involved in moving patients from the OR to the therapy room and back to
the OR. However, this solution has its own disadvantages: Remodeling an
OR and purchasing an accelerator is expensive. Moreover, IORT is
restricted to that one, specialized OR. Even so, the Mayo Clinic model
demonstrated that when therapy equipment is located within an OR, the
number of IOERT procedures will increase. In 1985, IOERT began in Italy
and involved a specialized method to facilitate surgery followed by
transport to the radiotherapy treatment room. Around the same time in
France, another IOERT method was developed using the Lyon
intra-operative device.

In 1982 the Joint Center for Radiation Therapy (JCRT), at Harvard
Medical School, attempted to reduce the cost of performing IORT in an OR
by using orthovoltage X-rays to provide the intraoperative dose, which
was similar to the approach used in Germany in 1915. But this was less
than ideal. While the shielding costs and the cost and weight of the
equipment compared favorably with conventional electron accelerators,
dose distributions were inferior; treatment times were longer; and bones
received a higher radiation dose. For these reasons, the centers
rejected IO orthovoltage (X-rays) radiation therapy machines. In
addition, these orthovoltage machines (300~kvp) were not designed to be
mobile.

\section{Advent of Portable Linear
Accelerators}\label{advent-of-portable-linear-accelerators}

\begin{itemize}
\item
  \emph{Currently, the Mobetron, LIAC, and NOVAC-7 linear accelerators
  are improving patient care by delivering intraoperative radiation
  electron beam therapy to cancer patients during surgery.}
\item
  \emph{In the 1990s, electron IORT experienced resurgence, due to the
  development of mobile linear accelerators that used electron
  beams---the Mobetron, LIAC, and NOVAC-7 -\/- and the increasing use of
  IOERT to treat breast cancer.}
\end{itemize}

In the 1990s, electron IORT experienced resurgence, due to the
development of mobile linear accelerators that used electron beams---the
Mobetron, LIAC, and NOVAC-7 -\/- and the increasing use of IOERT to
treat breast cancer. Prior to the invention of portable LINACs for
IOERT, clinicians could only treat IORT patients in specially shielded
operating rooms, which were expensive to build, or in a radiotherapy
room, which required transporting the anesthetized patient from the OR
to the LINAC for treatment. These factors were major obstructions to the
widespread adoption of IORT because they added significant cost to
treatment as well as logistical complications to surgery, including an
increased risk of infection to the patient.

Because portable LINACs for IOERT produced electron beams of energy less
than or equal to 12~MeV and did not use bending magnets, the secondary
radiation emitted was so low that it didn't require permanent shielding
in the operating room. This greatly reduced the cost of either
constructing a new OR or retrofitting an old one. By using mobile units,
the possibility of treating patients with IORT was no longer restricted
to the availability of special shielded operating rooms, but could be
done in regular unshielded ORs.

Currently, the Mobetron, LIAC, and NOVAC-7 linear accelerators are
improving patient care by delivering intraoperative radiation electron
beam therapy to cancer patients during surgery. All three units are
compact and mobile. Invented in the U.S. in 1997, the Mobetron uses
X-band technology and a soft docking system. The LIAC and NOVAC-7 are
robotic devices developed in Italy that use S-band technology and a
hard-docking system. The NOVAC-7 became available for clinical use in
the 1990s while the LIAC was introduced to a clinical environment in
2003.

Other non-IOERT mobile units have been developed as well. In 1998, a
technique called TARGIT (targeted intraoperative radio therapy) was
designed at the University College London for treating the tumor bed
after wide local excision (lumpectomy) of breast cancer. TARGIT uses a
miniature and mobile X-ray source that emits low energy X-ray radiation
(max. 50 kV) in isotropic distribution. (IO)-brachytherapy with
MammoSite is also used to treat breast cancer.

Interest in this treatment technique is growing, due in part to the
development of LINAC for IOERT by factories.

\section{See also}\label{see-also}

\begin{itemize}
\item
  \emph{Targeted intraoperative radiotherapy (TARGIT)}
\item
  \emph{Intraoperative radiation therapy (IORT)}
\end{itemize}

External beam radiotherapy (EBRT)

Intraoperative radiation therapy (IORT)

Targeted intraoperative radiotherapy (TARGIT)

\section{References}\label{references}

\section{External links}\label{external-links}

\begin{itemize}
\item
  \emph{Internal radiation therapy, cancer.org}
\end{itemize}

Internal radiation therapy, cancer.org
