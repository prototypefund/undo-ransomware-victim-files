\textbf{From Wikipedia, the free encyclopedia}

https://en.wikipedia.org/wiki/Robert\_Wharton\_\%28Philadelphia\%29\\
Licensed under CC BY-SA 3.0:\\
https://en.wikipedia.org/wiki/Wikipedia:Text\_of\_Creative\_Commons\_Attribution-ShareAlike\_3.0\_Unported\_License

\section{Robert Wharton
(Philadelphia)}\label{robert-wharton-philadelphia}

\begin{itemize}
\item
  \emph{Wharton was born in Philadelphia, January 12, 1757, the son of
  Joseph Wharton, a successful merchant.}
\item
  \emph{Robert Wharton (January 12, 1757 -- March 7, 1834) was the
  longest-serving Mayor of Philadelphia, Pennsylvania.}
\end{itemize}

Robert Wharton (January 12, 1757 -- March 7, 1834) was the
longest-serving Mayor of Philadelphia, Pennsylvania.

Wharton was born in Philadelphia, January 12, 1757, the son of Joseph
Wharton, a successful merchant. At an early age he left his studies, and
was apprenticed to a hatter. He entered the counting-house of his
brother Samuel, a Philadelphia merchant, but he spent much of his time
in outdoor sports, and until 1818 was president of the famous
fox-hunting club of Gloucester, New Jersey that was organized in 1766.
In 1790 he became a member of the Schuylkill Fishing Company, a social
club, of which he was president 1812--1828.

\section{Political career}\label{political-career}

\begin{itemize}
\item
  \emph{His 14 years of service and 15 elections to the office make him
  the longest-serving and most-elected mayor in Philadelphia's history.}
\item
  \emph{He was elected mayor of Philadelphia fifteen times between 1798
  and 1824.}
\item
  \emph{He was a member of the Philadelphia city council from 1792 till
  1795.}
\end{itemize}

He was a member of the Philadelphia city council from 1792 till 1795. In
1796 he was made alderman of that city, and in the same year quelled a
riot among sailors who had organized themselves into a body and demanded
higher wages. After reading the riot act, he requested they disperse,
and, being received with shouts of defiance, Wharton ordered each of his
men "to take his man," and the sailors were captured and imprisoned. He
quelled the Walnut Street Prison riot in 1798 and also took part in
suppressing others.

He was elected mayor of Philadelphia fifteen times between 1798 and
1824. After serving two one-year terms, 1798--1800, he declined
nomination in 1800. He served three more terms, 1806--1808 and
1810--1811, then on being re-elected in 1811, he declined to serve.
Subsequently, he served nine more terms, 1814--1819 and 1820--1824. His
14 years of service and 15 elections to the office make him the
longest-serving and most-elected mayor in Philadelphia's history.

\section{Military service}\label{military-service}

\begin{itemize}
\item
  \emph{He became a member of the First Troop Philadelphia City Cavalry
  in 1798 and served as its captain from 1803 to 1811.}
\item
  \emph{He died in Philadelphia, March 7, 1834.}
\item
  \emph{Wharton married Salome Chancellor.}
\item
  \emph{In 1810, the six troops of cavalry in the city were organized
  into a regiment, of which Wharton was elected colonel.}
\end{itemize}

He became a member of the First Troop Philadelphia City Cavalry in 1798
and served as its captain from 1803 to 1811. In 1810, the six troops of
cavalry in the city were organized into a regiment, of which Wharton was
elected colonel. Later, he was elected brigadier-general of the state
militia. He was vice-president of the Washington Benevolent Society, of
which he was an original member.

Wharton married Salome Chancellor. He died in Philadelphia, March 7,
1834.

\section{External links}\label{external-links}

\begin{itemize}
\item
  \emph{Biographical sketch, under Thomas Wharton}
\end{itemize}

Biographical sketch, under Thomas Wharton
