\textbf{From Wikipedia, the free encyclopedia}

https://en.wikipedia.org/wiki/Carroll\%20S.\%20Page\\
Licensed under CC BY-SA 3.0:\\
https://en.wikipedia.org/wiki/Wikipedia:Text\_of\_Creative\_Commons\_Attribution-ShareAlike\_3.0\_Unported\_License

\section{Carroll S. Page}\label{carroll-s.-page}

\begin{itemize}
\item
  \emph{He served as the 43rd Governor of Vermont and a United States
  Senator.}
\item
  \emph{Carroll Smalley Page (January 10, 1843~-- December 3, 1925) was
  an American businessman and politician.}
\end{itemize}

Carroll Smalley Page (January 10, 1843~-- December 3, 1925) was an
American businessman and politician. He served as the 43rd Governor of
Vermont and a United States Senator.

\section{Early life}\label{early-life}

\begin{itemize}
\item
  \emph{Carroll Page attended the local schools, People's Academy in
  Morrisville and Lamoille Central Academy in Hyde Park.}
\item
  \emph{Page was born in Westfield, Vermont, the son of Russel Smith
  Page (1813-1893) and Martha Malvina Smalley Page (1821-1907).}
\item
  \emph{Russel S. Page was a farmer, banker, businessman, and public
  official who served in several local offices, was a member of the
  Vermont House of Representatives, and held the county offices of
  sheriff and assistant judge.}
\end{itemize}

Page was born in Westfield, Vermont, the son of Russel Smith Page
(1813-1893) and Martha Malvina Smalley Page (1821-1907). Russel S. Page
was a farmer, banker, businessman, and public official who served in
several local offices, was a member of the Vermont House of
Representatives, and held the county offices of sheriff and assistant
judge. Carroll Page attended the local schools, People's Academy in
Morrisville and Lamoille Central Academy in Hyde Park. He married Ellen
Frances Patch on April 11, 1865, and they had three children.

\section{Career}\label{career}

\begin{itemize}
\item
  \emph{As a Republican, Page was elected Governor of Vermont and served
  from October 2, 1890 to October 6, 1892.}
\item
  \emph{During the American Civil War Page registered for the draft, and
  then served as a major in the 4th Regiment of Vermont Militia.}
\item
  \emph{From 1869 to 1872 Page was a member of the Vermont House of
  Representatives and from 1874 to 1876 he was a member of the Vermont
  Senate.}
\end{itemize}

Page went into the business of buying and selling raw animal hides for
the production of leather goods. Based in Hyde Park, Page's enterprise
grew until it was recognized as the largest calfskin dealer in the
world.

During the American Civil War Page registered for the draft, and then
served as a major in the 4th Regiment of Vermont Militia.

He was Lamoille County Treasurer from 1866 to 1872. Page was also
involved in the lumber business and served as President of the Lamoille
County Savings Bank and Trust Company and the Lamoille County National
Bank, both in Hyde Park. In addition, he was also a director of the St.
Johnsbury and Lake Champlain Railroad.

From 1869 to 1872 Page was a member of the Vermont House of
Representatives and from 1874 to 1876 he was a member of the Vermont
Senate. He was registrar of the Lamoille County probate court from 1880
to 1891. He was a state savings bank examiner from 1884 to 1888.

As a Republican, Page was elected Governor of Vermont and served from
October 2, 1890 to October 6, 1892. During his term, the office of
Governor of Vermont was empowered to appoint judges of all city and
municipal courts, and legislation was enacted providing for secret
ballots in elections.

In 1908, Page was elected as a Republican to the U.S. Senate to fill the
vacancy caused by the death of Redfield Proctor, easily defeating the
token Democratic candidate, Vernon A. Bullard; he was reelected in 1910
and 1916 and served from October 21, 1908, to March 3, 1923. He was not
a candidate for reelection in 1922. While in the Senate, Page was
chairman of the Committee on Standards, Weights and Measures
(Sixty-first Congress) and a member of the Committee on Cuban Relations
(Sixty-second Congress), the Committee on the Disposition of Useless
Executive Papers (Sixty-third Congress), the Committee on Transportation
and Sale of Meat Products (Sixty-fourth and Sixty-fifth Congresses), and
the Committee on Naval Affairs (Sixty-sixth and Sixty-seventh
Congresses).

\section{Death}\label{death}

\begin{itemize}
\item
  \emph{Page resided in Hyde Park until his death on December 3, 1925.}
\item
  \emph{He is interred at Hyde Park Cemetery, Hyde Park, Lamoille
  County, Vermont.}
\end{itemize}

Page resided in Hyde Park until his death on December 3, 1925. He is
interred at Hyde Park Cemetery, Hyde Park, Lamoille County, Vermont.

\section{Family}\label{family}

\begin{itemize}
\item
  \emph{Carroll Page was married to Ellen Frances Patch.}
\end{itemize}

Carroll Page was married to Ellen Frances Patch. They were the parents
of three children: Theophilus Hull (1871-1898), Russel Smith
(1877-1941), and Alice (1879-1929).

\section{References}\label{references}

\section{External links}\label{external-links}

\begin{itemize}
\item
  \emph{"Carroll S. Page (id: P000014)".}
\item
  \emph{United States Congress.}
\item
  \emph{Carroll S. Page at Find a Grave}
\item
  \emph{Biographical Directory of the United States Congress.}
\end{itemize}

United States Congress. "Carroll S. Page (id: P000014)". Biographical
Directory of the United States Congress.

Carroll S. Page at Find a Grave

National Governors Association

The Political Graveyard

Govetrack US Congress
