\textbf{From Wikipedia, the free encyclopedia}

https://en.wikipedia.org/wiki/Kerach\_9\\
Licensed under CC BY-SA 3.0:\\
https://en.wikipedia.org/wiki/Wikipedia:Text\_of\_Creative\_Commons\_Attribution-ShareAlike\_3.0\_Unported\_License

\section{Kerach 9}\label{kerach-9}

\begin{itemize}
\item
  \emph{Kerach tesha (Hebrew: קרח תשע‎ -- "Ice-nine";
  pronounced~{[}kerakh tesha{]}) is an Israeli Rock band active in the
  years 1993--2000, and reactivated in 2013.}
\end{itemize}

Kerach tesha (Hebrew: קרח תשע‎ -- "Ice-nine"; pronounced~{[}kerakh
tesha{]}) is an Israeli Rock band active in the years 1993--2000, and
reactivated in 2013.

\section{History}\label{history}

\section{Beginning}\label{beginning}

\begin{itemize}
\item
  \emph{Kerach tesha was formed in 1993 in Kfar Saba, Israel, when its
  members were at the age of 19.}
\item
  \emph{Among the cover versions the band played was Yossi Banai's song
  "I, Simon and little Moïse".}
\item
  \emph{In 1995 the band recorded first sketches under the production of
  Shai Lahav (from "Mofa Ha'arnavot Shel Dr. Kasper") and played as a
  warm-up band at Monica Sex's concerts.}
\end{itemize}

Kerach tesha was formed in 1993 in Kfar Saba, Israel, when its members
were at the age of 19. The name is taken from Kurt Vonnegut's novel
Cat's Cradle, where Ice-nine is a material that acts as a seed crystal
and causes the solidification of the entire body when it comes in
contact with water. In 1995 the band recorded first sketches under the
production of Shai Lahav (from "Mofa Ha'arnavot Shel Dr. Kasper") and
played as a warm-up band at Monica Sex's concerts. Among the cover
versions the band played was Yossi Banai's song "I, Simon and little
Moïse".

\section{Top time}\label{top-time}

\begin{itemize}
\item
  \emph{In June the band played as a warm-up band for "Suede" (which was
  one of the bands that had influenced "Kerach tesha" the most).}
\item
  \emph{The band then began playing as the warm-up band for Aviv Geffen
  and "The Mistakes" (Hatauyot).}
\item
  \emph{In 1999 the band released its second album, "Kerach tesha", also
  produced by Levi and Talmudi.}
\end{itemize}

In 1996 the band was signed to NMC Music label and in 1997 released its
debout album The Beginning of The Right Life (Hebrew: תחילתם של החיים
הנכונים, "Tkhilatam Shel HaKhaim HaNekonim"), musically produced by
Moshe Levi and Asaf Talmudi. The album had several radio hits: "Young
Mothers" (אמהות צעירות, "Imahot Tze'i'rot"), "Movies" (סרטים, "Sratim")
and "With Him Forever" (איתו לנצח, "Itto LaNetzakh"), which is
considered to be the first explicit gay pop song in Hebrew, and
therefore sparked considerable interest in the band and in the album,
which became a great success. The song "With Him Forever" reached the
Israeli annual Hebrew song chart that year. The band then began playing
as the warm-up band for Aviv Geffen and "The Mistakes" (Hatauyot). In
June the band played as a warm-up band for "Suede" (which was one of the
bands that had influenced "Kerach tesha" the most).

In 1999 the band released its second album, "Kerach tesha", also
produced by Levi and Talmudi. The album produced the hits "Sun" (שמש,
"Shemesh") and "My Ex-girlfriend" (החברה שלי לשעבר, "HaKhavera Sheli
Leshe'avar"). The album hosted Shlomi Shaban (piano) in his first
recording of non-classical music.

\section{Last hit and break-up}\label{last-hit-and-break-up}

\begin{itemize}
\item
  \emph{A few months after the single was released, the band broke up.}
\item
  \emph{In 2000 the band released its last hit, "Assaf Amdursky", which
  tells the story of a woman in love with the singer.}
\item
  \emph{In 2001 the band reunited for several concerts, and even
  recorded a new song titled "There's Not Enough Night" (אין מספיק לילה,
  "En Maspik Laïla"); the song was never officially released, but a
  bootleg recording can be found on the Internet.}
\end{itemize}

In 2000 the band released its last hit, "Assaf Amdursky", which tells
the story of a woman in love with the singer. Assaf Amdursky himself
used to play the song in the tour that accompanied his successful album
"Silent Engines" (מנועים שקטים, "Meno'im Shketim"). A few months after
the single was released, the band broke up.

In 2001 the band reunited for several concerts, and even recorded a new
song titled "There's Not Enough Night" (אין מספיק לילה, "En Maspik
Laïla"); the song was never officially released, but a bootleg recording
can be found on the Internet.

\section{Aftermath}\label{aftermath}

\begin{itemize}
\item
  \emph{The other three band members later formed a band named "The
  Koskim" (or "The Kosks"; הקוסקים, "HaKoskim"), which released an album
  called "Resting In Quick Worlds" (נחים בעולמות זריזים, "Nakhim
  Be'Olamot Zrizim"), with Ohad Koski, "Kerach tesha"'s guitarist, as
  the lead singer.}
\item
  \emph{After the band's break-up, lead singer Noam Rotem turned to a
  solo career.}
\end{itemize}

After the band's break-up, lead singer Noam Rotem turned to a solo
career. He released three albums: "Human Warmth" (חום אנושי, "Khom
Enoshi"), "Help Is On The Way" (עזרה בדרך, "Ezra BaDerekh") and "Iron
and Stones" (ברזל ואבנים, "Barzel Ve'Avanim").\\
The other three band members later formed a band named "The Koskim" (or
"The Kosks"; הקוסקים, "HaKoskim"), which released an album called
"Resting In Quick Worlds" (נחים בעולמות זריזים, "Nakhim Be'Olamot
Zrizim"), with Ohad Koski, "Kerach tesha"'s guitarist, as the lead
singer. Noam Rotem took part in writing some of the songs in the album,
and also played guitar.\\
In 2005 "The Koskim" released the album "Abramek" (in cooperation with
HaNoar HaOved VeHaLomed), which consists of poems written by young
Abramek Koplowicz, a Polish Jew from the Łódź Ghetto, who died in
Auschwitz in the age of 14. His most known song "Dream" tells about the
world's wonders that he wished to visit at age 20.

\section{International tour}\label{international-tour}

\begin{itemize}
\item
  \emph{In 2014 Kerach tesha is making first international tour, which
  is to Poland.}
\end{itemize}

In 2014 Kerach tesha is making first international tour, which is to
Poland. Tour is dedicated to the memory of Abramek Koplowicz and
includes concerts in Warsaw, Łódź, Zduńska Wola, Oświęcim and Kraków. In
Oświęcim they play as support for Eric Clapton at the Life Festival
Oświęcim.

\section{Film music}\label{film-music}

\begin{itemize}
\item
  \emph{In 2014 the flag song of Kerach tesha: "איתו לנצח", ("Itto
  LaNetzakh", eng: "With Him Forever") has been included as intro song
  to the Israeli film series "שושנה חלוץ מרכזי" (eng: Shoshana center
  forward).}
\end{itemize}

In 2014 the flag song of Kerach tesha: "איתו לנצח", ("Itto LaNetzakh",
eng: "With Him Forever") has been included as intro song to the Israeli
film series "שושנה חלוץ מרכזי" (eng: Shoshana center forward).

\section{Band members}\label{band-members}

\begin{itemize}
\item
  \emph{Noam Rotem (נעם רותם): lead singer and guitar}
\end{itemize}

Noam Rotem (נעם רותם): lead singer and guitar

Roy Hadas (רועי הדס): bass and vocals

Ohad Koski (אוהד קוסקי): guitar and vocals

Uri Meiselman (אורי מייזלמן): drums and cymbals

Musical production: Moshe Levi and Asaf Talmudi.

\section{Discography}\label{discography}

\begin{itemize}
\item
  \emph{1999 -- Kerach tesha}
\end{itemize}

1996 -- The Beginning of The Right Life (Hebrew: תחילתם של החיים
הנכונים, "Tkhilatam Shel HaKhaim HaNekonim")

1999 -- Kerach tesha

2000 -- "Assaf Amdursky" (single), attached to a special issue of the
second album

\section{References}\label{references}

\section{External links}\label{external-links}

\begin{itemize}
\item
  \emph{Band's page on Last.fm}
\item
  \emph{Yoav Kutner, Kerach 9 at MOOMA ‹See Tfd›(in Hebrew)}
\end{itemize}

Yoav Kutner, Kerach 9 at MOOMA ‹See Tfd›(in Hebrew)

Band's page on Last.fm
