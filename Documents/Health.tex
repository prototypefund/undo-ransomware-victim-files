\textbf{From Wikipedia, the free encyclopedia}

https://en.wikipedia.org/wiki/Health\\
Licensed under CC BY-SA 3.0:\\
https://en.wikipedia.org/wiki/Wikipedia:Text\_of\_Creative\_Commons\_Attribution-ShareAlike\_3.0\_Unported\_License

\section{Health}\label{health}

\begin{itemize}
\item
  \emph{Health may be defined as the ability to adapt and manage
  physical, mental and social challenges throughout life.}
\item
  \emph{Health, as defined by the World Health Organization (WHO), is "a
  state of complete physical, mental and social well-being and not
  merely the absence of disease or infirmity."}
\end{itemize}

Health, as defined by the World Health Organization (WHO), is "a state
of complete physical, mental and social well-being and not merely the
absence of disease or infirmity." This definition has been subject to
controversy, as it may have limited value for implementation. Health may
be defined as the ability to adapt and manage physical, mental and
social challenges throughout life.

\section{History}\label{history}

\begin{itemize}
\item
  \emph{The meaning of health has evolved over time.}
\item
  \emph{Systematic activities to prevent or cure health problems and
  promote good health in humans are undertaken by health care
  providers.}
\item
  \emph{In addition to health care interventions and a person's
  surroundings, a number of other factors are known to influence the
  health status of individuals, including their background, lifestyle,
  and economic, social conditions and spirituality; these are referred
  to as "determinants of health."}
\end{itemize}

The meaning of health has evolved over time. In keeping with the
biomedical perspective, early definitions of health focused on the theme
of the body's ability to function; health was seen as a state of normal
function that could be disrupted from time to time by disease. An
example of such a definition of health is: "a state characterized by
anatomic, physiologic, and psychological integrity; ability to perform
personally valued family, work, and community roles; ability to deal
with physical, biological, psychological, and social stress". Then in
1948, in a radical departure from previous definitions, the World Health
Organization (WHO) proposed a definition that aimed higher: linking
health to well-being, in terms of "physical, mental, and social
well-being, and not merely the absence of disease and infirmity".
Although this definition was welcomed by some as being innovative, it
was also criticized as being vague, excessively broad and was not
construed as measurable. For a long time, it was set aside as an
impractical ideal and most discussions of health returned to the
practicality of the biomedical model.

Just as there was a shift from viewing disease as a state to thinking of
it as a process, the same shift happened in definitions of health.
Again, the WHO played a leading role when it fostered the development of
the health promotion movement in the 1980s. This brought in a new
conception of health, not as a state, but in dynamic terms of
resiliency, in other words, as "a resource for living". 1984 WHO revised
the definition of health defined it as "the extent to which an
individual or group is able to realize aspirations and satisfy needs and
to change or cope with the environment. Health is a resource for
everyday life, not the objective of living; it is a positive concept,
emphasizing social and personal resources, as well as physical
capacities". Thus, health referred to the ability to maintain
homeostasis and recover from insults. Mental, intellectual, emotional
and social health referred to a person's ability to handle stress, to
acquire skills, to maintain relationships, all of which form resources
for resiliency and independent living. This opens up many possibilities
for health to be taught, strengthened and learned.

Since the late 1970s, the federal Healthy People Initiative has been a
visible component of the United States' approach to improving population
health. In each decade, a new version of Healthy People is issued,
featuring updated goals and identifying topic areas and quantifiable
objectives for health improvement during the succeeding ten years, with
assessment at that point of progress or lack thereof. Progress has been
limited to many objectives, leading to concerns about the effectiveness
of Healthy People in shaping outcomes in the context of a decentralized
and uncoordinated US health system. Healthy People 2020 gives more
prominence to health promotion and preventive approaches and adds a
substantive focus on the importance of addressing social determinants of
health. A new expanded digital interface facilitates use and
dissemination rather than bulky printed books as produced in the past.
The impact of these changes to Healthy People will be determined in the
coming years.

Systematic activities to prevent or cure health problems and promote
good health in humans are undertaken by health care providers.
Applications with regard to animal health are covered by the veterinary
sciences. The term "healthy" is also widely used in the context of many
types of non-living organizations and their impacts for the benefit of
humans, such as in the sense of healthy communities, healthy cities or
healthy environments. In addition to health care interventions and a
person's surroundings, a number of other factors are known to influence
the health status of individuals, including their background, lifestyle,
and economic, social conditions and spirituality; these are referred to
as "determinants of health." Studies have shown that high levels of
stress can affect human health.

In the first decade of the 21st century, the conceptualization of health
as an ability opened the door for self-assessments to become the main
indicators to judge the performance of efforts aimed at improving human
health. It also created the opportunity for every person to feel
healthy, even in the presence of multiple chronic diseases, or a
terminal condition, and for the re-examination of determinants of
health, away from the traditional approach that focuses on the reduction
of the prevalence of diseases.

\includegraphics[width=5.50000in,height=5.35071in]{media/image1.jpg}\\
\emph{Donald Henderson as part of the CDC's smallpox eradication team in
1966.}

\section{Determinants}\label{determinants}

\begin{itemize}
\item
  \emph{Sugar-sweetened beverages have become a target of anti-obesity
  initiatives with increasing evidence of their link to obesity.-- such
  as the 1974 Lalonde report from Canada; the Alameda County Study in
  California; and the series of World Health Reports of the World Health
  Organization, which focuses on global health issues including access
  to health care and improving public health outcomes, especially in
  developing countries.}
\end{itemize}

Generally, the context in which an individual lives is of great
importance for both his health status and quality of their life It is
increasingly recognized that health is maintained and improved not only
through the advancement and application of health science, but also
through the efforts and intelligent lifestyle choices of the individual
and society. According to the World Health Organization, the main
determinants of health include the social and economic environment, the
physical environment and the person's individual characteristics and
behaviors.

More specifically, key factors that have been found to influence whether
people are healthy or unhealthy include the following:

An increasing number of studies and reports from different organizations
and contexts examine the linkages between health and different factors,
including lifestyles, environments, health care organization and health
policy, one specific health policy brought into many countries in recent
years was the introduction of the sugar tax. Beverage taxes came into
light with increasing concerns about obesity, particularly among youth.
Sugar-sweetened beverages have become a target of anti-obesity
initiatives with increasing evidence of their link to obesity.-- such as
the 1974 Lalonde report from Canada; the Alameda County Study in
California; and the series of World Health Reports of the World Health
Organization, which focuses on global health issues including access to
health care and improving public health outcomes, especially in
developing countries.

The concept of the "health field," as distinct from medical care,
emerged from the Lalonde report from Canada. The report identified three
interdependent fields as key determinants of an individual's health.
These are:

Lifestyle: the aggregation of personal decisions (i.e., over which the
individual has control) that can be said to contribute to, or cause,
illness or death;

Environmental: all matters related to health external to the human body
and over which the individual has little or no control;

Biomedical: all aspects of health, physical and mental, developed within
the human body as influenced by genetic make-up.

The maintenance and promotion of health is achieved through different
combination of physical, mental, and social well-being, together
sometimes referred to as the "health triangle." The WHO's 1986 Ottawa
Charter for Health Promotion further stated that health is not just a
state, but also "a resource for everyday life, not the objective of
living. Health is a positive concept emphasizing social and personal
resources, as well as physical capacities."

Focusing more on lifestyle issues and their relationships with
functional health, data from the Alameda County Study suggested that
people can improve their health via exercise, enough sleep, spending
time in nature, maintaining a healthy body weight, limiting alcohol use,
and avoiding smoking. Health and illness can co-exist, as even people
with multiple chronic diseases or terminal illnesses can consider
themselves healthy.

The environment is often cited as an important factor influencing the
health status of individuals. This includes characteristics of the
natural environment, the built environment and the social environment.
Factors such as clean water and air, adequate housing, and safe
communities and roads all have been found to contribute to good health,
especially to the health of infants and children. Some studies have
shown that a lack of neighborhood recreational spaces including natural
environment leads to lower levels of personal satisfaction and higher
levels of obesity, linked to lower overall health and well being. It has
been demonstrated that increased time spent in natural environments is
associated with improved self-reported health , suggesting that the
positive health benefits of natural space in urban neighborhoods should
be taken into account in public policy and land use.

Genetics, or inherited traits from parents, also play a role in
determining the health status of individuals and populations. This can
encompass both the predisposition to certain diseases and health
conditions, as well as the habits and behaviors individuals develop
through the lifestyle of their families. For example, genetics may play
a role in the manner in which people cope with stress, either mental,
emotional or physical. For example, obesity is a significant problem in
the United States that contributes to bad mental health and causes
stress in the lives of great numbers of people. (One difficulty is the
issue raised by the debate over the relative strengths of genetics and
other factors; interactions between genetics and environment may be of
particular importance.)

\section{Potential issues}\label{potential-issues}

\begin{itemize}
\item
  \emph{Though the majority of these health issues are preventable, a
  major contributor to global ill health is the fact that approximately
  1 billion people lack access to health care systems (Shah, 2014).}
\item
  \emph{Another health issue that causes death or contributes to other
  health problems is malnutrition, especially among children.}
\end{itemize}

A number of types of health issues are common around the globe. Disease
is one of the most common. According to GlobalIssues.org, approximately
36~million people die each year from non-communicable (not contagious)
disease including cardiovascular disease, cancer, diabetes and chronic
lung disease (Shah, 2014).

Among communicable diseases, both viral and bacterial, AIDS/HIV,
tuberculosis, and malaria are the most common, causing millions of
deaths every year (Shah, 2014).

Another health issue that causes death or contributes to other health
problems is malnutrition, especially among children. One of the groups
malnutrition affects most is young children. Approximately 7.5~million
children under the age of 5 die from malnutrition, usually brought on by
not having the money to find or make food (Shah, 2014).

Bodily injuries are also a common health issue worldwide. These
injuries, including broken bones, fractures, and burns can reduce a
person's quality of life or can cause fatalities including infections
that resulted from the injury or the severity injury in general
(Moffett, 2013).

Lifestyle choices are contributing factors to poor health in many cases.
These include smoking cigarettes, and can also include a poor diet,
whether it is overeating or an overly constrictive diet. Inactivity can
also contribute to health issues and also a lack of sleep, excessive
alcohol consumption, and neglect of oral hygiene (Moffett2013).There are
also genetic disorders that are inherited by the person and can vary in
how much they affect the person and when they surface (Moffett, 2013).

Though the majority of these health issues are preventable, a major
contributor to global ill health is the fact that approximately 1
billion people lack access to health care systems (Shah, 2014).
Arguably, the most common and harmful health issue is that a great many
people do not have access to quality remedies.

\section{Mental health}\label{mental-health}

\begin{itemize}
\item
  \emph{Mental Health is not just the absence of mental illness.}
\item
  \emph{Other terms include: 'mental health problem', 'illness',
  'disorder', 'dysfunction'.}
\item
  \emph{There are many ways to prevent these health issues from
  occurring such as communicating well with a teen suffering from mental
  health issues.}
\item
  \emph{Family history of mental health problems}
\end{itemize}

The World Health Organization describes mental health as "a state of
well-being in which the individual realizes his or her own abilities,
can cope with the normal stresses of life, can work productively and
fruitfully, and is able to make a contribution to his or her community".
Mental Health is not just the absence of mental illness.

Mental illness is described as 'the spectrum of cognitive, emotional,
and behavioral conditions that interfere with social and emotional
well-being and the lives and productivity of people. Having a mental
illness can seriously impair, temporarily or permanently, the mental
functioning of a person. Other terms include: 'mental health problem',
'illness', 'disorder', 'dysfunction'.

Roughly one fifth of all adults 18 and over in the US are considered
diagnosable with mental illness. Mental illnesses are the leading cause
of disability in the US and Canada. Examples include, schizophrenia,
ADHD, major depressive disorder, bipolar disorder, anxiety disorder,
post-traumatic stress disorder and autism.

Many teens suffer from mental health issues in response to the pressures
of society and social problems they encounter. Some of the key mental
health issues seen in teens are: depression, eating disorders, and drug
abuse. There are many ways to prevent these health issues from occurring
such as communicating well with a teen suffering from mental health
issues. Mental health can be treated and be attentive to teens'
behavior.

~Many factors contribute to mental health problems, including:

Biological factors, such as genes or brain chemistry

Life experiences, such as trauma or abuse

Family history of mental health problems

\section{Maintaining}\label{maintaining}

\begin{itemize}
\item
  \emph{Achieving and maintaining health is an ongoing process, shaped
  by both the evolution of health care knowledge and practices as well
  as personal strategies and organized interventions for staying
  healthy.}
\end{itemize}

Achieving and maintaining health is an ongoing process, shaped by both
the evolution of health care knowledge and practices as well as personal
strategies and organized interventions for staying healthy.

\section{Diet}\label{diet}

\begin{itemize}
\item
  \emph{An important way to maintain your personal health is to have a
  healthy diet.}
\end{itemize}

An important way to maintain your personal health is to have a healthy
diet. A healthy diet includes a variety of plant-based and animal-based
foods that provide nutrients to your body. Such nutrients give you
energy and keep your body running. Nutrients help build and strengthen
bones, muscles, and tendons and also regulate body processes (i.e. blood
pressure). The food guide pyramid is a pyramid-shaped guide of healthy
foods divided into sections. Each section shows the recommended intake
for each food group (i.e. Protein, Fat, Carbohydrates, and Sugars).
Making healthy food choices is important because it can lower your risk
of heart disease, developing some types of cancer, and it will
contribute to maintaining a healthy weight.

The Mediterranean diet is commonly associated with health-promoting
effects due to the fact that it contains some bioactive compounds like
phenolic compounds, isoprenoids and alkaloids.

\section{Exercise}\label{exercise}

\begin{itemize}
\item
  \emph{Physical exercise enhances or maintains physical fitness and
  overall health and wellness.}
\item
  \emph{According to the National Institutes of Health, there are four
  types of exercise: endurance, strength, flexibility, and balance.}
\end{itemize}

Physical exercise enhances or maintains physical fitness and overall
health and wellness. It strengthens muscles and improves the
cardiovascular system. According to the National Institutes of Health,
there are four types of exercise: endurance, strength, flexibility, and
balance.

\section{Sleep}\label{sleep}

\begin{itemize}
\item
  \emph{In 2015, the National Sleep Foundation released updated
  recommendations for sleep duration requirements based on age and
  concluded that "Individuals who habitually sleep outside the normal
  range may be exhibiting signs or symptoms of serious health problems
  or, if done volitionally, may be compromising their health and
  well-being."}
\item
  \emph{Ongoing sleep deprivation has been linked to an increased risk
  for some chronic health problems.}
\item
  \emph{Sleep is an essential component to maintaining health.}
\end{itemize}

Sleep is an essential component to maintaining health. In children,
sleep is also vital for growth and development. Ongoing sleep
deprivation has been linked to an increased risk for some chronic health
problems. In addition, sleep deprivation has been shown to correlate
with both increased susceptibility to illness and slower recovery times
from illness. In one study, people with chronic insufficient sleep, set
as six hours of sleep a night or less, were found to be four times more
likely to catch a cold compared to those who reported sleeping for seven
hours or more a night. Due to the role of sleep in regulating
metabolism, insufficient sleep may also play a role in weight gain or,
conversely, in impeding weight loss. Additionally, in 2007, the
International Agency for Research on Cancer, which is the cancer
research agency for the World Health Organization, declared that
"shiftwork that involves circadian disruption is probably carcinogenic
to humans," speaking to the dangers of long-term nighttime work due to
its intrusion on sleep. In 2015, the National Sleep Foundation released
updated recommendations for sleep duration requirements based on age and
concluded that "Individuals who habitually sleep outside the normal
range may be exhibiting signs or symptoms of serious health problems or,
if done volitionally, may be compromising their health and well-being."

\section{Role of science}\label{role-of-science}

\begin{itemize}
\item
  \emph{Applied health sciences endeavor to better understand and
  improve human health through applications in areas such as health
  education, biomedical engineering, biotechnology and public health.}
\item
  \emph{Health science is the branch of science focused on health.}
\item
  \emph{Clinical practitioners focus mainly on the health of
  individuals, while public health practitioners consider the overall
  health of communities and populations.}
\end{itemize}

Health science is the branch of science focused on health. There are two
main approaches to health science: the study and research of the body
and health-related issues to understand how humans (and animals)
function, and the application of that knowledge to improve health and to
prevent and cure diseases and other physical and mental impairments. The
science builds on many sub-fields, including biology, biochemistry,
physics, epidemiology, pharmacology, medical sociology. Applied health
sciences endeavor to better understand and improve human health through
applications in areas such as health education, biomedical engineering,
biotechnology and public health.

Organized interventions to improve health based on the principles and
procedures developed through the health sciences are provided by
practitioners trained in medicine, nursing, nutrition, pharmacy, social
work, psychology, occupational therapy, physical therapy and other
health care professions. Clinical practitioners focus mainly on the
health of individuals, while public health practitioners consider the
overall health of communities and populations. Workplace wellness
programs are increasingly adopted by companies for their value in
improving the health and well-being of their employees, as are school
health services in order to improve the health and well-being of
children.

\includegraphics[width=3.39186in,height=5.50000in]{media/image2.jpg}\\
\emph{Postage stamp, New Zealand, 1933. Public health has been
promoted~-- and depicted~-- in a wide variety of ways.}

\section{Role of public health}\label{role-of-public-health}

\begin{itemize}
\item
  \emph{Environmental health, community health, behavioral health, and
  occupational health are also important areas of public health.}
\item
  \emph{It is concerned with threats to the overall health of a
  community based on population health analysis.}
\item
  \emph{Public health has many sub-fields, but typically includes the
  interdisciplinary categories of epidemiology, biostatistics and health
  services.}
\end{itemize}

Public health has been described as "the science and art of preventing
disease, prolonging life and promoting health through the organized
efforts and informed choices of society, organizations, public and
private, communities and individuals." It is concerned with threats to
the overall health of a community based on population health analysis.
The population in question can be as small as a handful of people or as
large as all the inhabitants of several continents (for instance, in the
case of a pandemic). Public health has many sub-fields, but typically
includes the interdisciplinary categories of epidemiology, biostatistics
and health services. Environmental health, community health, behavioral
health, and occupational health are also important areas of public
health.

The focus of public health interventions is to prevent and manage
diseases, injuries and other health conditions through surveillance of
cases and the promotion of healthy behavior, communities, and (in
aspects relevant to human health) environments. Its aim is to prevent
health problems from happening or re-occurring by implementing
educational programs, developing policies, administering services and
conducting research. In many cases, treating a disease or controlling a
pathogen can be vital to preventing it in others, such as during an
outbreak. Vaccination programs and distribution of condoms to prevent
the spread of communicable diseases are examples of common preventive
public health measures, as are educational campaigns to promote
vaccination and the use of condoms (including overcoming resistance to
such).

Public health also takes various actions to limit the health disparities
between different areas of the country and, in some cases, the continent
or world. One issue is the access of individuals and communities to
health care in terms of financial, geographical or socio-cultural
constraints to accessing and using services. Applications of the public
health system include the areas of maternal and child health, health
services administration, emergency response, and prevention and control
of infectious and chronic diseases.

The great positive impact of public health programs is widely
acknowledged. Due in part to the policies and actions developed through
public health, the 20th century registered a decrease in the mortality
rates for infants and children and a continual increase in life
expectancy in most parts of the world. For example, it is estimated that
life expectancy has increased for Americans by thirty years since 1900,
and worldwide by six years since 1990.

\includegraphics[width=4.58333in,height=5.50000in]{media/image3.jpg}\\
\emph{A lady washing her hands c. 1655}

\section{Self-care strategies}\label{self-care-strategies}

\begin{itemize}
\item
  \emph{Personal health depends partially on the active, passive, and
  assisted cues people observe and adopt about their own health.}
\item
  \emph{Personal health also depends partially on the social structure
  of a person's life.}
\item
  \emph{The maintenance of strong social relationships, volunteering,
  and other social activities have been linked to positive mental health
  and also increased longevity.}
\end{itemize}

Personal health depends partially on the active, passive, and assisted
cues people observe and adopt about their own health. These include
personal actions for preventing or minimizing the effects of a disease,
usually a chronic condition, through integrative care. They also include
personal hygiene practices to prevent infection and illness, such as
bathing and washing hands with soap; brushing and flossing teeth;
storing, preparing and handling food safely; and many others. The
information gleaned from personal observations of daily living~-- such
as about sleep patterns, exercise behavior, nutritional intake and
environmental features~-- may be used to inform personal decisions and
actions (e.g., "I feel tired in the morning so I am going to try
sleeping on a different pillow"), as well as clinical decisions and
treatment plans (e.g., a patient who notices his or her shoes are
tighter than usual may be having exacerbation of left-sided heart
failure, and may require diuretic medication to reduce fluid overload).

Personal health also depends partially on the social structure of a
person's life. The maintenance of strong social relationships,
volunteering, and other social activities have been linked to positive
mental health and also increased longevity. One American study among
seniors over age 70, found that frequent volunteering was associated
with reduced risk of dying compared with older persons who did not
volunteer, regardless of physical health status. Another study from
Singapore reported that volunteering retirees had significantly better
cognitive performance scores, fewer depressive symptoms, and better
mental well-being and life satisfaction than non-volunteering retirees.

Prolonged psychological stress may negatively impact health, and has
been cited as a factor in cognitive impairment with aging, depressive
illness, and expression of disease. Stress management is the application
of methods to either reduce stress or increase tolerance to stress.
Relaxation techniques are physical methods used to relieve stress.
Psychological methods include cognitive therapy, meditation, and
positive thinking, which work by reducing response to stress. Improving
relevant skills, such as problem solving and time management skills,
reduces uncertainty and builds confidence, which also reduces the
reaction to stress-causing situations where those skills are applicable.

\section{Occupational}\label{occupational}

\begin{itemize}
\item
  \emph{Examples of these include the British Health and Safety
  Executive and in the United States, the National Institute for
  Occupational Safety and Health, which conducts research on
  occupational health and safety, and the Occupational Safety and Health
  Administration, which handles regulation and policy relating to worker
  safety and health.}
\end{itemize}

In addition to safety risks, many jobs also present risks of disease,
illness and other long-term health problems. Among the most common
occupational diseases are various forms of pneumoconiosis, including
silicosis and coal worker's pneumoconiosis (black lung disease). Asthma
is another respiratory illness that many workers are vulnerable to.
Workers may also be vulnerable to skin diseases, including eczema,
dermatitis, urticaria, sunburn, and skin cancer. Other occupational
diseases of concern include carpal tunnel syndrome and lead poisoning.

As the number of service sector jobs has risen in developed countries,
more and more jobs have become sedentary, presenting a different array
of health problems than those associated with manufacturing and the
primary sector. Contemporary problems, such as the growing rate of
obesity and issues relating to stress and overwork in many countries,
have further complicated the interaction between work and health.

Many governments view occupational health as a social challenge and have
formed public organizations to ensure the health and safety of workers.
Examples of these include the British Health and Safety Executive and in
the United States, the National Institute for Occupational Safety and
Health, which conducts research on occupational health and safety, and
the Occupational Safety and Health Administration, which handles
regulation and policy relating to worker safety and health.

\section{See also}\label{see-also}

\section{References}\label{references}

\section{External links}\label{external-links}

\begin{itemize}
\item
  \emph{Media related to Health at Wikimedia Commons}
\end{itemize}

Media related to Health at Wikimedia Commons
