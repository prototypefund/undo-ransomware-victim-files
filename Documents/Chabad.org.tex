\textbf{From Wikipedia, the free encyclopedia}

https://en.wikipedia.org/wiki/Chabad.org\\
Licensed under CC BY-SA 3.0:\\
https://en.wikipedia.org/wiki/Wikipedia:Text\_of\_Creative\_Commons\_Attribution-ShareAlike\_3.0\_Unported\_License

\section{Chabad.org}\label{chabad.org}

\begin{itemize}
\item
  \emph{It was one of the first Jewish internet sites and the first and
  largest virtual congregation.}
\item
  \emph{Chabad.org is the flagship website of the Chabad-Lubavitch
  Hasidic movement.}
\end{itemize}

Chabad.org is the flagship website of the Chabad-Lubavitch Hasidic
movement. It serves its own members and Jews worldwide. It was one of
the first Jewish internet sites and the first and largest virtual
congregation.

\section{History}\label{history}

\begin{itemize}
\item
  \emph{In 1988, Yosef Yitzchak Kazen, a Chabad rabbi, began creating a
  Chabad-Lubavitch presence in cyberspace.}
\item
  \emph{Today, the Chabad Lubavitch Media Center maintains the flagship
  Chabad.org, specialized holiday sites, and over 1,400 customized sites
  for local Chabad houses.}
\item
  \emph{After Kazen's death in 1998, the site was rolled under the
  umbrella of the Chabad Lubavitch Media Center directed by Rabbi Zalman
  Shmotkin.}
\end{itemize}

In 1988, Yosef Yitzchak Kazen, a Chabad rabbi, began creating a
Chabad-Lubavitch presence in cyberspace. With the advent of computer
communication technology, Kazen recognized its potential for reaching an
almost limitless audience, unlimited by geographic and other
constraints. Kazen digitized thousands of documents into what became the
world's first virtual Jewish library, and enabling thousands of people
to learn about Judaism for the first time. Chabad.org served as a model
for other Jewish organizations that created their own educational
websites.

After Kazen's death in 1998, the site was rolled under the umbrella of
the Chabad Lubavitch Media Center directed by Rabbi Zalman Shmotkin.
Today, the Chabad Lubavitch Media Center maintains the flagship
Chabad.org, specialized holiday sites, and over 1,400 customized sites
for local Chabad houses.{[}citation needed{]}

\section{Jewish knowledge base}\label{jewish-knowledge-base}

\begin{itemize}
\item
  \emph{Chabad.org has a comprehensive Jewish knowledge base which
  includes over 100,000 articles of information ranging from basic
  Judaism to Hasidic philosophy taught from the Chabad point of view.}
\item
  \emph{There are comprehensive sections on Shabbat, Kosher, Tefillin,
  Mezuzah, the Jewish way in death and mourning and a synagogue
  companion.}
\end{itemize}

Chabad.org has a comprehensive Jewish knowledge base which includes over
100,000 articles of information ranging from basic Judaism to Hasidic
philosophy taught from the Chabad point of view. The major categories
are the human being, God and man, concepts and ideas, the Torah, the
physical world, the Jewish calendar, science and technology, people and
events.

There are comprehensive sections on Shabbat, Kosher, Tefillin, Mezuzah,
the Jewish way in death and mourning and a synagogue companion.

\section{Ask the Rabbi}\label{ask-the-rabbi}

\begin{itemize}
\item
  \emph{In 1994, Kazen launched the first version of Chabad's ``Ask the
  Rabbi'' website.}
\item
  \emph{Chabad.org was the pioneer of ``Ask the rabbi'' sites.}
\end{itemize}

Chabad.org was the pioneer of ``Ask the rabbi'' sites. Rabbi Yosef
Yitzchak Kazen reached out to thousands of people on Fidonet, an online
discussion network, as far back as 1988.

In 1994, Kazen launched the first version of Chabad's ``Ask the Rabbi''
website. Today's version, in which 40 rabbis and educators field
questions via e-mail, has answered more than 500,000 questions between
2001--2006, averaging about 270 a day. Many people take advantage of the
Web's anonymity to impart experiences and ask for advice from
chabad.org. Chabad.org also operates TheJewishWoman.org's ``Dear
Rachel'', a similar service which is run by women for women.

More than 2,000 questions and answers have been posted online.

\section{Features}\label{features}

\begin{itemize}
\item
  \emph{A section featuring reports in the media on the activities of
  Chabad Lubavitch Shluchim ("emissaries").}
\item
  \emph{Chabad.org provides daily, date-specific information relevant to
  each day from Jewish history, daily Torah study, candle-lighting
  times, and forthcoming Jewish holidays.}
\item
  \emph{An "Ask the Rabbi" feature.}
\end{itemize}

Chabad.org provides daily, date-specific information relevant to each
day from Jewish history, daily Torah study, candle-lighting times, and
forthcoming Jewish holidays.

Chabad.org maintains a number of sub-sites, including

Weekly Magazine email on Torah and contemporary life.

A search feature that enables the user to quickly find a Chabad House in
any part of the world.

An online Jewish library that contains some 100,000 articles.

An "Ask the Rabbi" feature.

A multimedia portal, Jewish.tv, where users can stream Jewish audio and
video.

A children's section.

A section featuring reports in the media on the activities of Chabad
Lubavitch Shluchim ("emissaries").

\section{Statistics}\label{statistics}

\begin{itemize}
\item
  \emph{Chabad.org and its affiliated sites claim over 43 million
  visitors per year, and over 365,000 email subscribers.}
\end{itemize}

Chabad.org and its affiliated sites claim over 43 million visitors per
year, and over 365,000 email subscribers.

\section{See also}\label{see-also}

\begin{itemize}
\item
  \emph{AskMoses.com}
\item
  \emph{Internet marketing}
\item
  \emph{Internet activism}
\item
  \emph{Kikar HaShabbat (website)}
\end{itemize}

AskMoses.com

Internet activism

Internet marketing

Kikar HaShabbat (website)

\section{References}\label{references}

\section{Sources}\label{sources}

\begin{itemize}
\item
  \emph{Wired.com interviews Chabad.org director Rabbi Zalman Shmotkin}
\item
  \emph{{[} `Ask the rabbi' -- online{]}}
\item
  \emph{Chabad launches Jewish women's site}
\end{itemize}

JTA News article

Wired.com interviews Chabad.org director Rabbi Zalman Shmotkin

Chabad launches Jewish women's site

Rabbi Schneerson's life for a history-gathering project

Let my people download! Passover texts available online

{[} `Ask the rabbi' -- online{]}

Yosef Kazen, Hasidic Rabbi And Web Pioneer, Dies at 44 The New York
Times December 13, 1998.

\section{External links}\label{external-links}

\begin{itemize}
\item
  \emph{Chabad.org}
\end{itemize}

Chabad.org
