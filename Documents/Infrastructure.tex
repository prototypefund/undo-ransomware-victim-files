\textbf{From Wikipedia, the free encyclopedia}

https://en.wikipedia.org/wiki/Infrastructure\\
Licensed under CC BY-SA 3.0:\\
https://en.wikipedia.org/wiki/Wikipedia:Text\_of\_Creative\_Commons\_Attribution-ShareAlike\_3.0\_Unported\_License

\section{Infrastructure}\label{infrastructure}

\begin{itemize}
\item
  \emph{Hard infrastructure refers to the physical networks necessary
  for the functioning of a modern industry.}
\item
  \emph{There are two general types of ways to view infrastructure, hard
  or soft.}
\item
  \emph{Soft infrastructure refers to all the institutions that maintain
  the economic, health, social, and cultural standards of a country.}
\end{itemize}

Infrastructure is the fundamental facilities and systems serving a
country, city, or other area, including the services and facilities
necessary for its economy to function. Infrastructure is composed of
public and private physical improvements such as roads, bridges,
tunnels, water supply, sewers, electrical grids, and telecommunications
(including Internet connectivity and broadband speeds). In general, it
has also been defined as "the physical components of interrelated
systems providing commodities and services essential to enable, sustain,
or enhance societal living conditions".

There are two general types of ways to view infrastructure, hard or
soft. Hard infrastructure refers to the physical networks necessary for
the functioning of a modern industry. This includes roads, bridges,
railways, etc. Soft infrastructure refers to all the institutions that
maintain the economic, health, social, and cultural standards of a
country. This includes educational programs, official statistics, parks
and recreational facilities, law enforcement agencies, and emergency
services.

The word infrastructure has been used in French since 1875 and in
English since 1887, originally meaning "The installations that form the
basis for any operation or system". The word was imported from French,
where it was already used for establishing a roadbed of substrate
material, required before railroad tracks or constructed pavement could
be laid on top of it. The word is a combination of the Latin prefix
"infra", meaning "below" and many of these constructions are
underground, for example, tunnels, water and gas systems, and railways
and the French word "structure" (derived from the Latin word
"structura"). The army use of the term achieved currency in the United
States after the formation of NATO in the 1940s, and by 1970 was adopted
by urban planners in its modern civilian sense.

\section{Classifications}\label{classifications}

\begin{itemize}
\item
  \emph{A 1987 US National Research Council panel adopted the term
  "public works infrastructure", referring to:}
\item
  \emph{The American Society of Civil Engineers publish a
  "Infrastructure Report Card" which represents the organizations
  opinion on the condition of various infrastructure every 2--4 years.}
\end{itemize}

A 1987 US National Research Council panel adopted the term "public works
infrastructure", referring to:

The American Society of Civil Engineers publish a "Infrastructure Report
Card" which represents the organizations opinion on the condition of
various infrastructure every 2--4 years. As of 2017{[}update{]} they
grade 16 categories, namely Aviation, Bridges, Dams, Drinking Water,
Energy, Hazardous Waste, Inland Waterways, Levees, Parks \& Recreation,
Ports, Rail, Roads, Schools, Solid Waste, Transit and Wastewater.:4

\section{Personal}\label{personal}

\begin{itemize}
\item
  \emph{A way to embody personal infrastructure is to think of it in
  term of human capital.}
\item
  \emph{); the importance of personal infrastructure for an individual
  (short and long-term consumption of education); and the social
  relevance of personal infrastructure.}
\item
  \emph{The goal of personal infrastructure is to determine the quality
  of the economic agents' values.}
\end{itemize}

A way to embody personal infrastructure is to think of it in term of
human capital. Human capital is defined by the Encyclopedia Britannica
as ``intangible collective resources possessed by individuals and groups
within a given population". The goal of personal infrastructure is to
determine the quality of the economic agents' values. This results in
three major tasks: the task of economic proxies' in the economic process
(teachers, unskilled and qualified labor, etc.); the importance of
personal infrastructure for an individual (short and long-term
consumption of education); and the social relevance of personal
infrastructure.

\section{Institutional}\label{institutional}

\begin{itemize}
\item
  \emph{According to Gianpiero Torrisi, Institutional infrastructure is
  the object of economic and legal policy.}
\item
  \emph{Institutional infrastructure branches from the term "economic
  constitution".}
\end{itemize}

Institutional infrastructure branches from the term "economic
constitution". According to Gianpiero Torrisi, Institutional
infrastructure is the object of economic and legal policy. It
compromises the grown and sets norms. It refers to the degree of actual
equal treatment of equal economic data and determines the framework
within which economic agents may formulate their own economic plans and
carry them out in co-operation with others.

\section{Material}\label{material}

\begin{itemize}
\item
  \emph{Material infrastructure is defined as ``those immobile,
  non-circulating capital goods that essentially contribute to the
  production of infrastructure goods and services needed to satisfy
  basic physical and social requirements of economic agents".}
\item
  \emph{The second characteristic is the non-availability of
  infrastructure goods and services.}
\end{itemize}

Material infrastructure is defined as ``those immobile, non-circulating
capital goods that essentially contribute to the production of
infrastructure goods and services needed to satisfy basic physical and
social requirements of economic agents". There are two distinct
qualities of material infrastructures: 1) Fulfillment of social needs
and 2) Mass production. The first characteristic deals with the basic
needs of human life. The second characteristic is the non-availability
of infrastructure goods and services.

\section{Economic}\label{economic}

\begin{itemize}
\item
  \emph{Economic infrastructure support productive activities and
  events.}
\item
  \emph{According to the business dictionary, economic infrastructure
  can be defined as "internal facilities of a country that make business
  activity possible, such as communication, transportation and
  distribution networks, financial institutions and markets, and energy
  supply systems".}
\end{itemize}

According to the business dictionary, economic infrastructure can be
defined as "internal facilities of a country that make business activity
possible, such as communication, transportation and distribution
networks, financial institutions and markets, and energy supply
systems". Economic infrastructure support productive activities and
events. This includes roads, highways, bridges, airports, water
distribution networks, sewer systems, irrigation plants, etc.

\section{Social}\label{social}

\begin{itemize}
\item
  \emph{Social infrastructure can be broadly defined as the construction
  and maintenance of facilities that support social services.}
\end{itemize}

Social infrastructure can be broadly defined as the construction and
maintenance of facilities that support social services. Social
infrastructures are created to increase social comfort and act on
economic activity. These being schools, parks and playgrounds,
structures for public safety, waste disposal plants, hospitals, sports
area, etc.

\section{Core}\label{core}

\begin{itemize}
\item
  \emph{Core Infrastructure incorporates all the main types of
  infrastructure.}
\end{itemize}

Core assets provide essential services and have monopolistic
characteristics. Investors seeking core infrastructure look for five
different characteristics: Income, Low volatility of returns,
Diversification, Inflation Protection, and Long-term liability matching.
Core Infrastructure incorporates all the main types of infrastructure.
For instance; roads, highways, railways, public transportation, water
and gas supply, etc.

\section{Basic}\label{basic}

\begin{itemize}
\item
  \emph{Basic infrastructure refers to main railways, roads, canals,
  harbors and docks, the electromagnetic telegraph, drainage, dikes, and
  land reclamation.}
\item
  \emph{It consist of the more well-known features of infrastructure.}
\end{itemize}

Basic infrastructure refers to main railways, roads, canals, harbors and
docks, the electromagnetic telegraph, drainage, dikes, and land
reclamation. It consist of the more well-known features of
infrastructure. The things in the world we come across everyday
(buildings, roads, docks, etc).

\section{Complementary}\label{complementary}

\begin{itemize}
\item
  \emph{So, complementary infrastructure deals with the little parts of
  the engineering world the brings more life.}
\item
  \emph{Complementary infrastructure refers to things like light
  railways, tramways, gas/electricity/water supply, etc.}
\end{itemize}

Complementary infrastructure refers to things like light railways,
tramways, gas/electricity/water supply, etc. To complement something,
means to bring to perfection or complete it. So, complementary
infrastructure deals with the little parts of the engineering world the
brings more life. The lights on the sidewalks, the landscaping around
buildings, the benches for pedestrians to rest, etc.

\section{Related concepts}\label{related-concepts}

\begin{itemize}
\item
  \emph{Land improvement and land development are general terms that in
  some contexts may include infrastructure, but in the context of a
  discussion of infrastructure would refer only to smaller scale systems
  or works that are not included in infrastructure, because they are
  typically limited to a single parcel of land, and are owned and
  operated by the land owner.}
\item
  \emph{Public services include both infrastructure and services
  generally provided by government.}
\end{itemize}

The term infrastructure may be confused with the following overlapping
or related concepts.

Land improvement and land development are general terms that in some
contexts may include infrastructure, but in the context of a discussion
of infrastructure would refer only to smaller scale systems or works
that are not included in infrastructure, because they are typically
limited to a single parcel of land, and are owned and operated by the
land owner. For example, an irrigation canal that serves a region or
district would be included with infrastructure, but the private
irrigation systems on individual land parcels would be considered land
improvements, not infrastructure. Service connections to municipal
service and public utility networks would also be considered land
improvements, not infrastructure.

The term public works includes government-owned and operated
infrastructure as well as public buildings, such as schools and court
houses. Public works generally refers to physical assets needed to
deliver public services. Public services include both infrastructure and
services generally provided by government.

\section{Ownership and financing}\label{ownership-and-financing}

\begin{itemize}
\item
  \emph{{[}citation needed{]} Publicly owned infrastructure may be paid
  for from taxes, tolls, or metered user fees, whereas private
  infrastructure is generally paid for by metered user fees.}
\item
  \emph{Many financial institutions invest in infrastructure.}
\item
  \emph{Infrastructure may be owned and managed by governments or by
  private companies, such as sole public utility or railway companies.}
\end{itemize}

Infrastructure may be owned and managed by governments or by private
companies, such as sole public utility or railway companies. Generally,
most roads, major airports and other ports, water distribution systems,
and sewage networks are publicly owned, whereas most energy and
telecommunications networks are privately owned.{[}citation needed{]}
Publicly owned infrastructure may be paid for from taxes, tolls, or
metered user fees, whereas private infrastructure is generally paid for
by metered user fees.{[}citation needed{]} Major investment projects are
generally financed by the issuance of long-term bonds.{[}citation
needed{]}

Government-owned and operated infrastructure may be developed and
operated in the private sector or in public-private partnerships, in
addition to in the public sector. As of 2008{[}update{]} in the United
States for example, public spending on infrastructure has varied between
2.3\% and 3.6\% of GDP since 1950. Many financial institutions invest in
infrastructure.

\section{Types}\label{types}

\section{Engineering and
construction}\label{engineering-and-construction}

\begin{itemize}
\item
  \emph{Engineers generally limit the term "infrastructure" to describe
  fixed assets that are in the form of a large network; in other words,
  hard infrastructure.}
\end{itemize}

Engineers generally limit the term "infrastructure" to describe fixed
assets that are in the form of a large network; in other words, hard
infrastructure.{[}citation needed{]} Efforts to devise more generic
definitions of infrastructures have typically referred to the network
aspects of most of the structures, and to the accumulated value of
investments in the networks as assets.{[}citation needed{]} One such
definition from 1998 defined infrastructure as the network of assets
"where the system as a whole is intended to be maintained indefinitely
at a specified standard of service by the continuing replacement and
refurbishment of its components".

\section{Civil defense and economic
development}\label{civil-defense-and-economic-development}

\begin{itemize}
\item
  \emph{The notion of infrastructure-based development combining
  long-term infrastructure investments by government agencies at central
  and regional levels with public private partnerships has proven
  popular among economists in Asia (notably Singapore and China),
  mainland Europe, and Latin America.}
\item
  \emph{Civil defense planners and developmental economists generally
  refer to both hard and soft infrastructure, including public services
  such as schools and hospitals, emergency services such as police and
  fire fighting, and basic financial services.}
\end{itemize}

Civil defense planners and developmental economists generally refer to
both hard and soft infrastructure, including public services such as
schools and hospitals, emergency services such as police and fire
fighting, and basic financial services. The notion of
infrastructure-based development combining long-term infrastructure
investments by government agencies at central and regional levels with
public private partnerships has proven popular among economists in Asia
(notably Singapore and China), mainland Europe, and Latin America.

\section{Military}\label{military}

\begin{itemize}
\item
  \emph{Military infrastructure is the buildings and permanent
  installations necessary for the support of military forces, whether
  they are stationed in bases, being deployed or engaged in operations.}
\end{itemize}

Military infrastructure is the buildings and permanent installations
necessary for the support of military forces, whether they are stationed
in bases, being deployed or engaged in operations. For example,
barracks, headquarters, airfields, communications facilities, stores of
military equipment, port installations, and maintenance stations.

\section{Communications}\label{communications}

\begin{itemize}
\item
  \emph{Examples include IT infrastructure, research infrastructure,
  terrorist infrastructure, employment infrastructure and tourism
  infrastructure.}
\end{itemize}

Communications infrastructure is the informal and formal channels of
communication, political and social networks, or beliefs held by members
of particular groups, as well as information technology, software
development tools. Still underlying these more conceptual uses is the
idea that infrastructure provides organizing structure and support for
the system or organization it serves, whether it is a city, a nation, a
corporation, or a collection of people with common interests. Examples
include IT infrastructure, research infrastructure, terrorist
infrastructure, employment infrastructure and tourism
infrastructure.{[}citation needed{]}

\section{In the developing world}\label{in-the-developing-world}

\begin{itemize}
\item
  \emph{Infrastructure investments and maintenance can be very
  expensive, especially in such areas as landlocked, rural and sparsely
  populated countries in Africa.}
\item
  \emph{According to researchers at the Overseas Development Institute,
  the lack of infrastructure in many developing countries represents one
  of the most significant limitations to economic growth and achievement
  of the Millennium Development Goals (MDGs).}
\end{itemize}

According to researchers at the Overseas Development Institute, the lack
of infrastructure in many developing countries represents one of the
most significant limitations to economic growth and achievement of the
Millennium Development Goals (MDGs). Infrastructure investments and
maintenance can be very expensive, especially in such areas as
landlocked, rural and sparsely populated countries in Africa. It has
been argued that infrastructure investments contributed to more than
half of Africa's improved growth performance between 1990 and 2005, and
increased investment is necessary to maintain growth and tackle poverty.
The returns to investment in infrastructure are very significant, with
on average thirty to forty percent returns for telecommunications (ICT)
investments, over forty percent for electricity generation, and eighty
percent for roads.

\section{Regional differences}\label{regional-differences}

\begin{itemize}
\item
  \emph{The infrastructure financing gap between what is invested in
  Asia-Pacific (around US\$48 billion) and what is needed (US\$228
  billion) is around US\$180 billion every year.}
\item
  \emph{There are severe constraints on the supply side of the provision
  of infrastructure in Asia.}
\item
  \emph{The demand for infrastructure, both by consumers and by
  companies is much higher than the amount invested.}
\end{itemize}

The demand for infrastructure, both by consumers and by companies is
much higher than the amount invested. There are severe constraints on
the supply side of the provision of infrastructure in Asia. The
infrastructure financing gap between what is invested in Asia-Pacific
(around US\$48 billion) and what is needed (US\$228 billion) is around
US\$180 billion every year.

In Latin America, three percent of GDP (around US\$71 billion) would
need to be invested in infrastructure in order to satisfy demand, yet in
2005, for example, only around two percent was invested leaving a
financing gap of approximately US\$24 billion.

In Africa, in order to reach the seven percent annual growth calculated
to be required to meet the MDGs by 2015 would require infrastructure
investments of about fifteen percent of GDP, or around US\$93 billion a
year. In fragile states, over thirty-seven percent of GDP would be
required.

\section{Sources of funding}\label{sources-of-funding}

\begin{itemize}
\item
  \emph{This means that the government spends less money on repairing
  old infrastructure and or on infrastructure as a whole.}
\item
  \emph{The private sector spending alone equals state capital
  expenditure, though the majority is focused on ICT infrastructure
  investments.}
\end{itemize}

The source of financing varies significantly across sectors. Some
sectors are dominated by government spending, others by overseas
development aid (ODA), and yet others by private investors. In
California, infrastructure financing districts are established by local
governments to pay for physical facilities and services within a
specified area by using property tax increases. In order to facilitate
investment of the private sector in developing countries' infrastructure
markets, it is necessary to design risk-allocation mechanisms more
carefully, given the higher risks of their markets.

The spending money that comes from the government is less than it used
to be. Compared to the global GDP percentages, The United States is tied
for second-to-last place, with an average percentage of 2.4\%. This
means that the government spends less money on repairing old
infrastructure and or on infrastructure as a whole.

In Sub-Saharan Africa, governments spend around US\$9.4 billion out of a
total of US\$24.9 billion. In irrigation, governments represent almost
all spending. In transport and energy a majority of investment is
government spending. In ICT and water supply and sanitation, the private
sector represents the majority of capital expenditure. Overall, between
them aid, the private sector, and non-OECD financiers exceed government
spending. The private sector spending alone equals state capital
expenditure, though the majority is focused on ICT infrastructure
investments. External financing increased in the 2000s (decade) and in
Africa alone external infrastructure investments increased from US\$7
billion in 2002 to US\$27 billion in 2009. China, in particular, has
emerged as an important investor.

\section{See also}\label{see-also}

\section{References}\label{references}

\section{Bibliography}\label{bibliography}

\begin{itemize}
\item
  \emph{A. Eberhard, "Infrastructure Regulation in Developing
  Countries", PPIAF Working Paper No.}
\item
  \emph{Georg Inderst, "Pension Fund Investment in Infrastructure", OECD
  Working Papers on Insurance and Private Pensions, No.}
\item
  \emph{Nurre, Sarah G. "Restoring infrastructure systems: An integrated
  network design and scheduling (INDS) problem."}
\item
  \emph{Infrastructure: the book of everything for the industrial
  landscape (1st ed.).}
\end{itemize}

Koh, Jae Myong (2018) Green Infrastructure Financing: Institutional
Investors, PPPs and Bankable Projects, London: Palgrave Macmillan.
ISBN~978-3-319-71769-2.

Nurre, Sarah G. "Restoring infrastructure systems: An integrated network
design and scheduling (INDS) problem." European Journal of Operational
Research. (12/2012), 223 (3), pp.~794--806.

Ascher, Kate; researched by Wendy Marech (2007). The works: anatomy of a
city (Reprint. ed.). New York: Penguin Press. ISBN~978-0-14-311270-9.

Larry W. Beeferman, "Pension Fund Investment in Infrastructure: A
Resource Paper", Capital Matter (Occasional Paper Series), No. 3
December 2008

A. Eberhard, "Infrastructure Regulation in Developing Countries", PPIAF
Working Paper No. 4 (2007) World Bank

M. Nicolas J. Firzli and Vincent Bazi, "Infrastructure Investments in an
Age of Austerity: The Pension and Sovereign Funds Perspective",
published jointly in Revue Analyse Financière, Q4 2011 issue, pp.~34--37
and USAK/JTW July 30, 2011 (online edition)

Hayes, Brian (2005). Infrastructure: the book of everything for the
industrial landscape (1st ed.). New York City: Norton.
ISBN~978-0-393-32959-9.

Huler, Scott (2010). On the grid: a plot of land, an average
neighborhood, and the systems that make our world work. Emmaus, PA:
Rodale. ISBN~978-1-60529-647-0.

Georg Inderst, "Pension Fund Investment in Infrastructure", OECD Working
Papers on Insurance and Private Pensions, No. 32 (2009)

Dalakoglou, Dimitris (2017). The Road: An Ethnography of (Im)mobility,
space and cross-border infrastructures. Manchester: Manchester
University Press/ Oxford university Press.

\section{External links}\label{external-links}

\begin{itemize}
\item
  \emph{Report Card on America's Infrastructure}
\item
  \emph{Body of Knowledge on Infrastructure Regulation}
\end{itemize}

Body of Knowledge on Infrastructure Regulation

Next Generation Infrastructures international research programme

Report Card on America's Infrastructure
