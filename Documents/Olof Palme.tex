\textbf{From Wikipedia, the free encyclopedia}

https://en.wikipedia.org/wiki/Olof\%20Palme\\
Licensed under CC BY-SA 3.0:\\
https://en.wikipedia.org/wiki/Wikipedia:Text\_of\_Creative\_Commons\_Attribution-ShareAlike\_3.0\_Unported\_License

\section{Olof Palme}\label{olof-palme}

\begin{itemize}
\item
  \emph{A longtime protégé of Prime Minister Tage Erlander, Palme led
  the Swedish Social Democratic Party from 1969 until his assassination
  in 1986, and was a two-term Prime Minister of Sweden, heading a Privy
  Council Government from 1969 to 1976 and a cabinet government from
  1982 until his death.}
\item
  \emph{Palme was a pivotal and polarizing figure domestically as well
  as in international politics from the 1960s.}
\item
  \emph{Sven Olof Joachim Palme (/ˈpɑːlmə/; Swedish:~{[}²uːlɔf
  ²palːmɛ{]} (listen); 30 January 1927 -- 28 February 1986) was a
  Swedish Social Democratic politician and statesman.}
\end{itemize}

Sven Olof Joachim Palme (/ˈpɑːlmə/; Swedish:~{[}²uːlɔf ²palːmɛ{]}
(listen); 30 January 1927 -- 28 February 1986) was a Swedish Social
Democratic politician and statesman. A longtime protégé of Prime
Minister Tage Erlander, Palme led the Swedish Social Democratic Party
from 1969 until his assassination in 1986, and was a two-term Prime
Minister of Sweden, heading a Privy Council Government from 1969 to 1976
and a cabinet government from 1982 until his death. Electoral defeats in
1976 and 1979 marked the end of Social Democratic hegemony in Swedish
politics, which had seen 40 years of unbroken rule by the party. While
leader of the opposition, he parted{[}clarification needed{]} domestic
and international interests and served as special mediator of the United
Nations in the Iran--Iraq War, and was President of the Nordic Council
in 1979. He returned as Prime Minister after electoral victories in 1982
and 1985.

Palme was a pivotal and polarizing figure domestically as well as in
international politics from the 1960s. He was steadfast in his
non-alignment policy towards the superpowers, accompanied by support for
numerous third world liberation movements following decolonization
including, most controversially, economic and vocal support for a number
of Third World governments. He was the first Western head of government
to visit Cuba after its revolution, giving a speech in Santiago praising
contemporary Cuban and Cambodian revolutionaries.

Frequently a critic of United States and Soviet foreign policy, he
resorted to fierce and often polarizing criticism in pinpointing his
resistance towards imperialist ambitions and authoritarian regimes,
including those of Francisco Franco of Spain, Leonid Brezhnev of the
Soviet Union, António de Oliveira Salazar of Portugal and Gustáv Husák
of Czechoslovakia, as well as B. J. Vorster and P. W. Botha of South
Africa. His 1972 condemnation of the Hanoi bombings, notably comparing
the tactic to the Treblinka extermination camp, resulted in a temporary
freeze in Sweden--United States relations.

Palme's murder on a Stockholm street on 28 February 1986 was the first
assassination of a national leader in Sweden since Gustav III, and had a
great impact across Scandinavia. Local convict and addict Christer
Pettersson was originally convicted of the murder in district court but
was acquitted on appeal to the Svea Court of Appeal.

\section{Early life}\label{early-life}

\begin{itemize}
\item
  \emph{A sickly child, Olof Palme received his education from private
  tutors.}
\item
  \emph{Palme was an atheist.}
\item
  \emph{Through her, Olof Palme claimed ancestry from King Frederick I
  of Denmark and Norway.}
\item
  \emph{Palme was born into an upper class, conservative Lutheran family
  in the Östermalm district of Stockholm, Sweden.}
\item
  \emph{His father Gunnar Palme was a businessman, son of Sven Theodore
  Palme and Baroness Hanna Maria von Born-Sarvilahti.}
\end{itemize}

Palme was born into an upper class, conservative Lutheran family in the
Östermalm district of Stockholm, Sweden. The Palme family is of Dutch
ancestry and is related to several other prominent Swedish families such
as the von Sydows and the Wallenbergs. His father Gunnar Palme was a
businessman, son of Sven Theodore Palme and Baroness Hanna Maria von
Born-Sarvilahti. Through her, Olof Palme claimed ancestry from King
Frederick I of Denmark and Norway. His mother, Elisabeth von Knieriem,
was descended from Baltic German tradesmen; she had arrived in Sweden
from Russia as a refugee in 1915. Elisabeth's great-great-great
grandfather Johann Melchior von Knieriem (1758--1817) had been ennobled
by the Emperor Alexander I of Russia in 1814. Her great-grandfather
Alexander von Knieriem (1837--1904) was an attorney general of the
Senate of Russian Empire, senator and member of the State Council of
Imperial Russia. The von Knieriem do not count as members of the
Baltische Ridderschaft. Palme's father died when he was six years old.
Despite his background, his political orientation came to be influenced
by Social Democratic attitudes. His travels in the Third World, as well
as the United States, where he saw deep economic inequality and racial
segregation, helped to develop these views.

A sickly child, Olof Palme received his education from private tutors.
Even as a child he gained knowledge of two foreign languages -- German
and English. He studied at the Sigtuna School of Liberal Arts, one of
Sweden's few residential high schools, and passed the university
entrance examination with high marks at the age of 17. He was called up
into the Army in January 1945 and did his compulsory military service at
A 1 between 1945 and 1947, became in 1956 a reserve officer with the
rank of Captain in the Artillery. After he was discharged from military
service in March 1947, he enrolled at the University of Stockholm.

On a scholarship, he studied at Kenyon College, a small liberal arts
school in central Ohio from 1947 to 1948, graduating with a BA Inspired
by radical debate in the student community, he wrote a critical essay on
Friedrich Hayek's The Road to Serfdom. Palme wrote his senior honour
thesis on the United Auto Workers union, led at the time by Walter
Reuther. After graduation he traveled throughout the country and
eventually ended up in Detroit, where his hero Reuther agreed to an
interview which lasted several hours. In later years, Palme regularly
remarked during his many subsequent American visits, that the United
States had made him a socialist, a remark that often has caused
confusion. Within the context of his American experience, it was not
that Palme was repelled by what he found in America, but rather that he
was inspired by it.

After hitchhiking through the USA and Mexico, he returned to Sweden to
study law at Stockholm University. In 1949 he became a member of the
Swedish Social Democratic Party. During his time at university, Palme
became involved in student politics, working with the Swedish National
Union of Students. In 1951, he became a member of the social democratic
student association in Stockholm, although it is asserted he did not
attend their political meetings at the time. The following year he was
elected President of the Swedish National Union of Students. As a
student politician he concentrated on international affairs and
travelled across Europe.

Palme attributed his becoming a socialist to three major influences:

In 1947, he attended a debate on taxes between the Social Democrat Ernst
Wigforss, the conservative Jarl Hjalmarson and the liberal Elon
Andersson;

The time he spent in the United States in the 1940s made him realise how
wide the class divide was in America, and the extent of racism against
black people; and,

A trip to Asia, specifically India, Ceylon (now Sri Lanka), Burma,
Thailand, Singapore, Indonesia, and Japan in 1953 had opened his eyes to
the consequences of colonialism and imperialism.

Palme was an atheist.

\includegraphics[width=5.50000in,height=4.14799in]{media/image1.JPG}\\
\emph{Palme in 1968}

\section{Political career}\label{political-career}

\begin{itemize}
\item
  \emph{In 1953, Palme was recruited by the social democratic prime
  minister Tage Erlander to work in his secretariat.}
\item
  \emph{When party leader Tage Erlander stepped down in 1969, Palme was
  elected as the new leader by the Social Democratic party congress and
  succeeded Erlander as Prime Minister.}
\item
  \emph{The protests culminated with the occupation of the Student Union
  Building in Stockholm; Palme came there and tried to comfort the
  students, urging them to use democratic methods for the pursuit of
  their cause.}
\end{itemize}

In 1953, Palme was recruited by the social democratic prime minister
Tage Erlander to work in his secretariat. From 1955 he was a board
member of the Swedish Social Democratic Youth League and lectured at the
Youth League College Bommersvik. He also was a member of the Worker's
Educational Association.

In 1957 he was elected as a member of parliament (Swedish:
riksdagsledamot) represented Jönköping County in the directly-elected
Second Chamber (Andra kammaren) of the Riksdag. In the early 1960s Palme
became a member of the Agency for International Assistance (NIB) and was
in charge of inquiries into assistance to the developing countries and
educational aid. In 1963, he became a member of the Cabinet as Minister
without Portfolio in the Cabinet Office, and retained his duties as a
close political adviser to Prime Minister Tage Erlander. In 1965, he
became Minister of Transport and Communications. One issue of special
interest to him was the further development of radio and television,
while ensuring their independence from commercial interests. In 1967 he
became Minister of Education, and the following year, he was the target
of strong criticism from left-wing students protesting against the
government's plans for university reform. The protests culminated with
the occupation of the Student Union Building in Stockholm; Palme came
there and tried to comfort the students, urging them to use democratic
methods for the pursuit of their cause. When party leader Tage Erlander
stepped down in 1969, Palme was elected as the new leader by the Social
Democratic party congress and succeeded Erlander as Prime Minister.

His protégé and political ally, Bernt Carlsson, who was appointed UN
Commissioner for Namibia in July 1987, was killed in the bombing of Pan
Am Flight 103 over Lockerbie, Scotland on 21 December 1988 en route to
the UN signing ceremony of the New York Accords the following day.

Palme was said to have had a profound impact on people's emotions; he
was very popular among the left, but harshly detested by most liberals
and conservatives. This was due in part to his international activities,
especially those directed against the US foreign policy, and in part to
his aggressive and outspoken debating style.

\includegraphics[width=5.50000in,height=4.04682in]{media/image2.jpg}\\
\emph{Olof Palme marching against the Vietnam War with the North Vietnam
ambassador in Stockholm, 1968}

\section{Policies and views}\label{policies-and-views}

\begin{itemize}
\item
  \emph{As leader of a new generation of Swedish Social Democrats, Palme
  was often described as a "revolutionary reformist".}
\item
  \emph{The protest was organized by the Swedish Committee for Vietnam
  and Palme and Nguyen were both invited as speakers.}
\item
  \emph{On the international scene, Palme was a widely recognised
  political figure because of his:}
\end{itemize}

As leader of a new generation of Swedish Social Democrats, Palme was
often described as a "revolutionary reformist". Domestically, his
democratic socialist views, especially the drive to expand Labour Union
influence over business ownership, engendered a great deal of hostility
from the organized business community.

During the tenure of Palme, several major reforms in the Swedish
constitution were carried out, such as orchestrating a switch from
bicameralism to unicameralism in 1971 and in 1975 replacing the 1809
Instrument of Government (at the time the oldest political constitution
in the world after that of the United States) with a new one officially
establishing parliamentary democracy rather than de jure monarchic
autocracy, abolishing the Cabinet meetings chaired by the King and
stripping the monarchy of all formal political powers.

His reforms on labour market included establishing a law which increased
job security. In the Swedish 1973 general election, the
Socialist-Communist and the Liberal-Conservative blocs got 175 places
each in the Riksdag. The Palme cabinet continued to govern the country
but several times they had to draw lots to decide on some issues,
although most important issues were decided through concessional
agreement. Tax rates also rose from being fairly low even by European
standards to the highest levels in the Western world.

Under Palme's premiership tenure, matters concerned with child care
centers, social security, protection of the elderly, accident safety,
and housing problems received special attention. Under Palme the public
health system in Sweden became efficient, with the infant mortality rate
standing at 12 per 1,000 live births. An ambitious redistributive
programme was carried out, with special help provided to the disabled,
immigrants, the low paid, single-parent families, and the old. The
Swedish welfare state was significantly expanded from a position already
one of the most far-reaching in the world during his time in office. As
noted by Isabela Mares, during the first half of the Seventies "the
level of benefits provided by every subsystem of the welfare state
improved significantly." Various policy changes increased the basic
old-age pension replacement rate from 42\% of the average wage in 1969
to 57\%, while a health care reform carried out in 1974 integrated all
health services and increased the minimum replacement rate from 64\% to
90\% of earnings. In 1974, supplementary unemployment assistance was
established, providing benefits to those workers ineligible for existing
benefits. In 1971, eligibility for invalidity pensions was extended with
greater opportunities for employees over the age of 60. In 1974,
universal dental insurance was introduced, and former maternity benefits
were replaced by a parental allowance. In 1974, housing allowances for
families with children were raised and these allowances were extended to
other low-income groups. Childcare centres were also expanded under
Palme, and separate taxation of husband and wife introduced. Access to
pensions for older workers in poor health was liberalised in 1970, and a
disability pension was introduced for older unemployed workers in 1972.

The Palme cabinet was also active in the field of education, introducing
such reforms as a system of loans and benefits for students, regional
universities, and preschool for all children. Under a law of 1970, in
the upper secondary school system "gymnasium," ``fackskola" and
vocational "yrkesskola" were integrated to form one school with 3
sectors (arts and social science, technical and natural sciences,
economic and commercial). In 1975, a law was passed that established
free admission to universities. A number of reforms were also carried
out to enhance workers' rights. An employment protection Act of 1974
introduced rules regarding consultation with unions, notice periods, and
grounds for dismissal, together with priority rules for dismissals and
re-employment in case of redundancies. That same year, work-environment
improvement grants were introduced and made available to modernising
firms "conditional upon the presence of union-appointed 'safety
stewards' to review the introduction of new technology with regard to
the health and safety of workers." In 1976, an Act on co-determination
at work was introduced that allowed unions to be consulted at various
levels within companies before major changes were enforced that would
affect employees, while management had to negotiate with labour for
joint rights in all matters concerning organisation of work, hiring and
firing, and key decisions affecting the workplace.

Palme's last government, elected during a time when Sweden's economy was
in difficult shape, sought to pursue a "third way," designed to
stimulate investment, production, and employment, having ruled out
classical Keynesian policies as a result of the growing burden of
foreign debt, together with the big balance of payments and budget
deficits. This involved "equality of sacrifice," whereby wage restraint
would be accompanied by increases in welfare provision and more
progressive taxation. For instance, taxes on wealth, gifts, and
inheritance were increased, while tax benefits to shareholders were
either reduced or eliminated. In addition, various welfare cuts carried
out before Olof's return to office were rescinded. The previous system
of indexing pensions and other benefits was restored, the grant-in-aid
scheme for municipal child care facilities was re-established,
unemployment insurance was restored in full, and the so-called "no
benefit days" for those drawing sickness benefits were cancelled.
Increases were also made to both food subsidies and child allowances,
while the employee investment funds (which represented a radical form of
profit-sharing) were introduced.

In 1968, Palme was a driving force behind the release of the documentary
Dom kallar oss mods ("They Call Us Misfits"). The controversial film,
depicting two social outcasts, was scheduled to be released in an edited
form but Palme thought the material was too socially important to be
cut.

An outspoken supporter of gender equality, Palme sparked interest for
women's rights issues by attending a World Women's Conference in Mexico.
He also made a feminist speech called "The Emancipation of Man" at a
meeting of the Woman's National Democratic Club on June 8, 1970; this
speech was later published in 1972.

As a forerunner in green politics, Palme was a firm believer in nuclear
power as a necessary form of energy, at least for a transitional period
to curb the influence of fossil fuel. His intervention in Sweden's 1980
referendum on the future of nuclear power is often pinpointed by
opponents of nuclear power as saving it. As of 2011, nuclear power
remains one of the most important sources of energy in Sweden, much
attributed to Palme's actions.

Shortly before his assassination, Palme had been accused of being
pro-Soviet and not sufficiently safeguarding Sweden's national interest.
Arrangements had therefore been made for him to go to Moscow to discuss
a number of contentious bilateral issues, including then ongoing Soviet
submarine incursions into Swedish waters (see US Psychological warfare
and U 137).

On the international scene, Palme was a widely recognised political
figure because of his:

harsh and emotional criticism of the United States over the Vietnam War;

vocal opposition to the crushing of the Prague Spring by the Soviet
Union;

criticism of European Communist regimes, including labeling the Husák
regime as "The Cattle of Dictatorship" (Swedish: "Diktaturens kreatur")
in 1975;

campaigning against nuclear weapons proliferation;

criticism of the Franco Regime in Spain, calling the regime "bloody
murderers" (Swedish: "satans mördare", literally Satan's murderers)
after its execution of ETA and FRAP terrorists in September 1975;

opposition to apartheid, branding it as "a particularly gruesome
system", and support for economic sanctions against South Africa;

support, both political and financial, for the African National Congress
(ANC), the Palestine Liberation Organization (PLO) and the POLISARIO
Front;

visiting Fidel Castro's Cuba in 1975, during which he denounced
Fulgencio Batista's government and praised contemporary Cuban
revolutionaries;

strong criticism of the Pinochet regime in Chile;

support, both political and financial, for the FMLN-FDR in El Salvador
and the FSLN in Nicaragua; and,

role as a mediator in the Iran--Iraq War.

All of this ensured that Palme had many opponents as well as many
friends abroad.

On 21 February 1968, Palme (then Minister of Education) participated in
a protest in Stockholm against U.S. involvement in the war in Vietnam
together with the North Vietnamese Ambassador to the Soviet Union Nguyen
Tho Chan. The protest was organized by the Swedish Committee for Vietnam
and Palme and Nguyen were both invited as speakers. As a result of this,
the U.S. recalled its Ambassador from Sweden and Palme was fiercely
criticised by the opposition for his participation in the protest.

On 23 December 1972, Palme (then Prime Minister) made a speech on
Swedish national radio where he compared the ongoing U.S. bombings of
Hanoi to historical atrocities, namely the bombing of Guernica, the
massacres of Oradour-sur-Glane, Babi Yar, Katyn, Lidice and Sharpeville,
and the extermination of Jews and other groups at Treblinka. The US
government called the comparison a "gross insult" and once again decided
to freeze its diplomatic relations with Sweden (this time the freeze
lasted for over a year).

Despite such associations and contrary to stated Social Democratic Party
policy, Sweden had in fact secretly maintained extensive military
co-operation with NATO over a long period, and was even under the
protection of a US military security guarantee (see Swedish neutrality
during the Cold War).

In response to Palme's remarks in a meeting with the US ambassador to
Sweden ahead of the Socialist International Meeting in Helsingør in
January 1976, Henry Kissinger, then United States Secretary of State,
asked the US ambassador to "convey my personal appreciation to Palme for
his frank presentation".

\section{Assassination}\label{assassination}

\begin{itemize}
\item
  \emph{Close to midnight on 28 February 1986, he was walking home from
  a cinema with his wife Lisbet Palme in the central Stockholm street
  Sveavägen when he was shot in the back at close range.}
\item
  \emph{He also took over the leadership of the Social Democratic Party,
  which he held until 1996.}
\item
  \emph{Political violence was little-known in Sweden at the time, and
  Olof Palme often went about without a bodyguard.}
\end{itemize}

Political violence was little-known in Sweden at the time, and Olof
Palme often went about without a bodyguard. Close to midnight on 28
February 1986, he was walking home from a cinema with his wife Lisbet
Palme in the central Stockholm street Sveavägen when he was shot in the
back at close range. A second shot grazed Lisbet's back.\\
He was pronounced dead on arrival at Sabbatsbergs sjukhus hospital at
00:06 CET. Lisbet survived without serious injuries.

Deputy Prime Minister Ingvar Carlsson immediately assumed the duties of
Prime Minister, a post he retained until 1991 (and then again in
1994--1996). He also took over the leadership of the Social Democratic
Party, which he held until 1996.

Two years later, Christer Pettersson (d. 2004), a small-time criminal
and drug addict, was convicted of Palme's murder, but his conviction was
overturned. The crime remains unsolved.

\section{See also}\label{see-also}

\begin{itemize}
\item
  \emph{IB affair, a political scandal involving Palme.}
\item
  \emph{Olof Palme International Center}
\item
  \emph{List of Olof Palme memorials, for a list of memorials and places
  named after Olof Palme.}
\item
  \emph{Olof Palme Street, for a list of streets named after Olof Palme}
\item
  \emph{Olof Palme Prize}
\end{itemize}

List of Olof Palme memorials, for a list of memorials and places named
after Olof Palme.

Olof Palme Street, for a list of streets named after Olof Palme

Olof Palme International Center

Olof Palme Prize

List of peace activists

Anna Lindh

Bernt Carlsson

Folke Bernadotte

Dag Hammarskjöld

Caleb J. Anderson

Jo Cox

IB affair, a political scandal involving Palme.

Ebbe Carlsson affair, a political scandal concerning non-official
inquiries into the murder.

\section{Notes}\label{notes}

\section{Further reading}\label{further-reading}

\begin{itemize}
\item
  \emph{Blood on the snow: The killing of Olof Palme (Cornell University
  Press, 2005).}
\item
  \emph{"How Ideas Influence Decision-Making: Olof Palme and Swedish
  Foreign Policy, 1965--1975."}
\item
  \emph{"Attitudes towards a fallen leader: Evaluations of Olof Palme
  before and after the assassination."}
\item
  \emph{"Three Swedish Prime Ministers: Tage Erlander, Olof Palme and
  Ingvar Carlsson."}
\end{itemize}

Bondeson, Jan. Blood on the snow: The killing of Olof Palme (Cornell
University Press, 2005).

Ekengren, Ann-Marie. "How Ideas Influence Decision-Making: Olof Palme
and Swedish Foreign Policy, 1965--1975." Scandinavian Journal of History
36\#2 (2011): 117--134.

Esaiasson, Peter, and Donald Granberg. "Attitudes towards a fallen
leader: Evaluations of Olof Palme before and after the assassination."
British Journal of Political Science 26\#3 (1996): 429--439.

Ruin, Olof. "Three Swedish Prime Ministers: Tage Erlander, Olof Palme
and Ingvar Carlsson." West European Politics 14\#3 (1991): 58--82.

Wilsford, David, ed. Political leaders of contemporary Western Europe: a
biographical dictionary (Greenwood, 1995) pp.~352--61.

\section{In Swedish}\label{in-swedish}

\section{External links}\label{external-links}

\begin{itemize}
\item
  \emph{Olof Palme Memorial Fund}
\item
  \emph{Olof Palme International Center}
\item
  \emph{Olof Palme Archives}
\end{itemize}

Olof Palme Archives

Olof Palme Memorial Fund

Olof Palme International Center
